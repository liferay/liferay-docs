\chapter{Generating New Business Using
Analytics}\label{generating-new-business-using-analytics}

Your company interacts with lots of people, including existing and
prospective customers. On your websites they browse products and
services, ask questions, leave feedback, and more. Presenting relevant,
interesting site content motivates people to do business with you.
Knowing your customers well and understanding how they interact with
your digital content can help you improve their experience.

Liferay Analytics Cloud (Analytics Cloud) is a cloud-based SaaS offering
that works with Liferay Digital Experience Platform (Liferay DXP) to
give you insights into customer attributes and behavior. It aggregates
your data from multiple sources and presents analytics, making it easier
to understand people and their interests. The centralized contact data
and analytics help you communicate intelligently with people and create
terrific user experiences that drive business and make people happier.

\begin{figure}
\centering
\includegraphics{./images/segment-interests.png}
\caption{As you learn about topics that interest your audience, you can
speak intelligently about the topics, through online content and in
person.}
\end{figure}

\section{Benefits}\label{benefits}

Here are key things Analytics Cloud helps you do:

\begin{itemize}
\item
  \textbf{Gain a well-rounded view of people:} Analytics Cloud gives you
  a clear picture of people by aggregating their profile properties and
  behavior data from multiple sources, including Liferay DXP sites.
\item
  \textbf{Identify and track target audiences:} You can focus on
  characteristics and behaviors common to targeted individuals,
  aggregate those individuals into Segments, and track Segment growth.
\item
  \textbf{Learn people's interests:} Once you know topics that interest
  people, you can engage them on those topics. You can also produce
  stimulating content that highlights how your products and services
  relate to the topics. These things help you generate new business and
  improve customer relationships.
\item
  \textbf{Measure content performance:} Identify content that gets the
  most activity so you can build off its success by creating similar
  content to grow other parts of your business. Spot under-performing
  content or content issues that prevent people from engaging more
  deeply with your brand. Fix or improve the content to get people on
  track, or remove the content from your site.
\item
  \textbf{Determine the best advertising channels:} Discover how
  customers are finding your site pages so you can advertise on the
  right channels.
\end{itemize}

These are just some ways you benefit from using Analytics Cloud. Before
getting started with Analytics Cloud, you should understand its key
components.

Analytics Cloud helps you understand people, their interaction with your
sites, and the paths (or channels) that draw them to your sites in these
ways:

\begin{itemize}
\item
  \hyperref[understanding-people]{Understanding People}
\item
  \hyperref[touchpoint-analytics]{Touchpoint Analytics}
\item
  \hyperref[path-analytics]{Path Analytics}
\end{itemize}

Start with understanding people.

\section{Understanding People}\label{understanding-people}

Existing and prospective customers interact with your company in person,
on your websites, and via email and messaging. On Liferay DXP sites,
people are registered users or visitors. Analytics Cloud integrates
contacts (people) from multiple sources, including Liferay DXP sites,
into a single view. It merges contact data into rich profiles of
\emph{Individuals}---unique contacts. You can name contact fields your
way, making them most useful to you and your company.

Analytics Cloud lets you aggregate Individuals into \emph{Segments}.
Segments are groups of Individuals with common characteristics. You can
create Segments dynamically based on criteria such as industry and
location, or statically (comprising hand-selected Individuals). Industry
Segment Analytics shows industries engaging with you most and identifies
your content that interests them.

\begin{figure}
\centering
\includegraphics{./images/segment-growth.png}
\caption{Dynamic Segment membership growth (and decline) reflects the
number of people matching the Segment's criteria.}
\end{figure}

Contact profiles are a big part of Analytics Cloud, but analytics about
people's interaction with your website pages and assets is a bigger
part. This type of analytics (called Touchpoint Analytics) helps you
determine how effective your content is in achieving your business
goals.

\section{Touchpoint Analytics}\label{touchpoint-analytics}

Touchpoints are the locations (pages) where people interact with
content. Here are some of the Touchpoint Analytics:

\begin{itemize}
\item
  \emph{Visitor count:} the number of users (guest and signed in)
  visiting a Touchpoint.
\item
  \emph{Session length:} amount of time people are spending on a page's
  site.
\item
  \emph{Number of sessions:} number of different people on a page's
  site.
\item
  \emph{Time on page:} amount of time people are spending on a page.
\item
  \emph{Bounce rate:} percentage of visitors to a particular website who
  navigate away from the site after viewing only one page.
\end{itemize}

Assets are specific types of content entities on site pages. Traditional
web analytics are based on static websites---everyone sees the same page
content. By additionally providing analytics on content articles,
Analytics Cloud helps you determine the article effectiveness. It also
provides insight into people's interaction with personalized content. It
generates charts on target audience engagement with the content.

Liferay Analytics Cloud (beta) supports these Liferay DXP Assets:

\begin{itemize}
\tightlist
\item
  Blogs
\item
  Documents and Media
\item
  Forms
\item
  Web Content
\end{itemize}

Regarding user behavior with a form, for example, Analytics Cloud shows
the number of views and submissions, the percentage of users abandoning
the form, and the form's average completion time. A line graph shows the
number of views over time.

\begin{figure}
\centering
\includegraphics{./images/page-views.png}
\caption{See how the number of views, visitors, and more are trending on
your pages.}
\end{figure}

As you view Page and Asset Analytics you can discern whether an action
is positive or negative. Here are some examples.

Positive:

\begin{itemize}
\tightlist
\item
  staying active in sessions
\item
  frequent sign in
\item
  submitting a form
\item
  giving an article five stars
\end{itemize}

Negative:

\begin{itemize}
\tightlist
\item
  signing out shortly after signing in
\item
  passing by key strategic content
\item
  abandoning a form
\item
  letting a session end
\end{itemize}

Learning how people get to pages and Assets is also useful. That's Path
Analytics.

\section{Path Analytics}\label{path-analytics}

People get to content through different channels (paths):

\begin{itemize}
\tightlist
\item
  Google search
\item
  Advertisements
\item
  Site navigation
\item
  Search within the site
\item
  Links inside other pages
\item
  Links within Assets
\end{itemize}

If you expect your advertisements on a channel to draw lots of people to
a page, for example, you can check its performance in the page's Path
Analytics. You may learn that a particular social network increasingly
drives site traffic, and thus advertise more on that network.

Path Analytics reports the device and browser types that render your
site content. It helps you determine which devices and browsers to test
more and optimize user experiences on.

You've been introduced to Liferay Analytics Cloud and what it offers:
Segments, Touchpoints, Paths, and more. It's time to experience
Analytics Cloud for yourself.

\chapter{Your New Home for Analytics}\label{your-new-home-for-analytics}

To get started using Liferay Analytics Cloud, the first thing you need
is an invitation. Liferay invites the Analytics Cloud owner (the person
who purchased the product) to your company's Analytics Cloud project. If
you're not the owner but still need access to the project, ask an
Analytics Cloud system administrator to invite you.

Your invitation links you to the Analytics Cloud
\href{https://analytics.liferay.com/}{sign in page}, which prompts you
for your Liferay credentials.

First time users see the Workspace creation page:

\begin{enumerate}
\def\labelenumi{\arabic{enumi}.}
\item
  Name your workspace, select a server location for hosting your data,
  and review/agree to the terms and conditions.
\item
  Create the workspace. A message appears, stating that your Analytics
  Cloud environment is being set up. Next your project home page
  appears.
\end{enumerate}

Subsequent visits to \url{https://analytics.liferay.com/} show the
Workspace welcome page, where you can select your workspace or any other
accessible workspaces for working on projects.

\begin{figure}
\centering
\includegraphics{./images/home-page-initial.png}
\caption{Your project's home page welcomes you to all Analytics Cloud
offers.}
\end{figure}

The project home page describes and links to each key component page:

\textbf{Segments:} Aggregate customers into Segments based on similar
profiles and behavior data. Track Segment growth and discover patterns
of interest. Discover your target audience.

\textbf{Accounts:} Learn how individuals that belong to Salesforce
accounts interact with your sites. By combining Salesforce account data
with data from other sources, Analytics Cloud presents a holistic view
of the account and those in it.

\textbf{Individuals:} Learn how individual customers interact with your
Sites. View holistic profiles that combine customer's profile data with
their page and asset interactions for deep insight into their interests.

\textbf{Sites:} Combine path analytics and traditional page analytics to
understand how customers interact with your content. Measure individual
web page performance to improve digital experiences.

\textbf{Assets:} Track and visualize engagement levels of digital assets
such as forms, blogs, web content, documents, and media to ensure that
assets are driving conversions.

\textbf{Data Sources:} Integrate your customer profile and behavior data
from multiple Liferay DXP sites and custom data sources.

Additional shortcuts and overviews are displayed on the right:

\textbf{Add Data Source:} Quick access to adding data sources for
customer profiles and behavior data.

\textbf{Usage:} Compares the number of Individuals and Pages you're
analyzing to your Liferay Analytics Cloud Plan.

\textbf{User Management:} Invite users and assign permissions.

The \emph{Liferay Analytics Cloud} icon
(\includegraphics{./images/icon-analytics-cloud.png}) at the top of the
navigation panel takes you to the home page from wherever you are in the
application. Beneath it are several other links:

\textbf{People}: Learn more about your customers by analyzing them
individually as \emph{Individuals}, aggregated as \emph{Segments}, or as
part of Salesforce \emph{Accounts}.

\textbf{Touchpoints}: Analyze \emph{page} and \emph{asset} performance
and view paths users take to your pages.

\textbf{User icon:} Switch projects or sign out.

\textbf{Settings}: Provides options for managing data sources, users,
and updating your Analytics Cloud usage.

If no data sources are connected, a message near the top of the main
area tells you so and links you to the page for adding data sources.
Adding data sources is next.

\section{Managing Data Sources}\label{managing-data-sources}

Liferay Analytics Cloud requires two kinds of data. First, you need web
analytics data on user interactions with Liferay DXP Pages and Assets.
Second, you need profile data on the users themselves. Analytics Cloud
can sync these two kinds of data so that you can see not only how users
are interacting with your site, but also who those users are. First,
however, you must provide it with appropriate data sources.

Both kinds of data can be obtained from a Liferay DXP instance. If you
have other user profile data---in addition to what is already stored in
Liferay DXP user models---you can import it from a CSV file. In a future
Analytics Cloud release, Liferay plans to support syncing with contact
data from Salesforce too.

The contact data is merged and consolidated into a single customer view.
Liferay DXP data sources let you select Organizations and User Groups to
sync. Individual contacts are matched by email address. You can define
the contact data model using whatever field names and types you want,
but Analytics Cloud makes it easy to define trivial model fields by
suggesting values.

Your data sources can be viewed from the navigation panel.

\begin{enumerate}
\def\labelenumi{\arabic{enumi}.}
\item
  Select \emph{Settings}.
\item
  Click on \emph{Data Sources}.
\end{enumerate}

The Data Sources page appears and lists all existing data sources.

\begin{figure}
\centering
\includegraphics{./images/data-source-list.png}
\caption{View, edit, and add data sources from the Data Sources page.}
\end{figure}

Unless a teammate has already added a data source, the list is empty. To
add a new data source, see the following tutorials:

\begin{itemize}
\tightlist
\item
  \href{https://github.com/liferay/liferay-docs/blob/7.1.x/discover/analytics-cloud/articles/02-getting-started/02-adding-a-liferay-dxp-data-source.markdown}{Adding
  a Liferay DXP Data Source}
\item
  \href{https://github.com/liferay/liferay-docs/blob/7.1.x/discover/analytics-cloud/articles/02-getting-started/03-adding-a-csv-data-source.markdown}{Adding
  a CSV Data Source}
\end{itemize}

Once you've created your data sources, you might need to modify them
from time to time.

\subsection{Modifying a Data Source}\label{modifying-a-data-source}

The Data Sources lets you page through the data source listing or use
\emph{Search} to find one that matches a keyword. Once you've found the
data source, click on it to edit it. Here's what you can modify:

\begin{itemize}
\tightlist
\item
  Name of the data source
\item
  Contact mapping
\item
  Enable/disable Contact or Analytics sync
\end{itemize}

Here's how to delete a data source:

\begin{enumerate}
\def\labelenumi{\arabic{enumi}.}
\item
  Click on the listed data source. The data source's page appears.
\item
  Click \emph{Delete Data Source}.
\end{enumerate}

Your Liferay DXP instances are rich sources of customer profile and
behavior data. Adding a Liferay DXP instance as a data source is next.

\section{Adding a Liferay DXP Data
Source}\label{adding-a-liferay-dxp-data-source}

Your Liferay DXP instances are rich with contact data from Users and web
analytics data on user interaction with Liferay DXP pages and assets.
When you Create Liferay DXP data sources, you can select web analytics
data from the Liferay DXP Sites you want. In Users and Organizations,
you can choose to use contact data from all Users or a specified subset
of them. Before adding an instance as a data source, though, you must
connect it to your Analytics Cloud project.

\subsection{Liferay DXP Data Source
Prerequisites}\label{liferay-dxp-data-source-prerequisites}

Follow these steps to connect your project to your Liferay DXP instance.

\subsubsection{Step 1: Install Required Liferay DXP Fix
Packs}\label{step-1-install-required-liferay-dxp-fix-packs}

Liferay DXP 7.1: \href{https://customer.liferay.com/downloads}{Download}
and
\href{/docs/7-1/deploy/-/knowledge_base/d/installing-patches}{install}
fix pack 10+.

Liferay DXP 7.0: \href{https://customer.liferay.com/downloads}{Download}
and
\href{/docs/7-0/deploy/-/knowledge_base/d/patching-tool\#installing-patches}{install}
fix pack 79+.

\subsubsection{Step 2: Register Analytics Cloud with your Liferay DXP
instance}\label{step-2-register-analytics-cloud-with-your-liferay-dxp-instance}

Liferay DXP 7.1:

\begin{enumerate}
\def\labelenumi{\arabic{enumi}.}
\item
  Download the
  \href{https://web.liferay.com/marketplace/-/mp/application/109571986}{Liferay
  Plugin for OAuth 2.0} version 1.1.0 (or newer) and
  \href{/docs/7-1/user/-/knowledge_base/u/installing-apps-manually}{install}
  it.
\item
  The plugin comes with Analytics Cloud pre-registered. Copy the
  \emph{Client ID} and \emph{Client Secret} for connecting DXP with
  Analytics Cloud, as described in the next section.
\end{enumerate}

Liferay DXP 7.0:

\begin{enumerate}
\def\labelenumi{\arabic{enumi}.}
\item
  Download the
  \href{https://web.liferay.com/marketplace/-/mp/application/45261909}{Liferay
  Connector to OAuth 1.0a} and
  \href{/docs/7-0/user/-/knowledge_base/u/installing-apps-manually}{install}
  it.
\item
  \href{/docs/7-0/deploy/-/knowledge_base/d/oauth}{Register} Analytics
  Cloud as an OAuth application with the \emph{Write} access level.
\item
  Copy the \emph{Consumer ID} and \emph{Consumer Secret} for connecting
  DXP with Analytics Cloud, as described in the next section.
\end{enumerate}

Congratulations on authorizing Analytics Cloud to connect to your
Liferay DXP instance! It's time to add your DXP instance as a data
source.

\noindent\hrulefill

\textbf{Tip:} If you have problems connecting to your DXP data source,
refer to
\href{https://github.com/liferay/liferay-docs/blob/7.1.x/discover/analytics-cloud/articles/06-troubleshooting/00-troubleshooting-data-sources-intro.markdown}{Troubleshooting
Liferay DXP Data Sources}.

\noindent\hrulefill

\subsection{Adding the DXP Data
Source}\label{adding-the-dxp-data-source}

Adding a Liferay DXP data source connects your Analytics Cloud project
with a Liferay DXP instance.

\begin{enumerate}
\def\labelenumi{\arabic{enumi}.}
\item
  Select \emph{Settings} → \emph{Data Sources}. A listing of your data
  sources appears.
\item
  Click \emph{Add Data Source}. The \emph{Add Data Source} page appears.
\item
  Select the \emph{Liferay DXP} icon. The \emph{Configure Liferay DXP}
  page appears.
\end{enumerate}

The Authorization tab is selected by default. It's time to authorize
your DXP instance as a data source.

\subsubsection{DXP Data Source
Authorization}\label{dxp-data-source-authorization}

Here's how to authorize your DXP instance as a data source:

\begin{enumerate}
\def\labelenumi{\arabic{enumi}.}
\item
  Fill in the data source and client credentials fields.

  \emph{Description}

  \textbf{Name:} A name for your data source.

  \textbf{URL:} The Liferay DXP instance URL.

  \emph{Client Credentials}

  \textbf{Consumer Key/Client ID:} Key/ID for Analytics Cloud to access
  your Liferay DXP instance.

  \textbf{Consumer Secret/Client Secret:} Secret for Analytics Cloud to
  access your Liferay DXP instance.

  In Liferay DXP 7.1, the Client ID and Secret are found at
  \emph{Control Panel} → \emph{Configuration} → \emph{OAuth 2 Admin}.

  In Liferay DXP 7.0, the Consumer Key and Secret are found at
  \emph{Control Panel} → \emph{OAuth Admin}.
\item
  Click \emph{Authorize}. A window appears and prompts you to sign in to
  the DXP instance.
\item
  Sign in by entering your DXP admin (user that has the Admin role)
  credentials and clicking \emph{Authorize}.
\item
  Click \emph{Save} to save the authorization options. Analytics Cloud
  advances you to the Configure Data Source tab's Data Configuration
  page. The data source's Current Status is \emph{AUTHENTICATED}.
\end{enumerate}

Here are the Configure Data Source options:

\textbf{Configure Contacts:} Configures the contact data only.

\textbf{Configure Analytics:} Configures the assets and touchpoints
only.

Start with configuring contacts.

\subsubsection{Configuring Contacts}\label{configuring-contacts}

Configuring contacts imports DXP user data.

\begin{enumerate}
\def\labelenumi{\arabic{enumi}.}
\item
  Select the \emph{Configure} button for \emph{Configure Contacts}. The
  Contacts configuration options appear.
\item
  Configuring contacts involves selecting contacts to sync from the
  Liferay DXP instance and its User Groups and Organizations. Contacts
  belonging to multiple User Groups and Organizations are only counted
  once.

  \textbf{Sync All Contacts:} Selects all Liferay DXP instance contacts
  and disables options for selecting specific User Groups and
  Organizations.

  \textbf{Sync By User Groups:} Selects contacts by User Group.

  \textbf{Sync By Organizations:} Selects contacts by Organization.

  \begin{figure}
  \centering
  \includegraphics{./images/select-dxp-contacts-by-org.png}
  \caption{Analytics Cloud lets you select and import contacts from a
  Liferay DXP instance and its Organizations and User Groups.}
  \end{figure}
\item
  Click \emph{Save and Continue} to import the selected contacts.
  Analytics Cloud imports the contact data and maps it to your Analytics
  Cloud contact data model. The initial contact data import can take 5
  1/2 minutes per 1,000 contacts.
\item
  Follow instructions for
  \href{https://github.com/liferay/liferay-docs/blob/7.1.x/discover/analytics-cloud/articles/02-getting-started/04-mapping-contact-data.markdown}{Mapping
  Contact Data} to map contact data from your Liferay DXP instance to
  your Analytics Cloud contact data model. Once you've mapped the data,
  click \emph{Save}. The Data Configuration page appears again; the
  button for Configure Contacts is labeled \emph{Edit}.
\end{enumerate}

You've configured Analytics Cloud to use your Liferay DXP contacts.

\subsubsection{Configuring Analytics}\label{configuring-analytics}

Configuring analytics imports asset and touchpoint data as it relates to
DXP contacts you've imported.

\begin{enumerate}
\def\labelenumi{\arabic{enumi}.}
\item
  Click \emph{Configure} for \emph{Configure Analytics}. The Liferay DXP
  site analytics registration page appears.
\item
  Select the Liferay DXP Sites to register for analytics and click
  \emph{Configure}.
\item
  Click the \emph{Done} button.
\end{enumerate}

The Contacts and Analytics data start syncing into Analytics Cloud.
\textbf{Initially the sync takes a while. After the initial sync,
changes are synced periodically.}

\noindent\hrulefill

\textbf{Tip:} If you have problems connecting to your DXP data source,
refer to
\href{https://github.com/liferay/liferay-docs/blob/7.1.x/discover/analytics-cloud/articles/06-troubleshooting/00-troubleshooting-data-sources-intro.markdown}{Troubleshooting
Liferay DXP Data Sources}.

\noindent\hrulefill

If you have contact profile data from other sources such as a database,
you might be able to export the data to a CSV file. Then you can add the
CSV file as a data source and import the contact profiles from it.
Adding a CSV data source is next.

\section{Adding a CSV Data Source}\label{adding-a-csv-data-source}

You can import contact profile data from CSV files to enrich customer
profiles with additional data related to a user's industry, job title,
annual income, or whatever metrics are important to your business. If
you have customer profile data in databases or collect it through web
forms, you can export the data into CSV files.

\textbf{Important:} The CSV files must have an email column.

Here's how to integrate contact data from a CSV file:

\begin{enumerate}
\def\labelenumi{\arabic{enumi}.}
\item
  In the Data Sources page, click \emph{Add Data Source}. A page appears
  showing the data source types.
\item
  Select the \emph{CSV File} icon. The CSV file upload page appears.
\item
  Upload your CSV file by dragging it into the upload area or browsing
  and selecting it from your file system. Click \emph{Next} to upload
  the file. The CSV configuration page appears.
\item
  Configure details and metadata about your CSV file data source, and
  click \emph{Next} when you're done.

  \begin{enumerate}
  \def\labelenumii{\arabic{enumii}.}
  \item
    \emph{Name:} Enter a name for your data source.
  \item
    \emph{Context:} Select a context that describes where the contact
    data is from. On selecting context, fields appear for you to further
    describe the data.
  \item
    \emph{View Data Preview:} Shows the raw data in table format. If
    it's not formatted the way you want, adjust your CSV file's format.
  \end{enumerate}
\item
  Follow instructions for
  \href{https://github.com/liferay/liferay-docs/blob/7.1.x/discover/analytics-cloud/articles/02-getting-started/04-mapping-contact-data.markdown}{Mapping
  Contact Data} to map contact data from your CSV file to your Analytics
  Cloud contact data model. Once you've mapped the data, click
  \emph{Next}.
\end{enumerate}

\begin{figure}
\centering
\includegraphics{./images/configure-csv-data-source.png}
\caption{When configuring a CSV file data source, you can describe the
data context and view the data to make sure it's formatted properly.}
\end{figure}

The contact profile data starts syncing. The time it takes to sync
depends on the number of contacts.

From time to time you might need to modify your data source
configurations. If you've already explored \emph{Mapping Contact Data}
(the next topic) skip ahead to \emph{Editing Data Sources}, to see how
easy it is.

\section{Adding a Salesforce Data
Source}\label{adding-a-salesforce-data-source}

Analytics Cloud can integrate Salesforce data with data from Liferay DXP
and other sources, automatically blending the data to build a
comprehensive customer profile. This lets you create more accurate
segmentation and enhance your site's personalization strategy. To do
this, you must first add a Salesforce instance as a data source.

\subsection{Adding the Data Source}\label{adding-the-data-source}

\begin{enumerate}
\def\labelenumi{\arabic{enumi}.}
\item
  Select \emph{Settings} → \emph{Data Sources} → \emph{Add Data Source}.
  This opens the Add Data Source screen.
\item
  Select \emph{Salesforce}. This opens the Configure Salesforce screen.

  \begin{figure}
  \centering
  \includegraphics{./images/salesforce-data-source.png}
  \caption{Select Salesforce from the Add Data Source screen.}
  \end{figure}
\item
  In the Authorization tab, enter the Salesforce instance's name and
  URL. Under the CLIENT CREDENTIALS section, enter the client ID and
  client secret of the OAuth connected app in Salesforce. Note that the
  Salesforce instance's administrator must create this connected app
  with the following settings:

  \textbf{Callback URL:}
  \texttt{https://analytics.liferay.com/oauth/receive}

  \textbf{OAuth Scopes:}

  \begin{itemize}
  \tightlist
  \item
    Access your basic information (id, profile, email address, phone)
  \item
    Access and manage your data (api)
  \item
    Perform requests on your behalf at any time (refresh\_token,
    offline\_access)
  \end{itemize}

  For instructions on creating an OAuth connected app in Salesforce and
  locating its client ID and client secret, see the
  \href{https://help.salesforce.com/articleView?id=connected_app_overview.htm&type=5}{Salesforce
  documentation}.
\item
  Click \emph{Authorize \& Save}. This begins importing the leads,
  contacts, and accounts data from Salesforce. This data is integrated
  with Analytics Cloud data as follows:

  \begin{itemize}
  \tightlist
  \item
    Data from Salesforce
    \href{https://help.salesforce.com/articleView?id=leads_def.htm&type=5}{leads}
    and
    \href{https://help.salesforce.com/articleView?id=contacts_overview.htm&type=5}{contacts}
    are merged with data for matching
    \href{https://help.liferay.com/hc/en-us/articles/360006946171-Profiling-Individuals}{individuals}
    in Analytics Cloud. The match is based on email address. For
    example, if a contact and/or lead has the same email address as an
    individual in Analytics Cloud, the data of that contact and/or lead
    is merged into that individual's Analytics Cloud data.
  \item
    If the email address of a lead and/or contact doesn't match with
    that of an existing individual in Analytics Cloud, a new individual
    is created in Analytics Cloud with the data from the lead and/or
    contact.
  \item
    Data from a contact is prioritized over that of a lead. For example,
    if the same field is populated in a matching lead and contact, then
    the data from the contact is imported into the individual in
    Analytics Cloud.
  \end{itemize}

  Note that importing the data may take some time, depending how much
  data exists in the Salesforce instance.
\item
  The import progress is visible on the \emph{CONFIGURE DATA SOURCE}
  tab. Click \emph{View} to see the mapping of each Salesforce field
  (for Accounts and Individuals) to its corresponding Analytics Cloud
  field. You can also view this information later from \emph{Settings} →
  \emph{Data Sources} → \emph{(Your Salesforce)} → \emph{CONFIGURE DATA
  SOURCE}.
\end{enumerate}

\begin{figure}
\centering
\includegraphics{./images/salesforce-auth.png}
\caption{Enter the information needed to connect to your Salesforce
instance.}
\end{figure}

\begin{figure}
\centering
\includegraphics{./images/salesforce-config.png}
\caption{The CONFIGURE DATA SOURCE tab shows the status of the accounts
and individuals imported from Salesforce, as well as the field mapping.}
\end{figure}

\section{Mapping Contact Data}\label{mapping-contact-data}

Analytics Cloud can map contact profile fields into a unified customer
data model. It starts you with a default model. The process for mapping
contact data is similar across data source types. Once you've selected
the contacts to sync, Analytics Cloud makes a best effort to map contact
data fields from the data source (e.g., Liferay DXP or CSV) to your
unified contact data model.

Analytics Cloud provides several contact data mapping options:

\begin{itemize}
\item
  Choose the most appropriate data model field---Analytics Cloud
  \emph{suggests} possible matches.
\item
  Add a new custom data model field and map source data to it.
\item
  Map source fields from multiple data sources to the same data model
  field.
\end{itemize}

\textbf{Note:} If you map source fields from multiple data sources to
the same model field, the latest modified value is used.

As you map source fields to the data model, you can search for and
select from the data model fields, use a \emph{suggested} field, or
create a new custom field.

\begin{figure}
\centering
\includegraphics{./images/mapping-contact-data.png}
\caption{Analytics Cloud facilitates finding appropriate data model
fields and offering suggestions.}
\end{figure}

Here's how to create custom contact data model fields:

\begin{enumerate}
\def\labelenumi{\arabic{enumi}.}
\item
  Click on the data model field's selector.
\item
  Click \emph{New Field}. A dialog appears for you to create the new
  field.
\item
  In the dialog, name the new field and select its type.
\item
  Click \emph{Create}.
\end{enumerate}

Your custom model field is ready to match with source fields.

\begin{figure}
\centering
\includegraphics{./images/new-contact-field.png}
\caption{Creating new data model fields is easy.}
\end{figure}

When you're done mapping the data, click the \emph{Done} button.
Congratulations on mapping contacts from your data source to your
Analytics Field contact model.

Once you've mapped your contact data and finished adding your data
sources, you can start learning more about your customers. See
\href{https://github.com/liferay/liferay-docs/blob/7.1.x/discover/analytics-cloud/articles/03-understanding-people/00-understanding-people-intro.markdown}{Understanding
People} for more.

\chapter{Learning What Makes Customers
Happier}\label{learning-what-makes-customers-happier}

Analytics Cloud creates a single source of truth that brings together
the various facets of customer knowledge to paint a more rounded, more
informed picture of each customer and his or her behavior. It gives you
a single customer view.

Here, your business users access a single location in which to view and
manage customer data compiled from your customer data sources---DXP
sites and customer CSV files. Analytics Cloud can merge customer data
your way, into a model that works best for you.

This foundation reveals insights into individual customer behavior, such
as how often individuals visit your sites, the content they read, and
the products they browse. The analytics identifies positive and negative
customer experiences. While you and your teammates might individually
know parts of these things, providing this information in one place
allows for highly targeted communication and marketing efforts.

\section{Profiling Individuals}\label{profiling-individuals}

Many individuals visit your sites. Learning who they are and their
activities and interests helps you improve business with them.
Individual profiles help your Sales, Customer Service, and Customer
Relations teams work personally with them. The profiles help you gain
more context on what people do on your websites and their interests, so
you can communicate with them intelligently according to their needs.

Here's how to view Individual profiles:

\begin{enumerate}
\def\labelenumi{\arabic{enumi}.}
\item
  Select \emph{Individuals} from the left navigation panel. The
  \emph{Individuals} page appears, listing individual contacts, so you
  can page through, search for, and select them to learn more about
  them.
\item
  Select an Individual to view these things about the person:

  \begin{itemize}
  \tightlist
  \item
    Activities on your sites
  \item
    Identity details and properties
  \item
    Topic interests
  \item
    Segment membership
  \end{itemize}
\end{enumerate}

Each Individual profile has these tabs:

\begin{itemize}
\tightlist
\item
  \hyperref[overview]{Overview}
\item
  \hyperref[details]{Details}
\item
  \hyperref[interests]{Interests}
\item
  \hyperref[segments]{Segments}
\end{itemize}

Let's explore them to learn more about people you do business with.

\subsection{Overview}\label{overview}

The Overview tab shows the person's recent activities using a histogram
of the customer's activities and frequency in the past 90 days. The
activities include page views, downloads, and form submissions. If you
click on a specific day, it lists the Touchpoints (pages and assets) the
customer acted on during that day only.

\begin{figure}
\centering
\includegraphics{./images/individual-overview.png}
\caption{An Individual's Overview tab is a great place to learn how a
person is interacting with your business.}
\end{figure}

The Overview tab also gives you a glimpse of a person's profile data,
topics of interest, and associated audience segments. Clicking the
\emph{View all \ldots{}} link for the Details, Interests, or Segments
summary opens its respective tab.

Click on the \emph{Details} tab to explore an Individual's details.

\subsection{Details}\label{details}

The Details tab shows all of the person's attributes, the data sources
they came from, and when they were last modified.

\subsection{Interests}\label{interests}

Click on the \emph{Interests} tab.

Depending on the web pages or assets a user interacts with, Analytics
Cloud uses an advanced algorithm to find out what an individual's
interests are with your brand. This helps you prepare for conversations
with the individual or create campaigns based on the interests.

See
\href{https://github.com/liferay/liferay-docs/blob/7.1.x/discover/analytics-cloud/articles/03-understanding-people/04-customer-insights.markdown}{Customer
Insights} for details on the Interests insight.

\subsection{Segments}\label{segments}

The Segments tab lists all the Segments the customer belongs to.
Analyzing Segments helps you learn about the Individual and similar
Individuals. Satisfying Segments of customers generates new business
fast. Segments are explained in detail next.

\section{Creating Segments}\label{creating-segments}

Segments aggregate Individuals based on common properties and behavior.
For example, you can create a Segment that contains users who are
interested in a given topic, work in a specific industry, or both.
Analytics Cloud then analyzes and
\href{https://help.liferay.com/hc/en-us/articles/360006947851-Profiling-Segments}{profiles}
Segments. You can then integrate your Analytics Cloud Segments with
\href{https://github.com/liferay/liferay-docs/blob/7.1.x/discover/analytics-cloud/articles/03-understanding-people/02-liferay-dxp-personalization.markdown}{personalization
in Liferay DXP}. This lets you deliver content of interest to each
Segment via Liferay DXP.

Here, you'll learn how to create segments in Analytics Cloud. In
Analytics Cloud's navigation panel, \emph{Segments} is at the top of the
\emph{People} section because it's where you'll spend most of your time
gaining customer insights.

\begin{figure}
\centering
\includegraphics{./images/ac-segments-panel.png}
\caption{You can access Segments in the navigation panel.}
\end{figure}

There are two types of Segments:

\hyperref[creating-dynamic-segments]{\textbf{Dynamic Segments:}}
Individuals aggregated automatically based on criteria that you specify.
Criteria can be based on Individuals' properties and interaction with
your Liferay DXP Site pages and assets. For example, you can use an
\emph{industry} property as criteria. Analytics on industry Segments
show which industries engage with you the most, and what content they're
interested in.

\hyperref[creating-static-segments]{\textbf{Static Segments:}}
Individuals aggregated manually. Static Segments are therefore comprised
of hand-selected individuals.

\subsection{Creating Dynamic Segments}\label{creating-dynamic-segments}

Follow these steps to create a Dynamic Segment:

\begin{enumerate}
\def\labelenumi{\arabic{enumi}.}
\item
  Click \emph{Segments} in the navigation panel. A table with existing
  Segments appears.
\item
  Click \emph{Create Segment} → \emph{Dynamic Segment}. The \emph{Create
  Individuals Segment} screen appears for creating a Dynamic Segment.
\item
  Click the \emph{Edit} icon next to \emph{Unnamed Segment}, and give
  your segment a name.
\item
  To create your Segment's criteria, drag and drop them from the panel
  on the right to the canvas on the center of the screen. The selector
  menu at the top of the panel lets you select from these criteria
  types:

  \textbf{Individual Properties:} Properties that belong to a person
  (e.g., age, country, industry, etc.).

  \textbf{Account Properties:} Properties that belong to a Salesforce
  Account (e.g., account name, industry, website, etc.).

  \textbf{Interests:} Topics and content types that interest the person.
  For more information on interests in Analytics Cloud, see
  \href{https://help.liferay.com/hc/en-us/articles/360006947951-Customer-Insights}{Customer
  Insights}.

  \textbf{Session Properties:} Properties that belong to a person's web
  session (e.g., browser, geolocation, etc.).

  \textbf{Web Behaviors:} Actions taken by a person (e.g., submitted a
  form, viewed a blog, etc.).

  For further instructions, see \hyperref[creating-criteria]{Creating
  Criteria}.
\item
  Anonymous users are excluded from Segments by default. To include
  them, enable the \emph{Include Anonymous} toggle. Note, however, that
  criteria with individual and/or account properties will exclude
  anonymous users regardless of your setting here. Such properties only
  apply to known users.

  \begin{figure}
  \centering
  \includegraphics{./images/anonymous-toggle.png}
  \caption{You can also include anonymous users in your segment.}
  \end{figure}
\item
  Click \emph{Save Segment} when you're finished.
\end{enumerate}

\subsubsection{Creating Criteria}\label{creating-criteria}

The criteria creation canvas is very flexible. Once added to the canvas,
you can move, delete, or duplicate any criterion:

\textbf{To move:} Click and drag the criterion using the vertical dots
(\includegraphics{./images/icon-handle.png}) on its left.

\textbf{To delete:} Click the criterion's trash icon
(\includegraphics{./images/icon-trash.png}). Alternatively, you can
click the criterion's Actions icon
(\includegraphics{./images/icon-actions.png}) and select \emph{Delete}.

\textbf{To duplicate:} Click the criterion's Actions icon
(\includegraphics{./images/icon-actions.png}) and select
\emph{Duplicate}.

Each criterion that you add contains fields that let you customize it to
your needs. The first field is typically a selector menu in which you
specify a condition for any remaining fields. The condition's values
depend on the data type for the remaining fields. Here are some common
condition values:

\begin{itemize}
\tightlist
\item
  Contains (text)
\item
  Equals
\item
  Greater than (number)
\item
  Is known
\item
  Less than (number)
\item
  Does not contain (text)
\item
  Does not equal
\item
  Is not known
\item
  Greater than or equals (number)
\item
  Less than or equals (number)
\item
  Is
\item
  Is not
\item
  Is before (date)
\item
  Is after (date)
\item
  Has (behavior)
\item
  Has not (behavior)
\end{itemize}

For example, the \texttt{birthDate} criterion's first field is a
selector menu that contains the options \emph{is before}, \emph{is}, and
\emph{is after}. The second field is a date field. You can therefore,
for example, specify a criterion in which only Individuals with a
birthday after 31 December 1980 are part of the Segment.

\begin{figure}
\centering
\includegraphics{./images/criteria-birthdate.png}
\caption{This criterion specifies Individuals with a birthday after 31
December 1980.}
\end{figure}

You can also control the way adjacent criteria interact with each other.
For example, if you place criteria next to each other, a small box
appears between them with the text \emph{AND}. This means that the two
criteria are joined by a logical \emph{And}. Clicking the box changes it
to \emph{OR}, which represents a logical \emph{Or}. Selecting \emph{And}
narrows the Segment's selection of Individuals; \emph{Or} broadens it.

For example, joining two \texttt{birthDate} criteria with the following
conditions creates a Segment targeting the Millennial generation (born
1981 - 1996):

\begin{itemize}
\tightlist
\item
  Is after 31 December 1980
\item
  AND
\item
  Is before 01 January 1997
\end{itemize}

You can also form subgroups of criteria by dragging and dropping
criteria onto each other. An AND/OR box then appears between the
subgroup and any adjacent criteria. Together, these tools let you build
complex criteria for your Segment.

\begin{figure}
\centering
\includegraphics{./images/criteria-groups.png}
\caption{Millennials interested in Liferay love avocado toast.}
\end{figure}

\subsection{Creating Static Segments}\label{creating-static-segments}

Follow these steps to create a Dynamic Segment:

\begin{enumerate}
\def\labelenumi{\arabic{enumi}.}
\item
  Click \emph{Segments} in the navigation panel. A table with existing
  Segments appears.
\item
  Click \emph{Create Segment} → \emph{Static Segment}. The \emph{Create
  Individuals Segment} screen appears for creating a Static Segment.
\item
  Name the Segment.
\item
  Click \emph{Add Members} to bring up the Add Members screen, which
  contains a searchable list of all Individuals.
\item
  Select the Individuals to add to the Segment, then click \emph{Add}.
\item
  To change or undo your selections, click the \emph{View Added Members}
  link and click \emph{Undo} for each Individual you want to remove.
  Alternatively, select each Individual and click the \emph{Undo
  Changes} button that appears. To remove all Individuals, click
  \emph{Undo All}.
\item
  Click \emph{Create} when you're finished.
\end{enumerate}

\section{Liferay DXP Personalization}\label{liferay-dxp-personalization}

When used with Liferay DXP's personalization features, the full power of
Analytics Cloud's Segments is realized. Personalization lets you target
Liferay DXP content at specific Segments. For example, if you have a
Segment for users in the finance industry, then you could use
personalization to show them content on investing.

Note that like Analytics Cloud, Liferay DXP can contain Segments.
Liferay DXP segments, however, are less powerful than those in Analytics
Cloud. Analytics Cloud Segments are more powerful because of the
comparatively greater computing power and resources that exist in the
cloud.

Here's a comparison of Segments in Liferay DXP and Analytics Cloud.

\noindent\hrulefill

\begin{longtable}[]{@{}
  >{\raggedright\arraybackslash}p{(\columnwidth - 2\tabcolsep) * \real{0.4545}}
  >{\raggedright\arraybackslash}p{(\columnwidth - 2\tabcolsep) * \real{0.5455}}@{}}
\toprule\noalign{}
\begin{minipage}[b]{\linewidth}\raggedright
Liferay DXP Segments
\end{minipage} & \begin{minipage}[b]{\linewidth}\raggedright
Analytics Cloud Segments
\end{minipage} \\
\midrule\noalign{}
\endhead
\bottomrule\noalign{}
\endlastfoot
Based only on user data in DXP & Based on user data from multiple
sources (e.g., DXP, Salesforce, CSV files, etc.) \\
Created from session attribute and cookies & Created from user interests
and historical behavior \\
User identity comes from only one DXP site & User identity is resolved
from multiple DXP sites \\
Can't include anonymous users & Includes anonymous users \\
Good for limited, short-term personalization & Good for expansive,
long-term personalization \\
\end{longtable}

\noindent\hrulefill

For information on using Analytics Cloud Segments in Liferay DXP, see
\href{https://help.liferay.com/hc/en-us/articles/360029041751-Using-Analytics-Cloud-With-User-Segments}{Using
Analytics Cloud with User Segments}. For information on personalization
in Liferay DXP, see
\href{https://help.liferay.com/hc/en-us/articles/360028721372-Introduction-to-Segmentation-and-Personalization\#personalizing-experiences}{Personalizing
Experiences}.

\section{Profiling Segments}\label{profiling-segments}

The Segment profiles are your gateway into how people interact with your
sites. They show who's matching your Segment criteria
(\href{https://github.com/liferay/liferay-docs/blob/7.1.x/discover/analytics-cloud/articles/03-understanding-people/02-creating-segments.markdown}{Dynamic
Segments}), so you can deliver content that wows them.

Here's how to view Segment profiles:

\begin{enumerate}
\def\labelenumi{\arabic{enumi}.}
\item
  Select \emph{Segments} from the navigation panel. The \emph{Segments}
  page appears, listing individual Segments. You can page through,
  search for, and select Segments to learn more about them.
\item
  Select a Segment to learn these things about it:

  \begin{itemize}
  \tightlist
  \item
    \emph{Membership:} Population fluctuation over time.\\
  \item
    \emph{Topics of interest:} Topics mentioned in the pages and assets
    the Segment visits most.
  \item
    \emph{Distribution:} Demographics based on Individual properties
  \end{itemize}
\end{enumerate}

The Segment's profile appears, showing its \emph{Overview} panel.

\subsection{Overview}\label{overview-1}

The \emph{Overview} tab's main area shows membership growth. As a
marketer, you might want to generate more customer prospects in a
specific industry. If you create a Dynamic Segment with that industry as
criterion, you can view its membership chart to monitor customer growth
in that industry.

The Overview tab also gives you a glimpse of the Segment's criteria,
interests, and demographics. Each one links to more details, or you can
click on their respective tabs.

\begin{figure}
\centering
\includegraphics{./images/segment-overview.png}
\caption{The Overview tab's main area shows the Segment's membership
growth.}
\end{figure}

\subsection{Interests}\label{interests-1}

As a Segment's members interact with your site pages and assets,
Analytics Cloud's interests algorithm determines the level of interest
the Segment has with your site's topics. Knowing these topics helps you
prepare for conversing with the Segment members and create content that
interests them.

See
\href{https://github.com/liferay/liferay-docs/blob/7.1.x/discover/analytics-cloud/articles/03-understanding-people/04-customer-insights.markdown}{Customer
Insights} for details on the Interests insight.

\subsection{Distribution}\label{distribution}

The \emph{Distribution} tab breaks down the Segment by a demographic
property.

For example, select a property such as \emph{Job Title} to dissect your
Segment further. A bar chart appears, displaying a distribution of the
Segment member population by Job Title. As a marketer, you could use
this information to identify a subset of the Segment to focus a
marketing campaign on.

You've completed touring Segment profiles. Creating Segments and
analyzing their profiles is key to learning more about your customers.

An in-depth look at the interest score insight is next.

\section{Customer Insights}\label{customer-insights}

Finding out topics people are interested in helps you relate your
products and services to them. Analytics Cloud's Interests insight
empowers you with rich, easy-to-understand visuals that provide this
information.

\subsection{Interests}\label{interests-2}

As an Individual interacts with the pages and assets, Analytics Cloud
notes the topics they contain. The more an Individual visits pages and
assets that contain a topic, the higher the topic's ranking is for that
Individual's Segments.

\begin{figure}
\centering
\includegraphics{./images/interests-insight.png}
\caption{Clicking on an Interest topic for a Segment (or Individual)
shows a histogram of the Segment's interaction with pages and assets
mentioning the topic. In this figure, \emph{Intranets} is the selected
topic. Clicking on a histogram point shows the pages and assets viewed
during that time period.}
\end{figure}

For example, say your site has pages that mention these topics:

\begin{itemize}
\tightlist
\item
  Clothing
\item
  Sports
\end{itemize}

If an Individual named Michael frequents these pages, the Interest
ranking for the \emph{clothing} and \emph{sports} topics is high. Here
are a couple of different ways you could use this information:

\begin{itemize}
\item
  If you sell sports apparel, you could prepare a campaign for selling
  sports apparel to Michael.
\item
  If you don't sell sports apparel, you could use the Interest data to
  convince sports apparel vendors to advertise on your site.
\end{itemize}

Either way, the Interests insight helps generate new business. The
Interests insight tells you the most common topics mentioned by the
pages and assets customers visit most.

\section{Profiling Accounts}\label{profiling-accounts}

If you have a Salesforce data source, Analytics Cloud can import and
analyze Salesforce account data. By combining Salesforce account data
with data from other sources, Analytics Cloud presents a holistic view
of the account and those in it.

To view account analytics, select \emph{Accounts} from the left
navigation panel. The accounts are displayed in a searchable table that
lists the following data for each account:

\begin{itemize}
\tightlist
\item
  Account Type
\item
  Individuals
\item
  Total Activities
\item
  30-Day Engagement Score
\end{itemize}

\begin{figure}
\centering
\includegraphics{./images/accounts.png}
\caption{The Accounts tab lists the Salesforce accounts imported in
Analytics Cloud.}
\end{figure}

Click an account to view its information, which is organized into these
tabs:

\begin{itemize}
\tightlist
\item
  Overview
\item
  Activities
\item
  Interests
\item
  Segments
\item
  Individuals
\item
  Details
\end{itemize}

The sections that follow describe each tab.

\subsection{Overview}\label{overview-2}

The Overview tab presents summaries of the account's data. It displays
these summaries via these panes:

\begin{itemize}
\tightlist
\item
  Account Activities
\item
  Account Firmographics
\item
  Contact Information
\item
  Account Interest Topics
\item
  Known Individuals
\item
  Associated Segments
\end{itemize}

If more information is available for each tab, you can access it via its
link. For example, a \emph{View All Activities} link is at the bottom of
the Account Activities pane. Clicking this link takes you to the
Activities tab, which shows more detailed information about activities.

\begin{figure}
\centering
\includegraphics{./images/accounts-overview.png}
\caption{The Overview tab displays summaries of the account's data.}
\end{figure}

\subsection{Activities}\label{activities}

The Activities tab shows the activities of the individuals in the
account. A histogram shows the number of activities by date and the
average engagement score for the account's individuals. A searchable
table below the histogram shows more detail about the activities.

\begin{figure}
\centering
\includegraphics{./images/accounts-activities.png}
\caption{The Activities tab displays data on activities of the account's
individuals.}
\end{figure}

\subsection{Interests}\label{interests-3}

The Interests tab shows the topics of interest for the account's
individuals. Analytics Cloud identifies these topics via the same
methodology it uses for individuals. For more information, see
\href{https://help.liferay.com/hc/en-us/articles/360006947951-Customer-Insights}{Customer
Insights}.

\subsection{Segments}\label{segments-1}

The Segments tab shows the segments for the account's individuals. These
segments function the same as segments in general. The only difference
is that the segments here apply to the individuals in the account. For
more information on segments, see the documentation on
\href{https://help.liferay.com/hc/en-us/articles/360006947671-Creating-Segments}{creating}
and
\href{https://help.liferay.com/hc/en-us/articles/360006947851-Profiling-Segments}{profiling}
segments.

\subsection{Individuals}\label{individuals}

The Individuals tab shows information about the individuals in the
account. For more information on individuals in Analytics Cloud, see
\href{https://help.liferay.com/hc/en-us/articles/360006946171-Profiling-Individuals}{Profiling
Individuals}.

\subsection{Details}\label{details-1}

The Details tab shows information about the account's properties. The
properties appear in a searchable table that shows each account property
in Analytics Cloud and its value. The table also has columns for these
values:

\textbf{Source Name:} The corresponding property name in the Salesforce
data source. For example, the Analytics Cloud \texttt{accountId}
property is \texttt{id} in Salesforce. Therefore, \texttt{id} appears in
the Source Name column of the \texttt{accountId} row.

\textbf{Data Source:} The data source the property's value originated
from.

\textbf{Last Modified:} The date the property's value was most recently
changed.

\begin{figure}
\centering
\includegraphics{./images/salesforce-accounts-details.png}
\caption{The Details tab contains a searchable table that shows the
account's properties and their values.}
\end{figure}

\chapter{Finding Analytics Data}\label{finding-analytics-data}

Combining traditional page analytics with site, path, and asset
analytics sets Liferay Analytics Cloud apart from other analytics tools.
Analytics Cloud's site-wide report feature provides a comprehensive
report of your entire site. This helps you understand how many unique
visitors your site has over time, which pages these visitors access,
what their interests are, and more.

Analytics cloud also provides analytics for individual pages. This
includes views, visitors, and bounce rate. Path analytics let you see
how visitors arrive at your pages. This includes information on which
pages they visit prior to yours, and their location and device type. In
addition, several Liferay DXP Assets have even more fine-grained metrics
to drill deeper than the page level to determine how you can improve
your site's performance.

Here, you'll learn how to find the site, page, and asset analytics:

\begin{itemize}
\tightlist
\item
  \hyperref[site-report]{Site Report}
\item
  \hyperref[viewing-page-data]{Viewing Page Data}
\item
  \hyperref[finding-asset-data]{Finding Asset Data}
\end{itemize}

The rest of the articles in this section explain these metrics in
detail.

\begin{figure}
\centering
\includegraphics{./images/pages-touchpoints-menu.png}
\caption{Analytics Cloud provides important data for Sites and Assets.}
\end{figure}

\section{Site Report}\label{site-report}

Liferay Analytics Cloud provides a single Site report for each connected
data source. For example, if you connect Analytics Cloud to one DXP
instance and configure analytics for two subsites, one Site report
aggregating both subsites is created.

Follow these steps to view the Site report:

\begin{enumerate}
\def\labelenumi{\arabic{enumi}.}
\item
  In the \emph{Touchpoints} section of the menu, click \emph{Sites}.
\item
  Click the \emph{Overview} tab (this tab is selected by default when
  you click \emph{Sites}). This tab contains the Site report.
\end{enumerate}

The Site report contains the following data:

\begin{itemize}
\tightlist
\item
  \hyperref[site-metrics]{Site Metrics}
\item
  \hyperref[top-pages]{Top Pages}
\item
  \hyperref[acquisitions]{Acquisitions}
\item
  \hyperref[visitors-by-day-and-time]{Visitors by Day and Time}
\item
  \hyperref[search-terms-and-interests]{Search Terms and Interests}
\item
  \hyperref[sessions-by-location]{Sessions by Location}
\item
  \hyperref[session-technology]{Session Technology}
\item
  \hyperref[cohort-analysis]{Cohort Analysis}
\end{itemize}

\subsection{Site Metrics}\label{site-metrics}

The Site Metrics panel presents a summary of how visitors interact with
your Site. This panel contains the following metrics:

\begin{itemize}
\tightlist
\item
  \textbf{Visitors:} Total unique visitors.
\item
  \textbf{Sessions per Visitor:} An average of the number of sessions
  for each unique visitor. A single user can open multiple sessions. A
  session ends after 30 minutes of inactivity, or at midnight.
\item
  \textbf{Session Duration:} The length of time an average session
  lasts.
\item
  \textbf{Bounce Rate:} The percentage of visitors who view your Site's
  first page, but do nothing else before the session ends.
\item
  \textbf{Engagement:} A proprietary metric that Analytics Cloud
  calculates to determine how engaged a visitor is with your Site. This
  is derived from session duration, pages visited, scroll depth, time on
  page, and more.
\end{itemize}

Clicking each metric changes the visualization in the panel to display
the selected metric.

\begin{figure}
\centering
\includegraphics{./images/site-metrics.png}
\caption{The Site Metrics panel shows how visitors are using your site.}
\end{figure}

\subsection{Top Pages}\label{top-pages}

The Top Pages panel lists the following:

\begin{itemize}
\tightlist
\item
  \textbf{Visited Pages:} Your Site's most visited pages and the number
  of visitors for each.
\item
  \textbf{Entrance Pages:} The most common pages that visitors view
  first upon entering your Site, and the number of entrances for each.
\item
  \textbf{Exit Pages:} The most common pages that visitors view when
  leaving your Site, and the exit percentage for each. The exit
  percentage is the percentage of visitors for which the page is the
  last page in their session.
\end{itemize}

To view more detailed page analytics, click \emph{View Pages} in the
panel. Alternatively, click the \emph{Pages} tab at the top of the
screen. Both take you to the page analytics discussed in
\hyperref[viewing-page-data]{Viewing Page Data}.

\begin{figure}
\centering
\includegraphics{./images/top-pages.png}
\caption{The Top Pages panel lists your Site's most relevant pages.}
\end{figure}

\subsection{Acquisitions}\label{acquisitions}

The Acquisitions panel shows how visitors arrive to your Site. It shows
data for the channels, source/medium, and referrers from which visitors
arrive to your Site.

This report works best if your marketing campaign utilizes UTM
parameters. UTM parameters allow Analytics Cloud to determine where
visitors arrive from (e.g., the specific referrer or ad campaign).

\begin{figure}
\centering
\includegraphics{./images/acquisitions.png}
\caption{The Acquisitions panel shows how visitors get to your Site.}
\end{figure}

\subsection{Visitors by Day and Time}\label{visitors-by-day-and-time}

The Visitors by Day and Time panel visualizes the days and times when
visitors come to your Site. This helps you understand when your Site is
most active. You can use this information, for example, to know when to
release important information or launch an advertising campaign.

The panel contains a grid with the days of the week on one axis, and the
time of day on the other axis. Darker cells in the grid indicate heavier
Site traffic at the corresponding day and time. Tooltips for each cell
show the number of visitors for that day and time.

\begin{figure}
\centering
\includegraphics{./images/day-and-time.png}
\caption{The Visitors by Day and Time panel shows your Site's traffic by
day and time.}
\end{figure}

\subsection{Search Terms and
Interests}\label{search-terms-and-interests}

The Search Terms and Interests panels show your visitors' most common
search terms and the topics they're interested in, respectively. Search
terms are collected from the search query parameter in your Site's URL.
Interest topics are derived from the keyword metadata of the pages that
visitors view. To view the full list of interest topics, click \emph{All
interests} in the Interests panel. Alternatively, click the
\emph{Interests} tab at the top of the screen.

\begin{figure}
\centering
\includegraphics{./images/search-terms-interests.png}
\caption{The Search Terms and Interests panels show your Site's most
common search terms and interest topics.}
\end{figure}

\subsection{Sessions by Location}\label{sessions-by-location}

The Sessions by Location panel shows the countries from which visitors
access your Site. Countries with more visitors are shaded darker on the
map. The countries are also listed below the map along with the number
and percentage of visitors for each.

\begin{figure}
\centering
\includegraphics{./images/sessions-by-location.png}
\caption{The Sessions by Location panel shows where your Site's visitors
are.}
\end{figure}

\subsection{Session Technology}\label{session-technology}

The Session Technology panel shows the devices, operating systems, and
browsers that visitors access your Site with. Tooltips for each graph
element display more detailed data for that element. On the
\emph{Devices} tab, for example, mouse over each bar on the bar graph to
see the operating system data for that device.

\begin{figure}
\centering
\includegraphics{./images/session-technology.png}
\caption{The Session Technology panel shows what visitors use to access
your Site.}
\end{figure}

\subsection{Cohort Analysis}\label{cohort-analysis}

The Cohort Analysis panel shows a
\href{https://en.wikipedia.org/wiki/Cohort_analysis}{cohort analysis}
based on visitors from a specific acquisition date (the cohort) and
whether they return to your Site over a given time period. Use the
selector menus at the top left of the panel to select the visitor type
(All, Anonymous, or Known) and time period (Day, Week, or Month).

For example, if you select \emph{All Visitors} and \emph{Day}, the
percentage of visitors from a given acquisition date that return to your
Site are listed in the chart for each following day.

\begin{figure}
\centering
\includegraphics{./images/cohort-analysis.png}
\caption{The cohort analysis is based on visitors from a specific date
and the percentage of those visitors that return over the time period
that follows.}
\end{figure}

\section{Viewing Page Data}\label{viewing-page-data}

Follow these steps to view the list of Pages with summary data:

\begin{enumerate}
\def\labelenumi{\arabic{enumi}.}
\item
  Find the Touchpoints section of the menu.
\item
  Click \emph{Sites}.
\item
  Click the \emph{Pages} tab.
\end{enumerate}

\begin{figure}
\centering
\includegraphics{./images/pages-list.png}
\caption{: The Page list contains useful summary data.}
\end{figure}

Order Pages in ascending or descending order based on any one of these
metrics. Click the metric's heading in the table to perform the sort:

\begin{itemize}
\tightlist
\item
  Average Engagement Score
\item
  Total Visitors
\item
  Total Views
\item
  Average Bounce Rate
\item
  Average Time on Page
\end{itemize}

The metrics for ordering the Page list are calculated based on the time
period selected in the time period menu (at the top-right of the table).
The following values are supported:

\begin{itemize}
\tightlist
\item
  Last 24 hours
\item
  Yesterday
\item
  Last 7 days
\item
  Last 28 days
\item
  Last 30 days (default)
\item
  Last 90 days
\end{itemize}

To view detailed metrics for a single page, click that page in the
table. See
\href{https://github.com/liferay/liferay-docs/blob/7.1.x/discover/analytics-cloud/articles/04-analyzing-touchpoints/01-page-analytics.markdown}{Page
Analytics} for more information.

\begin{figure}
\centering
\includegraphics{./images/pages-overview.png}
\caption{: Most Page metrics are available in the Overview tab.}
\end{figure}

\section{Finding Asset Data}\label{finding-asset-data}

\begin{enumerate}
\def\labelenumi{\arabic{enumi}.}
\item
  Find the Touchpoints section of the menu.
\item
  Click \emph{Assets} for a list of Assets with summary data.
\item
  Choose an Asset type: Blogs, Documents and Media, Forms, Web Content,
  or Custom.
\end{enumerate}

\begin{figure}
\centering
\includegraphics{./images/assets-list.png}
\caption{: The Assets appear in a table.}
\end{figure}

To order Assets in ascending or descending order based on any one of the
Asset's metrics, click the metric's heading in the table. The metrics
differ between Assets. See each Asset's documentation for more
information. In addition, you can calculate the metrics over a selected
time period by selecting one of the following values from the time
period menu (at the top-right of the table):

\begin{itemize}
\tightlist
\item
  Last 24 hours
\item
  Yesterday
\item
  Last 7 days
\item
  Last 28 days
\item
  Last 30 days (default)
\item
  Last 90 days
\end{itemize}

Click an Asset in the list to see more granular detail. Remember that
there are separate lists for different Asset types. More details on each
Asset type's metrics are covered in the subsequent articles.

\begin{figure}
\centering
\includegraphics{./images/assets-overview.png}
\caption{: Once you find an Asset, click it to see its metrics.}
\end{figure}

\section{Understanding Page
Analytics}\label{understanding-page-analytics}

How are your Site Pages performing? Are visitors abandoning your site
when they hit a certain Page? Is one Page constantly getting views,
engaging users, and drawing visitors back to your site? Liferay
Analytics Cloud answers these questions and more.

Analytics Cloud does the following for pages:

\begin{itemize}
\tightlist
\item
  Registers and tracks pages for analysis the first time the Analytics
  Cloud client detects a Page interaction.
\item
  Queries pages on a schedule for up-to-date data.
\item
  Reports on pages with the metrics described here.
\end{itemize}

For instructions on accessing the page analytics discussed here, see
\href{https://github.com/liferay/liferay-docs/blob/7.1.x/en/discover/analytics-cloud/articles/04-analyzing-touchpoints/00-analytics-intro.markdown\#viewing-page-data}{Viewing
Page Data}.

\subsection{Data Time Periods}\label{data-time-periods}

All Page data in Analytics Cloud appears for a specified time period.
The time period selector recalculates the metrics for the selected time
period. You must specify a time period to view the metrics. The default
is 30 days, but this is configurable. The following values are
supported:

\textbf{Last 24 hours:} Displays data generated over the last 24 hours.

\textbf{Yesterday:} Displays all data generated yesterday, beginning at
12:00 AM and ending at 11:59 PM.

\textbf{Last 7 days:} Displays data generated for the previous seven
full days (the current day is excluded), in daily increments.

\textbf{Last 28 days:} Displays data generated for the previous 28 full
days (the current day is excluded), in weekly increments.

\textbf{Last 30 days (default):} Displays data generated for the
previous 30 full days (the current day is excluded), in weekly
increments.

\textbf{Last 90 days:} Displays data generated over a 13 week period
beginning 90 days prior to the end of the current week. The time period
always begins on a Sunday, and ends with the current, incomplete week
(unless viewed on a Saturday approaching midnight).

Note that the first Sunday of the 90 day time period is not included.

\subsection{Visitor Behavior}\label{visitor-behavior}

The Overview tab of a Page's metrics has several data presentations. The
first chart, called Visitor Behavior, contains these metrics:

\begin{itemize}
\tightlist
\item
  Average Engagement Score
\item
  Unique Visitors
\item
  Total Views
\item
  Bounce Rate
\item
  Average Time on Page
\end{itemize}

Select the time period for the data displayed. There are always two
trend lines displayed: Selected Period and Previous Period. This
facilitates comparisons between time periods.

\begin{figure}
\centering
\includegraphics{./images/pages-visitors-behavior.png}
\caption{The Visitors Behavior chart contains interesting trend lines.}
\end{figure}

So what are these metrics in the Visitor Behavior chart?

\subsubsection{Metric: Engagement}\label{metric-engagement}

Engagement, or average Page Engagement is an aggregation of metrics into
one overall score. It incorporates these factors:

\begin{itemize}
\tightlist
\item
  Depth of scroll on the Page
\item
  Number of clicks
\item
  Time spent on the Page
\item
  More!
\end{itemize}

Use the engagement score as a high level view of the Page's overall
performance, as compared with other Pages. It might not tell you
specifically what's so effective (or weak) about a Page, but it can tell
you if the Page is performing as desired over the selected time period.

\textbf{Baseball Analogy:} Are you a baseball fan? If so you might be
familiar with the classic individual metrics used to describe a
non-pitcher's proficiency: Batting Average, Slugging Percentage,
Fielding Percentage, and a few more. Now there are higher level
aggregation metrics used to evaluate baseball players, such as BABIP
(Batting Average on Balls In Play) and WAR (Wins Above Replacement).
Discussion of those metrics is not the point here. Think of the simpler
metrics offered by Analytics Cloud (Views, Visitors, Time on Page) as
single snapshots into how useful a Page is, like looking at Batting
Average and Fielding Percentage to evaluate baseball players. Think of
Page Engagement as an advanced aggregation metric that captures the
overall usefulness of a Page, similar to BABIP or WAR in baseball. Maybe
the Analytics team can come up with a WAR-like metric next, to compare
how valuable a Page is as compared with the average Page. It could be
called TEAR (Page Engagement Above Replacement).

Page Engagement is useful to combine with time period filtering and
comparative time period features. Comparing the engagement score from
different periods is the best way to determine how your Page's
performance changes over time.

\subsubsection{Metric: Unique Visitors}\label{metric-unique-visitors}

Useful with the time period filter, Page Visitors is the number of
visitors that accessed a Page in a given period of time.

A unique visitor has a unique IP address in Analytics Cloud. Therefore,
if the same person views the Page from a different device, it's logged
as two unique visitors.

\subsubsection{Metric: Page Views}\label{metric-page-views}

Useful with the time period filter, Page View is the number of views for
a Page Page in a given period of time. It's not the same as the number
of visitors, because it doesn't try to count only unique IP addresses.
Over the last 30 days, one visitor (IP address) could come back to the
Page 100 times. That means there are 100 Page Views, but only one
visitor (assuming she used the same machine to access the Page each
time. However, a unique view isn't logged for a single user unless at
least 30 minutes of inactivity on the Page passes before the user
interacts with the Page again.

\subsubsection{Metric: Page Bounce Rate}\label{metric-page-bounce-rate}

Bounce Rate is the percentage of visitors to the Page that navigated
away from the site without any page interaction (including scrolling on
the page) after the initial page load. It's calculated as a daily rate
(percentage per day), and the daily rate trend line is displayed over
the selected time period.

\subsubsection{Metric: Time on Page}\label{metric-time-on-page}

Time on Page calculates the average time spent on a Page for all the
Views each day. It's displayed for the selected time period.

This metric is calculated like this for each 24 hour period:

\begin{verbatim}
(view-1-time + view-2-time + ...) / total-number-views
\end{verbatim}

That concludes the Visitors Behavior chart, but see below for more Page
data.

\subsection{Audience}\label{audience}

The Audience report uses charts to present information about the
audience interaction with the Page. It answers these questions:

\begin{itemize}
\tightlist
\item
  How many users are interacting with my content?
\item
  Of these users, how many are known or anonymous?
\item
  Of the known users who interacted with my content, how many belong to
  Analytics Cloud Segments?
\item
  Of the users in Segments, what are the top 15 segments?
\end{itemize}

Here are the charts in the Audience report:

\textbf{Unique Visitors:} A donut chart that presents the percentage of
total unique visitors who are known or anonymous. The center of the
chart shows the total number of unique visitors. Tooltips on each chart
section show the number of users for that section (e.g, the number of
known or anonymous users).

\textbf{Known Individuals:} A donut chart that presents the percentage
of known individuals who belong or do not belong to one or more
Analytics Cloud Segment. The center of the chart shows the total number
of known individuals. Tooltips on each chart section show the number of
users for that section (e.g, the number of users belonging to one or
more Segments). Click the chart title (Known Individuals) to view a list
of all the known individuals who comprise the chart's data.

\textbf{Known Individuals Segments:} A bar chart that shows the
percentage of known individuals that comprise each Segment. The chart
shows a bar for each of the top five Segments, and then aggregates the
remaining Segments into the last bar. A tooltip on the last bar shows
the values for each of the remaining Segments.

\begin{figure}
\centering
\includegraphics{./images/audience-report.png}
\caption{The Audience report visualizes how individuals interact with
the Page.}
\end{figure}

\subsection{Views by Location}\label{views-by-location}

The map in the Views by Location panel shows the number of views by
country in the selected time period.

\begin{figure}
\centering
\includegraphics{./images/pages-views-location.png}
\caption{See where the Page is most popular.}
\end{figure}

\subsection{Views by Technology}\label{views-by-technology}

View a stacked bar graph of the Page's views by operating system
(grouped by device type) in the default tab. Hover over each bar to see
the detailed breakdown of data.

\begin{figure}
\centering
\includegraphics{./images/pages-views-os.png}
\caption{Which OS is most commonly being used to access the Page?}
\end{figure}

Click \emph{Web Browser} to see a donut chart displaying up to the top
eight web browsers over the selected time period. If applicable,
remaining web browsers are aggregated in the ninth donut segment.

\begin{figure}
\centering
\includegraphics{./images/pages-views-browser.png}
\caption{Which browser should your page be optimized for?}
\end{figure}

\subsection{Assets}\label{assets}

View a list of the Assets on the Page by their number of Interactions
over the selected time period.

Depending on the Asset being viewed, a different Interaction metric is
reported:

\begin{itemize}
\tightlist
\item
  Blogs reports Views.
\item
  Documents and Media reports Downloads.
\item
  Forms reports Submissions.
\item
  Web Content reports Views.
\end{itemize}

\begin{figure}
\centering
\includegraphics{./images/pages-assets.png}
\caption{Which Assets on the Page are getting the most interactions?}
\end{figure}

After all those fundamental metrics, you're really getting to know your
Pages. But there's some interesting Page data you haven't seen. Discover
how people came to the Page in the first place. Learn about Path
Analytics next.

\section{Path Analytics}\label{path-analytics-1}

A weary sojourner winds his way through the twisted paths of the
Internet, seeking answers to life's most important questions. Nearly
ready to give up his search and instead look at Grumpy Cat memes, he
scrolls to the bottom of the page he's on, skimming the page's content.
His weary eyes come to rest on a promising link offered at the very
bottom of the page. He clicks it in desperation, hoping it offers the
solution to his pressing problem. Through this haphazard placement of an
advertisement, he is led to your site, finds exactly the answers he
seeks, and becomes a regular visitor or even a customer. If you saw this
become a discernible pattern, with a certain page consistently driving
visitors to your site, you'd want to target advertising on this page,
wouldn't you? Path Analytics identify where the sojourners to your site
most commonly originate. Now, instead of hoping they stumble upon your
links as you leave them in the gnarled and twisted tubes of the
Internet, you can target your placement of links, investing where you're
confident they'll be most effective and abandoning fruitless advertising
campaigns.

By knowing the Paths your page Visitors use to come to your Page, you'll
make better decisions about whether to continue, discontinue, or modify
your advertising.

\subsection{Viewing Page Path
Analytics}\label{viewing-page-path-analytics}

A Page's Path Analytics are reported in the Path tab of a Page's screen.
The diagram reports the most common pages from which visitors accessed
this Page.

\begin{figure}
\centering
\includegraphics{./images/paths-diagram.png}
\caption{On top of being pretty to look at, the Path Diagram contains
important information.}
\end{figure}

After the top three paths to the Page, the remaining paths are
aggregated to show how many views came from \emph{Other} pages.

If a Page in one of your Liferay DXP sites is also a Path to the current
Page, clicking the path URL brings you to its Overview screen. If it's
not a page that's loaded from the data source, nothing happens.

\subsection{Location and Device Type}\label{location-and-device-type}

Path Analytics can be filtered by Location and Device Type. Knowing the
Location and Device Type visitors most commonly use to access your
site's Pages helps you determine whether your site and its content are
optimized properly.

By default, all views of the Page in the selected time period are
represented in the Paths diagram. Filter the Paths being viewed by
Location and Device Type. Click the \emph{Filter} menu and select one
Location whose Path diagram you want to see. The Path diagram is updated
to represent the top Paths for only the selected Location. The same
filtering can be used for Device Type, and the Device Type and Location
filters can be combined. For example, view the Path diagram for views
coming from smartphones in Germany.

As you select filters, they're made visible at the top center of the
Path screen.

\begin{figure}
\centering
\includegraphics{./images/paths-filters.png}
\caption{One German smartphone User visited this Page in the last 30
days. The User came to the Page from \texttt{wwww.google.com.de}.}
\end{figure}

To remove a filter, click the \emph{x} next to the filter name.

\subsection{Assets}\label{assets-1}

Pages have Assets on them. If a Page containing the Analytics Cloud
client is reported in the Path diagram, a \emph{Show Top 5 Assets} link
is displayed. Clicking it opens a drop-down list. Up to the top five
assets on the Page is displayed, as determined by interaction with the
Assets. Each Asset uses the most appropriate interaction metric
available:

\begin{itemize}
\tightlist
\item
  Forms uses Submissions
\item
  Blogs uses Views
\item
  Documents and Media uses Downloads
\item
  Web Content uses Views
\end{itemize}

\begin{figure}
\centering
\includegraphics{./images/paths-assets.png}
\caption{The top 5 assets are listed for each page.}
\end{figure}

Clicking an asset brings you to the Asset's Overview page. See the
documentation on Assets for more information on analyzing Asset metrics
.

\section{Assets: Analyzing Content}\label{assets-analyzing-content}

Assets are individual pieces of content that exist on Site Pages in your
Liferay Analytics Cloud data sources. In summary, Assets are

\begin{itemize}
\tightlist
\item
  Content residing on a Page.
\item
  Registered, tracked, and reported on by Analytics Cloud the first time
  an interaction is detected by the Analytics Cloud client.
\item
  Queried and reported on at regular intervals.
\item
  Reported on in Analytics Cloud with the data reports shown in this
  tutorial.
\end{itemize}

Metrics for these Assets are currently reported in Analytics Cloud:

\begin{itemize}
\tightlist
\item
  \href{/docs/7-1/user/-/knowledge_base/u/forms}{Forms}
\item
  \href{/docs/7-1/user/-/knowledge_base/u/publishing-blogs}{Blogs}
\item
  \href{/docs/7-1/user/-/knowledge_base/u/managing-documents-and-media}{Documents
  and Media}
\item
  \href{/docs/7-1/user/-/knowledge_base/u/authoring-content-structured-and-inline-content}{Web
  Content}
\end{itemize}

Continue reading for details on how to interpret Analytics Cloud data to
better understand your assets' performance.

\begin{figure}
\centering
\includegraphics{./images/assets-table.png}
\caption{Each Asset has its own table.}
\end{figure}

Each Asset is covered in a separate tutorial.

\section{Form Analytics}\label{form-analytics}

\href{/docs/7-1/user/-/knowledge_base/u/forms}{Forms} are important
direct data gathering tools for enterprises. Are your web forms
providing you invaluable information or turning users away? Analytics
Cloud gives you important insights.

In Analytics Cloud, find the Touchpoints section of the menu and click
\emph{Assets}. There are four tabs, each displaying a paginated list of
the Asset Type indicated by the tab title. Click \emph{Forms}.

\begin{figure}
\centering
\includegraphics{./images/assets-forms.png}
\caption{By default, Forms are listed in descending order of
Submissions.}
\end{figure}

View the summary metrics for an Asset directly in the list, and click on
an Asset to view its detail page.

\subsection{Visitor Behavior}\label{visitor-behavior-1}

As with all Asset types, the Visitors Behavior chart is at the top of
its page and provides a line graph with several trend lines. Choose from
four important metrics:

\begin{itemize}
\tightlist
\item
  Total Submissions
\item
  Total Views
\item
  Abandonment rate
\item
  Average Completion Time
\end{itemize}

\begin{figure}
\centering
\includegraphics{./images/assets-forms-vb.png}
\caption{The Visitors Behavior chart contains important trend lines.}
\end{figure}

\subsubsection{Submissions}\label{submissions}

Submissions counts the number of times the Submit button was clicked on
a Form. It's the gold standard metric for a form, because that's why the
form was created in the first place: to collect data entered into the
form. If the Submit button isn't clicked, you don't get the data you
wanted.

The Submissions trend line shows the number of times the Submit button
was clicked each day (or hour, if \emph{Yesterday} or \emph{Last 24
Hours} are selected) over the selected time period.

\subsubsection{Views}\label{views}

Views is a common metric among all Assets (and Pages).

Useful with the time period filter, Views is the number of views for a
Form in a given period of time. It's not the same as the number of
visitors, because it doesn't try to count only unique IP addresses. So
over the last 30 days, one visitor (IP address) could come back to the
Page 100 times. That means there are 100 Page Views, but only one
visitor (assuming the same device was used to access the Page each
time). However, a unique view isn't logged for a single user unless at
least 30 minutes of inactivity on the Page passes before the user
interacts with the Page again.

\subsubsection{Abandonment}\label{abandonment}

Abandonment is the daily (or hourly if \emph{Yesterday} or \emph{Last 24
Hours} is selected) percentage of users that interacted with the form
but stopped short of submitting an entry over the selected time period.

\subsubsection{Completion Time}\label{completion-time}

Completion time is a daily average (or hourly if \emph{Yesterday} or
\emph{Last 24 Hours} is selected) of the time it took for form users to
go from their first interaction with the form until they hit the Submit
button.

The time series metrics displayed in the Visitors Behavior chart are
paramount to understanding Asset performance over time. But there's more
to Asset Analytics.

\subsection{Audience}\label{audience-1}

The Audience report uses charts to present information about the
audience interaction with the asset. It answers these questions:

\begin{itemize}
\tightlist
\item
  How many users are interacting with my content?
\item
  Of these users, how many are known or anonymous?
\item
  Of the known users who interacted with my content, how many belong to
  Analytics Cloud Segments?
\item
  Of the users in Segments, what are the top 15 segments?
\end{itemize}

Here are the charts in the Audience report:

\textbf{Unique Visitors:} A donut chart that presents the percentage of
total unique visitors who are known or anonymous. The center of the
chart shows the total number of unique visitors. Tooltips on each chart
section show the number of users for that section (e.g, the number of
known or anonymous users).

\textbf{Known Individuals:} A donut chart that presents the percentage
of known individuals who belong or do not belong to one or more
Analytics Cloud Segment. The center of the chart shows the total number
of known individuals. Tooltips on each chart section show the number of
users for that section (e.g, the number of users belonging to one or
more Segments). Click the chart title (Known Individuals) to view a list
of all the known individuals who comprise the chart's data.

\textbf{Known Individuals Segments:} A bar chart that shows the
percentage of known individuals that comprise each Segment. The chart
shows a bar for each of the top five Segments, and then aggregates the
remaining Segments into the last bar. A tooltip on the last bar shows
the values for each of the remaining Segments.

\begin{figure}
\centering
\includegraphics{./images/audience-report.png}
\caption{The Audience report visualizes how individuals interact with
the asset.}
\end{figure}

\subsection{Submissions by Location}\label{submissions-by-location}

The map in this panel shows the number of submissions by country in the
selected time period.

\begin{figure}
\centering
\includegraphics{./images/assets-interaction-location.png}
\caption{From which location do users submit the form most frequently?}
\end{figure}

\subsection{Submissions by Technology}\label{submissions-by-technology}

View a stacked bar graph of the Page's submissions by operating system
(grouped by device type) in the default tab.

\begin{figure}
\centering
\includegraphics{./images/assets-forms-sbt.png}
\caption{What technologies are used to interact with the Asset?}
\end{figure}

Click \emph{Web Browser} to see a donut chart displaying up to the top
eight web browsers over the selected time period. If applicable,
remaining web browsers are aggregated in the ninth donut segment.

\subsection{Field Analysis}\label{field-analysis}

The Field Analysis chart is unique to Forms.

Field analysis appears as a bar graph. The height of the first bar shows
the total number of people (Views) who have interacted with the form in
any way, even just scrolling past it on the page. The rest of the bars
show the percentage of those people that interacted with a specific
field by clicking into it. The gross numbers on the right y-axis are the
number of people who interacted with the field by clicking into it.

Field Analysis gives you these metrics on each form field:

\begin{itemize}
\item
  Number of clicks into the field
\item
  Percentage of Users who abandoned the form at this field.
\item
  The time spent interacting with the field (includes abandonments)
\item
  The refill percent for the field. Out of the total number that filled
  out the field, this answers the question, ``What percentage had to
  edit or enter new information into the field after moving on in the
  form?''
\end{itemize}

\begin{figure}
\centering
\includegraphics{./images/assets-forms-fa.png}
\caption{See how Form Users interact with the Form's fields.}
\end{figure}

Field Analysis helps you get down to the field level in diagnosing
potential issues with your forms. For example, in your Newsletter
Sign-Up form, if you included a required field that asked for users to
provide their first child's full name, you may see a lot of form
abandonment on that field, not only because some form users don't have
children. If you make that field not required, you'll get less
abandonment on that field, and if you remove that field from the form
entirely, you'll eliminate all abandonment from offended parents.

\subsection{Asset Appears On}\label{asset-appears-on}

The Asset Appears On table includes a Page Name and URL. Like any Page
data in Analytics Cloud, only Pages that were interacted with in some
way by site visitors are tracked and reported. It's important to note
that the link doesn't take you to the actual Page with the Asset on it.
Instead, it leads to the Page Analytics view of the Page. From there you
can click the URL to the actual Page.

\begin{figure}
\centering
\includegraphics{./images/assets-appears-on.png}
\caption{What Pages does the Asset appear on?}
\end{figure}

\section{Blogs Analytics}\label{blogs-analytics}

\href{/docs/7-1/user/-/knowledge_base/u/publishing-blogs}{Blogs Entries}
are important content on websites. They attract readers who return to
view new blog posts and interact in discussion.

Analytics Cloud gives you important insights into how engaged readers
are with your blog posts.

\begin{figure}
\centering
\includegraphics{./images/assets-blogs.png}
\caption{By default, Blogs are listed in descending order of Views.}
\end{figure}

\subsection{Visitor Behavior}\label{visitor-behavior-2}

The Overview tab of an Asset's detail presents several data
presentations. The first chart, called Visitor Behavior, contains five
important metrics:

\begin{itemize}
\tightlist
\item
  Total Views
\item
  Average Reading time
\item
  Total Shares
\item
  Total Comments
\item
  Average Rating
\end{itemize}

\begin{figure}
\centering
\includegraphics{./images/assets-blogs-vb.png}
\caption{The Visitors Behavior chart contains important trend lines.}
\end{figure}

\subsubsection{Views}\label{views-1}

The Views plot shows the trend-line for the number of times a Blogs
Entry was viewed each day (or hour, if \emph{Yesterday} or \emph{Last 24
Hours} are selected).

\subsubsection{Reading Time}\label{reading-time}

Reading Time is the average reading time per view of a Blogs Entry. The
daily (or hourly, if \emph{Yesterday} or \emph{Last 24 Hours} are
selected) average reading time per view is plotted for the time period.

\subsubsection{Shares}\label{shares}

The count of total Shares per day is plotted to establish the Shares
trend-line.

\subsubsection{Comments}\label{comments}

The Comments trend-line plots the total number of comments on a Blogs
Entry per day (or hour, if \emph{Yesterday} or \emph{Last 24 Hours} are
selected) over the selected time period.

\subsubsection{Rating}\label{rating}

\href{/docs/7-1/user/-/knowledge_base/u/displaying-blogs}{Readers can
rate blogs.}. Whether the rating is a simple Thumb Up/Thumbs Down, or a
Stars rating (0-5 Stars), the Rating in Analytics Cloud lets you compare
Blog posts to each other. A Blog post announcing Free Swag may have a
higher rating than a Blog post announcing that prices are increasing.
The rating is reported as a number out of 10.

\subsection{Audience}\label{audience-2}

The Audience report uses charts to present information about the
audience interaction with the asset. It answers these questions:

\begin{itemize}
\tightlist
\item
  How many users are interacting with my content?
\item
  Of these users, how many are known or anonymous?
\item
  Of the known users who interacted with my content, how many belong to
  Analytics Cloud Segments?
\item
  Of the users in Segments, what are the top 15 segments?
\end{itemize}

Here are the charts in the Audience report:

\textbf{Unique Visitors:} A donut chart that presents the percentage of
total unique visitors who are known or anonymous. The center of the
chart shows the total number of unique visitors. Tooltips on each chart
section show the number of users for that section (e.g, the number of
known or anonymous users).

\textbf{Known Individuals:} A donut chart that presents the percentage
of known individuals who belong or do not belong to one or more
Analytics Cloud Segment. The center of the chart shows the total number
of known individuals. Tooltips on each chart section show the number of
users for that section (e.g, the number of users belonging to one or
more Segments). Click the chart title (Known Individuals) to view a list
of all the known individuals who comprise the chart's data.

\textbf{Known Individuals Segments:} A bar chart that shows the
percentage of known individuals that comprise each Segment. The chart
shows a bar for each of the top five Segments, and then aggregates the
remaining Segments into the last bar. A tooltip on the last bar shows
the values for each of the remaining Segments.

\begin{figure}
\centering
\includegraphics{./images/audience-report.png}
\caption{The Audience report visualizes how individuals interact with
the asset.}
\end{figure}

\subsection{Views by Location}\label{views-by-location-1}

The map in the Views by Location panel shows the number of views by
country in the selected time period.

\begin{figure}
\centering
\includegraphics{./images/assets-interaction-location.png}
\caption{From which location do users interact with the Asset most
frequently?}
\end{figure}

\subsection{Views by Technology}\label{views-by-technology-1}

View a stacked bar graph of the Page's views by operating system
(grouped by device type) in the default tab.

\begin{figure}
\centering
\includegraphics{./images/assets-dm-dbt.png}
\caption{What technologies are used to interact with the Asset?}
\end{figure}

Click \emph{Web Browser} to see a donut chart displaying up to the top
eight web browsers over the selected time period. If applicable,
remaining web browsers are aggregated in the ninth donut segment.

\subsection{Asset Appears On}\label{asset-appears-on-1}

The Asset Appears On table includes a Page Name and URL. Like any Page
data in Analytics Cloud, only Pages that were interacted with in some
way by site visitors are tracked and reported. It's important to note
that the link doesn't take you to the actual Page with the Asset on it.
Instead, it leads to the Page Analytics view of the Page. From there you
can click the URL to the actual Page.

\begin{figure}
\centering
\includegraphics{./images/assets-appears-on.png}
\caption{What Pages does the Asset appear on?}
\end{figure}

\section{Documents and Media
Analytics}\label{documents-and-media-analytics}

\href{/docs/7-1/user/-/knowledge_base/u/managing-documents-and-media}{Documents
and Media files} are assets you want site users to look at or download.

Are your Documents and Media engaging site visitors? Analytics Cloud can
give you important insights.

\begin{figure}
\centering
\includegraphics{./images/assets-dm.png}
\caption{By default, Documents and Media files are listed in descending
order of Downloads.}
\end{figure}

\subsection{Visitor Behavior}\label{visitor-behavior-3}

The Overview tab of an Asset's detail presents several data
presentations. The first chart, called Visitor Behavior, contains four
important metrics:

\begin{itemize}
\tightlist
\item
  Total Downloads
\item
  Total Previews
\item
  Total Comments
\item
  Average Rating
\end{itemize}

\begin{figure}
\centering
\includegraphics{./images/assets-dm-vb.png}
\caption{The Visitors Behavior chart contains important trend lines.}
\end{figure}

\subsubsection{Downloads}\label{downloads}

The Downloads trend-line shows the total number of downloads per day (or
hour, if \emph{Yesterday} or \emph{Last 24 Hours} are selected) over the
selected time period.

\subsubsection{Previews}\label{previews}

Documents and Media Documents can be
\href{/docs/7-1/user/-/knowledge_base/u/viewing-file-previews}{previewed}.
Even if not ultimately downloaded, previewing shows interaction with the
file and is therefore an important metric. The Previews trend line shows
the total number of previews per day (or hour, if \emph{Yesterday} or
\emph{Last 24 Hours} are selected) for the selected time period.

\subsubsection{Comments}\label{comments-1}

Comments can be allowed on Documents and Media files. If enabled, the
Comments trend line shows the total number of Comments on a file per day
(or hour, if \emph{Yesterday} or \emph{Last 24 Hours} are selected) over
the selected time period.

\subsubsection{Rating}\label{rating-1}

Ratings can be enabled on Documents and Media. Whether the rating is a
simple Thumb Up/Thumbs Down, or a Stars rating (0-5 Stars), the Rating
in Analytics Cloud is a calculation that lets you compare how Users
evaluate the Site's Documents and Media Files.

\subsection{Audience}\label{audience-3}

The Audience report uses charts to present information about the
audience interaction with the asset. It answers these questions:

\begin{itemize}
\tightlist
\item
  How many users are interacting with my content?
\item
  Of these users, how many are known or anonymous?
\item
  Of the known users who interacted with my content, how many belong to
  Analytics Cloud Segments?
\item
  Of the users in Segments, what are the top 15 segments?
\end{itemize}

Here are the charts in the Audience report:

\textbf{Unique Visitors:} A donut chart that presents the percentage of
total unique visitors who are known or anonymous. The center of the
chart shows the total number of unique visitors. Tooltips on each chart
section show the number of users for that section (e.g, the number of
known or anonymous users).

\textbf{Known Individuals:} A donut chart that presents the percentage
of known individuals who belong or do not belong to one or more
Analytics Cloud Segment. The center of the chart shows the total number
of known individuals. Tooltips on each chart section show the number of
users for that section (e.g, the number of users belonging to one or
more Segments). Click the chart title (Known Individuals) to view a list
of all the known individuals who comprise the chart's data.

\textbf{Known Individuals Segments:} A bar chart that shows the
percentage of known individuals that comprise each Segment. The chart
shows a bar for each of the top five Segments, and then aggregates the
remaining Segments into the last bar. A tooltip on the last bar shows
the values for each of the remaining Segments.

\begin{figure}
\centering
\includegraphics{./images/audience-report.png}
\caption{The Audience report visualizes how individuals interact with
the asset.}
\end{figure}

\subsection{Downloads by Location}\label{downloads-by-location}

The map in this panel shows the number of downloads by country in the
selected time period.

\begin{figure}
\centering
\includegraphics{./images/assets-interaction-location.png}
\caption{From which location do users download the asset most
frequently?}
\end{figure}

\subsection{Views by Technology}\label{views-by-technology-2}

View a stacked bar graph of the Page's views by operating system
(grouped by device type) in the default tab.

\begin{figure}
\centering
\includegraphics{./images/assets-dm-dbt.png}
\caption{What technologies are used to interact with the Asset?}
\end{figure}

Click \emph{Web Browser} to see a donut chart displaying up to the top
eight web browsers over the selected time period. If applicable,
remaining web browsers are aggregated in the ninth donut segment.

\subsection{Asset Appears On}\label{asset-appears-on-2}

The Asset Appears On table includes a Page Name and URL. Like any Page
data in Analytics Cloud, only Pages that were interacted with in some
way by site visitors are tracked and reported. It's important to note
that the link doesn't take you to the actual Page with the Asset on it.
Instead, it leads to the Page Analytics view of the Page. From there you
can click the URL to the actual Page.

\begin{figure}
\centering
\includegraphics{./images/assets-appears-on.png}
\caption{What Pages does the Asset appear on?}
\end{figure}

\section{Web Content Analytics}\label{web-content-analytics}

\href{/docs/7-1/user/-/knowledge_base/u/introduction-web-content}{Web
Content Articles} are important assets for presenting site visitors with
information. Analytics Cloud gives you important insights into how site
users are engaging with your content.

\begin{figure}
\centering
\includegraphics{./images/assets-wc.png}
\caption{By default, Web Content is listed in descending order of
Views.}
\end{figure}

\subsection{Visitor Behavior}\label{visitor-behavior-4}

The Overview tab of an Asset's detail presents several data
presentations. The first chart, called Visitor Behavior, contains one
important metric: total Views.

\begin{figure}
\centering
\includegraphics{./images/assets-wc-vb.png}
\caption{The Visitors Behavior chart contains important trend lines.}
\end{figure}

The number of views per day (or hour, if \emph{Yesterday} or \emph{Last
24 Hours} are selected) is plotted over the selected time period to
establish a View trend line. Use this to evaluate how well the content
engages visitors over time.

\subsection{Audience}\label{audience-4}

The Audience report uses charts to present information about the
audience interaction with the asset. It answers these questions:

\begin{itemize}
\tightlist
\item
  How many users are interacting with my content?
\item
  Of these users, how many are known or anonymous?
\item
  Of the known users who interacted with my content, how many belong to
  Analytics Cloud Segments?
\item
  Of the users in Segments, what are the top 15 segments?
\end{itemize}

Here are the charts in the Audience report:

\textbf{Unique Visitors:} A donut chart that presents the percentage of
total unique visitors who are known or anonymous. The center of the
chart shows the total number of unique visitors. Tooltips on each chart
section show the number of users for that section (e.g, the number of
known or anonymous users).

\textbf{Known Individuals:} A donut chart that presents the percentage
of known individuals who belong or do not belong to one or more
Analytics Cloud Segment. The center of the chart shows the total number
of known individuals. Tooltips on each chart section show the number of
users for that section (e.g, the number of users belonging to one or
more Segments). Click the chart title (Known Individuals) to view a list
of all the known individuals who comprise the chart's data.

\textbf{Known Individuals Segments:} A bar chart that shows the
percentage of known individuals that comprise each Segment. The chart
shows a bar for each of the top five Segments, and then aggregates the
remaining Segments into the last bar. A tooltip on the last bar shows
the values for each of the remaining Segments.

\begin{figure}
\centering
\includegraphics{./images/audience-report.png}
\caption{The Audience report visualizes how individuals interact with
the asset.}
\end{figure}

\subsection{Views by Location}\label{views-by-location-2}

The map in the Views by Location panel shows the number of views by
country in the selected time period.

\begin{figure}
\centering
\includegraphics{./images/assets-interaction-location.png}
\caption{From which location do users interact with the Asset most
frequently?}
\end{figure}

\subsection{Views by Technology}\label{views-by-technology-3}

View a stacked bar graph of the Page's views by operating system
(grouped by device type) in the default tab.

\begin{figure}
\centering
\includegraphics{./images/assets-blogs-vbt.png}
\caption{What technologies are used to interact with the Asset?}
\end{figure}

Click \emph{Web Browser} to see a donut chart displaying up to the top
eight web browsers over the selected time period. If applicable,
remaining web browsers are aggregated in the ninth donut segment.

\subsection{Asset Appears On}\label{asset-appears-on-3}

The Asset Appears On table includes a Page Name and URL. Like any Page
data in Analytics Cloud, only Pages that were interacted with in some
way by site visitors are tracked and reported. It's important to note
that the link doesn't take you to the actual Page with the Asset on it.
Instead, it leads to the Page Analytics view of the Page. From there you
can click the URL to the actual Page.

\begin{figure}
\centering
\includegraphics{./images/assets-appears-on.png}
\caption{What Pages does the Asset appear on?}
\end{figure}

\section{A/B Testing Analytics}\label{ab-testing-analytics}

A/B Testing evaluates the effectiveness of Content Pages by testing
multiple versions/layouts of the Page at once. This is done by creating
Page Variants of the original Page, testing the Page with a goal (e.g.,
clicks), and publishing the most effective Variant. You can learn more
about creating an A/B test and configuring it for a Content Page in
Liferay DXP's \href{/docs/7-2/user/-/knowledge_base/u/a-b-testing}{A/B
Testing} documentation.

All results from an A/B test running in Liferay DXP are tracked by
Analytics Cloud. An A/B test is synced with Analytics Cloud once it's
created. From there, you can manage the A/B test from Analytics Cloud.
To view all drafted, running, terminated, and completed A/B tests, go to
the \emph{Tests} menu from the left column.

\begin{figure}
\centering
\includegraphics{./images/ab-test-view.png}
\caption{A complete history of your A/B tests are available in Analytics
Cloud.}
\end{figure}

For a drafted A/B test, you can manage its

\begin{itemize}
\tightlist
\item
  \emph{Target}: the Experience and User Segment.
\item
  \emph{Metric}: the goal to track (e.g., Bounce Rate or Click).
\item
  \emph{Variants}: the Page Variants for users to interact with.
\item
  \emph{Traffic Split}: the percentage of visitors that are randomly
  split between the Variants when visiting the Page.
\item
  \emph{Confidence Level}: the accuracy of the test results.
\end{itemize}

\begin{figure}
\centering
\includegraphics{./images/ab-test-draft-setup.png}
\caption{Analytics Cloud offers a visual checklist for an A/B test's
setup.}
\end{figure}

See Liferay DXP's
\href{/docs/7-2/user/-/knowledge_base/u/a-b-testing}{A/B Testing}
documentation for more information on an A/B test's setup.

Once your A/B test is running, Analytics Cloud offers several reports to
keep you up-to-date on your A/B test's progress:

\begin{itemize}
\tightlist
\item
  \emph{Summary}
\item
  \emph{Variant Report}
\item
  \emph{Test Sessions}
\end{itemize}

You'll learn about these next.

\subsection{Summary}\label{summary}

The Summary panel gives you an overview of your test. It provides you
with information like

\begin{itemize}
\tightlist
\item
  percent completion
\item
  running time (in days)
\item
  total visitor sessions
\end{itemize}

It also gives you a quick glance at your test metric and the best
current performing Variant.

\begin{figure}
\centering
\includegraphics{./images/ab-test-summary.png}
\caption{The Summary panel gives you a quick way to assess the status of
your A/B test.}
\end{figure}

\subsection{Variant Report}\label{variant-report}

The Variant Report panel provides a detailed breakdown of each Variant
and how well they're performing.

\begin{figure}
\centering
\includegraphics{./images/ab-test-variant-report.png}
\caption{Variants are tracked using multiple metrics.}
\end{figure}

Below are the metrics reported for each variant:

\textbf{Median:} the middle number in the set of sample values. This
estimates a typical user's behavior.

\textbf{Confidence Interval:} the range of values expected to contain
the true mean of the population. For example, a 95\% confidence interval
is a range of values that the system is 95\% sure contains the true
mean. This gives the range of possible values that seem plausible for
the measured goal.

\textbf{Improvement:} the relative improvement from the control group.
This metric may also be known as \emph{Lift}. For example, assume the
Control Page has a 15\% retention rate. The improvement calculation
would be \texttt{((16\ -\ 15)\ /\ 15)\ =\ \textasciitilde{}6.67\%}
improvement.

The lets you know the impact of a change. If there is only a small
improvement, it may not be worth implementing that change.

\textbf{Probability to Win:} predicts the likelihood that the Variant
will beat out all other participating Variants. This lets you see how
multiple metrics compare to each other. For example, consider a horse
racing event: each horse has a generated chance to win that is posted
before a race (i.e., odds of winning), calculated by simulating the race
thousands of times. This same method is used for your Variants to
calculate their probability of winning the A/B test.

\textbf{Unique Visitors:} the number of visitors contributing to the
Variant. A visitor randomly assigned a Variant always sees the same
Variant until the test is finished.

Besides knowing how much traffic is hitting a page, this metric also
helps determine if there is an issue with how the A/B test is
configured. For example, there could be too much traffic going to one
Variant (typically caused by a Segment misconfiguration).

\subsection{Test Sessions}\label{test-sessions}

The Test Sessions panel provides statistics showing how many sessions
view your test impressions per day over time. This helps you validate
that your audiences are being directed to your A/B test impressions. It
also portrays how your test affects the traffic to your page compared to
before.

\begin{figure}
\centering
\includegraphics{./images/ab-test-sessions.png}
\caption{This graph gives you an accurate depiction of your page
traffic.}
\end{figure}

Next, you'll learn about an A/B test's statuses.

\subsection{Test Status}\label{test-status}

An A/B test is always characterized with a status after it starts. These
include

\begin{itemize}
\tightlist
\item
  \emph{Test is Running}
\item
  \emph{Winner Declared}
\item
  \emph{No Clear Winner}
\end{itemize}

You'll explore each status next.

\subsubsection{Test is Running}\label{test-is-running}

This means that your test is still running and needs a larger sample
size before declaring a winner. You can still see which Variant is your
current best; however, the desired confidence level has not been met.

\begin{figure}
\centering
\includegraphics{./images/ab-test-current-best.png}
\caption{The leading Variant is clearly labeled as being the current
best.}
\end{figure}

When a test is running, you can terminate it by selecting
\emph{Terminate} from the Summary bar.

\begin{figure}
\centering
\includegraphics{./images/ab-test-terminate.png}
\caption{Terminating an A/B test allows you to delete the test, if
desired.}
\end{figure}

\subsubsection{Winner Declared}\label{winner-declared}

Once your A/B test successfully finishes, a Variant is declared a
winner. At this state, you can perform the following actions:

\begin{itemize}
\tightlist
\item
  publish the winning Variant as your default experience.
\item
  complete the test without publishing any Variants.
\end{itemize}

\begin{figure}
\centering
\includegraphics{./images/ab-test-winner-declared.png}
\caption{Click \emph{Publish Winner} to publish the winning Variant.}
\end{figure}

\subsubsection{No Clear Winner}\label{no-clear-winner}

Sometimes, Analytics Cloud cannot determine a winner because no Variant
has outperformed significantly over the Control Page. In this case, you
can complete the test without publishing anything. The control stays the
default experience.

\begin{figure}
\centering
\includegraphics{./images/ab-test-no-winner.png}
\caption{When the required confidence level is not met during the time
duration, there is no winning Variant.}
\end{figure}

By viewing the generated analytics for your A/B tests, you're constantly
informed on how they're progressing. With the provided data, you can
confidently choose the best Experience for your Site's users.

\section{Content Recommendation API}\label{content-recommendation-api}

The Content Recommendation API in Liferay Analytics Cloud suggests
content based on user interactions with content in a Liferay DXP
instance. This is possible because once Liferay DXP is connected to
Analytics Cloud, it sends interaction events each time a user visits a
page. Each of those events contains information about the content the
user consumes. Analytics Cloud collects and processes this information.

The Content Recommendation API contains services that enable the
following:

\begin{itemize}
\tightlist
\item
  Discover similar tags (content) based on the current tags the user is
  browsing.
\item
  Discover tags based on the user's interest over time.
\end{itemize}

\subsection{Discover Similar Tags}\label{discover-similar-tags}

This service returns a list of tags (terms) similar to those passed as
parameters. The list of similar terms is sorted by weight and returned
as JSON.

Here's this API's endpoint:

\textbf{GET:}
\texttt{\{url\}/api/1.0/interests/terms/related\{?page,size,terms\}}

Here are the parameters:

\texttt{int\ page\ (defaultValue\ =\ 0)}: The page of results. For
example, the default value \texttt{0} specifies the first page of
results.

\texttt{int\ size\ (defaultValue\ =\ 5)}: The number of results to
include on each page. The default value \texttt{5} specifies five
results on each page.

\texttt{List\textless{}String\textgreater{}\ terms}: The tags to use for
determining the list of similar tags. This is the only required
parameter.

\texttt{double\ termWeightThreshold\ (defaultValue\ =\ 0.01)}: The
relevance level (weight) for determining related terms. The default
value of \texttt{0.01} returns all tags with a weight above 1\%.

Together, the \texttt{page} and \texttt{size} parameters control the
number of similar terms to include in the response. Note that this API
can return up to 100 terms. If the number of terms exceeds that limit,
the API returns an error.

For example, here's a JSON response that contains related terms:

\begin{verbatim}
{
  "_embedded": {
    "interest-terms": [
      "jquery",
      "html",
      "sql",
      "mysql",
      "java"
    ]
  },
  "page": {
    "number": 0,
    "size": 5,
    "totalPages": 7,
    "totalElements": 35
  }
}
\end{verbatim}

\subsection{Discover Tags Based on
Interest}\label{discover-tags-based-on-interest}

This service returns a list of tags (terms) that are relevant to a
specific user. You must pass that user's ID as a parameter. The list of
similar terms is sorted by weight and returned as JSON.

Here's this API's endpoint:

\textbf{GET:} \texttt{\{url\}/api/1.0/interests/terms/\{userId\}}

There are also three optional parameters you can use. You can use these
parameters to fine-tune the terms that the API returns. Note that topics
are groups of terms:

\texttt{int\ termsPerTopic\ (defaultValue\ =\ 3)}: The number of terms
to consider per topic/subject.

\texttt{double\ termWeightThreshold\ (defaultValue\ =\ 0.01)}: The
relevance level for determining terms of interest. The default value of
\texttt{0.01} returns all terms with a weight above 1\%.

\texttt{int\ topicsLength\ (defaultValue\ =\ 3)}: The number of topics
to consider.

Decreasing \texttt{termsPerTopic} and increasing the
\texttt{topicsLength} might lead to an increment of the subject
variation (terms from different topics being recommended to users).

Here's an example request that contains only a user ID:

\begin{verbatim}
{url}/api/1.0/interests/terms/953be104-5540-abf8-59b8-55f895200acc
\end{verbatim}

And here's an example response in JSON:

\begin{verbatim}
{
  "_embedded": {
    "interest-topics": [
      {
        "terms": [
          {
            "weight": 0.0945945945945946,
            "keyword": "javascript"
          },
          {
            "weight": 0.08648648648648649,
            "keyword": "jquery"
          },
          {
            "weight": 0.07027027027027027,
            "keyword": "html"
          }
        ],
        "weight": 0.08653350323695352,
        "id": 7
      },
      {
        "terms": [
          {
            "weight": 0.1322314049586777,
            "keyword": "php"
          },
          {
            "weight": 0.06060606060606061,
            "keyword": "sql"
          },
          {
            "weight": 0.05509641873278237,
            "keyword": "mysql"
          }
        ],
        "weight": 0.08027610626914822,
        "id": 1
      },
      {
        "terms": [
          {
            "weight": 0.15204678362573099,
            "keyword": "java"
          },
          {
            "weight": 0.10526315789473684,
            "keyword": "android"
          },
          {
            "weight": 0.023391812865497075,
            "keyword": "multithreading"
          }
        ],
        "weight": 0.07511374008317741,
        "id": 9
      }
    ]
  }
}
\end{verbatim}

\chapter{Managing Your Analytics
Project}\label{managing-your-analytics-project}

Project administrators connect data sources to the project, manage
project users, and track project usage against the current plan limits.

The
\href{https://github.com/liferay/liferay-docs/blob/7.1.x/discover/analytics-cloud/articles/02-getting-started/00-getting-started-intro.markdown}{Getting
Started} articles demonstrate these processes:

\begin{itemize}
\item
  Adding
  \href{https://github.com/liferay/liferay-docs/blob/7.1.x/discover/analytics-cloud/articles/02-getting-started/02-adding-a-liferay-dxp-data-source.markdown}{Liferay
  DXP} and
  \href{https://github.com/liferay/liferay-docs/blob/7.1.x/discover/analytics-cloud/articles/02-getting-started/03-adding-a-csv-data-source.markdown}{CSV}
  data sources.
\item
  \href{https://github.com/liferay/liferay-docs/blob/7.1.x/discover/analytics-cloud/articles/02-getting-started/01-managing-data-sources.markdown}{Managing
  Data Sources}
\end{itemize}

The following articles demonstrate these administrative
responsibilities:

\begin{itemize}
\item
  \href{https://github.com/liferay/liferay-docs/blob/7.1.x/discover/analytics-cloud/articles/05-getting-started/02-managing-users.markdown}{Managing
  Users:} As you're starting your project and new people join your team,
  you'll want to invite teammates to participate in your project. This
  article shows you how to invite users and assign them appropriate
  permission Roles.
\item
  \href{https://github.com/liferay/liferay-docs/blob/7.1.x/discover/analytics-cloud/articles/05-getting-started/03-tracking-usage.markdown}{Tracking
  Usage:} As you sync contact and web analytics data from data sources,
  you must keep an eye on your project's usage compared to the Liferay
  Analytics Cloud plan you purchased. The Usage page shows your current
  usage and describes available plan upgrades and add-ons to address
  your analytics needs.
\end{itemize}

Next
\href{https://github.com/liferay/liferay-docs/blob/7.1.x/discover/analytics-cloud/articles/05-administering-liferay-analytics-cloud/02-managing-users.markdown}{Managing
Users} shows you how to get your team on-board your project.

\section{Managing Users}\label{managing-users}

If you're the project \emph{Owner} (Liferay invited you by email) or
have been assigned the \emph{Admin} permission, you can invite and
manage users. Here's how to bring up the User Management page:

\begin{enumerate}
\def\labelenumi{\arabic{enumi}.}
\item
  Click on \emph{Settings} in the navigation panel. The default Settings
  page appears.
\item
  Click on \emph{User Management} in the navigation area. The \emph{User
  Management} page appears, listing all the project's users.
\end{enumerate}

From the \emph{User Management} page, an Admin user (or Owner) can
invite users, edit their permissions, and delete them.

If your project is new, you'll want to invite your teammates to the
project.

\subsection{Inviting Users}\label{inviting-users}

It's exciting to give and receive invitations. Carry that excitement
with you as you invite your team's analysts, data integrators, and
marketers to your Analytics Cloud project.

\begin{enumerate}
\def\labelenumi{\arabic{enumi}.}
\item
  In the \emph{User Management} page, click on the \emph{Invite Users}
  button. The \emph{Invite Users} dialog appears.
\item
  Enter the email addresses (separated by a space or comma) of the
  people to invite to the project.
\item
  Click the \emph{Send} button. The invitation is sent to your lucky
  teammates.
\end{enumerate}

\begin{figure}
\centering
\includegraphics{./images/invitation-email.png}
\caption{The invitation email looks like this, except your project
number shows in place of the red rectangle.}
\end{figure}

You can set the user's permissions when their invitation acceptance is
pending.

\subsection{Managing Permissions}\label{managing-permissions}

Analytics Cloud projects have these permission Roles:

\textbf{Owner:} Can do all things, but cannot be deleted.

\textbf{Admin:} Can do all things, including manage other Admin users.
Admin users can be deleted.

\textbf{Member:} Can view all profiles and analytics, but cannot edit
anything under Settings. They cannot manage users, or add or remove data
sources.

Here's how to change a user's permission Role:

\begin{enumerate}
\def\labelenumi{\arabic{enumi}.}
\item
  Click the \emph{Edit} button in the user's list entry.
\item
  In the \emph{Permission} column, select the user's current permission.
  A permission Role selector appears.
\item
  Select the permission Role for the user.
\end{enumerate}

The user is assigned the selected Role. That's all there is to it!

\section{Tracking Usage}\label{tracking-usage}

Analytics Cloud plans are limited by the total amount of Individuals and
Pages Views synced from data sources. When either limit is exceeded, you
must either upgrade your plan or purchase add-ons to accommodate the
overage.

\begin{figure}
\centering
\includegraphics{./images/current-usage.png}
\caption{The Usage page reports when you're approaching or exceeding
your plan limits.}
\end{figure}

The plans and add-ons are described on the right side of the page.

\subsection{Plans}\label{plans}

The \emph{Plans} section describes each plan's limits and cost. The
higher tier plans cost more initially but offer more flexibility through
higher limits and less expensive add-ons. Your current plan is
highlighted. You can upgrade from lower tier to higher tier plans.

\subsection{Add-ons}\label{add-ons}

Add-ons increase limits on Individuals or Page Views, without requiring
you to upgrade to a new plan. Purchasing add-ons gives you
commitment-free analytics capacity boosts.

Contact your sales representative for further guidance on getting a plan
that fits you.

\section{Managing Interest Topics}\label{managing-interest-topics}

Analytics Cloud's
\href{https://github.com/liferay/liferay-docs/blob/7.1.x/discover/analytics-cloud/articles/04-analyzing-touchpoints/09-content-recommendation-api.markdown}{recommendation
API} suggests topics of interest based on user interactions with content
in Liferay DXP. Analytics Cloud administrators can exclude certain
keywords from the recommendation API, therefore preventing Analytics
Cloud from suggesting content based on those keywords. This is useful if
the administrator considers a given topic too broad, too narrow,
offensive, or otherwise not useful for suggesting content.

Follow these steps to add a keyword to the block list:

\begin{enumerate}
\def\labelenumi{\arabic{enumi}.}
\item
  In Analytics Cloud, select \emph{Settings} → \emph{Interest Topics}. A
  table lists any existing keywords in the block list.
\item
  Click \emph{Add Keyword} and enter one or more keywords in a
  comma-delimited list.
\item
  Click \emph{Send} when you're finished. Your keywords then appear in
  the table.
\end{enumerate}

To delete a keyword from the block list, click that keyword's trash icon
(\includegraphics{./images/icon-trash.png}). To delete multiple keywords
at once, select the checkbox for each and click the trash icon that
appears at the top of the table. You can select all keywords via the
checkbox at the top of the table.

\begin{figure}
\centering
\includegraphics{./images/interest-topics.png}
\caption{Cool beans aren't allowed.}
\end{figure}

\chapter{Troubleshooting Liferay DXP Data
Sources}\label{troubleshooting-liferay-dxp-data-sources}

Misconfigured environments or data sources can prevent or disrupt access
to Liferay DXP (DXP) data sources. Here's how to troubleshoot DXP data
source issues.

\section{Retry Authorization}\label{retry-authorization}

\textbf{Error Message:}
\texttt{Unknown\ error.\ Please\ retry\ authorization.}

DXP data source access requires that your DXP instance be
\href{https://github.com/liferay/liferay-docs/blob/7.1.x/discover/analytics-cloud/articles/02-getting-started/02-adding-a-liferay-dxp-data-source.markdown\#step-2-make-sure-liferay-dxp-and-its-json-web-services-are-accessible}{publicly
available} and that your Analytics Cloud instance be registered with the
DXP instance as an
\href{https://github.com/liferay/liferay-docs/blob/7.1.x/discover/analytics-cloud/articles/02-getting-started/02-adding-a-liferay-dxp-data-source.markdown\#step-2-register-analytics-cloud-with-your-liferay-dxp-instance}{OAuth
application}.

\textbf{Resolution:}

\begin{enumerate}
\def\labelenumi{\arabic{enumi}.}
\item
  Follow the steps for
  \href{https://github.com/liferay/liferay-docs/blob/7.1.x/discover/analytics-cloud/articles/02-getting-started/02-adding-a-liferay-dxp-data-source.markdown}{adding
  a Liferay DXP data source}.
\item
  \href{https://github.com/liferay/liferay-docs/blob/7.1.x/discover/analytics-cloud/articles/02-getting-started/02-adding-a-liferay-dxp-data-source.markdown\#step-2-register-analytics-cloud-with-your-liferay-dxp-instance}{Register
  Analytics Cloud with your DXP instance}.
\end{enumerate}

\section{Unsupported Version}\label{unsupported-version}

\textbf{Error Message:}
\texttt{Unsupported\ version.\ This\ method\ of\ connection\ does\ not\ \ support\ the\ data\ source\ Liferay\ version.\ Make\ sure\ you\ are\ connecting\ to\ Liferay\ 7.0/7.1\ instance\ or\ try\ a\ different\ method\ of\ connection.}

\textbf{Resolution:}

\begin{enumerate}
\def\labelenumi{\arabic{enumi}.}
\item
  Make sure to connect with a Liferay DXP 7.0 or 7.1 instance.
\item
  Follow the steps for
  \href{https://github.com/liferay/liferay-docs/blob/7.1.x/discover/analytics-cloud/articles/02-getting-started/02-adding-a-liferay-dxp-data-source.markdown}{adding
  a Liferay DXP data source}.
\item
  If the error persists, make sure JSON web services are enabled on your
  DXP instance. They're enabled by default. If you disabled them using a
  \href{https://docs.liferay.com/dxp/portal/7.1-latest/propertiesdoc/portal.properties.html}{portal
  property} setting \texttt{json.web.service.enabled=false} (e.g., set
  in a \texttt{portal-ext.properties} file), delete the setting or set
  the property value to \texttt{true}.
\end{enumerate}

\subsection{Invalid Credentials; the Authorization
Expired}\label{invalid-credentials-the-authorization-expired}

\textbf{Error Message:}
\texttt{Invalid\ Credentials.\ The\ authorization\ for\ this\ data\ source\ \ has\ expired.\ Please\ reauthorize\ the\ token\ in\ the\ OAuth\ tab.}

This message appears in the log:

\texttt{WARN\ {[}http-nio-8080-exec-14{]}{[}AbstractOAuthService:88{]}\ Unsecure\ HTTP,\ Transport\ Layer\ Security\ is\ recommended}

Connection to a DXP data source requires that the DXP instance's web
server protocol be forwarded.

\textbf{Resolution:}

\begin{enumerate}
\def\labelenumi{\arabic{enumi}.}
\item
  Follow steps for
  \href{https://github.com/liferay/liferay-docs/blob/7.1.x/discover/analytics-cloud/articles/02-getting-started/02-adding-a-liferay-dxp-data-source.markdown}{adding
  a DXP data source}, paying particular attention to
  \href{https://github.com/liferay/liferay-docs/blob/7.1.x/discover/analytics-cloud/articles/02-getting-started/02-adding-a-liferay-dxp-data-source.markdown\#step-2-register-analytics-cloud-with-your-liferay-dxp-instance}{register
  Analytics Cloud with your DXP instance}.
\item
  If the issue persists and the web server protocol is forwarded, set
  these
  \href{https://docs.liferay.com/dxp/portal/7.1-latest/propertiesdoc/portal.properties.html}{portal
  properties} in a \texttt{portal-ext.properties} file in your DXP
  instance.

\begin{verbatim}
web.server.forwarded.protocol.enabled=true
redirect.url.security.mode=domain
redirect.url.domains.allowed=
\end{verbatim}
\end{enumerate}
