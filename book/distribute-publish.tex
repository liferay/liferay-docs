\chapter{Publishing Your App}\label{publishing-your-app}

As you develop apps, you may want to sell them to or share them with
consumers. You can distribute your apps on the Liferay Marketplace. The
process is straightforward, but you'll need to make some decisions along
the way and prepare your app for publishing on the Marketplace.

This section's first tutorial explains Liferay Marketplace-related
concepts and informs you of decisions to make before starting the
publishing process from your portal. Then you'll learn how to get
resources, such as your app's icon, screenshots to show off your app,
and a comprehensive description of your app, ready for the publishing
process. After that, you'll learn the the step-by-step process of
submitting your app to the Marketplace. Lastly, you'll learn how to
monitor your app's success on Liferay Marketplace and modify or add new
versions of your app to the Marketplace.

\section{Planning Your App's
Distribution}\label{planning-your-apps-distribution}

When you start the formal process of submitting your app to the
Marketplace, in addition to uploading your app's files you'll need to
answer a host of important questions. For example, you'll need to
clarify who owns the app, specify pricing for the app, define its
licensing scheme (if it's a paid app), associate a person or company as
its owner and maintainer, and specify the versions of Liferay that the
app supports. Your answers to these questions will also help you
determine whether you'll need to package multiple versions of the app.
This tutorial prepares you by explaining the questions and ways you
might answer them.

\subsection{Selling Your App or Making it
Free}\label{selling-your-app-or-making-it-free}

Do you want to sell your app on the Marketplace? Or do you want to
freely share it with anyone on the Marketplace? It's up to you. Most of
the content that follows describes options for paid apps (apps you
sell).

If you're selling your app, you must publish using a Paid App Account.
You must also specify licensing, a price structure, and regional
availability for your paid apps.

Importantly, you can't change the app from free to paid or from paid to
free once the app is published to the Marketplace. In order to offer the
app in the other license type, you must submit another app under a
different name (title). If you wish to have both free and paid licenses
for your app, you must submit the app under one name for free licenses
and submit it under another name for paid licenses. Make sure to select
the license type (i.e., free or paid) that's best for your app.

Have you decided who's going to be listed as the app's author/owner?
Have you decided who's going to manage the app once it's on the
Marketplace? App ownership options are explained next.

\subsection{Publishing as an Individual or on Behalf of a
Company}\label{publishing-as-an-individual-or-on-behalf-of-a-company}

You can publish an app as yourself (an individual) or on behalf of a
\emph{company}. This determines the who is shown as the app's author and
owner. Your selection also determines who can access the app behind the
scenes, once it's published.

The default option is publishing on behalf of yourself. If you go with
this option, your name is shown as the app's author/owner in the
Marketplace. The term \emph{personal app} refers to an app published by
an individual. That individual is the only one who can manage the
personal app. Managing an app includes such duties as adding new
releases to it, adding new versions of it, and editing its details.

Publishing on behalf of a company effectively hands the keys over to the
company's administrators. The app shows on the company's Marketplace app
development page and in the company's list of apps on the company's
public profile page. Company admins have the same permission that an
individual author has to manage the app (add new releases, new versions,
edit details, etc). The company's name alone is shown as the app's
author/owner.

You can
\href{https://www.liferay.com/marketplace/become-a-developer}{register}
yourself as a Marketplace Developer or one of your company's
administrators can register the company as a Marketplace Developer. You
can either register for a Free Basic Account or register for a Paid App
Developer Account. A Paid App Developer Account lets you submit paid
apps to the Liferay Marketplace for sale to customers globally. It
enables you to monetize the benefits of the Marketplace and the benefits
of the Liferay distribution channel. The best part is, you can upgrade
to a Paid App Developer Account if and when you want to submit a paid
app to the Marketplace! Here's a comparison of the Marketplace developer
account options:

Free Basic Account

No cost to register as a Marketplace developer

Distribute free apps on the Marketplace

Access to developer-only resources (e.g., Developer Portal, Liferay IDE,
Liferay Developer License)

Participate in Marketplace promotions

Converting from Basic-to-Paid

When you're ready to submit a paid app to the Marketplace:

Agree to the Developer Agreement

Pay \$99 annual fee

Submit relevant tax documents (eg., W-9, W-8BEN) to receive payments

Enter PayPal info to receive payments (PayPal account must be a Verified
Business Account)

Paid App Developer Account

Upgrade when you're ready: only pay annual fee at time of paid app
submission

Offer paid licensing/support for your apps

Receive 80\% of app sales proceeds

Access to detailed transaction history

Paid apps customer management

All the benefits of a Basic Developer Account

Now that you've determined your app's owner and you've registered an
account to manage the app, you can learn about licensing options for
paid apps.

\subsection{Licensing and Pricing Your
App}\label{licensing-and-pricing-your-app}

You have significant control over how to price your app. You choose the
license term (perpetual vs.~annual), choose the license type (standard
vs.~developer), define a pricing structure (pricing and bundled
discounting based on a license unit), and specify regional availability.
Even after your app is on the Marketplace, you can tweak general pricing
or modify regional pricing.

\subsubsection{Determining a License
Term}\label{determining-a-license-term}

Here are the license term options:

\begin{itemize}
\item
  Perpetual license: does not expire.
\item
  Non-perpetual license: must be renewed annually.
\end{itemize}

Importantly, you can't change your app's license terms once the app is
approved. In order to release an approved app under a different license
term you must submit another app under a new name (title). So make sure
you think through the license term that makes the most sense for your
app.

\noindent\hrulefill 

\textbf{Note:} If you are a foreign developer based outside of
the United States, non-perpetual license sales (considered to be royalty
income) to US customers are subject to a 30\% withholding tax. This tax
does not apply to perpetual licenses. For more information on licensing
terms and fees, please refer to the \href{/distribute/faq}{FAQ}.

\noindent\hrulefill

\subsubsection{Determining License Type and License Unit
Pricing}\label{determining-license-type-and-license-unit-pricing}

Licenses are set to run on a permitted number of Instance Units (defined
as a single installation of the Liferay Portal, which corresponds to one
(1) Liferay Portal \texttt{.war} file). You can create tiers of bundled
options to accommodate the number of Instance Units to offer to
customers. You can also offer a discounted price for a bundle of
multiple Instance Units.

There are two types of licenses that you can offer for your app:
standard licenses and developer licenses. Standard licenses are intended
for production server environments. Developer licenses are limited to 10
unique IP addresses and, therefore, should not be used for full-scale
production deployments.

\begin{figure}
\centering
\includegraphics{./images/licenseview.png}
\caption{Liferay Marketplace lets you specify different types of
licenses and license quantities for pricing.}
\end{figure}

You can also offer subscription services or 30-day trials. Support,
maintenance, and updates should comprise subscription services.

Depending on how you decide to license your app, you have a few options
to consider with regard to subscription services:

\textbf{If you're offering a perpetual license\ldots{}}

You can offer annually renewable subscription services. During the app
submission process if you choose to \emph{offer subscription services},
you're asked to price subscription services on a \emph{per Instance Unit
per year} basis. The first year of subscription services is included
with a perpetual license. If a customer wants to continue receiving
services after the first year, the customer must renew subscription
services at the price you designate. Customers are entitled to support,
maintenance, and updates as long as they continue to annually renew
subscription services.

If you choose not to offer subscription services, customers are entitled
to only app updates, if and when updates become available. This type of
one-time, upfront payment model may work well for less complex apps and
themes/templates.

\textbf{If you're offering a non-perpetual (annual/renewable)
license\ldots{}}

You can offer annually renewable subscription services. During the app
submission process if you choose to \emph{offer subscription services},
you should build the price of subscription services into the annual
price of the non-perpetual license; please take this into account as you
price non-perpetual licenses. In effect, the customer pays one price
annually for both the app license and subscription services. Customers
is entitled to support, maintenance, and updates as long as they
continue to renew their non-perpetual license.

If you choose not to offer subscription services, customers are entitled
to only app updates if and when updates become available. They can
receive updates as long as they continue to have valid non-perpetual
licenses.

\subsubsection{Setting Prices for License Options by
Region}\label{setting-prices-for-license-options-by-region}

You can specify countries your app will be available in and the app's
price in in each of those countries. You can make it as simple (single
price offered globally) or as granular (different price in each country
offered) as you want.

\begin{figure}
\centering
\includegraphics{./images/pricing_screenshot.png}
\caption{Liferay Marteplace lets you specify prices for all of your
license options and lets you specify them by region.}
\end{figure}

Choose the currency to use with your pricing options. Decide on the
renewal cost for any support services you offer. The support services
price is per instance, so if you specify \$100 USD and the customer is
running 10 instances, their annual support services renewal cost will be
\$1000. Note: This only applies to perpetual licenses. For non-perpetual
licenses, you should include any support services cost in the annual
license price.

Even after an app has been approved, you can change the currency type
and currency price of its license bundles, and you can modify regional
availability of license bundles.

Although Liferay Marketplace supports major currencies and a broad list
of countries, not all currencies and countries are currently available.
Additional currencies and countries may become available at a later
time.

\subsubsection{Considering the Liferay Marketplace
Fee}\label{considering-the-liferay-marketplace-fee}

By selling your paid apps on the Liferay Marketplace, you're agreeing to
share app sales revenue with Liferay. For each app sale, you receive
80\% of the sales proceeds and Liferay receives 20\% of the sales
proceeds. We believe this type of fee structure is extremely competitive
compared to other app marketplaces. Liferay uses its share of app sale
proceeds to:

\begin{itemize}
\tightlist
\item
  Operate and improve the Liferay Marketplace.
\item
  Provide developer services, such as payment processing, license
  tracking, and performance metrics.
\item
  Continue investing in and growing the Liferay ecosystem.
\end{itemize}

\textbf{Comparison of Liferay's App Revenue Sharing with that of Other
App Marketplaces}

Share to Developer

Liferay Marketplace

80\%

Atlassian Marketplace

75\%

DNN Store

75\%

Concrete 5

75\%

Apple App Store

70\%

As you can see, Liferay's fee is reasonable.

You need a valid Paypal account to receive payment; you're asked to
provide this info when registering for or converting to a Paid App
Developer Account. Payments are issued no later than 90 days after the
transaction.

Now that you've decided on licensing options and pricing, you can
concentrate on what versions of Liferay your app will run on.

\subsection{Determining Editions and Versions of Liferay to
Target}\label{determining-editions-and-versions-of-liferay-to-target}

Of course, targeting the widest possible range of Liferay editions and
versions in an app typically draws larger audiences to the app. And
there may be certain features in these editions and versions that you
want to take advantage of. In your app's plugin
\href{/distribute/how-to-publish/-/knowledge_base/how-to-publish/preparing-your-app\#specifying-packaging-directives}{packaging
properties}, you specify packaging directives to indicate the
\href{/docs/6-2/deploy/-/knowledge_base/d/editions-of-liferay}{editions}
the app supports and the version that the app supports. To ensure the
widest audience for your app, we encourage you to make your app
compatible with both Liferay Community Edition (CE) and Liferay
Enterprise Edition (EE).

You can prepare a set of app files (including its
\texttt{liferay-plugin-package.properties} file) for each version of
Liferay you want to support. When uploading your app, you can specify
which versions of Liferay your app is compatible with and you can
appropriately upload the sets of app files that are designed for those
different versions. The next article in this guide explains how to go
about
\href{/distribute/how-to-publish/-/knowledge_base/how-to-publish/preparing-your-app\#specifying-packaging-directives}{specifying
packaging directives}.

Note that apps on the Liferay Marketplace must be designed for Liferay
Portal 6.1 or later. That's not to say that they can't work with prior
versions. However, only Liferay Portal 6.1 and later versions support
installing apps directly from the Marketplace and provide safeguards
against malicious apps. If you want to use an app with an earlier
version of Liferay, make sure that version of Liferay provides what your
app needs from Liferay.

Lastly, you should determine a versioning scheme for your app. How will
you refer to the first version of your app, the second version, and so
on.

\subsection{Decide on a Versioning
Scheme}\label{decide-on-a-versioning-scheme}

A version of an app represents the functionality of the app at a given
point in time. When you first create an app, you give it an initial
version (e.g., \texttt{1.0}). On updating the app, you increment its
version (e.g., from \texttt{1.0} to \texttt{1.1}), and you upload new
files representing that version of the app. In some cases, you specify
additional qualifiers in order to convey a special meaning. For example,
you may declare that the version of your app is always in x.y.z format
(where you've clearly defined the significance of x, y, and z). Liferay
Portal versions and official Liferay app versions use this format.

In any case, you're free to your app's assign version designators any
way you like. We recommend that you stick to a well known and easily
understandable format, such as \texttt{1.0}, \texttt{1.1}, \texttt{1.2},
and so on. Although you may want to include alphabetical characters
(e.g., \texttt{1.0\ Beta\ 2} or \texttt{6.3\ Patch\ 123235-01}), we
discourage it, as such characters may confuse customers as to how your
app's versions relate to one another.

Keep in mind that the releases of Liferay with which your app works must
be specified using Liferay's versioning scheme, as explained in
\href{/docs/6-2/deploy/-/knowledge_base/d/understanding-liferays-releases}{Understanding
Liferay's Releases}. See the later section \emph{Specify App Packaging
Directives} for details on specifying the releases of Liferay for which
your app is designed.

Congratulations on coming up with a sound game plan for your app! Next,
you should read the next article to learn how to prepare your app for
publishing.

\section{Preparing Your App}\label{preparing-your-app}

As a Liferay developer, you're undoubtedly already familiar with the
concept of plugins (portlets, themes, etc). If you're not familiar with
Liferay plugins, see the
\href{/docs/7-0/tutorials/-/knowledge_base/t/introduction-to-liferay-development}{introductory
section of Liferay developer tutorials}. A \emph{Liferay App} (sometimes
just called an \emph{app}) is a collection of one or more of these
plugins, packaged together to represent the full functionality of an
application on the Liferay platform. In addition to the plugins
contained within an app, apps have metadata such as names, descriptions,
versions, and other information used to describe and track the app
throughout its life cycle.

Much like standard Liferay plugins, Liferay apps are also
hot-deployable. Liferay Marketplace apps are distributed via a special
file type with a \texttt{.lpkg} extension. To deploy these files, drop
them into a running Liferay instance's hot-deploy folder
(\texttt{{[}Liferay\_Home{]}/deploy}), like any other plugin.

As an app developer, you're not required to create the \texttt{.lpkg}
files. Instead, your app's individual plugins (WAR files for traditional
plugins or JAR files for OSGi modules) are uploaded as part of the
publication process, along with information (name, description, version,
icon, etc.) that identifies the app. The publication process is
described in detail later.

At this point in preparing to publish your app, you've developed your
app. And if you're preparing a paid app, you've specified a permission
descriptor (a portal access control list
\href{/docs/6-2/tutorials/-/knowledge_base/t/plugin-security-and-pacl}{(PACL)}
for traditional plugins or a \texttt{OSGI-INF/permissions.perm} file for
OSGi modules), so that your app can be deployed on Liferay instances
that have their
\href{/docs/6-2/tutorials/-/knowledge_base/t/plugin-security-and-pacl\#enabling-the-security-manager}{Plugin
Security Manager} running. But before you start the formal publishing
process, you must prepare your app's files and app metadata.

\subsection{Marketplace App Metadata
Guidelines}\label{marketplace-app-metadata-guidelines}

The following app metadata guidelines are intended to ensure that apps
are submitted with important and necessary supporting information. The
metadata that you submit with your app will serve both as necessary
information for your app's buyers (e.g., your contact info) and as
promotional assets (e.g., description, screenshots, etc.) that can help
drive traffic to and downloads of your app!

\begin{figure}
\centering
\includegraphics{./images/dev-portal-app-metadata-guidelines.png}
\caption{Check out how good your app can look on the Markeplace.}
\end{figure}

You'll want to think of a good name and description of your app. If you
haven't already done so, take some screenshots, design an icon, and
create a website for your app. The table below can help guide you
address the Marketplace app metadata requirements and produce an
appealing advertisement for your app.

\textbf{Marketplace App Metadata Guidelines:}

Required Metadata

Description

App Name

This is probably the most important branding element of your app, so be
creative! Some important things to keep in mind:

In some views within the Marketplace, app titles longer than 18
characters will be shortened with an ellipsis.

Titles must not be longer than 50 characters.

App title may contain the word ``Liferay'' to describe its use or intent
as long as the name does not imply official certification or validation
from Liferay, Inc. (For example, names such as ``Exchange Connector for
Liferay'' or ``Integration Connector Kit for Liferay'' would be allowed,
while ``Liferay Mail Portlet,'' ``Liferay Management Console,'' or
``Liferay UI Kit'' would not be permissible without explicit approval).
Please refer to our trademark policy for details.

Please try to conform the app name, as closely as possible, to the
actual plugin (portlet, hook, theme, etc.) name.

Support Information

Please include an email address, contact information, and/or website URL
for the ``Support'' field. If someone encounters an issue with your app,
they will need a way to contact you.

Even if you choose not to offer Support Services (by unchecking ``Offer
Support'' during the app submission process), we require that you
provide support contact information so that buyers can reach you with
general questions about your app.

Description

Using English as the default langauge, you must provide a description
for your app. After providing the description in English, you can
provide other translations of the text.

At a minimum, the description should provide a concise overview of what
the app does. Great descriptions also list key functionalities and what
customers can expect to gain by deploying your app. If you'd like to see
an example, you can see our description of the Audience Targeting app on
the Markeplace.

It's important that you specify any plugin dependencies (e.g., plugins
that must be installed prior to running your app) and environment
compatibilities (e.g., compatibility with specific app servers) here, so
that potential buyers and the Liferay app review team are aware of these
requirements.

What's New?*

Using English as the default language, describe what's new and improved
in your app. After this, you can provide other translations of the text.
This field is shown on updating an app that you've already submitted,
regardless of whether the app has been published.

Compatibility with Future Minor Releases of Liferay Portal

Please include a ``+'' at the end of the latest version when specifying
version constraints in your liferay-plugin-package.properties file
(e.g., ``liferay-versions=6.1.1+, 6.1.20+''). This will ensure that your
app will continue to be deployable to future versions of Liferay within
a minor release. If, in the future, you discover your app does NOT work
with a particular Liferay version, you can modify the list to exclude
that version.

Increase Your Potential User Base

In most cases, an app that is compatible with CE portal will also run
under Liferay Digital Enterprise (DE) or EE portal, and vice versa.
Specifying compatibility with both DE/EE and CE versions of the portal
will ensure that a wider audience has access to your app!

You can request a Liferay DE/EE Developer License to support testing and
confirm compatibility.

Icon

Icons for your app must be exactly 90 pixels in both height and width
and must be in PNG, JPG, or GIF format. The image size cannot exceed
512kb. Animated images are prohibited.

The use of the Liferay logo, including any permitted alternate versions
of the Liferay logo, is permitted only with Liferay's explicit approval.
Please refer to our trademark policy for details.

Screenshots

Screenshots for your app must not exceed 1080 pixels in width x 678
pixels in height and must be in the JPG format. The file size of each
screenshot must not exceed 384KB. Each screenshot should preferably be
the same size (each will be automatically scaled to match the aspect
ratio of the above dimensions), and it is preferable if they are named
sequentially, for example fluffy-puppies-01.jpg, fluffy-puppies-02.jpg,
and so on.

Make sure your icons, images, descriptions, and tags are free of
profanity or other offensive material.

During the publication process, you upload your app's individual plugin
WAR files and module JAR files along with the app's metadata (name,
description, version, icon, etc.).

\textbf{Additional Requirements for Themes/Site Templates}

A theme without content is like an empty house. If you're trying to sell
an empty house, it may be difficult for prospective buyers to see its
full beauty. However, staging the house with some furniture and
decorations helps prospective buyers imagine what the house might look
like with their belongings. Liferay's resources importer application is
a tool that allows a theme developer to have files and web content
automatically imported into the portal when a theme is deployed.

All standalone themes that are uploaded to Liferay Marketplace must use
the resources importer to include files/web content to provide a sample
context for their theme. This ensures a uniform experience for
Marketplace users: a user can download a theme from Marketplace, install
it on their portal, go to Sites or Site Templates in the Control Panel
and immediately see their new theme in action. You can read more about
themes in the
\href{/docs/7-0/tutorials/-/knowledge_base/t/themes-and-layout-templates}{Themes
and Layout Templates tutorials}.

\subsection{Deployment Requirements}\label{deployment-requirements}

Liferay apps are ``normal'' Liferay plugins with additional information
about them. Therefore, most of the requirements are the same as those
that exist for other Liferay plugins, as explained in the tutorials on
creating
\href{/docs/7-0/tutorials/-/knowledge_base/t/liferay-mvc-portlet}{MVC
Portlets} and
\href{/docs/7-0/tutorials/-/knowledge_base/t/jsf-portlets-with-liferay-faces}{JSF
Portlets}.

In addition to those requirements, there are some Marketplace-specific
ones to keep in mind:

\begin{itemize}
\item
  \textbf{Target the Appropriate Java JRE}: Regardless of the tools you
  use to develop your app, your app's bytecode must be compatible with
  the target Java JRE for your version of Liferay. Your app will be
  rejected if its bytecode is not compatible with the Java JRE for the
  intended version of Liferay Digital Enterprise or Liferay Portal.
  Liferay Portal 6.2 targets Java 1.7, and Liferay Digital Enterprise
  7.0 and Liferay Portal CE 7.0 target Java 1.8. If you use the Liferay
  Plugins SDK to develop your app, you can set the Java version by
  overriding the \texttt{ant.build.javac.target} property in the Plugins
  SDK's \texttt{build.properties} file.
\item
  \textbf{WAR (\texttt{.war}) files}:

  \begin{itemize}
  \item
    WAR files must contain a
    \texttt{WEB-INF/liferay-plugin-package.properties} file.
  \item
    WAR files must not contain any
    \texttt{WEB-INF/liferay-plugin-package.xml} file.
  \item
    WAR file names must not contain any commas.
  \item
    WAR file names must conform to the following naming convention:

    \emph{context\_name}\texttt{-}\emph{plugin\_type}\texttt{-A.B.C.D.war}

    Where:
  \item
    \emph{context\_name}: Alpha-numeric (including \texttt{-} and
    \texttt{\_}) short name of your app. This name is used as the
    deployment context, and must not duplicate any other app's context
    (you'll be warned if you use a context name of any other app on the
    Marketplace).
  \item
    \emph{plugin\_type}: one of the following: \texttt{hook},
    \texttt{layouttpl}, \texttt{portlet}, \texttt{theme}, or
    \texttt{web}.
  \item
    \texttt{A.B.C.D}: The 4 digit version of your WAR file. 4 digits
    must be used.

    Example: \texttt{myapp-portlet-1.0.0.0.war}
  \end{itemize}
\item
  \textbf{\texttt{WEB-INF/}\href{http://docs.liferay.com/portal/6.2/propertiesdoc/liferay-plugin-package_6_2_0.properties.html}{\texttt{liferay-plugin-package.properties}}
  file:}

  \begin{itemize}
  \tightlist
  \item
    Property \texttt{recommended.deployment.context} must not be set.
  \item
    Setting property \texttt{security-manager-enabled} to \texttt{true}
    is mandatory for all paid apps on 6.1 CE GA3, 6.1 EE GA3, and later;
    the setting is optional for free apps. Setting this property to
    \texttt{true} enables Liferay's Plugin Security Manager. If you're
    enabling the security manager, you'll also need to define your
    Portal Access Control List (PACL) in this file. Read
    \href{/docs/6-2/tutorials/-/knowledge_base/t/plugin-security-and-pacl}{Plugins
    Security and PACL} for information on developing secure apps.
  \end{itemize}
\item
  \textbf{Deployment contexts}:

  \begin{itemize}
  \tightlist
  \item
    Liferay reserves the right to deny an application if any of its
    plugin deployment contexts is the same as a context of another
    plugin in the Marketplace.
  \item
    Liferay reserves the right to replace app plugin WAR files that have
    the same deployment context as plugins built by Liferay.
  \end{itemize}
\end{itemize}

There are some additional requirements for uploading Liferay Digital
Enterprise 7.0 and Liferay Portal CE 7.0 apps:

\begin{itemize}
\tightlist
\item
  \textbf{OSGi modules in JAR (\texttt{.jar}) files}:

  \begin{itemize}
  \tightlist
  \item
    For more information, see
    \href{/docs/7-0/tutorials/-/knowledge_base/t/osgi-and-modularity\#modules}{OSGi
    and Modularity - Modules}.
  \item
    The manifest file must at minimum contain the following manifest
    headers:

    \begin{itemize}
    \tightlist
    \item
      \texttt{Bundle-SymbolicName}: a unique name for the module
    \item
      \texttt{Bundle-Version}: the module's version
    \item
      \texttt{Web-ContextPath}: the servlet context path
    \end{itemize}
  \end{itemize}
\item
  \textbf{WAR (\texttt{.war}) files}:

  \begin{itemize}
  \tightlist
  \item
    WAR-based plugins must be adapted to run on Liferay Digital
    Enterprise 7.0 and Liferay Portal CE 7.0. See
    \href{/docs/7-0/tutorials/-/knowledge_base/t/liferay-upgrade-planner}{this
    tutorial} for information on adapting WAR plugins that run on
    Liferay Portal 6.2 and earlier to run on Liferay Digital Enterprise
    7.0 and Liferay Portal CE 7.0.
  \end{itemize}
\item
  \textbf{Apps containing multiple file types:}

  \begin{itemize}
  \tightlist
  \item
    Liferay Digital Enterprise 7.0 and Liferay Portal CE 7.0 apps can
    contain a mix of WAR-based plugins and OSGi JAR-based plugins.
    Regardless of file type, each plugin must be able to run on Liferay
    Digital Enterprise 7.0 and Liferay Portal CE 7.0.
  \end{itemize}
\end{itemize}

\noindent\hrulefill 

\textbf{Important:} If you're developing a paid app or want
your free app to satisfy Liferay's Plugin Security Manager, make sure to
specify PACLs for your traditional plugins and a
\texttt{OSGI-INF/permissions.perm} file for each of your modules. See
the article
\href{https://dev.liferay.com/participate/liferaypedia/-/wiki/Main/Portal+Security+Manager}{Plugin
Security Manager} for details. Give yourself adequate time to develop
your app's permission descriptors and time to test your app thoroughly
with the security manager enabled. 

\noindent\hrulefill

Apps usually consist of multiple plugins (e.g., multiple WAR or JAR
files) and plugin types. In addition, you may want to consider how to
package your app for running on different Liferay versions.

\subsection{Considering Package Variations to Target Different Versions
of
Liferay}\label{considering-package-variations-to-target-different-versions-of-liferay}

Apps can be written to work across many different versions of Liferay.
For example, suppose you want to publish version 1.0 of your app, which
you're supporting on Liferay Portal 6.1 and 6.2. Due to
incompatibilities between these Liferay versions, it may be impossible
to create a single binary WAR or JAR file that works across both Liferay
versions. In this case, you must compile your app twice: once against
Liferay Portal 6.1 and once against 6.2, producing 2 different
\emph{packages} (also called variations) of your version 1.0 app. Each
package has the same functionality, but they're different files. You can
upload such packages to support your app on different Liferay versions.
With regards to Liferay apps, packages are sometimes referred to as
files that make up your app.

Now that you've prepared your app's files and specified its metadata,
it's time to get it to submit it to Liferay for publishing on the
Marketplace!

\section{Submitting Your App}\label{submitting-your-app}

Now that you have everything in order for publishing your app, you can
start the Marketplace app submission process.

Go to your \emph{Account Home} on
\href{http://www.liferay.com}{liferay.com}. In the left side navigation
panel of your profile page, there are links to pages related to using
apps and developing apps. Links to \emph{Apps} and \emph{App Metrics}
are listed in the \emph{Development} section of the navigation panel.
You'll use these links heavily during development, so you may want to
bookmark this page too. Click \emph{Apps} from the \emph{Development}
section to access your app development page.

\begin{figure}
\centering
\includegraphics{./images/marketplace-my-app-manager.png}
\caption{Your app development page lists the apps you've developed and
enables you to add new apps for publishing to the Marketplace.}
\end{figure}

Now that you know how to get to your app development page, you can
submit your app!

To begin the process of publishing your app, click \emph{Add New App}.
The app wizard appears, so that you can fill in your app's details.

\subsection{Specify Your App's Initial
Details}\label{specify-your-apps-initial-details}

From this screen you enter your app's basic details. Previous articles
\href{/distribute/how-to-publish/-/knowledge_base/how-to-publish/planning-your-apps-distribution}{Planning
Your App's Distribution} and
\href{/distribute/how-to-publish/-/knowledge_base/how-to-publish/preparing-your-app}{Preparing
Your App} helped prepare you for filling in these details.

The top portion of the app wizard's initial screen is shown in the
figure below.

\begin{figure}
\centering
\includegraphics{./images/marketplace-add-app-details.png}
\caption{The app wizard lets you add details about your app, an icon,
and screen shots. Scroll down further to see options for specifying
relevant URLs, adding tags, and specifying the editions of Liferay that
your app supports.}
\end{figure}

Here are descriptions of this screen's fields:

\textbf{App Owner:} Choose to whom the app ``belongs'' once it is
uploaded--either yourself (personal) or a company. If you'd like to
submit on behalf of a company, click \emph{Create a company} to go to
the \emph{Join A Company} page. From this page, search to see if your
company already exists in the Liferay Marketplace. If you wish to
publish your app on behalf of your company, but your company does not
yet have a Marketplace profile, register your company by clicking
\emph{Register My Company}, filling out the registration form, and
submitting the form.

\textbf{App Pricing:} Choose whether you want the app to be free or
paid. If you choose paid, you'll have the option to specify pricing and
licensing details later in the submission process.

Importantly, you can't change the app from free to paid or from paid to
free once the app is published to the Marketplace. In order to offer the
app in the other license type, you must submit another app under a
different name (title). If you wish to have both free and paid licenses
for your app, you must submit the app under one name for free licenses
and submit it under another name for paid licenses. Make sure to select
the license type (i.e., free or paid) that's best for your app.

\textbf{Title:} Name of your application.

\textbf{Description:} Like the name says, this is a description of your
app. You can put anything you want here, but a good guideline is no more
than 4-5 paragraphs. This field does not allow any markup tags or other
textual adornments--it's just text.

\textbf{Icon:} Upload an icon image to represent your app.

\textbf{Screen Captures:} Upload one or more screenshots of your app.
The screenshots you upload here are displayed in a carousel of rotating
images when your app is viewed on the Marketplace.

\textbf{Category:} Choose the Marketplace category that most accurately
describes what your app does. Users looking for specific types of apps
often browse categories by clicking on a specific category name in the
main Marketplace home page. Having your app listed under the appropriate
category helps them find your app.

\textbf{Developer Website URL:} This is a URL that should reference the
web site associated with the development of the app. For open source
apps, this typically points at the project site where the source is
maintained. For others, links to a home page of information about this
app would be appropriate.

\textbf{Support URL:} This is a URL that should reference a location
where your app's purchasers or users can get support.

\textbf{Documentation URL:} What better way to showcase the amazing
capabilities of your app than to have a live, running version of it for
potential buyers to see and use? This field can house a URL pointing to
something as exciting as that and/or documentation for using your app.

\textbf{Source Code URL:} If you'd to provide a link to the source code
of your app, do so here.

\textbf{Labs:} You can denote an app as experimental by checking the
appropriate box.

\textbf{Security:} If your app does \emph{not} use Liferay's PACL
Security Manager, check the appropriate box. Otherwise, make sure to
enable the security manager in your app by including the setting
\texttt{security-manager-enabled=true} in the
\href{http://docs.liferay.com/portal/6.2/propertiesdoc/liferay-plugin-package_6_2_0.properties.html}{\texttt{liferay-plugin-package.properties}}
file in your WAR files and specify a \texttt{OSGI-INF/permissions.perm}
file in your module JAR files.

\textbf{Tags:} A set of descriptive words that categorize your app.
These tags are free-form and can help potential purchasers find your app
through keyword searches, tag clouds, and other search mechanisms. You
can click on \emph{Suggestions} to let Marketplace suggest some tags for
you based on the data you've already entered. Click \emph{Select} to
select from existing tags, or you can manually type in new tags. See the
\emph{Marketplace Basics} section of this chapter for detailed
requirements for tags.

\textbf{EULA:} You can use the default end user license agreement (EULA)
or provide your own. There's a link to the minimum terms which custom
EULAs must satisfy.

Remember to visit the articles
\href{/distribute/how-to-publish/-/knowledge_base/how-to-publish/planning-your-apps-distribution}{Planning
Your App's Distribution} and
\href{/distribute/how-to-publish/-/knowledge_base/how-to-publish/preparing-your-app}{Preparing
Your App} for information to help you decide how best to fill in many of
these fields.

Once you've entered all your app's details, click \emph{Next} to get to
the screen for uploading your app's files.

\subsection{Upload Your App's Plugin
Files}\label{upload-your-apps-plugin-files}

On this screen, you must specify your app's version and upload your
app's files. Note that the article
\href{/distribute/how-to-publish/-/knowledge_base/how-to-publish/planning-your-apps-distribution}{Planning
Your App's Distribution} helps you plan your app's versioning scheme.
Likewise, the article
\href{/distribute/how-to-publish/-/knowledge_base/how-to-publish/preparing-your-app}{Preparing
Your App} helps you determine which files to upload.

To start uploading, click \emph{Browse} and select the files that make
up your app. Each time you add a file, it automatically begins to upload
and its compatibility information is scanned. You must upload at least
one file before advancing beyond this screen. Once the files are
successfully uploaded, a check mark appears next to each plugin, and the
plugins are displayed based on their compatibility information.

\begin{figure}
\centering
\includegraphics{./images/marketplace-app-version-and-upload-files.png}
\caption{Specify a set of files for each Liferay version you wish to
support.}
\end{figure}

If you selected \emph{Free} for your app pricing, click \emph{Next} to
advance to the final screen. If you selected \emph{Paid}, you'll be
presented with additional options for licensing and pricing your app.

\subsection{Creating Your Licensing and Pricing
Model}\label{creating-your-licensing-and-pricing-model}

Carefully consider which licensing structure best meets your needs. Once
your app has been approved, these options, except for price updates,
cannot be changed.

\textbf{Choose a license term:}

\begin{figure}
\centering
\includegraphics{./images/marketplace-configure-app-license.png}
\caption{Choosing license terms for Marketplace apps is easy.}
\end{figure}

Choosing \emph{Perpetual} allows the app to continue running without
expiration. Choosing \emph{Non-Perpetual} expires the app's license one
year from the purchase date. Perpetual License also allows you to offer
Support Services, which the customer must renew annually to maintain
access to app updates and support. If you choose not to offer Support
Services with a Perpetual License, customers will be provided with app
updates only, whenever updates are available.

Importantly, you can't change your app's license terms (perpetual or
non-perpetual) once the app is approved. In order to release an approved
app under a different license term you must submit another app under a
new name (title). So make sure you think through the license term that
makes the most sense for your app.

\textbf{Creating license options:}

\begin{figure}
\centering
\includegraphics{./images/marketplace-create-license-types.png}
\caption{You can create multiple license options for your Marketplace
apps.}
\end{figure}

Creating license options allows you to design license bundles and to
specify discounts for customers who purchase more Liferay Instances for
your app (a Liferay Instance or Instance refers to a single installation
of the Liferay Portal). Also you can designate different pricing for
Standard Licenses vs. Developer Licenses. You must specify at least one
license option, but no more than 10 options per type. You'll price these
options on the next page.

You can add or remove bundles (quantities) of your app's license type
even after the app is approved. You must however, always honor any
support agreements you have with current customers that are using
bundles you've removed.

\textbf{Paid support:} You can offer additional paid support services
for your app. If you select this option, customers can contact you with
support requests and are entitled to regular updates.

Once an app that offers a support subscription offering is approved, you
can't remove that support subscription offering. You can however, add a
support subscription offering to an approved app.

\textbf{Offer a trial:} You can offer a free 30-day trial of your app,
restricted to one Instance and 25 users.

When you're finished selecting all the options for your license, proceed
to the next page to determine the app's pricing and availability.

\textbf{Pricing:}

\begin{figure}
\centering
\includegraphics{./images/marketplace-app-pricing.png}
\caption{Liferay makes it easy to price your app's license types and
specify their availability to countries around the world.}
\end{figure}

Based on your selections from the previous page, you'll have price
fields for each license option and for any support option you offered.

When you have completed your app's pricing and availability, click
\emph{Next} to advance to the app preview screen.

\subsection{Preview and Submit Your
App}\label{preview-and-submit-your-app}

The \emph{App Preview} screen lets you preview your app as it will
appear on the Marketplace. If you want to change things before finally
submitting your app, click \emph{Edit} to go back and continue making
changes until you're satisfied.

Before finalizing your app's submission, make sure that you're pleased
with the values you've set for the following app details, as these
details can't be modified once the app is approved:

\begin{itemize}
\tightlist
\item
  Free vs.~paid
\item
  License term (perpetual vs.~non-perpetual)
\item
  Support subscription offering (cannot be removed)
\end{itemize}

Once you are satisfied, click \emph{Submit for Review}.

At this point, the Liferay Marketplace staff starts to review your app
and test it. If you need to make changes to your app during its review
or if you're curious about the rigors of the review process, make sure
to read the next section of this guide.

\section{Understanding the App Review
Process}\label{understanding-the-app-review-process}

The Liferay Marketplace app QA/review process begins as soon as you
submit your app for review. Every third-party app submitted to the
Liferay Marketplace is reviewed by our team to ensure that certain
standards for information are upheld and the app installs as expected.
\emph{However, Liferay cannot be a substitute for your own testing and
debugging team}. Ultimately, it is the responsibility of the developer
to test, refine, and ensure that the app functions as promised and
performs as expected.

\noindent\hrulefill 

\textbf{Note:} Liferay is not responsible for the behavior (or
misbehavior) of apps on the Marketplace. For details regarding this,
consult the \emph{Liferay Marketplace User Agreement}, \emph{Liferay
Marketplace Developer Agreement}, and the individual \emph{End User
License Agreements} associated with each app. 

\noindent\hrulefill

Once you've submitted your app for review, your app will change its
status as it moves through the review process. You will be updated via
email as your app's status changes, and we will try to provide as much
detail as we can if we discover a potential issue with your app.
Overall, we don't want our app review process to feel like a barrier or
a black box. We love having new apps in the Marketplace, and we'll try
to be as helpful as we can to bring your app to approval!

\begin{figure}
\centering
\includegraphics{./images/app_review_process.png}
\caption{Here's a diagram of the the QA/review process.}
\end{figure}

If you submit an updated version of a previously approved app, the app
statuses will display as: Approved (Version Unsubmitted), Approved
(Version Pending), Approved (Version Pending QA), and Approved (Version
Denied).

The app review process consists of \textbf{Two Major Phases}:

\phantomsection\label{article-33460874}
Review Phase

Est. Time Frame

App metadata review

Our team will review your app's metadata to confirm that titles,
descriptions, images, etc. are appropriate.

\textasciitilde1 week

App QA test

Liferay will ensure that apps meet a minimal set of requirements:

Passes anti-virus scan

Deploys successfully on standard Liferay supported
environments/platforms without errors.

Basic functionality ``smoke'' test.

Liferay does not do source code review and will not ask for your source
code. Further, Liferay is not responsible for the behavior (or
misbehavior) of apps on the Marketplace. Please consult the Liferay
Marketplace User Agreement, Liferay Marketplace Developer Agreement, and
the individual End User License Agreements associated with each app.

\textasciitilde1-2 weeks

\textbf{Our QA Test Environments} are summarized below. At a minimum,
your app should be tested against these environments prior to
submission. If technical reasons prevent your app from running on
certain platforms (e.g., app server-specific issues), please specify
your app's requirements in the app description and documentation so that
our review team can test accordingly and exclude certain test
conditions, if necessary.

\phantomsection\label{article-33460919}
Automated QA

(Anti-virus \& Deployment)

Manual QA

(Deployment \& Basic Functionality)

Liferay Portal

QA'ed against Liferay Portal version specified in app submission up to
Liferay Portal 7.

QA'ed against Liferay Portal version specified in app submission up to
Liferay Portal 7.

Operating System

Ubuntu 11x

Windows 7 x64

Database

{For versions up to Liferay Portal 6.2}

MySQL 5.5.x

{For Liferay Digital Enterprise 7.0 and Liferay Portal CE 7.0}

MySQL 5.6.x

{For versions up to Liferay Portal 6.2}

MySQL 5.5.x

{For Liferay Digital Enterprise 7.0 and Liferay Portal CE 7.0}

MySQL 5.6.x

Application Server*

{For versions up to Liferay Portal 6.2}

Tomcat 7

Glassfish 3.1

Wildfly 10

{For Liferay Digital Enterprise 7.0 and Liferay Portal CE 7.0}

Tomcat 8

Wildfly 10

Tomcat 7

{For Liferay Digital Enterprise 7.0 and Liferay Portal CE 7.0}

Tomcat 8

Wildfly 10

JDK

{For versions up to Liferay Portal 6.2}

Oracle JDK 6

Oracle JDK 7

{For Liferay Digital Enterprise 7.0 and Liferay Portal CE 7.0}

Oracle JDK 8

{For versions up to Liferay Portal 6.2}

Oracle JDK 6

Oracle JDK 7

{For Liferay Digital Enterprise 7.0 and Liferay Portal CE 7.0}

Oracle JDK 8

Browser

Firefox

Firefox

\textbf{Important:} Unless specified otherwise in your app description
or documentation, your app must run under all specified app servers in
order to be approved.

You'll receive an email confirmation, once your app is approved by
Marketplace staff. And the moment your app is approved, it is made
available on Marketplace. The app is also shown on your public Profile
page, which lists all apps that you or your company's developed and
published.

If your app is rejected, an email is sent to the email address
associated with the app, along with a note explaining the reasons for
rejection. At that point, you can make the requested changes, and
re-submit the app for approval.

After you've successfully published your app, you'll likely get all
kinds of feedback from users and yourself about what's right and wrong
with it. The next article explores how to make changes once you have
published your app.

\section{Managing Your Published App}\label{managing-your-published-app}

You've launched your new app on the Marketplace. Congratulations! As you
settle down from your launch celebration, you start to wonder how your
app's doing in the Marketplace. Is there a way to monitor how many
people are viewing it and downloading it? And what about those features
that didn't make this release? How difficult will it be to upload your
app's next version with those features? Don't sweat it. The Liferay
Marketplace and app wizard are here to help.

The Liferay Marketplace lets you track app performance. You can see its
trends with respect to views, downloads, and purchases. And the app
wizard lets you upload new versions of your app and it lets you make
changes to some of its details. In this article you'll learn how all of
this works.

\subsection{Tracking App Performance}\label{tracking-app-performance}

One of the main reasons for developing and publishing apps on the
Marketplace is to drive downloads and adoption of your app. The
Marketplace enables you, as the developer of your app, to get detailed
reports about the number of views, downloads, and purchases of your
app(s). To access these metrics, navigate to \emph{Home} →
\emph{Metrics} (under \emph{Development}).

\begin{figure}
\centering
\includegraphics{./images/marketplace-app-metrics-over-time.png}
\caption{The App Performance view in Marketplace lets you see how many
times your apps has been viewed, downloaded, and purchased over a time
interval.}
\end{figure}

The view shown above is the default metrics view for a single app.
Across the top is a list of data series options (\emph{Views},
\emph{Downloads}, or \emph{Purchases}). Below that, a date range can be
chosen. In the middle, a graph is shown for the data within the date
range. Finally, the same data that is graphed is also shown in tabular
format, in case you want to know the exact values making up the graph.
The different types of data available to view are described below.

\subsubsection{Views}\label{views}

When someone searches or browses the Marketplace, they click on apps to
see detailed views of the apps they're interested in. When this occurs
for your app, a \emph{View} is recorded for the app, and this data is
what is shown on the App Metrics screen when \emph{Views} is selected at
the top. \emph{Views} is also the default view, as shown above. The
number of recorded views per day per user is unlimited.

\subsubsection{Downloads}\label{downloads}

A download is recorded for your app when someone downloads a specific
package of a specific version of your app. The number of recorded
downloads per day per user is unlimited.

\subsubsection{Purchases}\label{purchases}

The Marketplace makes an effort to count the total number of purchases
for your app.

It's great that the Liferay Marketplace provides these app monitoring
features. As you see areas to improve your app or when necessary changes
come up, you'll want to know how to make changes to your published app.
The next section explains what changes you can make.

\subsection{Making Changes to Published
Apps}\label{making-changes-to-published-apps}

After your app is published and approved, you will undoubtedly need to
make one or more of these kinds of changes during the life of the app:

\begin{itemize}
\tightlist
\item
  Editing your app details (e.g., description, icon, etc.)
\item
  Adding new license bundles (quantities)
\item
  Editing app prices of bundles
\item
  Modifying regional availability
\item
  Adding support for a new version of Liferay Portal
\item
  Releasing a new version of your app to fix bugs or offer new
  functionality
\item
  Disabling your apps
\end{itemize}

Your currently approved app remains available while you're in the
process of submitting changes to your app and while your submitted
changes are under review and testing.

Liferay Marketplace supports all of the above operations as described
below.

\subsubsection{Editing Your App's
Metadata}\label{editing-your-apps-metadata}

App metadata includes the name, description, icon, screenshots, and
other information that you supplied on the first screen during the app
creation process. To make changes to this content for your app, navigate
to \emph{Home} → \emph{Apps} and select the app you wish to edit. Click
the \emph{Edit} button to edit that app.

This screen shows you what the app looks like on the Marketplace. To
edit the detail information, click the \emph{Edit} button at the bottom
of the preview. This allows you to edit details (as well as add new
files to your existing version). Note that the current values as they
appear in your app are used to pre-fill the form. Make any changes as
needed on this screen and click \emph{Next}. If you do not need to edit
any more variations, you can continue clicking \emph{Next} until you
reach the final preview screen. Click \emph{Submit for Review} to submit
your changes for review. All changes must go through the submission
process. If your app revision only consists of metadata changes, it need
only go through the
\href{/distribute/how-to-publish/-/knowledge_base/how-to-publish/understanding-the-app-review-process}{\emph{App
metadata review}} phase of the review process.

Once approved, the changes you request appear for your app on the
Marketplace.

\subsubsection{Editing App Prices}\label{editing-app-prices}

You can change your app's prices, add or remove bundles, and modify
regional availability if you wish. This can be for a variety of reasons,
whether it's to run a promotional offer, or to adjust your pricing model
to better account for app demand. To make changes to your app, navigate
to \emph{Company Profile Home} → \emph{Apps}, then click the app you
wish to edit, and then at the bottom, click \emph{Edit} →
\emph{Pricing}. When you've finished editing your app's prices, click
\emph{Next}.

These changes do not require Liferay verification process to approve.
Simply make the changes you wish and save your app. These new prices are
reflected immediately.

You however, cannot modify the following attributes of an approved app:

\begin{itemize}
\tightlist
\item
  Free vs.~paid
\item
  License term (perpetual vs.~non-perpetual)
\item
  Support subscription offering (cannot be removed)
\end{itemize}

Next, you can consider adding support for new versions of Liferay.

\subsubsection{Adding Support for New Versions of Liferay
Portal}\label{adding-support-for-new-versions-of-liferay-portal}

If you need to add files in support of another Liferay release, the
process is similar. Navigate to navigate to \emph{Home} → \emph{Apps}
and select the app you wish to edit. Click the \emph{Edit} button to
edit that app. Click \emph{Next} to advance past the details screen
(making any changes as needed), and click \emph{Next} to advance past
the version edit screen (you can't actually edit the version number of
an already-approved version, but you can edit the ``What's New''
information if needed).

Once you advance past the version edit screen, you'll be at the File
Upload screen. This screen should look familiar--it's the same workflow
used when you initially created your app! The difference is that you
can't edit pre-approved files for specific Liferay releases. You can
only add \emph{new} files for a different Liferay release (if you
actually need to update existing files, you must create a new version of
the app--see the later section on adding versions for details on how to
do this).

Upload your new files (ensuring that your new plugins have updated
compatibility information: see the section on \emph{Specify App
Compatibility} for details on versions), click \emph{Next}, and observe
the newly-added files listed at the bottom of the preview screen. Click
\emph{Submit for Review} to submit your requested change (adding of
files). The files will be reviewed by Liferay, and once approved, the
new package becomes available for download in the Marketplace.

\subsubsection{Releasing a New Version of your
App}\label{releasing-a-new-version-of-your-app}

As time passes, you may wish to add new functionality to your app or fix
a batch of bugs. This can be accomplished by releasing a new version of
your app. New versions offer your users new functionality and bug fixes,
and users are generally encouraged to always use the latest version. In
addition, when a new version of your app becomes available, existing
users are notified automatically through Liferay's notification system.

New versions of your apps are created in a similar fashion to creating
new apps. To add a new version, navigate to \emph{Home} → \emph{Apps}
and select the app you wish to edit. Click the \emph{Edit} button to
edit that app. You are taken to the Details screen. At the bottom of the
Details screen, click the \emph{Add New Version} button. This button
begins the process of adding a new version, starting with the App
Details screen. In this case, the screen is pre-filled with data from
the current version of the app.

You can make any changes to the pre-filled data on this screen. Since
this is a new version of an existing app, making major changes (such as
completely changing the name or description) might be unsettling to your
existing users. It is common that you'll want to upload new screenshots
and refresh the icon. Note that you cannot change the app owner (such as
moving from a personally-developed app to a company-developed app).

Clicking \emph{Next} takes you through the same screens you've already
seen. On the \emph{Add App Version} screen, you can specify a new
version name for this version of your app. Also, note that when adding
new versions to an existing app, you have the option to add \emph{What's
New} text. This typically lists changes for this version, such as
significant new features or bug fix information. Clicking \emph{Next}
from here allows you to upload the files associated with the new version
of the app. For a new version of the app, you must upload all files for
all supported Liferay versions again, even if they have not changed
since the last version.

\subsubsection{Deactivating Your App}\label{deactivating-your-app}

When the time comes to retire your app, you can \emph{Deactivate} it.
Deactivating an app causes the app to become unavailable from the
Marketplace for new customers, and it doesn't appear in any public
Marketplace listings. Existing customers that have already downloaded
your app can continue downloading the legacy versions of the app they
have already acquired, but they can't download any versions they've not
already received. The app remains in your inventory, with all of its
history, in case you need to re-activate or reference it in the future.

To deactivate your app, navigate to \emph{Home} → \emph{Apps} and select
the app you wish to deactivate. Click the \emph{Deactivate} button.

In this set of articles, we looked at how to create, publish, maintain,
and track Liferay Marketplace apps. You can do this through the App
Manager that's available on your profile
\href{http://liferay.com}{liferay.com} page
(\href{http://liferay.com}{liferay.com} account required!). Then, we
covered the requirements for publishing apps, which did not differ
significantly from requirements for general Liferay development. Next,
we showed how you can publish apps on the Marketplace and how you can
modify it as the app evolves. Finally, we looked at how to track the
adoption of apps using view, download, and install metrics.

\subsection{Related Topics}\label{related-topics}

\href{/develop/learning-paths/-/knowledge_base/6-2/beginning-liferay-development}{Beginning
Liferay Development}

\href{/docs/6-2/tutorials/-/knowledge_base/t/plugins-sdk}{Developing
with the Plugins SDK}

\href{/docs/6-2/tutorials/-/knowledge_base/t/liferay-ide}{Developing
Plugins with Liferay IDE}

\href{/docs/6-2/tutorials/-/knowledge_base/t/maven}{Developing with
Maven}

\href{/docs/6-2/tutorials/-/knowledge_base/t/plugin-security-and-pacl}{Plugin
Security and PACL}
