\chapter{Development Reference}\label{development-reference}

Here you'll find reference documentation for Liferay DXP, Liferay
Screens, Liferay Faces, and technologies related to you as a third-party
developer.

The different types of reference docs you'll find in this section are as
follows:

\begin{itemize}
\tightlist
\item
  Descriptions of Java and JavaScript APIs, CSS, tags and tag libraries,
  and XML DTDs
\item
  Write ups on the latest Screenlets for Liferay Screens
\item
  Breaking changes
\item
  Cheat sheets and tips on

  \begin{itemize}
  \tightlist
  \item
    Plugin anatomy
  \item
    Design patterns
  \item
    Tools
  \item
    Adapting to new APIs
  \end{itemize}
\end{itemize}

Liferay's reference docs are at your fingertips.

\section{Java APIs}\label{java-apis}

Here you'll find Javadoc for Liferay DXP and Liferay DXP apps.

\subsection{7.0 Java APIs}\label{java-apis-1}

This table links you to the 7.0 API modules. Their root location is
here.{ (Opens New Window) } The reference doc Zip is available here.{
(Opens New Window) }

Core:

com.liferay.portal.kernel (portal-kernel):{ (Opens New Window) } ~for
developing applications on Liferay DXP

com.liferay.util.bridges (util-bridges):{ (Opens New Window) } ~for
using various non-proprietary computing languages, frameworks, and
utilities on Liferay DXP

com.liferay.util.java (util-java):{ (Opens New Window) } ~for using
various Java-related frameworks and utilities on Liferay DXP

com.liferay.util.slf4j (util-slf4j):{ (Opens New Window) } ~for using
the Simple Logging Facade for Java (SLF4J)

com.liferay.portal.impl (portal-impl):{ (Opens New Window) } ~refer to
this only if you are an advanced Liferay developer that needs a deeper
understanding of 7.0's implementation in order to contribute to it

\subsection{Liferay DXP App Java APIs}\label{liferay-dxp-app-java-apis}

This table links you to Liferay DXP application APIs. Their root
location is \href{https://docs.liferay.com/dxp/apps}{here}.

\noindent\hrulefill

App \textbar{} Packages \textbar{}
\href{https://docs.liferay.com/portal/7.1-latest/apps/announcements-1.0.1/javadocs/}{\textbf{Announcements:}}
\textbar{} com.liferay.announcements.constants \textbar{}
\href{https://docs.liferay.com/portal/7.1-latest/apps/apio-architect-1.0.1/javadocs/}{\textbf{Apio
Architect:}} \textbar{} com.liferay.apio.architect.api \textbar{}
\href{https://docs.liferay.com/portal/7.1-latest/apps/application-list-1.0.1/javadocs/}{\textbf{Application
List:}} \textbar{} com.liferay.application.list.api
com.liferay.application.list.taglib \textbar{}
\href{https://docs.liferay.com/portal/7.1-latest/apps/asset-1.0.1/javadocs/}{\textbf{Asset:}}
\textbar{} com.liferay.asset.api
com.liferay.asset.categories.navigation.api
com.liferay.asset.category.property.api com.liferay.asset.display.api
com.liferay.asset.display.page.api
com.liferay.asset.display.page.item.selector.api
com.liferay.asset.entry.rel.api com.liferay.asset.publisher.api
com.liferay.asset.tag.stats.api com.liferay.asset.taglib
com.liferay.asset.tags.api com.liferay.asset.tags.navigation.api
com.liferay.asset.test.util \textbar{}
\href{https://docs.liferay.com/portal/7.1-latest/apps/blogs-1.0.1/javadocs/}{\textbf{Blogs:}}
\textbar{} com.liferay.blogs.api com.liferay.blogs.demo.data.creator.api
com.liferay.blogs.item.selector.api
com.liferay.blogs.recent.bloggers.api com.liferay.blogs.test.util
\textbar{}
\href{https://docs.liferay.com/portal/7.1-latest/apps/calendar-3.0.1/javadocs/}{\textbf{Calendar:}}
\textbar{} com.liferay.calendar.api \textbar{}
\href{https://docs.liferay.com/portal/7.1-latest/apps/captcha-1.0.0/javadocs/}{\textbf{Captcha:}}
\textbar{} com.liferay.captcha.api com.liferay.captcha.taglib \textbar{}
\href{https://docs.liferay.com/portal/7.1-latest/apps/comment-1.0.1/javadocs/}{\textbf{Comment:}}
\textbar{} com.liferay.comment.api
com.liferay.comment.demo.data.creator.api com.liferay.comment.taglib
\textbar{}
\href{https://docs.liferay.com/portal/7.1-latest/apps/configuration-admin-1.0.1/javadocs/}{\textbf{Configuration
Admin:}} \textbar{} com.liferay.configuration.admin.api \textbar{}
\href{https://docs.liferay.com/portal/7.1-latest/apps/contacts-1.0.1/javadocs/}{\textbf{Contacts:}}
\textbar{} com.liferay.contacts.api \textbar{}
\href{https://docs.liferay.com/portal/7.1-latest/apps/document-library-1.0.1/javadocs/}{\textbf{Document
Library:}} \textbar{} com.liferay.document.library.api
com.liferay.document.library.content.api
com.liferay.document.library.demo.data.creator.api
com.liferay.document.library.file.rank.api
com.liferay.document.library.repository.authorization.api
com.liferay.document.library.repository.cmis.api
com.liferay.document.library.repository.external.api
com.liferay.document.library.sync.api
com.liferay.document.library.test.util \textbar{}
\href{https://docs.liferay.com/portal/7.1-latest/apps/dynamic-data-lists-1.0.1/javadocs/}{\textbf{Dynamic
Data Lists:}} \textbar{} com.liferay.dynamic.data.lists.api \textbar{}
\href{https://docs.liferay.com/portal/7.1-latest/apps/dynamic-data-mapping-1.0.1/javadocs/}{\textbf{Dynamic
Data Mapping:}} \textbar{} com.liferay.dynamic.data.mapping.api
com.liferay.dynamic.data.mapping.taglib
com.liferay.dynamic.data.mapping.test.util \textbar{}
\href{https://docs.liferay.com/portal/7.1-latest/apps/export-import-1.0.1/javadocs/}{\textbf{Export
Import:}} \textbar{} com.liferay.exportimport.api
com.liferay.exportimport.changeset.api
com.liferay.exportimport.changeset.taglib
com.liferay.exportimport.test.util \textbar{}
\href{https://docs.liferay.com/portal/7.1-latest/apps/flags-1.0.1/javadocs/}{\textbf{Flags:}}
\textbar{} com.liferay.flags.api com.liferay.flags.taglib \textbar{}
\href{https://docs.liferay.com/portal/7.1-latest/apps/fragment-1.0.1/javadocs/}{\textbf{Fragment:}}
\textbar{} com.liferay.fragment.api
com.liferay.fragment.demo.data.creator.api
com.liferay.fragment.item.selector.api \textbar{}
\href{https://docs.liferay.com/portal/7.1-latest/apps/friendly-url-1.0.1/javadocs/}{\textbf{Friendly
URL:}} \textbar{} com.liferay.friendly.url.api \textbar{}
\href{https://docs.liferay.com/portal/7.1-latest/apps/frontend-editor-1.0.1/javadocs/}{\textbf{Frontend
Editor:}} \textbar{} com.liferay.frontend.editor.api \textbar{}
\href{https://docs.liferay.com/portal/7.1-latest/apps/frontend-image-editor-1.0.1/javadocs/}{\textbf{Frontend
Image Editor:}} \textbar{} com.liferay.frontend.image.editor.capability
\textbar{}
\href{https://docs.liferay.com/portal/7.1-latest/apps/frontend-js-1.0.1/javadocs/}{\textbf{Frontend
JS:}} \textbar{} com.liferay.frontend.js.loader.modules.extender.npm
\textbar{}
\href{https://docs.liferay.com/portal/7.1-latest/apps/html-preview-1.0.1/javadocs/}{\textbf{HTML
Preview:}} \textbar{} com.liferay.html.preview.api \textbar{}
\href{https://docs.liferay.com/portal/7.1-latest/apps/invitation-1.0.1/javadocs/}{\textbf{Invitation:}}
\textbar{} com.liferay.invitation.invite.members.api \textbar{}
\href{https://docs.liferay.com/portal/7.1-latest/apps/item-selector-1.0.1/javadocs/}{\textbf{Item
Selector:}} \textbar{} com.liferay.item.selector.api
com.liferay.item.selector.criteria.api com.liferay.item.selector.taglib
\textbar{}
\href{https://docs.liferay.com/portal/7.1-latest/apps/journal-1.0.1/javadocs/}{\textbf{Journal:}}
\textbar{} com.liferay.journal.api
com.liferay.journal.content.asset.addon.entry.api
com.liferay.journal.demo.data.creator.api
com.liferay.journal.item.selector.api com.liferay.journal.taglib
com.liferay.journal.test.util \textbar{}
\href{https://docs.liferay.com/portal/7.1-latest/apps/layout-1.0.1/javadocs/}{\textbf{Layout:}}
\textbar{} com.liferay.layout.api com.liferay.layout.admin.api
com.liferay.layout.item.selector.api
com.liferay.layout.page.template.api com.liferay.layout.prototype.api
com.liferay.layout.set.prototype.api com.liferay.layout.taglib
\textbar{}
\href{https://docs.liferay.com/portal/7.1-latest/apps/map-1.0.1/javadocs/}{\textbf{Map:}}
\textbar{} com.liferay.map.api com.liferay.map.taglib \textbar{}
\href{https://docs.liferay.com/portal/7.1-latest/apps/mentions-1.0.1/javadocs/}{\textbf{Mentions:}}
\textbar{} com.liferay.mentions.api \textbar{}
\href{https://docs.liferay.com/portal/7.1-latest/apps/message-boards-1.0.1/javadocs/}{\textbf{Message
Boards:}} \textbar{} com.liferay.message.boards.api
com.liferay.message.boards.demo.data.creator.api
com.liferay.message.boards.test.util \textbar{}
\href{https://docs.liferay.com/portal/7.1-latest/apps/mobile-device-rules-1.0.1/javadocs/}{\textbf{Mobile
Device Rules:}} \textbar{} com.liferay.mobile.device.rules.api
\textbar{}
\href{https://docs.liferay.com/portal/7.1-latest/apps/organizations-1.0.1/javadocs/}{\textbf{Organizations:}}
\textbar{} com.liferay.organizations.api
com.liferay.organizations.item.selector.api \textbar{}
\href{https://docs.liferay.com/portal/7.1-latest/apps/password-policies-admin-1.0.1/javadocs/}{\textbf{Password
Policies Admin:}} \textbar{}
com.liferay.password.policies.admin.constants \textbar{}
\href{https://docs.liferay.com/portal/7.1-latest/apps/polls-1.0.1/javadocs/}{\textbf{Polls:}}
\textbar{} com.liferay.polls.api \textbar{}
\href{https://docs.liferay.com/portal/7.1-latest/apps/portal-1.0.1/javadocs/}{\textbf{Portal:}}
\textbar{} com.liferay.portal.custom.jsp.bag.api
com.liferay.portal.dao.orm.custom.sql.api
com.liferay.portal.instance.lifecycle.api com.liferay.portal.jmx.api
com.liferay.portal.output.stream.container.api
com.liferay.portal.spring.extender.api com.liferay.portal.upgrade.api
\textbar{}
\href{https://docs.liferay.com/portal/7.1-latest/apps/portal-background-task-1.0.1/javadocs/}{\textbf{Portal
Background Task:}} \textbar{} com.liferay.portal.background.task.api
\textbar{}
\href{https://docs.liferay.com/portal/7.1-latest/apps/portal-cache-1.0.1/javadocs/}{\textbf{Portal
Cache:}} \textbar{} com.liferay.portal.cache.api
com.liferay.portal.cache.ehcache.spi com.liferay.portal.cache.test.util
\textbar{}
\href{https://docs.liferay.com/portal/7.1-latest/apps/portal-configuration-1.0.1/javadocs/}{\textbf{Portal
Configuration:}} \textbar{} com.liferay.portal.configuration.test.util
com.liferay.portal.configuration.upgrade.api \textbar{}
\href{https://docs.liferay.com/portal/7.1-latest/apps/portal-instances-1.0.1/javadocs/}{\textbf{Portal
Instances:}} \textbar{} com.liferay.portal.instances.service \textbar{}
\href{https://docs.liferay.com/portal/7.1-latest/apps/portal-lock-1.0.1/javadocs/}{\textbf{Portal
Lock:}} \textbar{} com.liferay.portal.lock.api \textbar{}
\href{https://docs.liferay.com/portal/7.1-latest/apps/portal-reports-engine-1.0.1/javadocs/}{\textbf{Portal
Reports Engine:}} \textbar{} com.liferay.portal.reports.engine.api
\textbar{}
\href{https://docs.liferay.com/portal/7.1-latest/apps/portal-remote-1.0.1/javadocs/}{\textbf{Portal
Remote:}} \textbar{} com.liferay.portal.remote.soap.extender \textbar{}
\href{https://docs.liferay.com/portal/7.1-latest/apps/portal-rules-engine-1.0.1/javadocs/com/liferay/portal/rules/engine/package-summary.html}{\textbf{Portal
Rules:}} \textbar{} com.liferay.portal.rules.engine \textbar{}
\href{https://docs.liferay.com/portal/7.1-latest/apps/portal-scripting-1.0.1/javadocs/}{\textbf{Portal
Scripting:}} \textbar{} com.liferay.portal.scripting \textbar{}
\href{https://docs.liferay.com/portal/7.1-latest/apps/portal-search-1.0.1/javadocs/}{\textbf{Portal
Search:}} \textbar{} com.liferay.portal.search.api
com.liferay.portal.search.engine.adapter.api
com.liferay.portal.search.spi com.liferay.portal.search.test.util
com.liferay.portal.search.web.api \textbar{}
\href{https://docs.liferay.com/portal/7.1-latest/apps/portal-security-1.0.1/javadocs/}{\textbf{Portal
Security:}} \textbar{} com.liferay.portal.security.exportimport.api
com.liferay.portal.security.ldap.api
com.liferay.portal.security.permission.api
com.liferay.portal.security.service.access.policy.api
com.liferay.portal.security.service.access.quota.api \textbar{}
\href{https://docs.liferay.com/portal/7.1-latest/apps/portal-security-audit-1.0.1/javadocs/}{\textbf{Portal
Security Audit:}} \textbar{} com.liferay.portal.security.audit.api
com.liferay.portal.security.audit.event.generators.api
com.liferay.portal.security.audit.storage.api \textbar{}
\href{https://docs.liferay.com/portal/7.1-latest/apps/portal-security-sso-1.0.1/javadocs/}{\textbf{Portal
Security SSO:}} \textbar{} com.liferay.portal.security.sso.cas.api
com.liferay.portal.security.sso.facebook.connect.api
com.liferay.portal.security.sso.google.api
com.liferay.portal.security.sso.ntlm.api
com.liferay.portal.security.sso.openid.api
com.liferay.portal.security.sso.openid.connect.api
com.liferay.portal.security.sso.opensso.api
com.liferay.portal.security.sso.token.api \textbar{}
\href{https://docs.liferay.com/portal/7.1-latest/apps/portal-settings-1.0.1/javadocs/}{\textbf{Portal
Settings:}} \textbar{} com.liferay.portal.settings.api \textbar{}
\href{https://docs.liferay.com/portal/7.1-latest/apps/portal-template-1.0.1/javadocs/}{\textbf{Portal
Template:}} \textbar{} com.liferay.portal.template.soy.api \textbar{}
\href{https://docs.liferay.com/portal/7.1-latest/apps/portal-url-builder-1.0.1/javadocs/}{\textbf{Portal
URL Builder:}} \textbar{} com.liferay.portal.url.builder \textbar{}
\href{https://docs.liferay.com/portal/7.1-latest/apps/portal-workflow-1.0.1/javadocs/}{\textbf{Portal
Workflow:}} \textbar{} com.liferay.portal.workflow.api
com.liferay.portal.workflow.kaleo.api
com.liferay.portal.workflow.kaleo.definition.api
com.liferay.portal.workflow.kaleo.runtime.api \textbar{}
\href{https://docs.liferay.com/portal/7.1-latest/apps/portlet-display-template-1.0.1/javadocs/}{\textbf{Portlet
Display Template:}} \textbar{} com.liferay.portlet.display.template.api
\textbar{}
\href{https://docs.liferay.com/portal/7.1-latest/apps/product-navigation-1.0.1/javadocs/}{\textbf{Product
Navigation:}} \textbar{} com.liferay.product.navigation.control.menu.api
com.liferay.product.navigation.product.menu.api
com.liferay.product.navigation.simulation.api
com.liferay.product.navigation.taglib \textbar{}
\href{https://docs.liferay.com/portal/7.1-latest/apps/ratings-1.0.1/javadocs/}{\textbf{Ratings:}}
\textbar{} com.liferay.ratings.page.ratings.constants \textbar{}
\href{https://docs.liferay.com/portal/7.1-latest/apps/reading-time-1.0.1/javadocs/}{\textbf{Reading
Time:}} \textbar{} com.liferay.reading.time.api
com.liferay.reading.time.taglib \textbar{}
\href{https://docs.liferay.com/portal/7.1-latest/apps/roles-1.0.1/javadocs/}{\textbf{Roles:}}
\textbar{} com.liferay.roles.admin.api
com.liferay.roles.admin.demo.data.creator.api
com.liferay.roles.item.selector.api \textbar{}
\href{https://docs.liferay.com/portal/7.1-latest/apps/rss-1.0.1/javadocs/}{\textbf{RSS:}}
\textbar{} com.liferay.rss.api com.liferay.rss.taglib \textbar{}
\href{https://docs.liferay.com/portal/7.1-latest/apps/site-1.0.1/javadocs/}{\textbf{Site:}}
\textbar{} com.liferay.site.api com.liferay.site.demo.data.creator.api
com.liferay.site.item.selector.api com.liferay.site.taglib \textbar{}
\href{https://docs.liferay.com/portal/7.1-latest/apps/social-1.0.1/javadocs/}{\textbf{Social:}}
\textbar{} com.liferay.social.activities.api
com.liferay.social.activities.taglib com.liferay.social.activity.api
com.liferay.social.activity.test.util com.liferay.social.bookmarks.api
com.liferay.social.bookmarks.taglib
com.liferay.social.user.statistics.api \textbar{}
\href{https://docs.liferay.com/portal/7.1-latest/apps/staging-1.0.1/javadocs/}{\textbf{Staging:}}
\textbar{} com.liferay.staging.api com.liferay.staging.taglib \textbar{}
\href{https://docs.liferay.com/portal/7.1-latest/apps/subscription-1.0.1/javadocs/}{\textbf{Subscription:}}
\textbar{} com.liferay.subscription.api \textbar{}
\href{https://docs.liferay.com/portal/7.1-latest/apps/text-localizer-1.0.1/javadocs/}{\textbf{Text
Localizer:}} \textbar{} com.liferay.text.localizer.address.api
com.liferay.text.localizer.taglib \textbar{}
\href{https://docs.liferay.com/portal/7.1-latest/apps/trash-1.0.1/javadocs/}{\textbf{Trash:}}
\textbar{} com.liferay.trash.api com.liferay.trash.taglib
com.liferay.trash.test.util \textbar{}
\href{https://docs.liferay.com/portal/7.1-latest/apps/upload-1.0.1/javadocs/}{\textbf{Upload:}}
\textbar{} com.liferay.upload \textbar{}
\href{https://docs.liferay.com/portal/7.1-latest/apps/user-associated-data-1.0.1/javadocs/}{\textbf{User
Associated Data:}} \textbar{} com.liferay.user.associated.data.api
com.liferay.user.associated.data.test.util \textbar{}
\href{https://docs.liferay.com/portal/7.1-latest/apps/user-groups-admin-1.0.1/javadocs/}{\textbf{User
Groups Admin:}} \textbar{} com.liferay.user.groups.admin.api
com.liferay.user.groups.admin.item.selector.api \textbar{}
\href{https://docs.liferay.com/portal/7.1-latest/apps/users-admin-1.0.1/javadocs/}{\textbf{Users
Admin:}} \textbar{} com.liferay.users.admin.api
com.liferay.users.admin.demo.data.creator.api
com.liferay.users.admin.item.selector.api
com.liferay.users.admin.test.util \textbar{}
\href{https://docs.liferay.com/portal/7.1-latest/apps/wiki-1.0.1/javadocs/}{\textbf{Wiki:}}
\textbar{} com.liferay.wiki.api \textbar{}
\href{https://docs.liferay.com/portal/7.1-latest/apps/xstream-1.0.1/javadocs/}{\textbf{XStream:}}
\textbar{} com.liferay.xstream.configurator \textbar{}

\noindent\hrulefill

For help finding module attributes and configuring dependencies, see
\href{/docs/7-1/tutorials/-/knowledge_base/t/configuring-dependencies}{Configuring
Dependencies}.

\section{Taglibs}\label{taglibs}

Here you'll find tag library documentation for the Liferay DXP, Liferay
DXP apps, and Liferay Faces.

\subsection{7.0 Taglibs}\label{taglibs-1}

Util Taglibs{ (Opens New Window) }

JSTL core

aui

liferay-portlet

portlet

portlet\_1\_0

liferay-security

liferay-theme

liferay-ui

liferay-util

\subsection{Liferay DXP App Taglibs}\label{liferay-dxp-app-taglibs}

Adaptive Media:

liferay-application-list{ (Opens New Window) }

Application List:

liferay-application-list{ (Opens New Window) }

Assets:

liferay-asset{ (Opens New Window) }

liferay-trash{ (Opens New Window) }

Forms:

liferay-ddm{ (Opens New Window) }

Foundation:

liferay-map{ (Opens New Window) }

liferay-frontend{ (Opens New Window) }

Import, Export, \& Staging:

liferay-staging{ (Opens New Window) }

Item Selector:

liferay-item-selector{ (Opens New Window) }

Product Navigation:

liferay-product-navigation{ (Opens New Window) }

Sites:

liferay-layout{ (Opens New Window) }

liferay-site-navigation{ (Opens New Window) }

Social:

liferay-flags{ (Opens New Window) }

For help finding module attributes and configuring dependencies, see
\href{/docs/7-1/tutorials/-/knowledge_base/t/configuring-dependencies}{Configuring
Dependencies}.

\subsection{Faces Taglibs}\label{faces-taglibs}

\href{https://docs.liferay.com/faces/3.2/vdldoc/}{\textbf{Faces 3.2
Taglibs}}: the latest version of Liferay Faces JSF tag docs in View
Declaration Language (VDL) format. VDL docs for all versions of Liferay
Faces are available \href{http://docs.liferay.com/faces/}{here}.

\section{JavaScript and CSS}\label{javascript-and-css}

\href{https://lexicondesign.io/}{\textbf{Lexicon}}: A system for
building applications in and outside of Liferay DXP, designed to be
fluid and extensible, as well as provide a consistent and documented
API.

\href{https://clayui.com/}{\textbf{Clay}}: The web implementation of
Liferay's \href{https://lexicondesign.io/}{Lexicon Experience Language}.

\href{http://getbootstrap.com/}{\textbf{Bootstrap}}: The base CSS
library onto which Lexicon is built. Liferay DXP uses Bootstrap natively
and all of its CSS classes and JavaScript features are available within
portlets, templates, and themes.

\href{http://alloyui.com}{\textbf{AlloyUI}}: Liferay includes AlloyUI
and all of its JavaScript APIs are available within portlets, templates
and themes.

\subsection{Descriptor Definitions}\label{descriptor-definitions}

\href{https://docs.liferay.com/dxp/portal/7.1-latest/definitions/}{\textbf{DTDs}}:
Describes the XML files used in configuring Liferay DXP apps, 7.0
plugins, and Liferay DXP 7.1.

\chapter{Back-End}\label{back-end}

As you create portlets and customizations, it helps to reference backend
APIs, descriptors, and third-party artifacts. These articles provide
such references.

\section{Classes Moved from
portal-service.jar}\label{classes-moved-from-portal-service.jar}

To leverage the benefits of modularization in 7.0, many classes from
former Liferay Portal 6 JAR file portal-service.jar have been moved into
application and framework API modules. The table below provides details
about these classes and the modules they've moved to. Package changes
and each module's symbolic name (artifact ID) are listed, to facilitate
configuring dependencies.

Classes Moved from portal-service to modules

This information was generated based on comparing classes in
liferay-portal-src-6.2-ee-sp20 to classes in liferay-dxp-src-7.1.10-ga1.

Class

Package

Module Symbolic Name (Artifact ID)

ActionHandler

Old: com.liferay.portal.kernel.mobile.device.rulegroup.action New:
com.liferay.mobile.device.rules.action

com.liferay.mobile.device.rules.api

ActionHandlerManager

Old: com.liferay.portal.kernel.mobile.device.rulegroup New:
com.liferay.mobile.device.rules.action

com.liferay.mobile.device.rules.api

ActionHandlerManagerUtil

Old: com.liferay.portal.kernel.mobile.device.rulegroup New:
com.liferay.mobile.device.rules.action

com.liferay.mobile.device.rules.api

ActionTypeException

Old: com.liferay.portlet.mobiledevicerules New:
com.liferay.mobile.device.rules.exception

com.liferay.mobile.device.rules.api

AlternateKeywordQueryHitsProcessor

Old: com.liferay.portal.kernel.search New:
com.liferay.portal.search.internal.hits

com.liferay.portal.search

ArticleContentException

Old: com.liferay.portlet.journal New: com.liferay.journal.exception

com.liferay.journal.api

ArticleContentSizeException

Old: com.liferay.portlet.journal New: com.liferay.journal.exception

com.liferay.journal.api

ArticleCreateDateComparator

Old: com.liferay.portlet.journal.util.comparator New:
com.liferay.journal.util.comparator

com.liferay.journal.api

ArticleDisplayDateComparator

Old: com.liferay.portlet.journal.util.comparator New:
com.liferay.journal.util.comparator

com.liferay.journal.api

ArticleDisplayDateException

Old: com.liferay.portlet.journal New: com.liferay.journal.exception

com.liferay.journal.api

ArticleExpirationDateException

Old: com.liferay.portlet.journal New: com.liferay.journal.exception

com.liferay.journal.api

ArticleIDComparator

Old: com.liferay.portlet.journal.util.comparator New:
com.liferay.journal.util.comparator

com.liferay.journal.api

ArticleIdException

Old: com.liferay.portlet.journal New: com.liferay.journal.exception

com.liferay.journal.api

ArticleModifiedDateComparator

Old: com.liferay.portlet.journal.util.comparator New:
com.liferay.journal.util.comparator

com.liferay.journal.api

ArticleResourcePKComparator

Old: com.liferay.portlet.journal.util.comparator New:
com.liferay.journal.util.comparator

com.liferay.journal.api

ArticleReviewDateComparator

Old: com.liferay.portlet.journal.util.comparator New:
com.liferay.journal.util.comparator

com.liferay.journal.api

ArticleReviewDateException

Old: com.liferay.portlet.journal New: com.liferay.journal.exception

com.liferay.journal.api

ArticleSmallImageNameException

Old: com.liferay.portlet.journal New: com.liferay.journal.exception

com.liferay.journal.api

ArticleSmallImageSizeException

Old: com.liferay.portlet.journal New: com.liferay.journal.exception

com.liferay.journal.api

ArticleTitleComparator

Old: com.liferay.portlet.journal.util.comparator New:
com.liferay.journal.util.comparator

com.liferay.journal.api

ArticleTitleException

Old: com.liferay.portlet.journal New: com.liferay.journal.exception

com.liferay.journal.api

ArticleVersionComparator

Old: com.liferay.portlet.journal.util.comparator New:
com.liferay.journal.util.comparator

com.liferay.journal.api

ArticleVersionException

Old: com.liferay.portlet.journal New: com.liferay.journal.exception

com.liferay.journal.api

AssetPublisherUtil

Old: com.liferay.portlet.assetpublisher.util New:
com.liferay.asset.publisher.web.util

com.liferay.asset.publisher.web

AuditMessageProcessor

Old: com.liferay.portal.kernel.audit New:
com.liferay.portal.security.audit

com.liferay.portal.security.audit.api

AutoDeleteFileInputStream

Old: com.liferay.portal.kernel.io New: com.liferay.petra.io

com.liferay.petra.io

AverageStatistics

Old: com.liferay.portal.kernel.monitoring.statistics New:
com.liferay.portal.monitoring.internal.statistics

com.liferay.portal.monitoring

BackgroundTaskLocalService

Old: com.liferay.portal.service New:
com.liferay.portal.background.task.service

com.liferay.portal.background.task.api

BackgroundTaskLocalServiceUtil

Old: com.liferay.portal.service New:
com.liferay.portal.background.task.service

com.liferay.portal.background.task.api

BackgroundTaskLocalServiceWrapper

Old: com.liferay.portal.service New:
com.liferay.portal.background.task.service

com.liferay.portal.background.task.api

BackgroundTaskModel

Old: com.liferay.portal.model New:
com.liferay.portal.background.task.model

com.liferay.portal.background.task.api

BackgroundTaskPersistence

Old: com.liferay.portal.service.persistence New:
com.liferay.portal.background.task.service.persistence

com.liferay.portal.background.task.api

BackgroundTaskService

Old: com.liferay.portal.service New:
com.liferay.portal.background.task.service

com.liferay.portal.background.task.api

BackgroundTaskServiceUtil

Old: com.liferay.portal.service New:
com.liferay.portal.background.task.service

com.liferay.portal.background.task.api

BackgroundTaskServiceWrapper

Old: com.liferay.portal.service New:
com.liferay.portal.background.task.service

com.liferay.portal.background.task.api

BackgroundTaskSoap

Old: com.liferay.portal.model New:
com.liferay.portal.background.task.model

com.liferay.portal.background.task.api

BackgroundTaskUtil

Old: com.liferay.portal.service.persistence New:
com.liferay.portal.background.task.service.persistence

com.liferay.portal.background.task.api

BackgroundTaskWrapper

Old: com.liferay.portal.model New:
com.liferay.portal.background.task.model

com.liferay.portal.background.task.api

BannedUserException

Old: com.liferay.portlet.messageboards New:
com.liferay.message.boards.exception

com.liferay.message.boards.api

BaseCmisRepository

Old: com.liferay.portal.kernel.repository.cmis New:
com.liferay.document.library.repository.cmis

com.liferay.document.library.repository.cmis.api

BaseCmisSearchQueryBuilder

Old: com.liferay.portal.kernel.repository.cmis.search New:
com.liferay.document.library.repository.cmis.search

com.liferay.document.library.repository.cmis.api

BaseDDLExporter

Old: com.liferay.portlet.dynamicdatalists.util New:
com.liferay.dynamic.data.lists.internal.exporter

com.liferay.dynamic.data.lists.service

BaseDDMDisplay

Old: com.liferay.portlet.dynamicdatamapping.util New:
com.liferay.dynamic.data.mapping.util

com.liferay.dynamic.data.mapping.api

BaseFieldRenderer

Old: com.liferay.portlet.dynamicdatamapping.storage New:
com.liferay.dynamic.data.mapping.storage

com.liferay.dynamic.data.mapping.api

BaseScriptingExecutor

Old: com.liferay.portal.kernel.scripting New:
com.liferay.portal.scripting

com.liferay.portal.scripting.api

BaseStatistics

Old: com.liferay.portal.kernel.monitoring.statistics New:
com.liferay.portal.monitoring.internal.statistics

com.liferay.portal.monitoring

BaseStorageAdapter

Old: com.liferay.portlet.dynamicdatamapping.storage New:
com.liferay.dynamic.data.mapping.storage

com.liferay.dynamic.data.mapping.api

BlockingPortalCache

Old: com.liferay.portal.kernel.cache New: com.liferay.portal.cache

com.liferay.portal.cache.api

BlogsEntry

Old: com.liferay.portlet.blogs.model New: com.liferay.blogs.model

com.liferay.blogs.api

BlogsEntryFinder

Old: com.liferay.portlet.blogs.service.persistence New:
com.liferay.blogs.service.persistence

com.liferay.blogs.api

BlogsEntryLocalService

Old: com.liferay.portlet.blogs.service New: com.liferay.blogs.service

com.liferay.blogs.api

BlogsEntryLocalServiceUtil

Old: com.liferay.portlet.blogs.service New: com.liferay.blogs.service

com.liferay.blogs.api

BlogsEntryLocalServiceWrapper

Old: com.liferay.portlet.blogs.service New: com.liferay.blogs.service

com.liferay.blogs.api

BlogsEntryModel

Old: com.liferay.portlet.blogs.model New: com.liferay.blogs.model

com.liferay.blogs.api

BlogsEntryPersistence

Old: com.liferay.portlet.blogs.service.persistence New:
com.liferay.blogs.service.persistence

com.liferay.blogs.api

BlogsEntryService

Old: com.liferay.portlet.blogs.service New: com.liferay.blogs.service

com.liferay.blogs.api

BlogsEntryServiceUtil

Old: com.liferay.portlet.blogs.service New: com.liferay.blogs.service

com.liferay.blogs.api

BlogsEntryServiceWrapper

Old: com.liferay.portlet.blogs.service New: com.liferay.blogs.service

com.liferay.blogs.api

BlogsEntrySoap

Old: com.liferay.portlet.blogs.model New: com.liferay.blogs.model

com.liferay.blogs.api

BlogsEntryUtil

Old: com.liferay.portlet.blogs.service.persistence New:
com.liferay.blogs.service.persistence

com.liferay.blogs.api

BlogsEntryWrapper

Old: com.liferay.portlet.blogs.model New: com.liferay.blogs.model

com.liferay.blogs.api

BlogsStatsUser

Old: com.liferay.portlet.blogs.model New: com.liferay.blogs.model

com.liferay.blogs.api

BlogsStatsUserFinder

Old: com.liferay.portlet.blogs.service.persistence New:
com.liferay.blogs.service.persistence

com.liferay.blogs.api

BlogsStatsUserLocalService

Old: com.liferay.portlet.blogs.service New: com.liferay.blogs.service

com.liferay.blogs.api

BlogsStatsUserLocalServiceUtil

Old: com.liferay.portlet.blogs.service New: com.liferay.blogs.service

com.liferay.blogs.api

BlogsStatsUserLocalServiceWrapper

Old: com.liferay.portlet.blogs.service New: com.liferay.blogs.service

com.liferay.blogs.api

BlogsStatsUserModel

Old: com.liferay.portlet.blogs.model New: com.liferay.blogs.model

com.liferay.blogs.api

BlogsStatsUserPersistence

Old: com.liferay.portlet.blogs.service.persistence New:
com.liferay.blogs.service.persistence

com.liferay.blogs.api

BlogsStatsUserSoap

Old: com.liferay.portlet.blogs.model New: com.liferay.blogs.model

com.liferay.blogs.api

BlogsStatsUserUtil

Old: com.liferay.portlet.blogs.service.persistence New:
com.liferay.blogs.service.persistence

com.liferay.blogs.api

BlogsStatsUserWrapper

Old: com.liferay.portlet.blogs.model New: com.liferay.blogs.model

com.liferay.blogs.api

BookmarksEntry

Old: com.liferay.portlet.bookmarks.model New:
com.liferay.bookmarks.model

com.liferay.bookmarks.api

BookmarksEntryFinder

Old: com.liferay.portlet.bookmarks.service.persistence New:
com.liferay.bookmarks.service.persistence

com.liferay.bookmarks.api

BookmarksEntryLocalService

Old: com.liferay.portlet.bookmarks.service New:
com.liferay.bookmarks.service

com.liferay.bookmarks.api

BookmarksEntryLocalServiceUtil

Old: com.liferay.portlet.bookmarks.service New:
com.liferay.bookmarks.service

com.liferay.bookmarks.api

BookmarksEntryLocalServiceWrapper

Old: com.liferay.portlet.bookmarks.service New:
com.liferay.bookmarks.service

com.liferay.bookmarks.api

BookmarksEntryModel

Old: com.liferay.portlet.bookmarks.model New:
com.liferay.bookmarks.model

com.liferay.bookmarks.api

BookmarksEntryPersistence

Old: com.liferay.portlet.bookmarks.service.persistence New:
com.liferay.bookmarks.service.persistence

com.liferay.bookmarks.api

BookmarksEntryService

Old: com.liferay.portlet.bookmarks.service New:
com.liferay.bookmarks.service

com.liferay.bookmarks.api

BookmarksEntryServiceUtil

Old: com.liferay.portlet.bookmarks.service New:
com.liferay.bookmarks.service

com.liferay.bookmarks.api

BookmarksEntryServiceWrapper

Old: com.liferay.portlet.bookmarks.service New:
com.liferay.bookmarks.service

com.liferay.bookmarks.api

BookmarksEntrySoap

Old: com.liferay.portlet.bookmarks.model New:
com.liferay.bookmarks.model

com.liferay.bookmarks.api

BookmarksEntryUtil

Old: com.liferay.portlet.bookmarks.service.persistence New:
com.liferay.bookmarks.service.persistence

com.liferay.bookmarks.api

BookmarksEntryWrapper

Old: com.liferay.portlet.bookmarks.model New:
com.liferay.bookmarks.model

com.liferay.bookmarks.api

BookmarksFolder

Old: com.liferay.portlet.bookmarks.model New:
com.liferay.bookmarks.model

com.liferay.bookmarks.api

BookmarksFolderConstants

Old: com.liferay.portlet.bookmarks.model New:
com.liferay.bookmarks.model

com.liferay.bookmarks.api

BookmarksFolderFinder

Old: com.liferay.portlet.bookmarks.service.persistence New:
com.liferay.bookmarks.service.persistence

com.liferay.bookmarks.api

BookmarksFolderLocalService

Old: com.liferay.portlet.bookmarks.service New:
com.liferay.bookmarks.service

com.liferay.bookmarks.api

BookmarksFolderLocalServiceUtil

Old: com.liferay.portlet.bookmarks.service New:
com.liferay.bookmarks.service

com.liferay.bookmarks.api

BookmarksFolderLocalServiceWrapper

Old: com.liferay.portlet.bookmarks.service New:
com.liferay.bookmarks.service

com.liferay.bookmarks.api

BookmarksFolderModel

Old: com.liferay.portlet.bookmarks.model New:
com.liferay.bookmarks.model

com.liferay.bookmarks.api

BookmarksFolderPersistence

Old: com.liferay.portlet.bookmarks.service.persistence New:
com.liferay.bookmarks.service.persistence

com.liferay.bookmarks.api

BookmarksFolderService

Old: com.liferay.portlet.bookmarks.service New:
com.liferay.bookmarks.service

com.liferay.bookmarks.api

BookmarksFolderServiceUtil

Old: com.liferay.portlet.bookmarks.service New:
com.liferay.bookmarks.service

com.liferay.bookmarks.api

BookmarksFolderServiceWrapper

Old: com.liferay.portlet.bookmarks.service New:
com.liferay.bookmarks.service

com.liferay.bookmarks.api

BookmarksFolderSoap

Old: com.liferay.portlet.bookmarks.model New:
com.liferay.bookmarks.model

com.liferay.bookmarks.api

BookmarksFolderUtil

Old: com.liferay.portlet.bookmarks.service.persistence New:
com.liferay.bookmarks.service.persistence

com.liferay.bookmarks.api

BookmarksFolderWrapper

Old: com.liferay.portlet.bookmarks.model New:
com.liferay.bookmarks.model

com.liferay.bookmarks.api

ByteArrayReportResultContainer

Old: com.liferay.portal.kernel.bi.reporting New:
com.liferay.portal.reports.engine

com.liferay.portal.reports.engine.api

CMISBetweenExpression

Old: com.liferay.portal.kernel.repository.cmis.search New:
com.liferay.document.library.repository.cmis.search

com.liferay.document.library.repository.cmis.api

CMISConjunction

Old: com.liferay.portal.kernel.repository.cmis.search New:
com.liferay.document.library.repository.cmis.search

com.liferay.document.library.repository.cmis.api

CMISContainsExpression

Old: com.liferay.portal.kernel.repository.cmis.search New:
com.liferay.document.library.repository.cmis.search

com.liferay.document.library.repository.cmis.api

CMISContainsNotExpression

Old: com.liferay.portal.kernel.repository.cmis.search New:
com.liferay.document.library.repository.cmis.search

com.liferay.document.library.repository.cmis.api

CMISContainsValueExpression

Old: com.liferay.portal.kernel.repository.cmis.search New:
com.liferay.document.library.repository.cmis.search

com.liferay.document.library.repository.cmis.api

CMISCriterion

Old: com.liferay.portal.kernel.repository.cmis.search New:
com.liferay.document.library.repository.cmis.search

com.liferay.document.library.repository.cmis.api

CMISDisjunction

Old: com.liferay.portal.kernel.repository.cmis.search New:
com.liferay.document.library.repository.cmis.search

com.liferay.document.library.repository.cmis.api

CMISFullTextConjunction

Old: com.liferay.portal.kernel.repository.cmis.search New:
com.liferay.document.library.repository.cmis.search

com.liferay.document.library.repository.cmis.api

CMISInFolderExpression

Old: com.liferay.portal.kernel.repository.cmis.search New:
com.liferay.document.library.repository.cmis.search

com.liferay.document.library.repository.cmis.api

CMISInTreeExpression

Old: com.liferay.portal.kernel.repository.cmis.search New:
com.liferay.document.library.repository.cmis.search

com.liferay.document.library.repository.cmis.api

CMISJunction

Old: com.liferay.portal.kernel.repository.cmis.search New:
com.liferay.document.library.repository.cmis.search

com.liferay.document.library.repository.cmis.api

CMISNotExpression

Old: com.liferay.portal.kernel.repository.cmis.search New:
com.liferay.document.library.repository.cmis.search

com.liferay.document.library.repository.cmis.api

CMISParameterValueUtil

Old: com.liferay.portal.kernel.repository.cmis.search New:
com.liferay.document.library.repository.cmis.search

com.liferay.document.library.repository.cmis.api

CMISRepositoryHandler

Old: com.liferay.portal.kernel.repository.cmis New:
com.liferay.document.library.repository.cmis

com.liferay.document.library.repository.cmis.api

CMISRepositoryUtil

Old: com.liferay.portal.kernel.repository.cmis New:
com.liferay.document.library.repository.cmis.internal

com.liferay.document.library.repository.cmis.impl

CMISSearchQueryBuilder

Old: com.liferay.portal.kernel.repository.cmis.search New:
com.liferay.document.library.repository.cmis.search

com.liferay.document.library.repository.cmis.api

CMISSimpleExpression

Old: com.liferay.portal.kernel.repository.cmis.search New:
com.liferay.document.library.repository.cmis.search

com.liferay.document.library.repository.cmis.api

CMISSimpleExpressionOperator

Old: com.liferay.portal.kernel.repository.cmis.search New:
com.liferay.document.library.repository.cmis.search

com.liferay.document.library.repository.cmis.api

CharPool

Old: com.liferay.portal.kernel.util New: com.liferay.petra.string

com.liferay.petra.string

CharsetDecoderUtil

Old: com.liferay.portal.kernel.nio.charset New: com.liferay.petra.nio

com.liferay.petra.nio

CharsetEncoderUtil

Old: com.liferay.portal.kernel.nio.charset New: com.liferay.petra.nio

com.liferay.petra.nio

ClassLoaderPool

Old: com.liferay.portal.kernel.util New: com.liferay.petra.lang

com.liferay.petra.lang

ClassResolverUtil

Old: com.liferay.portal.kernel.util New: com.liferay.petra.lang

com.liferay.petra.lang

CollatedSpellCheckHitsProcessor

Old: com.liferay.portal.kernel.search New:
com.liferay.portal.search.internal.hits

com.liferay.portal.search

CompoundSessionIdServletRequest

Old: com.liferay.portal.kernel.servlet.filters.compoundsessionid New:
com.liferay.portal.compound.session.id.internal

com.liferay.portal.compound.session.id

Condition

Old: com.liferay.portlet.dynamicdatamapping.storage.query New:
com.liferay.adaptive.media.image.media.query

com.liferay.adaptive.media.image.api

ContactConverterKeys

Old: com.liferay.portal.security.ldap New:
com.liferay.portal.security.ldap

com.liferay.portal.security.ldap.api

ContentException

Old: com.liferay.portlet.dynamicdatamapping New:
com.liferay.dynamic.data.mapping.exception

com.liferay.dynamic.data.mapping.api

ContentNameException

Old: com.liferay.portlet.dynamicdatamapping New:
com.liferay.dynamic.data.mapping.exception

com.liferay.dynamic.data.mapping.api

ContextClassloaderReportDesignRetriever

Old: com.liferay.portal.kernel.bi.reporting New:
com.liferay.portal.reports.engine

com.liferay.portal.reports.engine.api

CountStatistics

Old: com.liferay.portal.kernel.monitoring.statistics New:
com.liferay.portal.monitoring.internal.statistics

com.liferay.portal.monitoring

DDL

Old: com.liferay.portlet.dynamicdatalists.util New:
com.liferay.dynamic.data.lists.util

com.liferay.dynamic.data.lists.api

DDLExporter

Old: com.liferay.portlet.dynamicdatalists.util New:
com.liferay.dynamic.data.lists.exporter

com.liferay.dynamic.data.lists.api

DDLExporterFactory

Old: com.liferay.portlet.dynamicdatalists.util New:
com.liferay.dynamic.data.lists.exporter

com.liferay.dynamic.data.lists.api

DDLRecord

Old: com.liferay.portlet.dynamicdatalists.model New:
com.liferay.dynamic.data.lists.model

com.liferay.dynamic.data.lists.api

DDLRecordConstants

Old: com.liferay.portlet.dynamicdatalists.model New:
com.liferay.dynamic.data.lists.model

com.liferay.dynamic.data.lists.api

DDLRecordFinder

Old: com.liferay.portlet.dynamicdatalists.service.persistence New:
com.liferay.dynamic.data.lists.service.persistence

com.liferay.dynamic.data.lists.api

DDLRecordLocalService

Old: com.liferay.portlet.dynamicdatalists.service New:
com.liferay.dynamic.data.lists.service

com.liferay.dynamic.data.lists.api

DDLRecordLocalServiceUtil

Old: com.liferay.portlet.dynamicdatalists.service New:
com.liferay.dynamic.data.lists.service

com.liferay.dynamic.data.lists.api

DDLRecordLocalServiceWrapper

Old: com.liferay.portlet.dynamicdatalists.service New:
com.liferay.dynamic.data.lists.service

com.liferay.dynamic.data.lists.api

DDLRecordModel

Old: com.liferay.portlet.dynamicdatalists.model New:
com.liferay.dynamic.data.lists.model

com.liferay.dynamic.data.lists.api

DDLRecordPersistence

Old: com.liferay.portlet.dynamicdatalists.service.persistence New:
com.liferay.dynamic.data.lists.service.persistence

com.liferay.dynamic.data.lists.api

DDLRecordService

Old: com.liferay.portlet.dynamicdatalists.service New:
com.liferay.dynamic.data.lists.service

com.liferay.dynamic.data.lists.api

DDLRecordServiceUtil

Old: com.liferay.portlet.dynamicdatalists.service New:
com.liferay.dynamic.data.lists.service

com.liferay.dynamic.data.lists.api

DDLRecordServiceWrapper

Old: com.liferay.portlet.dynamicdatalists.service New:
com.liferay.dynamic.data.lists.service

com.liferay.dynamic.data.lists.api

DDLRecordSet

Old: com.liferay.portlet.dynamicdatalists.model New:
com.liferay.dynamic.data.lists.model

com.liferay.dynamic.data.lists.api

DDLRecordSetConstants

Old: com.liferay.portlet.dynamicdatalists.model New:
com.liferay.dynamic.data.lists.model

com.liferay.dynamic.data.lists.api

DDLRecordSetFinder

Old: com.liferay.portlet.dynamicdatalists.service.persistence New:
com.liferay.dynamic.data.lists.service.persistence

com.liferay.dynamic.data.lists.api

DDLRecordSetLocalService

Old: com.liferay.portlet.dynamicdatalists.service New:
com.liferay.dynamic.data.lists.service

com.liferay.dynamic.data.lists.api

DDLRecordSetLocalServiceUtil

Old: com.liferay.portlet.dynamicdatalists.service New:
com.liferay.dynamic.data.lists.service

com.liferay.dynamic.data.lists.api

DDLRecordSetLocalServiceWrapper

Old: com.liferay.portlet.dynamicdatalists.service New:
com.liferay.dynamic.data.lists.service

com.liferay.dynamic.data.lists.api

DDLRecordSetModel

Old: com.liferay.portlet.dynamicdatalists.model New:
com.liferay.dynamic.data.lists.model

com.liferay.dynamic.data.lists.api

DDLRecordSetPersistence

Old: com.liferay.portlet.dynamicdatalists.service.persistence New:
com.liferay.dynamic.data.lists.service.persistence

com.liferay.dynamic.data.lists.api

DDLRecordSetService

Old: com.liferay.portlet.dynamicdatalists.service New:
com.liferay.dynamic.data.lists.service

com.liferay.dynamic.data.lists.api

DDLRecordSetServiceUtil

Old: com.liferay.portlet.dynamicdatalists.service New:
com.liferay.dynamic.data.lists.service

com.liferay.dynamic.data.lists.api

DDLRecordSetServiceWrapper

Old: com.liferay.portlet.dynamicdatalists.service New:
com.liferay.dynamic.data.lists.service

com.liferay.dynamic.data.lists.api

DDLRecordSetSoap

Old: com.liferay.portlet.dynamicdatalists.model New:
com.liferay.dynamic.data.lists.model

com.liferay.dynamic.data.lists.api

DDLRecordSetUtil

Old: com.liferay.portlet.dynamicdatalists.service.persistence New:
com.liferay.dynamic.data.lists.service.persistence

com.liferay.dynamic.data.lists.api

DDLRecordSetWrapper

Old: com.liferay.portlet.dynamicdatalists.model New:
com.liferay.dynamic.data.lists.model

com.liferay.dynamic.data.lists.api

DDLRecordSoap

Old: com.liferay.portlet.dynamicdatalists.model New:
com.liferay.dynamic.data.lists.model

com.liferay.dynamic.data.lists.api

DDLRecordUtil

Old: com.liferay.portlet.dynamicdatalists.service.persistence New:
com.liferay.dynamic.data.lists.service.persistence

com.liferay.dynamic.data.lists.api

DDLRecordVersion

Old: com.liferay.portlet.dynamicdatalists.model New:
com.liferay.dynamic.data.lists.model

com.liferay.dynamic.data.lists.api

DDLRecordVersionModel

Old: com.liferay.portlet.dynamicdatalists.model New:
com.liferay.dynamic.data.lists.model

com.liferay.dynamic.data.lists.api

DDLRecordVersionPersistence

Old: com.liferay.portlet.dynamicdatalists.service.persistence New:
com.liferay.dynamic.data.lists.service.persistence

com.liferay.dynamic.data.lists.api

DDLRecordVersionSoap

Old: com.liferay.portlet.dynamicdatalists.model New:
com.liferay.dynamic.data.lists.model

com.liferay.dynamic.data.lists.api

DDLRecordVersionUtil

Old: com.liferay.portlet.dynamicdatalists.service.persistence New:
com.liferay.dynamic.data.lists.service.persistence

com.liferay.dynamic.data.lists.api

DDLRecordVersionVersionComparator

Old: com.liferay.portlet.dynamicdatalists.util.comparator New:
com.liferay.dynamic.data.lists.util.comparator

com.liferay.dynamic.data.lists.api

DDLRecordVersionWrapper

Old: com.liferay.portlet.dynamicdatalists.model New:
com.liferay.dynamic.data.lists.model

com.liferay.dynamic.data.lists.api

DDLRecordWrapper

Old: com.liferay.portlet.dynamicdatalists.model New:
com.liferay.dynamic.data.lists.model

com.liferay.dynamic.data.lists.api

DDM

Old: com.liferay.portlet.dynamicdatamapping.util New:
com.liferay.dynamic.data.mapping.util

com.liferay.dynamic.data.mapping.api

DDMContent

Old: com.liferay.portlet.dynamicdatamapping.model New:
com.liferay.dynamic.data.mapping.model

com.liferay.dynamic.data.mapping.api

DDMContentLocalService

Old: com.liferay.portlet.dynamicdatamapping.service New:
com.liferay.dynamic.data.mapping.service

com.liferay.dynamic.data.mapping.api

DDMContentLocalServiceUtil

Old: com.liferay.portlet.dynamicdatamapping.service New:
com.liferay.dynamic.data.mapping.service

com.liferay.dynamic.data.mapping.api

DDMContentLocalServiceWrapper

Old: com.liferay.portlet.dynamicdatamapping.service New:
com.liferay.dynamic.data.mapping.service

com.liferay.dynamic.data.mapping.api

DDMContentModel

Old: com.liferay.portlet.dynamicdatamapping.model New:
com.liferay.dynamic.data.mapping.model

com.liferay.dynamic.data.mapping.api

DDMContentPersistence

Old: com.liferay.portlet.dynamicdatamapping.service.persistence New:
com.liferay.dynamic.data.mapping.service.persistence

com.liferay.dynamic.data.mapping.api

DDMContentSoap

Old: com.liferay.portlet.dynamicdatamapping.model New:
com.liferay.dynamic.data.mapping.model

com.liferay.dynamic.data.mapping.api

DDMContentUtil

Old: com.liferay.portlet.dynamicdatamapping.service.persistence New:
com.liferay.dynamic.data.mapping.service.persistence

com.liferay.dynamic.data.mapping.api

DDMContentWrapper

Old: com.liferay.portlet.dynamicdatamapping.model New:
com.liferay.dynamic.data.mapping.model

com.liferay.dynamic.data.mapping.api

DDMDisplay

Old: com.liferay.portlet.dynamicdatamapping.util New:
com.liferay.dynamic.data.mapping.util

com.liferay.dynamic.data.mapping.api

DDMDisplayRegistry

Old: com.liferay.portlet.dynamicdatamapping.util New:
com.liferay.dynamic.data.mapping.util

com.liferay.dynamic.data.mapping.api

DDMIndexer

Old: com.liferay.portlet.dynamicdatamapping.util New:
com.liferay.dynamic.data.mapping.util

com.liferay.dynamic.data.mapping.api

DDMStorageLink

Old: com.liferay.portlet.dynamicdatamapping.model New:
com.liferay.dynamic.data.mapping.model

com.liferay.dynamic.data.mapping.api

DDMStorageLinkLocalService

Old: com.liferay.portlet.dynamicdatamapping.service New:
com.liferay.dynamic.data.mapping.service

com.liferay.dynamic.data.mapping.api

DDMStorageLinkLocalServiceUtil

Old: com.liferay.portlet.dynamicdatamapping.service New:
com.liferay.dynamic.data.mapping.service

com.liferay.dynamic.data.mapping.api

DDMStorageLinkLocalServiceWrapper

Old: com.liferay.portlet.dynamicdatamapping.service New:
com.liferay.dynamic.data.mapping.service

com.liferay.dynamic.data.mapping.api

DDMStorageLinkModel

Old: com.liferay.portlet.dynamicdatamapping.model New:
com.liferay.dynamic.data.mapping.model

com.liferay.dynamic.data.mapping.api

DDMStorageLinkPersistence

Old: com.liferay.portlet.dynamicdatamapping.service.persistence New:
com.liferay.dynamic.data.mapping.service.persistence

com.liferay.dynamic.data.mapping.api

DDMStorageLinkSoap

Old: com.liferay.portlet.dynamicdatamapping.model New:
com.liferay.dynamic.data.mapping.model

com.liferay.dynamic.data.mapping.api

DDMStorageLinkUtil

Old: com.liferay.portlet.dynamicdatamapping.service.persistence New:
com.liferay.dynamic.data.mapping.service.persistence

com.liferay.dynamic.data.mapping.api

DDMStorageLinkWrapper

Old: com.liferay.portlet.dynamicdatamapping.model New:
com.liferay.dynamic.data.mapping.model

com.liferay.dynamic.data.mapping.api

DDMStructureConstants

Old: com.liferay.portlet.dynamicdatamapping.model New:
com.liferay.dynamic.data.mapping.model

com.liferay.dynamic.data.mapping.api

DDMStructureFinder

Old: com.liferay.portlet.dynamicdatamapping.service.persistence New:
com.liferay.dynamic.data.mapping.service.persistence

com.liferay.dynamic.data.mapping.api

DDMStructureLinkLocalService

Old: com.liferay.portlet.dynamicdatamapping.service New:
com.liferay.dynamic.data.mapping.service

com.liferay.dynamic.data.mapping.api

DDMStructureLinkLocalServiceUtil

Old: com.liferay.portlet.dynamicdatamapping.service New:
com.liferay.dynamic.data.mapping.service

com.liferay.dynamic.data.mapping.api

DDMStructureLinkLocalServiceWrapper

Old: com.liferay.portlet.dynamicdatamapping.service New:
com.liferay.dynamic.data.mapping.service

com.liferay.dynamic.data.mapping.api

DDMStructureLinkModel

Old: com.liferay.portlet.dynamicdatamapping.model New:
com.liferay.dynamic.data.mapping.model

com.liferay.dynamic.data.mapping.api

DDMStructureLinkPersistence

Old: com.liferay.portlet.dynamicdatamapping.service.persistence New:
com.liferay.dynamic.data.mapping.service.persistence

com.liferay.dynamic.data.mapping.api

DDMStructureLinkSoap

Old: com.liferay.portlet.dynamicdatamapping.model New:
com.liferay.dynamic.data.mapping.model

com.liferay.dynamic.data.mapping.api

DDMStructureLinkUtil

Old: com.liferay.portlet.dynamicdatamapping.service.persistence New:
com.liferay.dynamic.data.mapping.service.persistence

com.liferay.dynamic.data.mapping.api

DDMStructureLinkWrapper

Old: com.liferay.portlet.dynamicdatamapping.model New:
com.liferay.dynamic.data.mapping.model

com.liferay.dynamic.data.mapping.api

DDMStructureLocalService

Old: com.liferay.portlet.dynamicdatamapping.service New:
com.liferay.dynamic.data.mapping.service

com.liferay.dynamic.data.mapping.api

DDMStructureLocalServiceUtil

Old: com.liferay.portlet.dynamicdatamapping.service New:
com.liferay.dynamic.data.mapping.service

com.liferay.dynamic.data.mapping.api

DDMStructureLocalServiceWrapper

Old: com.liferay.portlet.dynamicdatamapping.service New:
com.liferay.dynamic.data.mapping.service

com.liferay.dynamic.data.mapping.api

DDMStructureModel

Old: com.liferay.portlet.dynamicdatamapping.model New:
com.liferay.dynamic.data.mapping.model

com.liferay.dynamic.data.mapping.api

DDMStructurePersistence

Old: com.liferay.portlet.dynamicdatamapping.service.persistence New:
com.liferay.dynamic.data.mapping.service.persistence

com.liferay.dynamic.data.mapping.api

DDMStructureService

Old: com.liferay.portlet.dynamicdatamapping.service New:
com.liferay.dynamic.data.mapping.service

com.liferay.dynamic.data.mapping.api

DDMStructureServiceUtil

Old: com.liferay.portlet.dynamicdatamapping.service New:
com.liferay.dynamic.data.mapping.service

com.liferay.dynamic.data.mapping.api

DDMStructureServiceWrapper

Old: com.liferay.portlet.dynamicdatamapping.service New:
com.liferay.dynamic.data.mapping.service

com.liferay.dynamic.data.mapping.api

DDMStructureSoap

Old: com.liferay.portlet.dynamicdatamapping.model New:
com.liferay.dynamic.data.mapping.model

com.liferay.dynamic.data.mapping.api

DDMStructureUtil

Old: com.liferay.portlet.dynamicdatamapping.service.persistence New:
com.liferay.dynamic.data.mapping.service.persistence

com.liferay.dynamic.data.mapping.api

DDMStructureWrapper

Old: com.liferay.portlet.dynamicdatamapping.model New:
com.liferay.dynamic.data.mapping.model

com.liferay.dynamic.data.mapping.api

DDMTemplateConstants

Old: com.liferay.portlet.dynamicdatamapping.model New:
com.liferay.dynamic.data.mapping.model

com.liferay.dynamic.data.mapping.api

DDMTemplateFinder

Old: com.liferay.portlet.dynamicdatamapping.service.persistence New:
com.liferay.dynamic.data.mapping.service.persistence

com.liferay.dynamic.data.mapping.api

DDMTemplateHelper

Old: com.liferay.portlet.dynamicdatamapping.util New:
com.liferay.dynamic.data.mapping.util

com.liferay.dynamic.data.mapping.api

DDMTemplateLocalService

Old: com.liferay.portlet.dynamicdatamapping.service New:
com.liferay.dynamic.data.mapping.service

com.liferay.dynamic.data.mapping.api

DDMTemplateLocalServiceUtil

Old: com.liferay.portlet.dynamicdatamapping.service New:
com.liferay.dynamic.data.mapping.service

com.liferay.dynamic.data.mapping.api

DDMTemplateLocalServiceWrapper

Old: com.liferay.portlet.dynamicdatamapping.service New:
com.liferay.dynamic.data.mapping.service

com.liferay.dynamic.data.mapping.api

DDMTemplateModel

Old: com.liferay.portlet.dynamicdatamapping.model New:
com.liferay.dynamic.data.mapping.model

com.liferay.dynamic.data.mapping.api

DDMTemplatePersistence

Old: com.liferay.portlet.dynamicdatamapping.service.persistence New:
com.liferay.dynamic.data.mapping.service.persistence

com.liferay.dynamic.data.mapping.api

DDMTemplateService

Old: com.liferay.portlet.dynamicdatamapping.service New:
com.liferay.dynamic.data.mapping.service

com.liferay.dynamic.data.mapping.api

DDMTemplateServiceUtil

Old: com.liferay.portlet.dynamicdatamapping.service New:
com.liferay.dynamic.data.mapping.service

com.liferay.dynamic.data.mapping.api

DDMTemplateServiceWrapper

Old: com.liferay.portlet.dynamicdatamapping.service New:
com.liferay.dynamic.data.mapping.service

com.liferay.dynamic.data.mapping.api

DDMTemplateSoap

Old: com.liferay.portlet.dynamicdatamapping.model New:
com.liferay.dynamic.data.mapping.model

com.liferay.dynamic.data.mapping.api

DDMTemplateUtil

Old: com.liferay.portlet.dynamicdatamapping.service.persistence New:
com.liferay.dynamic.data.mapping.service.persistence

com.liferay.dynamic.data.mapping.api

DDMTemplateWrapper

Old: com.liferay.portlet.dynamicdatamapping.model New:
com.liferay.dynamic.data.mapping.model

com.liferay.dynamic.data.mapping.api

DDMUtil

Old: com.liferay.portlet.dynamicdatamapping.util New:
com.liferay.dynamic.data.mapping.util

com.liferay.dynamic.data.mapping.api

DDMXML

Old: com.liferay.portlet.dynamicdatamapping.util New:
com.liferay.dynamic.data.mapping.util

com.liferay.dynamic.data.mapping.api

DLContent

Old: com.liferay.portlet.documentlibrary.model New:
com.liferay.document.library.content.model

com.liferay.document.library.content.api

DLContentDataBlobModel

Old: com.liferay.portlet.documentlibrary.model New:
com.liferay.document.library.content.model

com.liferay.document.library.content.api

DLContentLocalService

Old: com.liferay.portlet.documentlibrary.service New:
com.liferay.document.library.content.service

com.liferay.document.library.content.api

DLContentLocalServiceUtil

Old: com.liferay.portlet.documentlibrary.service New:
com.liferay.document.library.content.service

com.liferay.document.library.content.api

DLContentLocalServiceWrapper

Old: com.liferay.portlet.documentlibrary.service New:
com.liferay.document.library.content.service

com.liferay.document.library.content.api

DLContentModel

Old: com.liferay.portlet.documentlibrary.model New:
com.liferay.document.library.content.model

com.liferay.document.library.content.api

DLContentPersistence

Old: com.liferay.portlet.documentlibrary.service.persistence New:
com.liferay.document.library.content.service.persistence

com.liferay.document.library.content.api

DLContentSoap

Old: com.liferay.portlet.documentlibrary.model New:
com.liferay.document.library.content.model

com.liferay.document.library.content.api

DLContentUtil

Old: com.liferay.portlet.documentlibrary.service.persistence New:
com.liferay.document.library.content.service.persistence

com.liferay.document.library.content.api

DLContentVersionComparator

Old: com.liferay.portlet.documentlibrary.util.comparator New:
com.liferay.document.library.content.service.util.comparator

com.liferay.document.library.content.service

DLContentWrapper

Old: com.liferay.portlet.documentlibrary.model New:
com.liferay.document.library.content.model

com.liferay.document.library.content.api

DLFileRank

Old: com.liferay.portlet.documentlibrary.model New:
com.liferay.document.library.file.rank.model

com.liferay.document.library.file.rank.api

DLFileRankFinder

Old: com.liferay.portlet.documentlibrary.service.persistence New:
com.liferay.document.library.file.rank.service.persistence

com.liferay.document.library.file.rank.api

DLFileRankLocalService

Old: com.liferay.portlet.documentlibrary.service New:
com.liferay.document.library.file.rank.service

com.liferay.document.library.file.rank.api

DLFileRankLocalServiceUtil

Old: com.liferay.portlet.documentlibrary.service New:
com.liferay.document.library.file.rank.service

com.liferay.document.library.file.rank.api

DLFileRankLocalServiceWrapper

Old: com.liferay.portlet.documentlibrary.service New:
com.liferay.document.library.file.rank.service

com.liferay.document.library.file.rank.api

DLFileRankModel

Old: com.liferay.portlet.documentlibrary.model New:
com.liferay.document.library.file.rank.model

com.liferay.document.library.file.rank.api

DLFileRankPersistence

Old: com.liferay.portlet.documentlibrary.service.persistence New:
com.liferay.document.library.file.rank.service.persistence

com.liferay.document.library.file.rank.api

DLFileRankSoap

Old: com.liferay.portlet.documentlibrary.model New:
com.liferay.document.library.file.rank.model

com.liferay.document.library.file.rank.api

DLFileRankUtil

Old: com.liferay.portlet.documentlibrary.service.persistence New:
com.liferay.document.library.file.rank.service.persistence

com.liferay.document.library.file.rank.api

DLFileRankWrapper

Old: com.liferay.portlet.documentlibrary.model New:
com.liferay.document.library.file.rank.model

com.liferay.document.library.file.rank.api

DLSyncConstants

Old: com.liferay.portlet.documentlibrary.model New:
com.liferay.document.library.sync.constants

com.liferay.document.library.sync.api

DLSyncEvent

Old: com.liferay.portlet.documentlibrary.model New:
com.liferay.document.library.sync.model

com.liferay.document.library.sync.api

DLSyncEventLocalService

Old: com.liferay.portlet.documentlibrary.service New:
com.liferay.document.library.sync.service

com.liferay.document.library.sync.api

DLSyncEventLocalServiceUtil

Old: com.liferay.portlet.documentlibrary.service New:
com.liferay.document.library.sync.service

com.liferay.document.library.sync.api

DLSyncEventLocalServiceWrapper

Old: com.liferay.portlet.documentlibrary.service New:
com.liferay.document.library.sync.service

com.liferay.document.library.sync.api

DLSyncEventModel

Old: com.liferay.portlet.documentlibrary.model New:
com.liferay.document.library.sync.model

com.liferay.document.library.sync.api

DLSyncEventPersistence

Old: com.liferay.portlet.documentlibrary.service.persistence New:
com.liferay.document.library.sync.service.persistence

com.liferay.document.library.sync.api

DLSyncEventSoap

Old: com.liferay.portlet.documentlibrary.model New:
com.liferay.document.library.sync.model

com.liferay.document.library.sync.api

DLSyncEventUtil

Old: com.liferay.portlet.documentlibrary.service.persistence New:
com.liferay.document.library.sync.service.persistence

com.liferay.document.library.sync.api

DLSyncEventWrapper

Old: com.liferay.portlet.documentlibrary.model New:
com.liferay.document.library.sync.model

com.liferay.document.library.sync.api

Database

Old: com.liferay.portal.kernel.util New:
com.liferay.portal.tools.db.upgrade.client

com.liferay.portal.tools.db.upgrade.client

DefaultAttributesTransformer

Old: com.liferay.portal.security.ldap New:
com.liferay.portal.security.ldap.internal

com.liferay.portal.security.ldap.impl

DefaultMessageBus

Old: com.liferay.portal.kernel.messaging New:
com.liferay.portal.messaging.internal

com.liferay.portal.messaging

DefaultSingleDestinationMessageSender

Old: com.liferay.portal.kernel.messaging.sender New:
com.liferay.portal.messaging.internal.sender

com.liferay.portal.messaging

DefaultSingleDestinationSynchronousMessageSender

Old: com.liferay.portal.kernel.messaging.sender New:
com.liferay.portal.messaging.internal.sender

com.liferay.portal.messaging

DefaultSynchronousMessageSender

Old: com.liferay.portal.kernel.messaging.sender New:
com.liferay.portal.messaging.internal.sender

com.liferay.portal.messaging

DeleteFileFinalizeAction

Old: com.liferay.portal.kernel.memory New: com.liferay.petra.memory

com.liferay.petra.memory

DestinationStatisticsManager

Old: com.liferay.portal.kernel.messaging.jmx New:
com.liferay.portal.messaging.internal.jmx

com.liferay.portal.messaging

DestinationStatisticsManagerMBean

Old: com.liferay.portal.kernel.messaging.jmx New:
com.liferay.portal.messaging.internal.jmx

com.liferay.portal.messaging

DirectSynchronousMessageSender

Old: com.liferay.portal.kernel.messaging.sender New:
com.liferay.portal.messaging.internal.sender

com.liferay.portal.messaging

DummyContext

Old: com.liferay.portal.kernel.ldap New:
com.liferay.portal.security.ldap.dummy

com.liferay.portal.security.ldap.api

DummyDirContext

Old: com.liferay.portal.kernel.ldap New:
com.liferay.portal.security.ldap.dummy

com.liferay.portal.security.ldap.api

DummyFinalizeAction

Old: com.liferay.portal.kernel.memory New: com.liferay.petra.memory

com.liferay.petra.memory

DuplicateArticleIdException

Old: com.liferay.portlet.journal New: com.liferay.journal.exception

com.liferay.journal.api

DuplicateFeedIdException

Old: com.liferay.portlet.journal New: com.liferay.journal.exception

com.liferay.journal.api

DuplicateLDAPServerNameException

Old: com.liferay.portal.kernel.ldap New:
com.liferay.portal.security.ldap

com.liferay.portal.security.ldap.api

DuplicateNodeNameException

Old: com.liferay.portlet.wiki New: com.liferay.wiki.exception

com.liferay.wiki.api

DuplicatePageException

Old: com.liferay.portlet.wiki New: com.liferay.wiki.exception

com.liferay.wiki.api

DuplicateRuleGroupInstanceException

Old: com.liferay.portlet.mobiledevicerules New:
com.liferay.mobile.device.rules.exception

com.liferay.mobile.device.rules.api

DuplicateVoteException

Old: com.liferay.portlet.polls New: com.liferay.polls.exception

com.liferay.polls.api

EntryDisplayDateComparator

Old: com.liferay.portlet.blogs.util.comparator New:
com.liferay.blogs.util.comparator

com.liferay.blogs.api

EntryModifiedDateComparator

Old: com.liferay.portlet.bookmarks.util.comparator New:
com.liferay.bookmarks.util.comparator

com.liferay.bookmarks.api

EntryNameComparator

Old: com.liferay.portlet.bookmarks.util.comparator New:
com.liferay.bookmarks.util.comparator

com.liferay.bookmarks.api

EntryPriorityComparator

Old: com.liferay.portlet.bookmarks.util.comparator New:
com.liferay.bookmarks.util.comparator

com.liferay.bookmarks.api

EntrySmallImageNameException

Old: com.liferay.portlet.blogs New: com.liferay.blogs.exception

com.liferay.blogs.api

EntryURLComparator

Old: com.liferay.portlet.bookmarks.util.comparator New:
com.liferay.bookmarks.util.comparator

com.liferay.bookmarks.api

EntryVisitsComparator

Old: com.liferay.portlet.bookmarks.util.comparator New:
com.liferay.bookmarks.util.comparator

com.liferay.bookmarks.api

EqualityWeakReference

Old: com.liferay.portal.kernel.memory New: com.liferay.petra.memory

com.liferay.petra.memory

Fact

Old: com.liferay.portal.kernel.bi.rules New:
com.liferay.portal.rules.engine

com.liferay.portal.rules.engine.api

FeedContentFieldException

Old: com.liferay.portlet.journal New: com.liferay.journal.exception

com.liferay.journal.api

FeedIdException

Old: com.liferay.portlet.journal New: com.liferay.journal.exception

com.liferay.journal.api

FeedNameException

Old: com.liferay.portlet.journal New: com.liferay.journal.exception

com.liferay.journal.api

FeedTargetLayoutFriendlyUrlException

Old: com.liferay.portlet.journal New: com.liferay.journal.exception

com.liferay.journal.api

FeedTargetPortletIdException

Old: com.liferay.portlet.journal New: com.liferay.journal.exception

com.liferay.journal.api

FieldConstants

Old: com.liferay.portlet.dynamicdatamapping.storage New:
com.liferay.dynamic.data.mapping.storage

com.liferay.dynamic.data.mapping.api

FieldRenderer

Old: com.liferay.portlet.dynamicdatamapping.storage New:
com.liferay.dynamic.data.mapping.storage

com.liferay.dynamic.data.mapping.api

FieldRendererFactory

Old: com.liferay.portlet.dynamicdatamapping.storage New:
com.liferay.dynamic.data.mapping.storage

com.liferay.dynamic.data.mapping.api

Fields

Old: com.liferay.portlet.dynamicdatamapping.storage New:
com.liferay.dynamic.data.mapping.storage

com.liferay.dynamic.data.mapping.api

FinalizeAction

Old: com.liferay.portal.kernel.memory New: com.liferay.petra.memory

com.liferay.petra.memory

FinalizeManager

Old: com.liferay.portal.kernel.memory New: com.liferay.petra.memory

com.liferay.petra.memory

FlagsEntryService

Old: com.liferay.portlet.flags.service New: com.liferay.flags.service

com.liferay.flags.api

FlagsEntryServiceUtil

Old: com.liferay.portlet.flags.service New: com.liferay.flags.service

com.liferay.flags.api

FlagsEntryServiceWrapper

Old: com.liferay.portlet.flags.service New: com.liferay.flags.service

com.liferay.flags.api

FlagsRequest

Old: com.liferay.portlet.flags.messaging New:
com.liferay.flags.internal.messaging

com.liferay.flags.service

GroupConverterKeys

Old: com.liferay.portal.security.ldap New:
com.liferay.portal.security.ldap

com.liferay.portal.security.ldap.api

ImportFilesException

Old: com.liferay.portlet.wiki New: com.liferay.wiki.exception

com.liferay.wiki.api

JournalArticle

Old: com.liferay.portlet.journal.model New: com.liferay.journal.model

com.liferay.journal.api

JournalArticleConstants

Old: com.liferay.portlet.journal.model New: com.liferay.journal.model

com.liferay.journal.api

JournalArticleDisplay

Old: com.liferay.portlet.journal.model New: com.liferay.journal.model

com.liferay.journal.api

JournalArticleFinder

Old: com.liferay.portlet.journal.service.persistence New:
com.liferay.journal.service.persistence

com.liferay.journal.api

JournalArticleLocalService

Old: com.liferay.portlet.journal.service New:
com.liferay.journal.service

com.liferay.journal.api

JournalArticleLocalServiceUtil

Old: com.liferay.portlet.journal.service New:
com.liferay.journal.service

com.liferay.journal.api

JournalArticleLocalServiceWrapper

Old: com.liferay.portlet.journal.service New:
com.liferay.journal.service

com.liferay.journal.api

JournalArticleModel

Old: com.liferay.portlet.journal.model New: com.liferay.journal.model

com.liferay.journal.api

JournalArticlePersistence

Old: com.liferay.portlet.journal.service.persistence New:
com.liferay.journal.service.persistence

com.liferay.journal.api

JournalArticleResource

Old: com.liferay.portlet.journal.model New: com.liferay.journal.model

com.liferay.journal.api

JournalArticleResourceLocalService

Old: com.liferay.portlet.journal.service New:
com.liferay.journal.service

com.liferay.journal.api

JournalArticleResourceLocalServiceUtil

Old: com.liferay.portlet.journal.service New:
com.liferay.journal.service

com.liferay.journal.api

JournalArticleResourceLocalServiceWrapper

Old: com.liferay.portlet.journal.service New:
com.liferay.journal.service

com.liferay.journal.api

JournalArticleResourceModel

Old: com.liferay.portlet.journal.model New: com.liferay.journal.model

com.liferay.journal.api

JournalArticleResourcePersistence

Old: com.liferay.portlet.journal.service.persistence New:
com.liferay.journal.service.persistence

com.liferay.journal.api

JournalArticleResourceSoap

Old: com.liferay.portlet.journal.model New: com.liferay.journal.model

com.liferay.journal.api

JournalArticleResourceUtil

Old: com.liferay.portlet.journal.service.persistence New:
com.liferay.journal.service.persistence

com.liferay.journal.api

JournalArticleResourceWrapper

Old: com.liferay.portlet.journal.model New: com.liferay.journal.model

com.liferay.journal.api

JournalArticleService

Old: com.liferay.portlet.journal.service New:
com.liferay.journal.service

com.liferay.journal.api

JournalArticleServiceUtil

Old: com.liferay.portlet.journal.service New:
com.liferay.journal.service

com.liferay.journal.api

JournalArticleServiceWrapper

Old: com.liferay.portlet.journal.service New:
com.liferay.journal.service

com.liferay.journal.api

JournalArticleSoap

Old: com.liferay.portlet.journal.model New: com.liferay.journal.model

com.liferay.journal.api

JournalArticleUtil

Old: com.liferay.portlet.journal.service.persistence New:
com.liferay.journal.service.persistence

com.liferay.journal.api

JournalArticleWrapper

Old: com.liferay.portlet.journal.model New: com.liferay.journal.model

com.liferay.journal.api

JournalContent

Old: com.liferay.portlet.journalcontent.util New:
com.liferay.journal.util

com.liferay.journal.api

JournalContentSearch

Old: com.liferay.portlet.journal.model New: com.liferay.journal.model

com.liferay.journal.api

JournalContentSearchLocalService

Old: com.liferay.portlet.journal.service New:
com.liferay.journal.service

com.liferay.journal.api

JournalContentSearchLocalServiceUtil

Old: com.liferay.portlet.journal.service New:
com.liferay.journal.service

com.liferay.journal.api

JournalContentSearchLocalServiceWrapper

Old: com.liferay.portlet.journal.service New:
com.liferay.journal.service

com.liferay.journal.api

JournalContentSearchModel

Old: com.liferay.portlet.journal.model New: com.liferay.journal.model

com.liferay.journal.api

JournalContentSearchPersistence

Old: com.liferay.portlet.journal.service.persistence New:
com.liferay.journal.service.persistence

com.liferay.journal.api

JournalContentSearchSoap

Old: com.liferay.portlet.journal.model New: com.liferay.journal.model

com.liferay.journal.api

JournalContentSearchUtil

Old: com.liferay.portlet.journal.service.persistence New:
com.liferay.journal.service.persistence

com.liferay.journal.api

JournalContentSearchWrapper

Old: com.liferay.portlet.journal.model New: com.liferay.journal.model

com.liferay.journal.api

JournalConverter

Old: com.liferay.portlet.journal.util New: com.liferay.journal.util

com.liferay.journal.api

JournalFeed

Old: com.liferay.portlet.journal.model New: com.liferay.journal.model

com.liferay.journal.api

JournalFeedConstants

Old: com.liferay.portlet.journal.model New: com.liferay.journal.model

com.liferay.journal.api

JournalFeedFinder

Old: com.liferay.portlet.journal.service.persistence New:
com.liferay.journal.service.persistence

com.liferay.journal.api

JournalFeedLocalService

Old: com.liferay.portlet.journal.service New:
com.liferay.journal.service

com.liferay.journal.api

JournalFeedLocalServiceUtil

Old: com.liferay.portlet.journal.service New:
com.liferay.journal.service

com.liferay.journal.api

JournalFeedLocalServiceWrapper

Old: com.liferay.portlet.journal.service New:
com.liferay.journal.service

com.liferay.journal.api

JournalFeedModel

Old: com.liferay.portlet.journal.model New: com.liferay.journal.model

com.liferay.journal.api

JournalFeedPersistence

Old: com.liferay.portlet.journal.service.persistence New:
com.liferay.journal.service.persistence

com.liferay.journal.api

JournalFeedService

Old: com.liferay.portlet.journal.service New:
com.liferay.journal.service

com.liferay.journal.api

JournalFeedServiceUtil

Old: com.liferay.portlet.journal.service New:
com.liferay.journal.service

com.liferay.journal.api

JournalFeedServiceWrapper

Old: com.liferay.portlet.journal.service New:
com.liferay.journal.service

com.liferay.journal.api

JournalFeedSoap

Old: com.liferay.portlet.journal.model New: com.liferay.journal.model

com.liferay.journal.api

JournalFeedUtil

Old: com.liferay.portlet.journal.service.persistence New:
com.liferay.journal.service.persistence

com.liferay.journal.api

JournalFeedWrapper

Old: com.liferay.portlet.journal.model New: com.liferay.journal.model

com.liferay.journal.api

JournalFolder

Old: com.liferay.portlet.journal.model New: com.liferay.journal.model

com.liferay.journal.api

JournalFolderFinder

Old: com.liferay.portlet.journal.service.persistence New:
com.liferay.journal.service.persistence

com.liferay.journal.api

JournalFolderLocalService

Old: com.liferay.portlet.journal.service New:
com.liferay.journal.service

com.liferay.journal.api

JournalFolderLocalServiceUtil

Old: com.liferay.portlet.journal.service New:
com.liferay.journal.service

com.liferay.journal.api

JournalFolderLocalServiceWrapper

Old: com.liferay.portlet.journal.service New:
com.liferay.journal.service

com.liferay.journal.api

JournalFolderModel

Old: com.liferay.portlet.journal.model New: com.liferay.journal.model

com.liferay.journal.api

JournalFolderPersistence

Old: com.liferay.portlet.journal.service.persistence New:
com.liferay.journal.service.persistence

com.liferay.journal.api

JournalFolderService

Old: com.liferay.portlet.journal.service New:
com.liferay.journal.service

com.liferay.journal.api

JournalFolderServiceUtil

Old: com.liferay.portlet.journal.service New:
com.liferay.journal.service

com.liferay.journal.api

JournalFolderServiceWrapper

Old: com.liferay.portlet.journal.service New:
com.liferay.journal.service

com.liferay.journal.api

JournalFolderSoap

Old: com.liferay.portlet.journal.model New: com.liferay.journal.model

com.liferay.journal.api

JournalFolderUtil

Old: com.liferay.portlet.journal.service.persistence New:
com.liferay.journal.service.persistence

com.liferay.journal.api

JournalFolderWrapper

Old: com.liferay.portlet.journal.model New: com.liferay.journal.model

com.liferay.journal.api

JournalSearchConstants

Old: com.liferay.portlet.journal.model New: com.liferay.journal.model

com.liferay.journal.api

JournalStructureConstants

Old: com.liferay.portlet.journal.model New: com.liferay.journal.model

com.liferay.journal.api

LDAPFilterException

Old: com.liferay.portal.kernel.ldap New:
com.liferay.portal.security.ldap.validator

com.liferay.portal.security.ldap.api

LDAPGroup

Old: com.liferay.portal.security.ldap New:
com.liferay.portal.security.ldap.exportimport

com.liferay.portal.security.ldap.api

LDAPServerNameException

Old: com.liferay.portal.kernel.ldap New:
com.liferay.portal.security.ldap

com.liferay.portal.security.ldap.api

LDAPToPortalConverter

Old: com.liferay.portal.security.ldap New:
com.liferay.portal.security.ldap.exportimport

com.liferay.portal.security.ldap.api

LDAPUser

Old: com.liferay.portal.security.ldap New:
com.liferay.portal.security.ldap.exportimport

com.liferay.portal.security.ldap.api

LDAPUtil

Old: com.liferay.portal.kernel.ldap New:
com.liferay.portal.security.ldap.util

com.liferay.portal.security.ldap.api

LockLocalService

Old: com.liferay.portal.service New: com.liferay.portal.lock.service

com.liferay.portal.lock.api

LockLocalServiceUtil

Old: com.liferay.portal.service New: com.liferay.portal.lock.service

com.liferay.portal.lock.api

LockLocalServiceWrapper

Old: com.liferay.portal.service New: com.liferay.portal.lock.service

com.liferay.portal.lock.api

LockModel

Old: com.liferay.portal.model New: com.liferay.portal.lock.model

com.liferay.portal.lock.api

LockPersistence

Old: com.liferay.portal.service.persistence New:
com.liferay.portal.lock.service.persistence

com.liferay.portal.lock.api

LockSoap

Old: com.liferay.portal.model New: com.liferay.portal.lock.model

com.liferay.portal.lock.api

LockUtil

Old: com.liferay.portal.service.persistence New:
com.liferay.portal.lock.service.persistence

com.liferay.portal.lock.api

LockWrapper

Old: com.liferay.portal.model New: com.liferay.portal.lock.model

com.liferay.portal.lock.api

LockedThreadException

Old: com.liferay.portlet.messageboards New:
com.liferay.message.boards.exception

com.liferay.message.boards.api

MBBan

Old: com.liferay.portlet.messageboards.model New:
com.liferay.message.boards.model

com.liferay.message.boards.api

MBBanLocalService

Old: com.liferay.portlet.messageboards.service New:
com.liferay.message.boards.service

com.liferay.message.boards.api

MBBanLocalServiceUtil

Old: com.liferay.portlet.messageboards.service New:
com.liferay.message.boards.service

com.liferay.message.boards.api

MBBanLocalServiceWrapper

Old: com.liferay.portlet.messageboards.service New:
com.liferay.message.boards.service

com.liferay.message.boards.api

MBBanModel

Old: com.liferay.portlet.messageboards.model New:
com.liferay.message.boards.model

com.liferay.message.boards.api

MBBanPersistence

Old: com.liferay.portlet.messageboards.service.persistence New:
com.liferay.message.boards.service.persistence

com.liferay.message.boards.api

MBBanService

Old: com.liferay.portlet.messageboards.service New:
com.liferay.message.boards.service

com.liferay.message.boards.api

MBBanServiceUtil

Old: com.liferay.portlet.messageboards.service New:
com.liferay.message.boards.service

com.liferay.message.boards.api

MBBanServiceWrapper

Old: com.liferay.portlet.messageboards.service New:
com.liferay.message.boards.service

com.liferay.message.boards.api

MBBanSoap

Old: com.liferay.portlet.messageboards.model New:
com.liferay.message.boards.model

com.liferay.message.boards.api

MBBanUtil

Old: com.liferay.portlet.messageboards.service.persistence New:
com.liferay.message.boards.service.persistence

com.liferay.message.boards.api

MBBanWrapper

Old: com.liferay.portlet.messageboards.model New:
com.liferay.message.boards.model

com.liferay.message.boards.api

MBCategory

Old: com.liferay.portlet.messageboards.model New:
com.liferay.message.boards.model

com.liferay.message.boards.api

MBCategoryConstants

Old: com.liferay.portlet.messageboards.model New:
com.liferay.message.boards.constants

com.liferay.message.boards.api

MBCategoryDisplay

Old: com.liferay.portlet.messageboards.model New:
com.liferay.message.boards.web.internal.display

com.liferay.message.boards.web

MBCategoryFinder

Old: com.liferay.portlet.messageboards.service.persistence New:
com.liferay.message.boards.service.persistence

com.liferay.message.boards.api

MBCategoryLocalService

Old: com.liferay.portlet.messageboards.service New:
com.liferay.message.boards.service

com.liferay.message.boards.api

MBCategoryLocalServiceUtil

Old: com.liferay.portlet.messageboards.service New:
com.liferay.message.boards.service

com.liferay.message.boards.api

MBCategoryLocalServiceWrapper

Old: com.liferay.portlet.messageboards.service New:
com.liferay.message.boards.service

com.liferay.message.boards.api

MBCategoryModel

Old: com.liferay.portlet.messageboards.model New:
com.liferay.message.boards.model

com.liferay.message.boards.api

MBCategoryPersistence

Old: com.liferay.portlet.messageboards.service.persistence New:
com.liferay.message.boards.service.persistence

com.liferay.message.boards.api

MBCategoryService

Old: com.liferay.portlet.messageboards.service New:
com.liferay.message.boards.service

com.liferay.message.boards.api

MBCategoryServiceUtil

Old: com.liferay.portlet.messageboards.service New:
com.liferay.message.boards.service

com.liferay.message.boards.api

MBCategoryServiceWrapper

Old: com.liferay.portlet.messageboards.service New:
com.liferay.message.boards.service

com.liferay.message.boards.api

MBCategorySoap

Old: com.liferay.portlet.messageboards.model New:
com.liferay.message.boards.model

com.liferay.message.boards.api

MBCategoryUtil

Old: com.liferay.portlet.messageboards.service.persistence New:
com.liferay.message.boards.service.persistence

com.liferay.message.boards.api

MBCategoryWrapper

Old: com.liferay.portlet.messageboards.model New:
com.liferay.message.boards.model

com.liferay.message.boards.api

MBDiscussion

Old: com.liferay.portlet.messageboards.model New:
com.liferay.message.boards.model

com.liferay.message.boards.api

MBDiscussionLocalService

Old: com.liferay.portlet.messageboards.service New:
com.liferay.message.boards.service

com.liferay.message.boards.api

MBDiscussionLocalServiceUtil

Old: com.liferay.portlet.messageboards.service New:
com.liferay.message.boards.service

com.liferay.message.boards.api

MBDiscussionLocalServiceWrapper

Old: com.liferay.portlet.messageboards.service New:
com.liferay.message.boards.service

com.liferay.message.boards.api

MBDiscussionModel

Old: com.liferay.portlet.messageboards.model New:
com.liferay.message.boards.model

com.liferay.message.boards.api

MBDiscussionPersistence

Old: com.liferay.portlet.messageboards.service.persistence New:
com.liferay.message.boards.service.persistence

com.liferay.message.boards.api

MBDiscussionSoap

Old: com.liferay.portlet.messageboards.model New:
com.liferay.message.boards.model

com.liferay.message.boards.api

MBDiscussionUtil

Old: com.liferay.portlet.messageboards.service.persistence New:
com.liferay.message.boards.service.persistence

com.liferay.message.boards.api

MBDiscussionWrapper

Old: com.liferay.portlet.messageboards.model New:
com.liferay.message.boards.model

com.liferay.message.boards.api

MBMailingList

Old: com.liferay.portlet.messageboards.model New:
com.liferay.message.boards.model

com.liferay.message.boards.api

MBMailingListLocalService

Old: com.liferay.portlet.messageboards.service New:
com.liferay.message.boards.service

com.liferay.message.boards.api

MBMailingListLocalServiceUtil

Old: com.liferay.portlet.messageboards.service New:
com.liferay.message.boards.service

com.liferay.message.boards.api

MBMailingListLocalServiceWrapper

Old: com.liferay.portlet.messageboards.service New:
com.liferay.message.boards.service

com.liferay.message.boards.api

MBMailingListModel

Old: com.liferay.portlet.messageboards.model New:
com.liferay.message.boards.model

com.liferay.message.boards.api

MBMailingListPersistence

Old: com.liferay.portlet.messageboards.service.persistence New:
com.liferay.message.boards.service.persistence

com.liferay.message.boards.api

MBMailingListSoap

Old: com.liferay.portlet.messageboards.model New:
com.liferay.message.boards.model

com.liferay.message.boards.api

MBMailingListUtil

Old: com.liferay.portlet.messageboards.service.persistence New:
com.liferay.message.boards.service.persistence

com.liferay.message.boards.api

MBMailingListWrapper

Old: com.liferay.portlet.messageboards.model New:
com.liferay.message.boards.model

com.liferay.message.boards.api

MBMessage

Old: com.liferay.portlet.messageboards.model New:
com.liferay.message.boards.model

com.liferay.message.boards.api

MBMessageConstants

Old: com.liferay.portlet.messageboards.model New:
com.liferay.message.boards.constants

com.liferay.message.boards.api

MBMessageDisplay

Old: com.liferay.portlet.messageboards.model New:
com.liferay.message.boards.model

com.liferay.message.boards.api

MBMessageFinder

Old: com.liferay.portlet.messageboards.service.persistence New:
com.liferay.message.boards.service.persistence

com.liferay.message.boards.api

MBMessageLocalService

Old: com.liferay.portlet.messageboards.service New:
com.liferay.message.boards.service

com.liferay.message.boards.api

MBMessageLocalServiceUtil

Old: com.liferay.portlet.messageboards.service New:
com.liferay.message.boards.service

com.liferay.message.boards.api

MBMessageLocalServiceWrapper

Old: com.liferay.portlet.messageboards.service New:
com.liferay.message.boards.service

com.liferay.message.boards.api

MBMessageModel

Old: com.liferay.portlet.messageboards.model New:
com.liferay.message.boards.model

com.liferay.message.boards.api

MBMessagePersistence

Old: com.liferay.portlet.messageboards.service.persistence New:
com.liferay.message.boards.service.persistence

com.liferay.message.boards.api

MBMessageService

Old: com.liferay.portlet.messageboards.service New:
com.liferay.message.boards.service

com.liferay.message.boards.api

MBMessageServiceUtil

Old: com.liferay.portlet.messageboards.service New:
com.liferay.message.boards.service

com.liferay.message.boards.api

MBMessageServiceWrapper

Old: com.liferay.portlet.messageboards.service New:
com.liferay.message.boards.service

com.liferay.message.boards.api

MBMessageSoap

Old: com.liferay.portlet.messageboards.model New:
com.liferay.message.boards.model

com.liferay.message.boards.api

MBMessageUtil

Old: com.liferay.portlet.messageboards.service.persistence New:
com.liferay.message.boards.service.persistence

com.liferay.message.boards.api

MBMessageWrapper

Old: com.liferay.portlet.messageboards.model New:
com.liferay.message.boards.model

com.liferay.message.boards.api

MBStatsUser

Old: com.liferay.portlet.messageboards.model New:
com.liferay.message.boards.model

com.liferay.message.boards.api

MBStatsUserLocalService

Old: com.liferay.portlet.messageboards.service New:
com.liferay.message.boards.service

com.liferay.message.boards.api

MBStatsUserLocalServiceUtil

Old: com.liferay.portlet.messageboards.service New:
com.liferay.message.boards.service

com.liferay.message.boards.api

MBStatsUserLocalServiceWrapper

Old: com.liferay.portlet.messageboards.service New:
com.liferay.message.boards.service

com.liferay.message.boards.api

MBStatsUserModel

Old: com.liferay.portlet.messageboards.model New:
com.liferay.message.boards.model

com.liferay.message.boards.api

MBStatsUserPersistence

Old: com.liferay.portlet.messageboards.service.persistence New:
com.liferay.message.boards.service.persistence

com.liferay.message.boards.api

MBStatsUserSoap

Old: com.liferay.portlet.messageboards.model New:
com.liferay.message.boards.model

com.liferay.message.boards.api

MBStatsUserUtil

Old: com.liferay.portlet.messageboards.service.persistence New:
com.liferay.message.boards.service.persistence

com.liferay.message.boards.api

MBStatsUserWrapper

Old: com.liferay.portlet.messageboards.model New:
com.liferay.message.boards.model

com.liferay.message.boards.api

MBThread

Old: com.liferay.portlet.messageboards.model New:
com.liferay.message.boards.model

com.liferay.message.boards.api

MBThreadConstants

Old: com.liferay.portlet.messageboards.model New:
com.liferay.message.boards.constants

com.liferay.message.boards.api

MBThreadFinder

Old: com.liferay.portlet.messageboards.service.persistence New:
com.liferay.message.boards.service.persistence

com.liferay.message.boards.api

MBThreadFlag

Old: com.liferay.portlet.messageboards.model New:
com.liferay.message.boards.model

com.liferay.message.boards.api

MBThreadFlagLocalService

Old: com.liferay.portlet.messageboards.service New:
com.liferay.message.boards.service

com.liferay.message.boards.api

MBThreadFlagLocalServiceUtil

Old: com.liferay.portlet.messageboards.service New:
com.liferay.message.boards.service

com.liferay.message.boards.api

MBThreadFlagLocalServiceWrapper

Old: com.liferay.portlet.messageboards.service New:
com.liferay.message.boards.service

com.liferay.message.boards.api

MBThreadFlagModel

Old: com.liferay.portlet.messageboards.model New:
com.liferay.message.boards.model

com.liferay.message.boards.api

MBThreadFlagPersistence

Old: com.liferay.portlet.messageboards.service.persistence New:
com.liferay.message.boards.service.persistence

com.liferay.message.boards.api

MBThreadFlagSoap

Old: com.liferay.portlet.messageboards.model New:
com.liferay.message.boards.model

com.liferay.message.boards.api

MBThreadFlagUtil

Old: com.liferay.portlet.messageboards.service.persistence New:
com.liferay.message.boards.service.persistence

com.liferay.message.boards.api

MBThreadFlagWrapper

Old: com.liferay.portlet.messageboards.model New:
com.liferay.message.boards.model

com.liferay.message.boards.api

MBThreadLocalService

Old: com.liferay.portlet.messageboards.service New:
com.liferay.message.boards.service

com.liferay.message.boards.api

MBThreadLocalServiceUtil

Old: com.liferay.portlet.messageboards.service New:
com.liferay.message.boards.service

com.liferay.message.boards.api

MBThreadLocalServiceWrapper

Old: com.liferay.portlet.messageboards.service New:
com.liferay.message.boards.service

com.liferay.message.boards.api

MBThreadModel

Old: com.liferay.portlet.messageboards.model New:
com.liferay.message.boards.model

com.liferay.message.boards.api

MBThreadPersistence

Old: com.liferay.portlet.messageboards.service.persistence New:
com.liferay.message.boards.service.persistence

com.liferay.message.boards.api

MBThreadService

Old: com.liferay.portlet.messageboards.service New:
com.liferay.message.boards.service

com.liferay.message.boards.api

MBThreadServiceUtil

Old: com.liferay.portlet.messageboards.service New:
com.liferay.message.boards.service

com.liferay.message.boards.api

MBThreadServiceWrapper

Old: com.liferay.portlet.messageboards.service New:
com.liferay.message.boards.service

com.liferay.message.boards.api

MBThreadSoap

Old: com.liferay.portlet.messageboards.model New:
com.liferay.message.boards.model

com.liferay.message.boards.api

MBThreadUtil

Old: com.liferay.portlet.messageboards.service.persistence New:
com.liferay.message.boards.service.persistence

com.liferay.message.boards.api

MBThreadWrapper

Old: com.liferay.portlet.messageboards.model New:
com.liferay.message.boards.model

com.liferay.message.boards.api

MBTreeWalker

Old: com.liferay.portlet.messageboards.model New:
com.liferay.message.boards.model

com.liferay.message.boards.api

MBeanRegistry

Old: com.liferay.portal.kernel.jmx New: com.liferay.portal.jmx

com.liferay.portal.jmx.api

MDRAction

Old: com.liferay.portlet.mobiledevicerules.model New:
com.liferay.mobile.device.rules.model

com.liferay.mobile.device.rules.api

MDRActionLocalService

Old: com.liferay.portlet.mobiledevicerules.service New:
com.liferay.mobile.device.rules.service

com.liferay.mobile.device.rules.api

MDRActionLocalServiceUtil

Old: com.liferay.portlet.mobiledevicerules.service New:
com.liferay.mobile.device.rules.service

com.liferay.mobile.device.rules.api

MDRActionLocalServiceWrapper

Old: com.liferay.portlet.mobiledevicerules.service New:
com.liferay.mobile.device.rules.service

com.liferay.mobile.device.rules.api

MDRActionModel

Old: com.liferay.portlet.mobiledevicerules.model New:
com.liferay.mobile.device.rules.model

com.liferay.mobile.device.rules.api

MDRActionPersistence

Old: com.liferay.portlet.mobiledevicerules.service.persistence New:
com.liferay.mobile.device.rules.service.persistence

com.liferay.mobile.device.rules.api

MDRActionService

Old: com.liferay.portlet.mobiledevicerules.service New:
com.liferay.mobile.device.rules.service

com.liferay.mobile.device.rules.api

MDRActionServiceUtil

Old: com.liferay.portlet.mobiledevicerules.service New:
com.liferay.mobile.device.rules.service

com.liferay.mobile.device.rules.api

MDRActionServiceWrapper

Old: com.liferay.portlet.mobiledevicerules.service New:
com.liferay.mobile.device.rules.service

com.liferay.mobile.device.rules.api

MDRActionSoap

Old: com.liferay.portlet.mobiledevicerules.model New:
com.liferay.mobile.device.rules.model

com.liferay.mobile.device.rules.api

MDRActionUtil

Old: com.liferay.portlet.mobiledevicerules.service.persistence New:
com.liferay.mobile.device.rules.service.persistence

com.liferay.mobile.device.rules.api

MDRActionWrapper

Old: com.liferay.portlet.mobiledevicerules.model New:
com.liferay.mobile.device.rules.model

com.liferay.mobile.device.rules.api

MDRPermission

Old: com.liferay.portlet.mobiledevicerules.service.permission New:
com.liferay.mobile.device.rules.web.internal.security.permission.resource

com.liferay.mobile.device.rules.web

MDRRule

Old: com.liferay.portlet.mobiledevicerules.model New:
com.liferay.mobile.device.rules.model

com.liferay.mobile.device.rules.api

MDRRuleGroup

Old: com.liferay.portlet.mobiledevicerules.model New:
com.liferay.mobile.device.rules.model

com.liferay.mobile.device.rules.api

MDRRuleGroupFinder

Old: com.liferay.portlet.mobiledevicerules.service.persistence New:
com.liferay.mobile.device.rules.service.persistence

com.liferay.mobile.device.rules.api

MDRRuleGroupInstanceLocalService

Old: com.liferay.portlet.mobiledevicerules.service New:
com.liferay.mobile.device.rules.service

com.liferay.mobile.device.rules.api

MDRRuleGroupInstanceLocalServiceUtil

Old: com.liferay.portlet.mobiledevicerules.service New:
com.liferay.mobile.device.rules.service

com.liferay.mobile.device.rules.api

MDRRuleGroupInstanceLocalServiceWrapper

Old: com.liferay.portlet.mobiledevicerules.service New:
com.liferay.mobile.device.rules.service

com.liferay.mobile.device.rules.api

MDRRuleGroupInstanceModel

Old: com.liferay.portlet.mobiledevicerules.model New:
com.liferay.mobile.device.rules.model

com.liferay.mobile.device.rules.api

MDRRuleGroupInstancePermission

Old: com.liferay.portlet.mobiledevicerules.service.permission New:
com.liferay.mobile.device.rules.web.internal.security.permission.resource

com.liferay.mobile.device.rules.web

MDRRuleGroupInstancePersistence

Old: com.liferay.portlet.mobiledevicerules.service.persistence New:
com.liferay.mobile.device.rules.service.persistence

com.liferay.mobile.device.rules.api

MDRRuleGroupInstanceService

Old: com.liferay.portlet.mobiledevicerules.service New:
com.liferay.mobile.device.rules.service

com.liferay.mobile.device.rules.api

MDRRuleGroupInstanceServiceUtil

Old: com.liferay.portlet.mobiledevicerules.service New:
com.liferay.mobile.device.rules.service

com.liferay.mobile.device.rules.api

MDRRuleGroupInstanceServiceWrapper

Old: com.liferay.portlet.mobiledevicerules.service New:
com.liferay.mobile.device.rules.service

com.liferay.mobile.device.rules.api

MDRRuleGroupInstanceSoap

Old: com.liferay.portlet.mobiledevicerules.model New:
com.liferay.mobile.device.rules.model

com.liferay.mobile.device.rules.api

MDRRuleGroupInstanceUtil

Old: com.liferay.portlet.mobiledevicerules.service.persistence New:
com.liferay.mobile.device.rules.service.persistence

com.liferay.mobile.device.rules.api

MDRRuleGroupInstanceWrapper

Old: com.liferay.portlet.mobiledevicerules.model New:
com.liferay.mobile.device.rules.model

com.liferay.mobile.device.rules.api

MDRRuleGroupLocalService

Old: com.liferay.portlet.mobiledevicerules.service New:
com.liferay.mobile.device.rules.service

com.liferay.mobile.device.rules.api

MDRRuleGroupLocalServiceUtil

Old: com.liferay.portlet.mobiledevicerules.service New:
com.liferay.mobile.device.rules.service

com.liferay.mobile.device.rules.api

MDRRuleGroupLocalServiceWrapper

Old: com.liferay.portlet.mobiledevicerules.service New:
com.liferay.mobile.device.rules.service

com.liferay.mobile.device.rules.api

MDRRuleGroupModel

Old: com.liferay.portlet.mobiledevicerules.model New:
com.liferay.mobile.device.rules.model

com.liferay.mobile.device.rules.api

MDRRuleGroupPermission

Old: com.liferay.portlet.mobiledevicerules.service.permission New:
com.liferay.mobile.device.rules.web.internal.security.permission.resource

com.liferay.mobile.device.rules.web

MDRRuleGroupPersistence

Old: com.liferay.portlet.mobiledevicerules.service.persistence New:
com.liferay.mobile.device.rules.service.persistence

com.liferay.mobile.device.rules.api

MDRRuleGroupService

Old: com.liferay.portlet.mobiledevicerules.service New:
com.liferay.mobile.device.rules.service

com.liferay.mobile.device.rules.api

MDRRuleGroupServiceUtil

Old: com.liferay.portlet.mobiledevicerules.service New:
com.liferay.mobile.device.rules.service

com.liferay.mobile.device.rules.api

MDRRuleGroupServiceWrapper

Old: com.liferay.portlet.mobiledevicerules.service New:
com.liferay.mobile.device.rules.service

com.liferay.mobile.device.rules.api

MDRRuleGroupSoap

Old: com.liferay.portlet.mobiledevicerules.model New:
com.liferay.mobile.device.rules.model

com.liferay.mobile.device.rules.api

MDRRuleGroupUtil

Old: com.liferay.portlet.mobiledevicerules.service.persistence New:
com.liferay.mobile.device.rules.service.persistence

com.liferay.mobile.device.rules.api

MDRRuleGroupWrapper

Old: com.liferay.portlet.mobiledevicerules.model New:
com.liferay.mobile.device.rules.model

com.liferay.mobile.device.rules.api

MDRRuleLocalService

Old: com.liferay.portlet.mobiledevicerules.service New:
com.liferay.mobile.device.rules.service

com.liferay.mobile.device.rules.api

MDRRuleLocalServiceUtil

Old: com.liferay.portlet.mobiledevicerules.service New:
com.liferay.mobile.device.rules.service

com.liferay.mobile.device.rules.api

MDRRuleLocalServiceWrapper

Old: com.liferay.portlet.mobiledevicerules.service New:
com.liferay.mobile.device.rules.service

com.liferay.mobile.device.rules.api

MDRRuleModel

Old: com.liferay.portlet.mobiledevicerules.model New:
com.liferay.mobile.device.rules.model

com.liferay.mobile.device.rules.api

MDRRulePersistence

Old: com.liferay.portlet.mobiledevicerules.service.persistence New:
com.liferay.mobile.device.rules.service.persistence

com.liferay.mobile.device.rules.api

MDRRuleService

Old: com.liferay.portlet.mobiledevicerules.service New:
com.liferay.mobile.device.rules.service

com.liferay.mobile.device.rules.api

MDRRuleServiceUtil

Old: com.liferay.portlet.mobiledevicerules.service New:
com.liferay.mobile.device.rules.service

com.liferay.mobile.device.rules.api

MDRRuleServiceWrapper

Old: com.liferay.portlet.mobiledevicerules.service New:
com.liferay.mobile.device.rules.service

com.liferay.mobile.device.rules.api

MDRRuleSoap

Old: com.liferay.portlet.mobiledevicerules.model New:
com.liferay.mobile.device.rules.model

com.liferay.mobile.device.rules.api

MDRRuleUtil

Old: com.liferay.portlet.mobiledevicerules.service.persistence New:
com.liferay.mobile.device.rules.service.persistence

com.liferay.mobile.device.rules.api

MDRRuleWrapper

Old: com.liferay.portlet.mobiledevicerules.model New:
com.liferay.mobile.device.rules.model

com.liferay.mobile.device.rules.api

MailingListEmailAddressException

Old: com.liferay.portlet.messageboards New:
com.liferay.message.boards.exception

com.liferay.message.boards.api

MailingListInServerNameException

Old: com.liferay.portlet.messageboards New:
com.liferay.message.boards.exception

com.liferay.message.boards.api

MailingListInUserNameException

Old: com.liferay.portlet.messageboards New:
com.liferay.message.boards.exception

com.liferay.message.boards.api

MailingListOutEmailAddressException

Old: com.liferay.portlet.messageboards New:
com.liferay.message.boards.exception

com.liferay.message.boards.api

MailingListOutServerNameException

Old: com.liferay.portlet.messageboards New:
com.liferay.message.boards.exception

com.liferay.message.boards.api

MailingListOutUserNameException

Old: com.liferay.portlet.messageboards New:
com.liferay.message.boards.exception

com.liferay.message.boards.api

MemoryReportDesignRetriever

Old: com.liferay.portal.kernel.bi.reporting New:
com.liferay.portal.reports.engine

com.liferay.portal.reports.engine.api

MessageBodyException

Old: com.liferay.portlet.messageboards New:
com.liferay.message.boards.exception

com.liferay.message.boards.api

MessageBusManager

Old: com.liferay.portal.kernel.messaging.jmx New:
com.liferay.portal.messaging.internal.jmx

com.liferay.portal.messaging

MessageBusManagerMBean

Old: com.liferay.portal.kernel.messaging.jmx New:
com.liferay.portal.messaging.internal.jmx

com.liferay.portal.messaging

MessageCreateDateComparator

Old: com.liferay.portlet.messageboards.util.comparator New:
com.liferay.message.boards.util.comparator

com.liferay.message.boards.api

MessageSubjectException

Old: com.liferay.portlet.messageboards New:
com.liferay.message.boards.exception

com.liferay.message.boards.api

MessageThreadComparator

Old: com.liferay.portlet.messageboards.util.comparator New:
com.liferay.message.boards.util.comparator

com.liferay.message.boards.api

Modifications

Old: com.liferay.portal.security.ldap New:
com.liferay.portal.security.ldap.exportimport

com.liferay.portal.security.ldap.api

NoSuchArticleException

Old: com.liferay.portlet.journal New: com.liferay.journal.exception

com.liferay.journal.api

NoSuchArticleImageException

Old: com.liferay.portlet.journal New: com.liferay.journal.exception

com.liferay.journal.api

NoSuchArticleResourceException

Old: com.liferay.portlet.journal New: com.liferay.journal.exception

com.liferay.journal.api

NoSuchBanException

Old: com.liferay.portlet.messageboards New:
com.liferay.message.boards.exception

com.liferay.message.boards.api

NoSuchChoiceException

Old: com.liferay.portlet.polls New: com.liferay.polls.exception

com.liferay.polls.api

NoSuchContentException

Old: com.liferay.portlet.dynamicdatamapping New:
com.liferay.dynamic.data.mapping.exception

com.liferay.dynamic.data.mapping.api

NoSuchContentSearchException

Old: com.liferay.portlet.journal New: com.liferay.journal.exception

com.liferay.journal.api

NoSuchDiscussionException

Old: com.liferay.portlet.messageboards New:
com.liferay.message.boards.exception

com.liferay.message.boards.api

NoSuchFeedException

Old: com.liferay.portlet.journal New: com.liferay.journal.exception

com.liferay.journal.api

NoSuchFileRankException

Old: com.liferay.portlet.documentlibrary New:
com.liferay.document.library.file.rank.exception

com.liferay.document.library.file.rank.api

NoSuchMailingListException

Old: com.liferay.portlet.messageboards New:
com.liferay.message.boards.exception

com.liferay.message.boards.api

NoSuchNodeException

Old: com.liferay.portlet.wiki New: com.liferay.wiki.exception

com.liferay.wiki.api

NoSuchPageException

Old: com.liferay.portlet.wiki New: com.liferay.wiki.exception

com.liferay.wiki.api

NoSuchPageResourceException

Old: com.liferay.portlet.wiki New: com.liferay.wiki.exception

com.liferay.wiki.api

NoSuchQuestionException

Old: com.liferay.portlet.polls New: com.liferay.polls.exception

com.liferay.polls.api

NoSuchRecordException

Old: com.liferay.portlet.dynamicdatalists New:
com.liferay.dynamic.data.lists.exception

com.liferay.dynamic.data.lists.api

NoSuchRecordSetException

Old: com.liferay.portlet.dynamicdatalists New:
com.liferay.dynamic.data.lists.exception

com.liferay.dynamic.data.lists.api

NoSuchRecordVersionException

Old: com.liferay.portlet.dynamicdatalists New:
com.liferay.dynamic.data.lists.exception

com.liferay.dynamic.data.lists.api

NoSuchRuleException

Old: com.liferay.portlet.mobiledevicerules New:
com.liferay.mobile.device.rules.exception

com.liferay.mobile.device.rules.api

NoSuchRuleGroupException

Old: com.liferay.portlet.mobiledevicerules New:
com.liferay.mobile.device.rules.exception

com.liferay.mobile.device.rules.api

NoSuchRuleGroupInstanceException

Old: com.liferay.portlet.mobiledevicerules New:
com.liferay.mobile.device.rules.exception

com.liferay.mobile.device.rules.api

NoSuchStatsUserException

Old: com.liferay.portlet.blogs New: com.liferay.blogs.exception

com.liferay.blogs.api

NoSuchStorageLinkException

Old: com.liferay.portlet.dynamicdatamapping New:
com.liferay.dynamic.data.mapping.exception

com.liferay.dynamic.data.mapping.api

NoSuchStructureLinkException

Old: com.liferay.portlet.dynamicdatamapping New:
com.liferay.dynamic.data.mapping.exception

com.liferay.dynamic.data.mapping.api

NoSuchTemplateException

Old: com.liferay.portlet.dynamicdatamapping New:
com.liferay.dynamic.data.mapping.exception

com.liferay.dynamic.data.mapping.api

NoSuchThreadException

Old: com.liferay.portlet.messageboards New:
com.liferay.message.boards.exception

com.liferay.message.boards.api

NoSuchThreadFlagException

Old: com.liferay.portlet.messageboards New:
com.liferay.message.boards.exception

com.liferay.message.boards.api

NoSuchVoteException

Old: com.liferay.portlet.polls New: com.liferay.polls.exception

com.liferay.polls.api

NodeNameException

Old: com.liferay.portlet.wiki New: com.liferay.wiki.exception

com.liferay.wiki.api

PageContentException

Old: com.liferay.portlet.wiki New: com.liferay.wiki.exception

com.liferay.wiki.api

PageCreateDateComparator

Old: com.liferay.portlet.wiki.util.comparator New:
com.liferay.wiki.util.comparator

com.liferay.wiki.api

PageTitleComparator

Old: com.liferay.portlet.wiki.util.comparator New:
com.liferay.wiki.util.comparator

com.liferay.wiki.api

PageTitleException

Old: com.liferay.portlet.wiki New: com.liferay.wiki.exception

com.liferay.wiki.api

PageVersionComparator

Old: com.liferay.portlet.wiki.util.comparator New:
com.liferay.wiki.util.comparator

com.liferay.wiki.api

PageVersionException

Old: com.liferay.portlet.wiki New: com.liferay.wiki.exception

com.liferay.wiki.api

PollsChoice

Old: com.liferay.portlet.polls.model New: com.liferay.polls.model

com.liferay.polls.api

PollsChoiceLocalService

Old: com.liferay.portlet.polls.service New: com.liferay.polls.service

com.liferay.polls.api

PollsChoiceLocalServiceUtil

Old: com.liferay.portlet.polls.service New: com.liferay.polls.service

com.liferay.polls.api

PollsChoiceLocalServiceWrapper

Old: com.liferay.portlet.polls.service New: com.liferay.polls.service

com.liferay.polls.api

PollsChoiceModel

Old: com.liferay.portlet.polls.model New: com.liferay.polls.model

com.liferay.polls.api

PollsChoicePersistence

Old: com.liferay.portlet.polls.service.persistence New:
com.liferay.polls.service.persistence

com.liferay.polls.api

PollsChoiceService

Old: com.liferay.portlet.polls.service New: com.liferay.polls.service

com.liferay.polls.api

PollsChoiceServiceUtil

Old: com.liferay.portlet.polls.service New: com.liferay.polls.service

com.liferay.polls.api

PollsChoiceServiceWrapper

Old: com.liferay.portlet.polls.service New: com.liferay.polls.service

com.liferay.polls.api

PollsChoiceSoap

Old: com.liferay.portlet.polls.model New: com.liferay.polls.model

com.liferay.polls.api

PollsChoiceUtil

Old: com.liferay.portlet.polls.service.persistence New:
com.liferay.polls.service.persistence

com.liferay.polls.api

PollsChoiceWrapper

Old: com.liferay.portlet.polls.model New: com.liferay.polls.model

com.liferay.polls.api

PollsQuestion

Old: com.liferay.portlet.polls.model New: com.liferay.polls.model

com.liferay.polls.api

PollsQuestionLocalService

Old: com.liferay.portlet.polls.service New: com.liferay.polls.service

com.liferay.polls.api

PollsQuestionLocalServiceUtil

Old: com.liferay.portlet.polls.service New: com.liferay.polls.service

com.liferay.polls.api

PollsQuestionLocalServiceWrapper

Old: com.liferay.portlet.polls.service New: com.liferay.polls.service

com.liferay.polls.api

PollsQuestionModel

Old: com.liferay.portlet.polls.model New: com.liferay.polls.model

com.liferay.polls.api

PollsQuestionPersistence

Old: com.liferay.portlet.polls.service.persistence New:
com.liferay.polls.service.persistence

com.liferay.polls.api

PollsQuestionService

Old: com.liferay.portlet.polls.service New: com.liferay.polls.service

com.liferay.polls.api

PollsQuestionServiceUtil

Old: com.liferay.portlet.polls.service New: com.liferay.polls.service

com.liferay.polls.api

PollsQuestionServiceWrapper

Old: com.liferay.portlet.polls.service New: com.liferay.polls.service

com.liferay.polls.api

PollsQuestionSoap

Old: com.liferay.portlet.polls.model New: com.liferay.polls.model

com.liferay.polls.api

PollsQuestionUtil

Old: com.liferay.portlet.polls.service.persistence New:
com.liferay.polls.service.persistence

com.liferay.polls.api

PollsQuestionWrapper

Old: com.liferay.portlet.polls.model New: com.liferay.polls.model

com.liferay.polls.api

PollsVote

Old: com.liferay.portlet.polls.model New: com.liferay.polls.model

com.liferay.polls.api

PollsVoteLocalService

Old: com.liferay.portlet.polls.service New: com.liferay.polls.service

com.liferay.polls.api

PollsVoteLocalServiceUtil

Old: com.liferay.portlet.polls.service New: com.liferay.polls.service

com.liferay.polls.api

PollsVoteLocalServiceWrapper

Old: com.liferay.portlet.polls.service New: com.liferay.polls.service

com.liferay.polls.api

PollsVoteModel

Old: com.liferay.portlet.polls.model New: com.liferay.polls.model

com.liferay.polls.api

PollsVotePersistence

Old: com.liferay.portlet.polls.service.persistence New:
com.liferay.polls.service.persistence

com.liferay.polls.api

PollsVoteService

Old: com.liferay.portlet.polls.service New: com.liferay.polls.service

com.liferay.polls.api

PollsVoteServiceUtil

Old: com.liferay.portlet.polls.service New: com.liferay.polls.service

com.liferay.polls.api

PollsVoteServiceWrapper

Old: com.liferay.portlet.polls.service New: com.liferay.polls.service

com.liferay.polls.api

PollsVoteSoap

Old: com.liferay.portlet.polls.model New: com.liferay.polls.model

com.liferay.polls.api

PollsVoteUtil

Old: com.liferay.portlet.polls.service.persistence New:
com.liferay.polls.service.persistence

com.liferay.polls.api

PollsVoteWrapper

Old: com.liferay.portlet.polls.model New: com.liferay.polls.model

com.liferay.polls.api

PoolAction

Old: com.liferay.portal.kernel.memory New: com.liferay.petra.memory

com.liferay.petra.memory

PortalExecutorFactory

Old: com.liferay.portal.kernel.executor New:
com.liferay.portal.executor.internal

com.liferay.portal.executor

PortalToLDAPConverter

Old: com.liferay.portal.security.ldap New:
com.liferay.portal.security.ldap.exportimport

com.liferay.portal.security.ldap.api

PortletDisplayTemplate

Old: com.liferay.portlet.portletdisplaytemplate.util New:
com.liferay.portlet.display.template

com.liferay.portlet.display.template.api

PortletDisplayTemplateConstants

Old: com.liferay.portlet.portletdisplaytemplate.util New:
com.liferay.portlet.display.template.constants

com.liferay.portlet.display.template.api

PortletDisplayTemplateUtil

Old: com.liferay.portlet.portletdisplaytemplate.util New:
com.liferay.roles.admin.web.internal.util

com.liferay.roles.admin.web

PortletDisplayTemplateUtil

Old: com.liferay.portlet.portletdisplaytemplate.util New:
com.liferay.roles.admin.web.internal.util

com.liferay.roles.admin.web

PortletDisplayTemplateUtil

Old: com.liferay.portlet.portletdisplaytemplate.util New:
com.liferay.roles.admin.web.internal.util

com.liferay.roles.admin.web

QueryIndexingHitsProcessor

Old: com.liferay.portal.kernel.search New:
com.liferay.portal.search.internal.hits

com.liferay.portal.search

QuerySuggestionHitsProcessor

Old: com.liferay.portal.kernel.search New:
com.liferay.portal.search.internal.hits

com.liferay.portal.search

QueryType

Old: com.liferay.portal.kernel.bi.rules New:
com.liferay.portal.rules.engine

com.liferay.portal.rules.engine.api

QuestionChoiceException

Old: com.liferay.portlet.polls New: com.liferay.polls.exception

com.liferay.polls.api

QuestionDescriptionException

Old: com.liferay.portlet.polls New: com.liferay.polls.exception

com.liferay.polls.api

QuestionExpirationDateException

Old: com.liferay.portlet.polls New: com.liferay.polls.exception

com.liferay.polls.api

QuestionExpiredException

Old: com.liferay.portlet.polls New: com.liferay.polls.exception

com.liferay.polls.api

QuestionTitleException

Old: com.liferay.portlet.polls New: com.liferay.polls.exception

com.liferay.polls.api

RecordSetDDMStructureIdException

Old: com.liferay.portlet.dynamicdatalists New:
com.liferay.dynamic.data.lists.exception

com.liferay.dynamic.data.lists.api

RecordSetDuplicateRecordSetKeyException

Old: com.liferay.portlet.dynamicdatalists New:
com.liferay.dynamic.data.lists.exception

com.liferay.dynamic.data.lists.api

RecordSetNameException

Old: com.liferay.portlet.dynamicdatalists New:
com.liferay.dynamic.data.lists.exception

com.liferay.dynamic.data.lists.api

RegistryAwareMBeanServer

Old: com.liferay.portal.kernel.jmx New: com.liferay.portal.jmx.internal

com.liferay.portal.jmx

ReportCompilerRequestMessageListener

Old: com.liferay.portal.kernel.bi.reporting.messaging New:
com.liferay.portal.reports.engine.messaging

com.liferay.portal.reports.engine.api

ReportDataSourceType

Old: com.liferay.portal.kernel.bi.reporting New:
com.liferay.portal.reports.engine

com.liferay.portal.reports.engine.api

ReportDesignRetriever

Old: com.liferay.portal.kernel.bi.reporting New:
com.liferay.portal.reports.engine

com.liferay.portal.reports.engine.api

ReportEngine

Old: com.liferay.portal.kernel.bi.reporting New:
com.liferay.portal.reports.engine

com.liferay.portal.reports.engine.api

ReportExportException

Old: com.liferay.portal.kernel.bi.reporting New:
com.liferay.portal.reports.engine

com.liferay.portal.reports.engine.api

ReportFormat

Old: com.liferay.portal.kernel.bi.reporting New:
com.liferay.portal.reports.engine

com.liferay.portal.reports.engine.api

ReportFormatExporter

Old: com.liferay.portal.kernel.bi.reporting New:
com.liferay.portal.reports.engine

com.liferay.portal.reports.engine.api

ReportFormatExporterRegistry

Old: com.liferay.portal.kernel.bi.reporting New:
com.liferay.portal.reports.engine

com.liferay.portal.reports.engine.api

ReportGenerationException

Old: com.liferay.portal.kernel.bi.reporting New:
com.liferay.portal.reports.engine

com.liferay.portal.reports.engine.api

ReportRequest

Old: com.liferay.portal.kernel.bi.reporting New:
com.liferay.portal.reports.engine

com.liferay.portal.reports.engine.api

ReportRequestContext

Old: com.liferay.portal.kernel.bi.reporting New:
com.liferay.portal.reports.engine

com.liferay.portal.reports.engine.api

ReportRequestMessageListener

Old: com.liferay.portal.kernel.bi.reporting.messaging New:
com.liferay.portal.reports.engine.messaging

com.liferay.portal.reports.engine.api

ReportResultContainer

Old: com.liferay.portal.kernel.bi.reporting New:
com.liferay.portal.reports.engine

com.liferay.portal.reports.engine.api

RequestStatistics

Old: com.liferay.portal.kernel.monitoring.statistics New:
com.liferay.portal.monitoring.internal.statistics

com.liferay.portal.monitoring

RequiredMessageException

Old: com.liferay.portlet.messageboards New:
com.liferay.message.boards.exception

com.liferay.message.boards.api

RequiredNodeException

Old: com.liferay.portlet.wiki New: com.liferay.wiki.exception

com.liferay.wiki.api

RequiredTemplateException

Old: com.liferay.portlet.dynamicdatamapping New:
com.liferay.dynamic.data.mapping.exception

com.liferay.dynamic.data.mapping.api

RequiredTemplateException

Old: com.liferay.portlet.journal New:
com.liferay.dynamic.data.mapping.exception

com.liferay.dynamic.data.mapping.api

RuleGroupInstancePriorityComparator

Old: com.liferay.portlet.mobiledevicerules.util New:
com.liferay.mobile.device.rules.util.comparator

com.liferay.mobile.device.rules.api

RuleGroupProcessor

Old: com.liferay.portal.kernel.mobile.device.rulegroup New:
com.liferay.mobile.device.rules.rule

com.liferay.mobile.device.rules.api

RuleGroupProcessorUtil

Old: com.liferay.portal.kernel.mobile.device.rulegroup New:
com.liferay.mobile.device.rules.rule

com.liferay.mobile.device.rules.api

RuleHandler

Old: com.liferay.portal.kernel.mobile.device.rulegroup.rule New:
com.liferay.mobile.device.rules.rule

com.liferay.mobile.device.rules.api

RulesEngine

Old: com.liferay.portal.kernel.bi.rules New:
com.liferay.portal.rules.engine

com.liferay.portal.rules.engine.api

RulesEngineException

Old: com.liferay.portal.kernel.bi.rules New:
com.liferay.portal.rules.engine

com.liferay.portal.rules.engine.api

RulesEngineUtil

Old: com.liferay.portal.kernel.bi.rules New:
com.liferay.portal.rules.engine

com.liferay.portal.rules.engine.api

RulesLanguage

Old: com.liferay.portal.kernel.bi.rules New:
com.liferay.portal.rules.engine

com.liferay.portal.rules.engine.api

RulesResourceRetriever

Old: com.liferay.portal.kernel.bi.rules New:
com.liferay.portal.rules.engine

com.liferay.portal.rules.engine.api

SearchUtil

Old: com.liferay.portal.kernel.search.util New:
com.liferay.portal.search.web.internal.util

com.liferay.portal.search.web

ServletContextReportDesignRetriever

Old: com.liferay.portal.kernel.bi.reporting.servlet New:
com.liferay.portal.reports.engine.servlet

com.liferay.portal.reports.engine.api

SoftReferencePool

Old: com.liferay.portal.kernel.memory New: com.liferay.petra.memory

com.liferay.petra.memory

SortFactoryImpl

Old: com.liferay.portal.kernel.search New:
com.liferay.portal.search.internal

com.liferay.portal.search

SplitThreadException

Old: com.liferay.portlet.messageboards New:
com.liferay.message.boards.exception

com.liferay.message.boards.api

Statistics

Old: com.liferay.portal.kernel.monitoring.statistics New:
com.liferay.portal.monitoring.internal.statistics

com.liferay.portal.monitoring

StatsUserLastPostDateComparator

Old: com.liferay.portlet.blogs.util.comparator New:
com.liferay.blogs.util.comparator

com.liferay.blogs.api

StorageAdapter

Old: com.liferay.portlet.dynamicdatamapping.storage New:
com.liferay.dynamic.data.mapping.storage

com.liferay.dynamic.data.mapping.api

StorageEngine

Old: com.liferay.portlet.dynamicdatamapping.storage New:
com.liferay.dynamic.data.mapping.storage

com.liferay.dynamic.data.mapping.api

StorageException

Old: com.liferay.portlet.dynamicdatamapping New:
com.liferay.dynamic.data.mapping.exception

com.liferay.dynamic.data.mapping.api

StorageFieldNameException

Old: com.liferay.portlet.dynamicdatamapping New:
com.liferay.dynamic.data.mapping.exception

com.liferay.dynamic.data.mapping.api

StructureDuplicateStructureKeyException

Old: com.liferay.portlet.dynamicdatamapping New:
com.liferay.dynamic.data.mapping.exception

com.liferay.dynamic.data.mapping.api

StructureFieldException

Old: com.liferay.portlet.dynamicdatamapping New:
com.liferay.dynamic.data.mapping.exception

com.liferay.dynamic.data.mapping.api

StructureIdComparator

Old: com.liferay.portlet.dynamicdatamapping.util.comparator New:
com.liferay.dynamic.data.mapping.util.comparator

com.liferay.dynamic.data.mapping.api

StructureModifiedDateComparator

Old: com.liferay.portlet.dynamicdatamapping.util.comparator New:
com.liferay.dynamic.data.mapping.util.comparator

com.liferay.dynamic.data.mapping.api

StructureStructureKeyComparator

Old: com.liferay.portlet.dynamicdatamapping.util.comparator New:
com.liferay.dynamic.data.mapping.util.comparator

com.liferay.dynamic.data.mapping.api

SummaryStatistics

Old: com.liferay.portal.kernel.monitoring.statistics New:
com.liferay.portal.monitoring.internal.statistics

com.liferay.portal.monitoring

SynchronousMessageListener

Old: com.liferay.portal.kernel.messaging.sender New:
com.liferay.portal.messaging.internal.sender

com.liferay.portal.messaging

TemplateDuplicateTemplateKeyException

Old: com.liferay.portlet.dynamicdatamapping New:
com.liferay.dynamic.data.mapping.exception

com.liferay.dynamic.data.mapping.api

TemplateIdComparator

Old: com.liferay.portlet.dynamicdatamapping.util.comparator New:
com.liferay.dynamic.data.mapping.util.comparator

com.liferay.dynamic.data.mapping.api

TemplateModifiedDateComparator

Old: com.liferay.portlet.dynamicdatamapping.util.comparator New:
com.liferay.dynamic.data.mapping.util.comparator

com.liferay.dynamic.data.mapping.api

TemplateNameException

Old: com.liferay.portlet.dynamicdatamapping New:
com.liferay.dynamic.data.mapping.exception

com.liferay.dynamic.data.mapping.api

TemplateNameException

Old: com.liferay.portlet.journal New:
com.liferay.dynamic.data.mapping.exception

com.liferay.dynamic.data.mapping.api

TemplateScriptException

Old: com.liferay.portlet.dynamicdatamapping New:
com.liferay.dynamic.data.mapping.exception

com.liferay.dynamic.data.mapping.api

TemplateSmallImageNameException

Old: com.liferay.portlet.dynamicdatamapping New:
com.liferay.dynamic.data.mapping.exception

com.liferay.dynamic.data.mapping.api

TemplateSmallImageNameException

Old: com.liferay.portlet.journal New:
com.liferay.dynamic.data.mapping.exception

com.liferay.dynamic.data.mapping.api

TemplateSmallImageSizeException

Old: com.liferay.portlet.dynamicdatamapping New:
com.liferay.dynamic.data.mapping.exception

com.liferay.dynamic.data.mapping.api

TemplateSmallImageSizeException

Old: com.liferay.portlet.journal New:
com.liferay.dynamic.data.mapping.exception

com.liferay.dynamic.data.mapping.api

ThreadLastPostDateComparator

Old: com.liferay.portlet.messageboards.util.comparator New:
com.liferay.message.boards.util.comparator

com.liferay.message.boards.api

UnknownRuleHandlerException

Old: com.liferay.portal.kernel.mobile.device.rulegroup.rule New:
com.liferay.mobile.device.rules.rule

com.liferay.mobile.device.rules.api

UserConverterKeys

Old: com.liferay.portal.security.ldap New:
com.liferay.portal.security.ldap

com.liferay.portal.security.ldap.api

WikiFormatException

Old: com.liferay.portlet.wiki New: com.liferay.wiki.exception

com.liferay.wiki.api

WikiNode

Old: com.liferay.portlet.wiki.model New: com.liferay.wiki.model

com.liferay.wiki.api

WikiNodeLocalService

Old: com.liferay.portlet.wiki.service New: com.liferay.wiki.service

com.liferay.wiki.api

WikiNodeLocalServiceUtil

Old: com.liferay.portlet.wiki.service New: com.liferay.wiki.service

com.liferay.wiki.api

WikiNodeLocalServiceWrapper

Old: com.liferay.portlet.wiki.service New: com.liferay.wiki.service

com.liferay.wiki.api

WikiNodeModel

Old: com.liferay.portlet.wiki.model New: com.liferay.wiki.model

com.liferay.wiki.api

WikiNodePersistence

Old: com.liferay.portlet.wiki.service.persistence New:
com.liferay.wiki.service.persistence

com.liferay.wiki.api

WikiNodeService

Old: com.liferay.portlet.wiki.service New: com.liferay.wiki.service

com.liferay.wiki.api

WikiNodeServiceUtil

Old: com.liferay.portlet.wiki.service New: com.liferay.wiki.service

com.liferay.wiki.api

WikiNodeServiceWrapper

Old: com.liferay.portlet.wiki.service New: com.liferay.wiki.service

com.liferay.wiki.api

WikiNodeSoap

Old: com.liferay.portlet.wiki.model New: com.liferay.wiki.model

com.liferay.wiki.api

WikiNodeUtil

Old: com.liferay.portlet.wiki.service.persistence New:
com.liferay.wiki.service.persistence

com.liferay.wiki.api

WikiNodeWrapper

Old: com.liferay.portlet.wiki.model New: com.liferay.wiki.model

com.liferay.wiki.api

WikiPage

Old: com.liferay.portlet.wiki.model New: com.liferay.wiki.model

com.liferay.wiki.api

WikiPageConstants

Old: com.liferay.portlet.wiki.model New: com.liferay.wiki.model

com.liferay.wiki.api

WikiPageDisplay

Old: com.liferay.portlet.wiki.model New: com.liferay.wiki.model

com.liferay.wiki.api

WikiPageFinder

Old: com.liferay.portlet.wiki.service.persistence New:
com.liferay.wiki.service.persistence

com.liferay.wiki.api

WikiPageLocalService

Old: com.liferay.portlet.wiki.service New: com.liferay.wiki.service

com.liferay.wiki.api

WikiPageLocalServiceUtil

Old: com.liferay.portlet.wiki.service New: com.liferay.wiki.service

com.liferay.wiki.api

WikiPageLocalServiceWrapper

Old: com.liferay.portlet.wiki.service New: com.liferay.wiki.service

com.liferay.wiki.api

WikiPageModel

Old: com.liferay.portlet.wiki.model New: com.liferay.wiki.model

com.liferay.wiki.api

WikiPagePersistence

Old: com.liferay.portlet.wiki.service.persistence New:
com.liferay.wiki.service.persistence

com.liferay.wiki.api

WikiPageResource

Old: com.liferay.portlet.wiki.model New: com.liferay.wiki.model

com.liferay.wiki.api

WikiPageResourceLocalService

Old: com.liferay.portlet.wiki.service New: com.liferay.wiki.service

com.liferay.wiki.api

WikiPageResourceLocalServiceUtil

Old: com.liferay.portlet.wiki.service New: com.liferay.wiki.service

com.liferay.wiki.api

WikiPageResourceLocalServiceWrapper

Old: com.liferay.portlet.wiki.service New: com.liferay.wiki.service

com.liferay.wiki.api

WikiPageResourceModel

Old: com.liferay.portlet.wiki.model New: com.liferay.wiki.model

com.liferay.wiki.api

WikiPageResourcePersistence

Old: com.liferay.portlet.wiki.service.persistence New:
com.liferay.wiki.service.persistence

com.liferay.wiki.api

WikiPageResourceSoap

Old: com.liferay.portlet.wiki.model New: com.liferay.wiki.model

com.liferay.wiki.api

WikiPageResourceUtil

Old: com.liferay.portlet.wiki.service.persistence New:
com.liferay.wiki.service.persistence

com.liferay.wiki.api

WikiPageResourceWrapper

Old: com.liferay.portlet.wiki.model New: com.liferay.wiki.model

com.liferay.wiki.api

WikiPageService

Old: com.liferay.portlet.wiki.service New: com.liferay.wiki.service

com.liferay.wiki.api

WikiPageServiceUtil

Old: com.liferay.portlet.wiki.service New: com.liferay.wiki.service

com.liferay.wiki.api

WikiPageServiceWrapper

Old: com.liferay.portlet.wiki.service New: com.liferay.wiki.service

com.liferay.wiki.api

WikiPageSoap

Old: com.liferay.portlet.wiki.model New: com.liferay.wiki.model

com.liferay.wiki.api

WikiPageUtil

Old: com.liferay.portlet.wiki.service.persistence New:
com.liferay.wiki.service.persistence

com.liferay.wiki.api

WikiPageWrapper

Old: com.liferay.portlet.wiki.model New: com.liferay.wiki.model

com.liferay.wiki.api

\chapter{Front-End}\label{front-end}

Front-end development involves multiple frameworks and tools. Keeping
track of all the moving pieces in your project can be a daunting task.
This section of reference docs provides the following helpful
information for front-end development:

\begin{itemize}
\tightlist
\item
  Understanding the liferay-npm-bundler
\item
  The CKEditor plugins available for use in your
  \href{/docs/7-1/tutorials/-/knowledge_base/t/adding-buttons-to-alloyeditor-toolbars}{custom
  AlloyEditor configurations}.
\item
  AlloyEditor button reference guide
\item
  Fully qualified portlet IDs
\item
  FreeMarker taglib macros
\item
  Setting up your npm environment
\item
  Liferay JS Generator
\end{itemize}

\chapter{liferay-npm-bundler}\label{liferay-npm-bundler}

The liferay-npm-bundler is a bundler (like
\href{https://webpack.github.io/}{Webpack} or
\href{http://browserify.org/}{Browserify} ) that targets Liferay DXP as
a platform and assumes you're using your npm packages from portlets (as
opposed to typical web applications).

The workflow for running npm packages inside portlets is slightly
different from standard bundlers. Instead of bundling the JavaScript in
a single file, you must \emph{link} all packages together in the browser
when the full web page is assembled. This lets portlets share common
versions of modules instead of each one loading its own copy. The
liferay-npm-bundler handles this for you.

This section of reference docs covers how Portal supports npm-based
portlet projects with the liferay-npm-bundler.

\noindent\hrulefill

\textbf{Note:} You can also find information for the liferay-npm-bundler
in the project's
\href{https://github.com/liferay/liferay-npm-build-tools/wiki}{Wiki}.

\section{How the Liferay npm Bundler Works
Internally}\label{how-the-liferay-npm-bundler-works-internally}

The liferay-npm-bundler takes a portlet project and outputs its files
(including npm packages) to a build folder, so the standard portlet
build (Gradle) can produce an OSGi bundle. You can learn more about the
build folder's structure in
\href{/docs/7-1/reference/-/knowledge_base/r/the-structure-of-osgi-bundles-containing-npm-packages}{The
Structure of OSGi Bundles Containing NPM Packages} reference.

The liferay-npm-bundler uses the process below to create the OSGi
bundle:

\begin{enumerate}
\def\labelenumi{\arabic{enumi}.}
\item
  Copy the project's \texttt{package.json} file to the output directory.
\item
  Traverse the project's dependency tree to determine its dependencies.
\item
  For the project:

  \begin{enumerate}
  \def\labelenumii{\alph{enumii}.}
  \item
    Run the source files, specified in the \texttt{.npmbundlerrc}
    configuration, through the rules.
  \item
    Pre-process the project's package with any configured plugins.
  \item
    Run \href{https://babeljs.io/}{Babel} with configured plugins for
    each \texttt{.js} file inside the project.
  \item
    Post-process the project package with any configured plugins.
  \end{enumerate}
\item
  For each npm package dependency:

  \begin{enumerate}
  \def\labelenumii{\alph{enumii}.}
  \item
    Copy the npm package to the output folder and prefix the bundle's
    name to it. Note that the bundler stores packages in a plain
    \emph{bundle-name\$package}@\emph{version} format, rather than the
    standard node\_modules tree format. To determine what is copied, the
    bundler invokes a plugin to filter the package file list.
  \item
    Run rules on the package files.
  \item
    Pre-process the npm package with any configured plugins.
  \item
    Run \href{https://babeljs.io/}{Babel} with configured plugins for
    each \texttt{.js} file inside the npm package.
  \item
    Post-process the npm package with any configured plugins.
  \end{enumerate}
\end{enumerate}

The only difference between the pre-process and post-process steps are
when they are run (before or after Babel is run, respectively). During
this workflow, liferay-npm-bundler calls all the configured plugins so
they can perform transformations on the npm packages (for instance,
modifying their \texttt{package.json} files, or deleting or moving
files).

\noindent\hrulefill

\textbf{Note:} that the pre, post, and Babel phases were designed for
the old mode of operation (See the
\href{/docs/7-1/tutorials/-/knowledge_base/t/migrating-your-project-to-use-the-new-mode}{Migrating
Your Project to Use the New Mode} for more information) and they will
gradually be replaced with rules for the new mode.

\section{Configuring
liferay-npm-bundler}\label{configuring-liferay-npm-bundler}

The liferay-npm-bundler is configured via a \texttt{.npmbundlerrc} file
placed in the portlet project's root folder. You can create a complete
configuration manually or extend a configuration preset (via Babel).

This article explains the \texttt{.npmbundlerrc} file's structure. See
the
\href{/docs/7-1/reference/-/knowledge_base/r/how-the-default-preset-configures-the-liferay-npm-bundler}{default
preset reference} to learn how the default preset configures the
liferay-npm-bundler. See the
\href{/docs/7-1/reference/-/knowledge_base/r/how-the-default-preset-configures-the-liferay-npm-bundler}{Creating
JavaScript Portlets with JavaScript Tooling} tutorial to learn how to
use the liferay-npm-bundler to create JavaScript portlets.

\subsection{Understanding the .npmbundlerrc File's
Structure}\label{understanding-the-.npmbundlerrc-files-structure}

The \texttt{.npmbundlerrc} file has four possible phase definitions:
\emph{copy-process}, \emph{pre-process}, \emph{post-process}, and
\emph{babel}. These phase definitions are explained in more detail
below:

\textbf{Copy-Process:} Defined with the \texttt{copy-plugins} property
(only available for dependency packages). Specifies which files should
be copied or excluded from each given package.

\textbf{Pre-Process:} Defined with the \texttt{plugins} property.
Specifies plugins to run before the Babel phase is run.

\textbf{Babel:} Defined with the \texttt{.babelrc} definition. Specifies
the \texttt{.babelrc} file to use when running Babel through the
package's \texttt{.js} files.

\noindent\hrulefill

\textbf{Note:} During this phase, Babel transforms package files (for
example, to convert them to AMD format, if necessary), but doesn't
transpile them. In theory, you could also transpile them by configuring
the proper plugins. We recommend transpiling before running the bundler,
to avoid mixing both unrelated processes.

\noindent\hrulefill

\textbf{Post-Process:} Defined with the \texttt{post-plugins} property.
An alternative to using the \emph{pre-process} phase, this specifies
plugins to run after the Babel phase has completed.

Here's an example of a \texttt{.npmbundlerrc} configuration:

\begin{verbatim}
{
    "exclude": {
        "*": [
            "test/**/*"
        ],
        "some-package-name": [
            "test/**/*",
            "bin/**/*"
        ],
        "another-package-name@1.0.10": [
            "test/**/*",
            "bin/**/*",
            "lib/extras-1.0.10.js"
        ]
    },
    "include-dependencies": [
        "isobject", "isarray"
    ],
    "output": "build",
    "process-serially": false,
    "verbose": false,
    "dump-report": true,
    "config": {
        "imports": {
            "npm-angular5-provider": {
                "@angular/common": "^5.0.0",
                "@angular/core": "^5.0.0"
            }
        }
    },
    "/": {
        "plugins": ["resolve-linked-dependencies"],
        ".babelrc": {
            "presets": ["liferay-standard"]
        },
        "post-plugins": [
            "namespace-packages",
            "inject-imports-dependencies"
        ]
    },
    "*": {
        "copy-plugins": ["exclude-imports"],
        "plugins": ["replace-browser-modules"],
        ".babelrc": {
            "presets": ["liferay-standard"]
        },
        "post-plugins": [
            "namespace-packages",
            "inject-imports-dependencies",
            "inject-peer-dependencies"
        ]
    },
    "packages": {
        "a-package-name": [
        "copy-plugins": ["exclude-imports"],
            "plugins": ["replace-browser-modules"],
            ".babelrc": {
                "presets": ["liferay-standard"]
            },
            "post-plugins": [
                "namespace-packages",
                "inject-imports-dependencies",
                "inject-peer-dependencies"
            ]
        ],
        "other-package-name@1.0.10": [
          "copy-plugins": ["exclude-imports"],
            "plugins": ["replace-browser-modules"],
            ".babelrc": {
                "presets": ["liferay-standard"]
            },
            "post-plugins": [
                "namespace-packages",
                "inject-imports-dependencies",
                "inject-peer-dependencies"
            ]
        ]
    }
}
\end{verbatim}

\noindent\hrulefill

\textbf{Note:} Not all definition formats (\texttt{*},
\texttt{some-package-name}, and \texttt{some-package-name@version})
shown above are required. In most cases, the wildcard definition
(\texttt{*}) is enough. The non-wildcard formats
(\texttt{some-package-name} and \texttt{some-package-name@version}) are
rare exceptions for packages that require a more specific configuration
than the wildcard definition provides.

\noindent\hrulefill

\subsubsection{Standard Configuration
Options}\label{standard-configuration-options}

Below are the standard configuration options for the
\texttt{.npmbundlerrc} file:

\emph{config}: global configuration which is passed to all bundler and
Babel plugins. Please refer to each plugin's documentation to find the
available options for each specific plugin.

\emph{dump-report:} Sets whether to generate a debugging report. If
\texttt{true}, a \texttt{liferay-npm-bundler-report.html} file is
generated in the project directory with information such as what the
liferay-npm-bundler is doing with each package. Note that you can also
pass this as the build flag
\texttt{\$\ liferay-npm-bundler\ -\/-dump-report} or
\texttt{\$\ liferay-npm-bundler\ -r}. The default value is
\texttt{false}.

\emph{no-tracking:} whether to send usage analytics to our servers. Note
that you can also pass this as a build flag with the CLI argument
\texttt{\$\ liferay-npm-bundler\ -\/-no-tracking}, or by creating a
marker file called \texttt{.liferay-npm-bundler-no-tracking} in the
project's root folder or any of its ancestors, or by setting the
environment variable
\texttt{LIFERAY\_NPM\_BUNDLER\_NO\_TRACKING=\textquotesingle{}\textquotesingle{}}.
The default value is \texttt{false}.

\emph{output:} by default the bundler writes packages to the standard
Gradle resources folder:
\texttt{build/resources/main/META-INF/resources}. Set this value to
override the default output folder. Note that the dependency npm
packages are placed in a \texttt{node\_modules} folder inside the build
folder. Note if \texttt{create-jar} is set, the default output folder is
\texttt{build}.

\emph{preset:} specifies the \texttt{liferay-npm-bundler} preset to use
as a base configuration. Note that if a \texttt{.npmbundlerrc} file is
not provided, the default \texttt{liferay-npm-bundler-preset-standard}
preset is used.

\emph{verbose:} Sets whether to output detailed information about what
the tool is doing to the console. The default value is \texttt{false}.

\subsubsection{Package Processing
Options}\label{package-processing-options}

\emph{``/''}: plugins' configuration for the project's package.

\emph{``"}: plugins' configuration for dependency packages.

\emph{(asterisk)}: Defines the default plugin configuration for all npm
packages. It contains four values identified by a corresponding key.
Keys \texttt{copy-plugins}, \texttt{plugins} and \texttt{post-plugins}
identify arrays of \texttt{liferay-npm-bundler} plugins to apply in the
copy, pre and post process steps. Key \texttt{.babelrc} identifies an
object specifying the configuration to use in the Babel step and has the
same structure of a standard \texttt{.babelrc} file.

\emph{exclude:} defines glob expressions of files to exclude from
bundling from all or specific packages. Each list is an array identified
by one of the following keys: \texttt{*} (any package),
\texttt{\{package\ name\}} (any version of the package), or
\texttt{\{package\ name\}@\{version\}} (a specific version of a
package). Below is an example configuration:

\begin{verbatim}
{
  "exclude": {
    "*": ["__tests__/**/*"],
    "is-object": ["test/**/*"],
    "is-array@1.0.1": ["test/**/*", "Makefile"]
  }
}
\end{verbatim}

\emph{ignore:} skips processing the specified JavaScript files with
Babel for the project. An example configuration is shown below:

\begin{verbatim}
{
  "ignore": ["lib/legacy/**/*.js"]
}
\end{verbatim}

\emph{include-dependencies:} defines packages to include in bundling,
even if they are not listed under the \texttt{dependencies} section of
\texttt{package.json}. These packages must be available in the
\texttt{node\_modules} folder (i.e.~installed manually, without saving
them to \texttt{package.json}, or listed in the \texttt{devDependencies}
section).

\emph{max-parallel-files:} Defines the maximum number of files to
process in parallel to avoid EMFILE errors (especially on Windows). The
default value is \texttt{128}.

\emph{packages:} defines plugin configuration for npm packages, per
package.

\emph{process-serially:} \emph{removed} since v2.7.0. Replaced with
\texttt{max-parallel-files}.

\texttt{rules:} defines rules to apply to the projects source files with
the loader. Rules must have a \texttt{use} array property, which defines
the loader to use, which may be specified by just a package name or an
object with \texttt{loader} and \texttt{options} properties if
applicable, and one or more of the properties below:

\begin{itemize}
\tightlist
\item
  \texttt{test}: defines a regular expression to filter files in the
  \texttt{sources} folders to determine whether to apply rules to them.
  The project-relative path of each eligible file is compared against
  the regular expression and files that match are processed by the
  loaders.
\item
  \texttt{exclude}: refines the \texttt{test} expression by specifying
  files to exclude.
\item
  \texttt{include}: refines the \texttt{test} expression by specifying
  files to include.
\end{itemize}

An example configuration is shown below:

\begin{verbatim}
{
  "rules": [
    {
      "test": "\\.js$",
      "exclude": "node_modules",
      "use": [
        {
          "loader": "babel-loader",
          "options": {
            "presets": ["env", "react"]
          }
        }
      ]
    },
    {
      "test": "\\.css$",
      "use": ["style-loader"]
    },
    {
      "test": "\\.json$",
      "use": ["json-loader"]
    }
  ]
}
\end{verbatim}

\texttt{sources:} defines the folders in the project that contain the
source files to apply rules to. Folders can be nested
(e.g.~\texttt{/src/main/resources/}) and must be written using POSIX
path separators (i.e.~use \texttt{/} instead of
\texttt{\textbackslash{}} on Win32 systems). Note that rules are
automatically applied to package dependency files of the project.

An example configuration is shown below:

\begin{verbatim}
{
  "sources": ["src", "assets"]
}
\end{verbatim}

\subsubsection{OSGi Bundle Creation
Options}\label{osgi-bundle-creation-options}

Since version 2.2.0, the liferay-npm-bundler can create portlet OSGi
bundles for you. See the
\href{/docs/7-1/tutorials/-/knowledge_base/t/creating-and-bundling-javascript-portlets-with-javascript-tooling}{Creating
and Bundling JavaScript Portlets with JavaScript Tooling} tutorial for
complete instructions. The configuration options for OSGi bundle
creation are shown below:

\begin{itemize}
\item
  \textbf{create-jar}: Creates an OSGi bundle when set to a truthy
  value. When set to \texttt{true}, all sub-options take default values.
  When an object is passed, as shown in the example above, each
  sub-option can be configured individually. Note that you can also pass
  this as a build flag: \texttt{\$\ liferay-npm-bundler\ -\/-create-} or
  \texttt{\$\ liferay-npm-bundler\ -j}. The default value is
  \texttt{false}.

  \{ ``create-jar'': true \}
\item
  \textbf{create-jar.auto-deploy-portlet}: \textbf{Note} that this
  option is deprecated. Use the \texttt{create-jar.features.js-extender}
  option instead.
\item
  \textbf{create-jar.features.configuration}: specifies the file
  describing the system (OSGi) and widget instance (portlet preferences,
  as defined in the Portlet spec) configuration to use. (see
  \href{/docs/7-1/tutorials/-/knowledge_base/t/configuring-system-settings-and-instance-settings-for-your-js-portlet}{Configuring
  System Settings and Instance Settings for Your JavaScript Portlets}
  for more information on the required settings configuration). The
  default value is \texttt{features/configuration.json} if that file
  exists, otherwise the default is \texttt{undefined}.

  \{ ``create-jar'': \{ ``features'': \{ ``configuration'':
  ``features/configuration.json'' \} \} \}
\item
  \textbf{create-jar.output-dir:} specifies where to place the final JAR

  \{ ``create-jar'': \{ ``features'': \{ ``configuration'':
  ``features/configuration.json'' \} \} \}
\item
  \textbf{create-jar.features.js-extender:} controls whether to process
  the OSGi bundle with the JS Portlet Extender
  \href{https://web.liferay.com/marketplace/-/mp/application/115542926}{CE
  App}
\end{itemize}

\noindent\hrulefill

\noindent\hrulefill
\href{https://web.liferay.com/marketplace/-/mp/application/115543020}{DXP
App}. You can also specify the minimum required version of the Extender
to use for the bundle. This can be useful if you want to use advanced
features in your bundle, but you want it to be deployable in older
versions of the Extender. Pass the string \texttt{"any"} to let the
bundle deploy in any version of the Extender. If \texttt{true}, the
liferay-npm-bundler automatically determines the minimum version of the
Extender required for the features used in the bundle. the default value
is \texttt{true}. An example configuration is shown below:

\begin{verbatim}
{
  "create-jar": {
    "features": {
      "js-extender": "1.1.0"
    }
  }
}
\end{verbatim}

\begin{itemize}
\item
  \textbf{create-jar.features.web-context:} specifies the context path
  to use for publishing bundle's static resources. The default value is
  \texttt{/\{project\ name\}-\{project\ version\}}.

  \{ ``create-jar'': \{ ``features'': \{ ``web-context'':
  ``/my-project'' \} \} \}
\item
  \textbf{create-jar.features.localization:} specifies the L10N file to
  be used by the bundle (see the
  \href{Creating-OSGi-bundles\#localization}{Creating JS Portlets with
  JS Tooling} tutorial for more information on using localization in
  your portlet. The default value is
  \texttt{features/localization/Language} if a properties file with that
  base name exists, otherwise the default is \texttt{undefined}.

  \{ ``create-jar'': \{ ``features'': \{ ``localization'':
  ``features/localization/Language'' \} \} \}
\item
  \textbf{create-jar.features.settings:} specifies the JSON file
  describing the configuration structure (see the
  \href{Creating-OSGi-bundles\#settings-configuration}{Creating JS
  Portlets with JS Tooling} tutorial for more information on the
  required settings configuration). The default value is
  \texttt{features/settings.json} if that file exists, otherwise the
  default is \texttt{undefined}.
\end{itemize}

\noindent\hrulefill

\textbf{Note:} Plugins' configuration specifies the options for
configuring plugins in all the possible phases, as well as the
\texttt{.babelrc} file to use when running Babel (see
\href{https://babeljs.io/docs/usage/babelrc/}{Babel's documentation} for
more information on that file format).

\noindent\hrulefill

\noindent\hrulefill

\textbf{Note:} Prior to version 1.4.0 of the liferay-npm-bundler,
package configurations were placed next to the tools options
(\texttt{*}, \texttt{output}, \texttt{exclude}, etc.) To prevent package
name collisions, package configurations are now namespaced and placed
under the \texttt{packages} section. To maintain backwards
compatibility, the liferay-npm-bundler falls back to the root section
outside \texttt{packages} for package configuration, if no package
configurations (\texttt{package-name@version}, \texttt{package-name}, or
\texttt{*}) are found in the \texttt{packages} section.

\noindent\hrulefill

Now you know the structure of the \texttt{.npmbundlerrc} file!

\section{How the Default Preset Configures the
liferay-npm-bundler}\label{how-the-default-preset-configures-the-liferay-npm-bundler}

The liferay-npm-bundler comes with a default configuration preset:
\href{https://github.com/liferay/liferay-npm-build-tools/tree/master/packages/liferay-npm-bundler-preset-standard}{\texttt{liferay-npm-bundler-preset-standard}}
in your \texttt{.npmbundlerrc} file. This preset configures several
plugins for the build process and is automatically used (even if the
\texttt{.npmbundlerrc} is missing), unless you override it with one of
your own. Running the liferay-npm-bundler with this preset applies the
\href{https://github.com/liferay/liferay-npm-build-tools/blob/master/packages/liferay-npm-bundler-preset-standard/config.json}{config
file} from \texttt{liferay-npm-bundler-preset-standard}:

\begin{verbatim}
{
    "/": {
        "plugins": ["resolve-linked-dependencies"],
        ".babelrc": {
            "presets": ["liferay-standard"]
        },
        "post-plugins": ["namespace-packages", "inject-imports-dependencies"]
    },
    "*": {
        "copy-plugins": ["exclude-imports"],
        "plugins": ["replace-browser-modules"],
        ".babelrc": {
            "presets": ["liferay-standard"]
        },
        "post-plugins": [
            "namespace-packages",
            "inject-imports-dependencies",
            "inject-peer-dependencies"
        ]
    }
}
\end{verbatim}

The configuration above states that for all npm packages (\texttt{*})
the pre-process phase (\texttt{plugins}) must run the
\texttt{replace-browser-modules} plugin. Setting this to
\texttt{post-plugins} would run it during the post phase instead.

\noindent\hrulefill

\textbf{Note:} You can override configuration preset values by adding
your own configuration to your project's \texttt{.npmbundlerrc} file.
For instance, using the configuration preset example above, you can
define your own \texttt{.babelrc} value in \texttt{.npmbundlerrc} file
to override the defined ``liferay-standard'' babelrc preset.

\noindent\hrulefill

The
\href{https://github.com/liferay/liferay-npm-build-tools/tree/master/packages/babel-preset-liferay-standard}{\texttt{liferay-standard}
preset} applies the following plugins to packages:

\begin{itemize}
\item
  \href{https://github.com/liferay/liferay-npm-build-tools/tree/master/packages/liferay-npm-bundler-plugin-exclude-imports}{exclude-imports}:
  Exclude packages declared in the \texttt{imports} section from the
  build.
\item
  \href{https://github.com/liferay/liferay-npm-build-tools/tree/master/packages/liferay-npm-bundler-plugin-inject-imports-dependencies}{inject-imports-dependencies}:
  Inject dependencies declared in the \texttt{imports} section in the
  dependencies' \texttt{package.json} files.
\item
  \href{https://github.com/liferay/liferay-npm-build-tools/tree/master/packages/liferay-npm-bundler-plugin-inject-peer-dependencies}{inject-peer-dependencies}:
  Inject declared peer dependencies (as they are resolved in the
  project's \texttt{node\_modules} folder) in the dependencies'
  \texttt{package.json} files.
\item
  \href{https://github.com/liferay/liferay-npm-build-tools/tree/master/packages/liferay-npm-bundler-plugin-namespace-packages}{namespace-packages}:
  Namespace package names based on the root project's package name to
  isolate packages per project and avoid collisions. This prepends
  \texttt{\textless{}project-package-name\textgreater{}\$} to each
  package name appearance in \texttt{package.json} files.
\item
  \href{https://github.com/liferay/liferay-npm-build-tools/tree/master/packages/liferay-npm-bundler-plugin-replace-browser-modules}{replace-browser-modules}:
  Replaces the server side files for modules listed under
  \texttt{browser}/\texttt{unpkg}/\texttt{jsdelivr} section of
  \texttt{package.json} with their browser counterparts.
\item
  \href{https://github.com/liferay/liferay-npm-build-tools/tree/master/packages/liferay-npm-bundler-plugin-resolve-linked-dependencies}{resolve-linked-dependencies}:
  Replace linked dependencies versions appearing in
  \texttt{package.json} files (those obtained from local file system or
  GitHub, for example) by their real version number, as resolved in the
  project's \texttt{node\_modules} directory.
\end{itemize}

In addition, the bundler runs Babel with the
\href{https://github.com/liferay/liferay-npm-build-tools/tree/master/packages/babel-preset-liferay-standard}{babel-preset-liferay-standard}
preset, that invokes the following plugins:

\begin{itemize}
\item
  \href{https://github.com/liferay/liferay-npm-build-tools/tree/master/packages/babel-plugin-normalize-requires}{babel-plugin-normalize-requires}:
  Normalize AMD \texttt{require()} calls.
\item
  \href{https://github.com/babel/minify/tree/master/packages/babel-plugin-transform-node-env-inline}{babel-plugin-transform-node-env-inline}:
  Inline the \texttt{NODE\_ENV} environment variable, and if it's part
  of a binary expression (eg.
  \texttt{process.env.NODE\_ENV\ ===\ "development"}), then statically
  evaluate and replace it.
\item
  \href{https://www.npmjs.com/package/babel-plugin-minify-dead-code-elimination}{babel-plugin-minify-dead-code-elimination}:
  Inline bindings when possible. Tries to evaluate expressions and
  prunes unreachable as a result.
\item
  \href{https://github.com/liferay/liferay-npm-build-tools/tree/master/packages/babel-plugin-wrap-modules-amd}{babel-plugin-wrap-modules-amd}:
  Wrap modules inside an AMD \texttt{define()} module.
\item
  \href{https://github.com/liferay/liferay-npm-build-tools/tree/master/packages/babel-plugin-name-amd-modules}{babel-plugin-name-amd-modules}:
  Name AMD modules based on package name, version, and module path.
\item
  \href{https://github.com/liferay/liferay-npm-build-tools/tree/master/packages/babel-plugin-namespace-modules}{babel-plugin-namespace-modules}:
  Namespace modules based on the root project's package name, prepending
  \texttt{\textless{}project-package-name\textgreater{}\$}. Wrap modules
  inside an AMD \texttt{define()} module for each module name appearance
  (in \texttt{define()} or \texttt{require()} calls) so that the
  packages are localized per project and don't clash.
\item
  \href{https://github.com/liferay/liferay-npm-build-tools/tree/master/packages/babel-plugin-namespace-amd-define}{babel-plugin-namespace-amd-define}:
  Add a prefix to AMD \texttt{define()} calls (by default
  \texttt{Liferay.Loader.}).
\end{itemize}

Now you know the available configuration presets for
\texttt{.npmbundlerrc} and how they work.

\section{The Structure of OSGi Bundles Containing npm
Packages}\label{the-structure-of-osgi-bundles-containing-npm-packages}

To deploy JavaScript modules, you must create an OSGi bundle with the
npm dependencies extracted from the project's \texttt{node\_modules}
folder and modify them to work with the
\href{https://github.com/liferay/liferay-amd-loader}{Liferay AMD
Loader}. The liferay-npm-bundler automates this process for you,
creating a bundle similar to the one below:

\begin{itemize}
\tightlist
\item
  \texttt{my-bundle/}

  \begin{itemize}
  \tightlist
  \item
    \texttt{META-INF/}

    \begin{itemize}
    \tightlist
    \item
      \texttt{resources/}

      \begin{itemize}
      \tightlist
      \item
        \texttt{package.json}

        \begin{itemize}
        \tightlist
        \item
          name: my-bundle-package
        \item
          version: 1.0.0
        \item
          main: lib/index
        \item
          dependencies:

          \begin{itemize}
          \tightlist
          \item
            my-bundle-package\$isarray: 2.0.0
          \item
            my-bundle-package\$isobject: 2.1.0
          \end{itemize}
        \item
          \ldots{}
        \end{itemize}
      \item
        \texttt{lib/}

        \begin{itemize}
        \tightlist
        \item
          \texttt{index.js}
        \item
          \ldots{}
        \end{itemize}
      \item
        \ldots{}
      \item
        \texttt{node\_modules/}

        \begin{itemize}
        \tightlist
        \item
          \texttt{my-bundle-package\$isobject@2.1.0/}

          \begin{itemize}
          \tightlist
          \item
            \texttt{package.json}

            \begin{itemize}
            \tightlist
            \item
              name: my-bundle-package\$isobject
            \item
              version: 2.1.0
            \item
              main: lib/index
            \item
              dependencies:

              \begin{itemize}
              \tightlist
              \item
                my-bundle-package\$isarray: 1.0.0
              \end{itemize}
            \item
              \ldots{}
            \end{itemize}
          \item
            \ldots{}
          \end{itemize}
        \item
          \texttt{my-bundle-package\$isarray@1.0.0/}

          \begin{itemize}
          \tightlist
          \item
            \texttt{package.json}

            \begin{itemize}
            \tightlist
            \item
              name: my-bundle-package\$isarray
            \item
              version: 1.0.0
            \item
              \ldots{}
            \end{itemize}
          \item
            \ldots{}
          \end{itemize}
        \item
          \texttt{my-bundle-package\$isarray@2.0.0/}

          \begin{itemize}
          \tightlist
          \item
            \texttt{package.json}

            \begin{itemize}
            \tightlist
            \item
              name: my-bundle-package\$isarray
            \item
              version: 2.0.0
            \item
              \ldots{}
            \end{itemize}
          \item
            \ldots{}
          \end{itemize}
        \end{itemize}
      \end{itemize}
    \end{itemize}
  \end{itemize}
\end{itemize}

The packages inside \texttt{node\_modules} are the same format as the
npm tool and can be copied (after a little processing for things like
converting to AMD, for example) from a standard \texttt{node\_modules}
folder. The \texttt{node\_modules} folder can hold any number of npm
packages (even different versions of the same package), or no npm
packages at all.

Now that you know the structure for OSGi bundles containing npm
packages, you can learn how the liferay-npm-bundler handles inline
JavaScript packages.

\subsection{Inline JavaScript
packages}\label{inline-javascript-packages}

The resulting OSGi bundle that the liferay-npm-bundler creates lets you
deploy one inline JavaScript package (named \texttt{my-bundle-package}
in the example) with several npm packages that are placed inside the
\texttt{node\_modules} folder, one package per folder.

The inline package is nested in the OSGi standard
\texttt{META-INF/resources} folder and is defined by a standard npm
\texttt{package.json} file.

The inline package is optional, but only one inline package is allowed
per OSGi bundle. The inline package usually provides the JavaScript code
for a portlet, when the OSGi bundle contains one. Note that the
architecture does not differentiate between inline and npm packages once
they are published. The inline package is only used for organizational
purposes.

Now you know how the liferay-npm-bundler creates OSGi bundles for npm
packages!

\section{How the Liferay npm Bundler Publishes npm
Packages}\label{how-the-liferay-npm-bundler-publishes-npm-packages}

When you deploy an OSGi bundle with the specified structure, as
explained in
\href{/docs/7-1/reference/-/knowledge_base/r/the-structure-of-osgi-bundles-containing-npm-packages}{The
Structure of OSGi Bundles Containing NPM Packages} reference, its
modules are made available for consumption through canonical URLs. To
better illustrate resolved modules, the example structure below is the
standard structure that the liferay-npm-bundler 1.x generates, and
therefore doesn't have the namespaced packages that the 2.x version
generates. Please refer to the last sections of this article to know how
liferay-npm-bundler 2.0 overrides this de-duplication mechanism to
implement isolated dependencies and imports.

\begin{itemize}
\tightlist
\item
  \texttt{my-bundle/}

  \begin{itemize}
  \tightlist
  \item
    \texttt{META-INF/}

    \begin{itemize}
    \tightlist
    \item
      \texttt{resources/}

      \begin{itemize}
      \tightlist
      \item
        \texttt{package.json}

        \begin{itemize}
        \tightlist
        \item
          name: my-bundle-package
        \item
          version: 1.0.0
        \item
          main: lib/index
        \item
          dependencies:

          \begin{itemize}
          \tightlist
          \item
            isarray: 2.0.0
          \item
            isobject: 2.1.0
          \end{itemize}
        \item
          \ldots{}
        \end{itemize}
      \item
        \texttt{lib/}

        \begin{itemize}
        \tightlist
        \item
          \texttt{index.js}
        \item
          \ldots{}
        \end{itemize}
      \item
        \ldots{}
      \item
        \texttt{node\_modules/}

        \begin{itemize}
        \tightlist
        \item
          \texttt{isobject@2.1.0/}

          \begin{itemize}
          \tightlist
          \item
            \texttt{package.json}

            \begin{itemize}
            \tightlist
            \item
              name: isobject
            \item
              version: 2.1.0
            \item
              main: lib/index
            \item
              dependencies:

              \begin{itemize}
              \tightlist
              \item
                isarray: 1.0.0
              \end{itemize}
            \item
              \ldots{}
            \end{itemize}
          \item
            \ldots{}
          \end{itemize}
        \item
          \texttt{isarray@1.0.0/}

          \begin{itemize}
          \tightlist
          \item
            \texttt{package.json}

            \begin{itemize}
            \tightlist
            \item
              name: isarray
            \item
              version: 1.0.0
            \item
              \ldots{}
            \end{itemize}
          \item
            \ldots{}
          \end{itemize}
        \item
          \texttt{isarray@2.0.0/}

          \begin{itemize}
          \tightlist
          \item
            \texttt{package.json}

            \begin{itemize}
            \tightlist
            \item
              name: isarray
            \item
              version: 2.0.0
            \item
              \ldots{}
            \end{itemize}
          \item
            \ldots{}
          \end{itemize}
        \end{itemize}
      \end{itemize}
    \end{itemize}
  \end{itemize}
\end{itemize}

If you deploy the example OSGi bundle shown above, the following URLs
are made available (one for each module):

\begin{itemize}
\item
  \url{http://localhost/o/js/module/598/my-bundle-package@1.0.0/lib/index.js}
\item
  \url{http://localhost/o/js/module/598/isobject@2.1.0/index.js}
\item
  \url{http://localhost/o/js/module/598/isarray@1.0.0/index.js}
\item
  \url{http://localhost/o/js/module/598/isarray@2.0.0/index.js}
\end{itemize}

\noindent\hrulefill

\textbf{NOTE:} The OSGi bundle ID (598) may vary.

\noindent\hrulefill

You can learn about package de-duplication next.

\subsection{Package De-duplication}\label{package-de-duplication}

Since two or more OSGi modules may export multiple copies of the same
package and version, Liferay Portal must de-duplicate such collisions.
To accomplish de-duplication, a new concept called \emph{resolved
module} was created.

A resolved module is the reference package exported to Liferay Portal's
front-end, when multiple copies of the same package and version exist.
It's randomly referenced from one of the several bundles exporting the
same copies of the package.

Using the example from the previous section, for each group of canonical
URLs referring to the same module inside different OSGi bundles, there's
another canonical URL for the resolved module. The example structure has
the resolved module URLs shown below:

\begin{itemize}
\item
  \url{http://localhost/o/js/resolved-module/my-bundle-package@1.0.0/lib/index.js}
\item
  {[}http://localhost/o/js/resolved-module/my-bundle-package\(isobject@2.1.0/index.js](http://localhost/o/js/resolved-module/my-bundle-package\)isobject@2.1.0/index.js)
\item
  {[}http://localhost/o/js/resolved-module/my-bundle-package\(isarray@1.0.0/index.js](http://localhost/o/js/resolved-module/my-bundle-package\)isarray@1.0.0/index.js)
\item
  {[}http://localhost/o/js/resolved-module/my-bundle-package\(isarray@2.0.0/index.js](http://localhost/o/js/resolved-module/my-bundle-package\)isarray@2.0.0/index.js)
\end{itemize}

\noindent\hrulefill

\textbf{NOTE:} The OSGi bundle ID (598 in the example) is removed and
module is replaced by \texttt{resolved-module}.

\noindent\hrulefill

Next you can learn how the bundler (since version 2.0.0) isolates
package dependencies. See
\href{/docs/7-1/reference/-/knowledge_base/r/changes-between-liferay-npm-bundler-1x-and-2x}{What
Changed Between liferay-npm-bundler 1.x and 2.x} for more information on
why this change was made.

\subsection{Isolated Package
Dependencies}\label{isolated-package-dependencies}

A typical OSGi bundle structure generated with liferay-npm-bundler 2.x
is shown below:

\begin{itemize}
\tightlist
\item
  \texttt{my-bundle/}

  \begin{itemize}
  \tightlist
  \item
    \texttt{META-INF/}

    \begin{itemize}
    \tightlist
    \item
      \texttt{resources/}

      \begin{itemize}
      \tightlist
      \item
        \texttt{package.json}

        \begin{itemize}
        \tightlist
        \item
          name: my-bundle-package
        \item
          version: 1.0.0
        \item
          main: lib/index
        \item
          dependencies:

          \begin{itemize}
          \tightlist
          \item
            my-bundle-package\$isarray: 2.0.0
          \item
            my-bundle-package\$isobject: 2.1.0
          \end{itemize}
        \item
          \ldots{}
        \end{itemize}
      \item
        \texttt{lib/}

        \begin{itemize}
        \tightlist
        \item
          \texttt{index.js}
        \item
          \ldots{}
        \end{itemize}
      \item
        \ldots{}
      \item
        \texttt{node\_modules/}

        \begin{itemize}
        \tightlist
        \item
          \texttt{my-bundle-package\$isobject@2.1.0/}

          \begin{itemize}
          \tightlist
          \item
            \texttt{package.json}

            \begin{itemize}
            \tightlist
            \item
              name: my-bundle-package\$isobject
            \item
              version: 2.1.0
            \item
              main: lib/index
            \item
              dependencies:

              \begin{itemize}
              \tightlist
              \item
                my-bundle-package\$isarray: 1.0.0
              \end{itemize}
            \item
              \ldots{}
            \end{itemize}
          \item
            \ldots{}
          \end{itemize}
        \item
          \texttt{my-bundle-package\$isarray@1.0.0/}

          \begin{itemize}
          \tightlist
          \item
            \texttt{package.json}

            \begin{itemize}
            \tightlist
            \item
              name: my-bundle-package\$isarray
            \item
              version: 1.0.0
            \item
              \ldots{}
            \end{itemize}
          \item
            \ldots{}
          \end{itemize}
        \item
          \texttt{my-bundle-package\$isarray@2.0.0/}

          \begin{itemize}
          \tightlist
          \item
            \texttt{package.json}

            \begin{itemize}
            \tightlist
            \item
              name: my-bundle-package\$isarray
            \item
              version: 2.0.0
            \item
              \ldots{}
            \end{itemize}
          \item
            \ldots{}
          \end{itemize}
        \end{itemize}
      \end{itemize}
    \end{itemize}
  \end{itemize}
\end{itemize}

Note that each package dependency is namespaced with the bundle's name
(\texttt{my-bundle-package\$} in the example structure). This lets each
project load its own dependencies and avoid potential collisions with
projects that export the same package. For example, consider the two
portlet projects below:

\begin{verbatim}
- `my-portlet`
    - package.json
        - dependencies:
            - a-library 1.0.0
            - a-helper 1.0.0
    - node_modules
        - a-library
            - version: 1.0.0
            - dependencies:
                - a-helper ^1.0.0
        - a-helper
            - version: 1.0.0

- `another-portlet`
    - package.json
        - dependencies:
            - a-library 1.0.0
            - a-helper 1.2.0
    - node_modules
        - a-library
            - version: 1.0.0
            - dependencies:
                - a-helper ^1.0.0
        - a-helper
            - version: 1.2.0
\end{verbatim}

In this example, \texttt{a-library} depends on \texttt{a-helper} at
version 1.0.0 or higher (note the caret \^{} expression in the
dependencies). The bundler implements isolated dependencies by prefixing
the name of the bundle to the modules, so that \texttt{my-portlet} gets
its \texttt{a-helper} at 1.0.0, while \texttt{another-portlet} gets its
\texttt{a-helper} at 1.2.0.

The dependencies isolation not only avoids collisions between bundles,
but also makes peer dependencies behave deterministically as each
portlet gets what it had in its \texttt{node\_modules} folder when it
was developed.

Now that you understand how namespacing modules isolates bundle
dependencies, avoiding collisions, you can learn about de-duplication
next.

\subsection{De-duplication through
Importing}\label{de-duplication-through-importing}

Isolated dependencies are very useful, but there are times when sharing
the same package between modules would be more beneficial. To do this,
the liferay-npm-bundler lets you import packages from an external OSGi
bundle, instead of using your own. This lets you put shared dependencies
in one project and reference them from the rest.

Imagine that you have three portlets that compose the homepage of your
site: \texttt{my-toolbar}, \texttt{my-menu}, and \texttt{my-content}.
These portlets depend on the fake, but awesome, Wonderful UI Components
(WUI) framework. This quite limited framework is composed of only three
packages:

\begin{enumerate}
\def\labelenumi{\arabic{enumi}.}
\tightlist
\item
  \texttt{component-core}
\item
  \texttt{button}
\item
  \texttt{textfield}
\end{enumerate}

Since the bundler namespaces each dependency package with the portlet's
name by default, you would end up with three namespaced copies of the
WUI package on the page. This is not what you want. Since they share the
same package, instead you can create a fourth bundle that contains the
WUI package, and import the WUI package in the three portlets. This
results in the structure below:

\begin{itemize}
\tightlist
\item
  \texttt{my-toolbar/}

  \begin{itemize}
  \tightlist
  \item
    \texttt{.npmbundlerrc}

    \begin{itemize}
    \tightlist
    \item
      config:

      \begin{itemize}
      \tightlist
      \item
        imports:

        \begin{itemize}
        \tightlist
        \item
          wui-provider:

          \begin{itemize}
          \tightlist
          \item
            component-core: \^{}1.0.0
          \item
            button: \^{}1.0.0
          \item
            textfield: \^{}1.0.0
          \end{itemize}
        \end{itemize}
      \end{itemize}
    \end{itemize}
  \end{itemize}
\item
  \texttt{my-menu/}

  \begin{itemize}
  \tightlist
  \item
    \texttt{.npmbundlerrc}

    \begin{itemize}
    \tightlist
    \item
      config:

      \begin{itemize}
      \tightlist
      \item
        imports:

        \begin{itemize}
        \tightlist
        \item
          wui-provider:

          \begin{itemize}
          \tightlist
          \item
            component-core: \^{}1.0.0
          \item
            button: \^{}1.0.0
          \item
            textfield: \^{}1.0.0
          \end{itemize}
        \end{itemize}
      \end{itemize}
    \end{itemize}
  \end{itemize}
\item
  \texttt{my-content/}

  \begin{itemize}
  \tightlist
  \item
    \texttt{.npmbundlerrc}

    \begin{itemize}
    \tightlist
    \item
      config:

      \begin{itemize}
      \tightlist
      \item
        imports:

        \begin{itemize}
        \tightlist
        \item
          wui-provider:

          \begin{itemize}
          \tightlist
          \item
            component-core: \^{}1.0.0
          \item
            button: \^{}1.0.0
          \item
            textfield: \^{}1.0.0
          \end{itemize}
        \end{itemize}
      \end{itemize}
    \end{itemize}
  \end{itemize}
\item
  \texttt{wui-provider/}

  \begin{itemize}
  \tightlist
  \item
    \texttt{.package.json}

    \begin{itemize}
    \tightlist
    \item
      name: wui-provider
    \item
      dependencies:

      \begin{itemize}
      \tightlist
      \item
        component-core: 1.0.0
      \item
        button: 1.0.0
      \item
        textfield: 1.0.0
      \end{itemize}
    \end{itemize}
  \end{itemize}
\end{itemize}

The bundler switches the namespace of certain packages, thus pointing
them to an external bundle. Say that you have the following code in
\texttt{my-toolbar} portlet:

\begin{verbatim}
var Button = require('button');
\end{verbatim}

By default, the bundler 2.x transforms this into the following when not
imported from another bundle:

\begin{verbatim}
var Button = require('my-toolbar$button');
\end{verbatim}

But, because \texttt{button} is imported from \texttt{wui-provider}, it
is instead changed to the value below:

\begin{verbatim}
var Button = require('wui-provider$button');
\end{verbatim}

Also, a dependency on \texttt{wui-provider\$button} at version
\texttt{\^{}1.0.0} is included in \texttt{my-toolbar}'s
\texttt{package.json} file so that the loader finds the correct version.
That's all you need. Once \texttt{wui-provider\$button} is required at
runtime, it jumps to \texttt{wui-provider}'s context and loads the
subdependencies from there on, even if code is executed from
\texttt{my-toolbar}. This works because, as you can imagine,
\texttt{wui-provider}'s modules are namespaced too, and once you load a
module from it, it keeps requiring \texttt{wui-provider\$} prefixed
modules all the way down.

Next, you will learn possible strategies for importing.

\subsection{Strategies When Importing
Packages}\label{strategies-when-importing-packages}

De-duplication by importing is a powerful tool, but you must design a
versioning strategy suitable for you so that you don't run into errors.

First of all, you must decide if you want to declare imported
dependencies only in the \texttt{.npmbundlerrc} file or in the
\texttt{package.json} too. Listing an imported dependency in
\texttt{.npmbundlerrc} is enough, even if it isn't present in your
\texttt{node\_modules} folder because during runtime the loader will
find it. Listing an imported dependency in \texttt{.npmbundlerrc} is
enough, even if it isn't present in your \texttt{node\_modules} folder,
because during runtime the loader finds it. If you have previous
experience with dynamic linking support in standard operating systems,
think of it as a DLL or shared object.

You may need to install your dependencies in \texttt{node\_modules} too
if you use them for tests, or if they contain types needed to compile
(like in Typescript), etc. If that is the case, then you can place them
in the \texttt{dependencies} or \texttt{devDependencies} section of your
\texttt{package.json}. If you list them under the latter, they are
automatically excluded from the output bundle by the
liferay-npm-bundler. Otherwise, you need to exclude them in the
\texttt{.npmbundlerrc} file so they don't redundantly appear in the
output.

If you list dependencies both in \texttt{package.json} and
\texttt{.npmbundlerrc}, decide how to keep versions in sync. The best
advice is to use the same version constraints in both files, but you may
decide not to do so if it is necessary. For example, imagine that you
import one of your dependencies from another bundle during runtime to
run tests. Say you are using version constraint \^{}1.5.1. It would be
desirable that if you have tested your code with a version
\textgreater=1.5.1 and \textless2.0.0 (that's what \^{}1.5.1 means), you
get a compatible version during runtime. Thus, you would declare the
dependency with \^{}1.5.1 in \texttt{.npmbundlerrc} too.

However, there are times when you may want to be more lenient, and you
may need to get a lower version (1.4.0 for example) during runtime, even
if you are developing against \^{}1.5.1. In that case, you can declare
\^{}1.5.1 in your \texttt{package.json} and \^{}1.0.0 in
\texttt{.npmbundlerrc}.

In the end, it's up to you to decide how you want to handle your
dependencies:

\begin{enumerate}
\def\labelenumi{\arabic{enumi}.}
\item
  \texttt{package.json} (While developing)
\item
  \texttt{.npmbundlerrc} (During runtime)
\end{enumerate}

we recommend that you choose a versioning strategy and stick to it, to
ensure dependencies are satisfied at runtime.

\section{Understanding How liferay-npm-bundler Formats JavaScript
Modules for
AMD}\label{understanding-how-liferay-npm-bundler-formats-javascript-modules-for-amd}

Liferay AMD Loader is based on the
\href{https://github.com/amdjs/amdjs-api/wiki/AMD}{AMD specification}.
All modules inside an npm OSGi bundle must be in AMD format. This is
done for \href{http://www.commonjs.org/}{CommonJS} modules by wrapping
the module code inside a \texttt{define} call. The liferay-npm-bundler
helps automate this process by wrapping the module for you. This article
references the OSGi structure below as an example. You can learn more
about this structure in
\href{/docs/7-1/reference/-/knowledge_base/r/the-structure-of-osgi-bundles-containing-npm-packages}{The
Structure of OSGi Bundles Containing NPM Packages} reference.

\begin{itemize}
\tightlist
\item
  \texttt{my-bundle/}

  \begin{itemize}
  \tightlist
  \item
    \texttt{META-INF/}

    \begin{itemize}
    \tightlist
    \item
      \texttt{resources/}

      \begin{itemize}
      \tightlist
      \item
        \texttt{package.json}

        \begin{itemize}
        \tightlist
        \item
          name: my-bundle-package
        \item
          version: 1.0.0
        \item
          main: lib/index
        \item
          dependencies:

          \begin{itemize}
          \tightlist
          \item
            my-bundle-package\$isarray: 2.0.0
          \item
            my-bundle-package\$isobject: 2.1.0
          \end{itemize}
        \item
          \ldots{}
        \end{itemize}
      \item
        \texttt{lib/}

        \begin{itemize}
        \tightlist
        \item
          \texttt{index.js}
        \item
          \ldots{}
        \end{itemize}
      \item
        \ldots{}
      \item
        \texttt{node\_modules/}

        \begin{itemize}
        \tightlist
        \item
          \texttt{my-bundle-package\$isobject@2.1.0/}

          \begin{itemize}
          \tightlist
          \item
            \texttt{package.json}

            \begin{itemize}
            \tightlist
            \item
              name: my-bundle-package\$isobject
            \item
              version: 2.1.0
            \item
              main: lib/index
            \item
              dependencies:

              \begin{itemize}
              \tightlist
              \item
                my-bundle-package\$isarray: 1.0.0
              \end{itemize}
            \item
              \ldots{}
            \end{itemize}
          \item
            \ldots{}
          \end{itemize}
        \item
          \texttt{my-bundle-package\$isarray@1.0.0/}

          \begin{itemize}
          \tightlist
          \item
            \texttt{package.json}

            \begin{itemize}
            \tightlist
            \item
              name: my-bundle-package\$isarray
            \item
              version: 1.0.0
            \item
              \ldots{}
            \end{itemize}
          \item
            \ldots{}
          \end{itemize}
        \item
          \texttt{my-bundle-package\$isarray@2.0.0/}

          \begin{itemize}
          \tightlist
          \item
            \texttt{package.json}

            \begin{itemize}
            \tightlist
            \item
              name: my-bundle-package\$isarray
            \item
              version: 2.0.0
            \item
              \ldots{}
            \end{itemize}
          \item
            \ldots{}
          \end{itemize}
        \end{itemize}
      \end{itemize}
    \end{itemize}
  \end{itemize}
\end{itemize}

For example, the \texttt{my-bundle-package\$isobject@2.1.0} package's
\texttt{index.js} file contains the following code:

\begin{verbatim}
'use strict';

var isArray = require('my-bundle-package$isarray');

module.exports = function isObject(val) {
    return val != null && typeof val === 'object' && isArray(val) === false;
};
\end{verbatim}

The updated module code configured for AMD format is shown below:

\begin{verbatim}
define(
    'my-bundle-package$isobject@2.1.0/index', 
    ['module', 'require', 'my-bundle-package$isarray'], 
    function (module, require) {
        'use strict';

        var define = undefined;

        var isArray = require('my-bundle-package$isarray');

        module.exports = function isObject(val) {
            return val != null && typeof val === 'object' 
            && isArray(val) === false;
        };
    }
);
\end{verbatim}

\noindent\hrulefill

\textbf{Note:} The module's name must be based on its package, version,
and file path (for example
\texttt{my-bundle-package\$isobject@2.1.0/index}), otherwise Liferay AMD
Loader can't find it.

\noindent\hrulefill

Note the module's dependencies:
\texttt{{[}\textquotesingle{}module\textquotesingle{},\ \textquotesingle{}require\textquotesingle{},\ \textquotesingle{}my-bundle-package\$isarray\textquotesingle{}{]}}.

\texttt{module} and \texttt{require} must be used to get a reference to
the \texttt{module.exports} object and the local \texttt{require}
function, as defined in the AMD specification.

The subsequent dependencies state the modules on which this module
depends. Note that \texttt{my-bundle-package\$isarray} in the example is
not a package but rather an alias of the
\texttt{my-bundle-package\$isarray} package's main module (thus, it is
equivalent to \texttt{my-bundle-package\$isarray/index}).

Also note that there is enough information in the \texttt{package.json}
files to know that \texttt{my-bundle-package\$isarray} refers to
\texttt{my-bundle-package\$isarray/index}, but also that it must be
resolved to version \texttt{1.0.0} of such package, i.e., that
\texttt{my-bundle-package\$isarray/index} in this case refers to
\texttt{my-bundle-package\$isarray@1.0.0/index}.

You may also have noted the \texttt{var\ define\ =\ undefined;} addition
to the top of the file. This is introduced by
\texttt{liferay-npm-bundler} to make the module think that it is inside
a CommonJS environment (instead of an AMD one). This is because some npm
packages are written in UMD format and, because we are wrapping it
inside our AMD \texttt{define()} call, we don't want them to execute
their own \texttt{define()} but prefer them to take the CommonJS path,
where the exports are done through the \texttt{module.exports} global.

Now you have a better understanding of how liferay-npm-bundler formats
JavaScript modules for AMD!

\section{Understanding How Liferay AMD Loader Configuration is
Exported}\label{understanding-how-liferay-amd-loader-configuration-is-exported}

\textbf{NOTE:} This article is for users who know how Liferay AMD Loader
works under the hood. You can learn more about Liferay AMD Loader in the
\href{/docs/7-1/tutorials/-/knowledge_base/t/loading-amd-modules-in-liferay}{Liferay
AMD Module Loader} tutorial.

With
\href{/docs/7-1/reference/-/knowledge_base/r/how-liferay-portal-publishes-npm-packages\#package-de-duplication}{de-duplication}
in place, JavaScript modules are made available to Liferay AMD Loader
through the configuration returned by the
\texttt{/o/js\_loader\_modules} URL.

The OSGi bundle shown below is used for reference in this article:

\begin{itemize}
\tightlist
\item
  \texttt{my-bundle/}

  \begin{itemize}
  \tightlist
  \item
    \texttt{META-INF/}

    \begin{itemize}
    \tightlist
    \item
      \texttt{resources/}

      \begin{itemize}
      \tightlist
      \item
        \texttt{package.json}

        \begin{itemize}
        \tightlist
        \item
          name: my-bundle-package
        \item
          version: 1.0.0
        \item
          main: lib/index
        \item
          dependencies:

          \begin{itemize}
          \tightlist
          \item
            isarray: 2.0.0
          \item
            isobject: 2.1.0
          \end{itemize}
        \item
          \ldots{}
        \end{itemize}
      \item
        \texttt{lib/}

        \begin{itemize}
        \tightlist
        \item
          \texttt{index.js}
        \item
          \ldots{}
        \end{itemize}
      \item
        \ldots{}
      \item
        \texttt{node\_modules/}

        \begin{itemize}
        \tightlist
        \item
          \texttt{isobject@2.1.0/}

          \begin{itemize}
          \tightlist
          \item
            \texttt{package.json}

            \begin{itemize}
            \tightlist
            \item
              name: isobject
            \item
              version: 2.1.0
            \item
              main: lib/index
            \item
              dependencies:

              \begin{itemize}
              \tightlist
              \item
                isarray: 1.0.0
              \end{itemize}
            \item
              \ldots{}
            \end{itemize}
          \item
            \ldots{}
          \end{itemize}
        \item
          \texttt{isarray@1.0.0/}

          \begin{itemize}
          \tightlist
          \item
            \texttt{package.json}

            \begin{itemize}
            \tightlist
            \item
              name: isarray
            \item
              version: 1.0.0
            \item
              \ldots{}
            \end{itemize}
          \item
            \ldots{}
          \end{itemize}
        \item
          \texttt{isarray@2.0.0/}

          \begin{itemize}
          \tightlist
          \item
            \texttt{package.json}

            \begin{itemize}
            \tightlist
            \item
              name: isarray
            \item
              version: 2.0.0
            \item
              \ldots{}
            \end{itemize}
          \item
            \ldots{}
          \end{itemize}
        \end{itemize}
      \end{itemize}
    \end{itemize}
  \end{itemize}
\end{itemize}

For example, for the specified structure (shown above), as explained in
\href{/docs/7-1/reference/-/knowledge_base/r/the-structure-of-osgi-bundles-containing-npm-packages}{The
Structure of OSGi Bundles Containing npm Packages} reference, the
following configuration is published for Liferay AMD loader to consume:

\begin{verbatim}
Liferay.PATHS = {
  ...
  'my-bundle-package@1.0.0/lib/index': '/o/js/resolved-module/my-bundle-package@1.0.0/lib/index',
  'isobject@2.1.0/index': '/o/js/resolved-module/isobject@2.1.0/index',
  'isarray@1.0.0/index': '/o/js/resolved-module/isarray@1.0.0/index',
  'isarray@2.0.0/index': '/o/js/resolved-module/isarray@2.0.0/index',
  ...
}
Liferay.MODULES = {
  ...
  "my-bundle-package@1.0.0/lib/index.es": {
    "dependencies": ["exports", "isarray", "isobject"],
    "map": {
      "isarray": "isarray@2.0.0", 
      "isobject": "isobject@2.1.0"
    }
  },
  "isobject@2.1.0/index": {
    "dependencies": ["module", "require", "isarray"],
    "map": {
      "isarray": "isarray@1.0.0"
    }
  },
  "isarray@1.0.0/index": {
    "dependencies": ["module", "require"],
    "map": {}
  },
  "isarray@2.0.0/index": {
    "dependencies": ["module", "require"],
    "map": {}
  },
  ...
}
Liferay.MAPS = {
  ...
  'my-bundle-package@1.0.0': { value: 'my-bundle-package@1.0.0/lib/index', exactMatch: true}
  'isobject@2.1.0': { value: 'isobject@2.1.0/index', exactMatch: true},
  'isarray@2.0.0': { value: 'isarray@2.0.0/index', exactMatch: true},
  'isarray@1.0.0': { value: 'isarray@1.0.0/index', exactMatch: true},
  ...
}
\end{verbatim}

Note:

\begin{itemize}
\item
  The \texttt{Liferay.PATHS} property describes paths to the JavaScript
  module files.
\item
  The \texttt{Liferay.MODULES} property describes the dependency names
  and versions of each module.
\item
  The \texttt{Liferay.MAPS} property describes the aliases of the
  package's main modules.
\end{itemize}

\section{What Changed Between Liferay npm Bundler 1.x and
2.x}\label{what-changed-between-liferay-npm-bundler-1.x-and-2.x}

This reference doc outlines the key changes between liferay-npm-bundler
version 1.x and 2.x.

\subsection{Automatically Formatting Modules for
AMD}\label{automatically-formatting-modules-for-amd}

In version series 1.x of the bundler it was the developer's
responsibility to wrap project modules in an AMD \texttt{define()} call.
However, since 2.x the bundler does it for you, so the only requisite is
that the project's code is transpiled/written for CommonJS modules model
(the standard model for module handling in Node.js, that uses
\texttt{require()} calls to load modules).

\subsection{Isolating Project
Dependencies}\label{isolating-project-dependencies}

Package names are prefixed with the bundle name since version 2.0.0 of
the bundler, but were left intact in previous versions. This strategy is
used to isolate packages from different bundles. You can still deploy
bundler 1.x packages (without prefix), and they will still work as they
did for previous versions of the bundler.

\subsection{Improved Peer Dependency
Support}\label{improved-peer-dependency-support}

In bundler 1.x, there was only one shared peer dependency package
available between portlets. With isolated dependencies per portlet, it's
easy to honor peer dependencies perfectly. Peer dependencies can be
resolved exactly as stated in projects because their names are prefixed
with the project's name. This is possible because of the new
\href{https://github.com/liferay/liferay-npm-build-tools/tree/master/packages/liferay-npm-bundler-plugin-inject-peer-dependencies}{liferay-npm-bundler-plugin-inject-peer-dependencies}
plugin. It scans all JS modules for \texttt{require} calls. If the
bundler finds a required package in the \texttt{main.js} file, but it is
not declared in the \texttt{package.json}, it resolves it to the proper
version that is found in the \texttt{node\_modules} folder. The plugin
then injects a new dependency in the output \texttt{package.json} for
the required package.

Note that injected dependency version constraints are the specific
version number required, without caret or any other semantic version
operator. This is to honor the exact peer dependency found in the
project. Injecting more relaxed semantic version expressions could lead
to unstable results.

\subsection{Manually De-duplicating Through
Importing}\label{manually-de-duplicating-through-importing}

Namespacing means that each portlet gets its own dependencies. Only
using the bundler this way obtains the same functionality as standard
bundlers like webpack or Browserify, so you wouldn't need a specific
tool like liferay-npm-bundler. Since Liferay DXP is a portlet based
architecture, sharing dependencies among different portlets would be
very beneficial.

In bundler 1.x that deduplication was made automatically, but there was
no control over it. However, with version 2.x, you may now import
packages from an external OSGi bundle, instead of using your own. This
lets you put shared dependencies in one project, and reference them from
the rest. Though This new way of de-duplication is not automatic, it
leads to full control (during build time) of how each package is
resolved.

Now that you understand what changed between version 1.x and 2.x of the
liferay-npm-bundler, you can follow the steps in the
\href{/docs/7-1/tutorials/-/knowledge_base/t/migrating-a-liferay-npm-bundler-project-from-1-x-to-2-x}{Migrating
a liferay-npm-bundler Project from 1.x to 2.x} tutorial to migrate your
1.x projects to 2.x.

\section{Understanding liferay-npm-bundler's
Loaders}\label{understanding-liferay-npm-bundlers-loaders}

liferay-npm-bundler's mechanism is inspired by webpack. Like webpack,
the liferay-npm-bundler processes files with a set of rules, which
includes loaders that transform a project's source files before
producing the final output.

\noindent\hrulefill

\textbf{Note:} While webpack creates a single JS bundle file,
liferay-npm-bundler targets an AMD loader, so webpack and
liferay-npm-bundler loaders are not compatible.

\noindent\hrulefill

Loaders are npm packages that export a function in their main module,
which receives source files and returns modified files, and optionally
new files, based on the loader's configuration. For example, the
\href{https://github.com/liferay/liferay-js-toolkit/tree/master/packages/liferay-npm-bundler-loader-babel-loader}{babel-loader}
receives ES6+ JavaScript files, runs Babel on them, and returns
transpiled ES5 files along with a generated source map. You can use this
simple pattern to
\href{/docs/7-1/tutorials/-/knowledge_base/t/creating-custom-loaders-for-the-liferay-npm-bundler}{create
custom loaders}. A few example loader functions are shown below:

\begin{itemize}
\tightlist
\item
  Pass JS files through Babel or TSC
\item
  Convert CSS files into JS modules that dynamically inject the CSS into
  the HTML page
\item
  Process CSS files with SASS
\item
  Create tools that generate code based on
  \href{https://en.wikipedia.org/wiki/Interface_description_language}{IDL}
  files
\end{itemize}

Loaders are configured via the project's \texttt{.npmbundlerrc} file. A
loader's configuration is specified with two key options:
\texttt{sources} (the folders that contain the sources files to process)
and \texttt{rules} ( the loaders, options (if applicable), and regular
expressions that determine which files to process). See
\href{/docs/7-1/reference/-/knowledge_base/r/configuring-liferay-npm-bundler\#package-processing-options}{Understanding
the \texttt{.npmbundlerrc}'s Structure} for more information on the
configuration requirements and options.

Loaders can be chained. Files are processed by the loaders in the order
in which they are listed in the \texttt{use} property. The files are
passed to the first loader, processed, sent to the next loader, and so
on and so forth, until the files are processed by the rules. This lets
you run complex processes, such as converting a SASS file into CSS with
the sass-loader, and then convert it into a JavaScript module with the
style-loader. Once the rules are applied, the liferay-npm-bundler
continues with the pre, post, and babel phases of the bundler plugins.

\section{Default liferay-npm-bundler
Loaders}\label{default-liferay-npm-bundler-loaders}

Several
\href{/docs/7-1/reference/-/knowledge_base/r/understanding-liferay-npm-bundlers-loaders}{loaders}
are available for the liferay-npm-bundler by default. These loaders are
listed below:

\begin{itemize}
\item
  \href{https://github.com/liferay/liferay-js-toolkit/tree/master/packages/liferay-npm-bundler-loader-babel-loader}{\texttt{babel-loader}}:
  processes source files with \href{https://babeljs.io/}{Babel}. This
  avoids an extra build step before the bundler.
\item
  \href{https://github.com/liferay/liferay-js-toolkit/tree/master/packages/liferay-npm-bundler-loader-copy-loader}{\texttt{copy-loader}}:
  copies source files (static assets) to the output folder.
\item
  \href{https://github.com/liferay/liferay-js-toolkit/tree/master/packages/liferay-npm-bundler-loader-css-loader}{\texttt{css-loader}}:
  converts a CSS file into a JavaScript module that's inserted into the
  DOM once it's loaded.
\item
  \href{https://github.com/liferay/liferay-js-toolkit/tree/master/packages/liferay-npm-bundler-loader-json-loader}{\texttt{json-loader}}:
  generates JavaScript modules that export the contents of a JSON file
  as an object. This lets you include JSON files with the
  \texttt{require()} call.
\item
  \href{https://github.com/liferay/liferay-js-toolkit/tree/master/packages/liferay-npm-bundler-loader-sass-loader}{\texttt{sass-loader}}:
  runs \texttt{node-sass} or \texttt{sass} on source files. This lets
  you generate static CSS files. It can be chained before
  \texttt{style-loader}.
\item
  \href{https://github.com/liferay/liferay-js-toolkit/tree/master/packages/liferay-npm-bundler-loader-style-loader}{\texttt{style-loader}}:
  converts a CSS file into a JavaScript module that directly inserts the
  CSS contents into the DOM once it's loaded. This lets you include CSS
  files with a \texttt{require()} call.
\end{itemize}

See the
\href{https://github.com/izaera/liferay-js-toolkit-showcase/tree/loaders}{liferay-js-toolkit
loaders showcase} for an example use case of the liferay-npm-bundler's
loaders. If the default loaders don't meet your requirements, you can
follow the instructions in
\href{/docs/7-1/tutorials/-/knowledge_base/t/creating-custom-loaders-for-the-liferay-npm-bundler}{Creating
Custom Loaders for the Bundler} to create your own loaders.

\section{CKEditor Plugin Reference
Guide}\label{ckeditor-plugin-reference-guide}

This reference guide provides a list of the default CKEditor plugins
bundled with Liferay DXP's AlloyEditor. You can
\href{/docs/7-1/tutorials/-/knowledge_base/t/adding-buttons-to-alloyeditor-toolbars}{use
these existing CKEditor plugins in your custom AlloyEditor
configurations}. Each plugin below links to its \texttt{plugin.js} file
for reference, specifying the plugin's name and buttons if applicable:

\begin{itemize}
\tightlist
\item
  \href{https://github.com/ckeditor/ckeditor-dev/tree/master/plugins/about/plugin.js}{about}
\item
  \href{https://github.com/ckeditor/ckeditor-dev/tree/master/plugins/a11yhelp/plugin.js}{allyhelp}
\item
  \href{https://github.com/liferay/liferay-portal/tree/7.1.x/modules/apps/frontend-editor/frontend-editor-ckeditor-web/src/main/resources/META-INF/resources/_diffs/plugins/a11yhelpbtn/plugin.js}{allyhelpbtn}
\item
  \href{https://github.com/liferay/liferay-portal/tree/7.1.x/modules/apps/frontend-editor/frontend-editor-ckeditor-web/src/main/resources/META-INF/resources/_diffs/plugins/ajaxsave/plugin.js}{ajaxsave}
\item
  \href{https://github.com/liferay/liferay-portal/tree/7.1.x/modules/apps/frontend-editor/frontend-editor-ckeditor-web/src/main/resources/META-INF/resources/_diffs/plugins/autocomplete/plugin.js}{autocomplete}
\item
  \href{https://github.com/ckeditor/ckeditor-dev/tree/master/plugins/basicstyles/plugin.js}{basicstyles}
\item
  \href{https://github.com/liferay/liferay-portal/tree/7.1.x/modules/apps/frontend-editor/frontend-editor-ckeditor-web/src/main/resources/META-INF/resources/_diffs/plugins/bbcode/plugin.js}{bbcode}
\item
  \href{https://github.com/ckeditor/ckeditor-dev/tree/master/plugins/bidi/plugin.js}{bidi}
\item
  \href{https://github.com/ckeditor/ckeditor-dev/tree/master/plugins/blockquote/plugin.js}{blockquote}
\item
  \href{https://github.com/ckeditor/ckeditor-dev/tree/master/plugins/clipboard/plugin.js}{clipboard}
\item
  \href{https://github.com/ckeditor/ckeditor-dev/tree/master/plugins/colorbutton/plugin.js}{colorbutton}
\item
  \href{https://github.com/ckeditor/ckeditor-dev/tree/master/plugins/colordialog/plugin.js}{colordialog}
\item
  \href{https://github.com/ckeditor/ckeditor-dev/blob/master/plugins/contextmenu/plugin.js}{contextmenu}
\item
  \href{https://github.com/liferay/liferay-portal/blob/7.1.x/modules/apps/frontend-editor/frontend-editor-ckeditor-web/src/main/resources/META-INF/resources/_diffs/plugins/creole/plugin.js}{creole}
\item
  \href{https://github.com/ckeditor/ckeditor-dev/blob/master/plugins/dialogadvtab/plugin.js}{dialogadvtab}
\item
  \href{https://github.com/ckeditor/ckeditor-dev/blob/master/plugins/div/plugin.js}{div}
\item
  \href{https://github.com/ckeditor/ckeditor-dev/blob/master/plugins/elementspath/plugin.js}{elementspath}
\item
  \href{https://github.com/ckeditor/ckeditor-dev/blob/master/plugins/enterkey/plugin.js}{enterkey}
\item
  \href{https://github.com/ckeditor/ckeditor-dev/blob/master/plugins/entities/plugin.js}{entities}
\item
  \href{https://github.com/ckeditor/ckeditor-dev/blob/master/plugins/filebrowser/plugin.js}{filebrowse}
\item
  \href{https://github.com/ckeditor/ckeditor-dev/blob/master/plugins/find/plugin.js}{find}
\item
  \href{https://github.com/ckeditor/ckeditor-dev/blob/master/plugins/flash/plugin.js}{flash}
\item
  \href{https://github.com/ckeditor/ckeditor-dev/blob/master/plugins/floatingspace/plugin.js}{floatingspace}
\item
  \href{https://github.com/ckeditor/ckeditor-dev/blob/master/plugins/font/plugin.js}{font}
\item
  \href{https://github.com/ckeditor/ckeditor-dev/blob/master/plugins/format/plugin.js}{format}
\item
  \href{https://github.com/ckeditor/ckeditor-dev/blob/master/plugins/forms/plugin.js}{forms}
\item
  \href{https://github.com/ckeditor/ckeditor-dev/blob/master/plugins/horizontalrule/plugin.js}{horizontalrule}
\item
  \href{https://github.com/ckeditor/ckeditor-dev/blob/master/plugins/htmlwriter/plugin.js}{htmlwriter}
\item
  \href{https://github.com/ckeditor/ckeditor-dev/blob/master/plugins/image/plugin.js}{image}
\item
  \href{https://github.com/ckeditor/ckeditor-dev/blob/master/plugins/iframe/plugin.js}{iframe}
\item
  \href{https://github.com/ckeditor/ckeditor-dev/blob/master/plugins/indent/plugin.js}{indent}
\item
  \href{https://github.com/liferay/liferay-portal/blob/7.1.x/modules/apps/frontend-editor/frontend-editor-ckeditor-web/src/main/resources/META-INF/resources/_diffs/plugins/itemselector/plugin.js}{itemselector}
\item
  \href{https://github.com/ckeditor/ckeditor-dev/blob/master/plugins/justify/plugin.js}{justify}
\item
  \href{https://github.com/ckeditor/ckeditor-dev/blob/master/plugins/link/plugin.js}{link}
\item
  \href{https://github.com/ckeditor/ckeditor-dev/blob/master/plugins/list/plugin.js}{list}
\item
  \href{https://github.com/ckeditor/ckeditor-dev/blob/master/plugins/liststyle/plugin.js}{liststyle}
\item
  \href{https://github.com/liferay/liferay-portal/blob/7.1.x/modules/apps/frontend-editor/frontend-editor-ckeditor-web/src/main/resources/META-INF/resources/_diffs/plugins/lfrpopup/plugin.js}{lfrpopup}
\item
  \href{https://github.com/ckeditor/ckeditor-dev/blob/master/plugins/magicline/plugin.js}{magicline}
\item
  \href{https://github.com/liferay/liferay-portal/blob/7.1.x/modules/apps/frontend-editor/frontend-editor-ckeditor-web/src/main/resources/META-INF/resources/_diffs/plugins/media/plugin.js}{media}
\item
  \href{https://github.com/ckeditor/ckeditor-dev/blob/master/plugins/newpage/plugin.js}{newpage}
\item
  \href{https://github.com/ckeditor/ckeditor-dev/blob/master/plugins/pagebreak/plugin.js}{pagebreak}
\item
  \href{https://github.com/ckeditor/ckeditor-dev/blob/master/plugins/pastefromword/plugin.js}{pastefromword}
\item
  \href{https://github.com/ckeditor/ckeditor-dev/blob/master/plugins/pastetext/plugin.js}{pastetext}
\item
  \href{https://github.com/ckeditor/ckeditor-dev/blob/master/plugins/preview/plugin.js}{preview}
\item
  \href{https://github.com/ckeditor/ckeditor-dev/blob/master/plugins/removeformat/plugin.js}{removeformat}
\item
  \href{https://github.com/ckeditor/ckeditor-dev/blob/master/plugins/resize/plugin.js}{resize}
\item
  \href{https://github.com/liferay/liferay-portal/blob/7.1.x/modules/apps/frontend-editor/frontend-editor-ckeditor-web/src/main/resources/META-INF/resources/_diffs/plugins/restore/plugin.js}{restore}
\item
  \href{https://github.com/ckeditor/ckeditor-dev/blob/master/plugins/selectall/plugin.js}{selectall}
\item
  \href{https://github.com/ckeditor/ckeditor-dev/blob/master/plugins/showblocks/plugin.js}{showblocks}
\item
  \href{https://github.com/ckeditor/ckeditor-dev/blob/master/plugins/showborders/plugin.js}{showborders}
\item
  \href{https://github.com/ckeditor/ckeditor-dev/blob/master/plugins/smiley/plugin.js}{smiley}
\item
  \href{https://github.com/ckeditor/ckeditor-dev/blob/master/plugins/sourcearea/plugin.js}{sourcearea}
\item
  \href{https://github.com/ckeditor/ckeditor-dev/blob/master/plugins/specialchar/plugin.js}{specialchar}
\item
  \href{https://github.com/ckeditor/ckeditor-dev/blob/master/plugins/stylescombo/plugin.js}{stylescombo}
\item
  \href{https://github.com/ckeditor/ckeditor-dev/blob/master/plugins/tab/plugin.js}{tab}
\item
  \href{https://github.com/ckeditor/ckeditor-dev/blob/master/plugins/table/plugin.js}{table}
\item
  \href{https://github.com/ckeditor/ckeditor-dev/blob/master/plugins/tabletools/plugin.js}{tabletools}
\item
  \href{https://github.com/ckeditor/ckeditor-dev/blob/master/plugins/templates/plugin.js}{templates}
\item
  \href{https://github.com/ckeditor/ckeditor-dev/blob/master/plugins/toolbar/plugin.js}{toolbar}
\item
  \href{https://github.com/ckeditor/ckeditor-dev/blob/master/plugins/undo/plugin.js}{undo}
\item
  \href{https://github.com/liferay/liferay-portal/blob/7.1.x/modules/apps/frontend-editor/frontend-editor-ckeditor-web/src/main/resources/META-INF/resources/_diffs/plugins/wikilink/plugin.js}{wikilink}
\item
  \href{https://github.com/ckeditor/ckeditor-dev/blob/master/plugins/wysiwygarea/plugin.js}{wysiwygarea}
\end{itemize}

\noindent\hrulefill

\textbf{Note:} The following CKEditor plugins are not available for
inline mode in AlloyEditor at this time, but you can still use them in
the classic CKEditor:

\begin{itemize}
\tightlist
\item
  \href{https://github.com/ckeditor/ckeditor-dev/blob/master/plugins/maximize/plugin.js}{maximize}
\item
  \href{https://github.com/ckeditor/ckeditor-dev/blob/master/plugins/print/plugin.js}{print}
\item
  \href{https://github.com/ckeditor/ckeditor-dev/blob/master/plugins/save/plugin.js}{save}
\end{itemize}

To use the Classic CKEditor instead of AlloyEditor, there are a few
properties to set, depending on the portlet. Add the
\href{https://github.com/liferay/liferay-portal/blob/7.1.x/portal-impl/src/portal.properties\#L5484-L5493}{properties}
that you need to your \texttt{portal-ext.properties} file:

\begin{verbatim}
 editor.wysiwyg.default=ckeditor
 editor.wysiwyg.portal-impl.portlet.ddm.text_html.ftl=ckeditor
 editor.wysiwyg.portal-web.docroot.html.portlet.announcements.edit_entry.jsp=ckeditor
 editor.wysiwyg.portal-web.docroot.html.portlet.blogs.edit_entry.jsp=ckeditor
 editor.wysiwyg.portal-web.docroot.html.portlet.mail.edit.jsp=ckeditor
 editor.wysiwyg.portal-web.docroot.html.portlet.mail.edit_message.jsp=ckeditor
 editor.wysiwyg.portal-web.docroot.html.portlet.message_boards.edit_message.html.jsp=ckeditor
 editor.wysiwyg.portal-web.docroot.html.taglib.ui.discussion.jsp=ckeditor
 editor.wysiwyg.portal-web.docroot.html.taglib.ui.email_notification_settings.jsp=ckeditor
\end{verbatim}

\section{AlloyEditor Button Reference
Guide}\label{alloyeditor-button-reference-guide}

This reference guide provides additional information that you may find
helpful while
\href{/docs/7-1/tutorials/-/knowledge_base/t/creating-new-buttons-for-alloyeditor}{creating
new buttons for AlloyEditor}. In this guide provides useful information
on the following topics:

\begin{itemize}
\tightlist
\item
  Mixins
\end{itemize}

\subsection{Mixins}\label{mixins}

When creating a new button for the Alloy Editor, several mixins are
available that make it easy to provide additional functionality. The
available mixins, along with a brief description and a link to their API
docs, are listed below:

\begin{itemize}
\tightlist
\item
  \href{https://github.com/liferay/alloy-editor/tree/master/src/ui/react/src/components/base/button-action-style}{ButtonActionStyle}:
  provides applying style implementation for a button based on the
  \texttt{applyStyle} and \texttt{removeStyle} API of CKEDITOR
\item
  \href{https://alloyeditor.com/api/1.5.0/ButtonCommandActive.html}{ButtonCommandActive}:
  provides an \texttt{isActive} method to determine if a context-aware
  command is currently in an active state.
\item
  \href{https://alloyeditor.com/api/1.5.0/ButtonCommand.html}{ButtonCommand}:
  executes a command via CKEDITOR's API
\item
  \href{https://alloyeditor.com/api/1.5.0/ButtonKeystroke.html}{ButtonKeystroke}:
  provides a \texttt{keystroke} prop that allows configuring a function
  of the instance to be invoked upon the keystroke activation.
  https://docs-old.ckeditor.com/ckeditor\_api/symbols/CKEDITOR.dom.event.html\#getKeystroke
\item
  \href{https://alloyeditor.com/api/1.5.0/ButtonCfgProps.html}{ButtonCfgProps}:
  provides a style prop and some methods to apply the resulting style
  and checking if it is present in a given path or selection.
\item
  \href{https://alloyeditor.com/api/1.5.0/ButtonStateClasses.html}{ButtonStateClasses}:
  decorates the domElement of a component with different CSS classes
  based on the current state of the element.
\item
  \href{https://alloyeditor.com/api/1.5.0/ButtonStyle.html}{ButtonStyle}:
  provides a style prop and some methods to apply the resulting style
  and checking if it is present in a given path or selection.
\item
  \href{https://alloyeditor.com/api/1.5.0/ToolbarButtons.html}{ToolbarButtons}:
  provides a list of buttons which have to be displayed on the current
  toolbar depending on user preferences and given state.
\end{itemize}

\section{Fully Qualified Portlet IDs}\label{fully-qualified-portlet-ids}

Below is a listing of the portlet IDs for the default portlets in
Liferay DXP. You can use these IDs to embed portlets in your theme's
\href{/docs/7-1/tutorials/-/knowledge_base/t/defining-portlets-in-a-sitemap}{sitemap}.

\textbf{Collaboration}

\noindent\hrulefill

\begin{longtable}[]{@{}
  >{\raggedright\arraybackslash}p{(\columnwidth - 2\tabcolsep) * \real{0.5000}}
  >{\raggedright\arraybackslash}p{(\columnwidth - 2\tabcolsep) * \real{0.5000}}@{}}
\toprule\noalign{}
\begin{minipage}[b]{\linewidth}\raggedright
Portlet
\end{minipage} & \begin{minipage}[b]{\linewidth}\raggedright
ID
\end{minipage} \\
\midrule\noalign{}
\endhead
\bottomrule\noalign{}
\endlastfoot
Blogs & \texttt{com\_liferay\_blogs\_web\_portlet\_BlogsPortlet} \\
Blogs Aggregator &
\texttt{com\_liferay\_blogs\_web\_portlet\_BlogsAgreggatorPortlet} \\
Calendar &
\texttt{com\_liferay\_calendar\_web\_portlet\_CalendarPortlet} \\
Dynamic Data Lists Display &
\texttt{com\_liferay\_dynamic\_data\_lists\_web\_portlet\_DDLDisplayPortlet} \\
Form &
\texttt{com\_liferay\_dynamic\_data\_mapping\_form\_web\_portlet\_DDMFormPortlet} \\
Invite Members &
\texttt{com\_liferay\_invitation\_invite\_members\_web\_portlet\_InviteMembersPortlet} \\
Message Boards &
\texttt{com\_liferay\_message\_boards\_web\_portlet\_MBPortlet} \\
Recent Bloggers &
\texttt{com\_liferay\_blogs\_recent\_bloggers\_web\_portlet\_RecentBloggersPortlet} \\
\end{longtable}

\noindent\hrulefill

\textbf{Community}

\noindent\hrulefill

\begin{longtable}[]{@{}
  >{\raggedright\arraybackslash}p{(\columnwidth - 2\tabcolsep) * \real{0.5000}}
  >{\raggedright\arraybackslash}p{(\columnwidth - 2\tabcolsep) * \real{0.5000}}@{}}
\toprule\noalign{}
\begin{minipage}[b]{\linewidth}\raggedright
Portlet
\end{minipage} & \begin{minipage}[b]{\linewidth}\raggedright
ID
\end{minipage} \\
\midrule\noalign{}
\endhead
\bottomrule\noalign{}
\endlastfoot
My Sites &
\texttt{com\_liferay\_site\_my\_sites\_web\_portlet\_MySitesPortlet} \\
Page Comments &
\texttt{com\_liferay\_comment\_page\_comments\_web\_portlet\_PageCommentsPortlet} \\
Page Flags &
\texttt{com\_liferay\_flags\_web\_portlet\_PageFlagsPortlet} \\
Page Ratings &
\texttt{com\_liferay\_ratings\_page\_ratings\_web\_portlet\_PageRatingsPortlet} \\
\end{longtable}

\noindent\hrulefill

\textbf{Content Management}

\noindent\hrulefill

\begin{longtable}[]{@{}
  >{\raggedright\arraybackslash}p{(\columnwidth - 2\tabcolsep) * \real{0.5000}}
  >{\raggedright\arraybackslash}p{(\columnwidth - 2\tabcolsep) * \real{0.5000}}@{}}
\toprule\noalign{}
\begin{minipage}[b]{\linewidth}\raggedright
Portlet
\end{minipage} & \begin{minipage}[b]{\linewidth}\raggedright
ID
\end{minipage} \\
\midrule\noalign{}
\endhead
\bottomrule\noalign{}
\endlastfoot
Asset Publisher &
\texttt{com\_liferay\_asset\_publisher\_web\_portlet\_AssetPublisherPortlet} \\
Breadcrumb &
\texttt{com\_liferay\_site\_navigation\_breadcrumb\_web\_portlet\_SiteNavigationBreadcrumbPortlet} \\
Categories Navigation &
\texttt{com\_liferay\_asset\_categories\_navigation\_web\_portlet\_AssetCategoriesNavigationPortlet} \\
Documents and Media &
\texttt{com\_liferay\_document\_library\_web\_portlet\_DLPortlet} \\
Highest Rated Assets &
\texttt{com\_liferay\_asset\_publisher\_web\_portlet\_HighestRatedAssetsPortlet} \\
Knowledge Base Article &
\texttt{com\_liferay\_knowledge\_base\_web\_portlet\_ArticlePortlet} \\
Knowledge Base Display &
\texttt{com\_liferay\_knowledge\_base\_web\_portlet\_DisplayPortlet} \\
Knowledge Base Search &
\texttt{com\_liferay\_knowledge\_base\_web\_portlet\_SearchPortlet} \\
Knowledge Base Section &
\texttt{com\_liferay\_knowledge\_base\_web\_portlet\_SectionPortlet} \\
Media Gallery &
\texttt{com\_liferay\_document\_library\_web\_portlet\_IGDisplayPortlet} \\
Most Viewed Assets &
\texttt{com\_liferay\_asset\_publisher\_web\_portlet\_MostViewedAssetsPortlet} \\
Navigation Menu &
\texttt{com\_liferay\_site\_navigation\_menu\_web\_portlet\_SiteNavigationMenuPortlet} \\
Nested Applications &
\texttt{com\_liferay\_nested\_portlets\_web\_portlet\_NestedPortletsPortlet} \\
Polls Display Portlet &
\texttt{com\_liferay\_polls\_web\_portlet\_PollsDisplayPortlet} \\
Related Assets &
\texttt{com\_liferay\_asset\_publisher\_web\_portlet\_RelatedAssetsPortlet} \\
Site Map &
\texttt{com\_liferay\_site\_navigation\_site\_map\_web\_portlet\_SiteNavigationSiteMapPortlet} \\
Sites Directory &
\texttt{com\_liferay\_site\_navigation\_directory\_web\_portlet\_SitesDirectoryPortlet} \\
Tag Cloud &
\texttt{com\_liferay\_asset\_tags\_navigation\_web\_portlet\_AssetTagsCloudPortlet} \\
Tags Navigation &
\texttt{com\_liferay\_asset\_tags\_navigation\_web\_portlet\_AssetTagsNavigationPortlet} \\
Web Content Display &
\texttt{com\_liferay\_journal\_content\_web\_portlet\_JournalContentPortlet} \\
\end{longtable}

\noindent\hrulefill

\textbf{News}

\noindent\hrulefill

\begin{longtable}[]{@{}
  >{\raggedright\arraybackslash}p{(\columnwidth - 2\tabcolsep) * \real{0.5000}}
  >{\raggedright\arraybackslash}p{(\columnwidth - 2\tabcolsep) * \real{0.5000}}@{}}
\toprule\noalign{}
\begin{minipage}[b]{\linewidth}\raggedright
Portlet
\end{minipage} & \begin{minipage}[b]{\linewidth}\raggedright
ID
\end{minipage} \\
\midrule\noalign{}
\endhead
\bottomrule\noalign{}
\endlastfoot
Alerts &
\texttt{com\_liferay\_announcements\_web\_portlet\_AlertsPortlet} \\
Announcements &
\texttt{com\_liferay\_announcements\_web\_portlet\_AnnouncementsPortlet} \\
Recent Content Portlet &
\texttt{com\_liferay\_asset\_publisher\_web\_portlet\_RecentContentPortlet} \\
\end{longtable}

\noindent\hrulefill

\textbf{Sample}

\noindent\hrulefill

\begin{longtable}[]{@{}ll@{}}
\toprule\noalign{}
Portlet & ID \\
\midrule\noalign{}
\endhead
\bottomrule\noalign{}
\endlastfoot
Hello World &
\texttt{com\_liferay\_hello\_world\_web\_portlet\_HelloWorldPortlet} \\
IFrame & \texttt{com\_liferay\_iframe\_web\_portlet\_IFramePortlet} \\
\end{longtable}

\noindent\hrulefill

\textbf{Search}

\noindent\hrulefill

\begin{longtable}[]{@{}
  >{\raggedright\arraybackslash}p{(\columnwidth - 2\tabcolsep) * \real{0.5000}}
  >{\raggedright\arraybackslash}p{(\columnwidth - 2\tabcolsep) * \real{0.5000}}@{}}
\toprule\noalign{}
\begin{minipage}[b]{\linewidth}\raggedright
Portlet
\end{minipage} & \begin{minipage}[b]{\linewidth}\raggedright
ID
\end{minipage} \\
\midrule\noalign{}
\endhead
\bottomrule\noalign{}
\endlastfoot
Category Facet &
\texttt{com\_liferay\_portal\_search\_web\_category\_facet\_portlet\_CategoryFacetPortlet} \\
Custom Facet &
\texttt{com\_liferay\_portal\_search\_web\_custom\_facet\_portlet\_CustomFacetPortlet} \\
Folder Facet &
\texttt{com\_liferay\_portal\_search\_web\_folder\_facet\_portlet\_FolderFacetPortlet} \\
Modified Facet &
\texttt{com\_liferay\_portal\_search\_web\_modified\_facet\_portlet\_ModifiedFacetPortlet} \\
Search Bar &
\texttt{com\_liferay\_portal\_search\_web\_search\_bar\_portlet\_SearchBarPortlet} \\
Search Insights &
\texttt{com\_liferay\_portal\_search\_web\_search\_insights\_portlet\_SearchInsightsPortlet} \\
Search Options &
\texttt{com\_liferay\_portal\_search\_web\_search\_options\_portlet\_SearchOptionsPortlet} \\
Search Results &
\texttt{com\_liferay\_portal\_search\_web\_search\_results\_portlet\_SearchResultsPortlet} \\
Site Facet &
\texttt{com\_liferay\_portal\_search\_web\_site\_facet\_portlet\_SiteFacetPortlet} \\
Suggestions &
\texttt{com\_liferay\_portal\_search\_web\_suggestions\_portlet\_SuggestionsPortlet} \\
Tag Facet &
\texttt{com\_liferay\_portal\_search\_web\_tag\_facet\_portlet\_TagFacetPortlet} \\
Type Facet &
\texttt{com\_liferay\_portal\_search\_web\_type\_facet\_portlet\_TypeFacetPortlet} \\
User Facet &
\texttt{com\_liferay\_portal\_search\_web\_user\_facet\_portlet\_UserFacetPortlet} \\
\end{longtable}

\noindent\hrulefill

\textbf{Social}

\noindent\hrulefill

\begin{longtable}[]{@{}
  >{\raggedright\arraybackslash}p{(\columnwidth - 2\tabcolsep) * \real{0.5000}}
  >{\raggedright\arraybackslash}p{(\columnwidth - 2\tabcolsep) * \real{0.5000}}@{}}
\toprule\noalign{}
\begin{minipage}[b]{\linewidth}\raggedright
Portlet
\end{minipage} & \begin{minipage}[b]{\linewidth}\raggedright
ID
\end{minipage} \\
\midrule\noalign{}
\endhead
\bottomrule\noalign{}
\endlastfoot
Activities &
\texttt{com\_liferay\_social\_activities\_web\_portlet\_SocialActivitiesPortlet} \\
Contacts Center &
\texttt{com\_liferay\_contacts\_web\_portlet\_ContactsCenterPortlet} \\
Members &
\texttt{com\_liferay\_social\_networking\_web\_members\_portlet\_MembersPortlet} \\
My Contacts &
\texttt{com\_liferay\_contacts\_web\_portlet\_MyContactsPortlet} \\
Profile &
\texttt{com\_liferay\_contacts\_web\_portlet\_ProfilePortlet} \\
\end{longtable}

\noindent\hrulefill

\textbf{Tools}

\noindent\hrulefill

\begin{longtable}[]{@{}
  >{\raggedright\arraybackslash}p{(\columnwidth - 2\tabcolsep) * \real{0.5000}}
  >{\raggedright\arraybackslash}p{(\columnwidth - 2\tabcolsep) * \real{0.5000}}@{}}
\toprule\noalign{}
\begin{minipage}[b]{\linewidth}\raggedright
Portlet
\end{minipage} & \begin{minipage}[b]{\linewidth}\raggedright
ID
\end{minipage} \\
\midrule\noalign{}
\endhead
\bottomrule\noalign{}
\endlastfoot
Language Selector &
\texttt{com\_liferay\_site\_navigation\_language\_web\_portlet\_SiteNavigationLanguagePortlet} \\
Search &
\texttt{com\_liferay\_portal\_search\_web\_portlet\_SearchPortlet} \\
Sign In & \texttt{com\_liferay\_login\_web\_portlet\_LoginPortlet} \\
\end{longtable}

\noindent\hrulefill

\textbf{Wiki}

\noindent\hrulefill

\begin{longtable}[]{@{}
  >{\raggedright\arraybackslash}p{(\columnwidth - 2\tabcolsep) * \real{0.5000}}
  >{\raggedright\arraybackslash}p{(\columnwidth - 2\tabcolsep) * \real{0.5000}}@{}}
\toprule\noalign{}
\begin{minipage}[b]{\linewidth}\raggedright
Portlet
\end{minipage} & \begin{minipage}[b]{\linewidth}\raggedright
ID
\end{minipage} \\
\midrule\noalign{}
\endhead
\bottomrule\noalign{}
\endlastfoot
Page Menu &
\texttt{com\_liferay\_wiki\_navigation\_web\_portlet\_WikiNavigationPageMenuPortlet} \\
Tree Menu &
\texttt{com\_liferay\_wiki\_navigation\_web\_portlet\_WikiNavigationTreeMenuPortlet} \\
Wiki & \texttt{com\_liferay\_wiki\_web\_portlet\_WikiPortlet} \\
Wiki Display &
\texttt{com\_liferay\_wiki\_web\_portlet\_WikiDisplayPortlet} \\
\end{longtable}

\section{FreeMarker Taglib Macros}\label{freemarker-taglib-macros}

Liferay DXP's taglibs are mapped to FreeMarker macros, so you can use
them in your FreeMarker templates. See the
\href{/docs/7-1/tutorials/-/knowledge_base/t/front-end-taglibs}{Taglib
tutorials} for more information on using each taglib in your theme
templates. The taglib macros are defined in
\texttt{taglib-mappings.properties} files. For convenience, these macros
are listed in the table below:

Macro

Taglib

TLD

liferay\_aui

liferay-aui

liferay-aui.tld

liferay\_portlet

liferay-portlet

liferay-portlet-ext.tld

liferay\_security

liferay-security

liferay-security.tld

liferay\_theme

liferay-theme

liferay-theme.tld

liferay\_ui

liferay-ui

liferay-ui.tld

liferay\_util

liferay-util

liferay-util.tld

portlet

portlet

liferay-portlet.tld

liferay\_frontend

liferay-frontend

liferay-frontend.tld

clay

clay

liferay-clay.tld

liferay\_map

liferay-map

liferay-map.tld

liferay\_rss

liferay-rss

liferay-rss.tld

liferay\_flags

liferay-flags

liferay-flags.tld

liferay\_expando

liferay-expando

liferay-expando.tld

liferay\_journal

liferay-journal

liferay-journal.tld

liferay\_social\_bookmarks

liferay-social-bookmarks

liferay-social-bookmarks.tld

liferay\_site

liferay-site

liferay-site.tld

liferay\_comment

liferay-comment

liferay-comment.tld

liferay\_social\_activities

liferay-social-activities

liferay-social-activities.tld

liferay\_asset

liferay-asset

liferay-asset.tld

liferay\_trash

liferay-trash

liferay-trash.tld

liferay\_item\_selector

liferay-item-selector

liferay-item-selector.tld

liferay\_layout

liferay-layout

liferay-layout.tld

liferay\_editor

liferay-editor

liferay-editor.tld

liferay-fragment

liferay-fragment

liferay-fragment.tld

liferay\_reading\_time

liferay-reading-time

liferay-reading-time.tld

liferay\_site\_navigation

liferay-site-navigation

liferay-site-navigation.tld

adaptive\_media\_image

liferay-adaptive-media

liferay-adaptive-media.tld

liferay\_product\_navigation

liferay-product-navigation

liferay-product-navigation.tld

\section{Setting up Your npm
Environment}\label{setting-up-your-npm-environment}

If you're using npm for development in Liferay DXP, you should set up
your npm environment to avoid potential permissions issues. Follow these
steps to configure your npm environment:

\begin{enumerate}
\def\labelenumi{\arabic{enumi}.}
\item
  Create an \texttt{.npmrc} file in your user's home directory. This
  helps bypass npm permission-related issues.
\item
  In the \texttt{.npmrc} file, specify a \texttt{prefix} property based
  on your user's home directory, like the one shown below. This value
  specifies where to install global npm packages:

\begin{verbatim}
prefix=/Users/[username]/.npm-packages
\end{verbatim}
\item
  Set the \texttt{NPM\_PACKAGES} system environment variable to the
  \texttt{prefix} value you just specified:

\begin{verbatim}
NPM_PACKAGES=/Users/[username]/.npm-packages (same as prefix value)
\end{verbatim}
\item
  Since npm installs Yeoman and gulp executables to
  \texttt{\$\{NPM\_PACKAGES\}/bin} on UNIX and to
  \texttt{\%NPM\_PACKAGES\%} on Windows, make sure to add the
  appropriate directory to your system path. For example, on UNIX you'd
  set this:

\begin{verbatim}
PATH=${PATH}:${NPM_PACKAGES}/bin
\end{verbatim}
\end{enumerate}

\section{Liferay JS Generator}\label{liferay-js-generator}

This reference section covers these topics for the
\href{installing-the-bundle-generator-and-generating-a-bundle}{Liferay
JS Generator}:

\begin{itemize}
\tightlist
\item
  Understanding the JS Portlet Extender's Configuration
\item
  A reference list of available commands for the Liferay JS Generator
\item
  Configuration JSON options
\end{itemize}

\section{Understanding the JS Portlet Extender
Configuration}\label{understanding-the-js-portlet-extender-configuration}

Bundles generated with the Liferay JS Generator require specific method
signatures, MANIFEST headers, and configuration within their
\texttt{package.json} file to use the JS Portlet Extender. This
configuration is provided by default. For reference, this configuration
is covered in detail below.

\subsection{Manifest Header}\label{manifest-header}

The OSGi bundle is identified with the MANIFEST header shown below,
which specifies to process it with the JS Portlet Extender:

\begin{verbatim}
Require-Capability: osgi.extender;filter:="(osgi.extender=liferay.npm.portlet)"
\end{verbatim}

\subsection{Main Entry Point}\label{main-entry-point}

The main module of your JavaScript widget must export a JavaScript
function with the signature below. Bundles created with the Liferay JS
Generator have this out-of-the-box:

\begin{verbatim}
function({portletNamespace, contextPath, portletElementId, configuration}) {
  ...
}
\end{verbatim}

The entry point function receives one object parameter with four fields:

\begin{itemize}
\item
  \texttt{portletNamespace}: the unique namespace of the widget as
  defined in the Portlet specification.
\item
  \texttt{contextPath}: the URL path that can be used to retrieve bundle
  resources from the browser (it doesn't contain the protocol, host, or
  port, just the absolute path).
\item
  \texttt{portletElementId}: the DOM identifier of the widget's
  \texttt{\textless{}div\textgreater{}} node that can be used to render
  HTML.
\item
  \texttt{configuration} (optional): since JS Portlet Extender version
  1.1.0, this field contains the system (OSGi) and portlet instance
  (preferences as described in the Portlet spec) configuration for the
  widget. It has two subfields:

  \begin{itemize}
  \item
    \textbf{system:} contains the system level configuration (defined in
    Control Panel → System Settings)
  \item
    \textbf{portletInstance:} contains the per-widget configuration
    (defined in the Configuration menu option of the widget)
  \end{itemize}
\end{itemize}

Note that all values are received as strings, no matter what their type
is in OSGi configuration store.

The JavaScript-based widget's main \texttt{index.js} file configuration
is shown below for reference. Note that system settings and localization
are enabled in the example below:

\begin{verbatim}
export default function main({portletNamespace, contextPath, portletElementId, configuration}) {
    
    const node = document.getElementById(portletElementId);

    node.innerHTML =`
        <div>
            <span class="tag">${Liferay.Language.get('porlet-namespace')}:</span>
            <span class="value">${portletNamespace}</span>
        </div>
        <div>
            <span class="tag">${Liferay.Language.get('context-path')}:</span>
            <span class="value">${contextPath}</span>
        </div>
        <div>
            <span class="tag">${Liferay.Language.get('portlet-element-id')}:</span>
            <span class="value">${portletElementId}</span>
        </div>
        
        <div>
            <span class="tag">${Liferay.Language.get('configuration')}:</span>
            <span class="value">
                ${JSON.stringify(configuration, null, 2)}
            </span>
        </div>
        
    `;
    
}
\end{verbatim}

The JavaScript file containing the main entry point function is
specified in the \texttt{main} entry of the \texttt{package.json} file.
Below is the \texttt{main} entry for the \emph{JavaScript based
portlet}:

\begin{verbatim}
"main": "index.js"
\end{verbatim}

\section{Liferay JS Generator
Commands}\label{liferay-js-generator-commands}

The npm commands shown below are available for the Liferay JS Generator:

\begin{itemize}
\item
  \texttt{npm\ run\ build}: Places the output of liferay-npm-bundler in
  the designated output folder. The standard output is a JAR file that
  can be deployed manually to Liferay DXP.
\item
  \texttt{npm\ run\ deploy}: Deploys the bundle to the configured app
  server
\item
  \texttt{npm\ run\ start}: Tests the application in a local webpack
  installation instead of a Liferay DXP server. This speeds up
  development because you can see live changes without any need to
  deploy. Note, however, that because this is separate from a Liferay
  instance, you don't have access to Liferay's APIs.
\end{itemize}

\noindent\hrulefill

\textbf{Note:} By default, the webpack server uses port 8080. You can
point the webpack server to a different port by setting the
\texttt{port} key in \texttt{.npmbuildrc}:

\begin{verbatim}
 "webpack": {
   "port": 2070
 }
\end{verbatim}

\noindent\hrulefill

\begin{itemize}
\tightlist
\item
  \texttt{npm\ run\ translate}: Runs the translation features for your
  bundle. Note that this feature requires Microsoft Translator
  credentials. See
  \href{/docs/7-1/tutorials/-/knowledge_base/t/using-translation-features-in-your-portlet}{Using
  Translation Features in Your widget} for more information.
\end{itemize}

\section{Configuring System Settings for OSGi Bundles Created with the
liferay-npm-bundler}\label{configuring-system-settings-for-osgi-bundles-created-with-the-liferay-npm-bundler}

If you're
\href{/docs/7-1/tutorials/-/knowledge_base/t/creating-and-bundling-javascript-portlets-with-javascript-tooling}{creating
an OSGi bundle with the Liferay JS Generator} and want to provide system
settings for your widget, you must provide a \texttt{configuration.json}
file. This reference guide lists the available configuration options for
\texttt{configuration.json} along with example code.

\subsection{JSON Format}\label{json-format}

The \texttt{configuration.json} must follow the basic pattern shown
below:

\begin{verbatim}
{
  "system": {
    "category": "{category identifier}",
    "name": "{name of configuration}",
    "fields": {
      "{field id 1}": {
        "type": "{field type}",
        "name": "{field name}",
        "description": "{field description}",
        "default": "{default value}",
        "options": {
          "{option id 1}": "{option name 1}",
          "{option id 2}": "{option name 2}",

          "{option id n}": "{option name n}"
        }
      },
      "{field id 2}": {},

      "{field id n}": {}
    }
  },
  "portletInstance": {
    "name": "{name of configuration}",
    "fields": {
      "{field id 1}": {
        "type": "{field type}",
        "name": "{field name}",
        "description": "{field description}",
        "default": "{default value}",
        "options": {
          "{option id 1}": "{option name 1}",
          "{option id 2}": "{option name 2}",

          "{option id n}": "{option name n}"
        }
      },
      "{field id 2}": {},

      "{field id n}": {}
    }
  }
}
\end{verbatim}

The available options are described in the table below:

\noindent\hrulefill

\begin{longtable}[]{@{}
  >{\raggedright\arraybackslash}p{(\columnwidth - 2\tabcolsep) * \real{0.5000}}
  >{\raggedright\arraybackslash}p{(\columnwidth - 2\tabcolsep) * \real{0.5000}}@{}}
\toprule\noalign{}
\begin{minipage}[b]{\linewidth}\raggedright
Option
\end{minipage} & \begin{minipage}[b]{\linewidth}\raggedright
Value
\end{minipage} \\
\midrule\noalign{}
\endhead
\bottomrule\noalign{}
\endlastfoot
\texttt{\{category\ identifier\}} & Describes the identifier of the
configuration category where the settings must be placed. It's
equivalent to the category field of the
\texttt{@ExtendedObjectClassDefinition} annotation explained
\href{/docs/7-2/frameworks/-/knowledge_base/f/categorizing-the-configuration}{here}.
The category field of \texttt{configuration.json} is optional and, when
not set, the project's name specified in \texttt{package.json} is used.
You need JS Portlet Extender 1.1.0+ for this feature to work. Otherwise,
the system configuration will show up under \emph{Platform} →
\emph{Third Party} in System Settings. \\
\texttt{\{name\ of\ configuration\}} & the configuration's name as a
string or a localization key. If no value is given, the bundler falls
back to the project's name, then description given in
\texttt{package.json}. \\
\texttt{\{field\ id\}} & the field's name as a string or a localization
key \\
\texttt{\{field\ type\}} & specifies the field's type, which can be one
of the following types: ~- \texttt{number}: an integer number~-
\texttt{float}: a floating point number~- \texttt{string}: a string~-
\texttt{boolean}: true or false~- \texttt{password}: a password
(string) \\
\texttt{\{field\ name\}} & the field's name as a string or a
localization key \\
\texttt{\{field\ description\}} & an optional string or a localization
key that describes the field's purpose and appears as hint text near
it \\
\texttt{\{default\ value\}} & an optional default value for the field \\
\texttt{options} & an optional section that defines a fixed set of
values for the field \\
\texttt{\{option\ id\}} & a string that defines the option's ID \\
\texttt{\{option\ name\}} & the option's name as a string or a
localization key \\
\end{longtable}

\noindent\hrulefill

An example configuration is shown below:

\begin{verbatim}
{
  "system": {
    "category": "third-party",
    "name": "My project",
    "fields": {
      "a-number": {
        "type": "number",
        "name": "A number",
        "description": "An integer number",
        "default": "42"
      },
      "a-string": {
        "type": "string",
        "name": "A string",
        "description": "An arbitrary length string",
        "default": "this is a string"
      },
      "a-password": {
        "type": "password",
        "name": "A password",
        "description": "A secret string",
        "default": "s3.cr3t"
      },
      "a-boolean": {
        "type": "boolean",
        "name": "A boolean",
        "description": "A true|false value",
        "default": true
      },
      "an-option": {
        "type": "string",
        "name": "An option",
        "description": "A restricted values option",
        "required": true,
        "default": "A",
        "options": {
          "A": "Option a",
          "B": "Option b"
        }
      }
    }
  },
  "portletInstance": {
    "name": "Widget configuration",
    "fields": {
      "a-float": {
        "type": "float",
        "name": "A float",
        "description": "A floating point number",
        "default": "1.1"
      }
    }
  }
}
\end{verbatim}

\chapter{Screenlets in Liferay
Screens}\label{screenlets-in-liferay-screens}

Liferay Screens includes several Screenlets that you can use in your
mobile apps. Screenlets are ready-to-use components that contain a
complete UI and the code necessary to call Liferay DXP's remote services
for tasks like logging in, displaying portal content, submitting forms,
and much more.

This section contains each Screenlet's reference documentation in
separate sections for Android and iOS. Each document in these sections
lists a Screenlet's features, compatibility, available Views,
attributes, listener methods, and more:

\begin{itemize}
\tightlist
\item
  \href{/docs/7-1/reference/-/knowledge_base/r/screenlets-in-liferay-screens-for-android}{Screenlets
  in Liferay Screens for Android}
\item
  \href{/docs/7-1/reference/-/knowledge_base/r/screenlets-in-liferay-screens-for-ios}{Screenlets
  in Liferay Screens for iOS}
\end{itemize}

\noindent\hrulefill

\textbf{Note:} This section only contains Screenlet reference
documentation. For instructional information on installing, using, and
customizing Liferay Screens and its Screenlets, see the Screens
tutorials for
\href{/docs/7-1/tutorials/-/knowledge_base/t/android-apps-with-liferay-screens}{Android}
and
\href{/docs/7-1/tutorials/-/knowledge_base/t/ios-apps-with-liferay-screens}{iOS}.

\noindent\hrulefill

\chapter{Screenlets in Liferay Screens for
Android}\label{screenlets-in-liferay-screens-for-android}

Liferay Screens for Android contains several Screenlets that you can use
in your Android apps. This section contains the reference documentation
for each. If you're looking for instructions on using Screens, see the
\href{/docs/7-1/tutorials/-/knowledge_base/t/android-apps-with-liferay-screens}{Screens
tutorials}. The Screens tutorials contain instructions on
\href{/docs/7-1/tutorials/-/knowledge_base/t/using-screenlets-in-android-apps}{using
Screenlets} and
\href{/docs/7-1/tutorials/-/knowledge_base/t/using-views-in-android-screenlets}{using
views in Screenlets}. Each Screenlet reference document here lists the
Screenlet's features, compatibility, its module (if any), available
Views, attributes, listener methods, and more. The available Screenlets
are listed here with links to their reference documents:

\begin{itemize}
\item
  \href{/docs/7-1/reference/-/knowledge_base/r/loginscreenlet-for-android}{\textbf{Login
  Screenlet:}} Signs users in to a Liferay DXP instance.
\item
  \href{/docs/7-1/reference/-/knowledge_base/r/signupscreenlet-for-android}{\textbf{Sign
  Up Screenlet:}} Registers new users in a Liferay DXP instance.
\item
  \href{/docs/7-1/reference/-/knowledge_base/r/forgotpasswordscreenlet-for-android}{\textbf{Forgot
  Password Screenlet:}} Sends emails containing a new password or
  password reset link to users.
\item
  \href{/docs/7-1/reference/-/knowledge_base/r/userportraitscreenlet-for-android}{\textbf{User
  Portrait Screenlet:}} Show the user's portrait picture.
\item
  \href{/docs/7-1/reference/-/knowledge_base/r/ddlformscreenlet-for-android}{\textbf{DDL
  Form Screenlet:}} Presents dynamic forms to be filled out by users and
  submitted back to the server.
\item
  \href{/docs/7-1/reference/-/knowledge_base/r/ddllistscreenlet-for-android}{\textbf{DDL
  List Screenlet:}} Shows a list of records based on a pre-existing DDL
  in a Liferay DXP instance.
\item
  \href{/docs/7-1/reference/-/knowledge_base/r/assetlistscreenlet-for-android}{\textbf{Asset
  List Screenlet:}} Shows a list of assets managed by the
  \href{/docs/7-1/tutorials/-/knowledge_base/t/asset-framework}{Asset
  Framework}. This includes web content, blog entries, documents, users,
  and more.
\item
  \href{/docs/7-1/reference/-/knowledge_base/r/webcontentdisplayscreenlet-for-android}{\textbf{Web
  Content Display Screenlet:}} Shows the web content's HTML or
  structured content. This Screenlet uses the features available in
  \href{/docs/7-1/user/-/knowledge_base/u/introduction-web-content}{Web
  Content Management}.
\item
  \href{/docs/7-1/reference/-/knowledge_base/r/web-content-list-screenlet-for-android}{\textbf{Web
  Content List Screenlet:}} Shows a list of web contents from a folder,
  usually based on a pre-existing \texttt{DDMStructure}.
\item
  \href{/docs/7-1/reference/-/knowledge_base/r/image-gallery-screenlet-for-android}{\textbf{Image
  Gallery Screenlet:}} Shows a list of images from a folder. This
  Screenlet also lets users upload and delete images.
\item
  \href{/docs/7-1/reference/-/knowledge_base/r/rating-screenlet-for-android}{\textbf{Rating
  Screenlet:}} Shows the rating for an asset. This Screenlet also lets
  the user update or delete the rating.
\item
  \href{/docs/7-1/reference/-/knowledge_base/r/comment-list-screenlet-for-android}{\textbf{Comment
  List Screenlet:}} Shows a list of comments for an asset.
\item
  \href{/docs/7-1/reference/-/knowledge_base/r/comment-display-screenlet-for-android}{\textbf{Comment
  Display Screenlet:}} Shows a single comment for an asset.
\item
  \href{/docs/7-1/reference/-/knowledge_base/r/comment-add-screenlet-for-android}{\textbf{Comment
  Add Screenlet:}} Lets the user comment on an asset.
\item
  \href{/docs/7-1/reference/-/knowledge_base/r/asset-display-screenlet-for-android}{\textbf{Asset
  Display Screenlet:}} Displays an asset. Currently, this Screenlet can
  display Documents and Media Library files (\texttt{DLFileEntry}
  entities), blog articles (\texttt{BlogsEntry} entities), and web
  content articles (\texttt{WebContent} entities). You can also use it
  to display custom assets.
\item
  \href{/docs/7-1/reference/-/knowledge_base/r/blogs-entry-display-screenlet-for-android}{\textbf{Blogs
  Entry Display Screenlet:}} Shows a single blog entry.
\item
  \href{/docs/7-1/reference/-/knowledge_base/r/image-display-screenlet-for-android}{\textbf{Image
  Display Screenlet:}} Shows a single image file from the
  \href{/docs/7-1/user/-/knowledge_base/u/managing-documents-and-media}{Documents
  and Media Library}.
\item
  \href{/docs/7-1/reference/-/knowledge_base/r/video-display-screenlet-for-android}{\textbf{Video
  Display Screenlet:}} Shows a single video file from the
  \href{/docs/7-1/user/-/knowledge_base/u/managing-documents-and-media}{Documents
  and Media Library}.
\item
  \href{/docs/7-1/reference/-/knowledge_base/r/audio-display-screenlet-for-android}{\textbf{Audio
  Display Screenlet:}} Shows a single audio file from the
  \href{/docs/7-1/user/-/knowledge_base/u/managing-documents-and-media}{Documents
  and Media Library}.
\item
  \href{/docs/7-1/reference/-/knowledge_base/r/pdf-display-screenlet-for-android}{\textbf{PDF
  Display Screenlet:}} Shows a single PDF file from the
  \href{/docs/7-1/user/-/knowledge_base/u/managing-documents-and-media}{Documents
  and Media Library}.
\item
  \href{/docs/7-1/reference/-/knowledge_base/r/web-screenlet-for-android}{\textbf{Web
  Screenlet:}} Displays any web page. You can also customize the web
  page through injection of local and remote JavaScript and CSS files.
\end{itemize}

\section{Login Screenlet for Android}\label{login-screenlet-for-android}

\subsection{Requirements}\label{requirements}

\begin{itemize}
\tightlist
\item
  Android SDK 4.1 (API Level 16) or above
\item
  Liferay Portal 6.2 CE/EE, Liferay CE Portal 7.0/7.1, Liferay DXP
\end{itemize}

\subsection{Compatibility}\label{compatibility}

\begin{itemize}
\tightlist
\item
  Android SDK 4.1 (API Level 16) or above
\end{itemize}

\subsection{Xamarin Requirements}\label{xamarin-requirements}

\begin{itemize}
\tightlist
\item
  Visual Studio 7.2
\item
  Mono .NET framework 5.4.1.6
\end{itemize}

\subsection{Features}\label{features}

Login Screenlet lets you authenticate portal users in your Android app.
The following types of authentication are supported:

\begin{itemize}
\item
  \textbf{Basic:} uses user login and password according to
  \href{http://tools.ietf.org/html/rfc2617}{HTTP Basic Access
  Authentication specification}. Depending on the authentication method
  used by your Liferay instance, you need to provide the user's email
  address, screen name, or user ID. You also need to provide the user's
  password.
\item
  \textbf{OAuth:} implements \href{https://oauth.net/2/}{OAuth 2}.
\item
  \textbf{Cookie:} uses a cookie to log in. This lets you access
  documents and images in the portal's document library without the
  guest view permission in the portal. The other authentication types
  require this permission to access such files.
\end{itemize}

For instructions on configuring the Screenlet to use these
authentication types, see the below
\hyperref[portal-configuration]{Portal Configuration} and
\hyperref[attributes]{Screenlet Attributes} sections.

When a user successfully authenticates, their user attributes are
retrieved for use in the app. You can use the \texttt{SessionContext}
class to get the current user's attributes.

Note that user credentials and attributes can be stored in an app's data
store (see the \texttt{saveCredentials} attribute). Android's
\texttt{SharedPreferences} is currently the only data store implemented.
However, new and more secure data stores will be added in the future.
Stored user credentials can be used to automatically log the user in to
subsequent sessions. To do this, you can use the method
\texttt{SessionContext.loadStoredCredentials()}.

\subsection{JSON Services Used}\label{json-services-used}

Screenlets in Liferay Screens call the portal's JSON web services. This
Screenlet calls the following services and methods.

\noindent\hrulefill

\begin{longtable}[]{@{}lll@{}}
\toprule\noalign{}
Service & Method & Notes \\
\midrule\noalign{}
\endhead
\bottomrule\noalign{}
\endlastfoot
\texttt{UserService} & \texttt{getUserByEmailAddress} & Basic login \\
\texttt{UserService} & \texttt{getUserByScreenName} & Basic login \\
\texttt{UserService} & \texttt{getUserById} & Basic login \\
\texttt{UserService} & \texttt{getCurrentUser} & Cookie and OAuth
login \\
\end{longtable}

\noindent\hrulefill

\subsection{Module}\label{module}

\begin{itemize}
\tightlist
\item
  Auth
\end{itemize}

\subsection{Views}\label{views}

\begin{itemize}
\tightlist
\item
  Default
\item
  Material
\end{itemize}

For instructions on using these Views, see the \texttt{layoutId}
attribute in the \hyperref[attributes]{Attributes section below}.

\begin{figure}
\centering
\includegraphics{./images/screens-android-login.png}
\caption{The Login Screenlet using the Default (left) and Material
(right) Viewsets.}
\end{figure}

\subsection{Portal Configuration}\label{portal-configuration}

\subsubsection{Basic Authentication}\label{basic-authentication}

Before using Login Screenlet, you should make sure your portal is
configured with the authentication option you want to use. You can
choose email address, screen name, or user ID. You can set this in the
Control Panel by selecting \emph{Configuration} → \emph{Instance
Settings}, and then selecting the \emph{Authentication} section. The
authentication options are in the \emph{How do users authenticate?}
selector menu. For more information, see the User Guide's
\href{/docs/7-1/user/-/knowledge_base/u/authentication}{authentication
section}.

\begin{figure}
\centering
\includegraphics{./images/screens-portal-auth.png}
\caption{Set the authentication method in your Liferay DXP instance.}
\end{figure}

\subsubsection{OAuth Authentication}\label{oauth-authentication}

For instructions on using OAuth with Login Screenlet, see the tutorial
on
\href{/docs/7-1/tutorials/-/knowledge_base/t/using-oauth-2-in-liferay-screens-for-android}{using
OAuth 2 with Liferay Screens}.

\subsection{Offline}\label{offline}

This Screenlet doesn't support offline mode. It requires network
connectivity. If you need to log in users automatically, even when
there's no network connection, you can use the
\texttt{credentialsStorage} attribute together with the
\texttt{SessionContext.loadStoredCredentials} method.

\subsection{Required Attributes}\label{required-attributes}

\begin{itemize}
\tightlist
\item
  None
\end{itemize}

\subsection{Attributes}\label{attributes}

\noindent\hrulefill

Attribute \textbar{} Data type \textbar{} Explanation \textbar{}
\texttt{layoutId} \textbar{} \texttt{@layout} \textbar{} The ID of the
View's layout. You can set this attribute to
\texttt{@layout/login\_default} (Default View) or
\texttt{@layout/login\_material} (Material View). To use the Material
View, you must first install the Material View Set.
\href{/docs/7-1/tutorials/-/knowledge_base/t/using-views-in-android-screenlets}{Click
here} for instructions on installing and using Views and View Sets,
including the Material View Set. \textbar{} \texttt{companyId}
\textbar{} \texttt{number} \textbar{} The ID of the portal instance to
authenticate to. If you don't set this attribute or set it to
\texttt{0}, the Screenlet uses the \texttt{companyId} setting in
\texttt{LiferayServerContext}. \textbar{} \texttt{loginMode} \textbar{}
\texttt{enum} \textbar{} The Screenlet's authentication type. You can
set this attribute to \texttt{basic}, \texttt{cookie},
\texttt{oauth2Redirect}, or \texttt{oauth2UsernameAndPassword}. If you
don't set this attribute, the Screenlet defaults to basic
authentication. \textbar{} \texttt{basicAuthMethod} \textbar{}
\texttt{string} \textbar{} Specifies the authentication option to use
with basic or cookie authentication. You can set this attribute to
\texttt{email}, \texttt{screenName} or \texttt{userId}. This must match
the server's authentication option. If you don't set this attribute, and
don't set the \texttt{loginMode} attribute to one of the OAuth values or
\texttt{cookie}, the Screenlet defaults to basic authentication with the
\texttt{email} option. \textbar{} \texttt{oauth2Redirect} \textbar{}
\texttt{string} \textbar{} The URL that the mobile browser will redirect
the user to after successful login. You must configure this in the
portal's OAuth 2 Admin portlet, and associate the URL with the Android
app. \textbar{} \texttt{oauth2ClientId} \textbar{} \texttt{string}
\textbar{} The ID of the OAuth 2 application in the portal. You can find
this value in the portal's OAuth 2 Admin portlet. \textbar{}
\texttt{oauth2ClientSecret} \textbar{} \texttt{string} \textbar{} The
client secret of the OAuth 2 application in the portal. You can find
this value in the portal's OAuth 2 Admin portlet. \textbar{}
\texttt{oauth2Scopes} \textbar{} \texttt{string} \textbar{} The portal
permissions to request. You can define a set of permissions associated
with an OAuth 2 application in the portal's OAuth 2 Admin portlet. Use
this attribute to request a subset of those permissions. Separate
multiple scopes with a space (e.g., \texttt{"scope1\ scope2\ scope3"}).
\textbar{} \texttt{credentialsStorage} \textbar{} \texttt{enum}
\textbar{} Sets the mode for storing user credentials. The possible
values are \texttt{none}, \texttt{auto}, and
\texttt{shared\_preferences}. If set to \texttt{shared\_preferences},
the user credentials and attributes are stored using Android's
\texttt{SharedPreferences} class. If set to \texttt{none}, user
credentials and attributes aren't saved at all. If set to \texttt{auto},
the best of the available storage modes is used. Currently, this is
equivalent to \texttt{shared\_preferences}. The default value is
\texttt{none}. \textbar{} \texttt{shouldHandleCookieExpiration}
\textbar{} \texttt{bool} \textbar~Whether to refresh the cookie
automatically when using cookie login. When set to \texttt{true} (the
default value), the cookie refreshes as it's about to expire. \textbar{}
\texttt{cookieExpirationTime} \textbar{} \texttt{int} \textbar~How long
the cookie lasts, in seconds. This value depends on your portal
instance's configuration. The default value is \texttt{900}. \textbar{}
\texttt{authenticator} \textbar{} \texttt{Authenticator} \textbar~An
instance of a class that implements the \texttt{Authenticator}
interface. The \emph{Challenge-Response Authentication} section below
discusses this further. \textbar{}

\noindent\hrulefill

\subsection{Listener}\label{listener}

The Login Screenlet delegates some events to an object that implements
the \texttt{LoginListener} interface. This interface let you implement
the following methods:

\begin{itemize}
\item
  \texttt{onLoginSuccess(User\ user)}: Called when login successfully
  completes. The \texttt{user} parameter contains a set of the logged in
  user's attributes. The supported keys are the same as those in the
  \href{https://github.com/liferay/liferay-portal/blob/master/portal-impl/src/com/liferay/portal/service.xml\#L2575-L2737}{portal's
  User entity}.
\item
  \texttt{onLoginFailure(Exception\ e)}: Called when an error occurs in
  the process.
\end{itemize}

\subsection{Challenge-Response
Authentication}\label{challenge-response-authentication}

To support
\href{https://en.wikipedia.org/wiki/Challenge\%E2\%80\%93response_authentication}{challenge-response
authentication} when using a cookie to log in to the portal, Login
Screenlet has an \texttt{authenticator} attribute. As mentioned in the
above \emph{Attributes} table, this attribute's value is a class that
implements the
\href{https://square.github.io/okhttp/3.x/okhttp/okhttp3/Authenticator.html}{\texttt{Authenticator}
interface}.

Here's an example of such a class. It sends a basic authorization in
response to an authentication challenge:

\begin{verbatim}
public class BasicAuthAutenticator extends BasicAuthentication implements Authenticator {

    public BasicAuthAutenticator(String username, String password) {
        super(username, password);
    }

    @Override
    public Request authenticate(Proxy proxy, Response response) throws IOException {
        String credential = Credentials.basic(username, password);
        return response.request().newBuilder().header(Headers.AUTHORIZATION, credential).build();
    }

    @Override
    public Request authenticateProxy(Proxy proxy, Response response) throws IOException {
        return null;
    }
}
\end{verbatim}

\section{Sign Up Screenlet for
Android}\label{sign-up-screenlet-for-android}

\subsection{Requirements}\label{requirements-1}

\begin{itemize}
\tightlist
\item
  Android SDK 4.1 (API Level 16) or above
\item
  Liferay Portal 6.2 CE/EE, Liferay CE Portal 7.0/7.1, Liferay DXP
\end{itemize}

\subsection{Compatibility}\label{compatibility-1}

\begin{itemize}
\tightlist
\item
  Android SDK 4.1 (API Level 16) or above
\end{itemize}

\subsection{Xamarin Requirements}\label{xamarin-requirements-1}

\begin{itemize}
\tightlist
\item
  Visual Studio 7.2
\item
  Mono .NET framework 5.4.1.6
\end{itemize}

\subsection{Features}\label{features-1}

The Sign Up Screenlet creates a new user in your Liferay instance: a new
user of your app can become a new user in your portal. You can also use
this Screenlet to save new users' credentials on their devices. This
enables auto login for future sessions. The Screenlet also supports
navigation of form fields from the device's keyboard.

\subsection{JSON Services Used}\label{json-services-used-1}

Screenlets in Liferay Screens call JSON web services in the portal. This
Screenlet calls the following services and methods.

\noindent\hrulefill

\begin{longtable}[]{@{}lll@{}}
\toprule\noalign{}
Service & Method & Notes \\
\midrule\noalign{}
\endhead
\bottomrule\noalign{}
\endlastfoot
\texttt{UserService} & \texttt{addUser} & \\
\end{longtable}

\noindent\hrulefill

\subsection{Module}\label{module-1}

\begin{itemize}
\tightlist
\item
  Auth
\end{itemize}

\subsection{Views}\label{views-1}

\begin{itemize}
\tightlist
\item
  Default
\item
  Material
\end{itemize}

\begin{figure}
\centering
\includegraphics{./images/screens-android-signup.png}
\caption{The Sign Up Screenlet with the Default (left) and Material
(right) Viewsets.}
\end{figure}

\subsection{Portal Configuration}\label{portal-configuration-1}

Sign Up Screenlet's corresponding configuration in the Liferay instance
can be set in the Control Panel by selecting \emph{Configuration} →
\emph{Instance Settings}, and then selecting the \emph{Authentication}
section.

\begin{figure}
\centering
\includegraphics{./images/screens-portal-signup.png}
\caption{The Liferay instance's authentication settings.}
\end{figure}

For more details, see the
\href{/docs/7-1/user/-/knowledge_base/u/authentication}{Authentication}
section of the User Guide.

\subsection{Anonymous Requests}\label{anonymous-requests}

Anonymous requests are unauthenticated requests. Authentication is still
required, however, to call the API. To allow this operation, the portal
administrator should create a user with minimal permissions. To use Sign
Up Screenlet, you need to use that user in your layout. You should add
that user's credentials to \texttt{server\_context.xml}.

\subsection{Offline}\label{offline-1}

This Screenlet doesn't support offline mode. It requires network
connectivity.

\subsection{Required Attributes}\label{required-attributes-1}

\begin{itemize}
\tightlist
\item
  \texttt{anonymousApiUserName}
\item
  \texttt{anonymousApiPassword}
\end{itemize}

\subsection{Attributes}\label{attributes-1}

\noindent\hrulefill

Attribute \textbar{} Data type \textbar{} Explanation \textbar{}
\texttt{layoutId} \textbar{} \texttt{@layout} \textbar{} The layout used
to show the View.\textbar{} \texttt{anonymousApiUserName} \textbar{}
\texttt{string} \textbar{} The user's name, email address, or ID to use
for authenticating the request. The portal's authentication method
defines which of these is used. \textbar{} \texttt{anoymousApiPassword}
\textbar{} \texttt{string} \textbar{} The password used to authenticate
the request. \textbar{} \texttt{companyId} \textbar{} \texttt{number}
\textbar{} When set, a user in the specified company is authenticated.
If not set, the company specified in \texttt{LiferayServerContext} is
used. \textbar{} \texttt{autoLogin} \textbar{} \texttt{boolean}
\textbar{} Sets whether the user is logged in automatically after a
successful sign up. \textbar{} \texttt{credentialsStorage} \textbar{}
\texttt{enum} \textbar{} Sets the mode for storing user credentials. The
possible values are \texttt{none}, \texttt{auto}, and
\texttt{shared\_preferences}. If set to \texttt{shared\_preferences},
the user credentials and attributes are stored using Android's
\texttt{SharedPreferences} class. If set to \texttt{none}, user
credentials and attributes aren't saved at all. If set to \texttt{auto},
the best of the available storage modes is used. Currently, this is
equivalent to \texttt{shared\_preferences}. The default value is
\texttt{none}. \textbar{}
\texttt{basicAuthMethod}\textbar{}\texttt{enum}\textbar{} Specifies the
authentication method to use after a successful sign up. This must match
the authentication method configured on the server. You can set this
attribute to \texttt{email}, \texttt{screenName} or \texttt{userId}. The
default value is \texttt{email}. \textbar{}

\noindent\hrulefill

\subsection{Listener}\label{listener-1}

The Sign Up Screenlet delegates some events to an object that implements
the \texttt{SignUpListener} interface. This interface lets you implement
the following methods:

\begin{itemize}
\item
  \texttt{onSignUpSuccess(User\ user)}: Called when sign up successfully
  completes. The \texttt{user} parameter contains a set of the created
  user's attributes, as defined in the
  \href{https://docs.liferay.com/dxp/portal/7.1-latest/javadocs/portal-kernel/com/liferay/portal/kernel/model/User.html}{portal's
  \texttt{User} entity}.
\item
  \texttt{onSignUpFailure(Exception\ e)}: Called when an error occurs in
  the process.
\end{itemize}

\section{Forgot Password Screenlet for
Android}\label{forgot-password-screenlet-for-android}

\subsection{Requirements}\label{requirements-2}

\begin{itemize}
\tightlist
\item
  Android SDK 4.1 (API Level 16) or above
\item
  Liferay Portal 6.2 CE/EE, Liferay CE Portal 7.0/7.1, Liferay DXP
\item
  Liferay Screens Compatibility app
  (\href{http://www.liferay.com/marketplace/-/mp/application/54365664}{CE}
  or
  \href{http://www.liferay.com/marketplace/-/mp/application/54369726}{EE/DXP}).
  This app is preinstalled in Liferay CE Portal 7.0/7.1 and Liferay DXP.
\end{itemize}

\subsection{Compatibility}\label{compatibility-2}

\begin{itemize}
\tightlist
\item
  Android SDK 4.1 (API Level 16) or above
\end{itemize}

\subsection{Xamarin Requirements}\label{xamarin-requirements-2}

\begin{itemize}
\tightlist
\item
  Visual Studio 7.2
\item
  Mono .NET framework 5.4.1.6
\end{itemize}

\subsection{Features}\label{features-2}

The Forgot Password Screenlet sends an email to registered users with
their new passwords or password reset links, depending on the server
configuration. The available authentication methods are

\begin{itemize}
\tightlist
\item
  Email address
\item
  Screen name
\item
  User id
\end{itemize}

\subsection{JSON Services Used}\label{json-services-used-2}

Screenlets in Liferay Screens call JSON web services in the portal. This
Screenlet calls the following services and methods.

\noindent\hrulefill

\begin{longtable}[]{@{}lll@{}}
\toprule\noalign{}
Service & Method & Notes \\
\midrule\noalign{}
\endhead
\bottomrule\noalign{}
\endlastfoot
\texttt{UserService} & \texttt{sendPasswordByEmailAddress} & \\
\texttt{UserService} & \texttt{sendPasswordByUserId} & \\
\texttt{UserService} & \texttt{sendPasswordByScreenName} & \\
\end{longtable}

\noindent\hrulefill

\subsection{Module}\label{module-2}

\begin{itemize}
\tightlist
\item
  Auth
\end{itemize}

\subsection{Views}\label{views-2}

\begin{itemize}
\tightlist
\item
  Default
\item
  Material
\end{itemize}

\begin{figure}
\centering
\includegraphics{./images/screens-android-forgotpwd.png}
\caption{The Forgot Password Screenlet with the Default (left) and
Material (right) Viewsets.}
\end{figure}

\subsection{Portal Configuration}\label{portal-configuration-2}

To use Forgot Password Screenlet, the portal must be configured to allow
users to request new passwords. The below sections show you how to do
this.

\subsubsection{Authentication Method}\label{authentication-method}

The authentication method configured in the portal can be different from
the one used by this Screenlet. For example, it's \emph{perfectly fine}
to use \texttt{screenName} for sign in authentication, but allow users
to recover their password using the \texttt{email} authentication
method.

\subsubsection{Password Reset}\label{password-reset}

You can set the Liferay instance's corresponding password reset options
in the Control Panel by selecting \emph{Configuration} → \emph{Instance
Settings}, and then selecting the \emph{Authentication} section. The
Screenlet's password functionality depends on the authentication
settings in the portal:

\begin{figure}
\centering
\includegraphics{./images/screens-password-reset.png}
\caption{Checkboxes for the password recovery features in your Liferay
instance.}
\end{figure}

If these options are both unchecked, password recovery is disabled. If
both options are checked, an email containing a password reset link is
sent when a user requests it. If only the first option is checked, an
email containing a new password is sent when a user requests it.

For more details, see the
\href{/docs/7-1/user/-/knowledge_base/u/authentication}{Authentication}
section of the User Guide.

\subsubsection{Anonymous Request}\label{anonymous-request}

An anonymous request can be made without the user being logged in.
However, authentication is needed to call the API. To allow this
operation, the portal administrator should create a specific user with
minimal permissions.

\subsection{Offline}\label{offline-2}

This Screenlet doesn't support offline mode. It requires network
connectivity.

\subsection{Required Attributes}\label{required-attributes-2}

\begin{itemize}
\tightlist
\item
  \texttt{anonymousApiUserName}
\item
  \texttt{anonymousApiPassword}
\end{itemize}

\subsection{Attributes}\label{attributes-2}

\noindent\hrulefill

Attribute \textbar{} Data type \textbar{} Explanation \textbar{}
\texttt{layoutId} \textbar{} \texttt{@layout} \textbar{} The layout used
to show the View. \textbar{} \texttt{anonymousApiUserName} \textbar{}
\texttt{string} \textbar{} The user name, email address, or
\texttt{userId} to use for authenticating the request. This depends on
the portal's authentication settings. \textbar{}
\texttt{anonymousApiPassword} \textbar{} \texttt{string} \textbar{} The
password to use to authenticate the request. \textbar{}
\texttt{companyId} \textbar{} \texttt{number} \textbar{} When set, a
user within the specified company is authenticated. If the value is set
to \texttt{0}, the company specified in \texttt{LiferayServerContext} is
used. \textbar{} \texttt{basicAuthMethod} \textbar{} \texttt{string}
\textbar{} The authentication method presented to the user. This can be
\texttt{email}, \texttt{screenName}, or \texttt{userId}. The default
value is \texttt{email}. \textbar{}

\noindent\hrulefill

\subsection{Listener}\label{listener-2}

The Forgot Password Screenlet delegates some events to an object that
implements the \texttt{ForgotPasswordListener} interface. This interface
lets you implement the following methods:

\begin{itemize}
\item
  \texttt{onForgotPasswordRequestSuccess(boolean\ passwordSent)}: Called
  when a password reset email is successfully sent. The boolean
  parameter determines whether the email contains the new password or a
  password reset link.
\item
  \texttt{onForgotPasswordRequestFailure(Exception\ e)}: Called when an
  error occurs in the process.
\end{itemize}

\section{User Portrait Screenlet for
Android}\label{user-portrait-screenlet-for-android}

\subsection{Requirements}\label{requirements-3}

\begin{itemize}
\tightlist
\item
  Android SDK 4.1 (API Level 16) or above
\item
  Liferay Portal 6.2 CE/EE, Liferay CE Portal 7.0/7.1, Liferay DXP
\item
  Picasso library
\end{itemize}

\subsection{Compatibility}\label{compatibility-3}

\begin{itemize}
\tightlist
\item
  Android SDK 4.1 (API Level 16) or above
\end{itemize}

\subsection{Xamarin Requirements}\label{xamarin-requirements-3}

\begin{itemize}
\tightlist
\item
  Visual Studio 7.2
\item
  Mono .NET framework 5.4.1.6
\end{itemize}

\subsection{Features}\label{features-3}

The User Portrait Screenlet shows the users' profile pictures. If a user
doesn't have a profile picture, a placeholder image is shown. The
Screenlet allows the profile picture to be edited via the
\texttt{editable} property.

\subsection{JSON Services Used}\label{json-services-used-3}

Screenlets in Liferay Screens call JSON web services in the portal. This
Screenlet calls the following services and methods.

\noindent\hrulefill

\begin{longtable}[]{@{}lll@{}}
\toprule\noalign{}
Service & Method & Notes \\
\midrule\noalign{}
\endhead
\bottomrule\noalign{}
\endlastfoot
\texttt{UserService} & \texttt{getUserById} & \\
\end{longtable}

\noindent\hrulefill

\subsection{Module}\label{module-3}

\begin{itemize}
\tightlist
\item
  None
\end{itemize}

\subsection{Views}\label{views-3}

\begin{itemize}
\tightlist
\item
  Default
\item
  Material
\end{itemize}

\begin{figure}
\centering
\includegraphics{./images/screens-android-userportrait.png}
\caption{The User Portrait Screenlet using the Default (left) and
Material (right) Views.}
\end{figure}

\subsection{Portal Configuration}\label{portal-configuration-3}

No additional steps required.

\subsection{Activity Configuration}\label{activity-configuration}

The User Portrait Screenlet needs the following user permissions:

\begin{verbatim}
<uses-permission android:name="android.permission.CAMERA"/>
<uses-permission android:name="android.permission.WRITE_EXTERNAL_STORAGE"/>
\end{verbatim}

\subsection{Offline}\label{offline-3}

This Screenlet supports offline mode so it can function without a
network connection. For more information on how offline mode works, see
the
\href{/docs/7-1/tutorials/-/knowledge_base/t/architecture-of-offline-mode-in-liferay-screens}{tutorial
on its architecture}.

When loading the portrait, the Screenlet supports the following offline
mode policies:

\noindent\hrulefill

Policy \textbar{} What happens \textbar{} When to use \textbar{}
\texttt{REMOTE\_ONLY} \textbar{} The Screenlet loads the user portrait
from the portal. If a connection issue occurs, the Screenlet uses the
listener to notify the developer about the error. If the Screenlet loads
the portrait, it stores the received image in the local cache for later
use. \textbar{} Use this policy when you always need to show updated
portraits, and show the default placeholder when there's no connection.
\textbar{} \texttt{CACHE\_ONLY} \textbar{} The Screenlet loads the user
portrait from the local cache. If the portrait isn't there, the
Screenlet uses the listener to notify the developer about the error.
\textbar{} Use this policy to show local portraits, without retrieving
remote information under any circumstance. \textbar{}
\texttt{REMOTE\_FIRST} \textbar{} The Screenlet loads the user portrait
from the portal. The Screenlet displays the portrait to the user and
stores it in the local cache for later use. If a connection issue
occurs, the Screenlet retrieves the portrait from the local cache. If
the portrait doesn't exist there, the Screenlet uses the listener to
notify the developer about the error. \textbar{} Use this policy to show
the most recent portrait when connected, but show a potentially outdated
version when there's no connection. \textbar{} \texttt{CACHE\_FIRST}
\textbar{} If the portrait exists in the local cache, the Screenlet
loads it from there. If it doesn't exist there, the Screenlet requests
the portrait from the portal and uses the listener to notify the
developer about any connection errors. \textbar{} Use this policy to
save bandwidth and loading time in the event a local (but probably
outdated) portrait exists. \textbar{}

\noindent\hrulefill

When editing the portrait, the Screenlet supports the following offline
mode policies:

\noindent\hrulefill

Policy \textbar{} What happens \textbar{} When to use \textbar{}
\texttt{REMOTE\_ONLY} \textbar{} The Screenlet sends the user portrait
to the portal. If a connection issue occurs, the Screenlet uses the
listener to notify the developer about the error, but it also discards
the new portrait. \textbar{} Use this policy when you need to make sure
portal always has the most recent version of the portrait. \textbar{}
\texttt{CACHE\_ONLY} \textbar{} The Screenlet stores the user portrait
in the local cache. \textbar{} Use this policy when you need to save the
portrait locally, but don't want to change the portrait in the portal.
\textbar{} \texttt{REMOTE\_FIRST} \textbar{} The Screenlet sends the
user portrait to the portal. If this succeeds, the Screenlet also stores
the portrait in the local cache for later usage. If a connection issue
occurs, the Screenlet stores the portrait in the local cache with the
\emph{dirty flag} enabled. This causes the portrait to be sent to the
portal when the synchronization process runs. \textbar{} Use this policy
when you need to make sure the Screenlet sends the new portrait to the
portal as soon as the connection is restored. \textbar{}
\texttt{CACHE\_FIRST} \textbar{} The Screenlet stores the user portrait
in the local cache and then sends it to the portal. If a connection
issue occurs, the Screenlet stores the portrait in the local cache with
the \emph{dirty flag} enabled. This causes the portrait to be sent to
the portal when the synchronization process runs. \textbar{} Use this
policy when you need to make sure the Screenlet sends the new portrait
to the portal as soon as the connection is restored. Compared to
\texttt{REMOTE\_FIRST}, this policy always stores the portrait in the
cache. The \texttt{REMOTE\_FIRST} policy only stores the new image in
the cache in the event of a network error or a successful upload.
\textbar{}

\noindent\hrulefill

\subsection{Required Attributes}\label{required-attributes-3}

\begin{itemize}
\tightlist
\item
  None
\end{itemize}

Note that if you don't set any attributes, the Screenlet loads the
logged-in user's portrait.

\subsection{Attributes}\label{attributes-3}

\noindent\hrulefill

Attribute \textbar{} Data type \textbar{} Explanation \textbar{}
\texttt{layoutId} \textbar{} \texttt{@layout} \textbar{} The layout used
to show the View. \textbar{} \texttt{autoLoad} \textbar{}
\texttt{boolean} \textbar{} Whether the portrait should load when the
Screenlet is attached to the window. \textbar{} \texttt{userId}
\textbar{} \texttt{number} \textbar{} The ID of the user whose portrait
is being requested. If this attribute is set, the \texttt{male},
\texttt{portraitId}, and \texttt{uuid} attributes are ignored.
\textbar{} \texttt{male} \textbar{} \texttt{boolean} \textbar{} Whether
the default portrait placeholder shows a male or female outline. This
attribute is used if \texttt{userId} isn't specified. \textbar{}
\texttt{portraitId} \textbar{} \texttt{number} \textbar{} The ID of the
portrait to load. This attribute is used if \texttt{userId} isn't
specified. \textbar{} \texttt{uuid} \textbar{} \texttt{string}
\textbar{} The \texttt{uuid} of the user whose portrait is being
requested. This attribute is used if \texttt{userId} isn't specified.
\textbar{} \texttt{editable} \textbar{} \texttt{boolean} \textbar{} Lets
the user change the portrait image by taking a photo or selecting a
gallery picture. \textbar{} \texttt{offlinePolicy} \textbar{}
\texttt{enum} \textbar{} Configure the loading and saving behavior in
case of connectivity issues. For more details, read the ``Offline''
section below. \textbar{}

\noindent\hrulefill

\subsection{Methods}\label{methods}

\noindent\hrulefill

Method \textbar{} Return \textbar{} Explanation \textbar{}
\texttt{load()} \textbar{} \texttt{void} \textbar{} Starts the request
to load the user specified in the \texttt{userId} property, or the
portrait specified in the \texttt{portraitId}and \texttt{uuid}
properties. \textbar{}
\texttt{upload(int\ requestCode,}\texttt{Intent\ onActivityResultData)}
\textbar{} \texttt{void} \textbar{} Starts the request to upload a
profile picture from the source specified in the \texttt{requestCode}
property (gallery or camera), and with the path stored in the
\texttt{onActivityResultData} variable. \textbar{}

\noindent\hrulefill

\subsection{Listener}\label{listener-3}

The User Portrait Screenlet delegates some events to an object that
implements the \texttt{UserPortraitListener} interface. This interface
lets you implement the following methods:

\begin{itemize}
\item
  \texttt{onUserPortraitLoadReceived(Bitmap\ bitmap)}: Called when an
  image is received from the server. You can then apply image filters
  (grayscale, for example) and return the new image. You can return
  \texttt{null} or the original image supplied as the argument if you
  don't want to modify it.
\item
  \texttt{onUserPortraitUploaded()}: Called when the user portrait
  upload service finishes.
\item
  \texttt{error(Exception\ e,\ String\ userAction)}: Called when an
  error occurs in the process. For example, an error can occur when
  receiving or uploading a user portrait. The \texttt{userAction}
  argument distinguishes the specific action in which the error
  occurred.
\end{itemize}

\section{DDL Form Screenlet for
Android}\label{ddl-form-screenlet-for-android}

\subsection{Requirements}\label{requirements-4}

\begin{itemize}
\tightlist
\item
  Android SDK 4.1 (API Level 16) or above
\item
  Liferay Portal 6.2 CE/EE, Liferay CE Portal 7.0/7.1, Liferay DXP
\item
  Liferay Screens Compatibility app
  (\href{http://www.liferay.com/marketplace/-/mp/application/54365664}{CE}
  or
  \href{http://www.liferay.com/marketplace/-/mp/application/54369726}{EE/DXP}).
  This app is preinstalled in Liferay CE Portal 7.0/7.1 and Liferay DXP.
\end{itemize}

\subsection{Compatibility}\label{compatibility-4}

\begin{itemize}
\tightlist
\item
  Android SDK 4.1 (API Level 16) or above
\end{itemize}

\subsection{Xamarin Requirements}\label{xamarin-requirements-4}

\begin{itemize}
\tightlist
\item
  Visual Studio 7.2
\item
  Mono .NET framework 5.4.1.6
\end{itemize}

\subsection{Features}\label{features-4}

DDL Form Screenlet shows a set of fields that can be filled in by the
user. Initial or existing values can be shown in the fields. Fields of
the following data types are supported:

\begin{itemize}
\tightlist
\item
  \emph{Boolean}: A two state value typically represented by a checkbox.
\item
  \emph{Date}: A formatted date value. The format depends on the
  device's current locale.
\item
  \emph{Decimal, Integer, and Number}: A numeric value.
\item
  \emph{Documents \& Media}: A file stored on the device. It can be
  uploaded to a specific portal repository.
\item
  \emph{Radio}: A set of options to choose from. A single option must be
  chosen.
\item
  \emph{Select}: A selection box of options to choose from. A single
  option must be chosen.
\item
  \emph{Text}: A single line of text.
\item
  \emph{Text Area}: Multiple lines of text.
\end{itemize}

The DDL Form Screenlet also supports the following features:

\begin{itemize}
\tightlist
\item
  Stored records can support a specific workflow.
\item
  A Submit button can be shown at the end of the form.
\item
  Required values and validation for fields can be used.
\item
  Users can traverse the form fields from the keyboard.
\item
  Supports i18n in record values and labels.
\end{itemize}

There are also a few limitations that you should be aware of when using
DDL Form Screenlet. They are listed here:

\begin{itemize}
\tightlist
\item
  Nested fields in the data definition aren't supported.
\item
  Selection of multiple items in the Radio and Select data types isn't
  supported.
\end{itemize}

\subsection{JSON Services Used}\label{json-services-used-4}

Screenlets in Liferay Screens call JSON web services in the portal. This
Screenlet calls the following services and methods.

\noindent\hrulefill

\begin{longtable}[]{@{}
  >{\raggedright\arraybackslash}p{(\columnwidth - 4\tabcolsep) * \real{0.3889}}
  >{\raggedright\arraybackslash}p{(\columnwidth - 4\tabcolsep) * \real{0.3333}}
  >{\raggedright\arraybackslash}p{(\columnwidth - 4\tabcolsep) * \real{0.2778}}@{}}
\toprule\noalign{}
\begin{minipage}[b]{\linewidth}\raggedright
Service
\end{minipage} & \begin{minipage}[b]{\linewidth}\raggedright
Method
\end{minipage} & \begin{minipage}[b]{\linewidth}\raggedright
Notes
\end{minipage} \\
\midrule\noalign{}
\endhead
\bottomrule\noalign{}
\endlastfoot
\texttt{ScreensddlrecordService} (Screens compatibility plugin) &
\texttt{getDDMStructureVersion} & Load form \\
\texttt{ScreensddlrecordService} (Screens compatibility plugin) &
\texttt{getDdlRecord} & Load record \\
\texttt{DLAppService} & \texttt{addFileEntry} & Upload document \\
\texttt{DDLRecordService} & \texttt{addRecord} & Submit form \\
\texttt{DDLRecordService} & \texttt{updateRecord} & Update form \\
\end{longtable}

\noindent\hrulefill

\subsection{Module}\label{module-4}

\begin{itemize}
\tightlist
\item
  DDL
\end{itemize}

\subsection{Views}\label{views-4}

\begin{itemize}
\tightlist
\item
  Default
\item
  Material
\end{itemize}

The Default View uses a standard vertical \texttt{ScrollView} to show a
scrollable list of fields. Other Views may use different components,
such as \texttt{ViewPager} or others, to show the fields. You can find a
sample of this implementation in the \texttt{DDLFormScreenletPagerView}
class.

\begin{figure}
\centering
\includegraphics{./images/screens-android-ddlform.png}
\caption{DDL Form Screenlet's Default (left) and Material (right)
Views.}
\end{figure}

\subsubsection{Editor Types}\label{editor-types}

Each field defines an editor type. You must define each editor type's
layout by using the following attributes:

\begin{itemize}
\tightlist
\item
  \texttt{checkboxFieldLayoutId}: The layout to use for Boolean fields.
\item
  \texttt{dateFieldLayoutId}: The layout to use for Date fields.
\item
  \texttt{numberFieldLayoutId}: The layout to use for Number, Decimal,
  or Integer fields.
\item
  \texttt{radioFieldLayoutId}: The layout to use for Radio fields.
\item
  \texttt{selectFieldLayoutId}: The layout to use for Select fields.
\item
  \texttt{textFieldLayoutId}: The layout to use for Text fields.
\item
  \texttt{textAreaFieldLayoutId}: The layout to use for Text Box fields.
\item
  \texttt{textDocumentFieldLayoutId}: The layout to use for Documents \&
  Media fields.
\end{itemize}

If you don't define the editor type's layout in DDL Form Screenlet's
attributes, the default layout \texttt{ddlfield\_xxx\_default} is used,
where \texttt{xxx} is the name of the editor type. It's important to
note that you can change the layout used with any editor type at any
point.

\subsubsection{Custom Editors}\label{custom-editors}

If you want to have a unique appearance for one specific field, you can
customize your field's editor View by calling the Screenlet's
\texttt{setCustomFieldLayoutId(fieldName,\ layoutId)} method, where the
first parameter is the name of the field to customize and the second
parameter is the layout to use. You can also create custom editor Views.
For examples of this, see the files
\texttt{ddlfield\_custom\_rating\_number.xml} and
\texttt{CustomRatingNumberView.java}.

\subsection{Activity Configuration}\label{activity-configuration-1}

DDL Form Screenlet needs the following user permissions:

\begin{verbatim}
<uses-permission android:name="android.permission.CAMERA"/>
<uses-permission android:name="android.permission.WRITE_EXTERNAL_STORAGE"/>
\end{verbatim}

Both are used by the Documents and Media fields to take a picture/video
and store it locally before uploading it to the portal. The Documents
and Media fields also need to override the \texttt{onActivityResult}
method to receive the picture/video information. Here's an example
implementation:

\begin{verbatim}
@Override
protected void onActivityResult(int requestCode, int resultCode, Intent data) {
    super.onActivityResult(requestCode, resultCode, data);

    screenlet.startUploadByPosition(requestCode);
}
\end{verbatim}

\subsection{Portal Configuration}\label{portal-configuration-4}

Before using DDL Form Screenlet, you should make sure that Dynamic Data
Lists and Data Types are configured properly in the portal. Refer to the
\href{/docs/7-1/user/-/knowledge_base/u/creating-data-definitions}{Creating
Data Definitions} and
\href{/docs/7-1/user/-/knowledge_base/u/creating-data-lists}{Creating
Data Lists} sections of the User Guide for more details. If Workflow is
required, it must also be configured. See the
\href{/docs/7-1/user/-/knowledge_base/u/workflow}{Using Workflow}
section of the User Guide for details.

\subsubsection{Permissions}\label{permissions}

To use DDL Form Screenlet to add new records, you must grant the Add
Record permission in the Dynamic Data List in the portal. If you want to
use DDL Form Screenlet to view or edit record values, you must also
grant the View and Update permissions, respectively. The Add Record,
View, and Update permissions are highlighted by the red boxes in the
following screenshot:

\begin{figure}
\centering
\includegraphics{./images/screens-portal-permission-ddl.png}
\caption{The permissions for adding, viewing, and editing DDL records.}
\end{figure}

Also, if your form includes at least one Documents and Media field, you
must grant permissions in the target repository and folder. For more
details, see the \texttt{repositoryId} and \texttt{folderId} attributes
below.

\begin{figure}
\centering
\includegraphics{./images/screens-portal-permission-folder-add.png}
\caption{The permission for adding a document to a Documents and Media
folder.}
\end{figure}

For more details, see the User Guide sections
\href{/docs/7-1/user/-/knowledge_base/u/creating-data-definitions}{Creating
Data Definitions},
\href{/docs/7-1/user/-/knowledge_base/u/creating-data-lists}{Creating
Data Lists}, and \href{/docs/7-1/user/-/knowledge_base/u/workflow}{Using
Workflow}.

\subsection{Offline}\label{offline-4}

This Screenlet supports offline mode so it can function without a
network connection. For more information on how offline mode works, see
the
\href{/docs/7-1/tutorials/-/knowledge_base/t/architecture-of-offline-mode-in-liferay-screens}{tutorial
on its architecture}.

When loading the form or record, the Screenlet supports the following
offline mode policies:

\noindent\hrulefill

Policy \textbar{} What happens \textbar{} When to use \textbar{}
\texttt{REMOTE\_ONLY} \textbar{} The Screenlet loads the form or record
from the portal. If a connection issue occurs, the Screenlet uses the
listener to notify the developer about the error. If the Screenlet loads
the form or record, it stores the received data (record structure and
data) in the local cache for later use. \textbar{} Use this policy when
you always need to show updated data, and show nothing when there's no
connection. \textbar{} \texttt{CACHE\_ONLY} \textbar{} The Screenlet
loads the form or record from the local cache. If the form or record
isn't there, the Screenlet uses the listener to notify the developer
about the error. \textbar{} Use this policy when you always need to show
local data, without retrieving remote information under any
circumstance. \textbar{} \texttt{REMOTE\_FIRST} \textbar{} The Screenlet
requests the form or record from the portal. The Screenlet shows the
record or form to the user and stores it in the local cache for later
use. If a connection issue occurs, the Screenlet retrieves the form or
record from the local cache. If the form or record doesn't exist there,
the Screenlet uses the listener to notify the developer about the error.
\textbar{} Use this policy to show the most recent version of the data
when connected, but show an outdated version when there's no connection.
\textbar{} \texttt{CACHE\_FIRST} \textbar{} If the form or record exists
in the local cache, the Screenlet loads it from there. If it doesn't
exist there, the Screenlet requests it from the portal and notifies the
developer about any errors that occur (including connectivity errors).
\textbar{} Use this policy to save bandwidth and loading time in case
you have local (but probably outdated) data. \textbar{}

\noindent\hrulefill

When editing the record, the Screenlet supports the following offline
mode policies:

\noindent\hrulefill

Policy \textbar{} What happens \textbar{} When to use \textbar{}
\texttt{REMOTE\_ONLY} \textbar{} The Screenlet sends the record to the
portal. If a connection issue occurs, the Screenlet uses the listener to
notify the developer about the error, but it also discards the record.
\textbar{} Use this policy to make sure the portal always has the most
recent version of the record. \textbar{} \texttt{CACHE\_ONLY} \textbar{}
The Screenlet stores the record in the local cache. \textbar{} Use this
policy when you need to save the data locally, but don't want to update
the data in the portal (update or add record). \textbar{}
\texttt{REMOTE\_FIRST} \textbar{} The Screenlet sends the record to the
portal. If this succeeds, it also stores the record in the local cache
for later usage. If a connection issue occurs, then Screenlet stores the
record in the local cache with the \emph{dirty flag} enabled. This
causes the synchronization process to send the record to the portal when
it runs. \textbar{} Use this policy when you need to make sure the
Screenlet sends the record to the portal as soon as the connection is
restored. \textbar{} \texttt{CACHE\_FIRST} \textbar{} The Screenlet
stores the record in the local cache and then sends it to the remote
portal. If a connection issue occurs, then Screenlet stores the record
in the local cache with the \emph{dirty flag} enabled. This causes the
the synchronization process to send the record to the portal when it
runs. \textbar{} Use this policy when you need to make sure the
Screenlet sends the record to the portal as soon as the connection is
restored. Compared to \texttt{REMOTE\_FIRST}, this policy always stores
the record in the cache. The \texttt{REMOTE\_FIRST} policy only stores
the record in the event of a network error. \textbar{}

\noindent\hrulefill

\subsection{Required Attributes}\label{required-attributes-4}

\begin{itemize}
\tightlist
\item
  \texttt{structureId}
\item
  \texttt{recordSetId}
\end{itemize}

\subsection{Attributes}\label{attributes-4}

\noindent\hrulefill

Attribute \textbar{} Data Type \textbar{} Explanation \textbar{}
\texttt{layoutId} \textbar{} \texttt{@layout} \textbar{} The layout to
use to show the View. \textbar{} \texttt{checkboxFieldLayoutId}
\textbar{} \texttt{@layout} \textbar{} The layout to use to show the
view for Boolean fields. \textbar{} \texttt{dateFieldLayoutId}
\textbar{} \texttt{@layout} \textbar{} The layout to use to show the
view for Date fields. \textbar{} \texttt{numberFieldLayoutId} \textbar{}
\texttt{@layout} \textbar{} The layout to use to show the view for
Number, Decimal, and Integer fields. \textbar{}
\texttt{radioFieldLayoutId} \textbar{} \texttt{@layout} \textbar{} The
layout to use to show the view for Radio fields. \textbar{}
\texttt{selectFieldLayoutId} \textbar{} \texttt{@layout} \textbar{} The
layout to use to show the view for Select fields. \textbar{}
\texttt{textFieldLayoutId} \textbar{} \texttt{@layout} \textbar{} The
layout to use to show the view for Text fields. \textbar{}
\texttt{textAreaFieldLayoutId} \textbar{} \texttt{@layout} \textbar{}
The layout to use to show the view for Text Box fields. \textbar{}
\texttt{documentFieldLayoutId} \textbar{} \texttt{@layout} \textbar{}
The layout to use to show the view for Documents \& Media fields.
\textbar{} \texttt{structureId} \textbar{} \texttt{number} \textbar{}
The ID of a data definition in your Liferay site. To find the IDs for
your data definitions, click \emph{Admin} → \emph{Content} from the
Dockbar. Then click \emph{Dynamic Data Lists} on the left and click the
\emph{Manage Data Definitions} button. The ID of each data definition is
in the ID column of the table. \textbar{} \texttt{groupId} \textbar{}
\texttt{number} \textbar{} The ID of the site (group) where the record
is stored. If this value is \texttt{0}, the \texttt{groupId} specified
in \texttt{LiferayServerContext} is used. \textbar{}
\texttt{recordSetId} \textbar{} \texttt{number} \textbar{} A dynamic
data list's ID. To find your dynamic data lists' IDs, click \emph{Admin}
→ \emph{Content} from the Dockbar. Then click \emph{Dynamic Data Lists}
on the left. Each dynamic data list's ID is in the ID column of the
table. \textbar{} \texttt{recordId} \textbar{} \texttt{number}
\textbar{} The ID of the record you want to show. You can also allow the
record's values to be edited. This ID can be obtained from other methods
or listeners. \textbar{} \texttt{repositoryId} \textbar{}
\texttt{number} \textbar{} The ID of the Documents and Media repository
to upload to. If this value is \texttt{0}, the default repository for
the site specified by \texttt{groupId} is used. \textbar{}
\texttt{folderId} \textbar{} \texttt{number} \textbar{} The ID of the
folder where Documents and Media files are uploaded. If this value is
\texttt{0}, the root is used. \textbar{} \texttt{filePrefix} \textbar{}
\texttt{string} \textbar{} The prefix to attach to the names of files
uploaded to a Documents and Media repository. The upload date followed
by the original file name is appended following the prefix. \textbar{}
\texttt{autoLoad} \textbar{} \texttt{boolean} \textbar{} Sets whether
the form loads when the Screenlet is shown. If \texttt{recordId} is set,
the record value is loaded together with the form definition. The
default value is \texttt{false}. \textbar{}
\texttt{autoScrollOnValidation} \textbar{} \texttt{boolean} \textbar{}
Sets whether the form automatically scrolls to the first failed field
when validation is used. The default value is \texttt{true}. \textbar{}
\texttt{showSubmitButton} \textbar{} \texttt{boolean} \textbar{} Sets
whether the form shows a submit button at the bottom. If this is set to
\texttt{false}, you should call the \texttt{submitForm()} method. The
default value is \texttt{true}. \textbar{} \texttt{cachePolicy}
\textbar{} \texttt{string} \textbar{} The offline mode setting. See the
\href{/docs/7-1/reference/-/knowledge_base/r/ddlformscreenlet-for-android\#offline}{Offline
section} for details. \textbar{}

\noindent\hrulefill

\subsection{Methods}\label{methods-1}

\noindent\hrulefill

Method \textbar{} Return Type \textbar{} Explanation \textbar{}
\texttt{loadForm()} \textbar{} \texttt{void} \textbar{} Starts the
request to load the form definition. The form fields are shown when the
response is received. \textbar{} \texttt{loadRecord()} \textbar{}
\texttt{void} \textbar{} Starts the request to load the record specified
by \texttt{recordId}. If needed, the form definition also loads. When
the response is received, the form fields are shown filled with record
values. \textbar{} \texttt{load()} \textbar{} \texttt{void} \textbar{}
Starts the request to load the record if \texttt{recordId} is specified.
Otherwise, the form definition is loaded. \textbar{}
\texttt{submitForm()} \textbar{} \texttt{void} \textbar{} Starts the
request to submit form values to the dynamic data list specified by
\texttt{recordSetId}. If the record is new, a new record is added. If
\texttt{loadRecord} is used to retrieve the record, or the record
already exists, its values are updated. Fields are validated prior to
the request. If validation fails, the validation errors are shown and
the request is terminated. \textbar{}

\noindent\hrulefill

\subsection{Listener}\label{listener-4}

DDL Form Screenlet delegates some events to an object that implements to
the \texttt{DDLFormListener} interface. This interface lets you
implement the following methods:

\begin{itemize}
\item
  \texttt{onDDLFormLoaded(Record\ record)}: Called when the form
  definition successfully loads.
\item
  \texttt{onDDLFormRecordLoaded(Record\ record,\ Map\textless{}String,\ Object\textgreater{}\ valuesAndAttributes)}:
  Called when the form record data successfully loads.
\item
  \texttt{onDDLFormRecordAdded(Record\ record)}: Called when the form
  record is successfully added.
\item
  \texttt{onDDLFormRecordUpdated(Record\ record)}: Called when the form
  record data successfully updates.
\item
  \texttt{error(Exception\ e,\ String\ userAction)}: Called when an
  error occurs in the process. For example, this method is called when
  an error occurs while loading a form definition or record, or adding
  or updating a record. The \texttt{userAction} variable distinguishes
  these events.
\item
  \texttt{onDDLFormDocumentUploaded(DocumentField\ field)}: Called when
  a specified document field's upload completes.
\item
  \texttt{onDDLFormDocumentUploadFailed(DocumentField\ field,\ Exception\ e)}:
  Called when a specified document field's upload fails.
\end{itemize}

\section{DDL List Screenlet for
Android}\label{ddl-list-screenlet-for-android}

\subsection{Requirements}\label{requirements-5}

\begin{itemize}
\tightlist
\item
  Android SDK 4.1 (API Level 16) or above
\item
  Liferay Portal 6.2 CE/EE, Liferay CE Portal 7.0/7.1, Liferay DXP
\item
  Liferay Screens Compatibility app
  (\href{http://www.liferay.com/marketplace/-/mp/application/54365664}{CE}
  or
  \href{http://www.liferay.com/marketplace/-/mp/application/54369726}{EE/DXP}).
  This app is preinstalled in Liferay CE Portal 7.0/7.1 and Liferay DXP.
\end{itemize}

\subsection{Compatibility}\label{compatibility-5}

\begin{itemize}
\tightlist
\item
  Android SDK 4.1 (API Level 16) or above
\end{itemize}

\subsection{Xamarin Requirements}\label{xamarin-requirements-5}

\begin{itemize}
\tightlist
\item
  Visual Studio 7.2
\item
  Mono .NET framework 5.4.1.6
\end{itemize}

\subsection{Features}\label{features-5}

The DDL List Screenlet has the following features:

\begin{itemize}
\tightlist
\item
  Shows a scrollable collection of Dynamic Data List (DDL) records.
\item
  Implements
  \href{http://www.iosnomad.com/blog/2014/4/21/fluent-pagination}{fluent
  pagination} with configurable page size.
\item
  Allows record filtering by creator.
\item
  Supports i18n in record values.
\end{itemize}

\subsection{JSON Services Used}\label{json-services-used-5}

Screenlets in Liferay Screens call JSON web services in the portal. This
Screenlet calls the following services and methods.

\noindent\hrulefill

\begin{longtable}[]{@{}
  >{\raggedright\arraybackslash}p{(\columnwidth - 4\tabcolsep) * \real{0.3889}}
  >{\raggedright\arraybackslash}p{(\columnwidth - 4\tabcolsep) * \real{0.3333}}
  >{\raggedright\arraybackslash}p{(\columnwidth - 4\tabcolsep) * \real{0.2778}}@{}}
\toprule\noalign{}
\begin{minipage}[b]{\linewidth}\raggedright
Service
\end{minipage} & \begin{minipage}[b]{\linewidth}\raggedright
Method
\end{minipage} & \begin{minipage}[b]{\linewidth}\raggedright
Notes
\end{minipage} \\
\midrule\noalign{}
\endhead
\bottomrule\noalign{}
\endlastfoot
\texttt{ScreensddlrecordService} (Screens compatibility plugin) &
\texttt{getDdlRecords} & With \texttt{ddlRecordSetId}, or
\texttt{ddlRecordSetId} and \texttt{userId} \\
\texttt{ScreensddlrecordService} (Screens compatibility plugin) &
\texttt{getDdlRecordsCount} & \\
\end{longtable}

\noindent\hrulefill

\subsection{Module}\label{module-5}

\begin{itemize}
\tightlist
\item
  DDL
\end{itemize}

\subsection{Views}\label{views-5}

\begin{itemize}
\tightlist
\item
  Default
\item
  Material
\end{itemize}

The Default View uses a standard \texttt{RecyclerView} to show the
scrollable list. Other Views may use a different component, such as
\texttt{ViewPager} or others, to show the items.

\begin{figure}
\centering
\includegraphics{./images/screens-android-ddllist.png}
\caption{The DDL List Screenlet using the Default and Material Views.}
\end{figure}

\subsection{Portal Configuration}\label{portal-configuration-5}

DDLs and Data Types should be configured in the portal before using DDL
List Screenlet. For more details, see the Liferay User Guide sections
\href{/docs/7-1/user/-/knowledge_base/u/creating-data-definitions}{Creating
Data Definitions} and
\href{/docs/7-1/user/-/knowledge_base/u/creating-data-lists}{Creating
Data Lists} .

Also, to allow remote calls without the \texttt{userId}, the Liferay
Screens Compatibility app must be installed in your Liferay instance.
You can find this app on
\href{https://web.liferay.com/marketplace}{Liferay Marketplace}.

\subsection{Offline}\label{offline-5}

This Screenlet supports offline mode so it can function without a
network connection. For more information on how offline mode works, see
the
\href{/docs/7-1/tutorials/-/knowledge_base/t/architecture-of-offline-mode-in-liferay-screens}{tutorial
on its architecture}.

\noindent\hrulefill

Policy \textbar{} What happens \textbar{} When to use \textbar{}
\texttt{REMOTE\_ONLY} \textbar{} The Screenlet loads the list from the
portal. If a connection issue occurs, the Screenlet uses the listener to
notify the developer about the error. If the Screenlet successfully
loads the list, it stores the data in the local cache for later use.
\textbar{} Use this policy when you always need to show updated data,
and show nothing when there's no connection. \textbar{}
\texttt{CACHE\_ONLY} \textbar{} The Screenlet loads the list from the
local cache. If the list isn't there, the Screenlet uses the listener to
notify the developer about the error. \textbar{} Use this policy when
you always need to show local data, without retrieving remote
information under any circumstance. \textbar{} \texttt{REMOTE\_FIRST}
\textbar{} The Screenlet loads the list from the portal. If this
succeeds, the Screenlet shows the list to the user and stores it in the
local cache for later use. If a connection issue occurs, the Screenlet
retrieves the list from the local cache. If the list doesn't exist
there, the Screenlet uses the listener to notify the developer about the
error. \textbar{} Use this policy to show the most recent version of the
data when connected, but show an outdated version when there's no
connection. \textbar{} \texttt{CACHE\_FIRST} \textbar{} The Screenlet
loads the list from the local cache. If the list isn't there, the
Screenlet requests it from the portal and notifies the developer about
any errors that occur (including connectivity errors). \textbar{} Use
this policy to save bandwidth and loading time in case you have local
(but probably outdated) data. \textbar{}

\noindent\hrulefill

\subsection{Required Attributes}\label{required-attributes-5}

\begin{itemize}
\tightlist
\item
  \texttt{recordSetId}
\item
  \texttt{labelFields}
\end{itemize}

\subsection{Attributes}\label{attributes-5}

\noindent\hrulefill

Attribute \textbar{} Data type \textbar{} Explanation \textbar{}
\texttt{layoutId} \textbar{} \texttt{@layout} \textbar{} The layout to
use to show the View. \textbar{} \texttt{autoLoad} \textbar{}
\texttt{boolean} \textbar{} Defines whether the list should be loaded
when it's presented on the screen. The default value is \texttt{true}.
\textbar{} \texttt{recordSetId} \textbar{} \texttt{number} \textbar{}
The ID of the DDL being called. To find your DDLs' IDs, click
\emph{Admin} → \emph{Content} from the Dockbar. Then click \emph{Dynamic
Data Lists} on the left. Each DDL's ID is in the ID column of the table.
\textbar{} \texttt{userId} \textbar{} \texttt{number} \textbar{} The ID
of the user to filter records on. Records aren't filtered if the
\texttt{userId} is \texttt{0}. The default value is \texttt{0}.
\textbar{} \texttt{cachePolicy} \textbar{} \texttt{string} \textbar{}
The offline mode setting. See the
\href{/docs/7-1/reference/-/knowledge_base/r/ddllistscreenlet-for-android\#offline}{Offline
section} for details. \textbar{} \texttt{firstPageSize} \textbar{}
\texttt{number} \textbar{} The number of items to retrieve from the
server for display on the first page. The default value is \texttt{50}.
\textbar{} \texttt{pageSize} \textbar{} \texttt{number} \textbar{} The
number of items to retrieve from the server for display on the second
and subsequent pages. The default value is \texttt{25}. \textbar{}
\texttt{labelFields} \textbar{} \texttt{string} \textbar{} The
comma-separated names of the DDL fields to show. Refer to the list's
data definition to find the field names. For more information on this,
see
\href{/docs/7-1/user/-/knowledge_base/u/creating-data-definitions}{Creating
Data Definitions}. Note that the appearance of these values in your app
depends on the \texttt{layoutId} set. \textbar{} \texttt{obcClassName}
\textbar{} \texttt{string} \textbar{} The name of the
\texttt{OrderByComparator} class to use to sort the results. Omit this
property if you don't want to sort the results.
\href{https://github.com/liferay/liferay-portal/tree/master/modules/apps/forms-and-workflow/dynamic-data-lists/dynamic-data-lists-api/src/main/java/com/liferay/dynamic/data/lists/util/comparator}{Click
here} to see some comparator classes. Note, however, that not all of
these classes can be used with \texttt{obcClassName}. You can only use
comparator classes that extend
\texttt{OrderByComparator\textless{}DDLRecord\textgreater{}}. You can
also create your own comparator classes that extend
\texttt{OrderByComparator\textless{}DDLRecord\textgreater{}}. \textbar{}

\noindent\hrulefill

\subsection{Methods}\label{methods-2}

\noindent\hrulefill

Method \textbar{} Return \textbar{} Explanation \textbar{}
\texttt{loadPage(pageNumber)} \textbar{} \texttt{void} \textbar{} Starts
the request to load the specified page of records. The page is shown
when the response is received. \textbar{}

\noindent\hrulefill

\subsection{Listener}\label{listener-5}

DDL List Screenlet delegates some events to an object or a class that
implements
\href{https://github.com/liferay/liferay-screens/blob/master/android/library/src/main/java/com/liferay/mobile/screens/base/list/BaseListListener.java}{the
\texttt{BaseListListener} interface}. This interface lets you implement
the following methods:

\begin{itemize}
\item
  \texttt{onListPageFailed(int\ startRow,\ Exception\ e)}: Called when
  the server call to retrieve a page of items fails. This method's
  arguments include the \texttt{Exception} generated when the server
  call fails.
\item
  \texttt{onListPageReceived(int\ startRow,\ int\ endRow,\ List\textless{}Record\textgreater{}\ records,\ int\ rowCount)}:
  Called when the server call to retrieve a page of items succeeds. Note
  that this method may be called more than once; once for each page
  received. Because \texttt{startRow} and \texttt{endRow} change for
  each page, a \texttt{startRow} of \texttt{0} corresponds to the first
  item on the first page.
\item
  \texttt{onListItemSelected(Record\ records,\ View\ view)}: Called when
  an item is selected in the list. This method's arguments include the
  selected list item (\texttt{Record}).
\item
  \texttt{error(Exception\ e,\ String\ userAction)}: Called when an
  error occurs in the process. The \texttt{userAction} argument
  distinguishes the specific action in which the error occurred.
\end{itemize}

\section{Asset List Screenlet for
Android}\label{asset-list-screenlet-for-android}

\subsection{Requirements}\label{requirements-6}

\begin{itemize}
\tightlist
\item
  Android SDK 4.1 (API Level 16) or above
\item
  Liferay Portal 6.2 CE/EE, Liferay CE Portal 7.0/7.1, Liferay DXP
\item
  Liferay Screens Compatibility app
  (\href{http://www.liferay.com/marketplace/-/mp/application/54365664}{CE}
  or
  \href{http://www.liferay.com/marketplace/-/mp/application/54369726}{EE/DXP}).
  This app is preinstalled in Liferay CE Portal 7.0/7.1 and Liferay DXP.
\end{itemize}

\subsection{Compatibility}\label{compatibility-6}

\begin{itemize}
\tightlist
\item
  Android SDK 4.1 (API Level 16) or above
\end{itemize}

\subsection{Xamarin Requirements}\label{xamarin-requirements-6}

\begin{itemize}
\tightlist
\item
  Visual Studio 7.2
\item
  Mono .NET framework 5.4.1.6
\end{itemize}

\subsection{Features}\label{features-6}

The Asset List Screenlet can be used to show
\href{/tutorials/-/knowledge_base/7-1/asset-framework}{asset} lists from
a Liferay instance. For example, you can use the Screenlet to show a
scrollable list of assets. It also implements
\href{http://www.iosnomad.com/blog/2014/4/21/fluent-pagination}{fluent
pagination} with configurable page size. The Asset List Screenlet can
show assets belonging to the following classes:

\begin{itemize}
\tightlist
\item
  \texttt{BlogsEntry}
\item
  \texttt{BookmarksEntry}
\item
  \texttt{BookmarksFolder}
\item
  \texttt{CalendarEvent}
\item
  \texttt{DLFileEntry}
\item
  \texttt{DDLRecord}
\item
  \texttt{DDLRecordSet}
\item
  \texttt{Group}
\item
  \texttt{JournalArticle} (Web Content)
\item
  \texttt{JournalFolder}
\item
  \texttt{Layout}
\item
  \texttt{LayoutRevision}
\item
  \texttt{MBThread}
\item
  \texttt{MBCategory}
\item
  \texttt{MBDiscussion}
\item
  \texttt{MBMailingList}
\item
  \texttt{Organization}
\item
  \texttt{User}
\item
  \texttt{WikiPage}
\item
  \texttt{WikiPageResource}
\item
  \texttt{WikiNode}
\end{itemize}

The Asset List Screenlet also supports i18n in asset values.

\subsection{JSON Services Used}\label{json-services-used-6}

Screenlets in Liferay Screens call JSON web services in the portal. This
Screenlet calls the following services and methods.

\noindent\hrulefill

\begin{longtable}[]{@{}
  >{\raggedright\arraybackslash}p{(\columnwidth - 4\tabcolsep) * \real{0.3889}}
  >{\raggedright\arraybackslash}p{(\columnwidth - 4\tabcolsep) * \real{0.3333}}
  >{\raggedright\arraybackslash}p{(\columnwidth - 4\tabcolsep) * \real{0.2778}}@{}}
\toprule\noalign{}
\begin{minipage}[b]{\linewidth}\raggedright
Service
\end{minipage} & \begin{minipage}[b]{\linewidth}\raggedright
Method
\end{minipage} & \begin{minipage}[b]{\linewidth}\raggedright
Notes
\end{minipage} \\
\midrule\noalign{}
\endhead
\bottomrule\noalign{}
\endlastfoot
\texttt{ScreensddlrecordService} (Screens compatibility plugin) &
\texttt{getAssetEntries} & With \texttt{entryQuery} \\
\texttt{ScreensddlrecordService} (Screens compatibility plugin) &
\texttt{getAssetEntries} & With \texttt{companyId}, \texttt{groupId},
and \texttt{portletItemName} \\
\texttt{AssetEntryService} & \texttt{getEntriesCount} & \\
\end{longtable}

\noindent\hrulefill

\subsection{Module}\label{module-6}

\begin{itemize}
\tightlist
\item
  None
\end{itemize}

\subsection{Views}\label{views-6}

\begin{itemize}
\tightlist
\item
  Default
\item
  Material
\end{itemize}

The Default Views use a standard \texttt{RecyclerView} to show the
scrollable list. Other Views may use a different component, such as
\texttt{ViewPager} or others, to show the items.

\begin{figure}
\centering
\includegraphics{./images/screens-android-assetlist.png}
\caption{Asset List Screenlet using the Default (left) and Material
(right) Views.}
\end{figure}

\subsection{Portal Configuration}\label{portal-configuration-6}

Dynamic Data Lists (DDL) and Data Types should be configured properly in
the portal. Refer to the
\href{/docs/7-1/user/-/knowledge_base/u/creating-data-definitions}{Creating
Data Definitions}\\
and
\href{/docs/7-1/user/-/knowledge_base/u/creating-data-lists}{Creating
Data Lists} sections of the User Guide for more details.

Also, to allow remote calls without the \texttt{userId}, the Liferay
Screens Compatibility app must be installed in your Liferay instance.
You can find this app on
\href{https://web.liferay.com/marketplace}{Liferay Marketplace}.

\subsection{Offline}\label{offline-6}

This Screenlet supports offline mode so it can function without a
network connection. For more information on how offline mode works, see
the
\href{/docs/7-1/tutorials/-/knowledge_base/t/architecture-of-offline-mode-in-liferay-screens}{tutorial
on its architecture}.

\noindent\hrulefill

Policy \textbar{} What happens \textbar{} When to use \textbar{}
\texttt{REMOTE\_ONLY} \textbar{} The Screenlet loads the list from the
portal. If a connection issue occurs, the Screenlet uses the listener to
notify the developer about the error. If the Screenlet successfully
loads the list, it stores the data in the local cache for later use.
\textbar{} Use this policy when you always need to show updated data,
and show nothing when there's no connection. \textbar{}
\texttt{CACHE\_ONLY} \textbar{} The Screenlet loads the list from the
local cache. If the list isn't there, the Screenlet uses the listener to
notify the developer about the error. \textbar{} Use this policy when
you always need to show local data, without retrieving remote
information under any circumstance. \textbar{} \texttt{REMOTE\_FIRST}
\textbar{} The Screenlet loads the list from the portal. If this
succeeds, the Screenlet shows the list to the user and stores it in the
local cache for later use. If a connection issue occurs, the Screenlet
retrieves the list from the local cache. If the list doesn't exist
there, the Screenlet uses the listener to notify the developer about the
error. \textbar{} Use this policy to show the most recent version of the
data when connected, but show an outdated version when there's no
connection. \textbar{} \texttt{CACHE\_FIRST} \textbar{} The Screenlet
loads the list from the local cache. If the list isn't there, the
Screenlet requests it from the portal and notifies the developer about
any errors that occur (including connectivity errors). \textbar{} Use
this policy to save bandwidth and loading time in case you have local
(but probably outdated) data. \textbar{}

\noindent\hrulefill

\subsection{Required Attributes}\label{required-attributes-6}

\begin{itemize}
\tightlist
\item
  \texttt{classNameId}
\end{itemize}

If you don't set \texttt{classNameId}, you must set this attribute
instead:

\begin{itemize}
\tightlist
\item
  \texttt{portletItemName}
\end{itemize}

\subsection{Attributes}\label{attributes-6}

\noindent\hrulefill

Attribute \textbar{} Data type \textbar{} Explanation \textbar{}
\texttt{layoutId} \textbar{} \texttt{@layout} \textbar{} The layout to
use to show the View.\textbar{} \texttt{autoLoad} \textbar{}
\texttt{boolean} \textbar{} Whether the list should be loaded when it's
presented on the screen. The default value is \texttt{true}. \textbar{}
\texttt{groupId} \textbar{} \texttt{number} \textbar{} The asset's group
(site) ID. If this value is \texttt{0}, the \texttt{groupId} specified
in \texttt{LiferayServerContext} is used. The default value is
\texttt{0}. \textbar{} \texttt{cachePolicy} \textbar{} \texttt{string}
\textbar{} The offline mode setting. See the
\href{/docs/7-1/reference/-/knowledge_base/r/assetlistscreenlet-for-android\#offline}{Offline
section} for details. \textbar{} \texttt{portletItemName} \textbar{}
\texttt{string} \textbar{} The name of the
\href{/docs/7-1/user/-/knowledge_base/u/configuration-templates}{configuration
template} you used in the Asset Publisher. To use this feature, add an
Asset Publisher to one of your site's pages (it may be a hidden page),
configure the Asset Publisher's filter (in \emph{Configuration} →
\emph{Setup} → \emph{Asset Selection}), and then use the Asset
Publisher's \emph{Configuration Templates} option to save this
configuration with a name. Use this name in this attribute. \textbar{}
\texttt{classNameId} \textbar{} \texttt{number} \textbar{} The asset
class name's ID. Use values from the portal's \texttt{classname\_}
database table. \textbar{} \texttt{firstPageSize} \textbar{}
\texttt{number} \textbar{} The number of items to retrieve from the
server for display on the list's first page. The default value is
\texttt{50}. \textbar{} \texttt{pageSize} \textbar{} \texttt{number}
\textbar{} The number of items to retrieve from the server for display
on the second and subsequent pages. The default value is \texttt{25}.
\textbar{} \texttt{labelFields} \textbar{} \texttt{string} \textbar{}
The comma-separated names of the DDL fields to show. Refer to the list's
data definition to find the field names. For more information on this,
see
\href{/docs/7-1/user/-/knowledge_base/u/creating-data-definitions}{Creating
Data Definitions}. Note that the appearance of these values in your app
depends on the \texttt{layoutId} set. \textbar{}
\texttt{customEntryQuery} \textbar{} \texttt{HashMap} \textbar{} The set
of keys (string) and values (string or number) to be used in the
\href{https://docs.liferay.com/dxp/portal/7.1-latest/javadocs/portal-kernel/com/liferay/asset/kernel/service/persistence/AssetEntryQuery.html}{\texttt{AssetEntryQuery}
object}. These values filter the assets returned by the Liferay
instance. \textbar{}

\noindent\hrulefill

\subsection{Methods}\label{methods-3}

\noindent\hrulefill

Method \textbar{} Return \textbar{} Explanation \textbar{}
\texttt{loadPage(pageNumber)} \textbar{} \texttt{void} \textbar{} Starts
the request to load the specified page of assets. The page is shown when
the response is received. \textbar{}

\noindent\hrulefill

\subsection{Listener}\label{listener-6}

Asset List Screenlet delegates some events to an object or a class that
implements
\href{https://github.com/liferay/liferay-screens/blob/master/android/library/src/main/java/com/liferay/mobile/screens/base/list/BaseListListener.java}{the
\texttt{BaseListListener} interface}. This interface lets you implement
the following methods:

\begin{itemize}
\item
  \texttt{onListPageFailed(int\ startRow,\ Exception\ e)}: Called when
  the server call to retrieve a page of items fails. This method's
  arguments include the \texttt{Exception} generated when the server
  call fails.
\item
  \texttt{onListPageReceived(int\ startRow,\ int\ endRow,\ List\textless{}Model\textgreater{}\ entries,\ int\ rowCount)}:
  Called when the server call to retrieve a page of items succeeds. Note
  that this method may be called more than once; once for each page
  received. Because \texttt{startRow} and \texttt{endRow} change for
  each page, a \texttt{startRow} of \texttt{0} corresponds to the first
  item on the first page.
\item
  \texttt{onListItemSelected(Model\ entries,\ View\ view)}: Called when
  an item is selected in the list. This method's arguments include the
  selected list item (\texttt{Model}).
\item
  \texttt{error(Exception\ e,\ String\ userAction)}: Called when an
  error occurs in the process. The \texttt{userAction} argument
  distinguishes the specific action in which the error occurred.
\end{itemize}

\section{Web Content Display Screenlet for
Android}\label{web-content-display-screenlet-for-android}

\subsection{Requirements}\label{requirements-7}

\begin{itemize}
\tightlist
\item
  Android SDK 4.1 (API Level 16) or above
\item
  Liferay Portal 6.2 CE/EE, Liferay CE Portal 7.0/7.1, Liferay DXP
\item
  Liferay Screens Compatibility app
  (\href{http://www.liferay.com/marketplace/-/mp/application/54365664}{CE}
  or
  \href{http://www.liferay.com/marketplace/-/mp/application/54369726}{EE/DXP}).
  This app is preinstalled in Liferay CE Portal 7.0/7.1 and Liferay DXP.
\end{itemize}

\subsection{Compatibility}\label{compatibility-7}

\begin{itemize}
\tightlist
\item
  Android SDK 4.1 (API Level 16) or above
\end{itemize}

\subsection{Xamarin Requirements}\label{xamarin-requirements-7}

\begin{itemize}
\tightlist
\item
  Visual Studio 7.2
\item
  Mono .NET framework 5.4.1.6
\end{itemize}

\subsection{Features}\label{features-7}

The Web Content Display Screenlet shows web content elements in your
app, rendering the web content's inner HTML. The Screenlet also supports
i18n, rendering contents differently depending on the device's locale.

\subsection{JSON Services Used}\label{json-services-used-7}

Screenlets in Liferay Screens call JSON web services in the portal. This
Screenlet calls the following services and methods.

\noindent\hrulefill

\begin{longtable}[]{@{}
  >{\raggedright\arraybackslash}p{(\columnwidth - 4\tabcolsep) * \real{0.3889}}
  >{\raggedright\arraybackslash}p{(\columnwidth - 4\tabcolsep) * \real{0.3333}}
  >{\raggedright\arraybackslash}p{(\columnwidth - 4\tabcolsep) * \real{0.2778}}@{}}
\toprule\noalign{}
\begin{minipage}[b]{\linewidth}\raggedright
Service
\end{minipage} & \begin{minipage}[b]{\linewidth}\raggedright
Method
\end{minipage} & \begin{minipage}[b]{\linewidth}\raggedright
Notes
\end{minipage} \\
\midrule\noalign{}
\endhead
\bottomrule\noalign{}
\endlastfoot
\texttt{DDMStructureService} & \texttt{getStructure} & \\
\texttt{JournalArticleService} & \texttt{getArticle} & \\
\texttt{JournalArticleService} & \texttt{getArticleContent} & \\
\texttt{ScreensddlrecordService} (Screens compatibility plugin) &
\texttt{getJournalArticleContent} & With \texttt{entryQuery} \\
\end{longtable}

\noindent\hrulefill

\subsection{Module}\label{module-7}

\begin{itemize}
\tightlist
\item
  None
\end{itemize}

\subsection{Views}\label{views-7}

\begin{itemize}
\tightlist
\item
  Default
\end{itemize}

The Default View uses a standard \texttt{WebView} to render the HTML.

\begin{figure}
\centering
\includegraphics{./images/screens-android-webcontentdisplay.png}
\caption{Web Content Display Screenlet using the Default View.}
\end{figure}

\subsection{Portal Configuration}\label{portal-configuration-7}

For the Web Content Display Screenlet to function properly, there should
be web content in the Liferay instance your app connects to. For more
details on web content, see the
\href{/docs/7-1/user/-/knowledge_base/u/introduction-web-content}{web
content} section of the User Guide.

\subsection{Offline}\label{offline-7}

This Screenlet supports offline mode so it can function without a
network connection. For more information on how offline mode works, see
the
\href{/docs/7-1/tutorials/-/knowledge_base/t/architecture-of-offline-mode-in-liferay-screens}{tutorial
on its architecture}. Here are the offline mode policies that you can
use with this Screenlet:

\noindent\hrulefill

Policy \textbar{} What happens \textbar{} When to use \textbar{}
\texttt{REMOTE\_ONLY} \textbar{} The Screenlet loads the content from
the portal. If a connection issue occurs, the Screenlet uses the
listener to notify the developer about the error. If the Screenlet
successfully loads the content, it stores the data in the local cache
for later use. \textbar{} Use this policy when you always need to show
updated content, and show nothing when there's no connection. \textbar{}
\texttt{CACHE\_ONLY} \textbar{} The Screenlet loads the content from the
local cache. If the content isn't there, the Screenlet uses the listener
to notify the developer about the error. \textbar{} Use this policy when
you always need to show local content, without retrieving remote content
under any circumstance. \textbar{} \texttt{REMOTE\_FIRST} \textbar{} The
Screenlet loads the content from the portal. If this succeeds, the
Screenlet shows the content to the user and stores it in the local cache
for later use. If a connection issue occurs, the Screenlet retrieves the
content from the local cache. If the content doesn't exist there, the
Screenlet uses the listener to notify the developer about the error.
\textbar{} Use this policy to show the most recent version of the
content when connected, but show a possibly outdated version when
there's no connection. \textbar{} \texttt{CACHE\_FIRST} \textbar{} The
Screenlet loads the content from the local cache. If the content isn't
there, the Screenlet requests it from the portal and notifies the
developer about any errors that occur (including connectivity errors).
\textbar{} Use this policy to save bandwidth and loading time in case
you have local (but probably outdated) content. \textbar{}

\noindent\hrulefill

\subsection{Required Attributes}\label{required-attributes-7}

\begin{itemize}
\tightlist
\item
  \texttt{articleId}
\end{itemize}

Note that if your web content uses
\href{/docs/7-1/user/-/knowledge_base/u/designing-uniform-content}{structures
and templates}, you can use \texttt{templateId} or \texttt{structureId}
in conjunction with \texttt{articleId}.

\subsection{Attributes}\label{attributes-7}

\noindent\hrulefill

Attribute \textbar{} Data type \textbar{} Explanation \textbar{}
\texttt{layoutId} \textbar{} \texttt{@layout} \textbar{} The layout used
to show the View. \textbar{} \texttt{groupId} \textbar{} \texttt{number}
\textbar{} The site (group) identifier where the asset is stored. If
this value is \texttt{0}, the \texttt{groupId} specified in
\texttt{LiferayServerContext} is used. \textbar{} \texttt{articleId}
\textbar{} \texttt{string} \textbar{} The identifier of the web content
to display. You can find the identifier by clicking \emph{Edit} on the
web content in the portal. \textbar{} \texttt{classPK} \textbar{}
\texttt{number} \textbar{} The corresponding asset's class primary key.
If the web content is an asset (from Asset List Screenlet, for example),
this is the asset's identifier. This attribute is used only if
\texttt{articleId} is empty. \textbar{} \texttt{templateId} \textbar{}
\texttt{number} \textbar{} The identifier of the template used to render
the web content. This only applies to
\href{/docs/7-1/user/-/knowledge_base/u/designing-uniform-content}{structured
web content}. \textbar{} \texttt{structureId} \textbar{} \texttt{number}
\textbar{} The identifier of the \texttt{DDMStructure} used to model the
web content. This parameter lets the Screenlet retrieve and parse the
structure. \textbar{} \texttt{labelFields} \textbar{} \texttt{string}
\textbar{} A comma-delimited list of \texttt{DDMStructure} fields to
display in the Screenlet. \textbar{} \texttt{autoLoad} \textbar{}
\texttt{boolean} \textbar{} Whether the content should be retrieved from
the portal as soon as the screenlet appears. Default value is
\texttt{true}. \textbar{} \texttt{javascriptEnabled} \textbar{}
\texttt{boolean} \textbar{} Enables support for JavaScript. This is
disabled by default. \textbar{} \texttt{cachePolicy} \textbar{}
\texttt{string} \textbar{} The offline mode setting. See the
\href{/docs/7-1/reference/-/knowledge_base/r/webcontentdisplayscreenlet-for-android\#offline}{Offline
section} for details. \textbar{}

\noindent\hrulefill

\subsection{Methods}\label{methods-4}

\noindent\hrulefill

Method \textbar{} Return \textbar{} Explanation \textbar{}
\texttt{load()} \textbar{} \texttt{void} \textbar{} Starts the request
to load the web content. The HTML is rendered when the response is
received. \textbar{} \texttt{getLocalized(String\ name)} \textbar{}
\texttt{String} \textbar{} Returns the value, according to the device
locale, of a field of the \texttt{DDMStructure} used to render the web
content.

\noindent\hrulefill

\subsection{Listener}\label{listener-7}

The Web Content Display Screenlet delegates some events to an object
that implements the \texttt{WebContentDisplayListener} interface. This
interface lets you implement the following methods:

\begin{itemize}
\item
  \texttt{onWebContentReceived(WebContent\ webContent)}: Called when the
  web content's HTML or \texttt{DDMStructure} is received. The HTML is
  available by calling the \texttt{getHtml} method. To make some
  adaptations, the listener may return a modified version of the HTML.
  The original HTML is rendered if the listener returns \texttt{null}.
\item
  \texttt{onUrlClicked(String\ url)}: Called when a URL is clicked.
  Return \texttt{true} to replace the default behavior, or
  \texttt{false} to load the url.
\item
  \texttt{onWebContentTouched(View\ view,\ MotionEvent\ event)}: Called
  when something is touched in the web content. Return \texttt{true} to
  replace the default behavior, or \texttt{false} to keep processing the
  event.
\item
  \texttt{error(Exception\ e,\ String\ userAction)}: Called when an
  error occurs in the process. The \texttt{userAction} argument
  distinguishes the specific action in which the error occurred.
\end{itemize}

\section{Web Content List Screenlet for
Android}\label{web-content-list-screenlet-for-android}

\subsection{Requirements}\label{requirements-8}

\begin{itemize}
\tightlist
\item
  Android SDK 4.1 (API Level 16) or above
\item
  Liferay Portal 6.2 CE/EE, Liferay CE Portal 7.0/7.1, Liferay DXP
\end{itemize}

\subsection{Compatibility}\label{compatibility-8}

\begin{itemize}
\tightlist
\item
  Android SDK 4.1 (API Level 16) or above
\end{itemize}

\subsection{Xamarin Requirements}\label{xamarin-requirements-8}

\begin{itemize}
\tightlist
\item
  Visual Studio 7.2
\item
  Mono .NET framework 5.4.1.6
\end{itemize}

\subsection{Features}\label{features-8}

Web Content List Screenlet has the following features:

\begin{itemize}
\tightlist
\item
  Shows a scrollable collection of
  \href{/docs/7-1/user/-/knowledge_base/u/introduction-web-content}{web
  content} articles.
\item
  Implements
  \href{http://www.iosnomad.com/blog/2014/4/21/fluent-pagination}{fluent
  pagination} with configurable page size.
\item
  Supports i18n in web content values.
\end{itemize}

\subsection{JSON Services Used}\label{json-services-used-8}

Screenlets in Liferay Screens call JSON web services in the portal. This
Screenlet calls the following services and methods.

\noindent\hrulefill

\begin{longtable}[]{@{}lll@{}}
\toprule\noalign{}
Service & Method & Notes \\
\midrule\noalign{}
\endhead
\bottomrule\noalign{}
\endlastfoot
\texttt{JournalArticleService} & \texttt{getJournalArticles} & \\
\texttt{JournalArticleService} & \texttt{getJournalArticlesCount} & \\
\end{longtable}

\noindent\hrulefill

\subsection{Module}\label{module-8}

\begin{itemize}
\tightlist
\item
  None
\end{itemize}

\subsection{Views}\label{views-8}

\begin{itemize}
\tightlist
\item
  Default
\end{itemize}

The Default View uses a standard \texttt{RecyclerView} to show the
scrollable list. Other Views may use a different component, such as
\texttt{ViewPager} or others, to show the items.

\begin{figure}
\centering
\includegraphics{./images/screens-android-webcontentlist.png}
\caption{The Web Content List Screenlet using the Default View.}
\end{figure}

\subsection{Offline}\label{offline-8}

This Screenlet supports offline mode so it can function without a
network connection. For more information on how offline mode works, see
the
\href{/docs/7-1/tutorials/-/knowledge_base/t/architecture-of-offline-mode-in-liferay-screens}{tutorial
on its architecture}. Here are the offline mode policies that you can
use with this Screenlet:

\noindent\hrulefill

Policy \textbar{} What happens \textbar{} When to use \textbar{}
\texttt{REMOTE\_ONLY} \textbar{} The Screenlet loads the list from the
Liferay instance. If a connection issue occurs, the Screenlet uses the
listener to notify the developer about the error. If the Screenlet
successfully loads the list, it stores the data in the local cache for
later use. \textbar{} Use this policy when you always need to show
updated data, and show nothing when there's no connection. \textbar{}
\texttt{CACHE\_ONLY} \textbar{} The Screenlet loads the list from the
local cache. If the list isn't there, the Screenlet uses the listener to
notify the developer about the error. \textbar{} Use this policy when
you always need to show local data, without retrieving remote
information under any circumstance. \textbar{} \texttt{REMOTE\_FIRST}
\textbar{} The Screenlet loads the list from the Liferay instance. If
this succeeds, the Screenlet shows the list to the user and stores it in
the local cache for later use. If a connection issue occurs, the
Screenlet retrieves the list from the local cache. If the list doesn't
exist there, the Screenlet uses the listener to notify the developer
about the error. \textbar{} Use this policy to show the most recent
version of the data when connected, but show a possibly outdated version
when there's no connection. \textbar{} \texttt{CACHE\_FIRST} \textbar{}
The Screenlet loads the list from the local cache. If the list isn't
there, the Screenlet requests it from the Liferay instance and notifies
the developer about any errors that occur (including connectivity
errors). \textbar{} Use this policy to save bandwidth and loading time
in case you have local (but possibly outdated) data. \textbar{}

\noindent\hrulefill

\subsection{Required Attributes}\label{required-attributes-8}

\begin{itemize}
\tightlist
\item
  \texttt{folderId}
\item
  \texttt{labelFields}
\end{itemize}

\subsection{Attributes}\label{attributes-8}

\noindent\hrulefill

Attribute \textbar{} Data type \textbar{} Explanation \textbar{}
\texttt{layoutId} \textbar{} \texttt{@layout} \textbar{} The ID of the
layout to use to show the View. \textbar{} \texttt{autoLoad} \textbar{}
\texttt{boolean} \textbar{} Whether the list loads automatically when
the Screenlet appears in the app's UI. The default value is
\texttt{true}. \textbar{} \texttt{folderId} \textbar{} \texttt{number}
\textbar{} The ID of the web content folder to retrieve content from.
\textbar{} \texttt{groupId} \textbar{} \texttt{number} \textbar{} The ID
of the site (group) where the asset is stored. If set to \texttt{0}, the
\texttt{groupId} specified in \texttt{LiferayServerContext} is used. The
default value is \texttt{0}. \textbar{} \texttt{cachePolicy} \textbar{}
\texttt{string} \textbar{} The offline mode setting. See the
\href{/docs/7-1/reference/-/knowledge_base/r/web-content-list-screenlet-for-android\#offline}{Offline
section} for details. \textbar{} \texttt{firstPageSize} \textbar{}
\texttt{number} \textbar{} The number of items to retrieve from the
server for display on the first page. The default value is \texttt{50}.
\textbar{} \texttt{pageSize} \textbar{} \texttt{number} \textbar{} The
number of items to retrieve from the server for display on the second
and subsequent pages. The default value is \texttt{25}. \textbar{}
\texttt{labelFields} \textbar{} \texttt{string} \textbar{} The
comma-separated names of the DDM fields to show. Refer to the list's
data definition to find the field names. For more information on this,
see
\href{/docs/7-1/user/-/knowledge_base/u/designing-uniform-content}{the
article on structured web content}. Note that the appearance of data
from a structure's fields depends on the \texttt{layoutId}. \textbar{}
\texttt{obcClassName} \textbar{} \texttt{string} \textbar{} The name of
the \texttt{OrderByComparator} class to use to sort the results. Omit
this property if you don't want to sort the results.
\href{https://github.com/liferay/liferay-portal/tree/master/modules/apps/web-experience/journal/journal-api/src/main/java/com/liferay/journal/util/comparator}{Click
here} to see some comparator classes. Note, however, that not all of
these classes can be used with \texttt{obcClassName}. You can only use
comparator classes that extend
\texttt{OrderByComparator\textless{}JournalArticle\textgreater{}}. You
can also create your own comparator classes that extend
\texttt{OrderByComparator\textless{}JournalArticle\textgreater{}}.
\textbar{}

\noindent\hrulefill

\subsection{Methods}\label{methods-5}

\noindent\hrulefill

Method \textbar{} Return \textbar{} Explanation \textbar{}
\texttt{loadPage(pageNumber)} \textbar{} \texttt{void} \textbar{} Starts
the request to load the specified page of records. The page is shown
when the response is received. \textbar{}

\noindent\hrulefill

\subsection{Listener}\label{listener-8}

Web Content List Screenlet delegates some events to an object or a class
that implements
\href{https://github.com/liferay/liferay-screens/blob/master/android/library/src/main/java/com/liferay/mobile/screens/base/list/BaseListListener.java}{the
\texttt{BaseListListener} interface}. This interface lets you implement
the following methods:

\begin{itemize}
\item
  \texttt{onListPageFailed(int\ startRow,\ Exception\ e)}: Called when
  the server call to retrieve a page of items fails. This method's
  arguments include the \texttt{Exception} generated when the server
  call fails.
\item
  \texttt{onListPageReceived(int\ startRow,\ int\ endRow,\ List\textless{}Record\textgreater{}\ records,\ int\ rowCount)}:
  Called when the server call to retrieve a page of items succeeds. Note
  that this method may be called more than once; once for each page
  received. Because \texttt{startRow} and \texttt{endRow} change for
  each page, a \texttt{startRow} of \texttt{0} corresponds to the first
  item on the first page.
\item
  \texttt{onListItemSelected(Record\ records,\ View\ view)}: Called when
  an item is selected in the list. This method's arguments include the
  selected list item (\texttt{Record}).
\end{itemize}

\section{Image Gallery Screenlet for
Android}\label{image-gallery-screenlet-for-android}

\subsection{Requirements}\label{requirements-9}

\begin{itemize}
\tightlist
\item
  Android SDK 4.1 (API Level 16) or above
\item
  Liferay Portal 6.2 CE/EE, Liferay CE Portal 7.0/7.1, Liferay DXP
\item
  Liferay Screens Compatibility app
  (\href{http://www.liferay.com/marketplace/-/mp/application/54365664}{CE}
  or
  \href{http://www.liferay.com/marketplace/-/mp/application/54369726}{EE/DXP}).
  This app is preinstalled in Liferay CE Portal 7.0/7.1 and Liferay DXP.
\end{itemize}

\subsection{Compatibility}\label{compatibility-9}

\begin{itemize}
\tightlist
\item
  Android SDK 4.1 (API Level 16) or above
\end{itemize}

\subsection{Xamarin Requirements}\label{xamarin-requirements-9}

\begin{itemize}
\tightlist
\item
  Visual Studio 7.2
\item
  Mono .NET framework 5.4.1.6
\end{itemize}

\subsection{Features}\label{features-9}

Image Gallery Screenlet shows a list of images from a Documents and
Media folder in a Liferay instance. You can also use Image Gallery
Screenlet to upload images to and delete images from the same folder.
The Screenlet implements fluent pagination with configurable page size,
and supports i18n in asset values.

\subsection{JSON Services Used}\label{json-services-used-9}

Screenlets in Liferay Screens call JSON web services in the portal. This
Screenlet calls the following services and methods.

\noindent\hrulefill

\begin{longtable}[]{@{}lll@{}}
\toprule\noalign{}
Service & Method & Notes \\
\midrule\noalign{}
\endhead
\bottomrule\noalign{}
\endlastfoot
\texttt{DLAppService} & \texttt{getFileEntries} & Load \\
\texttt{DLAppService} & \texttt{getFileEntriesCount} & \\
\texttt{DLAppService} & \texttt{addFileEntry} & Upload \\
\texttt{DLAppService} & \texttt{deleteFileEntry} & Delete \\
\end{longtable}

\noindent\hrulefill

\subsection{Module}\label{module-9}

\begin{itemize}
\tightlist
\item
  None
\end{itemize}

\subsection{Views}\label{views-9}

The included Views use a standard Android \texttt{RecyclerView} to show
the scrollable list. Other custom Views may use a different component,
such as \texttt{ViewPager} or others, to show the items.

This Screenlet has three different Views:

\begin{enumerate}
\def\labelenumi{\arabic{enumi}.}
\tightlist
\item
  Grid (default)
\item
  Slideshow
\item
  List
\end{enumerate}

\begin{figure}
\centering
\includegraphics{./images/screens-android-imagegallery.png}
\caption{Image Gallery Screenlet using the Grid, Slideshow, and List
Views.}
\end{figure}

\subsection{Offline}\label{offline-9}

This Screenlet supports offline mode so it can function without a
network connection when loading or uploading images (deleting images
while offline is unsupported). For more information on how offline mode
works, see the
\href{/docs/7-1/tutorials/-/knowledge_base/t/architecture-of-offline-mode-in-liferay-screens}{tutorial
on its architecture}. This Screenlet supports the \texttt{REMOTE\_ONLY},
\texttt{CACHE\_ONLY}, \texttt{REMOTE\_FIRST}, and \texttt{CACHE\_FIRST}
offline mode policies.

These policies take the following actions when loading images from a
Liferay instance:

\noindent\hrulefill

Policy \textbar{} What happens \textbar{} When to use \textbar{}
\texttt{REMOTE\_ONLY} \textbar{} The Screenlet loads the list from the
Liferay instance. If a connection issue occurs, the Screenlet uses the
listener to notify the developer about the error. If the Screenlet
successfully loads the list, it stores the data in the local cache for
later use. \textbar{} Use this policy when you always need to show
updated data, and show nothing when there's no connection. \textbar{}
\texttt{CACHE\_ONLY} \textbar{} The Screenlet loads the list from the
local cache. If the list isn't there, the Screenlet uses the listener to
notify the developer about the error. \textbar{} Use this policy when
you always need to show local data, without retrieving remote
information under any circumstance. \textbar{} \texttt{REMOTE\_FIRST}
\textbar{} The Screenlet loads the list from the Liferay instance. If
this succeeds, the Screenlet shows the list to the user and stores it in
the local cache for later use. If a connection issue occurs, the
Screenlet retrieves the list from the local cache. If the list doesn't
exist there, the Screenlet uses the listener to notify the developer
about the error. \textbar{} Use this policy to show the most recent
version of the data when connected, but show an outdated version when
there's no connection. \textbar{} \texttt{CACHE\_FIRST} \textbar{} The
Screenlet loads the list from the local cache. If the list isn't there,
the Screenlet requests it from the Liferay instance and notifies the
developer about any errors that occur (including connectivity errors).
\textbar{} Use this policy to save bandwidth and loading time in case
you have local (but probably outdated) data. \textbar{}

\noindent\hrulefill

These policies take the following actions when uploading an image to a
Liferay instance:

\noindent\hrulefill

Policy \textbar{} What happens \textbar{} When to use \textbar{}
\texttt{REMOTE\_ONLY} \textbar{} The Screenlet sends the image to the
Liferay instance. If a connection issue occurs, the Screenlet uses the
delegate to notify the developer about the error, but it also discards
the image. \textbar{} Use this policy to make sure the Liferay instance
always has the most recent version of the image. \textbar{}
\texttt{CACHE\_ONLY} \textbar{} The Screenlet stores the image in the
local cache. \textbar{} Use this policy when you need to save the image
locally, but don't want to update the image in the Liferay instance
(delete or add image). \textbar{} \texttt{REMOTE\_FIRST} \textbar{} The
Screenlet sends the image to the Liferay instance. If this succeeds, it
also stores the image in the local cache for later use. If a connection
issue occurs, the Screenlet stores the image in the local cache and
sends it to the Liferay instance when the connection is re-established.
\textbar{} Use this policy when you need to make sure the Screenlet
sends the image to the Liferay instance as soon as the connection is
restored. \textbar{} \texttt{CACHE\_FIRST} \textbar{} The Screenlet
stores the image in the local cache and then attempts to send it to the
Liferay instance. If a connection issue occurs, the Screenlet sends the
image to the Liferay instance when the connection is re-established.
\textbar{} Use this policy when you need to make sure the Screenlet
sends the image to the Liferay instance as soon as the connection is
restored. Compared to \texttt{REMOTE\_FIRST}, this policy always stores
the image in the cache. The \texttt{REMOTE\_FIRST} policy only stores
the image in the event of a network error. \textbar{}

\noindent\hrulefill

\subsection{Required Attributes}\label{required-attributes-9}

\begin{itemize}
\tightlist
\item
  \texttt{folderId}
\item
  \texttt{repositoryId}
\end{itemize}

\subsection{Attributes}\label{attributes-9}

\noindent\hrulefill

Attribute \textbar{} Data type \textbar{} Explanation \textbar{}
\texttt{repositoryId} \textbar{} \texttt{number} \textbar{} The ID of
the Liferay instance's Documents and Media repository that contains the
image gallery. If you're using a site's default Documents and Media
repository, then the \texttt{repositoryId} matches the site ID
(\texttt{groupId}). \textbar{} \texttt{folderId} \textbar{}
\texttt{number} \textbar{} The ID of the Documents and Media repository
folder that contains the image gallery. When accessing the folder in
your browser, the \texttt{folderId} is at the end of the URL. \textbar{}
\texttt{cachePolicy} \textbar{} \texttt{string} \textbar{} The offline
mode setting. See the
\href{/docs/7-1/reference/-/knowledge_base/r/image-gallery-screenlet-for-android\#offline}{Offline
section} for details. \textbar{} \texttt{firstPageSize} \textbar{}
\texttt{number} \textbar{} The number of items to display on the first
page. The default value is \texttt{50}. \textbar{} \texttt{pageSize}
\textbar{} \texttt{number} \textbar{} The number of items to display on
second and subsequent pages. The default value is \texttt{25}.
\textbar{} \texttt{mimeTypes} \textbar{} \texttt{string} \textbar{} The
comma-separated list of MIME types for the Screenlet to support.
\textbar{} \texttt{autoLoad} \textbar{} \texttt{boolean} \textbar{}
Whether the list automatically loads when the Screenlet appears in the
app's UI. The default value is \texttt{true}. \textbar{}
\texttt{layoutId} \textbar{} \texttt{@layout} \textbar{} The layout to
use to show the View. \textbar{} \texttt{obcClassName} \textbar{}
\texttt{string} \textbar{} The name of the \texttt{OrderByComparator}
class to use to sort the results. Omit this property if you don't want
to sort the results. Note that you can only use comparator classes that
extend \texttt{OrderByComparator\textless{}DLFileEntry\textgreater{}}.
Liferay contains no such comparator classes. You must therefore create
your own by extending
\texttt{OrderByComparator\textless{}DLFileEntry\textgreater{}}. To see
examples of some comparator classes that extend other Document Library
classes,
\href{https://github.com/liferay/liferay-portal/tree/master/portal-impl/src/com/liferay/portlet/documentlibrary/util/comparator}{click
here}. \textbar{}

\noindent\hrulefill

\subsection{Methods}\label{methods-6}

\noindent\hrulefill

Method \textbar{} Return \textbar{} Explanation \textbar{}
\texttt{loadPage(pageNumber)} \textbar{} \texttt{void} \textbar{} Starts
the request to load the specified page of images. The page is shown when
the response is received. \textbar{}

\noindent\hrulefill

\subsection{Listener}\label{listener-9}

Image Gallery Screenlet delegates some events to an object or class that
implements its
\href{https://github.com/liferay/liferay-screens/blob/master/android/library/src/main/java/com/liferay/mobile/screens/imagegallery/ImageGalleryListener.java}{\texttt{ImageGalleryListener}
interface}. This interface extends
\href{https://github.com/liferay/liferay-screens/blob/master/android/library/src/main/java/com/liferay/mobile/screens/base/list/BaseListListener.java}{the
\texttt{BaseListListener} interface}. Therefore, Image Gallery
Screenlet's listener methods are as follows:

\begin{itemize}
\item
  \texttt{onListPageFailed(int\ startRow,\ Exception\ e)}: Called when
  the server call to retrieve a page of items fails. This method's
  arguments include the \texttt{Exception} generated when the server
  call fails.
\item
  \texttt{onListPageReceived(int\ startRow,\ int\ endRow,\ List\textless{}Record\textgreater{}\ records,\ int\ rowCount)}:
  Called when the server call to retrieve a page of items succeeds. Note
  that this method may be called more than once; once for each page
  received. Because \texttt{startRow} and \texttt{endRow} change for
  each page, a \texttt{startRow} of \texttt{0} corresponds to the first
  item on the first page.
\item
  \texttt{onListItemSelected(Record\ records,\ View\ view)}: Called when
  an item is selected in the list. This method's arguments include the
  selected list item (\texttt{Record}).
\item
  \texttt{onImageEntryDeleted(long\ imageEntryId)}: Called when an item
  in the list is deleted.
\item
  \texttt{onImageUploadStarted(String\ picturePath,\ String\ title,\ String\ description,\ String\ changelog)}:
  Called when an item is prepared for upload.
\item
  \texttt{onImageUploadProgress(int\ totalBytes,\ int\ totalBytesSent)}:
  Called when an item is uploading.
\item
  \texttt{onImageUploadEnd(ImageEntry\ entry)}: Called when an item
  finishes uploading.
\item
  \texttt{showUploadImageView(String\ actionName,\ String\ picturePath,\ int\ screenletId)}:
  Called when the View for uploading an image is instantiated. The
  default behavior is to show the default View in a dialog. To retain
  this behavior, all this method needs to do is return \texttt{false}.
  To change the default behavior, use the \texttt{initializeUploadView}
  method to initialize a custom View that extends
  \texttt{BaseDetailUploadView}. Then return \texttt{true} to prevent
  the Screenlet from executing the default behavior. For example, the
  following sample implementation uses \texttt{initializeUploadView} to
  initialize the custom View instance \texttt{uploadDetailView}. It then
  performs a custom UI action (\texttt{uploadImageCard.goRight()}) and
  returns \texttt{true}:

\begin{verbatim}
  @Override
  public boolean showUploadImageView(String actionName, String picturePath, int screenletId) {
      uploadDetailView.initializeUploadView(actionName, picturePath, screenletId);
      uploadImageCard.goRight();

      return true;
  }
\end{verbatim}
\item
  \texttt{provideImageUploadDetailView()}: Called when the Screenlet
  provides the image upload View. To inflate the default View, return
  \texttt{0} in this method. Alternatively, display this View with a
  custom layout by returning its layout ID. Such a layout must have
  \texttt{DefaultUploadDetailView} as its root class.
\item
  \texttt{error(Exception\ e,\ String\ userAction)}: Called when an
  error occurs in the process. The \texttt{userAction} argument
  distinguishes the specific action in which the error occurred.
\end{itemize}

\section{Rating Screenlet for
Android}\label{rating-screenlet-for-android}

\subsection{Requirements}\label{requirements-10}

\begin{itemize}
\tightlist
\item
  Android SDK 4.1 (API Level 16) or above
\item
  Liferay Portal 6.2 CE/EE, Liferay CE Portal 7.0/7.1, Liferay DXP
\item
  Liferay Screens Compatibility app
  (\href{http://www.liferay.com/marketplace/-/mp/application/54365664}{CE}
  or
  \href{http://www.liferay.com/marketplace/-/mp/application/54369726}{EE/DXP}).
  This app is preinstalled in Liferay CE Portal 7.0/7.1 and Liferay DXP.
\end{itemize}

\subsection{Compatibility}\label{compatibility-10}

\begin{itemize}
\tightlist
\item
  Android SDK 4.1 (API Level 16) or above
\end{itemize}

\subsection{Xamarin Requirements}\label{xamarin-requirements-10}

\begin{itemize}
\tightlist
\item
  Visual Studio 7.2
\item
  Mono .NET framework 5.4.1.6
\end{itemize}

\subsection{Features}\label{features-10}

Rating Screenlet shows an asset's rating. It also lets users update or
delete the rating. This Screenlet comes with different Views that
display ratings as thumbs, stars, and emojis.

\subsection{JSON Services Used}\label{json-services-used-10}

Screenlets in Liferay Screens call JSON web services in the portal. This
Screenlet calls the following services and methods.

\noindent\hrulefill

\begin{longtable}[]{@{}
  >{\raggedright\arraybackslash}p{(\columnwidth - 4\tabcolsep) * \real{0.3889}}
  >{\raggedright\arraybackslash}p{(\columnwidth - 4\tabcolsep) * \real{0.3333}}
  >{\raggedright\arraybackslash}p{(\columnwidth - 4\tabcolsep) * \real{0.2778}}@{}}
\toprule\noalign{}
\begin{minipage}[b]{\linewidth}\raggedright
Service
\end{minipage} & \begin{minipage}[b]{\linewidth}\raggedright
Method
\end{minipage} & \begin{minipage}[b]{\linewidth}\raggedright
Notes
\end{minipage} \\
\midrule\noalign{}
\endhead
\bottomrule\noalign{}
\endlastfoot
\texttt{ScreensratingsentryService} (Screens compatibility plugin) &
\texttt{getRatingsEntries} & With \texttt{entryId} \\
\texttt{ScreensratingsentryService} (Screens compatibility plugin) &
\texttt{getRatingsEntries} & With \texttt{classPK} and
\texttt{className} \\
\texttt{ScreensratingsentryService} (Screens compatibility plugin) &
\texttt{updateRatingsEntry} & \\
\texttt{ScreensratingsentryService} (Screens compatibility plugin) &
\texttt{deleteRatingsEntry} & \\
\end{longtable}

\noindent\hrulefill

\subsection{Module}\label{module-10}

\begin{itemize}
\tightlist
\item
  None
\end{itemize}

\subsection{Views}\label{views-10}

The default View uses an
\href{https://developer.android.com/reference/android/widget/RatingBar.html}{Android
\texttt{RatingBar}} to show an asset's rating. Other custom Views may
show the rating with a different Android component such as
\texttt{Button}, \texttt{ImageButton}, or others.

This Screenlet has five different Views:

\begin{enumerate}
\def\labelenumi{\arabic{enumi}.}
\tightlist
\item
  Like
\item
  Thumbs (default)
\item
  Stars
\item
  Reactions
\item
  Emojis
\end{enumerate}

\begin{figure}
\centering
\includegraphics{./images/screens-android-ratings.png}
\caption{Rating Screenlet's different Views.}
\end{figure}

\subsection{Offline}\label{offline-10}

This Screenlet supports offline mode so it can function without a
network connection. For more information on how offline mode works, see
the
\href{/docs/7-1/tutorials/-/knowledge_base/t/architecture-of-offline-mode-in-liferay-screens}{tutorial
on its architecture}. Here are the offline mode policies that you can
use with this Screenlet:

\noindent\hrulefill

Policy \textbar{} What happens \textbar{} When to use \textbar{}
\texttt{REMOTE\_ONLY} \textbar{} The Screenlet loads the data from the
Liferay instance. If a connection issue occurs, the Screenlet uses the
listener to notify the developer about the error. If the Screenlet
successfully loads the data, it stores it in the local cache for later
use. \textbar{} Use this policy when you always need to show updated
data, and show nothing when there's no connection. \textbar{}
\texttt{CACHE\_ONLY} \textbar{} The Screenlet loads the data from the
local cache. If the data isn't there, the Screenlet uses the listener to
notify the developer about the error. \textbar{} Use this policy when
you always need to show local data, without retrieving remote
information under any circumstance. \textbar{} \texttt{REMOTE\_FIRST}
\textbar{} The Screenlet loads the data from the Liferay instance. If
this succeeds, the Screenlet shows the data to the user and stores it in
the local cache for later use. If a connection issue occurs, the
Screenlet retrieves the data from the local cache. If the data doesn't
exist there, the Screenlet uses the listener to notify the developer
about the error. \textbar{} Use this policy to show the most recent
version of the data when connected, but show an outdated version when
there's no connection. \textbar{} \texttt{CACHE\_FIRST} \textbar{} The
Screenlet loads the data from the local cache. If the data isn't there,
the Screenlet requests it from the Liferay instance and notifies the
developer about any errors that occur (including connectivity errors).
\textbar{} Use this policy to save bandwidth and loading time in case
you have local (but probably outdated) data. \textbar{}

\noindent\hrulefill

\subsection{Required Attributes}\label{required-attributes-10}

\begin{itemize}
\tightlist
\item
  \texttt{entryId}
\end{itemize}

If you don't use \texttt{entryId}, you must use both of the following
attributes:

\begin{itemize}
\tightlist
\item
  \texttt{className}
\item
  \texttt{classPK}
\end{itemize}

\subsection{Attributes}\label{attributes-10}

\noindent\hrulefill

Attribute \textbar{} Data type \textbar{} Explanation \textbar{}
\texttt{layoutId} \textbar{} \texttt{@layout} \textbar{} The ID of the
layout to use to show the View. \textbar{} \texttt{autoLoad} \textbar{}
\texttt{boolean} \textbar{} Whether the rating loads automatically when
the Screenlet appears in the app's UI. The default value is
\texttt{true}. \textbar{} \texttt{editable} \textbar{} \texttt{boolean}
\textbar{} Whether the user can change the rating. \textbar{}
\texttt{entryId} \textbar{} \texttt{number} \textbar{} The primary key
of the asset with the rating to display. \textbar{} \texttt{className}
\textbar{} \texttt{string} \textbar{} The asset's fully qualified class
name. For example, a blog entry's \texttt{className} is
\texttt{com.liferay.blogs.model.BlogsEntry}. The \texttt{className}
attribute is required when using it with \texttt{classPK} to instantiate
the Screenlet. \textbar{} \texttt{classPK} \textbar{} \texttt{number}
\textbar{} The asset's unique identifier. Only use this attribute when
also using \texttt{className} to instantiate the Screenlet. \textbar{}
\texttt{groupId} \textbar{} \texttt{number} \textbar{} The ID of the
site (group) containing the asset. \textbar{} \texttt{cachePolicy}
\textbar{} \texttt{string} \textbar{} The offline mode setting. See the
\href{/docs/7-1/reference/-/knowledge_base/r/rating-screenlet-for-android\#offline}{Offline
section} for details. \textbar{}

\noindent\hrulefill

\subsection{Methods}\label{methods-7}

\noindent\hrulefill

Method \textbar{} Return \textbar{} Explanation \textbar{}
\texttt{load()} \textbar{} \texttt{void} \textbar{} Starts the request
to load the asset's ratings. \textbar{}

\noindent\hrulefill

\subsection{Listener}\label{listener-10}

Rating Screenlet delegates some events to an object or class that
implements
\href{https://github.com/liferay/liferay-screens/blob/master/android/library/src/main/java/com/liferay/mobile/screens/rating/RatingListener.java}{its
\texttt{RatingListener} interface}. Therefore, Rating Screenlet's
listener methods are as follows:

\begin{itemize}
\tightlist
\item
  \texttt{onRatingOperationSuccess(AssetRating\ assetRating)}: Called
  when the operation finishes successfully and the rating is loaded.
\end{itemize}

\section{Comment List Screenlet for
Android}\label{comment-list-screenlet-for-android}

\subsection{Requirements}\label{requirements-11}

\begin{itemize}
\tightlist
\item
  Android SDK 4.1 (API Level 16) or above
\item
  Liferay Portal 6.2 CE/EE, Liferay CE Portal 7.0/7.1, Liferay DXP
\item
  Liferay Screens Compatibility app
  (\href{http://www.liferay.com/marketplace/-/mp/application/54365664}{CE}
  or
  \href{http://www.liferay.com/marketplace/-/mp/application/54369726}{EE/DXP}).
  This app is preinstalled in Liferay CE Portal 7.0/7.1 and Liferay DXP.
\end{itemize}

\subsection{Compatibility}\label{compatibility-11}

\begin{itemize}
\tightlist
\item
  Android SDK 4.1 (API Level 16) or above
\end{itemize}

\subsection{Xamarin Requirements}\label{xamarin-requirements-11}

\begin{itemize}
\tightlist
\item
  Visual Studio 7.2
\item
  Mono .NET framework 5.4.1.6
\end{itemize}

\subsection{Features}\label{features-11}

Comment List Screenlet can list all the comments of an asset in a
Liferay instance. It also lets the user update or delete comments.

\subsection{JSON Services Used}\label{json-services-used-11}

Screenlets in Liferay Screens call JSON web services in the portal. This
Screenlet calls the following services and methods.

\noindent\hrulefill

\begin{longtable}[]{@{}
  >{\raggedright\arraybackslash}p{(\columnwidth - 4\tabcolsep) * \real{0.3889}}
  >{\raggedright\arraybackslash}p{(\columnwidth - 4\tabcolsep) * \real{0.3333}}
  >{\raggedright\arraybackslash}p{(\columnwidth - 4\tabcolsep) * \real{0.2778}}@{}}
\toprule\noalign{}
\begin{minipage}[b]{\linewidth}\raggedright
Service
\end{minipage} & \begin{minipage}[b]{\linewidth}\raggedright
Method
\end{minipage} & \begin{minipage}[b]{\linewidth}\raggedright
Notes
\end{minipage} \\
\midrule\noalign{}
\endhead
\bottomrule\noalign{}
\endlastfoot
\texttt{ScreenscommentService} (Screens compatibility plugin) &
\texttt{getComments} & \\
\texttt{ScreenscommentService} (Screens compatibility plugin) &
\texttt{getCommentsCount} & \\
\end{longtable}

\noindent\hrulefill

\subsection{Module}\label{module-11}

\begin{itemize}
\tightlist
\item
  None
\end{itemize}

\subsection{Views}\label{views-11}

\begin{itemize}
\tightlist
\item
  Default
\end{itemize}

The Default View uses an
\href{https://developer.android.com/training/material/lists-cards.html}{Android
\texttt{RecyclerView}} to show an asset's comments. Other Views may use
a different component, such as \texttt{TableView} or others, to show the
items.

\begin{figure}
\centering
\includegraphics{./images/screens-android-commentlist.png}
\caption{Comment List Screenlet using the Default View.}
\end{figure}

\subsection{Offline}\label{offline-11}

This Screenlet supports offline mode so it can function without a
network connection. For more information on how offline mode works, see
the
\href{/docs/7-1/tutorials/-/knowledge_base/t/architecture-of-offline-mode-in-liferay-screens}{tutorial
on its architecture}. Here are the offline mode policies that you can
use with this Screenlet:

\noindent\hrulefill

Policy \textbar{} What happens \textbar{} When to use \textbar{}
\texttt{REMOTE\_ONLY} \textbar{} The Screenlet loads the comments from
the Liferay instance. If a connection issue occurs, the Screenlet uses
the listener to notify the developer about the error. If the Screenlet
successfully loads the comments, it stores the data in the local cache
for later use. \textbar{} Use this policy when you always need to show
updated data, and show nothing when there's no connection. \textbar{}
\texttt{CACHE\_ONLY} \textbar{} The Screenlet loads the comments from
the local cache. If the data isn't there, the Screenlet uses the
listener to notify the developer about the error. \textbar{} Use this
policy when you always need to show local data, without retrieving
remote information under any circumstance. \textbar{}
\texttt{REMOTE\_FIRST} \textbar{} The Screenlet loads the comments from
the Liferay instance. If this succeeds, the Screenlet shows the data to
the user and stores it in the local cache for later use. If a connection
issue occurs, the Screenlet retrieves the data from the local cache. If
the data doesn't exist there, the Screenlet uses the listener to notify
the developer about the error. \textbar{} Use this policy to show the
most recent version of the data when connected, but show an outdated
version when there's no connection. \textbar{} \texttt{CACHE\_FIRST}
\textbar{} The Screenlet loads the comments from the local cache. If the
data isn't there, the Screenlet requests it from the Liferay instance
and notifies the developer about any errors that occur (including
connectivity errors). \textbar{} Use this policy to save bandwidth and
loading time in case you have local (but probably outdated) data.
\textbar{}

\noindent\hrulefill

\subsection{Required Attributes}\label{required-attributes-11}

\begin{itemize}
\tightlist
\item
  \texttt{className}
\item
  \texttt{classPK}
\end{itemize}

\subsection{Attributes}\label{attributes-11}

\noindent\hrulefill

Attribute \textbar{} Data type \textbar{} Explanation \textbar{}
\texttt{layoutId} \textbar{} \texttt{@layout} \textbar{} The layout to
use to show the View. \textbar{} \texttt{autoLoad} \textbar{}
\texttt{boolean} \textbar{} Whether the list should automatically load
when the Screenlet appears in the app's UI. The default value is
\texttt{true}. \textbar{} \texttt{cachePolicy} \textbar{}
\texttt{string} \textbar{} The offline mode setting. See
\href{/docs/7-1/reference/-/knowledge_base/r/comment-list-screenlet-for-android\#offline}{the
Offline section} for details. \textbar{} \texttt{className} \textbar{}
\texttt{string} \textbar{} The asset's fully qualified class name. For
example, a blog entry's \texttt{className} is
\texttt{com.liferay.blogs.model.BlogsEntry}. The \texttt{className} and
\texttt{classPK} attributes are required to instantiate the Screenlet.
\textbar{} \texttt{classPK} \textbar{} \texttt{number} \textbar{} The
asset's unique identifier. The \texttt{className} and \texttt{classPK}
attributes are required to instantiate the Screenlet. \textbar{}
\texttt{firstPageSize} \textbar{} \texttt{number} \textbar{} The number
of items to retrieve from the server for display on the first page. The
default value is \texttt{50}. \textbar{} \texttt{pageSize} \textbar{}
\texttt{number} \textbar{} The number of items to retrieve from the
server for display on the second and subsequent pages. The default value
is \texttt{25}. \textbar{} \texttt{labelFields} \textbar{}
\texttt{string} \textbar{} The comma-separated names of the DDL fields
to show. Refer to the list's data definition to find the field names.
For more information on this, see
\href{/docs/7-1/user/-/knowledge_base/u/designing-uniform-content}{the
article on structured web content}. Note that the appearance of data
from a structure's fields depends on the \texttt{layoutId}. \textbar{}
\texttt{editable} \textbar{} \texttt{boolean} \textbar{} Whether the
user can edit the comment. \textbar{}

\noindent\hrulefill

\subsection{Methods}\label{methods-8}

\noindent\hrulefill

Method \textbar{} Return \textbar{} Explanation \textbar{}
\texttt{loadPage(pageNumber)} \textbar{} \texttt{void} \textbar{} Starts
the request to load the specified page of records. The page is shown
when the response is received. \textbar{}

\noindent\hrulefill

\subsection{Listener}\label{listener-11}

Comment List Screenlet delegates some events to a class that implements
\texttt{CommentListListener}. This interface lets you implement the
following methods:

\begin{itemize}
\item
  \texttt{onDeleteCommentSuccess(CommentEntry\ commentEntry)}: Called
  when the Screenlet successfully deletes the comment.
\item
  \texttt{onUpdateCommentSuccess(CommentEntry\ commentEntry)}: Called
  when the Screenlet successfully updates the comment.
\item
  \texttt{onListPageFailed(int\ startRow,\ Exception\ e)}: Called when
  the server call to retrieve a page of items fails. This method's
  arguments include the \texttt{Exception} generated when the server
  call fails.
\item
  \texttt{onListPageReceived(int\ startRow,\ int\ endRow,\ List\textless{}CommentEntry\textgreater{}\ entries,\ int\ rowCount)}:
  Called when the server call to retrieve a page of items succeeds. Note
  that this method may be called more than once; once for each page
  received. Because \texttt{startRow} and \texttt{endRow} change for
  each page, a \texttt{startRow} of \texttt{0} corresponds to the first
  item on the first page.
\item
  \texttt{onListItemSelected(CommentEntry\ element,\ View\ view)}:
  Called when an item is selected in the list. This method's arguments
  include the selected list item (\texttt{CommentEntry}).
\item
  \texttt{error(Exception\ e,\ String\ userAction)}: Called when an
  error occurs in the process. The \texttt{userAction} argument
  distinguishes the specific action in which the error occurred.
\end{itemize}

\section{Comment Display Screenlet for
Android}\label{comment-display-screenlet-for-android}

\subsection{Requirements}\label{requirements-12}

\begin{itemize}
\tightlist
\item
  Android SDK 4.1 (API Level 16) or above
\item
  Liferay Portal 6.2 CE/EE, Liferay CE Portal 7.0/7.1, Liferay DXP
\item
  Liferay Screens Compatibility app
  (\href{http://www.liferay.com/marketplace/-/mp/application/54365664}{CE}
  or
  \href{http://www.liferay.com/marketplace/-/mp/application/54369726}{EE/DXP}).
  This app is preinstalled in Liferay CE Portal 7.0/7.1 and Liferay DXP.
\end{itemize}

\subsection{Compatibility}\label{compatibility-12}

\begin{itemize}
\tightlist
\item
  Android SDK 4.1 (API Level 16) or above
\end{itemize}

\subsection{Xamarin Requirements}\label{xamarin-requirements-12}

\begin{itemize}
\tightlist
\item
  Visual Studio 7.2
\item
  Mono .NET framework 5.4.1.6
\end{itemize}

\subsection{Features}\label{features-12}

Comment Display Screenlet can show one comment of an asset in a Liferay
instance. It also lets the user update or delete the comment.

\subsection{JSON Services Used}\label{json-services-used-12}

Screenlets in Liferay Screens call JSON web services in the portal. This
Screenlet calls the following services and methods.

\noindent\hrulefill

\begin{longtable}[]{@{}
  >{\raggedright\arraybackslash}p{(\columnwidth - 4\tabcolsep) * \real{0.3889}}
  >{\raggedright\arraybackslash}p{(\columnwidth - 4\tabcolsep) * \real{0.3333}}
  >{\raggedright\arraybackslash}p{(\columnwidth - 4\tabcolsep) * \real{0.2778}}@{}}
\toprule\noalign{}
\begin{minipage}[b]{\linewidth}\raggedright
Service
\end{minipage} & \begin{minipage}[b]{\linewidth}\raggedright
Method
\end{minipage} & \begin{minipage}[b]{\linewidth}\raggedright
Notes
\end{minipage} \\
\midrule\noalign{}
\endhead
\bottomrule\noalign{}
\endlastfoot
\texttt{ScreenscommentService} (Screens compatibility plugin) &
\texttt{getComment} & \\
\texttt{ScreenscommentService} (Screens compatibility plugin) &
\texttt{updateComment} & \\
\texttt{CommentmanagerjsonwsService} & \texttt{deleteComment} & \\
\end{longtable}

\noindent\hrulefill

\subsection{Module}\label{module-12}

\begin{itemize}
\tightlist
\item
  None
\end{itemize}

\subsection{Views}\label{views-12}

\begin{itemize}
\tightlist
\item
  Default
\end{itemize}

The Default View uses
\href{/docs/7-1/reference/-/knowledge_base/r/userportraitscreenlet-for-android}{User
Portrait Screenlet}, and \texttt{TextView} and \texttt{ImageButton}
elements to show an asset's comment. Other Views may different
components to show the comment.

\begin{figure}
\centering
\includegraphics{./images/screens-android-commentdisplay.png}
\caption{Comment Display Screenlet using the Default View.}
\end{figure}

\subsection{Offline}\label{offline-12}

This Screenlet supports offline mode so it can function without a
network connection. For more information on how offline mode works, see
the
\href{/docs/7-1/tutorials/-/knowledge_base/t/architecture-of-offline-mode-in-liferay-screens}{tutorial
on its architecture}. Here are the offline mode policies that you can
use with this Screenlet:

\noindent\hrulefill

Policy \textbar{} What happens \textbar{} When to use \textbar{}
\texttt{REMOTE\_ONLY} \textbar{} The Screenlet loads the data from the
Liferay instance. If a connection issue occurs, the Screenlet uses the
listener to notify the developer about the error. If the Screenlet
successfully loads the data, it stores it in the local cache for later
use. \textbar{} Use this policy when you always need to show updated
data, and show nothing when there's no connection. \textbar{}
\texttt{CACHE\_ONLY} \textbar{} The Screenlet loads the data from the
local cache. If the data isn't there, the Screenlet uses the listener to
notify the developer about the error. \textbar{} Use this policy when
you always need to show local data, without retrieving remote
information under any circumstance. \textbar{} \texttt{REMOTE\_FIRST}
\textbar{} The Screenlet loads the data from the Liferay instance. If
this succeeds, the Screenlet shows the data to the user and stores it in
the local cache for later use. If a connection issue occurs, the
Screenlet retrieves the data from the local cache. If the data doesn't
exist there, the Screenlet uses the listener to notify the developer
about the error. \textbar{} Use this policy to show the most recent
version of the data when connected, but show an outdated version when
there's no connection. \textbar{} \texttt{CACHE\_FIRST} \textbar{} The
Screenlet loads the data from the local cache. If the data isn't there,
the Screenlet requests it from the Liferay instance and notifies the
developer about any errors that occur (including connectivity errors).
\textbar{} Use this policy to save bandwidth and loading time in case
you have local (but probably outdated) data. \textbar{}

\noindent\hrulefill

\subsection{Required Attributes}\label{required-attributes-12}

\begin{itemize}
\tightlist
\item
  \texttt{commentId}
\end{itemize}

\subsection{Attributes}\label{attributes-12}

\noindent\hrulefill

Attribute \textbar{} Data type \textbar{} Explanation \textbar{}
\texttt{layoutId} \textbar{} \texttt{@layout} \textbar{} The layout to
use to show the View.\textbar{} \texttt{autoLoad} \textbar{}
\texttt{boolean} \textbar{} Whether the list should automatically load
when the Screenlet appears in the app's UI. The default value is
\texttt{true}. \textbar{} \texttt{cachePolicy} \textbar{}
\texttt{string} \textbar{} The offline mode setting. See
\href{/docs/7-1/reference/-/knowledge_base/r/comment-display-screenlet-for-android\#offline}{the
Offline section} for details. \textbar{} \texttt{commentId} \textbar{}
\texttt{number} \textbar{} The primary key of the comment to display.
\textbar{} \texttt{editable} \textbar{} \texttt{boolean} \textbar{}
Whether the user can edit the comment. \textbar{}

\noindent\hrulefill

\subsection{Methods}\label{methods-9}

\noindent\hrulefill

Method \textbar{} Return \textbar{} Explanation \textbar{}
\texttt{load()} \textbar{} \texttt{void} \textbar{} Starts the request
to load the comment. \textbar{}

\noindent\hrulefill

\subsection{Listener}\label{listener-12}

Comment Display Screenlet delegates some events to a class that
implements \texttt{CommentDisplayListener}. This interface lets you
implement the following methods:

\begin{itemize}
\item
  \texttt{onLoadCommentSuccess(CommentEntry\ commentEntry)}: Called when
  the Screenlet successfully loads the comment.
\item
  \texttt{onDeleteCommentSuccess(CommentEntry\ commentEntry)}: Called
  when the Screenlet successfully deletes the comment.
\item
  \texttt{onUpdateCommentSuccess(CommentEntry\ commentEntry)}: Called
  when the Screenlet successfully updates the comment.
\item
  \texttt{error(Exception\ e,\ String\ userAction)}: Called when an
  error occurs in the process. The \texttt{userAction} argument
  distinguishes the specific action in which the error occurred.
\end{itemize}

\section{Comment Add Screenlet for
Android}\label{comment-add-screenlet-for-android}

\subsection{Requirements}\label{requirements-13}

\begin{itemize}
\tightlist
\item
  Android SDK 4.1 (API Level 16) or above
\item
  Liferay Portal 6.2 CE/EE, Liferay CE Portal 7.0/7.1, Liferay DXP
\item
  Liferay Screens Compatibility app
  (\href{http://www.liferay.com/marketplace/-/mp/application/54365664}{CE}
  or
  \href{http://www.liferay.com/marketplace/-/mp/application/54369726}{EE/DXP}).
  This app is preinstalled in Liferay CE Portal 7.0/7.1 and Liferay DXP.
\end{itemize}

\subsection{Compatibility}\label{compatibility-13}

\begin{itemize}
\tightlist
\item
  Android SDK 4.1 (API Level 16) or above
\end{itemize}

\subsection{Xamarin Requirements}\label{xamarin-requirements-13}

\begin{itemize}
\tightlist
\item
  Visual Studio 7.2
\item
  Mono .NET framework 5.4.1.6
\end{itemize}

\subsection{Features}\label{features-13}

Comment Add Screenlet can add a comment to an asset in a Liferay
instance.

\subsection{JSON Services Used}\label{json-services-used-13}

Screenlets in Liferay Screens call JSON web services in the portal. This
Screenlet calls the following services and methods.

\noindent\hrulefill

\begin{longtable}[]{@{}
  >{\raggedright\arraybackslash}p{(\columnwidth - 4\tabcolsep) * \real{0.3889}}
  >{\raggedright\arraybackslash}p{(\columnwidth - 4\tabcolsep) * \real{0.3333}}
  >{\raggedright\arraybackslash}p{(\columnwidth - 4\tabcolsep) * \real{0.2778}}@{}}
\toprule\noalign{}
\begin{minipage}[b]{\linewidth}\raggedright
Service
\end{minipage} & \begin{minipage}[b]{\linewidth}\raggedright
Method
\end{minipage} & \begin{minipage}[b]{\linewidth}\raggedright
Notes
\end{minipage} \\
\midrule\noalign{}
\endhead
\bottomrule\noalign{}
\endlastfoot
\texttt{ScreenscommentService} (Screens compatibility plugin) &
\texttt{addComment} & \\
\end{longtable}

\noindent\hrulefill

\subsection{Module}\label{module-13}

\begin{itemize}
\tightlist
\item
  None
\end{itemize}

\subsection{Views}\label{views-13}

\begin{itemize}
\tightlist
\item
  Default
\end{itemize}

The Default View uses Android's \texttt{EditText} and \texttt{Button}
elements to show an add comment dialog. Other Views may use different
components to show this dialog.

\begin{figure}
\centering
\includegraphics{./images/screens-android-commentadd.png}
\caption{Comment Add Screenlet using the Default View.}
\end{figure}

\subsection{Offline}\label{offline-13}

This Screenlet supports offline mode so it can function without a
network connection. For more information on how offline mode works, see
the
\href{/docs/7-1/tutorials/-/knowledge_base/t/architecture-of-offline-mode-in-liferay-screens}{tutorial
on its architecture}. Here are the offline mode policies that you can
use with this Screenlet:

\noindent\hrulefill

Policy \textbar{} What happens \textbar{} When to use \textbar{}
\texttt{REMOTE\_ONLY} \textbar{} The Screenlet sends the data to the
Liferay instance. If a connection issue occurs, the Screenlet uses the
listener to notify the developer about the error. If the Screenlet
successfully sends the data, it also stores it in the local cache.
\textbar{} Use this policy when you always need to send updated data,
and send nothing when there's no connection. \textbar{}
\texttt{CACHE\_ONLY} \textbar{} The Screenlet sends the data to the
local cache. If an error occurs, the Screenlet uses the listener to
notify the developer. \textbar{} Use this policy when you always need to
store local data without sending remote information under any
circumstance. \textbar{} \texttt{REMOTE\_FIRST} \textbar{} The Screenlet
sends the data to the Liferay instance. If this succeeds, the Screenlet
also stores the data in the local cache. If a connection issue occurs,
the Screenlet stores the data to the local cache and sends it to the
Liferay instance when the connection is restored. If an error occurs,
the Screenlet uses the listener to notify the developer. \textbar{} Use
this policy to send the most recent version of the data when connected,
and store the data for later synchronization when there's no connection.
\textbar{} \texttt{CACHE\_FIRST} \textbar{} The Screenlet sends the data
to the local cache, then sends it to the Liferay instance. If sending
the data to the Liferay instance fails, the Screenlet still stores the
data locally and then notifies the developer about any errors that occur
(including connectivity errors). \textbar{} Use this policy to save
bandwidth and store local (but possibly outdated) data. \textbar{}

\noindent\hrulefill

\subsection{Required Attributes}\label{required-attributes-13}

\begin{itemize}
\tightlist
\item
  \texttt{className}
\item
  \texttt{classPK}
\end{itemize}

\subsection{Attributes}\label{attributes-13}

\noindent\hrulefill

Attribute \textbar{} Data type \textbar{} Explanation \textbar{}
\texttt{layoutId} \textbar{} \texttt{@layout} \textbar{} The layout to
use to show the View.\textbar{} \texttt{className} \textbar{}
\texttt{string} \textbar{} The asset's fully qualified class name. For
example, a blog entry's \texttt{className} is
\texttt{com.liferay.blogs.model.BlogsEntry}. The \texttt{className} and
\texttt{classPK} attributes are required to instantiate the Screenlet.
\textbar{} \texttt{classPK} \textbar{} \texttt{number} \textbar{} The
asset's unique identifier. The \texttt{className} and \texttt{classPK}
attributes are required to instantiate the Screenlet. \textbar{}
\texttt{cachePolicy} \textbar{} \texttt{string} \textbar{} The offline
mode setting. See
\href{/docs/7-1/reference/-/knowledge_base/r/comment-add-screenlet-for-android\#offline}{the
Offline section} for details. \textbar{}

\noindent\hrulefill

\subsection{Listener}\label{listener-13}

Comment Add Screenlet delegates some events to a class that implements
\texttt{CommentAddListener}. This interface lets you implement the
following methods:

\begin{itemize}
\item
  \texttt{onAddCommentSuccess(CommentEntry\ commentEntry)}: Called when
  the Screenlet successfully adds a comment to the asset.
\item
  \texttt{error(Exception\ e,\ String\ userAction)}: Called when an
  error occurs in the process. The \texttt{userAction} argument
  distinguishes the specific action in which the error occurred.
\end{itemize}

\section{Asset Display Screenlet for
Android}\label{asset-display-screenlet-for-android}

\subsection{Requirements}\label{requirements-14}

\begin{itemize}
\tightlist
\item
  Android SDK 4.1 (API Level 16) or above
\item
  Liferay Portal 6.2 CE/EE, Liferay CE Portal 7.0/7.1, Liferay DXP
\item
  Liferay Screens Compatibility app
  (\href{http://www.liferay.com/marketplace/-/mp/application/54365664}{CE}
  or
  \href{http://www.liferay.com/marketplace/-/mp/application/54369726}{EE/DXP}).
  This app is preinstalled in Liferay CE Portal 7.0/7.1 and Liferay DXP.
\end{itemize}

\subsection{Compatibility}\label{compatibility-14}

\begin{itemize}
\tightlist
\item
  Android SDK 4.1 (API Level 16) or above
\end{itemize}

\subsection{Xamarin Requirements}\label{xamarin-requirements-14}

\begin{itemize}
\tightlist
\item
  Visual Studio 7.2
\item
  Mono .NET framework 5.4.1.6
\end{itemize}

\subsection{Features}\label{features-14}

Asset Display Screenlet can display an asset from a Liferay instance.
The Screenlet can currently display Documents and Media files
(\texttt{DLFileEntry} images, videos, audio files, and PDFs), blogs
entries (\texttt{BlogsEntry}) and web content articles
(\texttt{WebContent}).

Asset Display Screenlet can also display your custom asset types. See
\href{/docs/7-1/reference/-/knowledge_base/r/asset-display-screenlet-for-android\#listener}{the
Listener section of this document} for details.

\subsection{JSON Services Used}\label{json-services-used-14}

Screenlets in Liferay Screens call JSON web services in the portal. This
Screenlet calls the following services and methods.

\noindent\hrulefill

\begin{longtable}[]{@{}
  >{\raggedright\arraybackslash}p{(\columnwidth - 4\tabcolsep) * \real{0.3889}}
  >{\raggedright\arraybackslash}p{(\columnwidth - 4\tabcolsep) * \real{0.3333}}
  >{\raggedright\arraybackslash}p{(\columnwidth - 4\tabcolsep) * \real{0.2778}}@{}}
\toprule\noalign{}
\begin{minipage}[b]{\linewidth}\raggedright
Service
\end{minipage} & \begin{minipage}[b]{\linewidth}\raggedright
Method
\end{minipage} & \begin{minipage}[b]{\linewidth}\raggedright
Notes
\end{minipage} \\
\midrule\noalign{}
\endhead
\bottomrule\noalign{}
\endlastfoot
\texttt{ScreensassetentryService} (Screens compatibility plugin) &
\texttt{getAssetEntry} & With \texttt{entryId} \\
\texttt{ScreensassetentryService} (Screens compatibility plugin) &
\texttt{getAssetEntry} & With \texttt{classPK} and \texttt{className} \\
\texttt{ScreensassetentryService} (Screens compatibility plugin) &
\texttt{getAssetEntries} & With \texttt{entryQuery} \\
\texttt{ScreensassetentryService} (Screens compatibility plugin) &
\texttt{getAssetEntries} & With \texttt{companyId}, \texttt{groupId},
and \texttt{portletItemName} \\
\end{longtable}

\noindent\hrulefill

\subsection{Module}\label{module-14}

\begin{itemize}
\tightlist
\item
  None
\end{itemize}

\subsection{Views}\label{views-14}

\begin{itemize}
\tightlist
\item
  Default
\end{itemize}

\begin{figure}
\centering
\includegraphics{./images/screens-android-assetdisplay.png}
\caption{Asset Display Screenlet using the Default View.}
\end{figure}

The Default View uses different UI elements to show each asset type. For
example, it displays images with \texttt{ImageView} and blogs with
\texttt{TextView}. Note that other Views may use different UI elements.

This Screenlet can also render other Screenlets as inner Screenlets:

\begin{itemize}
\tightlist
\item
  Images: Image Display Screenlet
\item
  Videos: Video Display Screenlet
\item
  Audio: Audio Display Screenlet
\item
  PDFs: PDF Display Screenlet
\item
  Blog entries: Blogs Entry Display Screenlet
\item
  Web content: Web Content Display Screenlet
\end{itemize}

These Screenlets can also be used alone without Asset Display Screenlet.

\subsection{Offline}\label{offline-14}

This Screenlet supports offline mode so it can function without a
network connection. For more information on how offline mode works, see
the
\href{/docs/7-1/tutorials/-/knowledge_base/t/architecture-of-offline-mode-in-liferay-screens}{tutorial
on its architecture}. Here are the offline mode policies that you can
use with this Screenlet:

\noindent\hrulefill

Policy \textbar{} What happens \textbar{} When to use \textbar{}
\texttt{REMOTE\_ONLY} \textbar{} The Screenlet loads the data from the
Liferay instance. If a connection issue occurs, the Screenlet uses the
listener to notify the developer about the error. If the Screenlet
successfully loads the data, it stores it in the local cache for later
use. \textbar{} Use this policy when you always need to show updated
data, and show nothing when there's no connection. \textbar{}
\texttt{CACHE\_ONLY} \textbar{} The Screenlet loads the data from the
local cache. If the data isn't there, the Screenlet uses the listener to
notify the developer about the error. \textbar{} Use this policy when
you always need to show local data, without retrieving remote
information under any circumstance. \textbar{} \texttt{REMOTE\_FIRST}
\textbar{} The Screenlet loads the data from the Liferay instance. If
this succeeds, the Screenlet shows the data to the user and stores it in
the local cache for later use. If a connection issue occurs, the
Screenlet retrieves the data from the local cache. If the data doesn't
exist there, the Screenlet uses the listener to notify the developer
about the error. \textbar{} Use this policy to show the most recent
version of the data when connected, but show an outdated version when
there's no connection. \textbar{} \texttt{CACHE\_FIRST} \textbar{} The
Screenlet loads the data from the local cache. If the data isn't there,
the Screenlet requests it from the Liferay instance and notifies the
developer about any errors that occur (including connectivity errors).
\textbar{} Use this policy to save bandwidth and loading time in case
you have local (but probably outdated) data. \textbar{}

\noindent\hrulefill

\subsection{Required Attributes}\label{required-attributes-14}

\begin{itemize}
\tightlist
\item
  \texttt{entryId}
\end{itemize}

Instead of \texttt{entryId}, you can use both of the following
attributes:

\begin{itemize}
\tightlist
\item
  \texttt{className}
\item
  \texttt{classPK}
\end{itemize}

If you don't use \texttt{entryId}, \texttt{className}, or
\texttt{classPK}, you must use this attribute:

\begin{itemize}
\tightlist
\item
  \texttt{portletItemName}
\end{itemize}

\subsection{Attributes}\label{attributes-14}

\noindent\hrulefill

Attribute \textbar{} Data type \textbar{} Explanation \textbar{}
\texttt{layoutId} \textbar{} \texttt{@layout} \textbar{} The layout to
use to show the View. \textbar{} \texttt{autoLoad} \textbar{}
\texttt{boolean} \textbar{} Whether the asset automatically loads when
the Screenlet appears in the app's UI. The default value is
\texttt{true}. \textbar{} \texttt{entryId} \textbar{} \texttt{number}
\textbar{} The primary key of the asset. \textbar{} \texttt{className}
\textbar{} \texttt{string} \textbar{} The asset's fully qualified class
name. For example, a blog entry's \texttt{className} is
\texttt{com.liferay.blogs.model.BlogsEntry}. The \texttt{className} and
\texttt{classPK} attributes are required to instantiate the Screenlet.
\textbar{} \texttt{classPK} \textbar{} \texttt{number} \textbar{} The
asset's unique identifier. The \texttt{className} and \texttt{classPK}
attributes are required to instantiate the Screenlet. \textbar{}
\texttt{portletItemName} \textbar{} \texttt{string} \textbar{} The name
of the
\href{/docs/7-1/user/-/knowledge_base/u/configuration-templates}{configuration
template} you used in the Asset Publisher. To use this feature, add an
Asset Publisher to one of your site's pages (it may be a hidden page),
configure the Asset Publisher's filter (in \emph{Configuration} →
\emph{Setup} → \emph{Asset Selection}), and then use the Asset
Publisher's \emph{Configuration Templates} option to save this
configuration with a name. Use this name in this attribute. If there is
more than one asset in the configuration, the Screenlet displays only
the first one. \textbar{} \texttt{cachePolicy} \textbar{}
\texttt{string} \textbar{} The offline mode setting. See
\href{/docs/7-1/reference/-/knowledge_base/r/asset-display-screenlet-for-android\#offline}{the
Offline section} for details. \textbar{} \texttt{imageLayoutId}
\textbar{} \texttt{@layout} \textbar{} The layout to use to show an
image (\texttt{DLFileEntry}). \textbar{} \texttt{videoLayoutId}
\textbar{} \texttt{@layout} \textbar{} The layout to use to show a video
(\texttt{DLFileEntry}). \textbar{} \texttt{audioLayoutId} \textbar{}
\texttt{@layout} \textbar{} The layout to use to show an audio file
(\texttt{DLFileEntry}). \textbar{} \texttt{pdfLayoutId} \textbar{}
\texttt{@layout} \textbar{} The layout to use to show a PDF
(\texttt{DLFileEntry}). \textbar{} \texttt{blogsLayoutId} \textbar{}
\texttt{@layout} \textbar{} The layout to use to show a blog entry
(\texttt{BlogsEntry}). \textbar{} \texttt{webDisplayLayoutId} \textbar{}
\texttt{@layout} \textbar{} The layout to use to show a web content
article (\texttt{WebContent}). \textbar{}

\noindent\hrulefill

\subsection{Methods}\label{methods-10}

\noindent\hrulefill

Method \textbar{} Return \textbar{} Explanation \textbar{}
\texttt{load(AssetEntry\ assetEntry)} \textbar{} \texttt{void}
\textbar{} Loads the given \texttt{AssetEntry} in the Screenlet. If no
inner Screenlet is instantiated, the method tries to load the asset with
a custom asset listener method. If this returns \texttt{null}, a new
\texttt{Intent} is called to display the asset. \textbar{}
\texttt{loadAsset()} \textbar{} \texttt{void} \textbar{} Searches for
the \texttt{AssetEntry} defined by the required attributes and loads it
in the Screenlet. \textbar{}

\noindent\hrulefill

\subsection{Listener}\label{listener-14}

Asset Display Screenlet delegates some events to a class that implements
\texttt{AssetDisplayListener}. This interface contains the following
methods:

\begin{itemize}
\tightlist
\item
  \texttt{onRetrieveAssetSuccess(AssetEntry\ assetEntry)}: Called when
  the Screenlet successfully loads the asset.
\end{itemize}

A second listener, \texttt{AssetDisplayInnerScreenletListener}, also
exists for configuring a child Screenlet (the Screenlet rendered inside
Asset Display Screenlet) or rendering a custom asset.

\begin{itemize}
\item
  \texttt{onConfigureChildScreenlet(AssetDisplayScreenlet\ screenlet,\ BaseScreenlet\ innerScreenlet,\ AssetEntry\ assetEntry)}:
  Called when the child Screenlet has been successfully initialized. Use
  this method to configure or customize the child Screenlet. The example
  implementation here sets the child Screenlet's background color to
  light gray if the asset is a blog entry entity (\texttt{BlogsEntry}):

\begin{verbatim}
  @Override
  public void onConfigureChildScreenlet(AssetDisplayScreenlet screenlet,
      BaseScreenlet innerScreenlet, AssetEntry assetEntry) {
          if ("blogsEntry".equals(assetEntry.getObjectType())) {
              innerScreenlet.setBackgroundColor(ContextCompat.getColor(this,
              R.color.light_gray));
          }
  }
\end{verbatim}
\item
  \texttt{onRenderCustomAsset(AssetEntry\ assetEntry)}: Called to render
  a custom asset. The following example implementation inflates and
  returns the custom View necessary to render a user from a Liferay
  instance (\texttt{User}):

\begin{verbatim}
  @Override
  public View onRenderCustomAsset(AssetEntry assetEntry) {
      if (assetEntry instanceof User) {
          View view = getLayoutInflater().inflate(R.layout.user_display, null);
          User user = (User) assetEntry;

          TextView username = (TextView) view.findViewById(R.id.liferay_username);

          username(user.getUsername());

          return view;
      }
      return null;
  }
\end{verbatim}
\end{itemize}

\section{Blogs Entry Display Screenlet for
Android}\label{blogs-entry-display-screenlet-for-android}

\subsection{Requirements}\label{requirements-15}

\begin{itemize}
\tightlist
\item
  Android SDK 4.1 (API Level 16) or above
\item
  Liferay Portal 6.2 CE/EE, Liferay CE Portal 7.0/7.1, Liferay DXP
\item
  Liferay Screens Compatibility app
  (\href{http://www.liferay.com/marketplace/-/mp/application/54365664}{CE}
  or
  \href{http://www.liferay.com/marketplace/-/mp/application/54369726}{EE/DXP}).
  This app is preinstalled in Liferay CE Portal 7.0/7.1 and Liferay DXP.
\end{itemize}

\subsection{Compatibility}\label{compatibility-15}

\begin{itemize}
\tightlist
\item
  Android SDK 4.1 (API Level 16) or above
\end{itemize}

\subsection{Xamarin Requirements}\label{xamarin-requirements-15}

\begin{itemize}
\tightlist
\item
  Visual Studio 7.2
\item
  Mono .NET framework 5.4.1.6
\end{itemize}

\subsection{Features}\label{features-15}

Blogs Entry Display Screenlet displays a single blog entry. Image
Display Screenlet renders any header image the blogs entry may have.

\subsection{JSON Services Used}\label{json-services-used-15}

Screenlets in Liferay Screens call JSON web services in the portal. This
Screenlet calls the following services and methods.

\noindent\hrulefill

\begin{longtable}[]{@{}
  >{\raggedright\arraybackslash}p{(\columnwidth - 4\tabcolsep) * \real{0.3889}}
  >{\raggedright\arraybackslash}p{(\columnwidth - 4\tabcolsep) * \real{0.3333}}
  >{\raggedright\arraybackslash}p{(\columnwidth - 4\tabcolsep) * \real{0.2778}}@{}}
\toprule\noalign{}
\begin{minipage}[b]{\linewidth}\raggedright
Service
\end{minipage} & \begin{minipage}[b]{\linewidth}\raggedright
Method
\end{minipage} & \begin{minipage}[b]{\linewidth}\raggedright
Notes
\end{minipage} \\
\midrule\noalign{}
\endhead
\bottomrule\noalign{}
\endlastfoot
\texttt{ScreensassetentryService} (Screens compatibility plugin) &
\texttt{getAssetEntry} & With \texttt{entryId} \\
\texttt{ScreensassetentryService} (Screens compatibility plugin) &
\texttt{getAssetEntry} & With \texttt{classPK} and \texttt{className} \\
\texttt{ScreensassetentryService} (Screens compatibility plugin) &
\texttt{getAssetEntries} & With \texttt{entryQuery} \\
\texttt{ScreensassetentryService} (Screens compatibility plugin) &
\texttt{getAssetEntries} & With \texttt{companyId}, \texttt{groupId},
and \texttt{portletItemName} \\
\end{longtable}

\noindent\hrulefill

\subsection{Module}\label{module-15}

\begin{itemize}
\tightlist
\item
  None
\end{itemize}

\subsection{Views}\label{views-15}

\begin{itemize}
\tightlist
\item
  Default
\end{itemize}

The Default View uses different components to show a blogs entry
(\texttt{BlogsEntry}). For example, it uses an Android \texttt{TextView}
to show the blog's text, and
\href{/docs/7-1/reference/-/knowledge_base/r/userportraitscreenlet-for-android}{User
Portrait Screenlet} to show the profile picture of the Liferay user who
posted it. Note that other custom Views may use different components.

\begin{figure}
\centering
\includegraphics{./images/screens-android-blogsentrydisplay.png}
\caption{Blogs Entry Display Screenlet using the Default View.}
\end{figure}

\subsection{Offline}\label{offline-15}

This Screenlet supports offline mode so it can function without a
network connection. For more information on how offline mode works, see
the
\href{/docs/7-1/tutorials/-/knowledge_base/t/architecture-of-offline-mode-in-liferay-screens}{tutorial
on its architecture}. Here are the offline mode policies that you can
use with this Screenlet:

\noindent\hrulefill

Policy \textbar{} What happens \textbar{} When to use \textbar{}
\texttt{REMOTE\_ONLY} \textbar{} The Screenlet loads the data from the
Liferay instance. If a connection issue occurs, the Screenlet uses the
listener to notify the developer about the error. If the Screenlet
successfully loads the data, it stores it in the local cache for later
use. \textbar{} Use this policy when you always need to show updated
data, and show nothing when there's no connection. \textbar{}
\texttt{CACHE\_ONLY} \textbar{} The Screenlet loads the data from the
local cache. If the data isn't there, the Screenlet uses the listener to
notify the developer about the error. \textbar{} Use this policy when
you always need to show local data, without retrieving remote data under
any circumstance. \textbar{} \texttt{REMOTE\_FIRST} \textbar{} The
Screenlet loads the data from the Liferay instance. If this succeeds,
the Screenlet shows the data to the user and stores it in the local
cache for later use. If a connection issue occurs, the Screenlet
retrieves the data from the local cache. If the data doesn't exist
there, the Screenlet uses the listener to notify the developer about the
error. \textbar{} Use this policy to show the most recent version of the
data when connected, but show an outdated version when there's no
connection. \textbar{} \texttt{CACHE\_FIRST} \textbar{} The Screenlet
loads the data from the local cache. If the data isn't there, the
Screenlet requests it from the Liferay instance and notifies the
developer about any errors that occur (including connectivity errors).
\textbar{} Use this policy to save bandwidth and loading time in case
you have local (but probably outdated) data. \textbar{}

\noindent\hrulefill

\subsection{Required Attributes}\label{required-attributes-15}

\begin{itemize}
\tightlist
\item
  \texttt{entryId}
\end{itemize}

If you don't use \texttt{entryId}, you must use both of the following
attributes:

\begin{itemize}
\tightlist
\item
  \texttt{className}
\item
  \texttt{classPK}
\end{itemize}

\subsection{Attributes}\label{attributes-15}

\noindent\hrulefill

Attribute \textbar{} Data type \textbar{} Explanation \textbar{}
\texttt{layoutId} \textbar{} \texttt{@layout} \textbar{} The layout to
use to show the View.\textbar{} \texttt{autoLoad} \textbar{}
\texttt{boolean} \textbar{} Whether the blog entry automatically loads
when the Screenlet appears in the app's UI. The default value is
\texttt{true}. \textbar{} \texttt{entryId} \textbar{} \texttt{number}
\textbar{} The primary key of the blog entry (\texttt{BlogsEntry}).
\textbar{} \texttt{className} \textbar{} \texttt{string} \textbar{} The
\texttt{BlogsEntry} object's fully qualified class name. This is
\texttt{com.liferay.blogs.model.BlogsEntry}. If you don't use
\texttt{entryId}, the \texttt{className} and \texttt{classPK} attributes
are required to instantiate the Screenlet. \textbar{} \texttt{classPK}
\textbar{} \texttt{number} \textbar{} The \texttt{BlogsEntry} object's
unique identifier. If you don't use \texttt{entryId}, the
\texttt{className} and \texttt{classPK} attributes are required to
instantiate the Screenlet. \textbar{} \texttt{cachePolicy} \textbar{}
\texttt{string} \textbar{} The offline mode setting. See
\href{/docs/7-1/reference/-/knowledge_base/r/blogs-entry-display-screenlet-for-android\#offline}{the
Offline section} for details. \textbar{}

\noindent\hrulefill

\subsection{Listener}\label{listener-15}

Because a blog entry is an asset, Blogs Entry Display Screenlet
delegates its events to a class that implements
\texttt{AssetDisplayListener}. This interface lets you implement the
following method:

\begin{itemize}
\item
  \texttt{onRetrieveAssetSuccess(AssetEntry\ assetEntry)}: Called when
  the Screenlet successfully loads the blog entry.
\item
  \texttt{error(Exception\ e,\ String\ userAction)}: Called when an
  error occurs in the process. The \texttt{userAction} argument
  distinguishes the specific action in which the error occurred.
\end{itemize}

\section{Image Display Screenlet for
Android}\label{image-display-screenlet-for-android}

\subsection{Requirements}\label{requirements-16}

\begin{itemize}
\tightlist
\item
  Android SDK 4.1 (API Level 16) or above
\item
  Liferay Portal 6.2 CE/EE, Liferay CE Portal 7.0/7.1, Liferay DXP
\item
  Liferay Screens Compatibility app
  (\href{http://www.liferay.com/marketplace/-/mp/application/54365664}{CE}
  or
  \href{http://www.liferay.com/marketplace/-/mp/application/54369726}{EE/DXP}).
  This app is preinstalled in Liferay CE Portal 7.0/7.1 and Liferay DXP.
\end{itemize}

\subsection{Compatibility}\label{compatibility-16}

\begin{itemize}
\tightlist
\item
  Android SDK 4.1 (API Level 16) or above
\end{itemize}

\subsection{Xamarin Requirements}\label{xamarin-requirements-16}

\begin{itemize}
\tightlist
\item
  Visual Studio 7.2
\item
  Mono .NET framework 5.4.1.6
\end{itemize}

\subsection{Features}\label{features-16}

Image Display Screenlet displays an image file from a Liferay instance's
Documents and Media Library.

\subsection{JSON Services Used}\label{json-services-used-16}

Screenlets in Liferay Screens call JSON web services in the portal. This
Screenlet calls the following services and methods.

\noindent\hrulefill

\begin{longtable}[]{@{}
  >{\raggedright\arraybackslash}p{(\columnwidth - 4\tabcolsep) * \real{0.3889}}
  >{\raggedright\arraybackslash}p{(\columnwidth - 4\tabcolsep) * \real{0.3333}}
  >{\raggedright\arraybackslash}p{(\columnwidth - 4\tabcolsep) * \real{0.2778}}@{}}
\toprule\noalign{}
\begin{minipage}[b]{\linewidth}\raggedright
Service
\end{minipage} & \begin{minipage}[b]{\linewidth}\raggedright
Method
\end{minipage} & \begin{minipage}[b]{\linewidth}\raggedright
Notes
\end{minipage} \\
\midrule\noalign{}
\endhead
\bottomrule\noalign{}
\endlastfoot
\texttt{ScreensassetentryService} (Screens compatibility plugin) &
\texttt{getAssetEntry} & With \texttt{entryId} \\
\texttt{ScreensassetentryService} (Screens compatibility plugin) &
\texttt{getAssetEntry} & With \texttt{classPK} and \texttt{className} \\
\texttt{ScreensassetentryService} (Screens compatibility plugin) &
\texttt{getAssetEntries} & With \texttt{entryQuery} \\
\texttt{ScreensassetentryService} (Screens compatibility plugin) &
\texttt{getAssetEntries} & With \texttt{companyId}, \texttt{groupId},
and \texttt{portletItemName} \\
\end{longtable}

\noindent\hrulefill

\subsection{Module}\label{module-16}

\begin{itemize}
\tightlist
\item
  None
\end{itemize}

\subsection{Views}\label{views-16}

\begin{itemize}
\tightlist
\item
  Default
\end{itemize}

The Default View uses an Android \texttt{ImageView} to display the
image.

\begin{figure}
\centering
\includegraphics{./images/screens-android-imagedisplay.png}
\caption{Image Display Screenlet using the Default View.}
\end{figure}

\subsection{Offline}\label{offline-16}

This Screenlet supports offline mode so it can function without a
network connection. For more information on how offline mode works, see
the
\href{/docs/7-1/tutorials/-/knowledge_base/t/architecture-of-offline-mode-in-liferay-screens}{tutorial
on its architecture}. Here are the offline mode policies that you can
use with this Screenlet:

\noindent\hrulefill

Policy \textbar{} What happens \textbar{} When to use \textbar{}
\texttt{REMOTE\_ONLY} \textbar{} The Screenlet loads the data from the
Liferay instance. If a connection issue occurs, the Screenlet uses the
listener to notify the developer about the error. If the Screenlet
successfully loads the data, it stores it in the local cache for later
use. \textbar{} Use this policy when you always need to show updated
data, and show nothing when there's no connection. \textbar{}
\texttt{CACHE\_ONLY} \textbar{} The Screenlet loads the data from the
local cache. If the data isn't there, the Screenlet uses the listener to
notify the developer about the error. \textbar{} Use this policy when
you always need to show local data, without retrieving remote
information under any circumstance. \textbar{} \texttt{REMOTE\_FIRST}
\textbar{} The Screenlet loads the data from the Liferay instance. If
this succeeds, the Screenlet shows the data to the user and stores it in
the local cache for later use. If a connection issue occurs, the
Screenlet retrieves the data from the local cache. If the data doesn't
exist there, the Screenlet uses the listener to notify the developer
about the error. \textbar{} Use this policy to show the most recent
version of the data when connected, but show an outdated version when
there's no connection. \textbar{} \texttt{CACHE\_FIRST} \textbar{} The
Screenlet loads the data from the local cache. If the data isn't there,
the Screenlet requests it from the Liferay instance and notifies the
developer about any errors that occur (including connectivity errors).
\textbar{} Use this policy to save bandwidth and loading time in case
you have local (but probably outdated) data. \textbar{}

\noindent\hrulefill

\subsection{Required Attributes}\label{required-attributes-16}

\begin{itemize}
\tightlist
\item
  \texttt{entryId} or \texttt{classPK}
\end{itemize}

\subsection{Attributes}\label{attributes-16}

\noindent\hrulefill

Attribute \textbar{} Data type \textbar{} Explanation \textbar{}
\texttt{layoutId} \textbar{} \texttt{@layout} \textbar{} The layout to
use to show the View. \textbar{} \texttt{autoLoad} \textbar{}
\texttt{boolean} \textbar{} Whether the image automatically loads when
the Screenlet appears in the app's UI. The default value is
\texttt{true}. \textbar{} \texttt{entryId} \textbar{} \texttt{number}
\textbar{} The primary key of the image. \textbar{} \texttt{classPK}
\textbar{} \texttt{number} \textbar{} The image's unique identifier.
\textbar{} \texttt{cachePolicy} \textbar{} \texttt{string} \textbar{}
The offline mode setting. See
\href{/docs/7-1/reference/-/knowledge_base/r/image-display-screenlet-for-android\#offline}{the
Offline section} for details. \textbar{} \texttt{imageScaleType}
\textbar{} \texttt{number} \textbar{} Lets you set a scale image type
like \texttt{CENTER}, \texttt{CENTER\_CROP}, \texttt{CENTER\_INSIDE},
\texttt{FIT\_CENTER}, \texttt{FIT\_END}, \texttt{FIT\_START},
\texttt{FIT\_XY}, \texttt{MATRIX}. \textbar{} \texttt{placeHolder}
\textbar{} \texttt{@resource} \textbar{} Image to load until the final
image loads. \textbar{} \texttt{placeHolderScaleType} \textbar{}
\texttt{number} \textbar{} Lets you set a scale image type for the
placeholder like \texttt{CENTER}, \texttt{CENTER\_CROP},
\texttt{CENTER\_INSIDE}, \texttt{FIT\_CENTER}, \texttt{FIT\_END},
\texttt{FIT\_START}, \texttt{FIT\_XY}, \texttt{MATRIX}. \textbar{}

\noindent\hrulefill

Note that the values for \texttt{imageScaleType} and
\texttt{placeHolderScaleType} match those
\href{https://developer.android.com/reference/android/widget/ImageView.ScaleType.html}{described
in Android's \texttt{ImageView.ScaleType}}.

\subsection{Listener}\label{listener-16}

Because images are assets, Image Display Screenlet delegates its events
to a class that implements \texttt{AssetDisplayListener}. This interface
lets you implement the following methods:

\begin{itemize}
\item
  \texttt{onRetrieveAssetSuccess(AssetEntry\ assetEntry)}: Called when
  the Screenlet successfully loads the image.
\item
  \texttt{error(Exception\ e,\ String\ userAction)}: Called when an
  error occurs in the process. The \texttt{userAction} argument
  distinguishes the specific action in which the error occurred.
\end{itemize}

\section{Video Display Screenlet for
Android}\label{video-display-screenlet-for-android}

\subsection{Requirements}\label{requirements-17}

\begin{itemize}
\tightlist
\item
  Android SDK 4.1 (API Level 16) or above
\item
  Liferay Portal 6.2 CE/EE, Liferay CE Portal 7.0/7.1, Liferay DXP
\item
  Liferay Screens Compatibility app
  (\href{http://www.liferay.com/marketplace/-/mp/application/54365664}{CE}
  or
  \href{http://www.liferay.com/marketplace/-/mp/application/54369726}{EE/DXP}).
  This app is preinstalled in Liferay CE Portal 7.0/7.1 and Liferay DXP.
\end{itemize}

\subsection{Compatibility}\label{compatibility-17}

\begin{itemize}
\tightlist
\item
  Android SDK 4.1 (API Level 16) or above
\end{itemize}

\subsection{Xamarin Requirements}\label{xamarin-requirements-17}

\begin{itemize}
\tightlist
\item
  Visual Studio 7.2
\item
  Mono .NET framework 5.4.1.6
\end{itemize}

\subsection{Features}\label{features-17}

Video Display Screenlet displays a video file from a Liferay instance's
Documents and Media Library.

\subsection{JSON Services Used}\label{json-services-used-17}

Screenlets in Liferay Screens call JSON web services in the portal. This
Screenlet calls the following services and methods.

\noindent\hrulefill

\begin{longtable}[]{@{}
  >{\raggedright\arraybackslash}p{(\columnwidth - 4\tabcolsep) * \real{0.3889}}
  >{\raggedright\arraybackslash}p{(\columnwidth - 4\tabcolsep) * \real{0.3333}}
  >{\raggedright\arraybackslash}p{(\columnwidth - 4\tabcolsep) * \real{0.2778}}@{}}
\toprule\noalign{}
\begin{minipage}[b]{\linewidth}\raggedright
Service
\end{minipage} & \begin{minipage}[b]{\linewidth}\raggedright
Method
\end{minipage} & \begin{minipage}[b]{\linewidth}\raggedright
Notes
\end{minipage} \\
\midrule\noalign{}
\endhead
\bottomrule\noalign{}
\endlastfoot
\texttt{ScreensassetentryService} (Screens compatibility plugin) &
\texttt{getAssetEntry} & With \texttt{entryId} \\
\texttt{ScreensassetentryService} (Screens compatibility plugin) &
\texttt{getAssetEntry} & With \texttt{classPK} and \texttt{className} \\
\texttt{ScreensassetentryService} (Screens compatibility plugin) &
\texttt{getAssetEntries} & With \texttt{entryQuery} \\
\texttt{ScreensassetentryService} (Screens compatibility plugin) &
\texttt{getAssetEntries} & With \texttt{companyId}, \texttt{groupId},
and \texttt{portletItemName} \\
\end{longtable}

\noindent\hrulefill

\subsection{Module}\label{module-17}

\begin{itemize}
\tightlist
\item
  None
\end{itemize}

\subsection{Views}\label{views-17}

\begin{itemize}
\tightlist
\item
  Default
\end{itemize}

The Default View uses an Android \texttt{VideoView} to display the
video.

\begin{figure}
\centering
\includegraphics{./images/screens-android-videodisplay.png}
\caption{Video Display Screenlet using the Default View.}
\end{figure}

\subsection{Offline}\label{offline-17}

This Screenlet supports offline mode so it can function without a
network connection. For more information on how offline mode works, see
the
\href{/docs/7-1/tutorials/-/knowledge_base/t/architecture-of-offline-mode-in-liferay-screens}{tutorial
on its architecture}. Here are the offline mode policies that you can
use with this Screenlet:

\noindent\hrulefill

Policy \textbar{} What happens \textbar{} When to use \textbar{}
\texttt{REMOTE\_ONLY} \textbar{} The Screenlet loads the data from the
Liferay instance. If a connection issue occurs, the Screenlet uses the
listener to notify the developer about the error. If the Screenlet
successfully loads the data, it stores it in the local cache for later
use. \textbar{} Use this policy when you always need to show updated
data, and show nothing when there's no connection. \textbar{}
\texttt{CACHE\_ONLY} \textbar{} The Screenlet loads the data from the
local cache. If the data isn't there, the Screenlet uses the listener to
notify the developer about the error. \textbar{} Use this policy when
you always need to show local data, without retrieving remote
information under any circumstance. \textbar{} \texttt{REMOTE\_FIRST}
\textbar{} The Screenlet loads the data from the Liferay instance. If
this succeeds, the Screenlet shows the data to the user and stores it in
the local cache for later use. If a connection issue occurs, the
Screenlet retrieves the data from the local cache. If the data doesn't
exist there, the Screenlet uses the listener to notify the developer
about the error. \textbar{} Use this policy to show the most recent
version of the data when connected, but show an outdated version when
there's no connection. \textbar{} \texttt{CACHE\_FIRST} \textbar{} The
Screenlet loads the data from the local cache. If the data isn't there,
the Screenlet requests it from the Liferay instance and notifies the
developer about any errors that occur (including connectivity errors).
\textbar{} Use this policy to save bandwidth and loading time in case
you have local (but probably outdated) data. \textbar{}

\noindent\hrulefill

\subsection{Required Attributes}\label{required-attributes-17}

\begin{itemize}
\tightlist
\item
  \texttt{entryId} or \texttt{classPK}
\end{itemize}

\subsection{Attributes}\label{attributes-17}

\noindent\hrulefill

Attribute \textbar{} Data type \textbar{} Explanation \textbar{}
\texttt{layoutId} \textbar{} \texttt{@layout} \textbar{} The layout to
use to show the View. \textbar{} \texttt{autoLoad} \textbar{}
\texttt{boolean} \textbar{} Whether the video automatically loads when
the Screenlet appears in the app's UI. The default value is
\texttt{true}. \textbar{} \texttt{entryId} \textbar{} \texttt{number}
\textbar{} The primary key of the video file. \textbar{}
\texttt{classPK} \textbar{} \texttt{number} \textbar{} The video file's
unique identifier. \textbar{} \texttt{cachePolicy} \textbar{}
\texttt{string} \textbar{} The offline mode setting. See
\href{/docs/7-1/reference/-/knowledge_base/r/video-display-screenlet-for-android\#offline}{the
Offline section} for details. \textbar{}

\noindent\hrulefill

\subsection{Listener}\label{listener-17}

Video Display Screenlet delegates its events to a class that implements
\texttt{VideoDisplayListener}. This interface lets you implement these
methods:

\begin{itemize}
\item
  \texttt{onVideoPrepared()}: Called when the video is ready for
  display.
\item
  \texttt{onVideoCompleted()}: Called when the video is completed.
\item
  \texttt{onVideoError(Exception\ e)}: Called when an error occurs
  displaying the video.
\item
  \texttt{onRetrieveAssetSuccess(AssetEntry\ assetEntry)}: Called when
  the Screenlet successfully loads the video.
\item
  \texttt{error(Exception\ e,\ String\ userAction)}: Called when an
  error occurs in the process. The \texttt{userAction} argument
  distinguishes the specific action in which the error occurred.
\end{itemize}

\section{Audio Display Screenlet for
Android}\label{audio-display-screenlet-for-android}

\subsection{Requirements}\label{requirements-18}

\begin{itemize}
\tightlist
\item
  Android SDK 4.1 (API Level 16) or above
\item
  Liferay Portal 6.2 CE/EE, Liferay CE Portal 7.0/7.1, Liferay DXP
\item
  Liferay Screens Compatibility app
  (\href{http://www.liferay.com/marketplace/-/mp/application/54365664}{CE}
  or
  \href{http://www.liferay.com/marketplace/-/mp/application/54369726}{EE/DXP}).
  This app is preinstalled in Liferay CE Portal 7.0/7.1 and Liferay DXP.
\end{itemize}

\subsection{Compatibility}\label{compatibility-18}

\begin{itemize}
\tightlist
\item
  Android SDK 4.1 (API Level 16) or above
\end{itemize}

\subsection{Xamarin Requirements}\label{xamarin-requirements-18}

\begin{itemize}
\tightlist
\item
  Visual Studio 7.2
\item
  Mono .NET framework 5.4.1.6
\end{itemize}

\subsection{Features}\label{features-18}

Audio Display Screenlet displays an audio file from a Liferay instance's
Documents and Media Library.

\subsection{JSON Services Used}\label{json-services-used-18}

Screenlets in Liferay Screens call JSON web services in the portal. This
Screenlet calls the following services and methods.

\noindent\hrulefill

\begin{longtable}[]{@{}
  >{\raggedright\arraybackslash}p{(\columnwidth - 4\tabcolsep) * \real{0.3889}}
  >{\raggedright\arraybackslash}p{(\columnwidth - 4\tabcolsep) * \real{0.3333}}
  >{\raggedright\arraybackslash}p{(\columnwidth - 4\tabcolsep) * \real{0.2778}}@{}}
\toprule\noalign{}
\begin{minipage}[b]{\linewidth}\raggedright
Service
\end{minipage} & \begin{minipage}[b]{\linewidth}\raggedright
Method
\end{minipage} & \begin{minipage}[b]{\linewidth}\raggedright
Notes
\end{minipage} \\
\midrule\noalign{}
\endhead
\bottomrule\noalign{}
\endlastfoot
\texttt{ScreensassetentryService} (Screens compatibility plugin) &
\texttt{getAssetEntry} & With \texttt{entryId} \\
\texttt{ScreensassetentryService} (Screens compatibility plugin) &
\texttt{getAssetEntry} & With \texttt{classPK} and \texttt{className} \\
\texttt{ScreensassetentryService} (Screens compatibility plugin) &
\texttt{getAssetEntries} & With \texttt{entryQuery} \\
\texttt{ScreensassetentryService} (Screens compatibility plugin) &
\texttt{getAssetEntries} & With \texttt{companyId}, \texttt{groupId},
and \texttt{portletItemName} \\
\end{longtable}

\noindent\hrulefill

\subsection{Module}\label{module-18}

\begin{itemize}
\tightlist
\item
  None
\end{itemize}

\subsection{Views}\label{views-18}

\begin{itemize}
\tightlist
\item
  Default
\end{itemize}

The Default View uses an Android \texttt{VideoView} to display the audio
file.

\begin{figure}
\centering
\includegraphics{./images/screens-android-audiodisplay.png}
\caption{Audio Display Screenlet using the Default View.}
\end{figure}

\subsection{Offline}\label{offline-18}

This Screenlet supports offline mode so it can function without a
network connection. For more information on how offline mode works, see
the
\href{/docs/7-1/tutorials/-/knowledge_base/t/architecture-of-offline-mode-in-liferay-screens}{tutorial
on its architecture}. Here are the offline mode policies that you can
use with this Screenlet:

\noindent\hrulefill

Policy \textbar{} What happens \textbar{} When to use \textbar{}
\texttt{REMOTE\_ONLY} \textbar{} The Screenlet loads the data from the
Liferay instance. If a connection issue occurs, the Screenlet uses the
listener to notify the developer about the error. If the Screenlet
successfully loads the data, it stores it in the local cache for later
use. \textbar{} Use this policy when you always need to show updated
data, and show nothing when there's no connection. \textbar{}
\texttt{CACHE\_ONLY} \textbar{} The Screenlet loads the data from the
local cache. If the data isn't there, the Screenlet uses the listener to
notify the developer about the error. \textbar{} Use this policy when
you always need to show local data, without retrieving remote
information under any circumstance. \textbar{} \texttt{REMOTE\_FIRST}
\textbar{} The Screenlet loads the data from the Liferay instance. If
this succeeds, the Screenlet shows the data to the user and stores it in
the local cache for later use. If a connection issue occurs, the
Screenlet retrieves the data from the local cache. If the data doesn't
exist there, the Screenlet uses the listener to notify the developer
about the error. \textbar{} Use this policy to show the most recent
version of the data when connected, but show an outdated version when
there's no connection. \textbar{} \texttt{CACHE\_FIRST} \textbar{} The
Screenlet loads the data from the local cache. If the data isn't there,
the Screenlet requests it from the Liferay instance and notifies the
developer about any errors that occur (including connectivity errors).
\textbar{} Use this policy to save bandwidth and loading time in case
you have local (but probably outdated) data. \textbar{}

\noindent\hrulefill

\subsection{Required Attributes}\label{required-attributes-18}

\begin{itemize}
\tightlist
\item
  \texttt{entryId} or \texttt{classPK}
\end{itemize}

\subsection{Attributes}\label{attributes-18}

\noindent\hrulefill

Attribute \textbar{} Data type \textbar{} Explanation \textbar{}
\texttt{layoutId} \textbar{} \texttt{@layout} \textbar{} The layout to
use to show the View. \textbar{} \texttt{autoLoad} \textbar{}
\texttt{boolean} \textbar{} Whether the audio file automatically loads
when the Screenlet appears in the app's UI. The default value is
\texttt{true}. \textbar{} \texttt{entryId} \textbar{} \texttt{number}
\textbar{} The primary key of the audio file. \textbar{}
\texttt{classPK} \textbar{} \texttt{number} \textbar{} The audio file's
unique identifier. \textbar{} \texttt{cachePolicy} \textbar{}
\texttt{string} \textbar{} The offline mode setting. See
\href{/docs/7-1/reference/-/knowledge_base/r/audio-display-screenlet-for-android\#offline}{the
Offline section} for details. \textbar{}

\noindent\hrulefill

\subsection{Listener}\label{listener-18}

Because audio files are assets, Audio Display Screenlet delegates its
events to a class that implements \texttt{AssetDisplayListener}. This
interface lets you implement the following methods:

\begin{itemize}
\item
  \texttt{onRetrieveAssetSuccess(AssetEntry\ assetEntry)}: Called when
  the Screenlet successfully loads the audio file.
\item
  \texttt{error(Exception\ e,\ String\ userAction)}: Called when an
  error occurs in the process. The \texttt{userAction} argument
  distinguishes the specific action in which the error occurred.
\end{itemize}

\section{PDF Display Screenlet for
Android}\label{pdf-display-screenlet-for-android}

\subsection{Requirements}\label{requirements-19}

\begin{itemize}
\tightlist
\item
  Android SDK 4.1 (API Level 16) or above
\item
  Liferay Portal 6.2 CE/EE, Liferay CE Portal 7.0/7.1, Liferay DXP
\item
  Liferay Screens Compatibility app
  (\href{http://www.liferay.com/marketplace/-/mp/application/54365664}{CE}
  or
  \href{http://www.liferay.com/marketplace/-/mp/application/54369726}{EE/DXP}).
  This app is preinstalled in Liferay CE Portal 7.0/7.1 and Liferay DXP.
\end{itemize}

\subsection{Compatibility}\label{compatibility-19}

\begin{itemize}
\tightlist
\item
  Android SDK 4.1 (API Level 16) or above
\end{itemize}

\subsection{Xamarin Requirements}\label{xamarin-requirements-19}

\begin{itemize}
\tightlist
\item
  Visual Studio 7.2
\item
  Mono .NET framework 5.4.1.6
\end{itemize}

\subsection{Features}\label{features-19}

PDF Display Screenlet displays a PDF file from a Liferay Instance's
Documents and Media Library.

\subsection{JSON Services Used}\label{json-services-used-19}

Screenlets in Liferay Screens call JSON web services in the portal. This
Screenlet calls the following services and methods.

\noindent\hrulefill

\begin{longtable}[]{@{}
  >{\raggedright\arraybackslash}p{(\columnwidth - 4\tabcolsep) * \real{0.3889}}
  >{\raggedright\arraybackslash}p{(\columnwidth - 4\tabcolsep) * \real{0.3333}}
  >{\raggedright\arraybackslash}p{(\columnwidth - 4\tabcolsep) * \real{0.2778}}@{}}
\toprule\noalign{}
\begin{minipage}[b]{\linewidth}\raggedright
Service
\end{minipage} & \begin{minipage}[b]{\linewidth}\raggedright
Method
\end{minipage} & \begin{minipage}[b]{\linewidth}\raggedright
Notes
\end{minipage} \\
\midrule\noalign{}
\endhead
\bottomrule\noalign{}
\endlastfoot
\texttt{ScreensassetentryService} (Screens compatibility plugin) &
\texttt{getAssetEntry} & With \texttt{entryId} \\
\texttt{ScreensassetentryService} (Screens compatibility plugin) &
\texttt{getAssetEntry} & With \texttt{classPK} and \texttt{className} \\
\texttt{ScreensassetentryService} (Screens compatibility plugin) &
\texttt{getAssetEntries} & With \texttt{entryQuery} \\
\texttt{ScreensassetentryService} (Screens compatibility plugin) &
\texttt{getAssetEntries} & With \texttt{companyId}, \texttt{groupId},
and \texttt{portletItemName} \\
\end{longtable}

\noindent\hrulefill

\subsection{Module}\label{module-19}

\begin{itemize}
\tightlist
\item
  None
\end{itemize}

\subsection{Views}\label{views-19}

\begin{itemize}
\tightlist
\item
  Default
\end{itemize}

The Default View uses Android's \texttt{PdfRenderer} to display the PDF.
Note that \texttt{PdfRenderer} requires an Android API level of 21 or
higher.

\begin{figure}
\centering
\includegraphics{./images/screens-android-pdfdisplay.png}
\caption{PDF Display Screenlet using the Default View.}
\end{figure}

\subsection{Offline}\label{offline-19}

This Screenlet supports offline mode so it can function without a
network connection. For more information on how offline mode works, see
the
\href{/docs/7-1/tutorials/-/knowledge_base/t/architecture-of-offline-mode-in-liferay-screens}{tutorial
on its architecture}. Here are the offline mode policies that you can
use with this Screenlet:

\noindent\hrulefill

Policy \textbar{} What happens \textbar{} When to use \textbar{}
\texttt{REMOTE\_ONLY} \textbar{} The Screenlet loads the data from the
Liferay instance. If a connection issue occurs, the Screenlet uses the
listener to notify the developer about the error. If the Screenlet
successfully loads the data, it stores it in the local cache for later
use. \textbar{} Use this policy when you always need to show updated
data, and show nothing when there's no connection. \textbar{}
\texttt{CACHE\_ONLY} \textbar{} The Screenlet loads the data from the
local cache. If the data isn't there, the Screenlet uses the listener to
notify the developer about the error. \textbar{} Use this policy when
you always need to show local data, without retrieving remote
information under any circumstance. \textbar{} \texttt{REMOTE\_FIRST}
\textbar{} The Screenlet loads the data from the Liferay instance. If
this succeeds, the Screenlet shows the data to the user and stores it in
the local cache for later use. If a connection issue occurs, the
Screenlet retrieves the data from the local cache. If the data doesn't
exist there, the Screenlet uses the listener to notify the developer
about the error. \textbar{} Use this policy to show the most recent
version of the data when connected, but show an outdated version when
there's no connection. \textbar{} \texttt{CACHE\_FIRST} \textbar{} The
Screenlet loads the data from the local cache. If the data isn't there,
the Screenlet requests it from the Liferay instance and notifies the
developer about any errors that occur (including connectivity errors).
\textbar{} Use this policy to save bandwidth and loading time in case
you have local (but probably outdated) data. \textbar{}

\noindent\hrulefill

\subsection{Required Attributes}\label{required-attributes-19}

\begin{itemize}
\tightlist
\item
  \texttt{entryId} or \texttt{classPK}
\end{itemize}

\subsection{Attributes}\label{attributes-19}

\noindent\hrulefill

Attribute \textbar{} Data type \textbar{} Explanation \textbar{}
\texttt{layoutId} \textbar{} \texttt{@layout} \textbar{} The layout to
use to show the View. \textbar{} \texttt{autoLoad} \textbar{}
\texttt{boolean} \textbar{} Whether the PDF automatically loads when the
Screenlet appears in the app's UI. The default value is \texttt{true}.
\textbar{} \texttt{entryId} \textbar{} \texttt{number} \textbar{} The
primary key of the PDF file. \textbar{} \texttt{classPK} \textbar{}
\texttt{number} \textbar{} The PDF file's unique identifier. \textbar{}
\texttt{cachePolicy} \textbar{} \texttt{string} \textbar{} The offline
mode setting. See
\href{/docs/7-1/reference/-/knowledge_base/r/pdf-display-screenlet-for-android\#offline}{the
Offline section} for details. \textbar{}

\noindent\hrulefill

\subsection{Listener}\label{listener-19}

Because PDF files are assets, PDF Display Screenlet delegates its events
to a class that implements \texttt{AssetDisplayListener}. This interface
lets you implement the following methods:

\begin{itemize}
\item
  \texttt{onRetrieveAssetSuccess(AssetEntry\ assetEntry)}: Called when
  the Screenlet successfully loads the PDF file.
\item
  \texttt{error(Exception\ e,\ String\ userAction)}: Called when an
  error occurs in the process. The \texttt{userAction} argument
  distinguishes the specific action in which the error occurred.
\end{itemize}

\section{Web Screenlet for Android}\label{web-screenlet-for-android}

\subsection{Requirements}\label{requirements-20}

\begin{itemize}
\tightlist
\item
  Android SDK 4.1 (API Level 16) or above
\item
  Liferay Portal 6.2 CE/EE, Liferay CE Portal 7.0/7.1, Liferay DXP
\item
  Liferay Screens Compatibility app
  (\href{http://www.liferay.com/marketplace/-/mp/application/54365664}{CE}
  or
  \href{http://www.liferay.com/marketplace/-/mp/application/54369726}{EE/DXP}).
  This app is preinstalled in Liferay CE Portal 7.0/7.1 and Liferay DXP.
\end{itemize}

\subsection{Compatibility}\label{compatibility-20}

\begin{itemize}
\tightlist
\item
  Android SDK 4.1 (API Level 16) or above
\end{itemize}

\subsection{Xamarin Requirements}\label{xamarin-requirements-20}

\begin{itemize}
\tightlist
\item
  Visual Studio 7.2
\item
  Mono .NET framework 5.4.1.6
\end{itemize}

\subsection{Features}\label{features-20}

Web Screenlet lets you display any web page. It also lets you customize
the web page through injection of local and remote JavaScript and CSS
files. If you're using Liferay DXP as backend, you can use
\href{/docs/7-1/user/-/knowledge_base/u/styling-apps-and-assets}{Application
Display Templates} in your page to customize its content from the server
side.

\subsection{Module}\label{module-20}

\begin{itemize}
\tightlist
\item
  None
\end{itemize}

\subsection{Views}\label{views-20}

\begin{itemize}
\tightlist
\item
  Default
\end{itemize}

\begin{figure}
\centering
\includegraphics{./images/screens-android-webscreenlet.png}
\caption{The Web Screenlet with the Default View Set.}
\end{figure}

\subsection{Configuration}\label{configuration}

To learn how to use Web Screenlet, follow the steps in the tutorial
\href{/docs/7-1/tutorials/-/knowledge_base/t/rendering-web-pages-in-your-android-app}{Rendering
Web Pages in Your Android App}. That tutorial gives detailed
instructions for using the configuration items described here.

Web Screenlet has \texttt{WebScreenletConfiguration} and
\texttt{WebScreenletConfiguration.Builder} classes that you can use
together to supply the parameters that the Screenlet needs to work.
\texttt{WebScreenletConfiguration.Builder} has the following methods,
which let you supply the described configuration parameters:

\noindent\hrulefill

Method \textbar{} Return \textbar{} Explanation \textbar{}
\texttt{addLocalJs(fileName)} \textbar{}
\texttt{WebScreenletConfiguration.Builder} \textbar{} Adds a local
JavaScript file with the supplied filename. The JavaScript files must be
in the first level of your app's \texttt{assets} folder. Create this
folder at the same level of the \texttt{res} folder. \textbar{}
\texttt{addLocalCss(fileName)} \textbar{}
\texttt{WebScreenletConfiguration.Builder} \textbar{} Adds a local CSS
file with the supplied filename. The CSS files must be in the first
level of your app's \texttt{assets} folder. Create this folder at the
same level of the \texttt{res} folder. \textbar{}
\texttt{addRawJs(rawJs,\ name)} \textbar{}
\texttt{WebScreenletConfiguration.Builder} \textbar{} Adds a JavaScript
file from your app's \texttt{res/raw} folder. Create this folder if it
doesn't exist. Reference the file using \texttt{R.raw.yourfilename}.
This method also takes a second parameter called \texttt{name}, which is
only for debugging purposes. If there's an error, the console displays
it with this \texttt{name} value. \textbar{}
\texttt{addRawCss(rawCss,\ name)} \textbar{}
\texttt{WebScreenletConfiguration.Builder} \textbar{} Adds a CSS file
from your app's \texttt{res/raw} folder. Create this folder if it
doesn't exist. Reference the file using \texttt{R.raw.yourfilename}.
This method also takes a second parameter called \texttt{name}, which is
only for debugging purposes. If there's an error, the console displays
it with this \texttt{name} value. \textbar{} \texttt{addRemoteJs(url)}
\textbar{} \texttt{WebScreenletConfiguration.Builder} \textbar{} Adds a
JavaScript file from the supplied URL. \textbar{}
\texttt{addRemoteCss(url)} \textbar{}
\texttt{WebScreenletConfiguration.Builder} \textbar{} Adds a CSS file
from the supplied URL. \textbar{} \texttt{setWebType(webType)}
\textbar{} \texttt{WebScreenletConfiguration.Builder} \textbar{} Sets
the
\href{/docs/7-1/reference/-/knowledge_base/r/web-screenlet-for-android\#webtype}{\texttt{WebType}}.
\textbar{} \texttt{enableCordova(observer)} \textbar{}
\texttt{WebScreenletConfiguration.Builder} \textbar{} Enables Cordova
inside the Web Screenlet. \textbar{} \texttt{load()} \textbar{}
\texttt{WebScreenletConfiguration} \textbar{} Returns the
\texttt{WebScreenletConfiguration} object that you can set to the
Screenlet instance. \textbar{}

\noindent\hrulefill

\noindent\hrulefill

\textbf{Note:} If you want to add comments in the scripts, use the
\texttt{/**/} notation.

\noindent\hrulefill

\subsubsection{WebType}\label{webtype}

\begin{itemize}
\item
  \textbf{WebType.LIFERAY\_AUTHENTICATED} (default): Displays a Liferay
  DXP page that requires authentication. The user must therefore be
  logged in with Screens via Login Screenlet or a
  \texttt{SessionContext} method. For this \texttt{WebType}, the URL you
  must pass to the \texttt{WebScreenletConfiguration.Builder}
  constructor is a relative URL. For example, if the full URL is
  \texttt{http://screens.liferay.org.es/web/guest/blog}, then the URL
  you must supply to the constructor is \texttt{/web/guest/blog}.
\item
  \textbf{WebType.OTHER}: Displays any other page. For this
  \texttt{WebType}, the URL you must pass to the
  \texttt{WebScreenletConfiguration.Builder} constructor is a full URL.
  For example, if the full URL is
  \texttt{http://screens.liferay.org.es/web/guest/blog}, then you must
  supply that URL to the constructor.
\end{itemize}

\subsection{Attributes}\label{attributes-20}

\noindent\hrulefill

Attribute \textbar{} Data type \textbar{} Explanation \textbar{}
\texttt{autoLoad} \textbar{} \texttt{boolean} \textbar{} Whether to load
the page automatically when the Screenlet appears in the app's UI. The
default value is \texttt{true}. \textbar{} \texttt{layoutId} \textbar{}
\texttt{@layout} \textbar{} The layout to use to show the View.
\textbar{} \texttt{isLoggingEnabled} \textbar{} \texttt{boolean}
\textbar{} Whether logging is enabled. \textbar{}
\texttt{isScrollEnabled} \textbar{} \texttt{boolean} \textbar{} Whether
to enable scrolling on the page inside the Screenlet. \textbar{}

\noindent\hrulefill

\subsection{Methods}\label{methods-11}

\noindent\hrulefill

Method \textbar{} Return \textbar{} Explanation \textbar{}
\texttt{load()} \textbar{} \texttt{void} \textbar{} Checks if the page's
URL is valid, and then loads it. The operation fails if the URL is
invalid. \textbar{} \texttt{clearCache()} \textbar{} \texttt{void}
\textbar{} Clears the Web Screenlet's cache. \textbar{}
\texttt{injectScript(script)} \textbar{} \texttt{void} \textbar{}
Injects a script when the page is already loaded. \textbar{}

\noindent\hrulefill

\subsection{Listener}\label{listener-20}

Web Screenlet delegates some events to an object or class that
implements its
\href{https://github.com/liferay/liferay-screens/blob/master/android/library/src/main/java/com/liferay/mobile/screens/web/WebListener.java}{\texttt{WebListener}
interface}. This interface extends the
\href{https://github.com/liferay/liferay-screens/blob/master/android/library/src/main/java/com/liferay/mobile/screens/base/interactor/listener/BaseCacheListener.java}{\texttt{BaseCacheListener}
interface}. Therefore, Web Screenlet's listener methods are as follows:

\begin{itemize}
\item
  \texttt{onPageLoaded(String\ url)}: Called when the Screenlet loads
  the page correctly.
\item
  \texttt{onScriptMessageHandler(String\ namespace,\ String\ body)}:
  Called when the \texttt{WebView} in the Screenlet sends a message. The
  \texttt{namespace} parameter is the source namespace key, and
  \texttt{body} is the source namespace body.
\item
  \texttt{error(Exception\ e,\ String\ userAction)}: Called when an
  error occurs in the process. The \texttt{userAction} argument
  distinguishes the specific action in which the error occurred.
\end{itemize}

\section{DDM Form Screenlet for
Android}\label{ddm-form-screenlet-for-android}

\subsection{Requirements}\label{requirements-21}

\begin{itemize}
\tightlist
\item
  Android SDK 4.1 (API Level 21) or above
\item
  Liferay DXP 7.1 SP2+
\item
  Liferay Hypermedia REST APIs. These APIs are installed but disabled by
  default. To enable them, follow the instructions in the tutorial
  \href{/docs/7-1/tutorials/-/knowledge_base/t/enabling-hypermedia-rest-apis}{Enabling
  Hypermedia REST APIs}.
\end{itemize}

\subsection{Compatibility}\label{compatibility-21}

\begin{itemize}
\tightlist
\item
  Android SDK 4.1 (API Level 21) or above
\end{itemize}

\subsection{Xamarin Requirements}\label{xamarin-requirements-21}

\begin{itemize}
\tightlist
\item
  Visual Studio 7.2
\item
  Mono .NET framework 5.4.1.6
\end{itemize}

\subsection{Features}\label{features-21}

DDM Form Screenlet shows a set of fields that can be filled in by the
user. The fields can contain initial or existing values. The following
fields are supported:

\textbf{Paragraph:} Add a title and/or text in your form.

\textbf{Text Field:} A single or multiline text area.

\textbf{Single Selection:} Select one item with a radio button.

\textbf{Select From List:} Choose one or more items in a list.

\textbf{Multiple Selection:} Select multiple items via a checkbox.

\textbf{Date:} Select a date from a date picker.

\textbf{Grid:} Select items in a matrix.

\textbf{Numeric:} Enter an integer or decimal number.

\textbf{Upload:} Upload files via Documents and Media.

DDM Form Screenlet also supports the following features:

\textbf{Element Sets:} Reuse pre-existing element sets in your form.

\textbf{Multiple Pages:} Use multi-page forms.

\textbf{Success Page:} Show friendly feedback at the end of your form.
\textbf{Autosave:} Automatically save any change in form values to a
draft.

\textbf{Restore Previous Draft:} Automatically restore the last draft
when opening the form, independent of platform.

\textbf{Rules:} Create complex rules in your form. For example, you can
show or hide fields depending on the input of other fields.

\textbf{Workflow:} Form submission can trigger a specific workflow.

\textbf{Required Values:} Require specific values and/or validate form
fields.

\textbf{Internationalization:} Support i18n in record values and labels.

\subsection{Module}\label{module-21}

\begin{itemize}
\tightlist
\item
  DDM
\end{itemize}

\subsection{Views}\label{views-21}

\begin{itemize}
\tightlist
\item
  Default
\item
  Lexicon
\item
  Material
\end{itemize}

\begin{figure}
\centering
\includegraphics{./images/screens-android-ddm-form-screenlet-lexicon-view.png}
\caption{The DDM Form Screenlet with the Lexicon View Set.}
\end{figure}

\subsubsection{Custom Layouts}\label{custom-layouts}

To create custom layouts for a field, create the new layout following
the naming pattern
\texttt{\textless{}field\_editor\_id\textgreater{}\_\textless{}view\_name\textgreater{}}.
The Screenlet automatically loads such layouts.

For example, this table lists the filename you should use when creating
custom layouts for each field type, for the Lexicon View. Note that
because some DDM fields inherit from DDL, they are referenced as DDL.

Editor Type

Field Editor ID

Example Using Lexicon View

Checkbox

ddlfield\_checkbox

ddlfield\_checkbox\_lexicon.xml

Checkbox Multiple

ddmfield\_checkbox

ddmfield\_checkbox\_multiple.xml

Date

ddlfield\_date

ddlfield\_date\_lexicon.xml

Number

ddlfield\_number

ddlfield\_number\_lexicon.xml

Integer

ddlfield\_number

ddlfield\_number\_lexicon.xml

Decimal

ddlfield\_number

ddlfield\_number\_lexicon.xml

Radio

ddlfield\_radio

ddlfield\_radio\_lexicon.xml

Text

ddlfield\_text

ddlfield\_text\_lexicon.xml

Select

ddlfield\_select

ddlfield\_select\_lexicon.xml

Text Area

ddlfield\_text\_area

ddlfield\_text\_area\_lexicon.xml

Paragraph

ddmfield\_paragraph

ddmfield\_paragraph\_lexicon.xml

Document

ddlfield\_document

ddlfield\_document\_lexicon.xml

Grid

ddmfield\_grid

ddmfield\_grid\_lexicon.xml

Repeatable

ddmfield\_repeatable

ddmfield\_repeatable\_lexicon.xml

\subsection{Application Configuration}\label{application-configuration}

DDM Form Screenlet needs the following user permissions:

\begin{verbatim}
<uses-permission android:name="android.permission.CAMERA"/>
<uses-permission android:name="android.permission.WRITE_EXTERNAL_STORAGE"/>
\end{verbatim}

The Documents and Media fields use both to take a picture/video and
store it locally before uploading it to the portal.

\subsection{Portal Configuration}\label{portal-configuration-8}

Before using DDM Form Screenlet, ensure that the following exist in the
portal:

\begin{itemize}
\item
  A form for the Screenlet to display. For instructions on this, see the
  article
  \href{/docs/7-1/user/-/knowledge_base/u/creating-and-managing-forms}{Creating
  and Managing Forms}.
\item
  If your form uses it, workflow must be configured. See the
  \href{/docs/7-1/user/-/knowledge_base/u/workflow}{Workflow} section of
  the user guide for instructions on configuring and using workflow.
\end{itemize}

\subsection{Required Attributes}\label{required-attributes-20}

\begin{itemize}
\tightlist
\item
  \texttt{formInstanceId}
\end{itemize}

\subsection{Attributes}\label{attributes-21}

Attribute

Data Type

Explanation

formInstanceId

number

The ID of the form to display in the Screenlet. To find the IDs for your
data definitions in the portal, select the site to work in and click
Content → Forms. The table that lists the site's forms also lists each
form's ID.

layoutId

@layout

The layout to use to show the View.

autoloadDraftEnabled

boolean

Sets whether the form loads the last draft for the current user when the
Screenlet is shown. The default value is \texttt{true}.

autosaveDraftEnabled

boolean

Sets whether the form should autosave a draft for the current user. The
default value is \texttt{true}.

syncFormTimeout

number

Time in milliseconds to start synchronize the form (save and evaluate
form rules). The default value is 500.

\begin{figure}
\centering
\includegraphics{./images/screens-portal-ddm-form-id.png}
\caption{The red box in this image highlights a form's ID.}
\end{figure}

\subsection{Permissions}\label{permissions-1}

If your form includes at least one Documents and Media field, you must
grant permissions in the target repository and folder. For more
information, see
\href{/docs/7-1/user/-/knowledge_base/u/adding-files-to-a-document-library\#granting-file-permissions-and-roles}{Granting
File Permissions and Roles}, and
\href{/docs/7-1/user/-/knowledge_base/u/creating-folders\#setting-folder-permissions}{Setting
Folder Permissions}. To set permissions for Documents and Media's Home
folder, navigate to Documents and Media and select \emph{Options}
(\includegraphics{./images/icon-options.png}) → \emph{Home Folder
Permissions}.

\begin{figure}
\centering
\includegraphics{./images/screens-portal-permission-folder-add.png}
\caption{Select which roles can add a document to a Documents and Media
folder.}
\end{figure}

\subsection{Methods}\label{methods-12}

Method

Return Type

Explanation

load()

void

Starts the request to load the form. The form fields are shown when the
response is received.

setDDMFormListener()

void

Sets the listener for this form.

\subsection{Listener}\label{listener-21}

DDM Form Screenlet delegates some events to an object that implements to
the \texttt{DDMFormListener} interface. This interface lets you
implement the following methods:

\texttt{onFormLoaded(FormInstance\ formInstance)}: Called when the form
instance successfully loads.

\texttt{onError(Exception\ e)}: Called when an error occurs in the
process. For example, this method is called when an error occurs while
loading a form instance.

\texttt{onDraftLoaded(FormInstanceRecord\ formInstanceRecord)}: Called
when a draft is retored.

\texttt{onDraftSaved(FormInstanceRecord\ formInstanceRecord)}: Called
when a draft is saved.

\texttt{onFormSubmitted(FormInstanceRecord\ formInstanceRecord)}: Called
when a form is successfully submitted.

\chapter{Screenlets in Liferay Screens for
iOS}\label{screenlets-in-liferay-screens-for-ios}

Liferay Screens for iOS contains several Screenlets that you can use in
your iOS apps. This section contains the reference documentation for
each. If you're looking for instructions on using Screens, see the
\href{/docs/7-1/tutorials/-/knowledge_base/t/ios-apps-with-liferay-screens}{Screens
tutorials}. The Screens tutorials contain instructions on
\href{/docs/7-1/tutorials/-/knowledge_base/t/using-screenlets-in-ios-apps}{using
Screenlets} and
\href{/docs/7-1/tutorials/-/knowledge_base/t/using-themes-in-ios-screenlets}{using
Themes in Screenlets}. Each Screenlet reference document here lists the
Screenlet's features, compatibility, its module (if any), available
Themes, attributes, delegate methods, and more. The available Screenlets
are listed here with links to their reference documents:

\begin{itemize}
\item
  \href{/docs/7-1/reference/-/knowledge_base/r/loginscreenlet-for-ios}{\textbf{Login
  Screenlet:}} Signs users in to a Liferay DXP instance.
\item
  \href{/docs/7-1/reference/-/knowledge_base/r/signupscreenlet-for-ios}{\textbf{Sign
  Up Screenlet:}} Registers new users in a Liferay DXP instance.
\item
  \href{/docs/7-1/reference/-/knowledge_base/r/forgotpasswordscreenlet-for-ios}{\textbf{Forgot
  Password Screenlet:}} Sends emails containing a new password or
  password reset link to users.
\item
  \href{/docs/7-1/reference/-/knowledge_base/r/userportraitscreenlet-for-ios}{\textbf{User
  Portrait Screenlet:}} Shows the user's portrait picture.
\item
  \href{/docs/7-1/reference/-/knowledge_base/r/ddlformscreenlet-for-ios}{\textbf{DDL
  Form Screenlet:}} Presents dynamic forms to be filled out by users and
  submitted back to the server.
\item
  \href{/docs/7-1/reference/-/knowledge_base/r/ddllistscreenlet-for-ios}{\textbf{DDL
  List Screenlet:}} Shows a list of records based on a pre-existing DDL
  in a Liferay DXP instance.
\item
  \href{/docs/7-1/reference/-/knowledge_base/r/assetlistscreenlet-for-ios}{\textbf{Asset
  List Screenlet:}} Shows a list of assets managed by the
  \href{/docs/7-1/tutorials/-/knowledge_base/t/asset-framework}{Asset
  Framework}. This includes web content, blog entries, documents, and
  more.
\item
  \href{/docs/7-1/reference/-/knowledge_base/r/webcontentdisplayscreenlet-for-ios}{\textbf{Web
  Content Display Screenlet:}} Shows the web content's HTML or
  structured content. This Screenlet uses the features available in
  \href{/docs/7-1/user/-/knowledge_base/u/introduction-web-content}{Web
  Content Management}.
\item
  \href{/docs/7-1/reference/-/knowledge_base/r/web-content-list-screenlet-for-ios}{\textbf{Web
  Content List Screenlet:}} Shows a list of web contents from a folder,
  usually based on a pre-existing \texttt{DDMStructure}.
\item
  \href{/docs/7-1/reference/-/knowledge_base/r/image-gallery-screenlet-for-ios}{\textbf{Image
  Gallery Screenlet:}} Shows a list of images from a folder. This
  Screenlet also lets users upload and delete images.
\item
  \href{/docs/7-1/reference/-/knowledge_base/r/rating-screenlet-for-ios}{\textbf{Rating
  Screenlet:}} Shows the rating for an asset. This Screenlet also lets
  the user update or delete the rating.
\item
  \href{/docs/7-1/reference/-/knowledge_base/r/comment-list-screenlet-for-ios}{\textbf{Comment
  List Screenlet:}} Shows a list of comments for an asset.
\item
  \href{/docs/7-1/reference/-/knowledge_base/r/comment-display-screenlet-for-ios}{\textbf{Comment
  Display Screenlet:}} Shows a single comment for an asset.
\item
  \href{/docs/7-1/reference/-/knowledge_base/r/comment-add-screenlet-for-ios}{\textbf{Comment
  Add Screenlet:}} Lets the user comment on an asset.
\item
  \href{/docs/7-1/reference/-/knowledge_base/r/asset-display-screenlet-for-ios}{\textbf{Asset
  Display Screenlet:}} Displays an asset. Currently, this Screenlet can
  display Documents and Media Library files (\texttt{DLFileEntry}
  entities), blog articles (\texttt{BlogsEntry} entities), and web
  content articles (\texttt{WebContent} entities). You can also use it
  to display custom assets.
\item
  \href{/docs/7-1/reference/-/knowledge_base/r/blogs-entry-display-screenlet-for-ios}{\textbf{Blogs
  Entry Display Screenlet:}} Shows a single blog entry.
\item
  \href{/docs/7-1/reference/-/knowledge_base/r/image-display-screenlet-for-ios}{\textbf{Image
  Display Screenlet:}} Shows a single image file from the
  \href{/docs/7-1/user/-/knowledge_base/u/managing-documents-and-media}{Documents
  and Media Library}.
\item
  \href{/docs/7-1/reference/-/knowledge_base/r/video-display-screenlet-for-ios}{\textbf{Video
  Display Screenlet:}} Shows a single video file from the
  \href{/docs/7-1/user/-/knowledge_base/u/managing-documents-and-media}{Documents
  and Media Library}.
\item
  \href{/docs/7-1/reference/-/knowledge_base/r/audio-display-screenlet-for-ios}{\textbf{Audio
  Display Screenlet:}} Shows a single audio file from the
  \href{/docs/7-1/user/-/knowledge_base/u/managing-documents-and-media}{Documents
  and Media Library}.
\item
  \href{/docs/7-1/reference/-/knowledge_base/r/pdf-display-screenlet-for-ios}{\textbf{PDF
  Display Screenlet:}} Shows a single PDF file from the
  \href{/docs/7-1/user/-/knowledge_base/u/managing-documents-and-media}{Documents
  and Media Library}.
\item
  \href{/docs/7-1/reference/-/knowledge_base/r/file-display-screenlet-for-ios}{\textbf{File
  Display Screenlet:}} Shows a single file from the
  \href{/docs/7-1/user/-/knowledge_base/u/managing-documents-and-media}{Documents
  and Media Library}. Use this Screenlet to display file types not
  covered by the other display Screenlets (e.g., DOC, PPT, XLS).
\item
  \href{/docs/7-1/reference/-/knowledge_base/r/web-screenlet-for-ios}{\textbf{Web
  Screenlet:}} Displays any web page. You can also customize the web
  page through injection of local and remote JavaScript and CSS files.
\end{itemize}

\section{Login Screenlet for iOS}\label{login-screenlet-for-ios}

\subsection{Requirements}\label{requirements-22}

\begin{itemize}
\tightlist
\item
  Xcode 9.3 or above
\item
  iOS 11 SDK
\item
  Liferay Portal 6.2 CE/EE, Liferay CE Portal 7.0/7.1, Liferay DXP
\end{itemize}

\subsection{Compatibility}\label{compatibility-22}

\begin{itemize}
\tightlist
\item
  iOS 9 and above
\end{itemize}

\subsection{Xamarin Requirements}\label{xamarin-requirements-22}

\begin{itemize}
\tightlist
\item
  Visual Studio 7.2
\item
  Mono .NET framework 5.4.1.6
\end{itemize}

\subsection{Features}\label{features-22}

The Login Screenlet authenticates portal users in your iOS app. The
following authentication methods are supported:

\begin{itemize}
\item
  \textbf{Basic:} uses user login and password according to
  \href{http://tools.ietf.org/html/rfc2617}{HTTP Basic Access
  Authentication specification}. Depending on the authentication method
  used by your Liferay instance, you need to provide the user's email
  address, screen name, or user ID. You also need to provide the user's
  password.
\item
  \textbf{OAuth:} implements \href{https://oauth.net/2/}{OAuth 2}.
\item
  \textbf{Cookie:} uses a cookie to log in. This lets you access
  documents and images in the portal's document library without the
  guest view permission in the portal. The other authentication types
  require this permission to access such files.
\end{itemize}

For instructions on configuring the Screenlet to use these
authentication types, see the below
\hyperref[portal-configuration]{Portal Configuration} and
\hyperref[attributes]{Screenlet Attributes} sections.

When a user successfully authenticates, their attributes are retrieved
for use in the app. You can use the \texttt{SessionContext} class to get
the current user's attributes.

Note that user credentials and attributes can be stored securely in the
keychain (see the \texttt{saveCredentials} attribute). Stored user
credentials can be used to automatically log the user in to subsequent
sessions. To do this, you can use the method
\texttt{SessionContext.loadStoredCredentials()} method.

\subsection{JSON Services Used}\label{json-services-used-20}

Screenlets in Liferay Screens call the portal's JSON web services. This
Screenlet calls the following services and methods.

\noindent\hrulefill

\begin{longtable}[]{@{}lll@{}}
\toprule\noalign{}
Service & Method & Notes \\
\midrule\noalign{}
\endhead
\bottomrule\noalign{}
\endlastfoot
\texttt{UserService} & \texttt{getUserByEmailAddress} & Basic login \\
\texttt{UserService} & \texttt{getUserByScreenName} & Basic login \\
\texttt{UserService} & \texttt{getUserById} & Basic login \\
\texttt{UserService} & \texttt{getCurrentUser} & Cookie and OAuth
login \\
\end{longtable}

\noindent\hrulefill

\subsection{Module}\label{module-22}

\begin{itemize}
\tightlist
\item
  Auth
\end{itemize}

\subsection{Themes}\label{themes}

\begin{itemize}
\tightlist
\item
  Default (\texttt{default})
\item
  Flat7 (\texttt{flat7})
\end{itemize}

For instructions on using Themes, see the tutorial
\href{/docs/7-1/tutorials/-/knowledge_base/t/using-themes-in-ios-screenlets}{Using
Themes in iOS Screenlets}.

\begin{figure}
\centering
\includegraphics{./images/screens-ios-login.png}
\caption{The Login Screenlet using the Default and Flat7 Themes.}
\end{figure}

\subsection{Portal Configuration}\label{portal-configuration-9}

\subsubsection{Basic Authentication}\label{basic-authentication-1}

Before using Login Screenlet, you should make sure your portal is
configured with the authentication option you want to use. You can
choose email address, screen name, or user ID. You can set this in the
Control Panel by selecting \emph{Configuration} → \emph{Instance
Settings}, and then selecting the \emph{Authentication} section. The
authentication options are in the \emph{How do users authenticate?}
selector menu. For more information, see the User Guide's
\href{/docs/7-1/user/-/knowledge_base/u/authentication}{authentication
section}.

\begin{figure}
\centering
\includegraphics{./images/screens-portal-auth.png}
\caption{Setting the authentication method in your Liferay instance.}
\end{figure}

\subsubsection{OAuth}\label{oauth}

For instructions on using OAuth with Login Screenlet, see the tutorial
on
\href{/docs/7-1/tutorials/-/knowledge_base/t/using-oauth-2-in-liferay-screens-for-ios}{using
OAuth 2 with Liferay Screens}.

\subsection{Offline}\label{offline-20}

This Screenlet doesn't support offline mode. It requires network
connectivity. If you need to log in users automatically, even when
there's no network connection, you can use the \texttt{saveCredentials}
attribute together with the
\texttt{SessionContext.loadStoredCredentials()} method.

\subsection{Attributes}\label{attributes-22}

\noindent\hrulefill

Attribute \textbar{} Data type \textbar{} Explanation \textbar{}
\texttt{companyId} \textbar{} \texttt{number} \textbar{} The ID of the
portal instance to authenticate to. If you don't set this attribute or
set it to \texttt{0}, the Screenlet uses the \texttt{companyId} setting
in \texttt{LiferayServerContext}. \textbar{} \texttt{loginMode}
\textbar{} \texttt{string} \textbar{} The Screenlet's authentication
type. You can set this attribute to \texttt{basic}, \texttt{cookie},
\texttt{oauth2Redirect}, or \texttt{oauth2UsernameAndPassword}. If you
don't set this attribute, the Screenlet defaults to basic
authentication. \textbar{} \texttt{basicAuthMethod} \textbar{}
\texttt{string} \textbar{} Specifies the basic authentication option to
use. You can set this attribute to \texttt{email}, \texttt{screenName}
or \texttt{userId}. This must match the server's authentication option.
If you don't set this attribute, and don't set the \texttt{loginMode}
attribute to one of the OAuth values or \texttt{cookie}, the Screenlet
defaults to basic authentication with the \texttt{email} option.
\textbar{} \texttt{oauth2clientId} \textbar{} \texttt{string} \textbar{}
The ID of the OAuth 2 application in the portal. You can find this value
in the portal's OAuth 2 Admin portlet. \textbar{}
\texttt{oauth2redirectUrl} \textbar{} \texttt{string} \textbar{} The URL
that the mobile browser will redirect the user to after successful
login. You must configure this in the portal's OAuth 2 Admin portlet,
and associate the URL with the iOS app. \textbar{}
\texttt{oauth2clientSecret} \textbar{} \texttt{string} \textbar{} The
client secret of the OAuth 2 application in the portal. You can find
this value in the portal's OAuth 2 Admin portlet. \textbar{}
\texttt{oauth2Scopes} \textbar{} \texttt{string} \textbar{} The portal
permissions to request. You can define a set of permissions associated
with an OAuth 2 application in the portal's OAuth 2 Admin portlet. Use
this attribute to request a subset of those permissions. Separate
multiple scopes with a space (e.g., \texttt{"scope1\ scope2\ scope3"}).
\textbar{} \texttt{saveCredentials} \textbar{} \texttt{boolean}
\textbar{} When set, the user credentials and attributes are stored
securely in the keychain. This information can then be loaded in
subsequent sessions by calling the
\texttt{SessionContext.loadStoredCredentials()} method. \textbar{}
\texttt{shouldHandleCookieExpiration} \textbar{} \texttt{bool}
\textbar~Whether to refresh the cookie automatically when using cookie
login. When set to \texttt{true} (the default value), the cookie
refreshes as it's about to expire. \textbar{}
\texttt{cookieExpirationTime} \textbar{} \texttt{int} \textbar~How long
the cookie lasts, in seconds. This value depends on your portal
instance's configuration. The default value is \texttt{900}. \textbar{}

\noindent\hrulefill

\subsection{Delegate}\label{delegate}

The Login Screenlet delegates some events to an object that conforms to
the \texttt{LoginScreenletDelegate} protocol. This protocol lets you
implement the following methods:

\begin{itemize}
\item
  \texttt{-\ screenlet:onLoginResponseUserAttributes:}: Called when
  login successfully completes. The user attributes are passed as a
  dictionary of keys (\texttt{String} or \texttt{NSStrings}) and values
  (\texttt{AnyObject} or \texttt{NSObject}). The supported keys are the
  same as the
  \href{https://github.com/liferay/liferay-portal/blob/master/portal-impl/src/com/liferay/portal/service.xml\#L2575-L2737}{portal's
  User entity}.
\item
  \texttt{-\ screenlet:onLoginError:}: Called when an error occurs
  during login. The \texttt{NSError} object describes the error.
\item
  \texttt{-\ screenlet:onCredentialsSavedUserAttributes:}: Called when
  the user credentials are stored after a successful login.
\item
  \texttt{-\ screenlet:onCredentialsLoadedUserAttributes:}: Called when
  the user credentials are retrieved. Note that this only occurs when
  the Screenlet is used and stored credentials are available.
\end{itemize}

\subsection{Challenge-Response
Authentication}\label{challenge-response-authentication-1}

To support
\href{https://en.wikipedia.org/wiki/Challenge\%E2\%80\%93response_authentication}{challenge-response
authentication} when using a cookie to log in to the portal, the
\texttt{SessionContext} class has a \texttt{challengeResolver}
attribute. For more information about how iOS handles challenge-response
authentication, see the article
\href{https://developer.apple.com/library/content/documentation/Cocoa/Conceptual/URLLoadingSystem/Articles/AuthenticationChallenges.html}{Authentication
Challenges and TLS Chain Validation}.

The challenge resolver type is a closure or block that receives two
parameters:

\begin{enumerate}
\def\labelenumi{\arabic{enumi}.}
\tightlist
\item
  \texttt{URLAuthenticationChallenge}
\item
  Another closure or block. You must call this to resolve the challenge
  (e.g., by passing credentials, canceling the challenge, etc.). You can
  do this by passing a \texttt{URLSession.AuthChallengeDisposition}.
\end{enumerate}

Here's an example that sends a basic authorization in response to an
authentication challenge:

\begin{verbatim}
SessionContext.challengeResolver = resolver

func resolver(challenge: URLAuthenticationChallenge,
    decisionCallback: (URLSession.AuthChallengeDisposition, URLCredential) -> Void) {

    // Use the challenge variable to get information about the challenge itself
    if challenge.previousFailureCount == 0 {
        // To solve the challenge, call the decision callback with your decision
        // Pass the credentials to the server
        decisionCallback(.useCredential, URLCredential(user: "user", password: "password", 
            persistence: .forSession))
    }
    else {
        // Something went wrong, so let the system handle the challenge
        decisionCallback(.performDefaultHandling, URLCredential(user: "these credentials", 
            password: "are ignored", persistence: .none))
    }

}
\end{verbatim}

\section{Sign Up Screenlet for iOS}\label{sign-up-screenlet-for-ios}

\subsection{Requirements}\label{requirements-23}

\begin{itemize}
\tightlist
\item
  Xcode 9.3 or above
\item
  iOS 11 SDK
\item
  Liferay Portal 6.2 CE/EE, Liferay CE Portal 7.0/7.1, Liferay DXP
\end{itemize}

\subsection{Compatibility}\label{compatibility-23}

\begin{itemize}
\tightlist
\item
  iOS 9 and above
\end{itemize}

\subsection{Xamarin Requirements}\label{xamarin-requirements-23}

\begin{itemize}
\tightlist
\item
  Visual Studio 7.2
\item
  Mono .NET framework 5.4.1.6
\end{itemize}

\subsection{Features}\label{features-23}

The Sign Up Screenlet creates a new user in your Liferay instance: a new
user of your app can become a new user in your portal. You can also use
this Screenlet to save the credentials of the new user in their
keychain. This enables auto login for future sessions. The Screenlet
also supports navigation of form fields from the keyboard of the user's
device.

\subsection{JSON Services Used}\label{json-services-used-21}

Screenlets in Liferay Screens call JSON web services in the portal. This
Screenlet calls the following services and methods.

\noindent\hrulefill

\begin{longtable}[]{@{}lll@{}}
\toprule\noalign{}
Service & Method & Notes \\
\midrule\noalign{}
\endhead
\bottomrule\noalign{}
\endlastfoot
\texttt{UserService} & \texttt{addUser} & \\
\end{longtable}

\noindent\hrulefill

\subsection{Module}\label{module-23}

\begin{itemize}
\tightlist
\item
  Auth
\end{itemize}

\subsection{Themes}\label{themes-1}

\begin{itemize}
\tightlist
\item
  Default (\texttt{default})
\item
  Flat7 (\texttt{flat7})
\end{itemize}

\begin{figure}
\centering
\includegraphics{./images/screens-ios-signup.png}
\caption{The Sign Up Screenlet with the Default and Flat7 Themes.}
\end{figure}

\subsection{Portal Configuration}\label{portal-configuration-10}

Sign Up Screenlet's corresponding configuration in the Liferay instance
can be set in the Control Panel by selecting \emph{Configuration} →
\emph{Instance Settings}, and then selecting the \emph{Authentication}
section.

\begin{figure}
\centering
\includegraphics{./images/screens-portal-signup.png}
\caption{The Liferay instance's authentication settings.}
\end{figure}

For more details, see the
\href{/docs/7-1/user/-/knowledge_base/u/authentication}{Authentication}
section of the User Guide.

\subsection{Anonymous Request}\label{anonymous-request-1}

Anonymous requests are unauthenticated requests. Authentication is
needed, however, to call the API. To allow this operation, the portal
administrator should create a specific user with minimal permissions.

\subsection{Offline}\label{offline-21}

This Screenlet doesn't support offline mode. It requires network
connectivity.

\subsection{Attributes}\label{attributes-23}

\noindent\hrulefill

Attribute \textbar{} Data type \textbar{} Explanation \textbar{}
\texttt{anonymousApiUserName} \textbar{} \texttt{string} \textbar{} The
user name, email address, or user ID (depending on the portal's
authentication method) to use for authenticating the request. \textbar{}
\texttt{anoymousApiPassword} \textbar{} \texttt{string} \textbar{} The
password for use in authenticating the request. \textbar{}
\texttt{companyId} \textbar{} \texttt{number} \textbar{} When set,
authentication is done for a user in the specified company. If the value
is \texttt{0}, the company specified in \texttt{LiferayServerContext} is
used. \textbar{} \texttt{autoLogin} \textbar{} \texttt{boolean}
\textbar{} Whether the user is logged in automatically after a
successful sign up. \textbar{} \texttt{saveCredentials} \textbar{}
\texttt{boolean} \textbar{} Sets whether or not the user's credentials
and attributes are stored in the keychain after a successful log in.
This attribute is ignored if \texttt{autologin} is disabled. \textbar{}

\noindent\hrulefill

\subsection{Delegate}\label{delegate-1}

The Sign Up Screenlet delegates some events to an object that conforms
to the \texttt{SignUpScreenletDelegate} protocol. If the
\texttt{autologin} attribute is enabled, login events are delegated to
an object conforming to the \texttt{LoginScreenletDelegate} protocol.
Refer to the \href{LoginScreenlet.md}{\texttt{LoginScreenlet}
documentation} for more details.

The \texttt{SignUpScreenletDelegate} protocol lets you implement the
following methods:

\begin{itemize}
\item
  \texttt{-\ screenlet:onSignUpResponseUserAttributes:}: Called when
  sign up successfully completes. The user attributes are passed as a
  dictionary of keys (\texttt{String} or \texttt{NSStrings}) and values
  (\texttt{AnyObject} or \texttt{NSObject}). The supported keys are the
  same as the attributes in the
  \href{https://docs.liferay.com/dxp/portal/7.1-latest/javadocs/portal-kernel/com/liferay/portal/kernel/model/User.html}{portal's
  \texttt{User} entity}.
\item
  \texttt{-\ screenlet:onSignUpError:}: Called when an error occurs in
  the process. The \texttt{NSError} object describes the error.
\end{itemize}

\section{Forgot Password Screenlet for
iOS}\label{forgot-password-screenlet-for-ios}

\subsection{Requirements}\label{requirements-24}

\begin{itemize}
\tightlist
\item
  Xcode 9.3 or above
\item
  iOS 11 SDK
\item
  Liferay Portal 6.2 CE/EE, Liferay CE Portal 7.0/7.1, Liferay DXP
\item
  Liferay Screens Compatibility app
  (\href{http://www.liferay.com/marketplace/-/mp/application/54365664}{CE}
  or
  \href{http://www.liferay.com/marketplace/-/mp/application/54369726}{EE/DXP}).
  This app is preinstalled in Liferay CE Portal 7.0/7.1 and Liferay DXP.
\end{itemize}

\subsection{Compatibility}\label{compatibility-24}

\begin{itemize}
\tightlist
\item
  iOS 9 and above
\end{itemize}

\subsection{Xamarin Requirements}\label{xamarin-requirements-24}

\begin{itemize}
\tightlist
\item
  Visual Studio 7.2
\item
  Mono .NET framework 5.4.1.6
\end{itemize}

\subsection{Features}\label{features-24}

The Forgot Password Screenlet sends emails to registered users with
their new passwords or password reset links, depending on the server
configuration. The available authentication methods are:

\begin{itemize}
\tightlist
\item
  Email address
\item
  Screen name
\item
  User id
\end{itemize}

\subsection{JSON Services Used}\label{json-services-used-22}

Screenlets in Liferay Screens call JSON web services in the portal. This
Screenlet calls the following services and methods.

\noindent\hrulefill

\begin{longtable}[]{@{}lll@{}}
\toprule\noalign{}
Service & Method & Notes \\
\midrule\noalign{}
\endhead
\bottomrule\noalign{}
\endlastfoot
\texttt{UserService} & \texttt{sendPasswordByEmailAddress} & \\
\texttt{UserService} & \texttt{sendPasswordByUserId} & \\
\texttt{UserService} & \texttt{sendPasswordByScreenName} & \\
\end{longtable}

\noindent\hrulefill

\subsection{Module}\label{module-24}

\begin{itemize}
\tightlist
\item
  Auth
\end{itemize}

\subsection{Themes}\label{themes-2}

\begin{itemize}
\tightlist
\item
  Default (\texttt{default})
\item
  Flat7 (\texttt{flat7})
\end{itemize}

\begin{figure}
\centering
\includegraphics{./images/screens-ios-forgotpwd.png}
\caption{The Forgot Password Screenlet with the Default and Flat7
Themes.}
\end{figure}

\subsection{Portal Configuration}\label{portal-configuration-11}

To use the Forgot Password Screenlet, you must allow users to request
new passwords in the portal. The next sections show you how to do this.

\subsubsection{Authentication Method}\label{authentication-method-1}

Note that the authentication method configured in the portal can be
different from the one used by this Screenlet. For example, it's
\emph{perfectly fine} to use \texttt{screenName} for sign in
authentication, but allow users to recover their password using the
\texttt{email} authentication method.

\subsubsection{Password Reset}\label{password-reset-1}

You can set the Liferay instance's corresponding password reset options
in the Control Panel by selecting \emph{Configuration} → \emph{Instance
Settings}, and then selecting the \emph{Authentication} section. The
Screenlet's password functionality depends on the authentication
settings in the portal:

\begin{figure}
\centering
\includegraphics{./images/screens-password-reset.png}
\caption{Checkboxes for the password recovery features in Liferay
Portal.}
\end{figure}

If both of these options are unchecked, password recovery is disabled.
If both options are checked, an email containing a password reset link
is sent when a user requests it. If only the first option is checked, an
email containing a new password is sent when a user requests it.

For more details, see the
\href{/docs/7-1/user/-/knowledge_base/u/authentication}{Authentication}
section of the User Guide.

\subsubsection{Anonymous Request}\label{anonymous-request-2}

An anonymous request can be made without the user being logged in.
However, authentication is needed to call the API. To allow this
operation, the portal administrator should create a specific user with
minimal permissions.

\subsection{Offline}\label{offline-22}

This Screenlet doesn't support offline mode. It requires network
connectivity.

\subsection{Attributes}\label{attributes-24}

\noindent\hrulefill

Attribute \textbar{} Data type \textbar{} Explanation \textbar{}
\texttt{anonymousApiUserName} \textbar{} \texttt{string} \textbar{} The
user name, email address, or userId (depending on the portal's
authentication method) to use for authenticating the request. \textbar{}
\texttt{anonymousApiPassword} \textbar{} \texttt{string} \textbar{} The
password to use to authenticate the request. \textbar{}
\texttt{companyId} \textbar{} \texttt{number} \textbar{} When set, the
authentication is done for a user within the specified company. If the
value is \texttt{0}, the company specified in
\texttt{LiferayServerContext} is used. \textbar{}
\texttt{basicAuthMethod} \textbar{} \texttt{string} \textbar{} The
authentication method that is presented to the user. This can be
\texttt{email}, \texttt{screenName}, or \texttt{userId}. \textbar{}

\noindent\hrulefill

\subsection{Delegate}\label{delegate-2}

The Forgot Password Screenlet delegates some events to an object that
conforms to the \texttt{ForgotPasswordScreenletDelegate} protocol. This
protocol lets you implement the following methods:

\begin{itemize}
\item
  \texttt{-\ screenlet:onForgotPasswordSent:}: Called when a password
  reset email is successfully sent. The Boolean parameter indicates
  whether the email contains the new password or a password reset link.
\item
  \texttt{-\ screenlet:onForgotPasswordError:}: Called when an error
  occurs in the process. The \texttt{NSError} object describes the
  error.
\end{itemize}

\section{User Portrait Screenlet for
iOS}\label{user-portrait-screenlet-for-ios}

\subsection{Requirements}\label{requirements-25}

\begin{itemize}
\tightlist
\item
  Xcode 9.3 or above
\item
  iOS 11 SDK
\item
  Liferay Portal 6.2 CE/EE, Liferay CE Portal 7.0/7.1, Liferay DXP
\end{itemize}

\subsection{Compatibility}\label{compatibility-25}

\begin{itemize}
\tightlist
\item
  iOS 9 and above
\end{itemize}

\subsection{Xamarin Requirements}\label{xamarin-requirements-25}

\begin{itemize}
\tightlist
\item
  Visual Studio 7.2
\item
  Mono .NET framework 5.4.1.6
\end{itemize}

\subsection{Features}\label{features-25}

The User Portrait Screenlet shows the user's portrait from Liferay
Portal. If the user doesn't have a portrait configured, a placeholder
image is shown.

\subsection{JSON Services Used}\label{json-services-used-23}

Screenlets in Liferay Screens call JSON web services in the portal. This
Screenlet calls the following services and methods.

\noindent\hrulefill

\begin{longtable}[]{@{}lll@{}}
\toprule\noalign{}
Service & Method & Notes \\
\midrule\noalign{}
\endhead
\bottomrule\noalign{}
\endlastfoot
\texttt{UserService} & \texttt{getUserById} & \\
\texttt{UserService} & \texttt{getUserByEmailAddress} & \\
\texttt{UserService} & \texttt{getUserByScreenName} & \\
\end{longtable}

\noindent\hrulefill

\subsection{Module}\label{module-25}

\begin{itemize}
\tightlist
\item
  None
\end{itemize}

\subsection{Themes}\label{themes-3}

\begin{itemize}
\tightlist
\item
  Default (\texttt{default})
\item
  Flat7 (\texttt{flat7})
\end{itemize}

\begin{figure}
\centering
\includegraphics{./images/screens-ios-portrait.png}
\caption{The User Portrait Screenlet using the Default and Flat7
Themes.}
\end{figure}

\subsection{Portal Configuration}\label{portal-configuration-12}

None

\subsection{Offline}\label{offline-23}

This Screenlet supports offline mode so it can function without a
network connection. For more information on how offline mode works, see
the
\href{/docs/7-1/tutorials/-/knowledge_base/t/architecture-of-offline-mode-in-liferay-screens}{tutorial
on its architecture}.

When loading the portrait, the Screenlet supports the following offline
mode policies:

\noindent\hrulefill

Policy \textbar{} What happens \textbar{} When to use \textbar{}
\texttt{remote-only} \textbar{} The Screenlet loads the user portrait
from the portal. If a connection issue occurs, the Screenlet uses the
delegate to notify the developer about the error. If the Screenlet loads
the portrait, it stores the received image in the local cache for later
use. \textbar{} Use this policy when you always need to show updated
portraits, and show the default placeholder when there's no connection.
\textbar{} \texttt{cache-only} \textbar{} The Screenlet loads the user
portrait from the local cache. If the portrait isn't there, the
Screenlet uses the delegate to notify the developer about the error.
\textbar{} Use this policy to show local portraits, without retrieving
remote information under any circumstance. \textbar{}
\texttt{remote-first} \textbar{} The Screenlet loads the user portrait
from the portal. The Screenlet displays the portrait to the user and
stores it in the local cache for later use. If a connection issue
occurs, the Screenlet retrieves the portrait from the local cache. If
the portrait doesn't exist there, the Screenlet uses the delegate to
notify the developer about the error. \textbar{} Use this policy to show
the most recent portrait when connected, but show a potentially outdated
version when there's no connection. \textbar{} \texttt{cache-first}
\textbar{} If the portrait exists in the local cache, the Screenlet
loads it from there. If it doesn't exist there, the Screenlet requests
the portrait from the portal and uses the delegate to notify the
developer about any connection errors. \textbar{} Use this policy to
save bandwidth and loading time in the event a local (but probably
outdated) portrait exists. \textbar{}

\noindent\hrulefill

When editing the portrait, the Screenlet supports the following offline
mode policies:

\noindent\hrulefill

Policy \textbar{} What happens \textbar{} When to use \textbar{}
\texttt{remote-only} \textbar{} The Screenlet sends the user portrait to
the portal. If a connection issue occurs, the Screenlet uses the
delegate to notify the developer about the error, but it also discards
the new portrait. \textbar{} Use this policy when you need to make sure
portal always has the most recent version of the portrait. \textbar{}
\texttt{cache-only} \textbar{} The Screenlet stores the user portrait in
the local cache. \textbar{} Use this policy when you need to save the
portrait locally, but don't want to change the portrait in the portal.
\textbar{} \texttt{remote-first} \textbar{} The Screenlet sends the user
portrait to the portal. If this succeeds, the Screenlet also stores the
portrait in the local cache for later usage. If a connection issue
occurs, the Screenlet stores the portrait in the local cache with the
\emph{dirty flag} enabled. This causes the portrait to be sent to the
portal when the synchronization process runs. \textbar{} Use this policy
when you need to make sure the Screenlet sends the new portrait to the
portal as soon as the connection is restored. \textbar{}
\texttt{cache-first} \textbar{} The Screenlet stores the user portrait
in the local cache and then sends it to the portal. If a connection
issue occurs, the Screenlet stores the portrait in the local cache with
the \emph{dirty flag} enabled. This causes the portrait to be sent to
the portal when the synchronization process runs. \textbar{} Use this
policy when you need to make sure the Screenlet sends the new portrait
to the portal as soon as the connection is restored. Compared to
\texttt{remote-first}, this policy always stores the portrait in the
cache. The \texttt{remote-first} policy only stores the new image in the
event of a network error. \textbar{}

\noindent\hrulefill

\subsection{Attributes}\label{attributes-25}

\noindent\hrulefill

Attribute \textbar{} Data type \textbar{} Explanation \textbar{}
\texttt{borderWidth} \textbar{} \texttt{number} \textbar{} The size in
pixels for the portrait's border. The default value is 1. Set this to
\texttt{0} if you want to hide the border.\textbar{}
\texttt{borderColor} \textbar{} \texttt{UIColor} \textbar{} The border's
color. Use the system's transparent color to hide the border. \textbar{}
\texttt{editable} \textbar{} \texttt{boolean} \textbar{} Lets the user
change the portrait image by taking a photo or selecting a gallery
picture. The default value is \texttt{false}. Portraits loaded with the
\texttt{load(portraitId,\ uuid,\ male)} method aren't editable.
\textbar{} \texttt{offlinePolicy} \textbar{} \texttt{string} \textbar{}
Configure the loading and saving behavior in case of connectivity
issues. For more details, read the ``Offline'' section below. \textbar{}

\noindent\hrulefill

\subsection{Methods}\label{methods-13}

\noindent\hrulefill

Method \textbar{} Return \textbar{} Explanation \textbar{}
\texttt{loadLoggedUserPortrait()} \textbar{} \texttt{boolean} \textbar{}
Starts the request to load the currently logged in user's portrait image
(see the \texttt{SessionContext} class). \textbar{}
\texttt{load(userId)} \textbar{} \texttt{boolean} \textbar{} Starts the
request to load the specified user's portrait image. \textbar{}
\texttt{load(portraitId,\ uuid,\ male)} \textbar{} \texttt{boolean}
\textbar{} Starts the request to load the portrait image using the
specified user's data. The parameters \texttt{portraitId} and
\texttt{uuid} can be retrieved by using the
\texttt{SessionContext.userAttributes()} method. \textbar{}
\texttt{load(companyId,\ emailAddress)} \textbar{} \texttt{boolean}
\textbar{} Starts the request to load the portrait image using the
user's email address. \textbar{} \texttt{load(companyId,\ screenName)}
\textbar{} \texttt{boolean} \textbar{} Starts the request to load the
portrait image using the user's screen name. \textbar{}

\noindent\hrulefill

\subsection{Delegate}\label{delegate-3}

The User Portrait Screenlet delegates some events to an object that
conforms to the \texttt{UserPortraitScreenletDelegate} protocol. This
protocol lets you implement the following methods:

\begin{itemize}
\item
  \texttt{-\ screenlet:onUserPortraitResponseImage:}: Called when an
  image is received from the server. You can then apply image filters
  (grayscale, for example) and return the new image. You can return the
  original image supplied as the argument if you don't want to modify
  it.
\item
  \texttt{-\ screenlet:onUserPortraitError:}: Called when an error
  occurs in the process. The \texttt{NSError} object describes the
  error.
\item
  \texttt{-\ screenlet:onUserPortraitUploaded:}: Called when a new
  portrait is uploaded to the server. You receive the user attributes as
  a parameter.
\item
  \texttt{-\ screenlet:onUserPortraitUploadError:}: Called when an error
  occurs in the upload process. The \texttt{NSError} object describes
  the error.
\end{itemize}

\section{DDL Form Screenlet for iOS}\label{ddl-form-screenlet-for-ios}

\subsection{Requirements}\label{requirements-26}

\begin{itemize}
\tightlist
\item
  Xcode 9.3 or above
\item
  iOS 11 SDK
\item
  Liferay Portal 6.2 CE/EE, Liferay CE Portal 7.0/7.1, Liferay DXP
\item
  Liferay Screens Compatibility app
  (\href{http://www.liferay.com/marketplace/-/mp/application/54365664}{CE}
  or
  \href{http://www.liferay.com/marketplace/-/mp/application/54369726}{EE/DXP}).
  This app is preinstalled in Liferay CE Portal 7.0/7.1 and Liferay DXP.
\end{itemize}

\subsection{Compatibility}\label{compatibility-26}

\begin{itemize}
\tightlist
\item
  iOS 9 and above
\end{itemize}

\subsection{Xamarin Requirements}\label{xamarin-requirements-26}

\begin{itemize}
\tightlist
\item
  Visual Studio 7.2
\item
  Mono .NET framework 5.4.1.6
\end{itemize}

\subsection{Features}\label{features-26}

DDL Form Screenlet can be used to show a collection of fields so that a
user can fill in their values. Initial or existing values may be shown
in the fields. Fields of the following data types are supported:

\begin{itemize}
\tightlist
\item
  \emph{Boolean}: A two state value typically shown using a checkbox.
\item
  \emph{Date}: A formatted date value. The format depends on the
  device's locale.
\item
  \emph{Decimal, Integer, and Number}: A numeric value.
\item
  \emph{Document and Media}: A file stored on the current device. It can
  be uploaded to a specific portal repository.
\item
  \emph{Radio}: A set of options to choose from. A single option must be
  chosen.
\item
  \emph{Select}: A selection box of options to choose from. A single
  option must be chosen.
\item
  \emph{Text}: A single line of text.
\item
  \emph{Text Box}: Supports multiple lines of text.
\end{itemize}

DDL Form Screenlet also supports the following features:

\begin{itemize}
\tightlist
\item
  Stored records can support a specific workflow.
\item
  A Submit button can be shown at the end of the form.
\item
  Required values and validation for fields can be used.
\item
  Users can traverse the form fields from the keyboard.
\item
  Supports i18n in record values and labels.
\end{itemize}

There are also a few limitations you should be aware of when using DDL
Form Screenlet. They are listed here:

\begin{itemize}
\tightlist
\item
  Nested fields in the data definition aren't supported.
\item
  Selection of multiple items in the Radio and Select data types isn't
  supported yet.
\end{itemize}

\subsection{JSON Services Used}\label{json-services-used-24}

Screenlets in Liferay Screens call JSON web services in the portal. This
Screenlet calls the following services and methods.

\noindent\hrulefill

\begin{longtable}[]{@{}
  >{\raggedright\arraybackslash}p{(\columnwidth - 4\tabcolsep) * \real{0.3889}}
  >{\raggedright\arraybackslash}p{(\columnwidth - 4\tabcolsep) * \real{0.3333}}
  >{\raggedright\arraybackslash}p{(\columnwidth - 4\tabcolsep) * \real{0.2778}}@{}}
\toprule\noalign{}
\begin{minipage}[b]{\linewidth}\raggedright
Service
\end{minipage} & \begin{minipage}[b]{\linewidth}\raggedright
Method
\end{minipage} & \begin{minipage}[b]{\linewidth}\raggedright
Notes
\end{minipage} \\
\midrule\noalign{}
\endhead
\bottomrule\noalign{}
\endlastfoot
\texttt{DDMStructureService} & \texttt{getStructureWithStructureId} &
Load form \\
\texttt{ScreensddlrecordService} (Screens compatibility plugin) &
\texttt{getDdlRecord} & Load record \\
\texttt{DLAppService} & \texttt{addFileEntry} & Upload document \\
\texttt{DDLRecordService} & \texttt{addRecord} & Submit form \\
\texttt{DDLRecordService} & \texttt{updateRecord} & Update form \\
\end{longtable}

\noindent\hrulefill

\subsection{Module}\label{module-26}

\begin{itemize}
\tightlist
\item
  DDL
\end{itemize}

\subsection{Themes}\label{themes-4}

\begin{itemize}
\tightlist
\item
  Default
\end{itemize}

The Default Theme uses a standard \texttt{UITableView} to show a
scrollable list of fields. Other Themes may use a different component,
such as \texttt{UICollectionView} or others, to show the fields.

\begin{figure}
\centering
\includegraphics{./images/screens-ios-ddlform.png}
\caption{DDL Form Screenlet using the Default (\texttt{default}) Theme.}
\end{figure}

\subsubsection{Custom Cells}\label{custom-cells}

A Theme needs to define a cell view for each field type. For instance,
the \texttt{xib} file \texttt{DDLFieldDateTableCell\_default} is used to
render \texttt{Date} fields in the Default Theme.

If you want a specific field to have a unique appearance, you can
customize your field's display by using the following filename pattern,
where \texttt{XXX} is your field's name:
\texttt{DDLCustomFieldXXXTableCell\_default}. For example, the ``Are you
a subscriber?'' field in screenshot above shows how text fields appear
in the Default Theme. If you want to customize this, you don't need to
create an entire Theme. You just need to create an \texttt{xib} file for
the field \texttt{subscriberName}. The filename is therefore
\texttt{DDLCustomFieldSubscriberNameTableCell\_default}. Be careful to
keep the same components and \texttt{IBOutlet} defined in the custom
file.

\subsection{Portal Configuration}\label{portal-configuration-13}

Before using DDL Form Screenlet, you should make sure that Dynamic Data
Lists and Data Types are configured properly in the portal. Refer to the
\href{/docs/7-1/user/-/knowledge_base/u/creating-data-definitions}{Creating
Data Definitions} and
\href{/docs/7-1/user/-/knowledge_base/u/creating-data-lists}{Creating
Data Lists} sections of the User Guide for more details. If Workflow is
required, it must also be configured. See the
\href{/docs/7-1/user/-/knowledge_base/u/workflow}{Using Workflow}
section of the User Guide for details.

\subsubsection{Permissions}\label{permissions-2}

To use DDL Form Screenlet to add new records, you must grant the Add
Record permission in the Dynamic Data List in the portal. If you want to
use DDL Form Screenlet to view or edit record values, you must also
grant the View and Update permissions, respectively. The Add Record,
View, and Update permissions are highlighted by the red boxes in the
following screenshot:

\begin{figure}
\centering
\includegraphics{./images/screens-portal-permission-ddl.png}
\caption{The permissions for adding, viewing, and editing DDL records.}
\end{figure}

Also, if your form includes at least one Documents and Media field, you
must grant permissions in the target repository and folder. For more
details, see the \texttt{repositoryId} and \texttt{folderId} attributes
below.

\begin{figure}
\centering
\includegraphics{./images/screens-portal-permission-folder-add.png}
\caption{The permission for adding a document to a Documents and Media
folder.}
\end{figure}

For more details, please see the User Guide sections
\href{/docs/7-1/user/-/knowledge_base/u/creating-data-definitions}{Creating
Data Definitions},
\href{/docs/7-1/user/-/knowledge_base/u/creating-data-lists}{Creating
Data Lists}, and \href{/docs/7-1/user/-/knowledge_base/u/workflow}{Using
Workflow}.

\subsection{Offline}\label{offline-24}

This Screenlet supports offline mode so it can function without a
network connection. For more information on how offline mode works, see
the
\href{/docs/7-1/tutorials/-/knowledge_base/t/architecture-of-offline-mode-in-liferay-screens}{tutorial
on its architecture}.

When loading the form or record, the Screenlet supports the following
offline mode policies:

\noindent\hrulefill

Policy \textbar{} What happens \textbar{} When to use \textbar{}
\texttt{remote-only} \textbar{} The Screenlet loads the form or record
from the portal. If a connection issue occurs, the Screenlet uses the
delegate to notify the developer about the error. If the Screenlet loads
the form or record, it stores the received data (record structure and
data) in the local cache for later use. \textbar{} Use this policy when
you always need to show updated data, and show nothing when there's no
connection.\textbar{} \texttt{cache-only} \textbar{} The Screenlet loads
the form or record from the local cache. If the form or record isn't
there, the Screenlet uses the delegate to notify the developer about the
error. \textbar{} Use this policy when you always need to show local
data, without retrieving remote information under any
circumstance.\textbar{} \texttt{remote-first} \textbar{} The Screenlet
requests the form or record from the portal. The Screenlet shows the
record or form to the user and stores it in the local cache for later
use. If a connection issue occurs, the Screenlet retrieves the form or
record from the local cache. If the form or record doesn't exist there,
the Screenlet uses the delegate to notify the developer about the error.
\textbar{} Use this policy to show the most recent version of the data
when connected, but show an outdated version when there's no connection.
\textbar{} \texttt{cache-first} \textbar{} If the form or record exists
in the local cache, the Screenlet loads it from there. If it doesn't
exist there, the Screenlet requests it from the portal and notifies the
developer about any errors that occur (including connectivity errors).
\textbar{} Use this policy to save bandwidth and loading time in case
you have local (but probably outdated) data. \textbar{}

\noindent\hrulefill

When editing the record, the Screenlet supports the following offline
mode policies:

\noindent\hrulefill

Policy \textbar{} What happens \textbar{} When to use \textbar{}
\texttt{remote-only} \textbar{} The Screenlet sends the record to the
portal. If a connection issue occurs, the Screenlet uses the delegate to
notify the developer about the error, but it also discards the record.
\textbar{} Use this policy to make sure the portal always has the most
recent version of the record. \textbar{} \texttt{cache-only} \textbar{}
The Screenlet stores the record in the local cache. \textbar{} Use this
policy when you need to save the data locally, but don't want to update
the data in the portal (update or add record). \textbar{}
\texttt{remote-first} \textbar{} The Screenlet sends the record to the
portal. If this succeeds, it also stores the record in the local cache
for later usage. If a connection issue occurs, then Screenlet stores the
record in the local cache with the \emph{dirty flag} enabled. This
causes the synchronization process to send the record to the portal when
it runs. \textbar{} Use this policy when you need to make sure the
Screenlet sends the record to the portal as soon as the connection is
restored. \textbar{} \texttt{cache-first} \textbar{} The Screenlet
stores the record in the local cache and then sends it to the remote
portal. If a connection issue occurs, then Screenlet stores the record
in the local cache with the \emph{dirty flag} enabled. This causes the
the synchronization process to send the record to the portal when it
runs. \textbar{} Use this policy when you need to make sure the
Screenlet sends the record to the portal as soon as the connection is
restored. Compared to \texttt{remote-first}, this policy always stores
the record in the cache. The \texttt{remote-first} policy only stores
the record in the event of a network error. \textbar{}

\noindent\hrulefill

\subsection{Required Attributes}\label{required-attributes-21}

\begin{itemize}
\tightlist
\item
  \texttt{structureId}
\item
  \texttt{recordSetId}
\end{itemize}

\subsection{Attributes}\label{attributes-26}

\noindent\hrulefill

Attribute \textbar{} Data Type \textbar{} Explanation \textbar{}
\texttt{structureId} \textbar{} \texttt{number} \textbar{} This is the
identifier of a data definition for your site in Liferay. To find the
identifiers for your data definitions, click \emph{Admin} from the
Dockbar and select \emph{Content}. Then click \emph{Dynamic Data Lists}
and click the \emph{Manage Data Definitions} button. The identifier of
each data definition is in the ID column of the table that appears.
\textbar{} \texttt{groupId} \textbar{} \texttt{number} \textbar{} The
site (group) identifier where the record is stored. If this value is
\texttt{0}, the \texttt{groupId} specified in
\texttt{LiferayServerContext} is used. \textbar{} \texttt{recordSetId}
\textbar{} \texttt{number} \textbar{} The identifier of a dynamic data
list. To find the identifiers for your dynamic data lists, click
\emph{Admin} from the Dockbar and select \emph{Content}. Then click
\emph{Dynamic Data Lists}. The identifier of each dynamic data list is
in the ID column of the table that appears. \textbar{} \texttt{recordId}
\textbar{} \texttt{number} \textbar{} The identifier of the record you
want to show. Setting the \texttt{editable} attribute to \texttt{true}
allows editing of the record's values. The \texttt{recordId} can be
obtained from other methods or delegates. \textbar{}
\texttt{repositoryId} \textbar{} \texttt{number} \textbar{} The
identifier of the Documents and Media repository to upload to. If this
value is \texttt{0}, the default repository for the site specified in
\texttt{groupId} is used. \textbar{} \texttt{folderId} \textbar{}
\texttt{number} \textbar{} The identifier of the folder where Documents
and Media files are uploaded. If this value is \texttt{0}, the root
folder is used. \textbar{} \texttt{filePrefix} \textbar{}
\texttt{string} \textbar{} The prefix to attach to the names of files
uploaded to a Documents and Media repository. A random GUID string is
appended following the prefix. \textbar{} \texttt{autoLoad} \textbar{}
\texttt{boolean} \textbar{} Sets whether or not the form is loaded when
the Screenlet is shown. If \texttt{recordId} is set, the record value is
loaded together with the form definition. \textbar{}
\texttt{autoscrollOnValidation} \textbar{} \texttt{boolean} \textbar{}
Sets whether or not the form automatically scrolls to the first failed
field when validation is used. \textbar{} \texttt{showSubmitButton}
\textbar{} \texttt{boolean} \textbar{} Sets whether or not the form
shows a submit button at the bottom. If this is set to \texttt{false},
you should call the \texttt{submitForm()} method. \textbar{}
\texttt{editable} \textbar{} \texttt{boolean} \textbar{} Sets whether
the values can be changed by the user. The default is \texttt{true}.
\textbar{}

\noindent\hrulefill

\subsection{Methods}\label{methods-14}

\noindent\hrulefill

Method \textbar{} Return Type \textbar{} Explanation \textbar{}
\texttt{loadForm()} \textbar{} \texttt{boolean} \textbar{} Starts the
request to load the form definition. The form fields are shown when the
response is received. This method returns \texttt{true} if the request
is sent. \textbar{} \texttt{loadRecord()} \textbar{} \texttt{boolean}
\textbar{} Starts the request to load the record specified in
\texttt{recordId}. If needed, the form definition is also loaded. The
form fields are shown filled with record values when the response is
received. This method returns \texttt{true} if the request is sent.
\textbar{} \texttt{submitForm()} \textbar{} \texttt{boolean} \textbar{}
Starts the request to submit form values to the dynamic data list
specified in \texttt{recordSetId}. All fields are validated prior to
submission. Validation errors stop the submit process. \textbar{}

\noindent\hrulefill

\subsection{Delegate}\label{delegate-4}

DDL Form Screenlet delegates some events to an object that conforms with
the \texttt{DDLFormScreenletDelegate} protocol. This protocol lets you
implement the following methods:

\begin{itemize}
\item
  \texttt{-\ screenlet:onFormLoaded:}: Called when the form is loaded.
  The second parameter (\texttt{record}) contains only field
  definitions.
\item
  \texttt{-\ screenlet:onFormLoadError:}: Called when an error occurs
  while loading the form. The \texttt{NSError} object describes the
  error.
\item
  \texttt{-\ screenlet:onRecordLoaded:}: Called when a form with values
  loads. The second parameter (\texttt{record}) contains field
  definitions and values. The method \texttt{onFormLoadResult} is called
  before \texttt{onRecordLoaded}.
\item
  \texttt{-\ screenlet:onRecordLoadError:}: Called when an error occurs
  while loading a record. The \texttt{NSError} object describes the
  error.
\item
  \texttt{-\ screenlet:onFormSubmitted:}: Called when the form values
  are successfully submitted to the server.
\item
  \texttt{-\ screenlet:onFormSubmitError:}: Called when an error occurs
  while submitting the form. The \texttt{NSError} object describes the
  error.
\item
  \texttt{-\ screenlet:onDocumentFieldUploadStarted:}: Called when the
  upload of a Documents and Media field begins.
\item
  \texttt{-\ screenlet:onDocumentField:uploadedBytes:totalBytes:}:
  Called when a block of bytes in a Documents and Media field is
  uploaded. This method is intended to track progress of the uploads.
\item
  \texttt{-\ screenlet:onDocumentField:uploadResult:}: Called when a
  Documents and Media field upload is completed.
\item
  \texttt{-\ screenlet:onDocumentField:uploadError:}: Called when an
  error occurs in the Documents and Media upload process. The
  \texttt{NSError} object describes the error.
\end{itemize}

\section{DDL List Screenlet for iOS}\label{ddl-list-screenlet-for-ios}

\subsection{Requirements}\label{requirements-27}

\begin{itemize}
\tightlist
\item
  Xcode 9.3 or above
\item
  iOS 11 SDK
\item
  Liferay Portal 6.2 CE/EE, Liferay CE Portal 7.0/7.1, Liferay DXP
\item
  Liferay Screens Compatibility app
  (\href{http://www.liferay.com/marketplace/-/mp/application/54365664}{CE}
  or
  \href{http://www.liferay.com/marketplace/-/mp/application/54369726}{EE/DXP}).
  This app is preinstalled in Liferay CE Portal 7.0/7.1 and Liferay DXP.
\end{itemize}

\subsection{Compatibility}\label{compatibility-27}

\begin{itemize}
\tightlist
\item
  iOS 9 and above
\end{itemize}

\subsection{Xamarin Requirements}\label{xamarin-requirements-27}

\begin{itemize}
\tightlist
\item
  Visual Studio 7.2
\item
  Mono .NET framework 5.4.1.6
\end{itemize}

\subsection{Features}\label{features-27}

The DDL List Screenlet enables the following features:

\begin{itemize}
\tightlist
\item
  Shows a scrollable collection of DDL records.
\item
  Implements
  \href{http://www.iosnomad.com/blog/2014/4/21/fluent-pagination}{fluent
  pagination} with configurable page size.
\item
  Allows filtering of records by creator.
\item
  Supports i18n in record values.
\end{itemize}

\subsection{JSON Services Used}\label{json-services-used-25}

Screenlets in Liferay Screens call JSON web services in the portal. This
Screenlet calls the following services and methods.

\noindent\hrulefill

\begin{longtable}[]{@{}
  >{\raggedright\arraybackslash}p{(\columnwidth - 4\tabcolsep) * \real{0.3889}}
  >{\raggedright\arraybackslash}p{(\columnwidth - 4\tabcolsep) * \real{0.3333}}
  >{\raggedright\arraybackslash}p{(\columnwidth - 4\tabcolsep) * \real{0.2778}}@{}}
\toprule\noalign{}
\begin{minipage}[b]{\linewidth}\raggedright
Service
\end{minipage} & \begin{minipage}[b]{\linewidth}\raggedright
Method
\end{minipage} & \begin{minipage}[b]{\linewidth}\raggedright
Notes
\end{minipage} \\
\midrule\noalign{}
\endhead
\bottomrule\noalign{}
\endlastfoot
\texttt{ScreensddlrecordService} (Screens compatibility plugin) &
\texttt{getDdlRecords} & With \texttt{ddlRecordSetId}, or
\texttt{ddlRecordSetId} and \texttt{userId} \\
\texttt{ScreensddlrecordService} (Screens compatibility plugin) &
\texttt{getDdlRecordsCount} & \\
\end{longtable}

\noindent\hrulefill

\subsection{Module}\label{module-27}

\begin{itemize}
\tightlist
\item
  DDL
\end{itemize}

\subsection{Themes}\label{themes-5}

\begin{itemize}
\tightlist
\item
  The Default Theme uses a standard \texttt{UITableView} to show the
  scrollable list. Other Themes may use a different component, such as
  \texttt{UICollectionView} or others, to show the items.
\end{itemize}

\begin{figure}
\centering
\includegraphics{./images/screens-ios-ddllist.png}
\caption{The DDL List Screenlet using the Default (\texttt{default})
Theme.}
\end{figure}

\subsection{Portal Configuration}\label{portal-configuration-14}

Dynamic Data Lists (DDL) and Data Types should be configured in the
portal. For more details, please refer to the Liferay User Guide
sections
\href{/docs/7-1/user/-/knowledge_base/u/creating-data-definitions}{Creating
Data Definitions} and
\href{/docs/7-1/user/-/knowledge_base/u/creating-data-lists}{Creating
Data Lists}.

\subsection{Offline}\label{offline-25}

This Screenlet supports offline mode so it can function without a
network connection. For more information on how offline mode works, see
the
\href{/docs/7-1/tutorials/-/knowledge_base/t/architecture-of-offline-mode-in-liferay-screens}{tutorial
on its architecture}. Here are the offline mode policies that you can
use with this Screenlet:

\noindent\hrulefill

Policy \textbar{} What happens \textbar{} When to use \textbar{}
\texttt{remote-only} \textbar{} The Screenlet loads the list from the
portal. If a connection issue occurs, the Screenlet uses the delegate to
notify the developer about the error. If the Screenlet successfully
loads the list, it stores the data in the local cache for later use.
\textbar{} Use this policy when you always need to show updated data,
and show nothing when there's no connection. \textbar{}
\texttt{cache-only} \textbar{} The Screenlet loads the list from the
local cache. If the list isn't there, the Screenlet uses the delegate to
notify the developer about the error. \textbar{} Use this policy when
you always need to show local data, without retrieving remote
information under any circumstance. \textbar{} \texttt{remote-first}
\textbar{} The Screenlet loads the list from the portal. If this
succeeds, the Screenlet shows the list to the user and stores it in the
local cache for later use. If a connection issue occurs, the Screenlet
retrieves the list from the local cache. If the list doesn't exist
there, the Screenlet uses the delegate to notify the developer about the
error. \textbar{} Use this policy to show the most recent version of the
data when connected, but show an outdated version when there's no
connection. \textbar{} \texttt{cache-first} \textbar{} The Screenlet
loads the list from the local cache. If the list isn't there, the
Screenlet requests it from the portal and notifies the developer about
any errors that occur (including connectivity errors). \textbar{} Use
this policy to save bandwidth and loading time in case you have local
(but probably outdated) data. \textbar{}

\noindent\hrulefill

\subsection{Required Attributes}\label{required-attributes-22}

\begin{itemize}
\tightlist
\item
  \texttt{recordSetId}
\item
  \texttt{labelFields}
\end{itemize}

\subsection{Attributes}\label{attributes-27}

\noindent\hrulefill

Attribute \textbar{} Data type \textbar{} Explanation \textbar{}
\texttt{recordSetId} \textbar{} \texttt{number} \textbar{} The ID of the
DDL being called. To find the IDs for your DDLs, first open the Product
Menu and select the site that contains your DDLs. Then click
\emph{Content} → \emph{Dynamic Data Lists}. Each DDL's ID is in the
table's ID column. \textbar{} \texttt{userId} \textbar{} \texttt{number}
\textbar{} The ID of the user to filter records on. Records aren't
filtered if the \texttt{userId} is \texttt{0}. The default value is
\texttt{0}. \textbar{} \texttt{labelFields} \textbar{} \texttt{string}
\textbar{} The comma-separated names of the DDL fields to show. Refer to
the list's data definition to find the field names. To do so, first open
the Product Menu and select the site that contains your DDLs. Then click
\emph{Content} → \emph{Dynamic Data Lists}, and find the find the icon
(\includegraphics{./images/icon-options.png}) for the Dynamic Data List
configuration menu at the upper right. Click this icon and select
\emph{Manage Data Definitions}. You can view the fields by clicking on
any of the data definitions in the table that appears. Note that the
appearance of these values in your app depends on the Theme selected by
the user. \textbar{} \texttt{offlinePolicy} \textbar{} \texttt{string}
\textbar{} The offline mode setting. The default value is
\texttt{remote-first}. See the
\href{/docs/7-1/reference/-/knowledge_base/r/ddllistscreenlet-for-ios\#offline}{Offline
section} for details. \textbar{} \texttt{autoLoad} \textbar{}
\texttt{boolean} \textbar{} Whether the list loads automatically when
the Screenlet appears in the app's UI. The default value is
\texttt{true}. \textbar{} \texttt{refreshControl} \textbar{}
\texttt{boolean} \textbar{} Whether a standard
\href{https://developer.apple.com/library/ios/documentation/UIKit/Reference/UIRefreshControl_class/}{iOS
\texttt{UIRefreshControl}} appears when the user performs the pull to
refresh gesture. The default value is \texttt{true}. \textbar{}
\texttt{firstPageSize} \textbar{} \texttt{number} \textbar{} The number
of items retrieved from the server for display on the first page. The
default value is \texttt{50}. \textbar{} \texttt{pageSize} \textbar{}
\texttt{number} \textbar{} The number of items retrieved from the server
for display on the second and subsequent pages. The default value is
\texttt{25}. \textbar{} \texttt{obcClassName} \textbar{} \texttt{string}
\textbar{} The name of the \texttt{OrderByComparator} class to use to
sort the results. Omit this property if you don't want to sort the
results.
\href{https://github.com/liferay/liferay-portal/tree/master/modules/apps/forms-and-workflow/dynamic-data-lists/dynamic-data-lists-api/src/main/java/com/liferay/dynamic/data/lists/util/comparator}{Click
here} to see some comparator classes. Note, however, that not all of
these classes can be used with \texttt{obcClassName}. You can only use
comparator classes that extend
\texttt{OrderByComparator\textless{}DDLRecord\textgreater{}}. You can
also create your own comparator classes that extend
\texttt{OrderByComparator\textless{}DDLRecord\textgreater{}}. \textbar{}

\noindent\hrulefill

\subsection{Methods}\label{methods-15}

\noindent\hrulefill

Method \textbar{} Return \textbar{} Explanation \textbar{}
\texttt{loadList()} \textbar{} \texttt{boolean} \textbar{} Starts the
request to load the list of records. The list is shown when the response
is received. This method returns \texttt{true} if the request is sent.
\textbar{}

\noindent\hrulefill

\subsection{Delegate}\label{delegate-5}

The DDL List Screenlet delegates some events in an object that conforms
to the \texttt{DDLListScreenletDelegate} protocol. This protocol lets
you implement the following methods:

\begin{itemize}
\item
  \texttt{-\ screenlet:onDDLListResponseRecords:}: Called when a page of
  contents is received. Note that this method may be called more than
  once; once for each retrieved page.
\item
  \texttt{-\ screenlet:onDDLListError:}: Called when an error occurs in
  the process. The \texttt{NSError} object describes the error.
\item
  \texttt{-\ screenlet:onDDLSelectedRecord:}: Called when an item in the
  list is selected.
\end{itemize}

\section{Asset List Screenlet for
iOS}\label{asset-list-screenlet-for-ios}

\subsection{Requirements}\label{requirements-28}

\begin{itemize}
\tightlist
\item
  Xcode 9.3 or above
\item
  iOS 11 SDK
\item
  Liferay Portal 6.2 CE/EE, Liferay CE Portal 7.0/7.1, Liferay DXP
\item
  Liferay Screens Compatibility app
  (\href{http://www.liferay.com/marketplace/-/mp/application/54365664}{CE}
  or
  \href{http://www.liferay.com/marketplace/-/mp/application/54369726}{EE/DXP}).
  This app is preinstalled in Liferay CE Portal 7.0/7.1 and Liferay DXP.
\end{itemize}

\subsection{Compatibility}\label{compatibility-28}

\begin{itemize}
\tightlist
\item
  iOS 9 and above
\end{itemize}

\subsection{Xamarin Requirements}\label{xamarin-requirements-28}

\begin{itemize}
\tightlist
\item
  Visual Studio 7.2
\item
  Mono .NET framework 5.4.1.6
\end{itemize}

\subsection{Features}\label{features-28}

The Asset List Screenlet can be used to show lists of
\href{/docs/7-1/tutorials/-/knowledge_base/t/asset-framework}{assets}
from a Liferay instance. For example, you can use the Screenlet to show
a scrollable collection of assets. It also implements
\href{http://www.iosnomad.com/blog/2014/4/21/fluent-pagination}{fluent
pagination} with configurable page size. The Asset List Screenlet can
show assets of the following classes:

\begin{itemize}
\tightlist
\item
  \texttt{BlogsEntry}
\item
  \texttt{BookmarksEntry}
\item
  \texttt{BookmarksFolder}
\item
  \texttt{CalendarEvent}
\item
  \texttt{DLFileEntry}
\item
  \texttt{DDLRecord}
\item
  \texttt{DDLRecordSet}
\item
  \texttt{Group}
\item
  \texttt{JournalArticle} (Web Content)
\item
  \texttt{JournalFolder}
\item
  \texttt{Layout}
\item
  \texttt{LayoutRevision}
\item
  \texttt{MBThread}
\item
  \texttt{MBCategory}
\item
  \texttt{MBDiscussion}
\item
  \texttt{MBMailingList}
\item
  \texttt{Organization}
\item
  \texttt{User}
\item
  \texttt{WikiPage}
\item
  \texttt{WikiPageResource}
\item
  \texttt{WikiNode}
\end{itemize}

The Asset List Screenlet also supports i18n in asset values.

\subsection{JSON Services Used}\label{json-services-used-26}

Screenlets in Liferay Screens call JSON web services in the portal. This
Screenlet calls the following services and methods.

\noindent\hrulefill

\begin{longtable}[]{@{}
  >{\raggedright\arraybackslash}p{(\columnwidth - 4\tabcolsep) * \real{0.3889}}
  >{\raggedright\arraybackslash}p{(\columnwidth - 4\tabcolsep) * \real{0.3333}}
  >{\raggedright\arraybackslash}p{(\columnwidth - 4\tabcolsep) * \real{0.2778}}@{}}
\toprule\noalign{}
\begin{minipage}[b]{\linewidth}\raggedright
Service
\end{minipage} & \begin{minipage}[b]{\linewidth}\raggedright
Method
\end{minipage} & \begin{minipage}[b]{\linewidth}\raggedright
Notes
\end{minipage} \\
\midrule\noalign{}
\endhead
\bottomrule\noalign{}
\endlastfoot
\texttt{ScreensddlrecordService} (Screens compatibility plugin) &
\texttt{getAssetEntries} & With \texttt{entryQuery} \\
\texttt{ScreensddlrecordService} (Screens compatibility plugin) &
\texttt{getAssetEntries} & With \texttt{companyId}, \texttt{groupId},
and \texttt{portletItemName} \\
\texttt{AssetEntryService} & \texttt{getEntriesCount} & \\
\end{longtable}

\noindent\hrulefill

\subsection{Module}\label{module-28}

\begin{itemize}
\tightlist
\item
  None
\end{itemize}

\subsection{Themes}\label{themes-6}

\begin{itemize}
\tightlist
\item
  Default
\end{itemize}

The Default Theme uses a standard \texttt{UITableView} to show the
scrollable list. Other Themes may use a different component, such as
\texttt{UICollectionView} or others, to show the items.

\begin{figure}
\centering
\includegraphics{./images/screens-ios-assetlist.png}
\caption{Asset List Screenlet using the Default (\texttt{default})
Theme.}
\end{figure}

\subsection{Offline}\label{offline-26}

This Screenlet supports offline mode so it can function without a
network connection. For more information on how offline mode works, see
the
\href{/docs/7-1/tutorials/-/knowledge_base/t/architecture-of-offline-mode-in-liferay-screens}{tutorial
on its architecture}. Here are the offline mode policies that you can
use with this Screenlet:

\noindent\hrulefill

Policy \textbar{} What happens \textbar{} When to use \textbar{}
\texttt{remote-only} \textbar{} The Screenlet loads the list from the
portal. If a connection issue occurs, the Screenlet uses the delegate to
notify the developer about the error. If the Screenlet successfully
loads the list, it stores the data in the local cache for later use.
\textbar{} Use this policy when you always need to show updated data,
and show nothing when there's no connection. \textbar{}
\texttt{cache-only} \textbar{} The Screenlet loads the list from the
local cache. If the list isn't there, the Screenlet uses the delegate to
notify the developer about the error. \textbar{} Use this policy when
you always need to show local data, without retrieving remote
information under any circumstance. \textbar{} \texttt{remote-first}
\textbar{} The Screenlet loads the list from the portal. If this
succeeds, the Screenlet shows the list to the user and stores it in the
local cache for later use. If a connection issue occurs, the Screenlet
retrieves the list from the local cache. If the list doesn't exist
there, the Screenlet uses the delegate to notify the developer about the
error. \textbar{} Use this policy to show the most recent version of the
data when connected, but show an outdated version when there's no
connection. \textbar{} \texttt{cache-first} \textbar{} The Screenlet
loads the list from the local cache. If the list isn't there, the
Screenlet requests it from the portal and notifies the developer about
any errors that occur (including connectivity errors). \textbar{} Use
this policy to save bandwidth and loading time in case you have local
(but probably outdated) data. \textbar{}

\noindent\hrulefill

\subsection{Required Attributes}\label{required-attributes-23}

\begin{itemize}
\tightlist
\item
  \texttt{classNameId}
\end{itemize}

If you don't use \texttt{classNameId}, you must use this attribute:

\begin{itemize}
\tightlist
\item
  \texttt{portletItemName}
\end{itemize}

\subsection{Attributes}\label{attributes-28}

\noindent\hrulefill

Attribute \textbar{} Data type \textbar{} Explanation \textbar{}
\texttt{groupId} \textbar{} \texttt{number} \textbar{} The ID of the
site (group) where the asset is stored. If set to \texttt{0}, the
\texttt{groupId} specified in \texttt{LiferayServerContext} is used. The
default value is \texttt{0}. \textbar{} \texttt{classNameId} \textbar{}
\texttt{number} \textbar{} The ID of the asset's class name. Use values
from the \texttt{AssetClassNameId} class or the Liferay Instance's
\texttt{classname\_} database table. \textbar{} \texttt{portletItemName}
\textbar{} \texttt{string} \textbar{} The name of the
\href{/docs/7-1/user/-/knowledge_base/u/configuration-templates}{configuration
template} you used in the Asset Publisher. To use this feature, add an
Asset Publisher to one of your site's pages (it may be a hidden page),
configure the Asset Publisher's filter (in \emph{Configuration} →
\emph{Setup} → \emph{Asset Selection}), and then use the Asset
Publisher's \emph{Configuration Templates} option to save this
configuration with a name. Use this name as this attribute's value.
\textbar{} \texttt{offlinePolicy} \textbar{} \texttt{string} \textbar{}
The offline mode setting. The default value is \texttt{remote-first}.
See the
\href{/docs/7-1/reference/-/knowledge_base/r/assetlistscreenlet-for-ios\#offline}{Offline
section} for details. \textbar{} \texttt{autoLoad} \textbar{}
\texttt{boolean} \textbar{} Whether the list loads automatically when
the Screenlet appears in the app's UI. The default value is
\texttt{true}. \textbar{} \texttt{refreshControl} \textbar{}
\texttt{boolean} \textbar{} Defines whether a standard
\href{https://developer.apple.com/library/ios/documentation/UIKit/Reference/UIRefreshControl_class/}{ios
\texttt{UIRefreshControl}} appears when the user does the pull to
refresh gesture. The default value is \texttt{true}. \textbar{}
\texttt{firstPageSize} \textbar{} \texttt{number} \textbar{} The number
of items retrieved from the server for display on the first page. The
default value is \texttt{50}. \textbar{} \texttt{pageSize} \textbar{}
\texttt{number} \textbar{} The number of items retrieved from the server
for display on the second and subsequent pages. The default value is
\texttt{25}. \textbar{} \texttt{customEntryQuery} \textbar{}
\texttt{Dictionary} \textbar{} The set of keys (string) and values
(string or number) to be used in the
\href{https://docs.liferay.com/dxp/portal/7.1-latest/javadocs/portal-kernel/com/liferay/asset/kernel/service/persistence/AssetEntryQuery.html}{\texttt{AssetEntryQuery}
object}. These values filter the assets returned by the Liferay
instance. \textbar{}

\noindent\hrulefill

\subsection{Methods}\label{methods-16}

\noindent\hrulefill

Method \textbar{} Return \textbar{} Explanation \textbar{}
\texttt{loadList()} \textbar{} \texttt{boolean} \textbar{} Starts the
request to load the list of assets. This list is shown when the response
is received. Returns \texttt{true} if the request is sent. \textbar{}

\noindent\hrulefill

\subsection{Delegate}\label{delegate-6}

The Asset List Screenlet delegates some events to an object that
conforms to the \texttt{AssetListScreenletDelegate} protocol. This
protocol lets you implement the following methods:

\begin{itemize}
\item
  \texttt{-\ screenlet:onAssetListResponse:}: Called when a page of
  assets is received. Note that this method may be called more than
  once; one call for each page received.
\item
  \texttt{-\ screenlet:onAssetListError:}: Called when an error occurs
  in the process. The \texttt{NSError} object describes the error.
\item
  \texttt{-\ screenlet:onAssetSelected:}: Called when an item in the
  list is selected.
\end{itemize}

\section{Web Content Display Screenlet for
iOS}\label{web-content-display-screenlet-for-ios}

\subsection{Requirements}\label{requirements-29}

\begin{itemize}
\tightlist
\item
  Xcode 9.3 or above
\item
  iOS 11 SDK
\item
  Liferay Portal 6.2 CE/EE, Liferay CE Portal 7.0/7.1, Liferay DXP
\item
  Liferay Screens Compatibility app
  (\href{http://www.liferay.com/marketplace/-/mp/application/54365664}{CE}
  or
  \href{http://www.liferay.com/marketplace/-/mp/application/54369726}{EE/DXP}).
  This app is preinstalled in Liferay CE Portal 7.0/7.1 and Liferay DXP.
\end{itemize}

\subsection{Compatibility}\label{compatibility-29}

\begin{itemize}
\tightlist
\item
  iOS 9 and above
\end{itemize}

\subsection{Xamarin Requirements}\label{xamarin-requirements-29}

\begin{itemize}
\tightlist
\item
  Visual Studio 7.2
\item
  Mono .NET framework 5.4.1.6
\end{itemize}

\subsection{Features}\label{features-29}

The Web Content Display Screenlet shows web content elements in your
app, rendering the inner HTML of the web content. The Screenlet also
supports i18n, rendering contents differently depending on the device's
current locale.

\subsection{JSON Services Used}\label{json-services-used-27}

Screenlets in Liferay Screens call JSON web services in the portal. This
Screenlet calls the following services and methods.

\noindent\hrulefill

\begin{longtable}[]{@{}
  >{\raggedright\arraybackslash}p{(\columnwidth - 4\tabcolsep) * \real{0.3889}}
  >{\raggedright\arraybackslash}p{(\columnwidth - 4\tabcolsep) * \real{0.3333}}
  >{\raggedright\arraybackslash}p{(\columnwidth - 4\tabcolsep) * \real{0.2778}}@{}}
\toprule\noalign{}
\begin{minipage}[b]{\linewidth}\raggedright
Service
\end{minipage} & \begin{minipage}[b]{\linewidth}\raggedright
Method
\end{minipage} & \begin{minipage}[b]{\linewidth}\raggedright
Notes
\end{minipage} \\
\midrule\noalign{}
\endhead
\bottomrule\noalign{}
\endlastfoot
\texttt{DDMStructureService} & \texttt{getStructureWithStructureId} & \\
\texttt{JournalArticleService} & \texttt{getArticleWithGroupId} & \\
\texttt{JournalArticleService} & \texttt{getArticleContent} & \\
\texttt{ScreensddlrecordService} (Screens compatibility plugin) &
\texttt{getJournalArticleContent} & With \texttt{entryQuery} \\
\end{longtable}

\noindent\hrulefill

\subsection{Module}\label{module-29}

\begin{itemize}
\tightlist
\item
  WebContent
\end{itemize}

\subsection{Themes}\label{themes-7}

\begin{itemize}
\tightlist
\item
  Default
\end{itemize}

The Default Theme uses a standard \texttt{UIWebView} to render the HTML.
Other Themes may use a different component, such as iOS 8's.

\begin{figure}
\centering
\includegraphics{./images/screens-ios-webcontent.png}
\caption{The Web Content Display Screenlet using the Default
(\texttt{default}) Theme}
\end{figure}

\subsection{Portal Configuration}\label{portal-configuration-15}

For the Web Content Display Screenlet to function properly, there should
be web content in the Liferay instance your app connects to. For more
details on web content, please refer to the
\href{/docs/7-1/user/-/knowledge_base/u/introduction-web-content}{web
content} section of the User Guide.

\subsection{Offline}\label{offline-27}

This Screenlet supports offline mode so it can function without a
network connection. For more information on how offline mode works, see
the
\href{/docs/7-1/tutorials/-/knowledge_base/t/architecture-of-offline-mode-in-liferay-screens}{tutorial
on its architecture}. Here are the offline mode policies that you can
use with this Screenlet:

\noindent\hrulefill

Policy \textbar{} What happens \textbar{} When to use \textbar{}
\texttt{remote-only} \textbar{} The Screenlet loads the content from the
portal. If a connection issue occurs, the Screenlet uses the delegate to
notify the developer about the error. If the Screenlet successfully
loads the content, it stores the data in the local cache for later use.
\textbar{} Use this policy when you always need to show updated content,
and show nothing when there's no connection. \textbar{}
\texttt{cache-only} \textbar{} The Screenlet loads the content from the
local cache. If the content isn't there, the Screenlet uses the delegate
to notify the developer about the error. \textbar{} Use this policy when
you always need to show local content, without retrieving remote content
under any circumstance. \textbar{} \texttt{remote-first} \textbar{} The
Screenlet loads the content from the portal. If this succeeds, the
Screenlet shows the content to the user and stores it in the local cache
for later use. If a connection issue occurs, the Screenlet retrieves the
content from the local cache. If the content doesn't exist there, the
Screenlet uses the delegate to notify the developer about the error.
\textbar{} Use this policy to show the most recent version of the
content when connected, but show a possibly outdated version when
there's no connection. \textbar{} \texttt{cache-first} \textbar{} The
Screenlet loads the content from the local cache. If the content isn't
there, the Screenlet requests it from the portal and notifies the
developer about any errors that occur (including connectivity errors).
\textbar{} Use this policy to save bandwidth and loading time in case
you have local (but probably outdated) content. \textbar{}

\noindent\hrulefill

\subsection{Required Attributes}\label{required-attributes-24}

\begin{itemize}
\tightlist
\item
  \texttt{articleId}
\end{itemize}

If you have
\href{/docs/7-1/user/-/knowledge_base/u/designing-uniform-content}{structured
web content}, you can alternatively use \texttt{templateId} or
\texttt{structureId} with \texttt{articleId}.

\subsection{Attributes}\label{attributes-29}

\noindent\hrulefill

Attribute \textbar{} Data type \textbar{} Explanation \textbar{}
\texttt{groupId} \textbar{} \texttt{number} \textbar{} The site (group)
identifier where the asset is stored. If this value is \texttt{0}, the
\texttt{groupId} specified in \texttt{LiferayServerContext} is used.
\textbar{} \texttt{articleId} \textbar{} \texttt{string} \textbar{} The
identifier of the web content to display. You can find the identifier by
clicking \emph{Edit} on the web content in the portal. \textbar{}
\texttt{templateId} \textbar{} \texttt{number} \textbar{} The identifier
of the template used to render the web content. This is applicable only
with
\href{/docs/7-1/user/-/knowledge_base/u/designing-uniform-content}{structured
web content}. \textbar{} \texttt{structureId} \textbar{} \texttt{number}
\textbar{} The identifier of the \texttt{DDMStructure} used to model the
web content. This parameter lets the Screenlet retrieve and parse the
structure. \textbar{} \texttt{autoLoad} \textbar{} \texttt{boolean}
\textbar{} Whether the content should be retrieved from the portal as
soon as the Screenlet appears. The default value is \texttt{true}.
\textbar{}

\noindent\hrulefill

\subsection{Methods}\label{methods-17}

\noindent\hrulefill

Method \textbar{} Return \textbar{} Explanation \textbar{}
\texttt{loadWebContent()} \textbar{} \texttt{boolean} \textbar{} Starts
the request to load the web content. The HTML is rendered when the
response is received. Returns \texttt{true} if the request is sent.
\textbar{}

\noindent\hrulefill

\subsection{Delegate}\label{delegate-7}

The Web Content Display Screenlet delegates some events to an object
that conforms to the \texttt{WebContentDisplayScreenletDelegate}
protocol. This protocol lets you implement the following methods:

\begin{itemize}
\item
  \texttt{-\ screenlet:onWebContentResponse:}: Called when the web
  content's HTML is received.
\item
  \texttt{-\ screenlet:onWebContentError:}: Called when an error occurs
  in the process. The \texttt{NSError} object describes the error.
\item
  \texttt{-\ screenlet:onRecordContentResponse:}: Called when a web
  content record is received.
\item
  \texttt{-\ screenlet:onUrlClicked:}: Called when a URL is clicked in
  the web content. Return \texttt{true} to handle the navigation, or
  \texttt{false} to cancel it.
\end{itemize}

\section{Web Content List Screenlet for
iOS}\label{web-content-list-screenlet-for-ios}

\subsection{Requirements}\label{requirements-30}

\begin{itemize}
\tightlist
\item
  Xcode 9.3 or above
\item
  iOS 11 SDK
\item
  Liferay Portal 6.2 CE/EE, Liferay CE Portal 7.0/7.1, Liferay DXP
\item
  Liferay Screens Compatibility app
  (\href{http://www.liferay.com/marketplace/-/mp/application/54365664}{CE}
  or
  \href{http://www.liferay.com/marketplace/-/mp/application/54369726}{EE/DXP}).
  This app is preinstalled in Liferay CE Portal 7.0/7.1 and Liferay DXP.
\end{itemize}

\subsection{Compatibility}\label{compatibility-30}

\begin{itemize}
\tightlist
\item
  iOS 9 and above
\end{itemize}

\subsection{Xamarin Requirements}\label{xamarin-requirements-30}

\begin{itemize}
\tightlist
\item
  Visual Studio 7.2
\item
  Mono .NET framework 5.4.1.6
\end{itemize}

\subsection{Features}\label{features-30}

Web Content List Screenlet can show lists of
\href{/docs/7-1/user/-/knowledge_base/u/introduction-web-content}{web
content} from a Liferay instance. It can show both basic and
\href{/docs/7-1/user/-/knowledge_base/u/designing-uniform-content}{structured
web content}. The Screenlet also implements
\href{http://www.iosnomad.com/blog/2014/4/21/fluent-pagination}{fluent
pagination} with configurable page size, and supports i18n in asset
values.

\subsection{JSON Services Used}\label{json-services-used-28}

Screenlets in Liferay Screens call JSON web services in the portal. This
Screenlet calls the following services and methods.

\noindent\hrulefill

\begin{longtable}[]{@{}lll@{}}
\toprule\noalign{}
Service & Method & Notes \\
\midrule\noalign{}
\endhead
\bottomrule\noalign{}
\endlastfoot
\texttt{JournalArticleService} & \texttt{getArticlesWithGroupId} & \\
\texttt{JournalArticleService} & \texttt{getArticlesCount} & \\
\end{longtable}

\noindent\hrulefill

\subsection{Module}\label{module-30}

\begin{itemize}
\tightlist
\item
  WebContent
\end{itemize}

\subsection{Themes}\label{themes-8}

\begin{itemize}
\tightlist
\item
  Default
\end{itemize}

The Default Theme uses a standard \texttt{UITableView} to show the
scrollable list. Other Themes may use a different component, such as
\texttt{UICollectionView} or others, to show the contents.

\begin{figure}
\centering
\includegraphics{./images/screens-ios-webcontent-list.png}
\caption{Web Content List Screenlet using the Default (\texttt{default})
Theme.}
\end{figure}

\subsection{Offline}\label{offline-28}

This Screenlet supports offline mode so it can function without a
network connection. For more information on how offline mode works, see
the
\href{/docs/7-1/tutorials/-/knowledge_base/t/architecture-of-offline-mode-in-liferay-screens}{tutorial
on its architecture}. Here are the offline mode policies that you can
use with this Screenlet:

\noindent\hrulefill

Policy \textbar{} What happens \textbar{} When to use \textbar{}
\texttt{remote-only} \textbar{} The Screenlet loads the list from the
Liferay instance. If a connection issue occurs, the Screenlet uses the
delegate to notify the developer about the error. If the Screenlet
successfully loads the list, it stores the data in the local cache for
later use. \textbar{} Use this policy when you always need to show
updated data, and show nothing when there's no connection. \textbar{}
\texttt{cache-only} \textbar{} The Screenlet loads the list from the
local cache. If the list isn't there, the Screenlet uses the delegate to
notify the developer about the error. \textbar{} Use this policy when
you always need to show local data, without retrieving remote
information under any circumstance. \textbar{} \texttt{remote-first}
\textbar{} The Screenlet loads the list from the Liferay instance. If
this succeeds, the Screenlet shows the list to the user and stores it in
the local cache for later use. If a connection issue occurs, the
Screenlet retrieves the list from the local cache. If the list doesn't
exist there, the Screenlet uses the delegate to notify the developer
about the error. \textbar{} Use this policy to show the most recent
version of the data when connected, but show a possibly outdated version
when there's no connection. \textbar{} \texttt{cache-first} \textbar{}
The Screenlet loads the list from the local cache. If the list isn't
there, the Screenlet requests it from the Liferay instance and notifies
the developer about any errors that occur (including connectivity
errors). \textbar{} Use this policy to save bandwidth and loading time
in case you have local (but possibly outdated) data. \textbar{}

\noindent\hrulefill

\subsection{Required Attributes}\label{required-attributes-25}

\begin{itemize}
\tightlist
\item
  \texttt{folderId}
\end{itemize}

\subsection{Attributes}\label{attributes-30}

\noindent\hrulefill

Attribute \textbar{} Data type \textbar{} Explanation \textbar{}
\texttt{groupId} \textbar{} \texttt{number} \textbar{} The ID of the
site (group) where the web content exists. If set to \texttt{0}, the
\texttt{groupId} specified in \texttt{LiferayServerContext} is used. The
default value is \texttt{0}. \textbar{} \texttt{folderId} \textbar{}
\texttt{number} \textbar{} The ID of the web content folder. If set to
\texttt{0}, the root folder is used. The default value is \texttt{0}.
\textbar{} \texttt{offlinePolicy} \textbar{} \texttt{string} \textbar{}
The offline mode setting. The default value is \texttt{remote-first}.
See the
\href{/docs/7-1/reference/-/knowledge_base/r/web-content-list-screenlet-for-ios\#offline}{Offline
section} for details. \textbar{} \texttt{autoLoad} \textbar{}
\texttt{boolean} \textbar{} Whether the list loads automatically when
the Screenlet appears in the app's UI. The default value is
\texttt{true}. \textbar{} \texttt{refreshControl} \textbar{}
\texttt{boolean} \textbar{} Whether a standard
\href{https://developer.apple.com/library/ios/documentation/UIKit/Reference/UIRefreshControl_class/}{iOS
\texttt{UIRefreshControl}} appears when the user does the pull to
refresh gesture. The default value is \texttt{true}. \textbar{}
\texttt{firstPageSize} \textbar{} \texttt{number} \textbar{} The number
of items to display on the first page. The default value is \texttt{50}.
\textbar{} \texttt{pageSize} \textbar{} \texttt{number} \textbar{} The
number of items to display on the second and subsequent pages. The
default value is \texttt{25}. \textbar{} \texttt{obcClassName}
\textbar{} \texttt{string} \textbar{} The name of the
\texttt{OrderByComparator} class to use to sort the results. Omit this
property if you don't want to sort the results.
\href{https://github.com/liferay/liferay-portal/tree/master/modules/apps/web-experience/journal/journal-api/src/main/java/com/liferay/journal/util/comparator}{Click
here} to see some comparator classes. Note, however, that not all of
these classes can be used with \texttt{obcClassName}. You can only use
comparator classes that extend
\texttt{OrderByComparator\textless{}JournalArticle\textgreater{}}. You
can also create your own comparator classes that extend
\texttt{OrderByComparator\textless{}JournalArticle\textgreater{}}.
\textbar{}

\noindent\hrulefill

\subsection{Methods}\label{methods-18}

\noindent\hrulefill

Method \textbar{} Return \textbar{} Explanation \textbar{}
\texttt{loadList()} \textbar{} \texttt{boolean} \textbar{} Starts the
request to load the web content list. This list is shown when the
response is received. Returns \texttt{true} if the request is sent
successfully. \textbar{}

\noindent\hrulefill

\subsection{Delegate}\label{delegate-8}

Web Content List Screenlet delegates some events to an object that
conforms to the \texttt{WebContentListScreenletDelegate} protocol. This
protocol lets you implement the following methods:

\begin{itemize}
\item
  \texttt{-\ screenlet:onWebContentListResponse:}: Called when a page of
  contents is received. Note that this method may be called more than
  once: one call for each page received.
\item
  \texttt{-\ screenlet:onWebContentListError:}: Called when an error
  occurs in the process. The \texttt{NSError} object describes the
  error.
\item
  \texttt{-\ screenlet:onWebContentSelected:}: Called when an item in
  the list is selected.
\end{itemize}

\section{Image Gallery Screenlet for
iOS}\label{image-gallery-screenlet-for-ios}

\subsection{Requirements}\label{requirements-31}

\begin{itemize}
\tightlist
\item
  Xcode 9.3 or above
\item
  iOS 11 SDK
\item
  Liferay Portal 6.2 CE/EE, Liferay CE Portal 7.0/7.1, Liferay DXP
\item
  Liferay Screens Compatibility app
  (\href{http://www.liferay.com/marketplace/-/mp/application/54365664}{CE}
  or
  \href{http://www.liferay.com/marketplace/-/mp/application/54369726}{EE/DXP}).
  This app is preinstalled in Liferay CE Portal 7.0/7.1 and Liferay DXP.
\end{itemize}

\subsection{Compatibility}\label{compatibility-31}

\begin{itemize}
\tightlist
\item
  iOS 9 and above
\end{itemize}

\subsection{Xamarin Requirements}\label{xamarin-requirements-31}

\begin{itemize}
\tightlist
\item
  Visual Studio 7.2
\item
  Mono .NET framework 5.4.1.6
\end{itemize}

\subsection{Features}\label{features-31}

Image Gallery Screenlet shows a list of images from a Documents and
Media folder in a Liferay instance. You can also use Image Gallery
Screenlet to upload images to and delete images from the same folder.
The Screenlet implements
\href{http://www.iosnomad.com/blog/2014/4/21/fluent-pagination}{fluent
pagination} with configurable page size, and supports i18n in asset
values.

\subsection{JSON Services Used}\label{json-services-used-29}

Screenlets in Liferay Screens call JSON web services in the portal. This
Screenlet calls the following services and methods.

\noindent\hrulefill

\begin{longtable}[]{@{}lll@{}}
\toprule\noalign{}
Service & Method & Notes \\
\midrule\noalign{}
\endhead
\bottomrule\noalign{}
\endlastfoot
\texttt{DLAppService} & \texttt{getFileEntries} & Load \\
\texttt{DLAppService} & \texttt{getFileEntriesCount} & \\
\texttt{DLAppService} & \texttt{addFileEntry} & Upload \\
\texttt{DLAppService} & \texttt{deleteFileEntry} & Delete \\
\end{longtable}

\noindent\hrulefill

\subsection{Module}\label{module-31}

\begin{itemize}
\tightlist
\item
  None
\end{itemize}

\subsection{Themes}\label{themes-9}

The default Theme uses a standard iOS \texttt{UICollectionView} to show
the scrollable list as a grid. Other Themes may use a different
component, such as \texttt{UITableView} or others, to show the contents.

This screenlet has three different Themes:

\begin{enumerate}
\def\labelenumi{\arabic{enumi}.}
\tightlist
\item
  Grid (default)
\item
  Slideshow
\item
  List
\end{enumerate}

\begin{figure}
\centering
\includegraphics{./images/screens-ios-imagegallery.png}
\caption{Image Gallery Screenlet using the Grid, Slideshow, and List
Themes.}
\end{figure}

\subsection{Offline}\label{offline-29}

This Screenlet supports offline mode so it can function without a
network connection when loading or uploading images (deleting images
while offline is unsupported). For more information on how offline mode
works, see the
\href{/docs/7-1/tutorials/-/knowledge_base/t/architecture-of-offline-mode-in-liferay-screens}{tutorial
on its architecture}. This Screenlet supports the \texttt{remote-only},
\texttt{cache-only}, \texttt{remote-first}, and \texttt{cache-first}
offline mode policies.

These policies take the following actions when loading images from a
Liferay instance:

\noindent\hrulefill

Policy \textbar{} What happens \textbar{} When to use \textbar{}
\texttt{remote-only} \textbar{} The Screenlet loads the data from the
Liferay instance. If a connection issue occurs, the Screenlet uses the
delegate to notify the developer about the error. If the Screenlet
successfully loads the data, it stores it in the local cache for later
use. \textbar{} Use this policy when you always need to show updated
data, and show nothing when there's no connection. \textbar{}
\texttt{cache-only} \textbar{} The Screenlet loads the data from the
local cache. If the data isn't there, the Screenlet uses the delegate to
notify the developer about the error. \textbar{} Use this policy when
you always need to show local data, without retrieving remote
information under any circumstance. \textbar{} \texttt{remote-first}
\textbar{} The Screenlet loads the data from the Liferay instance. If
this succeeds, the Screenlet shows the data to the user and stores it in
the local cache for later use. If a connection issue occurs, the
Screenlet retrieves the data from the local cache. If the data doesn't
exist there, the Screenlet uses the delegate to notify the developer
about the error. \textbar{} Use this policy to show the most recent
version of the data when connected, but show a possibly outdated version
when there's no connection. \textbar{} \texttt{cache-first} \textbar{}
The Screenlet loads the data from the local cache. If the data isn't
there, the Screenlet requests it from the Liferay instance and notifies
the developer about any errors that occur (including connectivity
errors). \textbar{} Use this policy to save bandwidth and loading time
in case you have local (but possibly outdated) data. \textbar{}

\noindent\hrulefill

These policies take the following actions when uploading an image to a
Liferay instance:

\noindent\hrulefill

Policy \textbar{} What happens \textbar{} When to use \textbar{}
\texttt{remote-only} \textbar{} The Screenlet sends the image to the
Liferay instance. If a connection issue occurs, the Screenlet uses the
delegate to notify the developer about the error, but it also discards
the image. \textbar{} Use this policy to make sure the Liferay instance
always has the most recent version of the image. \textbar{}
\texttt{cache-only} \textbar{} The Screenlet stores the image in the
local cache. \textbar{} Use this policy when you need to save the image
locally, but don't want to update it in the Liferay instance. \textbar{}
\texttt{remote-first} \textbar{} The Screenlet sends the image to the
Liferay instance. If this succeeds, it also stores the image in the
local cache for later use. If a connection issue occurs, the Screenlet
stores the image in the local cache and sends it to the Liferay instance
when the connection is re-established. \textbar{} Use this policy when
you need to make sure the Screenlet sends the image to the Liferay
instance as soon as the connection is restored. \textbar{}
\texttt{cache-first} \textbar{} The Screenlet stores the image in the
local cache and then attempts to send it to the Liferay instance. If a
connection issue occurs, the Screenlet sends the image to the Liferay
instance when the connection is re-established. \textbar{} Use this
policy when you need to make sure the Screenlet sends the image to the
Liferay instance as soon as the connection is restored. Compared to
\texttt{remote-first}, this policy always stores the image in the cache.
The \texttt{remote-first} policy only stores the image in the event of a
network error. \textbar{}

\noindent\hrulefill

\subsection{Required Attributes}\label{required-attributes-26}

\begin{itemize}
\tightlist
\item
  \texttt{repositoryId}
\item
  \texttt{folderId}
\end{itemize}

\subsection{Attributes}\label{attributes-31}

\noindent\hrulefill

Attribute \textbar{} Data type \textbar{} Explanation \textbar{}
\texttt{repositoryId} \textbar{} \texttt{number} \textbar{} The ID of
the Liferay instance's Documents and Media repository that contains the
image gallery. If you're using a site's default Documents and Media
repository, then the \texttt{repositoryId} matches the site ID
(\texttt{groupId}). \textbar{} \texttt{folderId} \textbar{}
\texttt{number} \textbar{} The ID of the Documents and Media repository
folder that contains the image gallery. When accessing the folder in
your browser, the \texttt{folderId} is at the end of the URL. \textbar{}
\texttt{mimeTypes} \textbar{} \texttt{string} \textbar{} The
comma-separated list of MIME types for the Screenlet to support.
\textbar{} \texttt{filePrefix} \textbar{} \texttt{string} \textbar{} The
prefix to use on uploaded image file names. \textbar{}
\texttt{offlinePolicy} \textbar{} \texttt{string} \textbar{} The offline
mode setting. The default value is \texttt{remote-first}. See the
\href{/docs/7-1/reference/-/knowledge_base/r/image-gallery-screenlet-for-ios\#offline}{Offline
section} for details. \textbar{} \texttt{autoLoad} \textbar{}
\texttt{boolean} \textbar{} Whether the list automatically loads when
the Screenlet appears in the app's UI. The default value is
\texttt{true}. \textbar{} \texttt{refreshControl} \textbar{}
\texttt{boolean} \textbar{} Whether a standard
\href{https://developer.apple.com/library/ios/documentation/UIKit/Reference/UIRefreshControl_class/}{iOS
\texttt{UIRefreshControl}} appears when the user does the pull to
refresh gesture. The default value is \texttt{true}. \textbar{}
\texttt{firstPageSize} \textbar{} \texttt{number} \textbar{} The number
of items to display on the first page. The default value is \texttt{50}.
\textbar{} \texttt{pageSize} \textbar{} \texttt{number} \textbar{} The
number of items to display on the second and subsequent pages. The
default value is \texttt{25}. \textbar{} \texttt{obcClassName}
\textbar{} \texttt{string} \textbar{} The name of the
\texttt{OrderByComparator} class to use to sort the results. Omit this
property if you don't want to sort the results. Note that you can only
use comparator classes that extend
\texttt{OrderByComparator\textless{}DLFileEntry\textgreater{}}. Liferay
contains no such comparator classes. You must therefore create your own
by extending
\texttt{OrderByComparator\textless{}DLFileEntry\textgreater{}}. To see
examples of some comparator classes that extend other Document Library
classes,
\href{https://github.com/liferay/liferay-portal/tree/master/portal-impl/src/com/liferay/portlet/documentlibrary/util/comparator}{click
here}. \textbar{}

\noindent\hrulefill

\subsection{Methods}\label{methods-19}

\noindent\hrulefill

Method \textbar{} Return \textbar{} Explanation \textbar{}
\texttt{loadList()} \textbar{} \texttt{boolean} \textbar{} Starts the
request to load the list of images. This list is shown when the response
is received. Returns \texttt{true} if the request is sent successfully.
\textbar{}

\noindent\hrulefill

\subsection{Delegate}\label{delegate-9}

Image Gallery Screenlet delegates some events to an object that conforms
to the \texttt{ImageGalleryScreenletDelegate} protocol. This protocol
lets you implement the following methods:

\begin{itemize}
\item
  \texttt{-\ screenlet:onImageEntriesResponse:}: Called when a page of
  contents is received. Note that this method may be called more than
  once: one call for each page received.
\item
  \texttt{-\ screenlet:onImageEntriesError:}: Called when an error
  occurs in the process. The \texttt{NSError} object describes the
  error.
\item
  \texttt{-\ screenlet:onImageEntrySelected:}: Called when an item in
  the list is selected.
\item
  \texttt{-\ screenlet:onImageEntryDeleted:}: Called when an image in
  the list is deleted.
\item
  \texttt{-\ screenlet:onImageEntryDeleteError:}: Called when an error
  occurs during image file deletion. The \texttt{NSError} object
  describes the error.
\item
  \texttt{-\ screenlet:onImageUploadStart:}: Called when an image is
  prepared for upload.
\item
  \texttt{-\ screenlet:onImageUploadProgress:}: Called when the image
  upload progress changes.
\item
  \texttt{-\ screenlet:onImageUploadError:}: Called when an error occurs
  in the image upload process. The \texttt{NSError} object describes the
  error.
\item
  \texttt{-\ screenlet:onImageUploaded:}: Called when the image upload
  finishes.
\item
  \texttt{-\ screenlet:onImageUploadDetailViewCreated:}: Called when the
  image upload View is instantiated. By default, the Screenlet uses a
  modal view controller to present this View. You only need to implement
  this method if you want to override this behavior. This method should
  present the View, passed as parameter, and then return \texttt{true}.
  For example, the following example implementation presents
  \texttt{ImageUploadDetailViewBase} as a parameter, and then uses it to
  customize the View's appearance:

\begin{verbatim}
  func screenlet(screenlet: ImageGalleryScreenlet, 
      onImageUploadDetailViewCreated uploadView: ImageUploadDetailViewBase) -> Bool {
          self.cardDeck?.cards[safe: 0]?.addPage(uploadView)
          self.cardDeck?.cards[safe: 0]?.moveRight()
          return true
  }
\end{verbatim}
\end{itemize}

\section{Rating Screenlet for iOS}\label{rating-screenlet-for-ios}

\subsection{Requirements}\label{requirements-32}

\begin{itemize}
\tightlist
\item
  Xcode 9.3 or above
\item
  iOS 11 SDK
\item
  Liferay Portal 6.2 CE/EE, Liferay CE Portal 7.0/7.1, Liferay DXP
\item
  Liferay Screens Compatibility app
  (\href{http://www.liferay.com/marketplace/-/mp/application/54365664}{CE}
  or
  \href{http://www.liferay.com/marketplace/-/mp/application/54369726}{EE/DXP}).
  This app is preinstalled in Liferay CE Portal 7.0/7.1 and Liferay DXP.
\end{itemize}

\subsection{Compatibility}\label{compatibility-32}

\begin{itemize}
\tightlist
\item
  iOS 9 and above
\end{itemize}

\subsection{Xamarin Requirements}\label{xamarin-requirements-32}

\begin{itemize}
\tightlist
\item
  Visual Studio 7.2
\item
  Mono .NET framework 5.4.1.6
\end{itemize}

\subsection{Features}\label{features-32}

Rating Screenlet shows an asset's rating. It also lets users update or
delete the rating. This Screenlet comes with different Themes that
display ratings as thumbs, stars, and emojis.

\subsection{JSON Services Used}\label{json-services-used-30}

Screenlets in Liferay Screens call JSON web services in the portal. This
Screenlet calls the following services and methods.

\noindent\hrulefill

\begin{longtable}[]{@{}
  >{\raggedright\arraybackslash}p{(\columnwidth - 4\tabcolsep) * \real{0.3889}}
  >{\raggedright\arraybackslash}p{(\columnwidth - 4\tabcolsep) * \real{0.3333}}
  >{\raggedright\arraybackslash}p{(\columnwidth - 4\tabcolsep) * \real{0.2778}}@{}}
\toprule\noalign{}
\begin{minipage}[b]{\linewidth}\raggedright
Service
\end{minipage} & \begin{minipage}[b]{\linewidth}\raggedright
Method
\end{minipage} & \begin{minipage}[b]{\linewidth}\raggedright
Notes
\end{minipage} \\
\midrule\noalign{}
\endhead
\bottomrule\noalign{}
\endlastfoot
\texttt{ScreensratingsentryService} (Screens compatibility plugin) &
\texttt{getRatingsEntries} & With \texttt{entryId} \\
\texttt{ScreensratingsentryService} (Screens compatibility plugin) &
\texttt{getRatingsEntries} & With \texttt{classPK} and
\texttt{className} \\
\texttt{ScreensratingsentryService} (Screens compatibility plugin) &
\texttt{updateRatingsEntry} & \\
\texttt{ScreensratingsentryService} (Screens compatibility plugin) &
\texttt{deleteRatingsEntry} & \\
\end{longtable}

\noindent\hrulefill

\subsection{Module}\label{module-32}

\begin{itemize}
\tightlist
\item
  None
\end{itemize}

\subsection{Themes}\label{themes-10}

The default Theme uses \href{https://github.com/marketplacer/Cosmos}{the
\texttt{CosmosView} library} to show an asset's rating. Other custom
Themes may use a different component, such as \texttt{UIButton} or
others, to show the items.

This screenlet has four different Themes:

\begin{enumerate}
\def\labelenumi{\arabic{enumi}.}
\tightlist
\item
  Like
\item
  Thumbs (default)
\item
  Stars
\item
  Emojis
\end{enumerate}

\begin{figure}
\centering
\includegraphics{./images/screens-ios-ratings.png}
\caption{Rating Screenlet's different Themes.}
\end{figure}

\subsection{Offline}\label{offline-30}

This Screenlet supports offline mode so it can function without a
network connection. For more information on how offline mode works, see
the
\href{/docs/7-1/tutorials/-/knowledge_base/t/architecture-of-offline-mode-in-liferay-screens}{tutorial
on its architecture}. Here are the offline mode policies that you can
use with this Screenlet:

\noindent\hrulefill

Policy \textbar{} What happens \textbar{} When to use \textbar{}
\texttt{remote-only} \textbar{} The Screenlet loads the data from the
Liferay instance. If a connection issue occurs, the Screenlet uses the
delegate to notify the developer about the error. If the Screenlet
successfully loads the data, it stores it in the local cache for later
use. \textbar{} Use this policy when you always need to show updated
data, and show nothing when there's no connection. \textbar{}
\texttt{cache-only} \textbar{} The Screenlet loads the data from the
local cache. If the data isn't there, the Screenlet uses the delegate to
notify the developer about the error. \textbar{} Use this policy when
you always need to show local data, without retrieving remote
information under any circumstance. \textbar{} \texttt{remote-first}
\textbar{} The Screenlet loads the data from the Liferay instance. If
this succeeds, the Screenlet shows the data to the user and stores it in
the local cache for later use. If a connection issue occurs, the
Screenlet retrieves the data from the local cache. If the data doesn't
exist there, the Screenlet uses the delegate to notify the developer
about the error. \textbar{} Use this policy to show the most recent
version of the data when connected, but show a possibly outdated version
when there's no connection. \textbar{} \texttt{cache-first} \textbar{}
The Screenlet loads the data from the local cache. If the data isn't
there, the Screenlet requests it from the Liferay instance and notifies
the developer about any errors that occur (including connectivity
errors). \textbar{} Use this policy to save bandwidth and loading time
in case you have local (but possibly outdated) data. \textbar{}

\noindent\hrulefill

\subsection{Required Attributes}\label{required-attributes-27}

\begin{itemize}
\tightlist
\item
  \texttt{entryId}
\end{itemize}

If you don't use \texttt{entryId}, you must use these attributes:

\begin{itemize}
\tightlist
\item
  \texttt{className}
\item
  \texttt{classPK}
\end{itemize}

\subsection{Attributes}\label{attributes-32}

\noindent\hrulefill

Attribute \textbar{} Data type \textbar{} Explanation \textbar{}
\texttt{layoutId} \textbar{} \texttt{@layout} \textbar{} The ID of the
layout to use to show the Theme. \textbar{} \texttt{autoLoad} \textbar{}
\texttt{boolean} \textbar{} Whether the rating loads automatically when
the Screenlet appears in the app's UI. The default value is
\texttt{true}. \textbar{} \texttt{editable} \textbar{} \texttt{boolean}
\textbar{} Whether the user can change the rating. \textbar{}
\texttt{entryId} \textbar{} \texttt{number} \textbar{} The primary key
of the asset with the rating to display. \textbar{} \texttt{className}
\textbar{} \texttt{string} \textbar{} The asset's fully qualified class
name. For example, a blog entry's \texttt{className} is
\texttt{com.liferay.blogs.model.BlogsEntry}. The \texttt{className}
attribute is required when using it with \texttt{classPK} to instantiate
the Screenlet. \textbar{} \texttt{classPK} \textbar{} \texttt{number}
\textbar{} The asset's unique identifier. Only use this attribute when
also using \texttt{className} to instantiate the Screenlet. \textbar{}
\texttt{groupId} \textbar{} \texttt{number} \textbar{} The ID of the
site (group) containing the asset. \textbar{} \texttt{offlinePolicy}
\textbar{} \texttt{string} \textbar{} The offline mode setting. See the
\href{/docs/7-1/reference/-/knowledge_base/r/rating-screenlet-for-ios\#offline}{Offline
section} for details. \textbar{}

\noindent\hrulefill

\subsection{Methods}\label{methods-20}

\noindent\hrulefill

Method \textbar{} Return \textbar{} Explanation \textbar{}
\texttt{loadRatings()} \textbar{} \texttt{boolean} \textbar{} Starts the
request to load the asset's ratings. \textbar{}

\noindent\hrulefill

\subsection{Delegate}\label{delegate-10}

Rating Screenlet delegates some events to an object that conforms to the
\texttt{RatingScreenletDelegate} protocol. This protocol lets you
implement the following methods:

\begin{itemize}
\item
  \texttt{-\ screenlet:onRatingRetrieve:}: Called when the ratings are
  received.
\item
  \texttt{-\ screenlet:onRatingDeleted:}: Called when a rating is
  deleted.
\item
  \texttt{-\ screenlet:onRatingUpdated:}: Called when a rating is
  updated.
\item
  \texttt{-\ screenlet:onRatingError:}: Called when an error occurs in
  the process. The \texttt{NSError} object describes the error.
\end{itemize}

\section{Comment List Screenlet for
iOS}\label{comment-list-screenlet-for-ios}

\subsection{Requirements}\label{requirements-33}

\begin{itemize}
\tightlist
\item
  Xcode 9.3 or above
\item
  iOS 11 SDK
\item
  Liferay Portal 6.2 CE/EE, Liferay CE Portal 7.0/7.1, Liferay DXP
\item
  Liferay Screens Compatibility app
  (\href{http://www.liferay.com/marketplace/-/mp/application/54365664}{CE}
  or
  \href{http://www.liferay.com/marketplace/-/mp/application/54369726}{EE/DXP}).
  This app is preinstalled in Liferay CE Portal 7.0/7.1 and Liferay DXP.
\end{itemize}

\subsection{Compatibility}\label{compatibility-33}

\begin{itemize}
\tightlist
\item
  iOS 9 and above
\end{itemize}

\subsection{Xamarin Requirements}\label{xamarin-requirements-33}

\begin{itemize}
\tightlist
\item
  Visual Studio 7.2
\item
  Mono .NET framework 5.4.1.6
\end{itemize}

\subsection{Features}\label{features-33}

Comment List Screenlet can list all the comments of an asset in a
Liferay instance. It also lets the user update or delete comments.

\subsection{JSON Services Used}\label{json-services-used-31}

Screenlets in Liferay Screens call JSON web services in the portal. This
Screenlet calls the following services and methods.

\noindent\hrulefill

\begin{longtable}[]{@{}
  >{\raggedright\arraybackslash}p{(\columnwidth - 4\tabcolsep) * \real{0.3889}}
  >{\raggedright\arraybackslash}p{(\columnwidth - 4\tabcolsep) * \real{0.3333}}
  >{\raggedright\arraybackslash}p{(\columnwidth - 4\tabcolsep) * \real{0.2778}}@{}}
\toprule\noalign{}
\begin{minipage}[b]{\linewidth}\raggedright
Service
\end{minipage} & \begin{minipage}[b]{\linewidth}\raggedright
Method
\end{minipage} & \begin{minipage}[b]{\linewidth}\raggedright
Notes
\end{minipage} \\
\midrule\noalign{}
\endhead
\bottomrule\noalign{}
\endlastfoot
\texttt{ScreenscommentService} (Screens compatibility plugin) &
\texttt{getCommentsWithClassName} & \\
\texttt{ScreenscommentService} (Screens compatibility plugin) &
\texttt{getCommentsCount} & \\
\end{longtable}

\noindent\hrulefill

\subsection{Module}\label{module-33}

\begin{itemize}
\tightlist
\item
  None
\end{itemize}

\subsection{Themes}\label{themes-11}

\begin{itemize}
\tightlist
\item
  Default
\end{itemize}

The Default Theme uses an
\href{https://developer.apple.com/reference/uikit/uitableview}{iOS
\texttt{UITableView}} to show an asset's comments. Other Themes may use
a different component, such as
\href{https://developer.apple.com/reference/uikit/uicollectionview}{iOS's
\texttt{UICollectionView}} or others, to show the items.

\begin{figure}
\centering
\includegraphics{./images/screens-ios-commentlist.png}
\caption{Comment List Screenlet using the Default Theme.}
\end{figure}

\subsection{Offline}\label{offline-31}

This Screenlet supports offline mode so it can function without a
network connection. For more information on how offline mode works, see
the
\href{/docs/7-1/tutorials/-/knowledge_base/t/architecture-of-offline-mode-in-liferay-screens}{tutorial
on its architecture}. Here are the offline mode policies that you can
use with this Screenlet:

\noindent\hrulefill

Policy \textbar{} What happens \textbar{} When to use \textbar{}
\texttt{remote-only} \textbar{} The Screenlet loads the list from the
Liferay instance. If a connection issue occurs, the Screenlet uses the
delegate to notify the developer about the error. If the Screenlet
successfully loads the list, it stores the data in the local cache for
later use. \textbar{} Use this policy when you always need to show
updated data, and show nothing when there's no connection. \textbar{}
\texttt{cache-only} \textbar{} The Screenlet loads the list from the
local cache. If the list isn't there, the Screenlet uses the delegate to
notify the developer about the error. \textbar{} Use this policy when
you always need to show local data, without retrieving remote
information under any circumstance. \textbar{} \texttt{remote-first}
\textbar{} The Screenlet loads the list from the Liferay instance. If
this succeeds, the Screenlet shows the list to the user and stores it in
the local cache for later use. If a connection issue occurs, the
Screenlet retrieves the list from the local cache. If the list doesn't
exist there, the Screenlet uses the delegate to notify the developer
about the error. \textbar{} Use this policy to show the most recent
version of the data when connected, but show a possibly outdated version
when there's no connection. \textbar{} \texttt{cache-first} \textbar{}
The Screenlet loads the list from the local cache. If the list isn't
there, the Screenlet requests it from the Liferay instance and notifies
the developer about any errors that occur (including connectivity
errors). \textbar{} Use this policy to save bandwidth and loading time
in case you have local (but possibly outdated) data. \textbar{}

\noindent\hrulefill

\subsection{Required Attributes}\label{required-attributes-28}

\begin{itemize}
\tightlist
\item
  \texttt{className}
\item
  \texttt{classPK}
\end{itemize}

\subsection{Attributes}\label{attributes-33}

\noindent\hrulefill

Attribute \textbar{} Data type \textbar{} Explanation \textbar{}
\texttt{className} \textbar{} \texttt{string} \textbar{} The asset's
fully qualified class name. For example, a blog entry's
\texttt{className} is \texttt{com.liferay.blogs.model.BlogsEntry}. The
\texttt{className} and \texttt{classPK} attributes are required to
instantiate the Screenlet. \textbar{} \texttt{classPK} \textbar{}
\texttt{number} \textbar{} The asset's unique identifier. The
\texttt{className} and \texttt{classPK} attributes are required to
instantiate the Screenlet. \textbar{} \texttt{offlinePolicy} \textbar{}
\texttt{string} \textbar{} The offline mode setting. The default is
\texttt{remote-first}. See
\href{/docs/7-1/reference/-/knowledge_base/r/comment-list-screenlet-for-ios\#offline}{the
Offline section} for details. \textbar{} \texttt{editable} \textbar{}
\texttt{boolean} \textbar{} Whether the user can edit the comment.
\textbar{} \texttt{autoLoad} \textbar{} \texttt{boolean} \textbar{}
Whether the list should automatically load when the Screenlet appears in
the app's UI. The default value is \texttt{true}. \textbar{}
\texttt{refreshControl} \textbar{} \texttt{boolean} \textbar{} Defines
whether a standard
\href{https://developer.apple.com/library/ios/documentation/UIKit/Reference/UIRefreshControl_class/}{iOS
\texttt{UIRefreshControl}} is shown when the user does the pull to
refresh gesture. The default value is \texttt{true}. \textbar{}
\texttt{firstPageSize} \textbar{} \texttt{number} \textbar{} The number
of items retrieved from the server for display on the first page. The
default value is \texttt{50}. \textbar{} \texttt{pageSize} \textbar{}
\texttt{number} \textbar{} The number of items retrieved from the server
for display on the second and subsequent pages. The default value is
\texttt{25}. \textbar{} \texttt{obcClassName} \textbar{} \texttt{string}
\textbar{} The name of the
\href{https://docs.liferay.com/dxp/portal/7.1-latest/javadocs/portal-kernel/com/liferay/portal/kernel/util/OrderByComparator.html}{\texttt{OrderByComparator}
class} to use to sort the results. You can only use classes that extend
\texttt{OrderByComparator\textless{}MBMessage\textgreater{}}. If you
don't want to sort the results, you can omit this property. \textbar{}

\noindent\hrulefill

\subsection{Methods}\label{methods-21}

\noindent\hrulefill

Method \textbar{} Return \textbar{} Explanation \textbar{}
\texttt{loadList()} \textbar{} \texttt{boolean} \textbar{} Starts the
request to load the list. This list is shown when the response is
received. Returns \texttt{true} if the request is sent. \textbar{}

\noindent\hrulefill

\subsection{Delegate}\label{delegate-11}

Comment List Screenlet delegates some events to an object that conforms
to the \texttt{ComentListScreenletDelegate} protocol. This protocol lets
you implement the following methods:

\begin{itemize}
\item
  \texttt{-\ screenlet:onListResponseComments:}: Called when the
  Screenlet receives the comments.
\item
  \texttt{-\ screenlet:onCommentListError:}: Called when an error occurs
  in the process. The \texttt{NSError} object describes the error.
\item
  \texttt{-\ screenlet:onSelectedComment:}: Called when a comment is
  selected.
\item
  \texttt{-\ screenlet:onDeletedComment:}: Called when a comment is
  deleted.
\item
  \texttt{-\ screenlet:onCommentDelete:}: Called when the Screenlet
  prepares a comment for deletion.
\item
  \texttt{-\ screenlet:onUpdatedComment:}: Called when a comment is
  updated.
\item
  \texttt{-\ screenlet:onCommentUpdate:}: Called when the Screenlet
  prepares a comment for update.
\end{itemize}

\section{Comment Display Screenlet for
iOS}\label{comment-display-screenlet-for-ios}

\subsection{Requirements}\label{requirements-34}

\begin{itemize}
\tightlist
\item
  Xcode 9.3 or above
\item
  iOS 11 SDK
\item
  Liferay Portal 6.2 CE/EE, Liferay CE Portal 7.0/7.1, Liferay DXP
\item
  Liferay Screens Compatibility app
  (\href{http://www.liferay.com/marketplace/-/mp/application/54365664}{CE}
  or
  \href{http://www.liferay.com/marketplace/-/mp/application/54369726}{EE/DXP}).
  This app is preinstalled in Liferay CE Portal 7.0/7.1 and Liferay DXP.
\end{itemize}

\subsection{Compatibility}\label{compatibility-34}

\begin{itemize}
\tightlist
\item
  iOS 9 and above
\end{itemize}

\subsection{Xamarin Requirements}\label{xamarin-requirements-34}

\begin{itemize}
\tightlist
\item
  Visual Studio 7.2
\item
  Mono .NET framework 5.4.1.6
\end{itemize}

\subsection{Features}\label{features-34}

Comment Display Screenlet can show one comment of an asset in a Liferay
instance. It also lets the user update or delete the comment.

\subsection{JSON Services Used}\label{json-services-used-32}

Screenlets in Liferay Screens call JSON web services in the portal. This
Screenlet calls the following services and methods.

\noindent\hrulefill

\begin{longtable}[]{@{}
  >{\raggedright\arraybackslash}p{(\columnwidth - 4\tabcolsep) * \real{0.3889}}
  >{\raggedright\arraybackslash}p{(\columnwidth - 4\tabcolsep) * \real{0.3333}}
  >{\raggedright\arraybackslash}p{(\columnwidth - 4\tabcolsep) * \real{0.2778}}@{}}
\toprule\noalign{}
\begin{minipage}[b]{\linewidth}\raggedright
Service
\end{minipage} & \begin{minipage}[b]{\linewidth}\raggedright
Method
\end{minipage} & \begin{minipage}[b]{\linewidth}\raggedright
Notes
\end{minipage} \\
\midrule\noalign{}
\endhead
\bottomrule\noalign{}
\endlastfoot
\texttt{ScreenscommentService} (Screens compatibility plugin) &
\texttt{getCommentWithCommentId} & \\
\texttt{ScreenscommentService} (Screens compatibility plugin) &
\texttt{updateComment} & \\
\texttt{CommentmanagerjsonwsService} & \texttt{deleteComment} & \\
\end{longtable}

\noindent\hrulefill

\subsection{Module}\label{module-34}

\begin{itemize}
\tightlist
\item
  None
\end{itemize}

\subsection{Themes}\label{themes-12}

\begin{itemize}
\tightlist
\item
  Default
\end{itemize}

The Default Theme uses
\href{/docs/7-1/reference/-/knowledge_base/r/userportraitscreenlet-for-ios}{User
Portrait Screenlet} and iOS \texttt{UILabel} elements to show an asset's
comment. Other Themes may use different components to show the comment.

\begin{figure}
\centering
\includegraphics{./images/screens-ios-commentdisplay.png}
\caption{Comment Display Screenlet using the Default Theme.}
\end{figure}

\subsection{Offline}\label{offline-32}

This Screenlet supports offline mode so it can function without a
network connection. For more information on how offline mode works, see
the
\href{/docs/7-1/tutorials/-/knowledge_base/t/architecture-of-offline-mode-in-liferay-screens}{tutorial
on its architecture}. This Screenlet supports the \texttt{remote-only},
\texttt{cache-only}, \texttt{remote-first}, and \texttt{cache-first}
offline mode policies.

These policies take the following actions when loading a comment from a
Liferay instance:

\noindent\hrulefill

Policy \textbar{} What happens \textbar{} When to use \textbar{}
\texttt{remote-only} \textbar{} The Screenlet loads the data from the
Liferay instance. If a connection issue occurs, the Screenlet uses the
listener to notify the developer about the error. If the Screenlet
successfully loads the data, it stores it in the local cache for later
use. \textbar{} Use this policy when you always need to show updated
data, and show nothing when there's no connection. \textbar{}
\texttt{cache-only} \textbar{} The Screenlet loads the data from the
local cache. If the data isn't there, the Screenlet uses the listener to
notify the developer about the error. \textbar{} Use this policy when
you always need to show local data, without retrieving remote
information under any circumstance. \textbar{} \texttt{remote-first}
\textbar{} The Screenlet loads the data from the Liferay instance. If
this succeeds, the Screenlet shows the data to the user and stores it in
the local cache for later use. If a connection issue occurs, the
Screenlet retrieves the data from the local cache. If the data doesn't
exist there, the Screenlet uses the listener to notify the developer
about the error. \textbar{} Use this policy to show the most recent
version of the data when connected, but show an outdated version when
there's no connection. \textbar{} \texttt{cache-first} \textbar{} The
Screenlet loads the data from the local cache. If the data isn't there,
the Screenlet requests it from the Liferay instance and notifies the
developer about any errors that occur (including connectivity errors).
\textbar{} Use this policy to save bandwidth and loading time in case
you have local (but probably outdated) data. \textbar{}

\noindent\hrulefill

These policies take the following actions when updating or deleting a
comment in a Liferay instance:

\noindent\hrulefill

Policy \textbar{} What happens \textbar{} When to use \textbar{}
\texttt{remote-only} \textbar{} The Screenlet sends the data to the
Liferay instance. If a connection issue occurs, the Screenlet uses the
delegate to notify the developer about the error, but it also discards
the data. \textbar{} Use this policy to make sure the Liferay instance
always has the most recent version of the data. \textbar{}
\texttt{cache-only} \textbar{} The Screenlet stores the data in the
local cache. \textbar{} Use this policy when you need to save the data
locally, but don't want to update it in the Liferay instance. \textbar{}
\texttt{remote-first} \textbar{} The Screenlet sends the data to the
Liferay instance. If this succeeds, it also stores the data in the local
cache for later use. If a connection issue occurs, the Screenlet stores
the data in the local cache and sends it to the Liferay instance when
the connection is re-established. \textbar{} Use this policy when you
need to make sure the Screenlet sends the data to the Liferay instance
as soon as the connection is restored. \textbar{} \texttt{cache-first}
\textbar{} The Screenlet stores the data in the local cache and then
attempts to send it to the Liferay instance. If a connection issue
occurs, the Screenlet sends the data to the Liferay instance when the
connection is re-established. \textbar{} Use this policy when you need
to make sure the Screenlet sends the data to the Liferay instance as
soon as the connection is restored. Compared to \texttt{remote-first},
this policy always stores the data in the cache. The
\texttt{remote-first} policy only stores the data in the event of a
network error. \textbar{}

\noindent\hrulefill

\subsection{Required Attributes}\label{required-attributes-29}

\begin{itemize}
\tightlist
\item
  \texttt{commentId}
\end{itemize}

\subsection{Attributes}\label{attributes-34}

\noindent\hrulefill

Attribute \textbar{} Data type \textbar{} Explanation \textbar{}
\texttt{commentId} \textbar{} \texttt{number} \textbar{} The primary key
of the comment to display. \textbar{} \texttt{autoLoad} \textbar{}
\texttt{boolean} \textbar{} Whether the list should automatically load
when the Screenlet appears in the app's UI. The default value is
\texttt{true}. \textbar{} \texttt{editable} \textbar{} \texttt{boolean}
\textbar{} Whether the user can edit the comment. \textbar{}
\texttt{offlinePolicy} \textbar{} \texttt{string} \textbar{} The offline
mode setting. The default is \texttt{remote-first}. See
\href{/docs/7-1/reference/-/knowledge_base/r/comment-display-screenlet-for-ios\#offline}{the
Offline section} for details. \textbar{}

\noindent\hrulefill

\subsection{Methods}\label{methods-22}

\noindent\hrulefill

Method \textbar{} Return \textbar{} Explanation \textbar{}
\texttt{load()} \textbar{} none \textbar{} Starts the request to load
the comment. \textbar{}

\noindent\hrulefill

\subsection{Delegate}\label{delegate-12}

Comment Display Screenlet delegates some events to an object that
conforms to the \texttt{CommentDisplayScreenletDelegate} protocol. This
protocol lets you implement the following methods:

\begin{itemize}
\item
  \texttt{-\ screenlet:onCommentLoaded:}: Called when the Screenlet
  loads the comment.
\item
  \texttt{-\ screenlet:onLoadCommentError:}: Called when an error occurs
  in the process. The \texttt{NSError} object describes the error.
\item
  \texttt{-\ screenlet:onSelectedComment:}: Called when a comment is
  selected.
\item
  \texttt{-\ screenlet:onDeletedComment:}: Called when a comment is
  deleted.
\item
  \texttt{-\ screenlet:onCommentDelete:}: Called when the Screenlet
  prepares the comment for deletion.
\item
  \texttt{-\ screenlet:onUpdatedComment:}: Called when a comment is
  updated.
\item
  \texttt{-\ screenlet:onCommentUpdate:}: Called when the Screenlet
  prepares the comment for update.
\end{itemize}

\section{Comment Add Screenlet for
iOS}\label{comment-add-screenlet-for-ios}

\subsection{Requirements}\label{requirements-35}

\begin{itemize}
\tightlist
\item
  Xcode 9.3 or above
\item
  iOS 11 SDK
\item
  Liferay Portal 6.2 CE/EE, Liferay CE Portal 7.0/7.1, Liferay DXP
\item
  Liferay Screens Compatibility app
  (\href{http://www.liferay.com/marketplace/-/mp/application/54365664}{CE}
  or
  \href{http://www.liferay.com/marketplace/-/mp/application/54369726}{EE/DXP}).
  This app is preinstalled in Liferay CE Portal 7.0/7.1 and Liferay DXP.
\end{itemize}

\subsection{Compatibility}\label{compatibility-35}

\begin{itemize}
\tightlist
\item
  iOS 9 and above
\end{itemize}

\subsection{Xamarin Requirements}\label{xamarin-requirements-35}

\begin{itemize}
\tightlist
\item
  Visual Studio 7.2
\item
  Mono .NET framework 5.4.1.6
\end{itemize}

\subsection{Features}\label{features-35}

Comment Add Screenlet can add a comment to an asset in a Liferay
instance.

\subsection{JSON Services Used}\label{json-services-used-33}

Screenlets in Liferay Screens call JSON web services in the portal. This
Screenlet calls the following services and methods.

\noindent\hrulefill

\begin{longtable}[]{@{}
  >{\raggedright\arraybackslash}p{(\columnwidth - 4\tabcolsep) * \real{0.3889}}
  >{\raggedright\arraybackslash}p{(\columnwidth - 4\tabcolsep) * \real{0.3333}}
  >{\raggedright\arraybackslash}p{(\columnwidth - 4\tabcolsep) * \real{0.2778}}@{}}
\toprule\noalign{}
\begin{minipage}[b]{\linewidth}\raggedright
Service
\end{minipage} & \begin{minipage}[b]{\linewidth}\raggedright
Method
\end{minipage} & \begin{minipage}[b]{\linewidth}\raggedright
Notes
\end{minipage} \\
\midrule\noalign{}
\endhead
\bottomrule\noalign{}
\endlastfoot
\texttt{ScreenscommentService} (Screens compatibility plugin) &
\texttt{addComment} & \\
\end{longtable}

\noindent\hrulefill

\subsection{Module}\label{module-35}

\begin{itemize}
\tightlist
\item
  None
\end{itemize}

\subsection{Themes}\label{themes-13}

\begin{itemize}
\tightlist
\item
  Default
\end{itemize}

The Default Theme uses the iOS elements \texttt{UITextField} and
\texttt{UIButton} to add a comment to an asset. Other Themes may use
other components to show the comment.

\begin{figure}
\centering
\includegraphics{./images/screens-ios-commentadd.png}
\caption{Comment Add Screenlet using the Default Theme.}
\end{figure}

\subsection{Offline}\label{offline-33}

This Screenlet supports offline mode so it can function without a
network connection. For more information on how offline mode works, see
the
\href{/docs/7-1/tutorials/-/knowledge_base/t/architecture-of-offline-mode-in-liferay-screens}{tutorial
on its architecture}. Here are the offline mode policies that you can
use with this Screenlet:

\noindent\hrulefill

Policy \textbar{} What happens \textbar{} When to use \textbar{}
\texttt{remote-only} \textbar{} The Screenlet sends the data to the
Liferay instance. If a connection issue occurs, the Screenlet uses the
listener to notify the developer about the error. If the Screenlet
successfully sends the data, it also stores it in the local cache.
\textbar{} Use this policy when you always need to send updated data,
and send nothing when there's no connection. \textbar{}
\texttt{cache-only} \textbar{} The Screenlet sends the data to the local
cache. If an error occurs, the Screenlet uses the listener to notify the
developer. \textbar{} Use this policy when you always need to store
local data without sending remote information under any circumstance.
\textbar{} \texttt{remote-first} \textbar{} The Screenlet sends the data
to the Liferay instance. If this succeeds, the Screenlet also stores the
data in the local cache. If a connection issue occurs, the Screenlet
stores the data to the local cache and sends it to the Liferay instance
when the connection is restored. If an error occurs, the Screenlet uses
the listener to notify the developer. \textbar{} Use this policy to send
the most recent version of the data when connected, and store the data
for later synchronization when there's no connection. \textbar{}
\texttt{cache-first} \textbar{} The Screenlet sends the data to the
local cache, then sends it to the Liferay instance. If sending the data
to the Liferay instance fails, the Screenlet still stores the data
locally and then notifies the developer about any errors that occur
(including connectivity errors). \textbar{} Use this policy to save
bandwidth and store local (but possibly outdated) data. \textbar{}

\noindent\hrulefill

\subsection{Required Attributes}\label{required-attributes-30}

\begin{itemize}
\tightlist
\item
  \texttt{className}
\item
  \texttt{classPK}
\end{itemize}

\subsection{Attributes}\label{attributes-35}

\noindent\hrulefill

Attribute \textbar{} Data type \textbar{} Explanation \textbar{}
\texttt{className} \textbar{} \texttt{string} \textbar{} The asset's
fully qualified class name. For example, a blog entry's
\texttt{className} is \texttt{com.liferay.blogs.model.BlogsEntry}. The
\texttt{className} and \texttt{classPK} attributes are required to
instantiate the Screenlet. \textbar{} \texttt{classPK} \textbar{}
\texttt{number} \textbar{} The asset's unique identifier. The
\texttt{className} and \texttt{classPK} attributes are required to
instantiate the Screenlet. \textbar{} \texttt{offlinePolicy} \textbar{}
\texttt{string} \textbar{} The offline mode setting. The default value
is \texttt{remote-first}. See
\href{/docs/7-1/reference/-/knowledge_base/r/comment-add-screenlet-for-ios\#offline}{the
Offline section} for details. \textbar{}

\noindent\hrulefill

\subsection{Delegate}\label{delegate-13}

Comment Add Screenlet delegates some events to an object that conforms
to the \texttt{CommentAddScreenletDelegate} protocol. This protocol lets
you implement the following methods:

\begin{itemize}
\item
  \texttt{-\ screenlet:onCommentAdded:}: Called when the Screenlet adds
  a comment.
\item
  \texttt{-\ screenlet:onAddCommentError:}: Called when an error occurs
  while adding a comment. The \texttt{NSError} object describes the
  error.
\item
  \texttt{-\ screenlet:onCommentUpdated:}: Called when the Screenlet
  prepares a comment for update.
\item
  \texttt{-\ screenlet:onUpdateCommentError:}: Called when an error
  occurs while updating a comment. The \texttt{NSError} object describes
  the error.
\end{itemize}

\section{Asset Display Screenlet for
iOS}\label{asset-display-screenlet-for-ios}

\subsection{Requirements}\label{requirements-36}

\begin{itemize}
\tightlist
\item
  Xcode 9.3 or above
\item
  iOS 11 SDK
\item
  Liferay Portal 6.2 CE/EE, Liferay CE Portal 7.0/7.1, Liferay DXP
\item
  Liferay Screens Compatibility app
  (\href{http://www.liferay.com/marketplace/-/mp/application/54365664}{CE}
  or
  \href{http://www.liferay.com/marketplace/-/mp/application/54369726}{EE/DXP}).
  This app is preinstalled in Liferay CE Portal 7.0/7.1 and Liferay DXP.
\end{itemize}

\subsection{Compatibility}\label{compatibility-36}

\begin{itemize}
\tightlist
\item
  iOS 9 and above
\end{itemize}

\subsection{Xamarin Requirements}\label{xamarin-requirements-36}

\begin{itemize}
\tightlist
\item
  Visual Studio 7.2
\item
  Mono .NET framework 5.4.1.6
\end{itemize}

\subsection{Features}\label{features-36}

Asset Display Screenlet can display an asset from a Liferay instance.
The Screenlet can currently display Documents and Media files
(\texttt{DLFileEntry} images, videos, audio files, and PDFs), blogs
entries (\texttt{BlogsEntry}) and web content articles
(\texttt{WebContent}).

Asset Display Screenlet can also display your custom asset types. See
\href{/docs/7-1/reference/-/knowledge_base/r/asset-display-screenlet-for-ios\#delegate}{the
delegate section of this document} for details.

\subsection{JSON Services Used}\label{json-services-used-34}

Screenlets in Liferay Screens call JSON web services in the portal. This
Screenlet calls the following services and methods.

\noindent\hrulefill

\begin{longtable}[]{@{}
  >{\raggedright\arraybackslash}p{(\columnwidth - 4\tabcolsep) * \real{0.3889}}
  >{\raggedright\arraybackslash}p{(\columnwidth - 4\tabcolsep) * \real{0.3333}}
  >{\raggedright\arraybackslash}p{(\columnwidth - 4\tabcolsep) * \real{0.2778}}@{}}
\toprule\noalign{}
\begin{minipage}[b]{\linewidth}\raggedright
Service
\end{minipage} & \begin{minipage}[b]{\linewidth}\raggedright
Method
\end{minipage} & \begin{minipage}[b]{\linewidth}\raggedright
Notes
\end{minipage} \\
\midrule\noalign{}
\endhead
\bottomrule\noalign{}
\endlastfoot
\texttt{ScreensassetentryService} (Screens compatibility plugin) &
\texttt{getAssetEntry} & With \texttt{entryId} \\
\texttt{ScreensassetentryService} (Screens compatibility plugin) &
\texttt{getAssetEntry} & With \texttt{classPK} and \texttt{className} \\
\texttt{ScreensassetentryService} (Screens compatibility plugin) &
\texttt{getAssetEntries} & With \texttt{entryQuery} \\
\texttt{ScreensassetentryService} (Screens compatibility plugin) &
\texttt{getAssetEntries} & With \texttt{companyId}, \texttt{groupId},
and \texttt{portletItemName} \\
\end{longtable}

\noindent\hrulefill

\subsection{Module}\label{module-36}

\begin{itemize}
\tightlist
\item
  None
\end{itemize}

\subsection{Themes}\label{themes-14}

\begin{itemize}
\tightlist
\item
  Default
\end{itemize}

The Default Theme uses different UI elements to show each asset type.
For example, it displays images with \texttt{UIImageView}, and blogs
with \texttt{UILabel}.

This Screenlet can also render other Screenlets:

\begin{itemize}
\tightlist
\item
  Images: Image Display Screenlet
\item
  Videos: Video Display Screenlet
\item
  Audio: Audio Display Screenlet
\item
  PDFs: PDF Display Screenlet
\item
  Blog entries: Blogs Entry Display Screenlet
\item
  Web content: Web Content Display Screenlet
\end{itemize}

These Screenlets can also be used alone without Asset Display Screenlet.

\begin{figure}
\centering
\includegraphics{./images/screens-ios-assetdisplay.png}
\caption{Asset Display Screenlet using the Default Theme.}
\end{figure}

\subsection{Offline}\label{offline-34}

This Screenlet supports offline mode so it can function without a
network connection. For more information on how offline mode works, see
the
\href{/docs/7-1/tutorials/-/knowledge_base/t/architecture-of-offline-mode-in-liferay-screens}{tutorial
on its architecture}. Here are the offline mode policies that you can
use with this Screenlet:

\noindent\hrulefill

Policy \textbar{} What happens \textbar{} When to use \textbar{}
\texttt{remote-only} \textbar{} The Screenlet loads the data from the
Liferay instance. If a connection issue occurs, the Screenlet uses the
listener to notify the developer about the error. If the Screenlet
successfully loads the data, it stores it in the local cache for later
use. \textbar{} Use this policy when you always need to show updated
data, and show nothing when there's no connection. \textbar{}
\texttt{cache-only} \textbar{} The Screenlet loads the data from the
local cache. If the data isn't there, the Screenlet uses the listener to
notify the developer about the error. \textbar{} Use this policy when
you always need to show local data, without retrieving remote
information under any circumstance. \textbar{} \texttt{remote-first}
\textbar{} The Screenlet loads the data from the Liferay instance. If
this succeeds, the Screenlet shows the data to the user and stores it in
the local cache for later use. If a connection issue occurs, the
Screenlet retrieves the data from the local cache. If the data doesn't
exist there, the Screenlet uses the listener to notify the developer
about the error. \textbar{} Use this policy to show the most recent
version of the data when connected, but show an outdated version when
there's no connection. \textbar{} \texttt{cache-first} \textbar{} The
Screenlet loads the data from the local cache. If the data isn't there,
the Screenlet requests it from the Liferay instance and notifies the
developer about any errors that occur (including connectivity errors).
\textbar{} Use this policy to save bandwidth and loading time in case
you have local (but probably outdated) data. \textbar{}

\noindent\hrulefill

\subsection{Required Attributes}\label{required-attributes-31}

\begin{itemize}
\tightlist
\item
  \texttt{assetEntryId}
\end{itemize}

Instead of \texttt{assetEntryId}, you can use both of these attributes:

\begin{itemize}
\tightlist
\item
  \texttt{className}
\item
  \texttt{classPK}
\end{itemize}

If you don't use the above attributes, you must use this attribute:

\begin{itemize}
\tightlist
\item
  \texttt{portletItemName}
\end{itemize}

\subsection{Attributes}\label{attributes-36}

\noindent\hrulefill

Attribute \textbar{} Data type \textbar{} Explanation \textbar{}
\texttt{assetEntryId} \textbar{} \texttt{number} \textbar{} The primary
key of the asset. \textbar{} \texttt{className} \textbar{}
\texttt{string} \textbar{} The asset's fully qualified class name. For
example, a blog entry's \texttt{className} is
\texttt{com.liferay.blogs.model.BlogsEntry}. The \texttt{className} and
\texttt{classPK} attributes are required to instantiate the Screenlet.
\textbar{} \texttt{classPK} \textbar{} \texttt{number} \textbar{} The
asset's unique identifier. The \texttt{className} and \texttt{classPK}
attributes are required to instantiate the Screenlet. \textbar{}
\texttt{portletItemName} \textbar{} \texttt{string} \textbar{} The name
of the
\href{/docs/7-1/user/-/knowledge_base/u/configuration-templates}{configuration
template} you used in the Asset Publisher. To use this feature, add an
Asset Publisher to one of your site's pages (it may be a hidden page),
configure the Asset Publisher's filter (in \emph{Configuration} →
\emph{Setup} → \emph{Asset Selection}), and then use the Asset
Publisher's \emph{Configuration Templates} option to save this
configuration with a name. Use this name as this attribute's value. If
there is more than one asset in the configuration, the Screenlet
displays only the first one. \textbar{} \texttt{assetEntry} \textbar{}
\texttt{Asset} \textbar{} The \texttt{Asset} object to display, selected
from a list of assets. Note that if you use this attribute, the
Screenlet doesn't need to call the server. \textbar{} \texttt{autoLoad}
\textbar{} \texttt{boolean} \textbar{} Whether the asset automatically
loads when the Screenlet appears in the app's UI. The default value is
\texttt{true}. \textbar{} \texttt{offlinePolicy} \textbar{}
\texttt{string} \textbar{} The offline mode setting. The default value
is \texttt{remote-first}. See
\href{/docs/7-1/reference/-/knowledge_base/r/asset-display-screenlet-for-ios\#offline}{the
Offline section} for details. \textbar{}

\noindent\hrulefill

\subsection{Delegate}\label{delegate-14}

Asset Display Screenlet delegates some events to an object that conforms
to the \texttt{AssetDisplayScreenletDelegate} protocol. This protocol
lets you implement the following methods:

\begin{itemize}
\item
  \texttt{-\ screenlet:onAssetResponse:}: Called when the Screenlet
  receives the asset.
\item
  \texttt{-\ screenlet:onAssetError:}: Called when an error occurs in
  the process. The \texttt{NSError} object describes the error.
\item
  \texttt{-\ screenlet:onConfigureScreenlet:}: Called when the child
  Screenlet (the Screenlet rendered inside Asset Display Screenlet) has
  been successfully initialized. Use this method to configure or
  customize it. The example implementation here sets the child Blogs
  Entry Display Screenlet's background color to gray:

\begin{verbatim}
  func screenlet(screenlet: AssetDisplayScreenlet, onConfigureScreenlet, 
      childScreenlet: BaseScreenlet?, onAsset asset: Asset) {
          if childScreenlet is BlogsEntryDisplayScreenlet {
              childScreenlet?.screenletView?.backgroundColor = UIColor.grayColor()
          }
  }
\end{verbatim}
\item
  \texttt{-\ screenlet:onAsset:}: Called to render a custom asset. The
  following example implementation renders a portal user
  (\texttt{User}). If the asset is a user, this method instantiates a
  custom \texttt{UserProfileView} to render that user:

\begin{verbatim}
  public func screenlet(screenlet: AssetDisplayScreenlet, onAsset asset: Asset) -> UIView? {
      if let type = asset.attributes["object"]?.allKeys.first as? String {
          if type == "user" {
              let view = NSBundle.mainBundle().loadNibNamed("UserProfileView", owner: self, 
                  options: nil)![0] as? UserProfileView

              view?.user = User(attributes: asset.attributes)

              return view
          }
      }
      return nil
  }
\end{verbatim}
\end{itemize}

\section{Blogs Entry Display Screenlet for
iOS}\label{blogs-entry-display-screenlet-for-ios}

\subsection{Requirements}\label{requirements-37}

\begin{itemize}
\tightlist
\item
  Xcode 9.3 or above
\item
  iOS 11 SDK
\item
  Liferay Portal 6.2 CE/EE, Liferay CE Portal 7.0/7.1, Liferay DXP
\item
  Liferay Screens Compatibility app
  (\href{http://www.liferay.com/marketplace/-/mp/application/54365664}{CE}
  or
  \href{http://www.liferay.com/marketplace/-/mp/application/54369726}{EE/DXP}).
  This app is preinstalled in Liferay CE Portal 7.0/7.1 and Liferay DXP.
\end{itemize}

\subsection{Compatibility}\label{compatibility-37}

\begin{itemize}
\tightlist
\item
  iOS 9 and above
\end{itemize}

\subsection{Xamarin Requirements}\label{xamarin-requirements-37}

\begin{itemize}
\tightlist
\item
  Visual Studio 7.2
\item
  Mono .NET framework 5.4.1.6
\end{itemize}

\subsection{Features}\label{features-37}

Blogs Entry Display Screenlet displays a single blog entry. Image
Display Screenlet renders any header image the blogs entry may have.

\subsection{JSON Services Used}\label{json-services-used-35}

Screenlets in Liferay Screens call JSON web services in the portal. This
Screenlet calls the following services and methods.

\noindent\hrulefill

\begin{longtable}[]{@{}
  >{\raggedright\arraybackslash}p{(\columnwidth - 4\tabcolsep) * \real{0.3889}}
  >{\raggedright\arraybackslash}p{(\columnwidth - 4\tabcolsep) * \real{0.3333}}
  >{\raggedright\arraybackslash}p{(\columnwidth - 4\tabcolsep) * \real{0.2778}}@{}}
\toprule\noalign{}
\begin{minipage}[b]{\linewidth}\raggedright
Service
\end{minipage} & \begin{minipage}[b]{\linewidth}\raggedright
Method
\end{minipage} & \begin{minipage}[b]{\linewidth}\raggedright
Notes
\end{minipage} \\
\midrule\noalign{}
\endhead
\bottomrule\noalign{}
\endlastfoot
\texttt{ScreensassetentryService} (Screens compatibility plugin) &
\texttt{getAssetEntry} & With \texttt{entryId} \\
\texttt{ScreensassetentryService} (Screens compatibility plugin) &
\texttt{getAssetEntry} & With \texttt{classPK} and \texttt{className} \\
\texttt{ScreensassetentryService} (Screens compatibility plugin) &
\texttt{getAssetEntries} & With \texttt{entryQuery} \\
\texttt{ScreensassetentryService} (Screens compatibility plugin) &
\texttt{getAssetEntries} & With \texttt{companyId}, \texttt{groupId},
and \texttt{portletItemName} \\
\end{longtable}

\noindent\hrulefill

\subsection{Module}\label{module-37}

\begin{itemize}
\tightlist
\item
  None
\end{itemize}

\subsection{Themes}\label{themes-15}

\begin{itemize}
\tightlist
\item
  Default
\end{itemize}

The Default Theme can use different components to show a blogs entry
(\texttt{BlogsEntry}). For example, it uses \texttt{UILabel} to show a
blog's text, and
\href{/docs/7-1/reference/-/knowledge_base/r/userportraitscreenlet-for-ios}{User
Portrait Screenlet} to show the profile picture of the Liferay user who
posted it. Note that other Themes may use different components.

\begin{figure}
\centering
\includegraphics{./images/screens-ios-blogsentrydisplay.png}
\caption{Blogs Entry Display Screenlet using the Default Theme.}
\end{figure}

\subsection{Offline}\label{offline-35}

This Screenlet supports offline mode so it can function without a
network connection. For more information on how offline mode works, see
the
\href{/docs/7-1/tutorials/-/knowledge_base/t/architecture-of-offline-mode-in-liferay-screens}{tutorial
on its architecture}. Here are the offline mode policies that you can
use with this Screenlet:

\noindent\hrulefill

Policy \textbar{} What happens \textbar{} When to use \textbar{}
\texttt{remote-only} \textbar{} The Screenlet loads the data from the
Liferay instance. If a connection issue occurs, the Screenlet uses the
listener to notify the developer about the error. If the Screenlet
successfully loads the data, it stores it in the local cache for later
use. \textbar{} Use this policy when you always need to show updated
data, and show nothing when there's no connection. \textbar{}
\texttt{cache-only} \textbar{} The Screenlet loads the data from the
local cache. If the data isn't there, the Screenlet uses the listener to
notify the developer about the error. \textbar{} Use this policy when
you always need to show local data, without retrieving remote data under
any circumstance. \textbar{} \texttt{remote-first} \textbar{} The
Screenlet loads the data from the Liferay instance. If this succeeds,
the Screenlet shows the data to the user and stores it in the local
cache for later use. If a connection issue occurs, the Screenlet
retrieves the data from the local cache. If the data doesn't exist
there, the Screenlet uses the listener to notify the developer about the
error. \textbar{} Use this policy to show the most recent version of the
data when connected, but show an outdated version when there's no
connection. \textbar{} \texttt{cache-first} \textbar{} The Screenlet
loads the data from the local cache. If the data isn't there, the
Screenlet requests it from the Liferay instance and notifies the
developer about any errors that occur (including connectivity errors).
\textbar{} Use this policy to save bandwidth and loading time in case
you have local (but probably outdated) data. \textbar{}

\noindent\hrulefill

\subsection{Required Attributes}\label{required-attributes-32}

\begin{itemize}
\tightlist
\item
  \texttt{assetEntryId} or \texttt{classPK}
\end{itemize}

\subsection{Attributes}\label{attributes-37}

\noindent\hrulefill

Attribute \textbar{} Data type \textbar{} Explanation \textbar{}
\texttt{assetEntryId} \textbar{} \texttt{number} \textbar{} The primary
key of the blog entry (\texttt{BlogsEntry}). \textbar{} \texttt{classPK}
\textbar{} \texttt{number} \textbar{} The \texttt{BlogsEntry} object's
unique identifier. \textbar{} \texttt{autoLoad} \textbar{}
\texttt{boolean} \textbar{} Whether the blog entry automatically loads
when the Screenlet appears in the app's UI. The default value is
\texttt{true}. \textbar{} \texttt{offlinePolicy} \textbar{}
\texttt{string} \textbar{} The offline mode setting. The default value
is \texttt{remote-first}. See
\href{/docs/7-1/reference/-/knowledge_base/r/blogs-entry-display-screenlet-for-ios\#offline}{the
Offline section} for details. \textbar{}

\noindent\hrulefill

\subsection{Delegate}\label{delegate-15}

Blogs Entry Display Screenlet delegates some events to an object that
conforms to the \texttt{BlogsEntryDisplayScreenletDelegate} protocol.
This protocol lets you implement the following methods:

\begin{itemize}
\item
  \texttt{-\ screenlet:onBlogEntryResponse:}: Called when the Screenlet
  receives the \texttt{BlogsEntry} object.
\item
  \texttt{-\ screenlet:onBlogEntryError:}: Called when an error occurs
  in the process. The \texttt{NSError} object describes the error.
\end{itemize}

\section{Image Display Screenlet for
iOS}\label{image-display-screenlet-for-ios}

\subsection{Requirements}\label{requirements-38}

\begin{itemize}
\tightlist
\item
  Xcode 9.3 or above
\item
  iOS 11 SDK
\item
  Liferay Portal 6.2 CE/EE, Liferay CE Portal 7.0/7.1, Liferay DXP
\item
  Liferay Screens Compatibility app
  (\href{http://www.liferay.com/marketplace/-/mp/application/54365664}{CE}
  or
  \href{http://www.liferay.com/marketplace/-/mp/application/54369726}{EE/DXP}).
  This app is preinstalled in Liferay CE Portal 7.0/7.1 and Liferay DXP.
\end{itemize}

\subsection{Compatibility}\label{compatibility-38}

\begin{itemize}
\tightlist
\item
  iOS 9 and above
\end{itemize}

\subsection{Xamarin Requirements}\label{xamarin-requirements-38}

\begin{itemize}
\tightlist
\item
  Visual Studio 7.2
\item
  Mono .NET framework 5.4.1.6
\end{itemize}

\subsection{Features}\label{features-38}

Image Display Screenlet displays an image file from a Liferay instance's
Documents and Media Library.

\subsection{JSON Services Used}\label{json-services-used-36}

Screenlets in Liferay Screens call JSON web services in the portal. This
Screenlet calls the following services and methods.

\noindent\hrulefill

\begin{longtable}[]{@{}
  >{\raggedright\arraybackslash}p{(\columnwidth - 4\tabcolsep) * \real{0.3889}}
  >{\raggedright\arraybackslash}p{(\columnwidth - 4\tabcolsep) * \real{0.3333}}
  >{\raggedright\arraybackslash}p{(\columnwidth - 4\tabcolsep) * \real{0.2778}}@{}}
\toprule\noalign{}
\begin{minipage}[b]{\linewidth}\raggedright
Service
\end{minipage} & \begin{minipage}[b]{\linewidth}\raggedright
Method
\end{minipage} & \begin{minipage}[b]{\linewidth}\raggedright
Notes
\end{minipage} \\
\midrule\noalign{}
\endhead
\bottomrule\noalign{}
\endlastfoot
\texttt{ScreensassetentryService} (Screens compatibility plugin) &
\texttt{getAssetEntry} & With \texttt{entryId} \\
\texttt{ScreensassetentryService} (Screens compatibility plugin) &
\texttt{getAssetEntry} & With \texttt{classPK} and \texttt{className} \\
\texttt{ScreensassetentryService} (Screens compatibility plugin) &
\texttt{getAssetEntries} & With \texttt{entryQuery} \\
\texttt{ScreensassetentryService} (Screens compatibility plugin) &
\texttt{getAssetEntries} & With \texttt{companyId}, \texttt{groupId},
and \texttt{portletItemName} \\
\end{longtable}

\noindent\hrulefill

\subsection{Module}\label{module-38}

\begin{itemize}
\tightlist
\item
  None
\end{itemize}

\subsection{Themes}\label{themes-16}

\begin{itemize}
\tightlist
\item
  Default
\end{itemize}

The Default Theme uses an iOS \texttt{UIImageView} for displaying the
image.

\begin{figure}
\centering
\includegraphics{./images/screens-ios-imagedisplay.png}
\caption{Image Display Screenlet using the Default Theme.}
\end{figure}

\subsection{Offline}\label{offline-36}

This Screenlet supports offline mode so it can function without a
network connection. For more information on how offline mode works, see
the
\href{/docs/7-1/tutorials/-/knowledge_base/t/architecture-of-offline-mode-in-liferay-screens}{tutorial
on its architecture}. Here are the offline mode policies that you can
use with this Screenlet:

\noindent\hrulefill

Policy \textbar{} What happens \textbar{} When to use \textbar{}
\texttt{remote-only} \textbar{} The Screenlet loads the data from the
Liferay instance. If a connection issue occurs, the Screenlet uses the
listener to notify the developer about the error. If the Screenlet
successfully loads the data, it stores it in the local cache for later
use. \textbar{} Use this policy when you always need to show updated
data, and show nothing when there's no connection. \textbar{}
\texttt{cache-only} \textbar{} The Screenlet loads the data from the
local cache. If the data isn't there, the Screenlet uses the listener to
notify the developer about the error. \textbar{} Use this policy when
you always need to show local data, without retrieving remote
information under any circumstance. \textbar{} \texttt{remote-first}
\textbar{} The Screenlet loads the data from the Liferay instance. If
this succeeds, the Screenlet shows the data to the user and stores it in
the local cache for later use. If a connection issue occurs, the
Screenlet retrieves the data from the local cache. If the data doesn't
exist there, the Screenlet uses the listener to notify the developer
about the error. \textbar{} Use this policy to show the most recent
version of the data when connected, but show an outdated version when
there's no connection. \textbar{} \texttt{cache-first} \textbar{} The
Screenlet loads the data from the local cache. If the data isn't there,
the Screenlet requests it from the Liferay instance and notifies the
developer about any errors that occur (including connectivity errors).
\textbar{} Use this policy to save bandwidth and loading time in case
you have local (but probably outdated) data. \textbar{}

\noindent\hrulefill

\subsection{Required Attributes}\label{required-attributes-33}

\begin{itemize}
\tightlist
\item
  \texttt{assetEntryId}
\end{itemize}

If you don't use \texttt{assetEntryId}, you must use these attributes:

\begin{itemize}
\tightlist
\item
  \texttt{className}
\item
  \texttt{classPK}
\end{itemize}

\subsection{Attributes}\label{attributes-38}

\noindent\hrulefill

Attribute \textbar{} Data type \textbar{} Explanation \textbar{}
\texttt{assetEntryId} \textbar{} \texttt{number} \textbar{} The primary
key of the image. \textbar{} \texttt{className} \textbar{}
\texttt{string} \textbar{} The image's fully qualified class name. Since
files in a Documents and Media Library are \texttt{DLFileEntry} objects,
their \texttt{className} is
\href{https://docs.liferay.com/dxp/portal/7.1-latest/javadocs/portal-kernel/com/liferay/document/library/kernel/model/DLFileEntry.html}{\texttt{com.liferay.document.library.kernel.model.DLFileEntry}}.
The \texttt{className} and \texttt{classPK} attributes are required to
instantiate the Screenlet. \textbar{} \texttt{classPK} \textbar{}
\texttt{number} \textbar{} The image's unique identifier. The
\texttt{className} and \texttt{classPK} attributes are required to
instantiate the Screenlet. \textbar{} \texttt{autoLoad} \textbar{}
\texttt{boolean} \textbar{} Whether the image automatically loads when
the Screenlet appears in the app's UI. The default value is
\texttt{true}. \textbar{} \texttt{offlinePolicy} \textbar{}
\texttt{string} \textbar{} The offline mode setting. The default value
is \texttt{remote-first}. See the
\href{/docs/7-1/reference/-/knowledge_base/r/image-display-screenlet-for-ios\#offline}{Offline
section} for details. \textbar{}

\noindent\hrulefill

\subsection{Delegate}\label{delegate-16}

Because images are files, Image Display Screenlet delegates its events
to an object that conforms to the \texttt{FileDisplayScreenletDelegate}
protocol. This protocol lets you implement the following methods:

\begin{itemize}
\item
  \texttt{-\ screenlet:onFileAssetResponse:}: Called when the Screenlet
  receives the image file.
\item
  \texttt{-\ screenlet:onFileAssetError:}: Called when an error occurs
  in the process. The \texttt{NSError} object describes the error.
\end{itemize}

\section{Video Display Screenlet for
iOS}\label{video-display-screenlet-for-ios}

\subsection{Requirements}\label{requirements-39}

\begin{itemize}
\tightlist
\item
  Xcode 9.3 or above
\item
  iOS 11 SDK
\item
  Liferay Portal 6.2 CE/EE, Liferay CE Portal 7.0/7.1, Liferay DXP
\item
  Liferay Screens Compatibility app
  (\href{http://www.liferay.com/marketplace/-/mp/application/54365664}{CE}
  or
  \href{http://www.liferay.com/marketplace/-/mp/application/54369726}{EE/DXP}).
  This app is preinstalled in Liferay CE Portal 7.0/7.1 and Liferay DXP.
\end{itemize}

\subsection{Compatibility}\label{compatibility-39}

\begin{itemize}
\tightlist
\item
  iOS 9 and above
\end{itemize}

\subsection{Xamarin Requirements}\label{xamarin-requirements-39}

\begin{itemize}
\tightlist
\item
  Visual Studio 7.2
\item
  Mono .NET framework 5.4.1.6
\end{itemize}

\subsection{Features}\label{features-39}

Video Display Screenlet displays a video file from a Liferay instance's
Documents and Media Library.

\subsection{JSON Services Used}\label{json-services-used-37}

Screenlets in Liferay Screens call JSON web services in the portal. This
Screenlet calls the following services and methods.

\noindent\hrulefill

\begin{longtable}[]{@{}
  >{\raggedright\arraybackslash}p{(\columnwidth - 4\tabcolsep) * \real{0.3889}}
  >{\raggedright\arraybackslash}p{(\columnwidth - 4\tabcolsep) * \real{0.3333}}
  >{\raggedright\arraybackslash}p{(\columnwidth - 4\tabcolsep) * \real{0.2778}}@{}}
\toprule\noalign{}
\begin{minipage}[b]{\linewidth}\raggedright
Service
\end{minipage} & \begin{minipage}[b]{\linewidth}\raggedright
Method
\end{minipage} & \begin{minipage}[b]{\linewidth}\raggedright
Notes
\end{minipage} \\
\midrule\noalign{}
\endhead
\bottomrule\noalign{}
\endlastfoot
\texttt{ScreensassetentryService} (Screens compatibility plugin) &
\texttt{getAssetEntry} & With \texttt{entryId} \\
\texttt{ScreensassetentryService} (Screens compatibility plugin) &
\texttt{getAssetEntry} & With \texttt{classPK} and \texttt{className} \\
\texttt{ScreensassetentryService} (Screens compatibility plugin) &
\texttt{getAssetEntries} & With \texttt{entryQuery} \\
\texttt{ScreensassetentryService} (Screens compatibility plugin) &
\texttt{getAssetEntries} & With \texttt{companyId}, \texttt{groupId},
and \texttt{portletItemName} \\
\end{longtable}

\noindent\hrulefill

\subsection{Module}\label{module-39}

\begin{itemize}
\tightlist
\item
  None
\end{itemize}

\subsection{Themes}\label{themes-17}

\begin{itemize}
\tightlist
\item
  Default
\end{itemize}

The Default Theme uses an iOS \texttt{AVPlayerViewController} to display
the video.

\begin{figure}
\centering
\includegraphics{./images/screens-ios-videodisplay.png}
\caption{Video Display Screenlet using the Default Theme.}
\end{figure}

\subsection{Offline}\label{offline-37}

This Screenlet supports offline mode so it can function without a
network connection. For more information on how offline mode works, see
the
\href{/docs/7-1/tutorials/-/knowledge_base/t/architecture-of-offline-mode-in-liferay-screens}{tutorial
on its architecture}. Here are the offline mode policies that you can
use with this Screenlet:

\noindent\hrulefill

Policy \textbar{} What happens \textbar{} When to use \textbar{}
\texttt{remote-only} \textbar{} The Screenlet loads the data from the
Liferay instance. If a connection issue occurs, the Screenlet uses the
listener to notify the developer about the error. If the Screenlet
successfully loads the data, it stores it in the local cache for later
use. \textbar{} Use this policy when you always need to show updated
data, and show nothing when there's no connection. \textbar{}
\texttt{cache-only} \textbar{} The Screenlet loads the data from the
local cache. If the data isn't there, the Screenlet uses the listener to
notify the developer about the error. \textbar{} Use this policy when
you always need to show local data, without retrieving remote
information under any circumstance. \textbar{} \texttt{remote-first}
\textbar{} The Screenlet loads the data from the Liferay instance. If
this succeeds, the Screenlet shows the data to the user and stores it in
the local cache for later use. If a connection issue occurs, the
Screenlet retrieves the data from the local cache. If the data doesn't
exist there, the Screenlet uses the listener to notify the developer
about the error. \textbar{} Use this policy to show the most recent
version of the data when connected, but show an outdated version when
there's no connection. \textbar{} \texttt{cache-first} \textbar{} The
Screenlet loads the data from the local cache. If the data isn't there,
the Screenlet requests it from the Liferay instance and notifies the
developer about any errors that occur (including connectivity errors).
\textbar{} Use this policy to save bandwidth and loading time in case
you have local (but probably outdated) data. \textbar{}

\noindent\hrulefill

\subsection{Required Attributes}\label{required-attributes-34}

\begin{itemize}
\tightlist
\item
  \texttt{assetEntryId}
\end{itemize}

If you don't use \texttt{assetEntryId}, you must use these attributes:

\begin{itemize}
\tightlist
\item
  \texttt{className}
\item
  \texttt{classPK}
\end{itemize}

\subsection{Attributes}\label{attributes-39}

\noindent\hrulefill

Attribute \textbar{} Data type \textbar{} Explanation \textbar{}
\texttt{assetEntryId} \textbar{} \texttt{number} \textbar{} The primary
key of the video file. \textbar{} \texttt{className} \textbar{}
\texttt{string} \textbar{} The video file's fully qualified class name.
Since files in a Documents and Media Library are \texttt{DLFileEntry}
objects, the \texttt{className} is
\href{https://docs.liferay.com/dxp/portal/7.1-latest/javadocs/portal-kernel/com/liferay/document/library/kernel/model/DLFileEntry.html}{\texttt{com.liferay.document.library.kernel.model.DLFileEntry}}.
The \texttt{className} and \texttt{classPK} attributes are required to
instantiate the Screenlet. \textbar{} \texttt{classPK} \textbar{}
\texttt{number} \textbar{} The video file's unique identifier. The
\texttt{className} and \texttt{classPK} attributes are required to
instantiate the Screenlet. \textbar{} \texttt{autoLoad} \textbar{}
\texttt{boolean} \textbar{} Whether the video automatically loads when
the Screenlet appears in the app's UI. The default value is
\texttt{true}. \textbar{} \texttt{offlinePolicy} \textbar{}
\texttt{string} \textbar{} The offline mode setting. See
\href{/docs/7-1/reference/-/knowledge_base/r/video-display-screenlet-for-ios\#offline}{the
Offline section} for details. \textbar{}

\noindent\hrulefill

\subsection{Delegate}\label{delegate-17}

Because images are files, Video Display Screenlet delegates its events
to an object that conforms to the \texttt{FileDisplayScreenletDelegate}
protocol. This protocol lets you implement the following methods:

\begin{itemize}
\item
  \texttt{-\ screenlet:onFileAssetResponse:}: Called when the Screenlet
  receives the image file.
\item
  \texttt{-\ screenlet:onFileAssetError:}: Called when an error occurs
  in the process. The \texttt{NSError} object describes the error.
\end{itemize}

\section{Audio Display Screenlet for
iOS}\label{audio-display-screenlet-for-ios}

\subsection{Requirements}\label{requirements-40}

\begin{itemize}
\tightlist
\item
  Xcode 9.3 or above
\item
  iOS 11 SDK
\item
  Liferay Portal 6.2 CE/EE, Liferay CE Portal 7.0/7.1, Liferay DXP
\item
  Liferay Screens Compatibility app
  (\href{http://www.liferay.com/marketplace/-/mp/application/54365664}{CE}
  or
  \href{http://www.liferay.com/marketplace/-/mp/application/54369726}{EE/DXP}).
  This app is preinstalled in Liferay CE Portal 7.0/7.1 and Liferay DXP.
\end{itemize}

\subsection{Compatibility}\label{compatibility-40}

\begin{itemize}
\tightlist
\item
  iOS 9 and above
\end{itemize}

\subsection{Xamarin Requirements}\label{xamarin-requirements-40}

\begin{itemize}
\tightlist
\item
  Visual Studio 7.2
\item
  Mono .NET framework 5.4.1.6
\end{itemize}

\subsection{Features}\label{features-40}

Audio Display Screenlet displays an audio file from a Liferay instance's
Documents and Media Library.

\subsection{JSON Services Used}\label{json-services-used-38}

Screenlets in Liferay Screens call JSON web services in the portal. This
Screenlet calls the following services and methods.

\noindent\hrulefill

\begin{longtable}[]{@{}
  >{\raggedright\arraybackslash}p{(\columnwidth - 4\tabcolsep) * \real{0.3889}}
  >{\raggedright\arraybackslash}p{(\columnwidth - 4\tabcolsep) * \real{0.3333}}
  >{\raggedright\arraybackslash}p{(\columnwidth - 4\tabcolsep) * \real{0.2778}}@{}}
\toprule\noalign{}
\begin{minipage}[b]{\linewidth}\raggedright
Service
\end{minipage} & \begin{minipage}[b]{\linewidth}\raggedright
Method
\end{minipage} & \begin{minipage}[b]{\linewidth}\raggedright
Notes
\end{minipage} \\
\midrule\noalign{}
\endhead
\bottomrule\noalign{}
\endlastfoot
\texttt{ScreensassetentryService} (Screens compatibility plugin) &
\texttt{getAssetEntry} & With \texttt{entryId} \\
\texttt{ScreensassetentryService} (Screens compatibility plugin) &
\texttt{getAssetEntry} & With \texttt{classPK} and \texttt{className} \\
\texttt{ScreensassetentryService} (Screens compatibility plugin) &
\texttt{getAssetEntries} & With \texttt{entryQuery} \\
\texttt{ScreensassetentryService} (Screens compatibility plugin) &
\texttt{getAssetEntries} & With \texttt{companyId}, \texttt{groupId},
and \texttt{portletItemName} \\
\end{longtable}

\noindent\hrulefill

\subsection{Module}\label{module-40}

\begin{itemize}
\tightlist
\item
  None
\end{itemize}

\subsection{Themes}\label{themes-18}

\begin{itemize}
\tightlist
\item
  Default
\end{itemize}

The Default Theme uses an iOS \texttt{AVAudioPlayer} to display the
audio player. For the player components, this Theme uses
\texttt{UIButton}, \texttt{UISlider}, and several \texttt{UILabel}
instances.

\begin{figure}
\centering
\includegraphics{./images/screens-ios-audiodisplay.png}
\caption{Audio Display Screenlet using the Default Theme.}
\end{figure}

\subsection{Offline}\label{offline-38}

This Screenlet supports offline mode so it can function without a
network connection. For more information on how offline mode works, see
the
\href{/docs/7-1/tutorials/-/knowledge_base/t/architecture-of-offline-mode-in-liferay-screens}{tutorial
on its architecture}. Here are the offline mode policies that you can
use with this Screenlet:

\noindent\hrulefill

Policy \textbar{} What happens \textbar{} When to use \textbar{}
\texttt{remote-only} \textbar{} The Screenlet loads the data from the
Liferay instance. If a connection issue occurs, the Screenlet uses the
listener to notify the developer about the error. If the Screenlet
successfully loads the data, it stores it in the local cache for later
use. \textbar{} Use this policy when you always need to show updated
data, and show nothing when there's no connection. \textbar{}
\texttt{cache-only} \textbar{} The Screenlet loads the data from the
local cache. If the data isn't there, the Screenlet uses the listener to
notify the developer about the error. \textbar{} Use this policy when
you always need to show local data, without retrieving remote
information under any circumstance. \textbar{} \texttt{remote-first}
\textbar{} The Screenlet loads the data from the Liferay instance. If
this succeeds, the Screenlet shows the data to the user and stores it in
the local cache for later use. If a connection issue occurs, the
Screenlet retrieves the data from the local cache. If the data doesn't
exist there, the Screenlet uses the listener to notify the developer
about the error. \textbar{} Use this policy to show the most recent
version of the data when connected, but show an outdated version when
there's no connection. \textbar{} \texttt{cache-first} \textbar{} The
Screenlet loads the data from the local cache. If the data isn't there,
the Screenlet requests it from the Liferay instance and notifies the
developer about any errors that occur (including connectivity errors).
\textbar{} Use this policy to save bandwidth and loading time in case
you have local (but probably outdated) data. \textbar{}

\noindent\hrulefill

\subsection{Required Attributes}\label{required-attributes-35}

\begin{itemize}
\tightlist
\item
  \texttt{assetEntryId}
\end{itemize}

If you don't use \texttt{assetEntryId}, you must use these attributes:

\begin{itemize}
\tightlist
\item
  \texttt{className}
\item
  \texttt{classPK}
\end{itemize}

\subsection{Attributes}\label{attributes-40}

\noindent\hrulefill

Attribute \textbar{} Data type \textbar{} Explanation \textbar{}
\texttt{assetEntryId} \textbar{} \texttt{number} \textbar{} The primary
key of the audio file. \textbar{} \texttt{className} \textbar{}
\texttt{string} \textbar{} The audio file's fully qualified class name.
Since files in a Documents and Media Library are \texttt{DLFileEntry}
objects, their \texttt{className} is
\href{https://docs.liferay.com/dxp/portal/7.1-latest/javadocs/portal-kernel/com/liferay/document/library/kernel/model/DLFileEntry.html}{\texttt{com.liferay.document.library.kernel.model.DLFileEntry}}.
The \texttt{className} and \texttt{classPK} attributes are required to
instantiate the Screenlet. \textbar{} \texttt{classPK} \textbar{}
\texttt{number} \textbar{} The audio file's unique identifier. The
\texttt{className} and \texttt{classPK} attributes are required to
instantiate the Screenlet. \textbar{} \texttt{autoLoad} \textbar{}
\texttt{boolean} \textbar{} Whether the audio file automatically loads
when the Screenlet appears in the app's UI. The default value is
\texttt{true}. \textbar{} \texttt{offlinePolicy} \textbar{}
\texttt{string} \textbar{} The offline mode setting. See
\href{/docs/7-1/reference/-/knowledge_base/r/audio-display-screenlet-for-ios\#offline}{the
Offline section} for details. \textbar{}

\noindent\hrulefill

\subsection{Delegate}\label{delegate-18}

Audio Display Screenlet delegates its events to an object that conforms
to the \texttt{FileDisplayScreenletDelegate} protocol. This protocol
lets you implement the following methods:

\begin{itemize}
\item
  \texttt{-\ screenlet:onFileAssetResponse:}: Called when the Screenlet
  receives the audio file.
\item
  \texttt{-\ screenlet:onFileAssetError:}: Called when an error occurs
  in the process. An \texttt{NSError} object describes the error.
\end{itemize}

\section{PDF Display Screenlet for
iOS}\label{pdf-display-screenlet-for-ios}

\subsection{Requirements}\label{requirements-41}

\begin{itemize}
\tightlist
\item
  Xcode 9.3 or above
\item
  iOS 11 SDK
\item
  Liferay Portal 6.2 CE/EE, Liferay CE Portal 7.0/7.1, Liferay DXP
\item
  Liferay Screens Compatibility app
  (\href{http://www.liferay.com/marketplace/-/mp/application/54365664}{CE}
  or
  \href{http://www.liferay.com/marketplace/-/mp/application/54369726}{EE/DXP}).
  This app is preinstalled in Liferay CE Portal 7.0/7.1 and Liferay DXP.
\end{itemize}

\subsection{Compatibility}\label{compatibility-41}

\begin{itemize}
\tightlist
\item
  iOS 9 and above
\end{itemize}

\subsection{Xamarin Requirements}\label{xamarin-requirements-41}

\begin{itemize}
\tightlist
\item
  Visual Studio 7.2
\item
  Mono .NET framework 5.4.1.6
\end{itemize}

\subsection{Features}\label{features-41}

PDF Display Screenlet displays a PDF file from a Liferay Instance's
Documents and Media Library.

\subsection{JSON Services Used}\label{json-services-used-39}

Screenlets in Liferay Screens call JSON web services in the portal. This
Screenlet calls the following services and methods.

\noindent\hrulefill

\begin{longtable}[]{@{}
  >{\raggedright\arraybackslash}p{(\columnwidth - 4\tabcolsep) * \real{0.3889}}
  >{\raggedright\arraybackslash}p{(\columnwidth - 4\tabcolsep) * \real{0.3333}}
  >{\raggedright\arraybackslash}p{(\columnwidth - 4\tabcolsep) * \real{0.2778}}@{}}
\toprule\noalign{}
\begin{minipage}[b]{\linewidth}\raggedright
Service
\end{minipage} & \begin{minipage}[b]{\linewidth}\raggedright
Method
\end{minipage} & \begin{minipage}[b]{\linewidth}\raggedright
Notes
\end{minipage} \\
\midrule\noalign{}
\endhead
\bottomrule\noalign{}
\endlastfoot
\texttt{ScreensassetentryService} (Screens compatibility plugin) &
\texttt{getAssetEntry} & With \texttt{entryId} \\
\texttt{ScreensassetentryService} (Screens compatibility plugin) &
\texttt{getAssetEntry} & With \texttt{classPK} and \texttt{className} \\
\texttt{ScreensassetentryService} (Screens compatibility plugin) &
\texttt{getAssetEntries} & With \texttt{entryQuery} \\
\texttt{ScreensassetentryService} (Screens compatibility plugin) &
\texttt{getAssetEntries} & With \texttt{companyId}, \texttt{groupId},
and \texttt{portletItemName} \\
\end{longtable}

\noindent\hrulefill

\subsection{Module}\label{module-41}

\begin{itemize}
\tightlist
\item
  None
\end{itemize}

\subsection{Themes}\label{themes-19}

\begin{itemize}
\tightlist
\item
  Default
\end{itemize}

The Default Theme uses an iOS \texttt{UIWebView} for displaying the PDF
file.

\begin{figure}
\centering
\includegraphics{./images/screens-ios-pdfdisplay.png}
\caption{PDF Display Screenlet using the Default Theme.}
\end{figure}

\subsection{Offline}\label{offline-39}

This Screenlet supports offline mode so it can function without a
network connection. For more information on how offline mode works, see
the
\href{/docs/7-1/tutorials/-/knowledge_base/t/architecture-of-offline-mode-in-liferay-screens}{tutorial
on its architecture}. Here are the offline mode policies that you can
use with this Screenlet:

\noindent\hrulefill

Policy \textbar{} What happens \textbar{} When to use \textbar{}
\texttt{remote-only} \textbar{} The Screenlet loads the data from the
Liferay instance. If a connection issue occurs, the Screenlet uses the
listener to notify the developer about the error. If the Screenlet
successfully loads the data, it stores it in the local cache for later
use. \textbar{} Use this policy when you always need to show updated
data, and show nothing when there's no connection. \textbar{}
\texttt{cache-only} \textbar{} The Screenlet loads the data from the
local cache. If the data isn't there, the Screenlet uses the listener to
notify the developer about the error. \textbar{} Use this policy when
you always need to show local data, without retrieving remote
information under any circumstance. \textbar{} \texttt{remote-first}
\textbar{} The Screenlet loads the data from the Liferay instance. If
this succeeds, the Screenlet shows the data to the user and stores it in
the local cache for later use. If a connection issue occurs, the
Screenlet retrieves the data from the local cache. If the data doesn't
exist there, the Screenlet uses the listener to notify the developer
about the error. \textbar{} Use this policy to show the most recent
version of the data when connected, but show an outdated version when
there's no connection. \textbar{} \texttt{cache-first} \textbar{} The
Screenlet loads the data from the local cache. If the data isn't there,
the Screenlet requests it from the Liferay instance and notifies the
developer about any errors that occur (including connectivity errors).
\textbar{} Use this policy to save bandwidth and loading time in case
you have local (but probably outdated) data. \textbar{}

\noindent\hrulefill

\subsection{Required Attributes}\label{required-attributes-36}

\begin{itemize}
\tightlist
\item
  \texttt{assetEntryId}
\end{itemize}

If you don't use \texttt{assetEntryId}, you must use these attributes:

\begin{itemize}
\tightlist
\item
  \texttt{className}
\item
  \texttt{classPK}
\end{itemize}

\subsection{Attributes}\label{attributes-41}

\noindent\hrulefill

Attribute \textbar{} Data type \textbar{} Explanation \textbar{}
\texttt{assetEntryId} \textbar{} \texttt{number} \textbar{} The primary
key of the PDF file. \textbar{} \texttt{className} \textbar{}
\texttt{string} \textbar{} The PDF file's fully qualified class name.
Since files in a Documents and Media Library are \texttt{DLFileEntry}
objects, their \texttt{className} is
\href{https://docs.liferay.com/dxp/portal/7.1-latest/javadocs/portal-kernel/com/liferay/document/library/kernel/model/DLFileEntry.html}{\texttt{com.liferay.document.library.kernel.model.DLFileEntry}}.
The \texttt{className} and \texttt{classPK} attributes are required to
instantiate the Screenlet. \textbar{} \texttt{classPK} \textbar{}
\texttt{number} \textbar{} The PDF file's unique identifier. The
\texttt{className} and \texttt{classPK} attributes are required to
instantiate the Screenlet. \textbar{} \texttt{autoLoad} \textbar{}
\texttt{boolean} \textbar{} Whether the PDF automatically loads when the
Screenlet appears in the app's UI. The default value is \texttt{true}.
\textbar{} \texttt{offlinePolicy} \textbar{} \texttt{string} \textbar{}
The offline mode setting. See
\href{/docs/7-1/reference/-/knowledge_base/r/pdf-display-screenlet-for-ios\#offline}{the
Offline section} for details. \textbar{}

\noindent\hrulefill

\subsection{Delegate}\label{delegate-19}

Because PDFs are files, PDF Display Screenlet delegates some events to
an object that conforms to the \texttt{FileDisplayScreenletDelegate}
protocol. This protocol lets you implement the following methods:

\begin{itemize}
\item
  \texttt{-\ screenlet:onFileAssetResponse:}: Called when the Screenlet
  receives the PDF.
\item
  \texttt{-\ screenlet:onFileAssetError:}: Called when an error occurs
  in the process. An \texttt{NSError} object describes the error.
\end{itemize}

\section{File Display Screenlet for
iOS}\label{file-display-screenlet-for-ios}

\subsection{Requirements}\label{requirements-42}

\begin{itemize}
\tightlist
\item
  Xcode 9.3 or above
\item
  iOS 11 SDK
\item
  Liferay Portal 6.2 CE/EE, Liferay CE Portal 7.0/7.1, Liferay DXP
\item
  Liferay Screens Compatibility app
  (\href{http://www.liferay.com/marketplace/-/mp/application/54365664}{CE}
  or
  \href{http://www.liferay.com/marketplace/-/mp/application/54369726}{EE/DXP}).
  This app is preinstalled in Liferay CE Portal 7.0/7.1 and Liferay DXP.
\end{itemize}

\subsection{Compatibility}\label{compatibility-42}

\begin{itemize}
\tightlist
\item
  iOS 9 and above
\end{itemize}

\subsection{Xamarin Requirements}\label{xamarin-requirements-42}

\begin{itemize}
\tightlist
\item
  Visual Studio 7.2
\item
  Mono .NET framework 5.4.1.6
\end{itemize}

\subsection{Features}\label{features-42}

File Display Screenlet shows a single file from a Liferay DXP instance's
Documents and Media Library. Use this Screenlet to display file types
not covered by the other display Screenlets (e.g., DOC, PPT, XLS).

\subsection{JSON Services Used}\label{json-services-used-40}

Screenlets in Liferay Screens call JSON web services in the portal. This
Screenlet calls the following services and methods.

\noindent\hrulefill

\begin{longtable}[]{@{}
  >{\raggedright\arraybackslash}p{(\columnwidth - 4\tabcolsep) * \real{0.3889}}
  >{\raggedright\arraybackslash}p{(\columnwidth - 4\tabcolsep) * \real{0.3333}}
  >{\raggedright\arraybackslash}p{(\columnwidth - 4\tabcolsep) * \real{0.2778}}@{}}
\toprule\noalign{}
\begin{minipage}[b]{\linewidth}\raggedright
Service
\end{minipage} & \begin{minipage}[b]{\linewidth}\raggedright
Method
\end{minipage} & \begin{minipage}[b]{\linewidth}\raggedright
Notes
\end{minipage} \\
\midrule\noalign{}
\endhead
\bottomrule\noalign{}
\endlastfoot
\texttt{ScreensassetentryService} (Screens compatibility plugin) &
\texttt{getAssetEntry} & With \texttt{entryId} \\
\texttt{ScreensassetentryService} (Screens compatibility plugin) &
\texttt{getAssetEntry} & With \texttt{classPK} and \texttt{className} \\
\texttt{ScreensassetentryService} (Screens compatibility plugin) &
\texttt{getAssetEntries} & With \texttt{entryQuery} \\
\texttt{ScreensassetentryService} (Screens compatibility plugin) &
\texttt{getAssetEntries} & With \texttt{companyId}, \texttt{groupId},
and \texttt{portletItemName} \\
\end{longtable}

\noindent\hrulefill

\subsection{Module}\label{module-42}

\begin{itemize}
\tightlist
\item
  None
\end{itemize}

\subsection{Themes}\label{themes-20}

\begin{itemize}
\tightlist
\item
  Default
\end{itemize}

The Default View uses an iOS \texttt{UIWebView} for displaying the file.

\begin{figure}
\centering
\includegraphics{./images/screens-ios-filedisplay.png}
\caption{File Display Screenlet using the Default View.}
\end{figure}

\subsection{Offline}\label{offline-40}

This Screenlet supports offline mode so it can function without a
network connection. For more information on how offline mode works, see
the
\href{/docs/7-1/tutorials/-/knowledge_base/t/architecture-of-offline-mode-in-liferay-screens}{tutorial
on its architecture}. Here are the offline mode policies that you can
use with this Screenlet:

\noindent\hrulefill

Policy \textbar{} What happens \textbar{} When to use \textbar{}
\texttt{remote-only} \textbar{} The Screenlet loads the data from the
Liferay instance. If a connection issue occurs, the Screenlet uses the
listener to notify the developer about the error. If the Screenlet
successfully loads the data, it stores it in the local cache for later
use. \textbar{} Use this policy when you always need to show updated
data, and show nothing when there's no connection. \textbar{}
\texttt{cache-only} \textbar{} The Screenlet loads the data from the
local cache. If the data isn't there, the Screenlet uses the listener to
notify the developer about the error. \textbar{} Use this policy when
you always need to show local data, without retrieving remote
information under any circumstance. \textbar{} \texttt{remote-first}
\textbar{} The Screenlet loads the data from the Liferay instance. If
this succeeds, the Screenlet shows the data to the user and stores it in
the local cache for later use. If a connection issue occurs, the
Screenlet retrieves the data from the local cache. If the data doesn't
exist there, the Screenlet uses the listener to notify the developer
about the error. \textbar{} Use this policy to show the most recent
version of the data when connected, but show an outdated version when
there's no connection. \textbar{} \texttt{cache-first} \textbar{} The
Screenlet loads the data from the local cache. If the data isn't there,
the Screenlet requests it from the Liferay instance and notifies the
developer about any errors that occur (including connectivity errors).
\textbar{} Use this policy to save bandwidth and loading time in case
you have local (but probably outdated) data. \textbar{}

\noindent\hrulefill

\subsection{Attributes}\label{attributes-42}

\noindent\hrulefill

Attribute \textbar{} Data type \textbar{} Explanation \textbar{}
\texttt{assetEntryId} \textbar{} \texttt{number} \textbar{} The primary
key of the file. \textbar{} \texttt{className} \textbar{}
\texttt{string} \textbar{} The file's fully qualified class name. Since
files in a Documents and Media Library are \texttt{DLFileEntry} objects,
their \texttt{className} is
\href{https://docs.liferay.com/dxp/portal/7.1-latest/javadocs/portal-kernel/com/liferay/document/library/kernel/model/DLFileEntry.html}{\texttt{com.liferay.document.library.kernel.model.DLFileEntry}}.
The \texttt{className} and \texttt{classPK} attributes are required to
instantiate the Screenlet. \textbar{} \texttt{classPK} \textbar{}
\texttt{number} \textbar{} The file's unique identifier. The
\texttt{className} and \texttt{classPK} attributes are required to
instantiate the Screenlet. \textbar{} \texttt{autoLoad} \textbar{}
\texttt{boolean} \textbar{} Whether the file automatically loads when
the Screenlet appears in the app's UI. The default value is
\texttt{true}. \textbar{} \texttt{offlinePolicy} \textbar{}
\texttt{string} \textbar{} The offline mode setting. See
\href{/docs/7-1/reference/-/knowledge_base/r/file-display-screenlet-for-ios\#offline}{the
Offline section} for details. \textbar{}

\noindent\hrulefill

\subsection{Delegate}\label{delegate-20}

File Display Screenlet delegates some events to an object that conforms
to the \texttt{FileDisplayScreenletDelegate} protocol. This protocol
lets you implement the following methods:

\begin{itemize}
\item
  \texttt{-\ screenlet:onFileAssetResponse:}: Called when the Screenlet
  receives the file.
\item
  \texttt{-\ screenlet:onFileAssetError:}: Called when an error occurs
  in the process. An \texttt{NSError} object describes the error.
\end{itemize}

\section{Web Screenlet for iOS}\label{web-screenlet-for-ios}

\subsection{Requirements}\label{requirements-43}

\begin{itemize}
\tightlist
\item
  Xcode 9.3 or above
\item
  iOS 11 SDK
\item
  Liferay Portal 6.2 CE/EE, Liferay CE Portal 7.0/7.1, Liferay DXP
\item
  Liferay Screens Compatibility app
  (\href{http://www.liferay.com/marketplace/-/mp/application/54365664}{CE}
  or
  \href{http://www.liferay.com/marketplace/-/mp/application/54369726}{EE/DXP}).
  This app is preinstalled in Liferay CE Portal 7.0/7.1 and Liferay DXP.
\end{itemize}

\subsection{Compatibility}\label{compatibility-43}

\begin{itemize}
\tightlist
\item
  iOS 9 and above
\end{itemize}

\subsection{Xamarin Requirements}\label{xamarin-requirements-43}

\begin{itemize}
\tightlist
\item
  Visual Studio 7.2
\item
  Mono .NET framework 5.4.1.6
\end{itemize}

\subsection{Features}\label{features-43}

Web Screenlet lets you display any web page. It also lets you customize
the web page through injection of local and remote JavaScript and CSS
files. If you're using Liferay DXP as backend, you can use
\href{/docs/7-1/user/-/knowledge_base/u/styling-apps-and-assets}{Application
Display Templates} in your page to customize its content from the server
side.

\subsection{Module}\label{module-43}

\begin{itemize}
\tightlist
\item
  None
\end{itemize}

\subsection{Themes}\label{themes-21}

\begin{itemize}
\tightlist
\item
  Default
\end{itemize}

The Default Theme uses an iOS \texttt{WKWebView} for displaying the web
page.

\begin{figure}
\centering
\includegraphics{./images/screens-ios-webscreenlet.png}
\caption{Web Screenlet using the Default Theme.}
\end{figure}

\subsection{Configuration}\label{configuration-1}

To learn how to use Web Screenlet, follow the steps in the tutorial
\href{/docs/7-1/tutorials/-/knowledge_base/t/rendering-web-pages-in-your-ios-app}{Rendering
Web Pages in Your iOS App}. That tutorial gives detailed instructions
for using the configuration items described here.

Web Screenlet has \texttt{WebScreenletConfiguration} and
\texttt{WebScreenletConfigurationBuilder} objects that you can use
together to supply the parameters that the Screenlet needs to work.
\texttt{WebScreenletConfigurationBuilder} has the following methods,
which let you supply the described configuration parameters:

\noindent\hrulefill

Method \textbar{} Returns \textbar{} Explanation \textbar{}
\texttt{addJs(localFile:\ String)} \textbar{}
\texttt{WebScreenletConfigurationBuilder} \textbar{} Adds a local
JavaScript file with the supplied filename. \textbar{}
\texttt{addCss(localFile:\ String)} \textbar{}
\texttt{WebScreenletConfigurationBuilder} \textbar{} Adds a local CSS
file with the supplied filename. \textbar{} \texttt{addJs(url:\ String)}
\textbar{} \texttt{WebScreenletConfigurationBuilder} \textbar{} Adds a
JavaScript file from the supplied URL. \textbar{}
\texttt{addCss(url:\ String)} \textbar{}
\texttt{WebScreenletConfigurationBuilder} \textbar{} Adds a CSS file
from the supplied URL. \textbar{} \texttt{set(webType:\ WebType)}
\textbar{} \texttt{WebScreenletConfigurationBuilder} \textbar{} Sets the
\href{/docs/7-1/reference/-/knowledge_base/r/web-screenlet-for-ios\#webtype}{\texttt{WebType}}.
\textbar{} \texttt{enableCordova()} \textbar{}
\texttt{WebScreenletConfigurationBuilder} \textbar{} Enables Cordova
inside the Web Screenlet. \textbar{} \texttt{load()} \textbar{}
\texttt{WebScreenletConfiguration} \textbar{} Returns the
\texttt{WebScreenletConfiguration} object that you can set to the
Screenlet instance. \textbar{}

\noindent\hrulefill

\noindent\hrulefill

\textbf{Note:} If you want to add comments in the scripts, use the
\texttt{/**/} notation.

\noindent\hrulefill

\subsubsection{WebType}\label{webtype-1}

\begin{itemize}
\item
  \textbf{WebType.liferayAuthenticated} (default): Displays a Liferay
  DXP page that requires authentication. The user must therefore be
  logged in with Screens via Login Screenlet or a
  \texttt{SessionContext} method. For this \texttt{WebType}, the URL you
  must pass to the \texttt{WebScreenletConfigurationBuilder} constructor
  is a relative URL. For example, if the full URL is
  \texttt{http://screens.liferay.org.es/web/guest/blog}, then the URL
  you must supply to the constructor is \texttt{/web/guest/blog}.
\item
  \textbf{WebType.other}: Displays any other page. For this
  \texttt{WebType}, the URL you must pass to the
  \texttt{WebScreenletConfigurationBuilder} constructor is a full URL.
  For example, if the full URL is
  \texttt{http://screens.liferay.org.es/web/guest/blog}, then you must
  supply that URL to the constructor.
\end{itemize}

\subsection{Attributes}\label{attributes-43}

\noindent\hrulefill

Attribute \textbar{} Data type \textbar{} Explanation \textbar{}
\texttt{autoLoad} \textbar{} \texttt{boolean} \textbar{} Whether to load
the page automatically when the Screenlet appears in the app's UI. The
default value is \texttt{true}. \textbar{} \texttt{loggingEnabled}
\textbar{} \texttt{boolean} \textbar{} Whether logging is enabled.
\textbar{} \texttt{isScrollEnabled} \textbar{} \texttt{boolean}
\textbar{} Whether to enable scrolling on the page inside the Screenlet.
\textbar{}

\noindent\hrulefill

\subsection{Delegate}\label{delegate-21}

Web Screenlet delegates some events to an object that conforms to the
\texttt{WebScreenletDelegate} protocol. This protocol lets you implement
the following methods:

\begin{itemize}
\item
  \texttt{onWebLoad(\_:url:)}: Called when the Screenlet loads the page.

\begin{verbatim}
func onWebLoad(_ screenlet: WebScreenlet, url: String) {
    ...
}
\end{verbatim}
\item
  \texttt{screenlet(\_:onScriptMessageNamespace:onScriptMessage:)}:
  Called when the \texttt{WKWebView} sends a message.

\begin{verbatim}
func screenlet(_ screenlet: WebScreenlet,
           onScriptMessageNamespace namespace: String,
           onScriptMessage message: String) {
    ...
}
\end{verbatim}
\item
  \texttt{screenlet(\_:onError:)}: Called when an error occurs in the
  process. The \texttt{NSError} object describes the error.

\begin{verbatim}
func screenlet(_ screenlet: WebScreenlet, onError error: NSError) {
    ...
}
\end{verbatim}
\end{itemize}

\section{SyncManagerDelegate}\label{syncmanagerdelegate}

The \texttt{SyncManagerDelegate} class is required to
\href{/docs/7-1/tutorials/-/knowledge_base/t/using-offline-mode-in-ios}{use
Screenlets with offline mode}. This class receives the events produced
in the synchronization process. This document describes the class's
methods.

\subsection{Methods}\label{methods-23}

The following method is invoked when the synchronization process is
started. The number of items to be synced are passed.

\begin{verbatim}
syncManager(manager: SyncManager, itemsCount: UInt)
\end{verbatim}

The following method is invoked when an item synchronization is about to
start.

\begin{verbatim}
syncManager(manager: SyncManager, onItemSyncScreenlet screenlet: String, 
    startKey: String, attributes: [String:AnyObject])
\end{verbatim}

\begin{itemize}
\tightlist
\item
  \texttt{screenlet}: the screenlet name that stored this cache element
\item
  \texttt{startKey}: the cache key where the item is stored
\item
  \texttt{attributes}: some attributes stored together with the element.
  The specific attributes depend on the type of the entry. For more
  details, read the screenlet reference documentation.
\end{itemize}

The following method is invoked when an item synchronization is
successfully completed.

\begin{verbatim}
syncManager(manager: SyncManager, onItemSyncScreenlet screenlet: String, 
    completedKey: String, attributes: [String:AnyObject])
\end{verbatim}

\begin{itemize}
\tightlist
\item
  \texttt{screenlet}: the screenlet name that stored this cache element
\item
  \texttt{completedKey}: the cache key where the item is stored
\item
  \texttt{attributes}: some attributes stored together with the element.
  The specific attributes depend on the type of the entry. For more
  details, read the screenlet reference documentation.
\end{itemize}

The following method is invoked when an item synchronization fails.

\begin{verbatim}
syncManager(manager: SyncManager, onItemSyncScreenlet screenlet: String, 
    failedKey: String, attributes: [String:AnyObject], error: NSError)
\end{verbatim}

\begin{itemize}
\tightlist
\item
  \texttt{screenlet}: the screenlet name that stored this cache element
\item
  \texttt{failedKey}: the cache key where the item is stored
\item
  \texttt{attributes}: some attributes stored together with the element.
  The specific attributes will depend on the type of the entry. For more
  details, read the screenlet reference documentation.
\item
  \texttt{error}: the error occurred in the synchronization
\end{itemize}

The following method is invoked when an item synchronization detects a
conflict. The method must invoke asynchronously a
\href{https://en.wikipedia.org/wiki/Continuation-passing_style}{continuation}
argument with the conflict action result.

\begin{verbatim}
syncManager(manager: SyncManager, onItemSyncScreenlet screenlet: String, 
    conflictedKey: String, remoteValue: AnyObject, localValue: AnyObject, 
    resolve: SyncConflictResolution -> ())
\end{verbatim}

\begin{itemize}
\tightlist
\item
  \texttt{screenlet}: the screenlet name that stored this cache element
\item
  \texttt{conflictedKey}: the cache key where the item is stored
\item
  \texttt{remoteValue}: the value stored in the server for the item
  being synchronized
\item
  \texttt{localValue}: the value stored in the cache for the item being
  synchronized
\item
  \texttt{resolve}: this is the continuation function to be called with
  the action result.
\end{itemize}

Supported values for \texttt{resolve} are:

\begin{itemize}
\tightlist
\item
  \texttt{UseRemote}: the remote version is overwritten with the local
  one. Both the local cache and the portal have the same version.
\item
  \texttt{UseLocal}: the local version is overwritten with the remote
  one. Both the local cache and the portal have the same version
\item
  \texttt{Discard}: the local version is removed and the remote one
  isn't overwritten.
\item
  \texttt{Ignore}: data is not changed, so the next synchronization will
  detect the conflict again.
\end{itemize}

\chapter{Themes}\label{themes-22}

Theme development is a multistep process, involving many tools and
endless possibilities. This section of reference docs provides the
following helpful information for theme development:

\begin{itemize}
\tightlist
\item
  A Theme anatomy reference guide
\end{itemize}

\section{Theme Reference Guide}\label{theme-reference-guide}

A theme is made up of several files. Although most of the files are
named after their matching components, their functions may be unclear.
This reference guide explains each file's usage to make clear which
files to modify.

\subsection{Theme Anatomy}\label{theme-anatomy}

There are two main approaches to theme development for 7.0: themes built
using the Node.js build tools with the
\href{/docs/7-1/tutorials/-/knowledge_base/t/creating-themes}{theme
generator} and
\href{/docs/7-1/tutorials/-/knowledge_base/t/creating-themes-with-liferay-ide}{themes
built using Dev Studio DXP}.

Themes developed with the theme generator have the anatomy shown below.
Although themes developed with Dev Studio DXP have a slightly different
anatomy built with the
\href{/docs/7-1/reference/-/knowledge_base/r/theme-template}{theme
project template}, the core theme files are the same.

\begin{itemize}
\tightlist
\item
  \texttt{theme-name/}

  \begin{itemize}
  \tightlist
  \item
    \texttt{src/}

    \begin{itemize}
    \tightlist
    \item
      \texttt{css/}

      \begin{itemize}
      \tightlist
      \item
        \href{/docs/7-1/reference/-/knowledge_base/r/theme-reference-guide\#\#-clay-customscss}{\texttt{\_clay\_custom.scss}}
      \item
        \href{/docs/7-1/reference/-/knowledge_base/r/theme-reference-guide\#-clay-variablesscss}{\texttt{\_clay\_variables.scss}}
      \item
        \href{/docs/7-1/reference/-/knowledge_base/r/theme-reference-guide\#-customscss}{\texttt{\_custom.scss}}
      \item
        \href{/docs/7-1/reference/-/knowledge_base/r/theme-reference-guide\#-liferay-variables-customscss}{\texttt{\_liferay\_variables\_custom.scss}}
      \end{itemize}
    \item
      \texttt{images/}

      \begin{itemize}
      \tightlist
      \item
        (custom images)
      \end{itemize}
    \item
      \texttt{js/}

      \begin{itemize}
      \tightlist
      \item
        \href{/docs/7-1/reference/-/knowledge_base/r/theme-reference-guide\#mainjs}{\texttt{main.js}}
      \end{itemize}
    \item
      \texttt{templates/}

      \begin{itemize}
      \tightlist
      \item
        \href{/docs/7-1/reference/-/knowledge_base/r/theme-reference-guide\#init-customftl}{\texttt{init\_custom.ftl}}
      \item
        \href{/docs/7-1/reference/-/knowledge_base/r/theme-reference-guide\#navigationftl}{\texttt{navigation.ftl}}
      \item
        \href{/docs/7-1/reference/-/knowledge_base/r/theme-reference-guide\#portal-normalftl}{\texttt{portal\_normal.ftl}}
      \item
        \href{/docs/7-1/reference/-/knowledge_base/r/theme-reference-guide\#portal-pop-upftl}{\texttt{portal\_pop\_up.ftl}}
      \item
        \href{/docs/7-1/reference/-/knowledge_base/r/theme-reference-guide\#portletftl}{\texttt{portlet.ftl}}
      \end{itemize}
    \item
      \texttt{WEB-INF/}

      \begin{itemize}
      \tightlist
      \item
        \href{/docs/7-1/reference/-/knowledge_base/r/theme-reference-guide\#liferay-look-and-feelxml}{\texttt{liferay-look-and-feel.xml}}
      \item
        \href{/docs/7-1/reference/-/knowledge_base/r/theme-reference-guide\#liferay-plugin-packageproperties}{\texttt{liferay-plugin-package.properties}}
      \item
        \texttt{src/}

        \begin{itemize}
        \tightlist
        \item
          \texttt{resources-importer/}

          \begin{itemize}
          \tightlist
          \item
            (Many directories)
          \end{itemize}
        \end{itemize}
      \end{itemize}
    \end{itemize}
  \item
    \href{/docs/7-1/reference/-/knowledge_base/r/theme-reference-guide\#liferay-themejson}{\texttt{liferay-theme.json}}
  \item
    \href{/docs/7-1/reference/-/knowledge_base/r/theme-reference-guide\#packagejson}{\texttt{package.json}}
  \end{itemize}
\end{itemize}

Regarding CSS files, you should only modify
\texttt{\_clay\_custom.scss}, \texttt{\_clay\_variables.scss},
\texttt{\_custom.scss}, and \texttt{\_liferay\_variables\_custom.scss}.

You can of course overwrite any CSS file you want, but if you modify any
other files, you're removing styling that 7.0 needs to work properly.

\subsection{Theme Files}\label{theme-files}

\subsubsection{\_clay\_custom.scss}\label{clay_custom.scss}

Used for Clay custom styles, i.e.~styles for a third party Bootstrap
theme. Anything written in this file is compiled in the same scope as
Bootstrap/Lexicon, so you can use their variables, mixins, etc. You can
also implement any of the variables you define in
\texttt{\_clay\_variables.scss}.

\subsubsection{\_clay\_variables.scss}\label{clay_variables.scss}

Used to store custom Sass variables. This file gets injected into the
Bootstrap/Lexicon build, so you can overwrite variables and change how
those libraries are compiled.

\subsubsection{\_custom.scss}\label{custom.scss}

Used for custom CSS styles. You should place all of your custom CSS
modifications in this file.

\subsubsection{\_liferay\_variables\_custom.scss}\label{liferay_variables_custom.scss}

Used for overwriting variables defined in
\texttt{\_liferay\_variables.scss} without wiping out the whole file.

\subsubsection{init\_custom.ftl}\label{init_custom.ftl}

Used for custom FreeMarker variables i.e.~
\href{/docs/7-1/tutorials/-/knowledge_base/t/making-configurable-theme-settings}{theme
setting} variables.

\subsubsection{navigation.ftl}\label{navigation.ftl}

The theme template for the theme's navigation.

\subsubsection{portal\_normal.ftl}\label{portal_normal.ftl}

Similar to a static site's \texttt{index.html}, this file acts as a hub
for all theme templates.

\subsubsection{portal\_pop\_up.ftl}\label{portal_pop_up.ftl}

The theme template for pop up dialogs for the theme's portlets.

\subsubsection{portlet.ftl}\label{portlet.ftl}

The theme template for the theme's portlets. If your theme uses
Application Decorators, you can modify this file to create application
decorator-specific theme settings. See the
\href{/docs/7-1/tutorials/-/knowledge_base/t/creating-configurable-styles-for-portlet-wrappers}{Portlet
Decorators tutorial} for more info.

\subsubsection{liferay-theme.json}\label{liferay-theme.json}

Contains the configuration settings for your app server, in Node.js
tool-based themes. You can change this file manually at any time to
update your server settings. The file can also be updated via the
\href{/docs/7-1/tutorials/-/knowledge_base/t/configuring-your-themes-app-server}{\texttt{gulp\ init}
task}.

\subsubsection{package.json}\label{package.json}

Contains theme setting information such as the theme template language,
version, and base theme, for Node.js tool developed themes. You can
update this file manually. The
\href{/docs/7-1/tutorials/-/knowledge_base/t/changing-your-base-theme}{\texttt{gulp\ extend}
task} can also be used to change the base theme.

\subsubsection{main.js}\label{main.js}

Used for custom JavaScript.

\subsubsection{liferay-look-and-feel.xml}\label{liferay-look-and-feel.xml}

Contains basic information for the theme. If your theme has
\href{/docs/7-1/tutorials/-/knowledge_base/t/making-configurable-theme-settings}{theme
settings}, they are defined in this file. For a full explanation of this
file, please see the
\href{https://docs.liferay.com/dxp/portal/7.1-latest/definitions/liferay-look-and-feel_7_1_0.dtd.html}{Definitions
docs}.

\subsubsection{liferay-plugin-package.properties}\label{liferay-plugin-package.properties}

Contains general properties for the theme.
\href{/docs/7-1/tutorials/-/knowledge_base/t/importing-resources-with-a-theme}{Resources
Importer} configuration settings are also placed in this file. For a
full explanation of the properties available for this file please see
the
\href{https://docs.liferay.com/dxp/portal/7.1-latest/propertiesdoc/liferay-plugin-package_7_1_0.properties.html}{7.1
Properties documentation}.

\section{Theme Components and
Workflow}\label{theme-components-and-workflow}

If you want to develop a website, you must have three key components:
CSS, JavaScript, and HTML. Liferay DXP supports SASS as well as multiple
JavaScript frameworks. The HTML, however, is rendered via
\href{https://freemarker.apache.org/}{FreeMarker} theme templates. This
reference guide provides an overview of Liferay DXP's theme development
components and workflow, covering the following topics:

\begin{itemize}
\tightlist
\item
  Theme templates
\item
  Theme customizations and extensions
\item
  Portlet customizations and extensions
\item
  Theme workflow
\item
  CSS Frameworks and extensions
\end{itemize}

\subsection{Theme Templates}\label{theme-templates}

Liferay DXP provides several default FreeMarker templates that each
handle a key piece of functionality for the page:

\begin{itemize}
\item
  \texttt{portal\_normal.ftl}: Similar to a static site's
  \texttt{index.html}, this file acts as a hub for all theme templates
  and provides the overall markup for the page.
\item
  \texttt{init.ftl}: Contains common FreeMarker variables that can be
  used in your theme templates. Useful for reference if you need access
  to theme objects. \textbf{We recommended that you DO NOT override this
  file}.
\item
  \texttt{init\_custom.ftl}: Used to override FreeMarker variables in
  \texttt{init.ftl} and to define new variables, such as
  \href{/docs/7-1/tutorials/-/knowledge_base/t/making-configurable-theme-settings}{theme
  settings}.
\item
  \texttt{portlet.ftl}: This template controls the theme's portlets. If
  your theme uses Portlet Decorators, you can modify this file to create
  application decorator-specific theme settings. See the
  \href{/docs/7-1/tutorials/-/knowledge_base/t/creating-configurable-styles-for-portlet-wrappers}{Portlet
  Decorators} tutorial for more info.
\item
  \texttt{navigation.ftl}: Contains the navigation markup. To customize
  pages in the navigation, you must use the
  \texttt{liferay.navigation\_menu} macro. Then you can leverage
  \href{https://github.com/liferay/liferay-portal/tree/7.1.x/modules/apps/site-navigation/site-navigation-menu-web/src/main/resources/com/liferay/site/navigation/menu/web/portlet/template/dependencies}{ADTs}
  for the navigation menu. Note that \texttt{navigation.ftl} also
  defines the hamburger icon and \texttt{navbar-collapse} class that
  provides the simplified navigation toggle for mobile viewports, as
  shown in the snippet below for the Classic theme:

\begin{verbatim}
<#if has_navigation && is_setup_complete>
  <button aria-controls="navigationCollapse" aria-expanded="false" 
  aria-label="Toggle navigation" class="navbar-toggler navbar-toggler-right" 
  data-target="#navigationCollapse" data-toggle="collapse" type="button">
    <span class="navbar-toggler-icon"></span>
  </button>

  <div aria-expanded="false" class="collapse navbar-collapse" id="navigationCollapse">
    <@liferay.navigation_menu default_preferences="${preferences}" />
  </div>
</#if>
\end{verbatim}
\end{itemize}

\begin{figure}
\centering
\includegraphics{./images/portal-layout-mobile-nav.png}
\caption{The collapsed navbar provides simplified user-friendly
navigation for mobile devices.}
\end{figure}

\begin{itemize}
\tightlist
\item
  \texttt{portal\_pop\_up.ftl}: The theme template controlling pop up
  dialogs for the theme's portlets. Similar to
  \texttt{portal\_normal.ftl}, \texttt{portal\_pop\_up.ftl} provides the
  markup template for all pop-up dialogs, such as a portlet's
  Configuration menu. It also has access to the FreeMarker variables
  defined in \texttt{init.ftl} and \texttt{init\_custom.ftl}.
\end{itemize}

\begin{figure}
\centering
\includegraphics{./images/portal-layout-theme-templates.png}
\caption{Each theme template provides a portion of the page's markup and
functionality.}
\end{figure}

\subsubsection{Theme Template Utilities}\label{theme-template-utilities}

Liferay DXP provides several FreeMarker variables and macros that you
can use in your theme templates to include portlets, use taglibs, access
theme objects, and more. You can see examples of these in
\texttt{portal\_normal.ftl}. These utilities are included in the files
listed below:

\begin{itemize}
\tightlist
\item
  \href{https://github.com/liferay/liferay-portal/blob/7.1.x/modules/apps/frontend-theme/frontend-theme-unstyled/src/main/resources/META-INF/resources/_unstyled/templates/init.ftl}{\texttt{Init.ftl}}:
  Provides access to common theme variables
\item
  \href{https://github.com/liferay/liferay-portal/blob/7.1.x/modules/apps/portal-template/portal-template-freemarker/src/main/resources/FTL_liferay.ftl}{\texttt{FTL\_Liferay.ftl}}:
  Provides macros for commonly used portlets and theme resources. See
  the
  \href{/docs/7-1/tutorials/-/knowledge_base/t/using-liferays-macros-in-your-theme}{Macros
  tutorial} for more information.
\item
  \texttt{taglib-mappings.properties}: Maps the portal taglibs to
  FreeMarker macros. Taglibs let you quickly create common UI
  components. This properties file is also provided separately for each
  app taglib. For convenience, these FreeMarker macros appear in the
  \href{/docs/7-1/reference/-/knowledge_base/r/freemarker-taglib-macros}{FreeMarker
  Taglib Mappings reference guide}. See the
  \href{/docs/7-1/tutorials/-/knowledge_base/t/front-end-taglibs}{Taglib
  tutorials} for more information on using each taglib in your theme
  templates.
\end{itemize}

\subsection{CSS Frameworks and
Extensions}\label{css-frameworks-and-extensions}

As noted above, Liferay DXP supports the Sass CSS extension, so you can
take full advantage of Sass mixins, nesting, partials, and variables in
your CSS.

Also important to note is \href{https://clayui.com/}{Clay CSS}, the web
implementation of Liferay's \href{https://lexicondesign.io/}{Lexicon
design language}. An extension of Bootstrap, Clay CSS fills the gaps
between Bootstrap and the needs of Liferay DXP, providing additional
components and CSS patterns that you can use in your themes. Clay base,
Liferay's Bootstrap API extension, along with Atlas, a custom Bootstrap
theme, creates Liferay DXP's Classic theme. See the
\href{/docs/7-1/tutorials/-/knowledge_base/t/importing-clay-css-into-a-theme}{importing
Clay CSS tutorial} for more information.

\subsection{Theme Customizations and
Extensions}\label{theme-customizations-and-extensions}

The theme templates, along with the CSS, provide much of the overall
look and feel for the page, but additional extension
points/customizations are available. The following extensions and
mechanisms are available for themes:

\begin{itemize}
\tightlist
\item
  \textbf{Color Schemes:} specifies configurable color scheme settings
  for Administrator's to configure via the Look and Feel menu. See the
  \href{/docs/7-1/tutorials/-/knowledge_base/t/creating-color-schemes-for-your-theme}{color
  scheme tutorial} for more information.
\item
  \textbf{Configurable Theme Settings:} settings that let Administrators
  configure aspects of a theme that may need changed frequently, such as
  controlling the visibility of certain elements, changing a daily
  quote, etc. See the
  \href{/docs/7-1/tutorials/-/knowledge_base/t/making-configurable-theme-settings}{Configurable
  Theme Settings tutorial} for more information.
\item
  \textbf{Context Contributor:} Exposes Java variables and functionality
  for you to use in your FreeMarker templates. This lets you use non-JSP
  templating languages for themes, ADTs, and any other templates used in
  Liferay DXP. See the
  \href{/docs/7-1/tutorials/-/knowledge_base/t/injecting-additional-context-variables-into-your-templates}{Context
  Contributors tutorial} or more information.
\item
  \textbf{Theme Contributor:} a package containing UI resources, not
  attached to a theme, that you want to include on every page. See the
  \href{/docs/7-1/tutorials/-/knowledge_base/t/packaging-independent-ui-resources-for-your-site}{Theme
  Contributors tutorial} for more information.
\item
  \textbf{Themelet:} small, extendable, and reusable pieces of code that
  contain CSS and JavaScript. It can be shared with other developers to
  provide common components for themes, and it only requires the files
  you want to extend. See the
  \href{/docs/7-1/tutorials/-/knowledge_base/t/creating-reusable-pieces-of-code-for-your-themes}{Themelets
  tutorial} for more information.
\end{itemize}

\subsection{Portlet Customizations and
Extensions}\label{portlet-customizations-and-extensions}

You can customize portlets with these mechanisms and extensions:

\begin{itemize}
\tightlist
\item
  \textbf{Portlet FTL Customizations:} customize the base template
  markup for all portlets. See the
  \href{/docs/7-1/tutorials/-/knowledge_base/t/theming-portlets\#portlet-ftl}{Theming
  Portlets tutorial} for more information.
\item
  \textbf{Application Display Templates (ADTs):} provides an alternate
  display style for a portlet. Note that not all portlets support ADTs.
  See the
  \href{/docs/7-1/user/-/knowledge_base/u/styling-widgets-with-application-display-templates}{Application
  Display Templates (ADTs) User Guide} for more information.
\item
  \textbf{Portlet Decorator:} lets you customize the exterior decoration
  for a portlet. See the
  \href{/docs/7-1/tutorials/-/knowledge_base/t/creating-configurable-styles-for-portlet-wrappers}{Portlet
  Decorators tutorial} for more information.
\item
  \textbf{Web Content Template:} defines how structures are displayed
  for web content. See the
  \href{/docs/7-1/user/-/knowledge_base/u/designing-web-content-with-templates}{Web
  Content Templates User Guide articles} for more information.
\end{itemize}

\begin{figure}
\centering
\includegraphics{./images/portal-layout-portlet-customizations.png}
\caption{There are several extension points for customizing portlets}
\end{figure}

\subsection{Theme Workflow}\label{theme-workflow}

Themes are built on top of one of the following base themes:

\begin{itemize}
\tightlist
\item
  \textbf{Unstyled:} provides basic markup, functions, and images for
  Portal
\item
  \textbf{Styled:} inherits from the Unstyled base theme and adds some
  styling on top
\end{itemize}

You can use the development tools you're most comfortable with so you
can focus on creating a well designed theme. The following Liferay tools
help you build themes:

\begin{itemize}
\tightlist
\item
  \href{/docs/7-1/reference/-/knowledge_base/r/theme-builder-gradle-plugin}{Theme
  Builder Gradle Plugin}
\item
  \href{/docs/7-1/tutorials/-/knowledge_base/t/creating-themes}{Liferay
  Theme Generator}
\item
  \href{/docs/7-1/tutorials/-/knowledge_base/t/creating-themes-with-liferay-ide}{Dev
  Studio}
\item
  \href{/docs/7-1/tutorials/-/knowledge_base/t/blade-cli}{Blade CLI}'s
  \href{/docs/7-1/reference/-/knowledge_base/r/theme-template}{Theme
  Template}.
\end{itemize}

Depending on the tool you choose (
\href{/docs/7-1/reference/-/knowledge_base/r/theme-reference-guide}{Theme
Generator},
\href{/docs/7-1/reference/-/knowledge_base/r/theme-builder-gradle-plugin}{Gradle},
\href{/docs/7-1/reference/-/knowledge_base/r/theme-template}{Blade CLI},
\href{/docs/7-1/reference/-/knowledge_base/r/theme-template}{Maven}, or
\href{/docs/7-1/reference/-/knowledge_base/r/theme-template}{Dev Studio}
), the theme anatomy is a bit different. The overall development process
is the same though:

\begin{enumerate}
\def\labelenumi{\arabic{enumi}.}
\item
  Mirror the structure of the files you want to modify. The main
  modifications are placed in the following files:

  \begin{itemize}
  \tightlist
  \item
    \texttt{portal\_normal.ftl}: main theme markup
  \item
    \texttt{\_custom.scss}: custom CSS styling
  \item
    \texttt{main.js}: the theme's JavaScript
  \end{itemize}
\item
  Build and deploy the theme to your Liferay DXP server.
\item
  Apply the theme
  \href{/docs/7-1/user/-/knowledge_base/u/page-set-look-and-feel}{through
  the Look and Feel menu} by selecting your
  \href{/docs/7-1/tutorials/-/knowledge_base/t/creating-a-thumbnail-preview-for-your-theme}{theme's
  thumbnail}.
\end{enumerate}

The finished theme is bundled as a WAR (Web application ARchive) file.

\noindent\hrulefill

\textbf{Note:} While developing your theme, we recommend that you enable
\href{/docs/7-1/tutorials/-/knowledge_base/t/using-developer-mode-with-themes}{Developer
Mode}. This un-minifies JS files and disable caching for CSS and
FreeMarker template files, making the debugging process much easier.

\noindent\hrulefill

During theme development, if you've built your theme with the Liferay
Theme Generator, you can use some helpful Gulp tasks to make the process
easier:

\begin{itemize}
\tightlist
\item
  \textbf{build:} builds your theme's files based on the specified base
  theme. See the
  \href{/docs/7-1/tutorials/-/knowledge_base/t/building-your-themes-files}{gulp
  build tutorial} for more information.
\item
  \textbf{extend:} sets the base theme or themelet to extend. See the
  \href{/docs/7-1/tutorials/-/knowledge_base/t/changing-your-base-theme}{gulp
  extend tutorial} for more information.
\item
  \textbf{init:} specifies the app server to deploy your theme to
  (automatically run during the initial creation of the theme). See the
  \href{/docs/7-1/tutorials/-/knowledge_base/t/configuring-your-themes-app-server}{gulp
  init tutorial} for more information.
\item
  \textbf{kickstart:} copies files from an existing theme into your
  theme to help kickstart it. See the
  \href{/docs/7-1/tutorials/-/knowledge_base/t/copying-an-existing-themes-files}{gulp
  kickstart tutorial} for more information.
\item
  \textbf{status:} lists the base theme/themelets that your theme
  extends. See the
  \href{/docs/7-1/tutorials/-/knowledge_base/t/listing-your-themes-extensions}{gulp
  status tutorial} for more information.
\item
  \textbf{watch:} watches for changes to your theme's files and
  automatically deploys them to the server when a change is made. See
  the
  \href{/docs/7-1/tutorials/-/knowledge_base/t/automatically-deploying-theme-changes}{gulp
  watch tutorial} for more information.
\end{itemize}

\section{Understanding the Page
Layout}\label{understanding-the-page-layout}

Knowing the layout's structure is crucial to targeting the correct
markup for styling, organizing your content, and creating your site.
Your page layout is unique to the requirements and design for your site.
The Unstyled theme's default page layout is organized into three key
sections:

\begin{itemize}
\tightlist
\item
  \textbf{Header:} contains the navigation, site logo and title (if
  shown), and sign-in link when the user isn't logged in
\item
  \textbf{Main Content:} contains the portlets or fragments for the page
\item
  \textbf{Footer:} contains additional information, such as the
  copyright or author
\end{itemize}

\begin{figure}
\centering
\includegraphics{./images/portal-layout-sections.png}
\caption{The page layout is broken into three key sections.}
\end{figure}

\subsection{Portlets or Fragments}\label{portlets-or-fragments}

The \texttt{\#content} \texttt{Section} makes up the majority of the
page. Portlets or fragments are contained inside the
\texttt{\#main-content} \texttt{div}. Liferay DXP ships with a default
set of applications that provide common functionality, such as forums
and Wikis, documents and media, blogs, and more. For more information on
using Liferay DXP and its native portlets, see the
\href{/docs/7-1/user}{User \& Admin documentation}. You can also create
custom portlets for your site. Portlets can be added via the Add Menu
(referred to as widget), included in a sitemap through the
\href{/docs/7-1/tutorials/-/knowledge_base/t/importing-resources-with-a-theme}{Resources
Importer}, or they can be
\href{/docs/7-1/tutorials/-/knowledge_base/t/embedding-portlets-in-themes}{embedded
in the page's theme}. See the
\href{/docs/7-1/tutorials/-/knowledge_base/t/portlets}{portlet tutorials
section} for more information on creating and developing portlets.

You can target the elements and IDs shown in the table below to style
the page:

\noindent\hrulefill

\begin{longtable}[]{@{}
  >{\raggedright\arraybackslash}p{(\columnwidth - 4\tabcolsep) * \real{0.3333}}
  >{\raggedright\arraybackslash}p{(\columnwidth - 4\tabcolsep) * \real{0.3333}}
  >{\raggedright\arraybackslash}p{(\columnwidth - 4\tabcolsep) * \real{0.3333}}@{}}
\toprule\noalign{}
\begin{minipage}[b]{\linewidth}\raggedright
Element
\end{minipage} & \begin{minipage}[b]{\linewidth}\raggedright
ID
\end{minipage} & \begin{minipage}[b]{\linewidth}\raggedright
Description
\end{minipage} \\
\midrule\noalign{}
\endhead
\bottomrule\noalign{}
\endlastfoot
div & \#wrapper & The container div for the page contents \\
header & \#banner & The page's header \\
section & \#content \textgreater{} \#main-content & The main contents of
the page (portlets or fragments) \\
footer & \#footer & The page's footer \\
\end{longtable}

\noindent\hrulefill

\begin{figure}
\centering
\includegraphics{./images/portal-layout-elements.png}
\caption{Each section of the page has elements and IDs that you can
target for styling.}
\end{figure}

As shown in the diagram above, you can also add
\href{/docs/7-1/user/-/knowledge_base/u/creating-fragments}{fragments}
to a page. You can have a page that contains portlets or a content page
that contains fragments, not both. Fragments are components---composed
of CSS, JavaScript, and HTML---that provide key pieces of functionality
for the page (i.e.~a carousel or banner). Liferay DXP provides an editor
for creating collections of fragments that you can then add to the page.
These fragments can be edited on the page to suit your vision.

\subsection{Layout Templates, Page Templates, and Site
Templates}\label{layout-templates-page-templates-and-site-templates}

The page layout within the \texttt{\#content} Section is determined by
the
\href{/docs/7-1/tutorials/-/knowledge_base/t/layout-templates-intro}{Layout
Template}. Several layout templates are included out-of-the-box. You can
also
\href{/docs/7-1/tutorials/-/knowledge_base/t/creating-layout-templates-manually}{create
custom layout templates manually} or with the
\href{/docs/7-1/tutorials/-/knowledge_base/t/creating-layout-templates-with-the-themes-generator}{Liferay
Theme Generator's layout sub-generator}.

Layout templates can be pre-configured depending on the
\href{/docs/7-1/user/-/knowledge_base/u/page-types-and-templates}{page
type} you choose when the page is created. Along with setting the types
of portlets to include on the page, the page template may also define
the default layout template for the page. Climbing further up the scope
chain, you can create
\href{/docs/7-1/user/-/knowledge_base/u/building-sites-from-templates}{site
templates}, which can define the pages, page templates, layout
templates, and theme(s) to use for site pages.

\subsection{Product Navigation Sidebars and
Panels}\label{product-navigation\noindent\hrulefills-and-panels}

The main page layout also contains a few notable sidebars an
administrative user can trigger through the Control Menu. These are
listed below:

\begin{itemize}
\tightlist
\item
  \textbf{Add Menu:} lets you add portlets (widgets) and fragments (if
  applicable) to the page
\item
  \textbf{Control Menu:} provides the main navigation for accessing the
  Add Menu, Product Menu, and Simulation Panel
\item
  \textbf{Product Menu:} contains administrative apps, configuration
  settings, and user account settings, profile, and dashboard page
\item
  \textbf{Simulation Panel:} simulates how the page appears on different
  devices
\end{itemize}

\begin{figure}
\centering
\includegraphics{./images/portal-layout-nav-control-menu.png}
\caption{Remember to account for the product navigation sidebars and
panels when styling your site.}
\end{figure}

\begin{figure}
\centering
\includegraphics{./images/portal-layout-nav-add-menu.png}
\caption{The Add Menu pushes the main contents to the left.}
\end{figure}

\begin{figure}
\centering
\includegraphics{./images/portal-layout-nav-product-menu.png}
\caption{The Product Menu pushes the main contents to the right.}
\end{figure}

\begin{figure}
\centering
\includegraphics{./images/portal-layout-nav-simulation-panel.png}
\caption{The Simulation Panel pushes the main contents to the left.}
\end{figure}

When styling the page, you must keep the navigation menus in mind,
especially for absolutely positioned elements, such as a fixed navbar.
If the user is logged in and can view the Control Menu, the fixed navbar
must have a top margin equal to the Control Menu's height. See the
\href{/docs/7-1/tutorials/-/knowledge_base/t/product-navigation}{Product
Navigation tutorials section} for more information on customizing these
menus.

\chapter{Gradle}\label{gradle}

Liferay provides plugins that you can apply to your Gradle project. This
reference documentation describes how to apply and use Liferay's Gradle
plugins.

\textbf{Important:} If you're using
\href{/docs/7-1/tutorials/-/knowledge_base/t/liferay-workspace}{Liferay
Workspace} to create Liferay apps, most of the Liferay Gradle plugins
covered in this section are already applied by default. The
\href{https://github.com/liferay/liferay-portal/tree/master/modules/sdk/gradle-plugins-workspace}{com.liferay.gradle.plugins.workspace}
and
\href{https://github.com/liferay/liferay-portal/tree/master/modules/sdk/gradle-plugins}{com.liferay.gradle.plugins}
dependencies provide them, both of which are preset in workspace by
default.

Do not apply a Liferay Gradle plugin to an app that already has access
to it.

Each article in this section describes how to apply the plugin, what
Gradle tasks the plugin provides, the plugin's configuration properties,
and the plugin's dependencies.

\section{Resolving Common Output Errors Reported by the resolve
Task}\label{resolving-common-output-errors-reported-by-the-resolve-task}

\href{/docs/7-1/tutorials/-/knowledge_base/t/liferay-workspace}{Liferay
Workspace} provides the \texttt{resolve} Gradle task to
\href{/docs/7-1/tutorials/-/knowledge_base/t/resolving-your-modules}{validate
modules}. This is very useful for finding issues and reporting them as
output before deployment. For more information on running this task from
Liferay Workspace, see the
\href{/docs/7-1/tutorials/-/knowledge_base/t/validating-modules-against-the-target-platform}{Validating
Modules Against the Target Platform} tutorial section. For general help
with OSGi related issues, visit the
\href{/docs/7-0/tutorials/-/knowledge_base/t/troubleshooting}{Troubleshooting
FAQ} tutorial section.

For help interpreting the \texttt{resolve} task's output, see the list
below for common output errors, what they mean, and how to fix them.

\subsection{Missing Import Error}\label{missing-import-error}

When your module refers to an unavailable import, the container throws
this error. For example, suppose you have a module \texttt{test-service}
that depends on the \texttt{com.google.common.base} package. If the
container can't find that package, it throws this error:

\begin{verbatim}
Resolution exception in project 'modules:test-service': Unresolved requirements in root project 'modules:test-service':
    Mandatory:
        [osgi.wiring.package ] com.google.common.base; version=[23.0.0,24.0.0)
        [osgi.identity       ] test.service
\end{verbatim}

This kind of error can also occur when separate modules require
different versions of another module. If you have \emph{module A}
requiring \emph{module Test version 1} and \emph{module B} requiring
\emph{module Test version 4}, without running the resolver, both modules
A and B would compile successfully. When they were deployed, however,
one would fail in the OSGi runtime because both dependencies cannot be
satisfied. These types of scenarios are difficult to diagnose, but with
the \texttt{resolve} task, can be found with ease.

To fix missing import errors, you may need to adjust the
\href{/docs/7-1/tutorials/-/knowledge_base/t/exporting-packages}{export}
and/or
\href{/docs/7-1/tutorials/-/knowledge_base/t/importing-packages}{import}
configuration of your modules. Also, see the
\href{/docs/7-1/tutorials/-/knowledge_base/t/adding-third-party-libraries-to-a-module}{Resolving
Third Party Library Package Dependencies} tutorial for more information
on resolving import errors. Sometimes, this kind of error can be solved
by editing the \texttt{resolve} task's list of capabilities. See the
\href{/docs/7-1/tutorials/-/knowledge_base/t/adding-third-party-libraries-to-a-module}{Resolving
Third Party Library Package Dependencies} section to learn how to do
this.

\subsection{Missing Service Reference}\label{missing-service-reference}

If your module references a non-existent service, an error is thrown.
This is helpful because service reference issues are hard to diagnose
during deployment without using the
\href{/docs/7-1/reference/-/knowledge_base/r/using-the-felix-gogo-shell}{Gogo
Shell}.

For example, if your module \texttt{test-portlet} references a service
(e.g., \texttt{test.api.TestApi}) it does not have access to, the
following error is thrown:

\begin{verbatim}
Resolution exception in project 'modules:test-portlet': Unresolved requirements in project 'modules:test-portlet':
    Mandatory:
        [osgi.identity ] test.portlet
        [osgi.service  ] objectClass=test.api.TestApi
\end{verbatim}

To fix this, you must make the service available to your module. If
you're expecting the service to be provided by your target platform,
check to make sure it's being provided. If it's a service provided by a
custom module, check that service provider module and ensure it's
correctly providing that service to your module. To check the target
platform for available services, follow the steps below:

\begin{enumerate}
\def\labelenumi{\arabic{enumi}.}
\item
  Start your target platform instance.
\item
  Open the Gogo shell.
\item
  List all services containing a keyword by running
  \texttt{services\ \textbar{}\ grep\ \ \ \ \ "SERVICE\_NAME"}. It's
  easiest to do this rather than listing all services since there are
  usually too many to sift through.
\item
  You can also list services provided by a component. Run
  \texttt{lb\ -s} to list all provided bundles by their bundle symbolic
  name (BSN). Find the BSN for the desired component and then run
  \texttt{scr:info\ \textless{}BSN\textgreater{}}.
\end{enumerate}

If you're unable to track down your missing service, it may be provided
by a customized Liferay DXP core feature or an external Liferay DXP
feature. If this is the case, it isn't included in the target platform's
default capabilities. You can make the custom service capability
available to reference by
\href{/docs/7-1/tutorials/-/knowledge_base/t/modifying-the-target-platforms-capabilities\#depending-on-a-customized-distribution-of-product}{generating
a new custom distro JAR}.

\subsection{Missing Fragment Host}\label{missing-fragment-host}

Referring to a non-existent fragment host throws an error. For example,
if your \texttt{test.login} fragment is configured to modify a fragment
host named \texttt{com.liferay.login.web} that cannot be referenced, the
following error is thrown:

\begin{verbatim}
Resolution exception in project 'modules:test.login': Unresolved requirements in project 'modules:test-login':
    Mandatory:
        [osgi.identity    ] test.login
        [osgi.wiring.host ] com.liferay.login.web; version=1.0.10
\end{verbatim}

Configuring a fragment host in your module is typically done with the
\texttt{Fragment-Host} header in the \texttt{bnd.bnd} file:

\begin{verbatim}
Fragment-Host: com.liferay.login.web;bundle-version="[1.0.0,1.0.1)"
\end{verbatim}

To fix this, inspect your target platform to ensure it includes the JAR
you're attempting to add a fragment for. Your fragment host header may
be referencing an incorrect bundle symbolic name (BSN) or version. The
easiest way to check this is by using the
\href{/docs/7-1/reference/-/knowledge_base/r/using-the-felix-gogo-shell}{Gogo
Shell}. Follow the steps below to find the bundle symbolic name:

\begin{enumerate}
\def\labelenumi{\arabic{enumi}.}
\item
  Start your target platform instance.
\item
  Open the Gogo shell.
\item
  List all installed bundles by BSN with the command \texttt{lb\ -s}.
  You can search through the output to find the BSN. If you already know
  the BSN and want to check the version, run
  \texttt{lb\ -s\ \textbar{}\ grep\ "\textless{}BSN\textgreater{}"}.
\end{enumerate}

Once you know the correct BSN/version to reference, update your
\texttt{Fragment-Host} header to resolve the error.

For more information on fragments, see the
\href{/docs/7-0/tutorials/-/knowledge_base/t/overriding-a-modules-jsps}{JSP
Overrides Using OSGi Fragments} tutorial.

\section{App Javadoc Builder Gradle
Plugin}\label{app-javadoc-builder-gradle-plugin}

The App Javadoc Builder Gradle plugin lets you generate API
documentation as a single, combined HTML document for an application
that spans different subprojects, each one representing a different
component of the same application.

The plugin has been successfully tested with Gradle 4.10.2.

\subsection{Usage}\label{usage}

To use the plugin, include it in the build script of the root project:

\begin{verbatim}
buildscript {
    dependencies {
        classpath group: "com.liferay", name: "com.liferay.gradle.plugins.app.javadoc.builder", version: "1.2.2"
    }

    repositories {
        maven {
            url "https://repository-cdn.liferay.com/nexus/content/groups/public"
        }
    }
}

apply plugin: "com.liferay.app.javadoc.builder"
\end{verbatim}

The App Javadoc Builder plugin automatically applies the
\href{https://docs.gradle.org/current/userguide/standard_plugins.html\#N135C1}{\texttt{base}}
and \texttt{reporting-base} plugins.

\subsection{Project Extension}\label{project-extension}

The App Javadoc Builder plugin exposes the following properties through
the extension named \texttt{appJavadocBuilder}:

Property Name \textbar{} Type \textbar{} Default Value \textbar{}
Description \texttt{copyTags} \textbar{} \texttt{boolean} \textbar{}
\texttt{true} \textbar{} Whether to copy the custom block tags
configuration from the subprojects. It sets the Javadoc
\href{http://docs.oracle.com/javase/8/docs/technotes/tools/windows/javadoc.html\#tag}{\texttt{-tag}}
argument for the \hyperref[appjavadoc]{\texttt{appJavadoc}} task.
\texttt{doclintDisabled} \textbar{} \texttt{boolean} \textbar{}
\texttt{true} on JDK8+, \texttt{false} otherwise. \textbar{} Whether to
ignore Javadoc errors. It sets the Javadoc
\href{docs.oracle.com/javase/8/docs/technotes/tools/windows/javadoc.html\#BEJEFABE}{\texttt{-Xdoclint}}
and
\href{http://docs.oracle.com/javase/8/docs/technotes/tools/windows/javadoc.html\#CHDGFHAA}{\texttt{-quiet}}
arguments for the \hyperref[appjavadoc]{\texttt{appJavadoc}} task.
\texttt{groupNameClosure} \textbar{}
\texttt{Closure\textless{}String\textgreater{}} \textbar{} The
subproject's description, or the subproject's name if the description is
empty. \textbar{} The closure invoked in order to get the group heading
for a subproject. The given closure is passed a
\href{https://docs.gradle.org/current/javadoc/org/gradle/api/Project.html}{\texttt{Project}}
as its parameter. If \texttt{groupPackages} is \texttt{false}, this
property is not used. \texttt{groupPackages} \textbar{} \texttt{boolean}
\textbar{} \texttt{true} \textbar{} Whether to separate packages on the
overview page based on the subprojects they belong to. It sets the
\href{docs.oracle.com/javase/8/docs/technotes/tools/unix/javadoc.html\#CHDIGGII}{\texttt{-group}}
argument for the \hyperref[appjavadoc]{\texttt{appJavadoc}} task.
\texttt{subprojects} \textbar{}
\texttt{Set\textless{}Project\textgreater{}} \textbar{}
\texttt{project.subprojects} \textbar{} The subprojects to include in
the API documentation of the app.

The same extension exposes the following methods:

Method \textbar{} Description
\texttt{AppJavadocBuilderExtension\ onlyIf(Closure\textless{}Boolean\textgreater{}\ onlyIfClosure)}
\textbar{} Includes a subproject in the API documentation if the given
closure returns \texttt{true}. The closure is evaluated at the end of
the subproject configuration phase and is passed a single parameter: the
subproject. If the closure returns \texttt{false}, the subproject is not
included in the API documentation.
\texttt{AppJavadocBuilderExtension\ onlyIf(Spec\textless{}Project\textgreater{}\ onlyIfSpec)}
\textbar{} Includes a subproject in the API documentation if the given
spec is satisfied. The spec is evaluated at the end of the subproject
configuration phase. If the spec is not satisfied, the subproject is not
included in the API documentation.
\texttt{AppJavadocBuilderExtension\ subprojects(Iterable\textless{}Project\textgreater{}\ subprojects)}
\textbar{} Include additional projects in the API documentation of the
app.
\texttt{AppJavadocBuilderExtension\ subprojects(Project...\ subprojects)}
\textbar{} Include additional projects in the API documentation of the
app.

\subsection{Tasks}\label{tasks}

The plugin adds two tasks to your project:

Name \textbar{} Depends On \textbar{} Type \textbar{} Description
\texttt{appJavadoc} \textbar{} The \texttt{javadoc} tasks of the
subprojects. \textbar{}
\href{https://docs.gradle.org/current/dsl/org.gradle.api.tasks.javadoc.Javadoc.html}{\texttt{Javadoc}}
\textbar{} Generates Javadoc API documentation for the app.
\texttt{jarAppJavadoc} \textbar{} \texttt{appJavadoc} \textbar{}
\href{https://docs.gradle.org/current/dsl/org.gradle.api.tasks.bundling.Jar.html}{\texttt{Jar}}
\textbar{} Assembles a JAR archive containing the Javadoc files for this
app.

The \texttt{appJavadoc} task is automatically configured with sensible
defaults:

Property Name \textbar{} Default Value
\href{https://docs.gradle.org/current/dsl/org.gradle.api.tasks.javadoc.Javadoc.html\#org.gradle.api.tasks.javadoc.Javadoc:classpath}{\texttt{classpath}}
\textbar{} The \texttt{javadoc.classpath} of all the subprojects.
\href{https://docs.gradle.org/current/dsl/org.gradle.api.tasks.javadoc.Javadoc.html\#org.gradle.api.tasks.javadoc.Javadoc:destinationDir}{\texttt{destinationDir}}
\textbar{} \texttt{\$\{project.buildDir\}/docs/javadoc}
\href{https://docs.gradle.org/current/javadoc/org/gradle/external/javadoc/MinimalJavadocOptions.html\#getEncoding()}{\texttt{options.encoding}}
\textbar{} \texttt{"UTF-8"}
\href{https://docs.gradle.org/current/dsl/org.gradle.api.tasks.javadoc.Javadoc.html\#org.gradle.api.tasks.javadoc.Javadoc:source}{\texttt{source}}
\textbar{} The \texttt{javadoc.source} of all the subprojects.
\href{https://docs.gradle.org/current/dsl/org.gradle.api.tasks.javadoc.Javadoc.html\#org.gradle.api.tasks.javadoc.Javadoc:title}{\texttt{title}}
\textbar{} \texttt{project.reporting.apiDocTitle}

\section{Baseline Gradle Plugin}\label{baseline-gradle-plugin}

The Baseline Gradle plugin lets you verify that the OSGi
\href{http://semver.org/}{semantic versioning} rules are obeyed by your
OSGi bundle.

When you run the \hyperref[baseline]{\texttt{baseline}} task, the plugin
\emph{baselines} the new bundle against the latest released non-snapshot
bundle (i.e., the \emph{baseline}). That is, it compares the public
exported API of the new bundle with the baseline. If there are any
changes, it uses the OSGi semantic versioning rules to calculate the
minimum new version. If the new bundle has a lower version, errors are
thrown.

The plugin has been successfully tested with Gradle 4.10.2.

\subsection{Usage}\label{usage-1}

To use the plugin, include it in your build script:

\begin{verbatim}
buildscript {
    dependencies {
        classpath group: "com.liferay", name: "com.liferay.gradle.plugins.baseline", version: "2.1.0"
    }

    repositories {
        maven {
            url "https://repository-cdn.liferay.com/nexus/content/groups/public"
        }
    }
}

apply plugin: "com.liferay.baseline"
\end{verbatim}

The Baseline plugin automatically applies the
\href{https://docs.gradle.org/current/userguide/java_plugin.html}{\texttt{java}}
and
\href{https://docs.gradle.org/current/userguide/standard_plugins.html\#sec:base_plugins}{\texttt{reporting-base}}
plugins.

Since the plugin needs to download the baseline, you have to configure a
\href{https://docs.gradle.org/current/userguide/artifact_dependencies_tutorial.html\#sec:repositories_tutorial}{repository}
that hosts it; for example, the central Maven 2 repository:

\begin{verbatim}
repositories {
    mavenCentral()
}
\end{verbatim}

\subsection{Project Extension}\label{project-extension-1}

The Baseline plugin exposes the following properties through the
\texttt{baselineConfiguration} extension:

Property Name \textbar{} Type \textbar{} Default Value \textbar{}
Description \texttt{allowMavenLocal} \textbar{} \texttt{boolean}
\textbar{} \texttt{false} \textbar{} Whether to let the baseline come
from the local Maven cache (by default: \texttt{\$\{user.home\}/.m2}).
If the local Maven cache is not
\href{https://docs.gradle.org/current/userguide/dependency_management.html\#sub:maven_local}{configured}
as a project repository, this property has no effect.
\texttt{lowestBaselineVersion} \textbar{} \texttt{String} \textbar{}
\texttt{"1.0.0"} \textbar{} The greatest project version to ignore for
the baseline check. If the
\href{https://docs.gradle.org/current/dsl/org.gradle.api.tasks.bundling.Jar.html\#org.gradle.api.tasks.bundling.Jar:version}{project
version} is less than or equal to the value of this property, the
\hyperref[baseline]{\texttt{baseline}} task is skipped.
\texttt{lowestMajorVersion} \textbar{} \texttt{Integer} \textbar{}
Content of the file
\texttt{\$\{project.projectDir\}/.lfrbuild-lowest-major-version}, where
the default file name can be changed by setting the project property
\texttt{baseline.lowest.major.version.file}. \textbar{} The lowest major
version of the released artifact to use in the baseline check.
\texttt{lowestMajorVersionRequired} \textbar{} \texttt{boolean}
\textbar{} \texttt{false} \textbar{} Whether to fail the build if the
\hyperref[lowestmajorversion]{\texttt{lowestMajorVersion}} is not
specified.

If the \texttt{lowestMajorVersion} is not specified, the plugin runs the
check using the most recent released non-snapshot bundle as baseline,
which matches the
\href{http://ant.apache.org/ivy/history/latest-milestone/settings/version-matchers.html}{version
range} \texttt{(,\$\{project.version\})}. Otherwise, if the
\texttt{lowestMajorVersion} is equal to a value \texttt{L} and the
project has version \texttt{M.x.y} (with \texttt{L} less or equal than
\texttt{M}), multiple checks are performed in order, using the following
version ranges as baseline:

\begin{enumerate}
\def\labelenumi{\arabic{enumi}.}
\tightlist
\item
  \texttt{{[}L.0.0,\ (L\ +\ 1).0.0)}
\item
  \texttt{{[}(L\ +\ 1).0.0,\ (L\ +\ 2).0.0)}
\item
  \ldots{}
\item
  \texttt{{[}(M\ -\ 2).0.0,\ (M\ -\ 1).0.0)}
\item
  \texttt{{[}(M\ -\ 1).0.0,\ M.0.0)}
\item
  \texttt{{[}M.0.0,\ M.x.y)}
\end{enumerate}

The first failing check fails the whole build.

\subsection{Tasks}\label{tasks-1}

The plugin adds one task to your project:

Name \textbar{} Depends On \textbar{} Type \textbar{} Description
\texttt{baseline} \textbar{}
\href{(https://docs.gradle.org/current/userguide/java_plugin.html\#sec:jar)}{\texttt{jar}}
\textbar{} \hyperref[baselinetask]{\texttt{BaselineTask}} \textbar{}
Compares the public API of this project with the public API of the
previous released version, if found.

The \texttt{baseline} task is automatically configured with sensible
defaults:

Property Name \textbar{} Default Value
\hyperref[baselineconfiguration]{\texttt{baselineConfiguration}}
\textbar{}
\hyperref[baseline-dependency]{\texttt{configurations.baseline}}
\hyperref[bndfile]{\texttt{bndFile}} \textbar{}
\texttt{\$\{project.projectDir\}/bnd.bnd}
\hyperref[newjarfile]{\texttt{newJarFile}} \textbar{}
\href{https://docs.gradle.org/current/dsl/org.gradle.api.tasks.bundling.Jar.html\#org.gradle.api.tasks.bundling.Jar:archivePath}{\texttt{project.tasks.jar.archivePath}}
\hyperref[sourcedir]{\texttt{sourceDir}} \textbar{} The first
\texttt{resources} directory of the \texttt{main} source set (by
default: \texttt{src/main/resources}).

\subsubsection{BaselineTask}\label{baselinetask}

\paragraph{Task Properties}\label{task-properties}

Property Name \textbar{} Type \textbar{} Default Value \textbar{}
Description \texttt{baselineConfiguration} \textbar{}
\texttt{Configuration} \textbar{} \texttt{null} \textbar{} The
configuration that contains exactly one dependency to the baseline
bundle. \texttt{bndFile} \textbar{} \texttt{File} \textbar{}
\texttt{null} \textbar{} The BND file of the project. If provided, the
task will automatically update the
\href{http://bnd.bndtools.org/heads/bundle_version.html}{\texttt{Bundle-Version}}
header. \texttt{forceCalculatedVersion} \textbar{} \texttt{boolean}
\textbar{} \texttt{false} \textbar{} Whether to fail the baseline check
if the \texttt{Bundle-Version} has been excessively increased.
\texttt{ignoreExcessiveVersionIncreases} \textbar{} \texttt{boolean}
\textbar{} \texttt{false} \textbar{} Whether to ignore excessive package
version increase warnings. \texttt{ignoreFailures} \textbar{}
\texttt{boolean} \textbar{} \texttt{false} \textbar{} Whether the build
should not break when semantic versioning errors are found.
\texttt{logFile} \textbar{} \texttt{File} \textbar{} \texttt{null}
\textbar{} The file to which the results of the baseline check are
written. \emph{(Read-only)} \texttt{logFileName} \textbar{}
\texttt{String} \textbar{} \texttt{"baseline/\$\{task.name\}.log"}
\textbar{} The name of the file to which the results of the baseline
check are written. If the \texttt{reporting-base} plugin is applied, the
file name is relative to
\href{https://docs.gradle.org/current/dsl/org.gradle.api.reporting.ReportingExtension.html\#org.gradle.api.reporting.ReportingExtension:baseDir}{\texttt{reporting.baseDir}};
otherwise, it's relative to the project directory. \texttt{newJarFile}
\textbar{} \texttt{File} \textbar{} \texttt{null} \textbar{} The file of
the new OSGi bundle. \texttt{reportDiff} \textbar{} \texttt{boolean}
\textbar{} \texttt{true} if the project property
\texttt{baseline.jar.report.level} has either value \texttt{"diff"} or
\texttt{"persist"}; \texttt{false} otherwise \textbar{} Whether to show
a granular, differential report of all changes that occurred in the
exported packages of the OSGi bundle. \texttt{reportOnlyDirtyPackages}
\textbar{} \texttt{boolean} \textbar{} Value of the project property
\texttt{baseline.jar.report.only.dirty.packages} if specified;
\texttt{true} otherwise. \textbar{} Whether to show only packages with
API changes in the report. \texttt{sourceDir} \textbar{} \texttt{File}
\textbar{} \texttt{null} \textbar{} The directory to which the
\href{http://bnd.bndtools.org/chapters/170-versioning.html\#versioning-packages}{\texttt{packageinfo}}
files are generated or updated.

The properties of type \texttt{File} support any type that can be
resolved by
\href{https://docs.gradle.org/current/dsl/org.gradle.api.Project.html\#org.gradle.api.Project:file(java.css.Object)}{\texttt{project.file}}.
Moreover, it is possible to use Closures and Callables as values for the
\texttt{String} properties to defer evaluation until task execution.

\subsubsection{Helper Tasks}\label{helper-tasks}

If the \hyperref[lowestmajorversion]{\texttt{lowestMajorVersion}}
property is specified with a value \texttt{L}, the plugin creates a
series of helper tasks of type
\hyperref[baselinetask]{\texttt{BaselineTask}} at the end of the
\href{https://docs.gradle.org/current/userguide/build_lifecycle.html\#N11BAE}{project
evaluation}, one for each major version between \texttt{L} and the major
version \texttt{M} of the project:

\begin{enumerate}
\def\labelenumi{\arabic{enumi}.}
\tightlist
\item
  Task \texttt{baseline\$\{L\ +\ 1\}}, which depends on
  \texttt{baseline\$\{L\ +\ 2\}} and uses the version range
  \texttt{{[}(L\ +\ 1).0.0,\ (L\ +\ 2).0.0)} as baseline.
\item
  Task \texttt{baseline\$\{L\ +\ 2\}}, which depends on
  \texttt{baseline\$\{L\ +\ 3\}} and uses the version range
  \texttt{{[}(L\ +\ 2).0.0,\ (L\ +\ 3).0.0)} as baseline.
\item
  \ldots{}
\item
  Task \texttt{baseline\$\{M\ -\ 2\}}, which depends on
  \texttt{baseline\$\{M\ -\ 1\}} and uses the version range
  \texttt{{[}(M\ -\ 2).0.0,\ (M\ -\ 1).0.0)} as baseline.
\item
  Task \texttt{baseline\$\{M\ -\ 1\}}, which depends on
  \texttt{baseline\$\{M\}} and uses the version range
  \texttt{{[}(M\ -\ 1).0.0,\ M.0.0)} as baseline.
\item
  Task \texttt{baseline\$\{M\}}, which uses the version range
  \texttt{{[}M.0.0,\ M.x.y)} as baseline.
\end{enumerate}

The \texttt{baseline} task is also configured to use the version range
\texttt{{[}L.0.0,\ (L\ +\ 1).0.0)} as baseline, and to depend on the
task \texttt{baseline\$\{L\ +\ 1\}}. This means that running the
\texttt{baseline} task runs the baseline check against multiple
versions, starting from the most recent \texttt{M} and going back to
\texttt{L}.

Moreover, all tasks except \texttt{baseline\$\{M\}} have the property
\hyperref[ignoreexcessiveversionincreases]{\texttt{ignoreExcessiveVersionIncreases}}
set to \texttt{true}.

\subsection{Additional Configuration}\label{additional-configuration}

There are additional configurations that can help you baseline your OSGi
bundle.

\subsubsection{Baseline Dependency}\label{baseline-dependency}

The plugin creates a configuration called \texttt{baseline} with a
default dependency to a released non-snapshot version of the bundle:

\begin{itemize}
\tightlist
\item
  version range \texttt{{[}L.0.0,\ (L\ +\ 1).0.0)} if the
  \hyperref[lowestmajorversion]{\texttt{lowestMajorVersion}} property is
  specified with a value \texttt{L}.
\item
  version range \texttt{(,\$\{project.version\})} otherwise.
\end{itemize}

It is possible to override this setting and use a different version of
the bundle as baseline.

\subsubsection{System Properties}\label{system-properties}

It is possible to set the default values of the
\hyperref[ignorefailures]{\texttt{ignoreFailures}} property for a
\texttt{BaselineTask} task via system properties:

\begin{verbatim}
-D${task.name}.ignoreFailures=true
\end{verbatim}

For example, run the following Bash command to execute the baseline
check without breaking the build, in case of errors:

\begin{verbatim}
./gradlew baseline -Dbaseline.ignoreFailures=true
\end{verbatim}

\section{Change Log Builder Gradle
Plugin}\label{change-log-builder-gradle-plugin}

The Change Log Builder Gradle plugin lets you generate and maintain a
change log file based on the Git commits in your project. A change log
file generated by this plugin looks like this

\begin{verbatim}
    #
    # Bundle Version 1.0.1
    #
    9c77ff4c95cb1a325db3bdd089be105206e8b63c^..b421f00ac84b065685b131833fecc594fc01c760=LPS-123 LPS-1321

    #
    # Bundle Version 1.0.2
    #
    b421f00ac84b065685b131833fecc594fc01c760^..bc15d8d84e12b9544f78e4e3743c510dbaec2d89=LPS-456
\end{verbatim}

Every time the \hyperref[buildchangelog]{\texttt{buildChangeLog}} task
is executed, a new line is added to the change log, which lists all Git
\hyperref[ticketidprefixes]{commit prefixes} (usually issue ticket IDs)
that occurred in a certain range. The end of the range is always the tip
of the current branch. The start range can vary, depending on the case:

\begin{itemize}
\tightlist
\item
  If \texttt{buildChangeLog} has never been executed for the project,
  the change log does not exist. Therefore, the most recent commit from
  two years ago is used for the range start.
\item
  If a change log already exists for your project, the start range
  begins at the range end of the last line in the change log.
\end{itemize}

The plugin has been successfully tested with Gradle 4.10.2.

\subsection{Usage}\label{usage-2}

To use the plugin, include it in your build script:

\begin{verbatim}
buildscript {
    dependencies {
        classpath group: "com.liferay", name: "com.liferay.gradle.plugins.change.log.builder", version: "1.1.3"
    }

    repositories {
        maven {
            url "https://repository-cdn.liferay.com/nexus/content/groups/public"
        }
    }
}

apply plugin: "com.liferay.change.log.builder"
\end{verbatim}

\subsection{Tasks}\label{tasks-2}

The plugin adds one task to your project:

Name \textbar{} Depends On \textbar{} Type \textbar{} Description
\texttt{buildChangeLog} \textbar{} - \textbar{}
\hyperref[buildchangelogtask]{\texttt{BuildChangeLogTask}} \textbar{}
Builds the change log file for this project.

The \texttt{buildChangeLog} task is automatically configured with
sensible defaults, depending on whether the
\href{https://docs.gradle.org/current/userguide/java_plugin.html}{\texttt{java}}
plugin is applied:

Property Name \textbar{} Default Value
\hyperref[changelogheader]{\texttt{changeLogHeader}} \textbar{}
\texttt{"Bundle\ Version\ \$\{project.version\}"}
\hyperref[changelogfile]{\texttt{changeLogFile}} \textbar{}

\textbf{If the \texttt{java} plugin is applied:} The
\texttt{META-INF/liferay-releng.changelog} file in the first
\texttt{resources} directory of the \texttt{main} source set (by
default, \texttt{src/main/resources/META-INF/liferay-releng.changelog}).

\textbf{Otherwise:}
\texttt{"\$\{project.projectDir\}/liferay-releng.changelog"}

\hyperref[dirs]{\texttt{dirs}} \textbar{}
\texttt{{[}project.projectDir{]}}

\subsubsection{BuildChangeLogTask}\label{buildchangelogtask}

\paragraph{Task Properties}\label{task-properties-1}

Property Name \textbar{} Type \textbar{} Default Value \textbar{}
Description \texttt{changeLogFile} \textbar{} \texttt{File} \textbar{}
\texttt{null} \textbar{} The change log file to build.
\texttt{changeLogHeader} \textbar{} \texttt{String} \textbar{}
\texttt{null} \textbar{} The header for the new line in the change log.
\texttt{dirs} \textbar{} \texttt{FileCollection} \textbar{}
\texttt{{[}{]}} \textbar{} The directories to consider when listing the
commits in the range specified. \texttt{gitDir} \textbar{} \texttt{File}
\textbar{} \texttt{project.rootDir} \textbar{} The base directory to
start searching for the \texttt{.git} directory. The search proceeds in
all the ancestors of the directory specified. \texttt{rangeEnd}
\textbar{} \texttt{String} \textbar{} \texttt{null} \textbar{} The hash
of the last commit to consider. If not set, it corresponds to the range
end of the last line in the change log, or the most recent commit from
at least two years ago if the change log file does not exist yet.
\texttt{rangeStart} \textbar{} \texttt{String} \textbar{} \texttt{null}
\textbar{} The hash of the first commit to consider. If not set, it
corresponds to the hash of the tip of the current branch.
\texttt{ticketIdPrefixes} \textbar{}
\texttt{Set\textless{}String\textgreater{}} \textbar{}
\texttt{{[}"CLDSVCS",\ "LPS",\ "SOS",\ "SYNC"{]}} \textbar{} The valid
prefix of the Git commit messages to add to the change log. For example,
if a commit message is \texttt{"LPS-123\ Bugfix"}, \texttt{"LPS-123"}
will be added to the change log.

The properties of type \texttt{File} support any type that can be
resolved by
\href{https://docs.gradle.org/current/dsl/org.gradle.api.Project.html\#org.gradle.api.Project:file(java.css.Object)}{\texttt{project.file}}.
Moreover, it is possible to use Closures and Callables as values for the
\texttt{String} properties to defer evaluation until task execution.

\paragraph{Task Methods}\label{task-methods}

Method \textbar{} Description
\texttt{BuildChangeLogTask\ dirs(Iterable\textless{}?\textgreater{}\ dirs)}
\textbar{} Adds directories to consider when listing the commits in the
range specified. \texttt{BuildChangeLogTask\ dirs(Object...\ dirs)}
\textbar{} Adds directories to consider when listing the commits in the
range specified.
\texttt{BuildChangeLogTask\ ticketIdPrefixes(Iterable\textless{}String\textgreater{}\ ticketIdPrefixes)}
\textbar{} Adds valid prefixes of the Git commit messages to add to the
change log.
\texttt{BuildChangeLogTask\ ticketIdPrefixes(String...\ ticketIdPrefixes)}
\textbar{} Adds valid prefixes of the Git commit messages to add to the
change log.

\section{CSS Builder Gradle Plugin}\label{css-builder-gradle-plugin}

The CSS Builder Gradle plugin lets you run the
\href{https://github.com/liferay/liferay-portal/tree/master/modules/util/css-builder}{Liferay
CSS Builder} tool to compile \href{http://sass-lang.com/}{Sass} files in
your project.

The plugin has been successfully tested with Gradle 4.10.2.

\subsection{Usage}\label{usage-3}

To use the plugin, include it in your build script:

\begin{verbatim}
buildscript {
    dependencies {
        classpath group: "com.liferay", name: "com.liferay.gradle.plugins.css.builder", version: "3.0.0"
    }

    repositories {
        maven {
            url "https://repository-cdn.liferay.com/nexus/content/groups/public"
        }
    }
}

apply plugin: "com.liferay.css.builder"
\end{verbatim}

Since the plugin automatically resolves the Liferay CSS Builder library
as a dependency, you have to configure a repository that hosts the
library and its transitive dependencies. The Liferay CDN repository
hosts them all:

\begin{verbatim}
repositories {
    maven {
        url "https://repository-cdn.liferay.com/nexus/content/groups/public"
    }
}
\end{verbatim}

\subsection{Tasks}\label{tasks-3}

The plugin adds one task to your project:

Name \textbar{} Depends On \textbar{} Type \textbar{} Description
\texttt{buildCSS} \textbar{} - \textbar{}
\hyperref[buildcsstask]{\texttt{BuildCSSTask}} \textbar{} Compiles the
Sass files in this project.

The plugin also adds the following dependencies to tasks defined by the
\href{https://docs.gradle.org/current/userguide/java_plugin.html}{\texttt{java}}
plugin:

Name \textbar{} Depends On \texttt{processResources} \textbar{}
\texttt{buildCSS}

The \texttt{buildCSS} task is automatically configured with sensible
defaults, depending on whether the
\href{https://docs.gradle.org/current/userguide/java_plugin.html}{\texttt{java}}
or the
\href{https://docs.gradle.org/current/userguide/war_plugin.html}{\texttt{war}}
plugins are applied:

Property Name \textbar{} Default Value
\hyperref[basedir]{\texttt{baseDir}} \textbar{}

\textbf{If the \texttt{java} plugin is applied:} The first
\texttt{resources} directory of the \texttt{main} source set (by
default: \texttt{src/main/resources}).

\textbf{If the \texttt{war} plugin is applied:}
\texttt{project.webAppDir}.

\textbf{Otherwise:} \texttt{null}

\subsubsection{BuildCSSTask}\label{buildcsstask}

Tasks of type \texttt{BuildCSSTask} extend
\href{https://docs.gradle.org/current/dsl/org.gradle.api.tasks.JavaExec.html}{\texttt{JavaExec}},
so all its properties and methods, such as
\href{https://docs.gradle.org/current/dsl/org.gradle.api.tasks.JavaExec.html\#org.gradle.api.tasks.JavaExec:args(java.css.Iterable)}{\texttt{args}}
and
\href{https://docs.gradle.org/current/dsl/org.gradle.api.tasks.JavaExec.html\#org.gradle.api.tasks.JavaExec:maxHeapSize}{\texttt{maxHeapSize}},
are available. They also have the following properties set by default:

Property Name \textbar{} Default Value
\href{https://docs.gradle.org/current/dsl/org.gradle.api.tasks.JavaExec.html\#org.gradle.api.tasks.JavaExec:args}{\texttt{args}}
\textbar{} CSS Builder command line arguments
\href{https://docs.gradle.org/current/dsl/org.gradle.api.tasks.JavaExec.html\#org.gradle.api.tasks.JavaExec:classpath}{\texttt{classpath}}
\textbar{}
\hyperref[liferay-css-builder-dependency]{\texttt{project.configurations.cssBuilder}}
\href{https://docs.gradle.org/current/javadoc/org/gradle/api/tasks/JavaExec.html\#setDefaultCharacterEncoding(java.lang.String)}{\texttt{defaultCharacterEncoding}}
\textbar{} \texttt{"UTF-8"}
\href{https://docs.gradle.org/current/dsl/org.gradle.api.tasks.JavaExec.html\#org.gradle.api.tasks.JavaExec:main}{\texttt{main}}
\textbar{} \texttt{"com.liferay.css.builder.CSSBuilder"}
\href{https://docs.gradle.org/current/dsl/org.gradle.api.tasks.JavaExec.html\#org.gradle.api.tasks.JavaExec:systemProperties}{\texttt{systemProperties}}
\textbar{} \texttt{{[}"sass.compiler.jni.clean.temp.dir",\ true{]}}

\paragraph{Task Properties}\label{task-properties-2}

Property Name \textbar{} Type \textbar{} Default Value \textbar{}
Description \texttt{appendCssImportTimestamps} \textbar{}
\texttt{boolean} \textbar{} \texttt{true} \textbar{} Whether to append
the current timestamp to the URLs in the \texttt{@import} CSS at-rules.
It sets the \texttt{sass.append.css.import.timestamps} argument.
\texttt{baseDir} \textbar{} \texttt{File} \textbar{} \texttt{null}
\textbar{} The base directory that contains the SCSS files to compile.
It sets the \texttt{sass.docroot.dir} argument. \texttt{cssFiles}
\textbar{} \texttt{FileCollection} \textbar{} - \textbar{} The SCSS
files to compile. \emph{(Read-only)} \texttt{dirNames} \textbar{}
\texttt{List\textless{}String\textgreater{}} \textbar{}
\texttt{{[}"/"{]}} \textbar{} The name of the directories, relative to
\hyperref[basedir]{\texttt{baseDir}}, which contain the SCSS files to
compile. All sub-directories are searched for SCSS files as well. It
sets the \texttt{sass.dir} argument. \texttt{generateSourceMap}
\textbar{} \texttt{boolean} \textbar{} \texttt{false} \textbar{} Whether
to generate
\href{https://developers.google.com/web/tools/chrome-devtools/debug/readability/source-maps}{source
maps} for easier debugging. It sets the
\texttt{sass.generate.source.map} argument. \texttt{importDir}
\textbar{} \texttt{File} \textbar{} \texttt{null} \textbar{} The
\texttt{META-INF/resources} directory of the
\href{https://github.com/liferay/liferay-portal/tree/master/modules/apps/frontend-css/frontend-css-common}{Liferay
Frontend Common CSS} artifact. This is required in order to make
\href{http://bourbon.io}{Bourbon} and other CSS libraries available to
the compilation. \texttt{importFile} \textbar{} \texttt{File} \textbar{}
\hyperref[liferay-frontend-common-css-dependency]{\texttt{configurations.portalCommonCSS.singleFile}}
\textbar{} The Liferay Frontend Common CSS JAR file. If
\hyperref[importdir]{\texttt{importDir}} is set, this property has no
effect. \texttt{importPath} \textbar{} \texttt{File} \textbar{} -
\textbar{} The value of the \texttt{importDir} property if set;
otherwise \texttt{importFile}. It sets the
\texttt{sass.portal.common.path} argument. \emph{(Read-only)}
\texttt{outputDirName} \textbar{} \texttt{String} \textbar{}
\texttt{".sass-cache/"} \textbar{} The name of the sub-directories where
the SCSS files are compiled to. For each directory that contains SCSS
files, a sub-directory with this name is created. It sets the
\texttt{sass.output.dir} argument. \texttt{outputDirs} \textbar{}
\texttt{FileCollection} \textbar{} - \textbar{} The directories where
the SCSS files are compiled to. Usually, these directories are ignored
by the Version Control System. \emph{(Read-only)} \texttt{precision}
\textbar{} \texttt{int} \textbar{} \texttt{5} \textbar{} The numeric
precision of numbers in Sass. It sets the \texttt{sass.precision}
argument. \texttt{rtlExcludedPathRegexps} \textbar{}
\texttt{List\textless{}String\textgreater{}} \textbar{} \texttt{{[}{]}}
\textbar{} The SCSS file patterns to exclude when converting for
right-to-left (RTL) support. It sets the
\texttt{sass.rtl.excluded.path.regexps} argument.
\texttt{sassCompilerClassName} \textbar{} \texttt{String} \textbar{}
\texttt{null} \textbar{} The type of Sass compiler to use. Supported
values are \texttt{"jni"} and \texttt{"ruby"}. If not set, defaults to
\texttt{"jni"}. It sets the \texttt{sass.compiler.class.name} argument.

\noindent\hrulefill

\textbf{Note:} Liferay's CSS Builder is supported for Oracle's JDK and
uses a native compiler for increased speed. If you're using an IBM JDK,
you may experience issues when building your Sass files (e.g., when
building a theme). It's recommended to switch to using the Oracle JDK,
but if you prefer using the IBM JDK, you must use the fallback Ruby
compiler. You can do this two ways:

\begin{itemize}
\tightlist
\item
  If you're working in a
  \href{/docs/7-1/tutorials/-/knowledge_base/t/liferay-workspace}{Liferay
  Workspace} or using the
  \href{https://github.com/liferay/liferay-portal/tree/master/modules/sdk/gradle-plugins}{Liferay
  Gradle Plugins} plugin, set \texttt{sass.compiler.class.name=ruby} in
  your \texttt{gradle.properties} file.
\item
  Otherwise, set
  \texttt{buildCSS.sassCompilerClassName=\textquotesingle{}ruby\textquotesingle{}}
  in the project's \texttt{build.gradle} file.
\end{itemize}

The \texttt{sass.compiler.class.name=ruby} Gradle property only works
for modules, so if you're using the Ruby compiler in a WAR project
(e.g., theme), you must use the second option.

Be aware that the Ruby-based compiler doesn't perform as well as the
native compiler, so expect longer compile times.

\noindent\hrulefill

The properties of type \texttt{File} support any type that can be
resolved by
\href{https://docs.gradle.org/current/dsl/org.gradle.api.Project.html\#org.gradle.api.Project:file(java.css.Object)}{\texttt{project.file}}.
Moreover, it is possible to use Closures and Callables as values for the
\texttt{int} and \texttt{String} properties, to defer evaluation until
task execution.

\paragraph{Task Methods}\label{task-methods-1}

Method \textbar{} Description
\texttt{BuildCSSTask\ dirNames(Iterable\textless{}Object\textgreater{}\ dirNames)}
\textbar{} Adds sub-directory names, relative to
\hyperref[basedir]{\texttt{baseDir}}, which contain the SCSS files to
compile. \texttt{BuildCSSTask\ dirNames(Object...\ dirNames)} \textbar{}
Adds sub-directory names, relative to
\hyperref[basedir]{\texttt{baseDir}}, which contain the SCSS files to
compile.
\texttt{BuildCSSTask\ rtlExcludedPathRegexps(Iterable\textless{}Object\textgreater{}\ rtlExcludedPathRegexps)}
\textbar{} Adds SCSS file patterns to exclude when converting for
right-to-left (RTL) support.
\texttt{BuildCSSTask\ rtlExcludedPathRegexps(Object...\ rtlExcludedPathRegexps)}
\textbar{} Adds SCSS file patterns to exclude when converting for
right-to-left (RTL) support.

\subsection{Additional Configuration}\label{additional-configuration-1}

There are additional configurations that can help you use the CSS
Builder.

\subsubsection{Liferay CSS Builder
Dependency}\label{liferay-css-builder-dependency}

By default, the plugin creates a configuration called
\texttt{cssBuilder} and adds a dependency to the latest released version
of the Liferay CSS Builder. It is possible to override this setting and
use a specific version of the tool by manually adding a dependency to
the \texttt{cssBuilder} configuration:

\begin{verbatim}
dependencies {
    cssBuilder group: "com.liferay", name: "com.liferay.css.builder", version: "3.0.0"
}
\end{verbatim}

\subsubsection{Liferay Frontend Common CSS
Dependency}\label{liferay-frontend-common-css-dependency}

By default, the plugin creates a configuration called
\texttt{portalCommonCSS} and adds a dependency to the latest released
version of the Liferay Frontend Common CSS artifact. It is possible to
override this setting and use a specific version of the artifact by
manually adding a dependency to the \texttt{portalCommonCSS}
configuration:

\begin{verbatim}
dependencies {
    portalCommonCSS group: "com.liferay", name: "com.liferay.frontend.css.common", version: "2.0.1"
}
\end{verbatim}

\section{DB Support Gradle Plugin}\label{db-support-gradle-plugin}

The DB Support Gradle plugin lets you run the
\href{https://github.com/liferay/liferay-portal/tree/master/modules/util/portal-tools-db-support}{Liferay
DB Support} tool to execute certain actions on a local Liferay database.
So far, the following actions are available:

\begin{itemize}
\tightlist
\item
  Cleans the Liferay database from the Service Builder tables and rows
  of a module.
\end{itemize}

The plugin has been successfully tested with Gradle 4.10.2.

\subsection{Usage}\label{usage-4}

To use the plugin, include it in your build script:

\begin{verbatim}
buildscript {
    dependencies {
        classpath group: "com.liferay", name: "com.liferay.gradle.plugins.db.support", version: "1.0.5"
    }

    repositories {
        maven {
            url "https://repository-cdn.liferay.com/nexus/content/groups/public"
        }
    }
}

apply plugin: "com.liferay.portal.tools.db.support"
\end{verbatim}

Since the plugin automatically resolves the Liferay DB Support library
as a dependency, you have to configure a repository that hosts the
library and its transitive dependencies. The Liferay CDN repository
hosts them all:

\begin{verbatim}
repositories {
    maven {
        url "https://repository-cdn.liferay.com/nexus/content/groups/public"
    }
}
\end{verbatim}

\subsection{Tasks}\label{tasks-4}

The plugin adds one task to your project:

Name \textbar{} Depends On \textbar{} Type \textbar{} Description
\texttt{cleanServiceBuilder} \textbar{} - \textbar{}
\hyperref[cleanservicebuildertask]{\texttt{CleanServiceBuilderTask}}
\textbar{} Cleans the Liferay database from the Service Builder tables
and rows of a module.

The \texttt{cleanServiceBuilder} task is automatically configured with
sensible defaults, depending on whether the
\href{https://docs.gradle.org/current/userguide/standard_plugins.html\#N135C1}{\texttt{base}}
plugin is applied:

Property Name \textbar{} Default Value
\hyperref[servletcontextname]{\texttt{servletContextName}} \textbar{}

\textbf{If the \texttt{base} plugin is applied:} The bundle symbolic
name of the project inferred via the
\href{https://github.com/gradle/gradle/blob/master/subprojects/osgi/src/main/java/org/gradle/api/internal/plugins/osgi/OsgiHelper.java}{\texttt{OsgiHelper}}
class.

\textbf{Otherwise:} \texttt{null}

\hyperref[servicexmlfile]{\texttt{serviceXmlFile}} \textbar{}
\texttt{"\$\{project.projectDir\}/service.xml"}

\subsubsection{CleanServiceBuilderTask}\label{cleanservicebuildertask}

Tasks of type \texttt{BuildDeploymentHelperTask} extend
\href{https://docs.gradle.org/current/dsl/org.gradle.api.tasks.JavaExec.html}{\texttt{JavaExec}},
so all its properties and methods, such as
\href{https://docs.gradle.org/current/dsl/org.gradle.api.tasks.JavaExec.html\#org.gradle.api.tasks.JavaExec:args(java.lang.Iterable)}{\texttt{args}}
and
\href{https://docs.gradle.org/current/dsl/org.gradle.api.tasks.JavaExec.html\#org.gradle.api.tasks.JavaExec:maxHeapSize}{\texttt{maxHeapSize}},
are available. They also have the following properties set by default:

Property Name \textbar{} Default Value
\href{https://docs.gradle.org/current/dsl/org.gradle.api.tasks.JavaExec.html\#org.gradle.api.tasks.JavaExec:args}{\texttt{args}}
\textbar{} The DB Support command line arguments.
\href{https://docs.gradle.org/current/dsl/org.gradle.api.tasks.JavaExec.html\#org.gradle.api.tasks.JavaExec:classpath}{\texttt{classpath}}
\textbar{}
\hyperref[jdbc-drivers-dependency]{\texttt{project.configurations.dbSupport}}
+
\hyperref[liferay-db-support-dependency]{\texttt{project.configurations.dbSupportTool}}
\href{https://docs.gradle.org/current/dsl/org.gradle.api.tasks.JavaExec.html\#org.gradle.api.tasks.JavaExec:main}{\texttt{main}}
\textbar{} \texttt{"com.liferay.portal.tools.db.support.DBSupport"}

\paragraph{Task Properties}\label{task-properties-3}

Property Name \textbar{} Type \textbar{} Default Value \textbar{}
Description \texttt{password} \textbar{} \texttt{String} \textbar{}
\texttt{null} \textbar{} The user password for connecting to the Liferay
database. It sets the \texttt{-\/-password} argument. If
\hyperref[propertiesfile]{\texttt{propertiesFile}} is set, this property
has no effect. \texttt{propertiesFile} \textbar{} \texttt{File}
\textbar{} \texttt{null} \textbar{} The \texttt{portal-ext.properties}
file that contains the JDBC settings for connecting to the Liferay
database. It sets the \texttt{-\/-properties-file} argument.
\texttt{servletContextName} \textbar{} \texttt{String} \textbar{}
\texttt{null} \textbar{} The servlet context name (usually the value of
the \texttt{Bundle-Symbolic-Name} manifest header) of the module. It
sets the \texttt{-\/-servlet-context-name} argument.
\texttt{serviceXmlFile} \textbar{} \texttt{File} \textbar{}
\texttt{null} \textbar{} The \texttt{service.xml} file of the module. It
sets the \texttt{-\/-service-xml-file} argument. \texttt{url} \textbar{}
\texttt{String} \textbar{} \texttt{null} \textbar{} The JDBC URL for
connecting to the Liferay database. It sets the \texttt{-\/-url}
argument. If \hyperref[propertiesfile]{\texttt{propertiesFile}} is set,
this property has no effect. \texttt{userName} \textbar{}
\texttt{String} \textbar{} \texttt{null} \textbar{} The user name for
connecting to the Liferay database. It sets the \texttt{-\/-user-name}
argument. If \hyperref[propertiesfile]{\texttt{propertiesFile}} is set,
this property has no effect.

The properties of type \texttt{File} support any type that can be
resolved by
\href{https://docs.gradle.org/current/dsl/org.gradle.api.Project.html\#org.gradle.api.Project:file(java.css.Object)}{\texttt{project.file}}.
Moreover, it is possible to use Closures and Callables as values for the
\texttt{int} and \texttt{String} properties to defer evaluation until
task execution.

\subsection{Additional Configuration}\label{additional-configuration-2}

There are additional configurations that can help you use the Deployment
Helper.

\subsubsection{JDBC Drivers Dependency}\label{jdbc-drivers-dependency}

The plugin creates a configuration called \texttt{dbSupport}, which can
be used to provide the suitable JDBC driver for your Liferay database:

\begin{verbatim}
dependencies {
    dbSupport group: "mysql", name: "mysql-connector-java", version: "5.1.23"
    dbSupport group: "org.mariadb.jdbc", name: "mariadb-java-client", version: "1.1.9"
    dbSupport group: "org.postgresql", name: "postgresql", version: "9.4-1201-jdbc41"
}
\end{verbatim}

\subsubsection{Liferay DB Support
Dependency}\label{liferay-db-support-dependency}

By default, the plugin creates a configuration called
\texttt{dbSupportTool} and adds a dependency to the latest released
version of the Liferay DB Support. It is possible to override this
setting and use a specific version of the tool by manually adding a
dependency to the \texttt{dbSupportTool} configuration:

\begin{verbatim}
dependencies {
    dbSupportTool group: "com.liferay", name: "com.liferay.portal.tools.db.support", version: "1.0.8"
}
\end{verbatim}

\section{Dependency Checker Gradle
Plugin}\label{dependency-checker-gradle-plugin}

The Dependency Checker Gradle plugin lets you warn users if a specific
configuration dependency is not the latest one available from the Maven
central repository. The plugin eventually fails the build if the
dependency age (the difference between the timestamp of the current
version and the latest version) is above a predetermined threshold.

The plugin has been successfully tested with Gradle 4.10.2.

\subsection{Usage}\label{usage-5}

To use the plugin, include it in your build script:

\begin{verbatim}
buildscript {
    dependencies {
        classpath group: "com.liferay", name: "com.liferay.gradle.plugins.dependency.checker", version: "1.0.3"
    }

    repositories {
        maven {
            url "https://repository-cdn.liferay.com/nexus/content/groups/public"
        }
    }
}

apply plugin: "com.liferay.dependency.checker"
\end{verbatim}

\subsection{Project Extension}\label{project-extension-2}

The Dependency Checker Gradle plugin exposes the following properties
through the extension named \texttt{dependencyChecker}:

Property Name \textbar{} Type \textbar{} Default Value \textbar{}
Description \texttt{ignoreFailures} \textbar{} \texttt{boolean}
\textbar{} \texttt{true} \textbar{} Whether to print an error message
instead of failing the build when the dependency check fails, either for
a network error or because the dependency is out-of-date.

The same extension exposes the following methods:

Method \textbar{} Description
\texttt{void\ maxAge(Map\textless{}?,\ ?\textgreater{}\ args)}
\textbar{} Declares the max age allowed for a dependency. The
\texttt{args} map must contain the following entries:

\texttt{configuration}: the configuration name

\texttt{group}: the dependency group

\texttt{name}: the dependency name

\texttt{maxAge}: an instance of
\href{http://docs.groovy-lang.org/latest/html/api/groovy/time/Duration.html}{\texttt{groovy.time.Duration}}
that represents the maximum age allowed for the dependency

\texttt{throwError}: a \texttt{boolean} value representing whether to
throw an error if the dependency is out-of-date

\subsection{Additional Configuration}\label{additional-configuration-3}

There are additional configurations that can help you use the Deployment
Helper.

\subsubsection{Project Properties}\label{project-properties}

It is possible to set the default values of the
\hyperref[ignorefailures]{\texttt{ignoreFailures}} property via the
project property \texttt{dependencyCheckerIgnoreFailures}:

\begin{verbatim}
-PdependencyCheckerIgnoreFailures=false
\end{verbatim}

\section{Deployment Helper Gradle
Plugin}\label{deployment-helper-gradle-plugin}

The Deployment Helper Gradle plugin lets you run the
\href{https://github.com/liferay/liferay-portal/tree/master/modules/util/deployment-helper}{Liferay
Deployment Helper} tool to create a cluster deployable WAR from your
OSGi artifacts.

The plugin has been successfully tested with Gradle 4.10.2.

\subsection{Usage}\label{usage-6}

To use the plugin, include it in your build script:

\begin{verbatim}
buildscript {
    dependencies {
        classpath group: "com.liferay", name: "com.liferay.gradle.plugins.deployment.helper", version: "1.0.5"
    }

    repositories {
        maven {
            url "https://repository-cdn.liferay.com/nexus/content/groups/public"
        }
    }
}

apply plugin: "com.liferay.deployment.helper"
\end{verbatim}

Since the plugin automatically resolves the Liferay Deployment Helper
library as a dependency, you have to configure a repository that hosts
the library and its transitive dependencies. The Liferay CDN repository
hosts them all:

\begin{verbatim}
repositories {
    maven {
        url "https://repository-cdn.liferay.com/nexus/content/groups/public"
    }
}
\end{verbatim}

\subsection{Tasks}\label{tasks-5}

The plugin adds one task to your project:

Name \textbar{} Depends On \textbar{} Type \textbar{} Description
\texttt{buildDeploymentHelper} \textbar{} - \textbar{}
\hyperref[builddeploymenthelpertask]{\texttt{BuildDeploymentHelperTask}}
\textbar{} Builds a WAR which contains one or more files that are copied
once the WAR is deployed.

\subsubsection{BuildDeploymentHelperTask}\label{builddeploymenthelpertask}

Tasks of type \texttt{BuildDeploymentHelperTask} extend
\href{https://docs.gradle.org/current/dsl/org.gradle.api.tasks.JavaExec.html}{\texttt{JavaExec}},
so all its properties and methods, such as
\href{https://docs.gradle.org/current/dsl/org.gradle.api.tasks.JavaExec.html\#org.gradle.api.tasks.JavaExec:args(java.lang.Iterable)}{\texttt{args}}
and
\href{https://docs.gradle.org/current/dsl/org.gradle.api.tasks.JavaExec.html\#org.gradle.api.tasks.JavaExec:maxHeapSize}{\texttt{maxHeapSize}},
are available. They also have the following properties set by default:

Property Name \textbar{} Default Value
\href{https://docs.gradle.org/current/dsl/org.gradle.api.tasks.JavaExec.html\#org.gradle.api.tasks.JavaExec:args}{\texttt{args}}
\textbar{} The Deployment Helper command line arguments.
\href{https://docs.gradle.org/current/dsl/org.gradle.api.tasks.JavaExec.html\#org.gradle.api.tasks.JavaExec:classpath}{\texttt{classpath}}
\textbar{}
\hyperref[liferay-deployment-helper-dependency]{\texttt{project.configurations.deploymentHelper}}
\hyperref[deploymentfiles]{\texttt{deploymentFiles}} \textbar{} The
output files of the
\href{https://docs.gradle.org/current/userguide/java_plugin.html\#sec:jar}{\texttt{jar}}
tasks of this project and all its sub-projects.
\href{https://docs.gradle.org/current/dsl/org.gradle.api.tasks.JavaExec.html\#org.gradle.api.tasks.JavaExec:main}{\texttt{main}}
\textbar{} \texttt{"com.liferay.deployment.helper.DeploymentHelper"}
\hyperref[outputfile]{\texttt{outputFile}} \textbar{}
\texttt{"\$\{project.buildDir\}/\$\{project.name\}.war"}

\paragraph{Task Properties}\label{task-properties-4}

Property Name \textbar{} Type \textbar{} Default Value \textbar{}
Description \texttt{deploymentFiles} \textbar{} \texttt{FileCollection}
\textbar{} \texttt{{[}{]}} \textbar{} The files or directories to
include in the WAR and copy once the WAR is deployed. If a directory is
added to this collection, all the JAR files contained in the directory
are included in the WAR. \texttt{deploymentPath} \textbar{}
\texttt{File} \textbar{} \texttt{null} \textbar{} The directory to which
the included files are copied. \texttt{outputFile} \textbar{}
\texttt{File} \textbar{} \texttt{null} \textbar{} The WAR file to build.

The properties of type \texttt{File} support any type that can be
resolved by
\href{https://docs.gradle.org/current/dsl/org.gradle.api.Project.html\#org.gradle.api.Project:file(java.css.Object)}{\texttt{project.file}}.

\paragraph{Task Methods}\label{task-methods-2}

Method \textbar{} Description
\texttt{BuildDeploymentHelperTask\ deploymentFiles(Iterable\textless{}?\textgreater{}\ deploymentFiles)}
\textbar{} Adds files or directories to include in the WAR and copy once
the WAR is deployed. The values are evaluated as per
\href{https://docs.gradle.org/current/dsl/org.gradle.api.Project.html\#org.gradle.api.Project:files(java.lang.Object\%5B\%5D)}{\texttt{project.files}}.
\texttt{BuildDeploymentHelperTask\ deploymentFiles(Object...\ deploymentFiles)}
\textbar{} Adds files or directories to include in the WAR and copy once
the WAR is deployed. The values are evaluated as per
\href{https://docs.gradle.org/current/dsl/org.gradle.api.Project.html\#org.gradle.api.Project:files(java.lang.Object\%5B\%5D)}{\texttt{project.files}}.

\subsection{Additional Configuration}\label{additional-configuration-4}

There are additional configurations that can help you use the Deployment
Helper.

\subsubsection{Liferay Deployment Helper
Dependency}\label{liferay-deployment-helper-dependency}

By default, the plugin creates a configuration called
\texttt{deploymentHelper} and adds a dependency to the latest released
version of the Liferay Deployment Helper. It is possible to override
this setting and use a specific version of the tool by manually adding a
dependency to the \texttt{deploymentHelper} configuration:

\begin{verbatim}
dependencies {
    deploymentHelper group: "com.liferay", name: "com.liferay.deployment.helper", version: "1.0.4"
}
\end{verbatim}

\section{Go Gradle Plugin}\label{go-gradle-plugin}

The Go Gradle plugin lets you run \href{https://golang.org/}{Go} as part
of your build.

The plugin has been successfully tested with Gradle 3.5.1 up to 4.10.2.

\subsection{Usage}\label{usage-7}

To use the plugin, include it in your build script:

\begin{verbatim}
buildscript {
    dependencies {
        classpath group: "com.liferay", name: "com.liferay.gradle.plugins.go", version: "1.0.0"
    }

    repositories {
        maven {
            url "https://repository-cdn.liferay.com/nexus/content/groups/public"
        }
    }
}

apply plugin: "com.liferay.go"
\end{verbatim}

\subsection{Project Extension}\label{project-extension-3}

The Go Gradle plugin exposes the following properties through the
extension named \texttt{go}:

Property Name \textbar{} Type \textbar{} Default Value \textbar{}
Description \texttt{goDir} \textbar{} \texttt{File} \textbar{}
\texttt{"\$\{project.buildDir\}/go"} \textbar{} The directory where the
Go distribution is unpacked. \texttt{goUrl} \textbar{} \texttt{String}
\textbar{}
\texttt{"https://dl.google.com/go/go\$\{go.goVersion\}.\$\{platform\}-\$\{bitMode\}.\$\{extension\}}
\textbar{} The URL of the Go distribution to download.
\texttt{goVersion} \textbar{} \texttt{String} \textbar{}
\texttt{"1.11.4"} \textbar{} The Go distribution's version to use.
\texttt{workingDir} \textbar{} \texttt{File} \textbar{}
\texttt{"\$\{project.projectDir\}"} \textbar{} The directory that
contains the project's Go source code.

\subsection{Tasks}\label{tasks-6}

The plugin adds a series of tasks to your project:

Name \textbar{} Depends On \textbar{} Type \textbar{} Description
\texttt{downloadGo} \textbar{} - \textbar{}
\hyperref[downloadgotask]{\texttt{DownloadGoTask}} \textbar{} Downloads
and unpacks the local Go distribution for the project.
\hyperref[gocommandprogramname-task]{\texttt{goBuild\$\{programName\}}}
\textbar{} \texttt{downloadGo} \textbar{}
\hyperref[executegotask]{\texttt{ExecuteGoTask}} \textbar{} Compiles
packages and dependencies for the Go program.
\hyperref[gocommandprogramname-task]{\texttt{goClean\$\{programName\}}}
\textbar{} \texttt{downloadGo} \textbar{}
\hyperref[executegotask]{\texttt{ExecuteGoTask}} \textbar{} Removes
object files for the Go program.
\hyperref[gocommandprogramname-task]{\texttt{goRun\$\{programName\}}}
\textbar{} \texttt{downloadGo} \textbar{}
\hyperref[executegotask]{\texttt{ExecuteGoTask}} \textbar{} Compiles and
runs the Go program.
\hyperref[gocommandprogramname-task]{\texttt{goTest\$\{programName\}}}
\textbar{} \texttt{downloadGo} \textbar{}
\hyperref[executegotask]{\texttt{ExecuteGoTask}} \textbar{} Tests
packages for the Go program.

\subsubsection{DownloadGoTask}\label{downloadgotask}

The purpose of this task is to download and unpack a Go distribution.

\paragraph{Task Properties}\label{task-properties-5}

Property Name \textbar{} Type \textbar{} Default Value \textbar{}
Description \texttt{goDir} \textbar{} \texttt{File} \textbar{}
\texttt{null} \textbar{} The directory where the Go distribution is
unpacked. \texttt{goUrl} \textbar{} \texttt{String} \textbar{}
\texttt{null} \textbar{} The URL of the Go distribution to download.

The \texttt{File} type support any type that can be resolved by
\href{https://docs.gradle.org/current/dsl/org.gradle.api.Project.html\#org.gradle.api.Project:file(java.css.Object)}{\texttt{project.file}}.
Moreover, it is possible to use Closures and Callables as values for the
\texttt{String} properties, to defer evaluation until task execution.

\subsubsection{ExecuteGoTask}\label{executegotask}

This is the base task to run Go in a Gradle build. All tasks of type
\texttt{ExecuteGoTask} automatically depend on
\hyperref[downloadgo]{\texttt{downloadGo}}.

\paragraph{Task Properties}\label{task-properties-6}

Property Name \textbar{} Type \textbar{} Default Value \textbar{}
Description \texttt{args} \textbar{}
\texttt{List\textless{}Object\textgreater{}} \textbar{} \texttt{{[}{]}}
\textbar{} The arguments for the Go invocation. \texttt{command}
\textbar{} \texttt{String} \textbar{} \texttt{"go"} \textbar{} The file
name of the executable to invoke. \texttt{environment} \textbar{}
\texttt{Map\textless{}Object,\ Object\textgreater{}} \textbar{}
\texttt{{[}{]}} \textbar{} The environment variables for the Go
invocation. \texttt{inheritProxy} \textbar{} \texttt{boolean} \textbar{}
\texttt{true} \textbar{} Whether to set the \texttt{http\_proxy},
\texttt{https\_proxy}, and \texttt{no\_proxy} environment variables in
the Go invocation based on the values of the system properties
\texttt{https.proxyHost}, \texttt{https.proxyPort},
\texttt{https.proxyUser}, \texttt{https.proxyPassword},
\texttt{https.nonProxyHosts}, \texttt{https.proxyHost},
\texttt{https.proxyPort}, \texttt{https.proxyUser},
\texttt{https.proxyPassword}, and \texttt{https.nonProxyHosts}. If these
environment variables are already set, their values will not be
overwritten. \texttt{goDir} \textbar{} \texttt{File} \textbar{}
\texttt{go.goDir}{]}(\#godir) \textbar{} The directory that contains the
executable to invoke. \texttt{useGradleExec} \textbar{} \texttt{boolean}
\textbar{}

\textbf{If running in a
\href{https://docs.gradle.org/current/userguide/gradle_daemon.html}{Gradle
Daemon}:} \texttt{true}

\textbf{Otherwise:} \texttt{false}

\textbar{} Whether to invoke Go using
\href{https://docs.gradle.org/current/dsl/org.gradle.api.Project.html\#org.gradle.api.Project:exec(org.gradle.api.Action)}{\texttt{project.exec}},
which can solve hanging problems with the Gradle Daemon.
\texttt{workingDir} \textbar{} \texttt{File} \textbar{}
\texttt{go.workingDir}{]}(\#workingdir) \textbar{} The working directory
to use in the Go invocation.

The type \texttt{File} properties support any type that can be resolved
by
\href{https://docs.gradle.org/current/dsl/org.gradle.api.Project.html\#org.gradle.api.Project:file(java.css.Object)}{\texttt{project.file}}.
Moreover, it is possible to use Closures and Callables as values for the
\texttt{String} properties to defer evaluation until task execution.

\paragraph{Task Methods}\label{task-methods-3}

Method \textbar{} Description
\texttt{ExecuteGoTask\ args(Iterable\textless{}?\textgreater{}\ args)}
\textbar{} Adds arguments for the Go invocation.
\texttt{ExecuteGoTask\ args(Object...\ args)} \textbar{} Adds arguments
for the Go invocation.
\texttt{ExecuteGoTask\ environment(Map\textless{}?,\ ?\textgreater{}\ environment)}
\textbar{} Adds environment variables for the Go invocation.
\texttt{ExecuteGoTask\ environment(Object\ key,\ Object\ value)}
\textbar{} Adds an environment variable for the Go invocation.

\subsubsection{\texorpdfstring{go\({command}\)\{programName\}
Task}{go\{command\}\{programName\} Task}}\label{gocommandprogramname-task}

For each Go program in
\hyperref[workingdirproperty]{\texttt{workingDir}}, four tasks of type
\hyperref[executegotask]{\texttt{ExecuteGoTask}} are added. Each of
these tasks are automatically configured with sensible defaults:

Property Name \textbar{} Default Value \texttt{args} \textbar{}
\texttt{{[}"\$\{command\}",\ "\$\{programFile.absolutePath\}"{]}}

\section{Gulp Gradle Plugin}\label{gulp-gradle-plugin}

The Gulp Gradle plugin lets you run \href{http://gulpjs.com/}{Gulp}
tasks as part of your build.

The plugin has been successfully tested with Gradle 4.10.2.

\subsection{Usage}\label{usage-8}

To use the plugin, include it in your build script:

\begin{verbatim}
buildscript {
    dependencies {
        classpath group: "com.liferay", name: "com.liferay.gradle.plugins.gulp", version: "2.0.59"
    }

    repositories {
        maven {
            url "https://repository-cdn.liferay.com/nexus/content/groups/public"
        }
    }
}

apply plugin: "com.liferay.gulp"
\end{verbatim}

The Gulp plugin automatically applies the
\href{https://github.com/liferay/liferay-portal/tree/master/modules/sdk/gradle-plugins-node}{\texttt{com.liferay.node}}
plugin.

\subsection{Tasks}\label{tasks-7}

The plugin adds one
\href{https://docs.gradle.org/current/userguide/more_about_tasks.html\#sec:task_rules}{task
rule} to your project:

Name \textbar{} Depends On \textbar{} Type \textbar{} Description
\texttt{gulp\textless{}Task\textgreater{}} \textbar{}
\texttt{downloadNode}, \texttt{npmInstall} \textbar{}
\hyperref[executegulptask]{\texttt{ExecuteGulpTask}} \textbar{} Executes
a named Gulp task.

\subsubsection{ExecuteGulpTask}\label{executegulptask}

Tasks of type \texttt{ExecuteGulpTask} extend
\href{/docs/7-1/reference/-/knowledge_base/r/node-gradle-plugin\#executenodescripttask}{\texttt{ExecuteNodeScriptTask}},
so all its properties and methods, such as \texttt{args} and
\texttt{inheritProxy}, are available. They also have the following
properties set by default:

Property Name \textbar{} Default Value \texttt{scriptFile} \textbar{}
\texttt{"node\_modules/gulp/bin/gulp.js"}

Gulp must be already installed in the \texttt{node\_modules} directory
of the project; otherwise, it will not be downloaded by the task. In
order to ensure Gulp is installed, you can add the Gulp dependency to
the project's \texttt{package.json} file.

\paragraph{Task Properties}\label{task-properties-7}

Property Name \textbar{} Type \textbar{} Default Value \textbar{}
Description \texttt{gulpCommand} \textbar{} \texttt{String} \textbar{}
\texttt{null} \textbar{} The Gulp task to execute.

It is possible to use Closures and Callables as values for the
\texttt{String} properties to defer evaluation until task execution.

\section{Jasper JSPC Gradle Plugin}\label{jasper-jspc-gradle-plugin}

The Jasper JSPC Gradle plugin lets you run the
\href{https://github.com/liferay/liferay-portal/tree/master/modules/util/jasper-jspc}{Liferay
Jasper JSPC} tool to compile the JavaServer Pages (JSP) files in your
project. This can be useful to

\begin{itemize}
\tightlist
\item
  check for errors in the JSP files.
\item
  pre-compile the JSP files for better performance.
\end{itemize}

The plugin has been successfully tested with Gradle 4.10.2.

\subsection{Usage}\label{usage-9}

To use the plugin, include it in your build script:

\begin{verbatim}
buildscript {
    dependencies {
        classpath group: "com.liferay", name: "com.liferay.gradle.plugins.jasper.jspc", version: "2.0.5"
    }

    repositories {
        maven {
            url "https://repository-cdn.liferay.com/nexus/content/groups/public"
        }
    }
}

apply plugin: "com.liferay.jasper.jspc"
\end{verbatim}

The Jasper JSPC plugin automatically applies the
\href{https://docs.gradle.org/current/userguide/java_plugin.html}{\texttt{java}}
plugin.

Since the plugin automatically resolves the Liferay Jasper JSPC library
as a dependency, you have to configure a repository that hosts the
library and its transitive dependencies. The Liferay CDN repository
hosts them all:

\begin{verbatim}
repositories {
    maven {
        url "https://repository-cdn.liferay.com/nexus/content/groups/public"
    }
}
\end{verbatim}

\subsection{Tasks}\label{tasks-8}

The plugin adds two tasks to your project:

Name \textbar{} Depends On \textbar{} Type \textbar{} Description
\texttt{compileJSP} \textbar{} \texttt{generateJSPJava} \textbar{}
\href{https://docs.gradle.org/current/dsl/org.gradle.api.tasks.compile.JavaCompile.html}{\texttt{JavaCompile}}
\textbar{} Compiles JSP files to check for errors.
\texttt{generateJSPJava} \textbar{}
\href{https://docs.gradle.org/current/userguide/java_plugin.html\#sec:jar}{\texttt{jar}}
\textbar{} \hyperref[compilejsptask]{\texttt{CompileJSPTask}} \textbar{}
Compiles JSP files to Java source files to check for errors.

The \texttt{generateJSPJava} task is automatically configured with
sensible defaults, depending on whether the
\href{https://docs.gradle.org/current/userguide/war_plugin.html}{\texttt{war}}
plugin is applied:

Property Name \textbar{} Default Value
\href{https://docs.gradle.org/current/dsl/org.gradle.api.tasks.JavaExec.html\#org.gradle.api.tasks.JavaExec:classpath}{\texttt{classpath}}
\textbar{}
\hyperref[liferay-jasper-jspc-dependency]{\texttt{project.configurations.jspCTool}}
\hyperref[destinationdir]{\texttt{destinationDir}} \textbar{}
\texttt{"\$\{project.buildDir\}/jspc"}
\hyperref[jspcclasspath]{\texttt{jspCClasspath}} \textbar{}
\hyperref[jsp-compilation-classpath]{\texttt{project.configurations.jspC}}
\hyperref[webappdir]{\texttt{webAppDir}} \textbar{}

\textbf{If the \texttt{war} plugin is applied:}
\texttt{project.webAppDir}.

\textbf{Otherwise:} The first \texttt{resources} directory of the
\texttt{main} source set (by default, \texttt{src/main/resources}).

The \texttt{compileJSP} task is also configured with the following
defaults:

Property Name \textbar{} Default Value
\href{https://docs.gradle.org/current/dsl/org.gradle.api.tasks.compile.JavaCompile.html\#org.gradle.api.tasks.compile.JavaCompile:classpath}{\texttt{classpath}}
\textbar{}
\texttt{project.configurations.jspCTool\ +\ project.configurations.jspC}
\href{https://docs.gradle.org/current/dsl/org.gradle.api.tasks.compile.JavaCompile.html\#org.gradle.api.tasks.compile.JavaCompile:destinationDir}{\texttt{destinationDir}}
\textbar{} \texttt{compileJSP.temporaryDir}
\href{https://docs.gradle.org/current/dsl/org.gradle.api.tasks.compile.JavaCompile.html\#org.gradle.api.tasks.compile.JavaCompile:source}{\texttt{source}}
\textbar{} \texttt{generateJSPJava.outputs}

\subsubsection{CompileJSPTask}\label{compilejsptask}

Tasks of type \texttt{CompileJSPTask} extend
\href{https://docs.gradle.org/current/dsl/org.gradle.api.tasks.JavaExec.html}{\texttt{JavaExec}},
so all its properties and methods, such as
\href{https://docs.gradle.org/current/dsl/org.gradle.api.tasks.JavaExec.html\#org.gradle.api.tasks.JavaExec:args(java.css.Iterable)}{\texttt{args}}
and
\href{https://docs.gradle.org/current/dsl/org.gradle.api.tasks.JavaExec.html\#org.gradle.api.tasks.JavaExec:maxHeapSize}{\texttt{maxHeapSize}},
are available. They also have the following properties set by default:

Property Name \textbar{} Default Value
\href{https://docs.gradle.org/current/dsl/org.gradle.api.tasks.JavaExec.html\#org.gradle.api.tasks.JavaExec:main}{\texttt{main}}
\textbar{} \texttt{"com.liferay.jasper.jspc.JspC"}

\paragraph{Task Properties}\label{task-properties-8}

Property Name \textbar{} Type \textbar{} Default Value \textbar{}
Description \texttt{destinationDir} \textbar{} \texttt{File} \textbar{}
\texttt{null} \textbar{} The directory where the the JSP files are
compiled to. Package directories are automatically generated based on
the directories containing the uncompiled JSP files. It sets the
\texttt{-d} argument. \texttt{jspCClasspath} \textbar{}
\texttt{FileCollection} \textbar{} \texttt{null} \textbar{} The
classpath to use for the JSP files compilation. \texttt{webAppDir}
\textbar{} \texttt{File} \textbar{} \texttt{null} \textbar{} The
directory containing the web application. All JSP files in the directory
and its subdirectories are compiled. It sets the \texttt{-webapp}
argument.

The properties of type \texttt{File} support any type that can be
resolved by
\href{https://docs.gradle.org/current/dsl/org.gradle.api.Project.html\#org.gradle.api.Project:file(java.css.Object)}{\texttt{project.file}}.

\subsection{Additional Configuration}\label{additional-configuration-5}

There are additional configurations that can help you use Jasper JSPC.

\subsubsection{JSP Compilation
Classpath}\label{jsp-compilation-classpath}

The plugin creates a configuration called \texttt{jspC} and adds several
dependencies at the end of the configuration phase of the project:

\begin{itemize}
\tightlist
\item
  the JAR file of the project generated by the
  \href{https://docs.gradle.org/current/userguide/java_plugin.html\#sec:jar}{\texttt{jar}}
  task.
\item
  the output files of the \texttt{main} source set.
\item
  the \texttt{compileClasspath} file collection of the \texttt{main}
  source set.
\end{itemize}

If necessary, it is possible to add more dependencies to the
\texttt{jspC} configuration.

\subsubsection{Liferay Jasper JSPC
Dependency}\label{liferay-jasper-jspc-dependency}

By default, the plugin creates a configuration called \texttt{jspCTool}
and adds a dependency to the latest released version of the Liferay
Jasper JSPC. It is possible to override this setting and use a specific
version of the tool by manually adding a dependency to the
\texttt{jspCTool} configuration:

\begin{verbatim}
dependencies {
    jspCTool group: "com.liferay", name: "com.liferay.jasper.jspc", version: "1.0.11"
    jspCTool group: "org.apache.ant", name: "ant", version: "1.9.4"
}
\end{verbatim}

\section{Javadoc Formatter Gradle
Plugin}\label{javadoc-formatter-gradle-plugin}

The Javadoc Formatter Gradle plugin lets you format project Javadoc
comments using the
\href{https://github.com/liferay/liferay-portal/tree/master/modules/util/javadoc-formatter}{Liferay
Javadoc Formatter tool}. The tool lets you generate:

\begin{itemize}
\tightlist
\item
  Default
  \href{http://www.oracle.com/technetwork/java/javase/documentation/index-137868.html\#@author}{\texttt{@author}}
  tags to all classes.
\item
  Comment stubs to classes, fields, and methods.
\item
  Missing
  \href{https://docs.oracle.com/javase/8/docs/api/java/lang/Override.html}{\texttt{@Override}}
  annotations.
\item
  An XML representation of the Javadoc comments, which can be used by
  tools in order to index the Javadocs of the project.
\end{itemize}

The plugin has been successfully tested with Gradle 4.10.2.

\subsection{Usage}\label{usage-10}

To use the plugin, include it in your build script:

\begin{verbatim}
buildscript {
    dependencies {
        classpath group: "com.liferay", name: "com.liferay.gradle.plugins.javadoc.formatter", version: "1.0.27"
    }

    repositories {
        maven {
            url "https://repository-cdn.liferay.com/nexus/content/groups/public"
        }
    }
}

apply plugin: "com.liferay.javadoc.formatter"
\end{verbatim}

Since the plugin automatically resolves the Liferay Javadoc Formatter
library as a dependency, you have to configure a repository that hosts
the library and its transitive dependencies. The Liferay CDN repository
hosts them all:

\begin{verbatim}
repositories {
    maven {
        url "https://repository-cdn.liferay.com/nexus/content/groups/public"
    }
}
\end{verbatim}

\subsection{Tasks}\label{tasks-9}

The plugin adds one task to your project:

Name \textbar{} Depends On \textbar{} Type \textbar{} Description
\texttt{formatJavadoc} \textbar{} - \textbar{}
\hyperref[formatjavadoctask]{\texttt{FormatJavadocTask}} \textbar{} Runs
the Liferay Javadoc Formatter to format files.

\subsubsection{FormatJavadocTask}\label{formatjavadoctask}

Tasks of type \texttt{FormatJavadocTask} extend
\href{https://docs.gradle.org/current/dsl/org.gradle.api.tasks.JavaExec.html}{\texttt{JavaExec}},
so all its properties and methods, like
\href{https://docs.gradle.org/current/dsl/org.gradle.api.tasks.JavaExec.html\#org.gradle.api.tasks.JavaExec:args(java.lang.Iterable)}{\texttt{args}}
and
\href{https://docs.gradle.org/current/dsl/org.gradle.api.tasks.JavaExec.html\#org.gradle.api.tasks.JavaExec:maxHeapSize}{\texttt{maxHeapSize}},
are available. They also have the following properties set by default:

Property Name \textbar{} Default Value
\href{https://docs.gradle.org/current/dsl/org.gradle.api.tasks.JavaExec.html\#org.gradle.api.tasks.JavaExec:args}{\texttt{args}}
\textbar{} Javadoc Formatter command line arguments
\href{https://docs.gradle.org/current/dsl/org.gradle.api.tasks.JavaExec.html\#org.gradle.api.tasks.JavaExec:classpath}{\texttt{classpath}}
\textbar{}
\hyperref[liferay-javadoc-formatter-dependency]{\texttt{project.configurations.javadocFormatter}}
\href{https://docs.gradle.org/current/dsl/org.gradle.api.tasks.JavaExec.html\#org.gradle.api.tasks.JavaExec:main}{\texttt{main}}
\textbar{} \texttt{"com.liferay.javadoc.formatter.JavadocFormatter"}

\paragraph{Task Properties}\label{task-properties-9}

Property Name \textbar{} Type \textbar{} Default Value \textbar{}
Description \texttt{author} \textbar{} \texttt{String} \textbar{}
\texttt{"Brian\ Wing\ Shun\ Chan"} \textbar{} The value of the
\texttt{@author} tag to add at class level if missing. It sets the
\texttt{javadoc.author} argument. \texttt{generateXML} \textbar{}
\texttt{boolean} \textbar{} \texttt{false} \textbar{} Whether to
generate a XML representation of the Javadoc comments. The XML files are
generated in the \texttt{src/main/resources} directory only if the Java
files are contained in \texttt{src/main/java}. It sets the
\texttt{javadoc.generate.xml} argument.
\texttt{initializeMissingJavadocs} \textbar{} \texttt{boolean}
\textbar{} \texttt{false} \textbar{} Whether to add comment stubs at the
class, field, and method levels. If \texttt{false}, only the class-level
\texttt{@author} is added. It sets the \texttt{javadoc.init} argument.
\texttt{limits} \textbar{} \texttt{List\textless{}String\textgreater{}}
\textbar{} \texttt{{[}{]}} \textbar{} The Java file name patterns,
relative to
\href{https://docs.gradle.org/current/dsl/org.gradle.api.tasks.JavaExec.html\#org.gradle.api.tasks.JavaExec:workingDir}{\texttt{workingDir}},
to include when formatting Javadoc comments. The patterns must be
specified without the \texttt{.java} file type suffix. If empty, all
Java files are formatted. It sets the \texttt{javadoc.limit} argument.
\texttt{lowestSupportedJavaVersion} \textbar{} \texttt{double}
\textbar{} \texttt{1.7} \textbar{} If a method is annotated with the
\href{https://github.com/liferay/liferay-portal/blob/master/modules/util/javadoc-formatter/src/main/java/com/liferay/javadoc/formatter/SinceJava.java}{\texttt{@SinceJava}}
annotation and its \texttt{value} argument is greater than the value
specified for the \texttt{lowestSupportedJavaVersion} property, then the
\texttt{@Override} annotation is not automatically added, even if it is
missing. It sets the \texttt{javadoc.lowest.supported.java.version}
argument. See
\href{https://issues.liferay.com/browse/LPS-37353}{LPS-37353}.
\texttt{outputFilePrefix} \textbar{} \texttt{String} \textbar{}
\texttt{"javadocs"} \textbar{} The file name prefix of the XML
representation of the Javadoc comments. If \texttt{generateXML} is
\texttt{false}, this property is not used. It sets the
\texttt{javadoc.output.file.prefix} argument. \texttt{updateJavadocs}
\textbar{} \texttt{boolean} \textbar{} \texttt{false} \textbar{} Whether
to fix existing comment blocks by adding missing tags. It sets the
\texttt{javadoc.update} argument.

It is possible to use Closures and Callables as values for the
\texttt{String} properties, to defer evaluation until task execution.

\paragraph{Task Methods}\label{task-methods-4}

Method \textbar{} Description
\texttt{FormatJavadocTask\ dirNames(Iterable\textless{}Object\textgreater{}\ limits)}
\textbar{} Adds Java file name patterns, relative to
\texttt{workingDir}, to include when formatting Javadoc comments.
\texttt{FormatJavadocTask\ dirNames(Object...\ limits)} \textbar{} Adds
Java file name patterns, relative to \texttt{workingDir}, to include
when formatting Javadoc comments.

\subsection{Additional Configuration}\label{additional-configuration-6}

There are additional configurations that can help you use the Javadoc
Formatter.

\subsubsection{Liferay Javadoc Formatter
Dependency}\label{liferay-javadoc-formatter-dependency}

By default, the plugin creates a configuration called
\texttt{javadocFormatter} and adds a dependency to the latest released
version of the Liferay Javadoc Formatter. It is possible to override
this setting and use a specific version of the tool by manually adding a
dependency to the \texttt{javadocFormatter} configuration:

\begin{verbatim}
dependencies {
    javadocFormatter group: "com.liferay", name: "com.liferay.javadoc.formatter", version: "1.0.32"
}
\end{verbatim}

If the
\href{https://docs.gradle.org/current/userguide/java_plugin.html}{\texttt{java}}
plugin is applied, the \texttt{javadocFormatter} configuration
automatically extends from the
\href{https://docs.gradle.org/current/userguide/java_plugin.html\#sec:java_plugin_and_dependency_management}{\texttt{compile}}
configuration.

\subsubsection{System Properties}\label{system-properties-1}

It is possible to set the default values of the \texttt{generateXML},
\texttt{initializeMissingJavadocs}, \texttt{limits}, and
\texttt{updateJavadocs} properties for a \texttt{FormatJavadocTask} task
via system properties:

\begin{itemize}
\tightlist
\item
  \texttt{-D\$\{task.name\}.generate.xml=true}
\item
  \texttt{-D\$\{task.name\}.init=SomeClassName1,SomeClassName2,com.liferay.portal.**}
\item
  \texttt{-D\$\{task.name\}.limit=**/com/example/}
\item
  \texttt{-D\$\{task.name\}.update=true}
\end{itemize}

\section{JS Module Config Generator Gradle
Plugin}\label{js-module-config-generator-gradle-plugin}

The JS Module Config Generator Gradle plugin lets you run the
\href{https://github.com/liferay/liferay-module-config-generator}{Liferay
AMD Module Config Generator} to generate the configuration file needed
to load AMD files via combo loader in Liferay.

The plugin has been successfully tested with Gradle 4.10.2.

\subsection{Usage}\label{usage-11}

To use the plugin, include it in your build script:

\begin{verbatim}
buildscript {
    dependencies {
        classpath group: "com.liferay", name: "com.liferay.gradle.plugins.js.module.config.generator", version: "2.1.57"
    }

    repositories {
        maven {
            url "https://repository-cdn.liferay.com/nexus/content/groups/public"
        }
    }
}

apply plugin: "com.liferay.js.module.config.generator"
\end{verbatim}

The JS Module Config Generator plugin automatically applies the
\href{https://github.com/liferay/liferay-portal/tree/master/modules/sdk/gradle-plugins-node}{\texttt{com.liferay.node}}
plugin.

\subsection{Project Extension}\label{project-extension-4}

The JS Module Config Generator plugin exposes the following properties
through the extension named \texttt{jsModuleConfigGenerator}:

Property Name \textbar{} Type \textbar{} Default Value \textbar{}
Description \texttt{version} \textbar{} \texttt{String} \textbar{}
\texttt{"1.2.1"} \textbar{} The version of the Liferay AMD Module Config
Generator to use.

\subsection{Tasks}\label{tasks-10}

The plugin adds two tasks to your project:

Name \textbar{} Depends On \textbar{} Type \textbar{} Description
\texttt{configJSModules} \textbar{}
\texttt{downloadLiferayModuleConfigGenerator}, \texttt{processResources}
\textbar{} \hyperref[configjsmodulestask]{\texttt{ConfigJSModulesTask}}
\textbar{} Generates the configuration file needed to load AMD files via
combo loader in Liferay. \texttt{downloadLiferayModuleConfigGenerator}
\textbar{} \texttt{downloadNode} \textbar{}
\texttt{DownloadNodeModuleTask} \textbar{} Downloads the Liferay AMD
Module Config Generator in the project's \texttt{node\_modules}
directory.

By default, the \texttt{downloadLiferayModuleConfigGenerator} task
downloads the version of \texttt{liferay-module-config-generator}
declared in the
\hyperref[version]{\texttt{jsModuleConfigGenerator.version}} property.
If the project's \texttt{package.json} file, however, already lists the
\texttt{liferay-module-config-generator} package in its
\texttt{dependencies} or \texttt{devDependencies}, the
\texttt{downloadLiferayModuleConfigGenerator} task is disabled.

The \texttt{configJSModules} task is automatically configured with
sensible defaults, depending on whether the
\href{https://docs.gradle.org/current/userguide/java_plugin.html}{\texttt{java}}
plugin is applied:

Property Name \textbar{} Default Value
\hyperref[moduleconfigfile]{\texttt{moduleConfigFile}} \textbar{}
\texttt{"\$\{project.projectDir\}/package.json"}
\hyperref[outputfile]{\texttt{outputFile}} \textbar{}
\texttt{"\$\{sourceSets.main.output.resourcesDir\}/META-INF/config.json"}
\hyperref[sourcedir]{\texttt{sourceDir}} \textbar{}
\texttt{"\$\{sourceSets.main.output.resourcesDir\}/META-INF/resources"}

The plugin also adds the following dependencies to tasks defined by the
\texttt{java} plugin:

Name \textbar{} Depends On \texttt{classes} \textbar{}
\texttt{configJSModules}

If the
\href{https://github.com/liferay/liferay-portal/tree/master/modules/sdk/gradle-plugins-js-transpiler}{\texttt{com.liferay.js.transpiler}}
plugin is applied, the \texttt{configJSModules} task is configured to
always run after the \texttt{transpileJS} task.

\subsubsection{ConfigJSModulesTask}\label{configjsmodulestask}

Tasks of type \texttt{ConfigJSModulesTask} extend
\texttt{ExecuteNodeScriptTask}, so all its properties and methods, such
as \texttt{args}, \texttt{inheritProxy}, and \texttt{workingDir}, are
available. The \texttt{ConfigJSModulesTask} instances also implement the
\href{https://docs.gradle.org/current/javadoc/org/gradle/api/tasks/util/PatternFilterable.html}{\texttt{PatternFilterable}}
interface, which lets you specify include and exclude patterns for the
files in \hyperref[sourcedir]{\texttt{sourceDir}} to process.

They also have the following properties set by default:

Property Name \textbar{} Default Value
\href{https://docs.gradle.org/current/javadoc/org/gradle/api/tasks/util/PatternFilterable.html\#getIncludes()}{\texttt{includes}}
\textbar{} \texttt{{[}"**/*.es.js*",\ "**/*.soy.js*"{]}}
\texttt{scriptFile} \textbar{}
\texttt{"\$\{downloadLiferayModuleConfigGenerator.moduleDir\}/bin/index.js"}

The purpose of this task is to run the Liferay AMD Module Config
Generator from the included files in
\hyperref[sourcedir]{\texttt{sourceDir}}. The generator processes these
files and creates a configuration file in the location specified by the
\hyperref[outputfile]{\texttt{outputFile}} property.

\paragraph{Task Properties}\label{task-properties-10}

Property Name \textbar{} Type \textbar{} Default Value \textbar{}
Description \texttt{configVariable} \textbar{} \texttt{String}
\textbar{} \texttt{null} \textbar{}
The~configuration~variable~to~which~the~modules~should~be~added. It sets
the \texttt{-\/-config} argument. \texttt{customDefine} \textbar{}
\texttt{String} \textbar{} \texttt{"Liferay.Loader"} \textbar{} The
namespace of the \texttt{define(...)} call to use in the JS files. It
sets the \texttt{-\/-namespace} argument. \texttt{ignorePath} \textbar{}
\texttt{boolean} \textbar{} \texttt{false} \textbar{} Whether not to
create module \texttt{path} and \texttt{fullPath} properties. It sets
the \texttt{-\/-ignorePath} argument. \texttt{keepFileExtension}
\textbar{} \texttt{boolean} \textbar{} \texttt{false} \textbar{} Whether
to keep the file extension when generating the module name. It sets the
\texttt{-\/-keepExtension} argument. \texttt{lowerCase} \textbar{}
\texttt{boolean} \textbar{} \texttt{false} \textbar{} Whether to convert
file name to lower case before using it as the module name. It sets the
\texttt{-\/-lowerCase} argument. \texttt{moduleConfigFile} \textbar{}
\texttt{File} \textbar{} \texttt{null} \textbar{} The JSON file which
contains configuration data for the modules. It sets the
\texttt{-\/-moduleConfig} argument. \texttt{moduleExtension} \textbar{}
\texttt{String} \textbar{} \texttt{null} \textbar{} The extension for
the module file (e.g., \texttt{.js}). If specified, use the provided
string~as~an~extension~instead~to~get~it~automatically~from~the~file~name.
It sets the \texttt{-\/-extension} argument. \texttt{moduleFormat}
\textbar{} \texttt{String} \textbar{} \texttt{null} \textbar{} The
regular expression and value to apply to the file name when generating
the module name. It sets the \texttt{-\/-format} argument.
\texttt{outputFile} \textbar{} \texttt{File} \textbar{} \texttt{null}
\textbar{} The file where the generated configuration is stored. It sets
the \texttt{-\/-output} argument. \texttt{sourceDir} \textbar{}
\texttt{File} \textbar{} \texttt{null} \textbar{} The directory that
contains the files to process.

The properties of type \texttt{File} support any type that can be
resolved by
\href{https://docs.gradle.org/current/dsl/org.gradle.api.Project.html\#org.gradle.api.Project:file(java.css.Object)}{\texttt{project.file}}.
Moreover, it is possible to use Closures and Callables as values for the
\texttt{int} and \texttt{String} properties to defer evaluation until
task execution.

\section{JS Transpiler Gradle Plugin}\label{js-transpiler-gradle-plugin}

The JS Transpiler Gradle plugin lets you run
\href{https://github.com/metal/metal-cli}{\texttt{metal-cli}} to build
\href{http://metaljs.com/}{Metal.js} code, compile Soy files, and
transpile ES6 to ES5.

The plugin has been successfully tested with Gradle 4.10.2.

\subsection{Usage}\label{usage-12}

To use the plugin, include it in your build script:

\begin{verbatim}
buildscript {
    dependencies {
        classpath group: "com.liferay", name: "com.liferay.gradle.plugins.js.transpiler", version: "2.4.36"
    }

    repositories {
        maven {
            url "https://repository-cdn.liferay.com/nexus/content/groups/public"
        }
    }
}

apply plugin: "com.liferay.js.transpiler"
\end{verbatim}

There are two JS Transpiler Gradle plugins you can apply to your
project:

\begin{itemize}
\item
  \hyperref[js-transpiler-plugin]{\emph{JS Transpiler Plugin}}: builds
  Metal.js code, compiles Soy files, and transpiles ES6 to ES5:

\begin{verbatim}
apply plugin: "com.liferay.js.transpiler"
\end{verbatim}
\item
  \hyperref[js-transpiler-base-plugin]{\emph{JS Transpiler Base
  Plugin}}: provides a way to use Gradle dependencies (such as an
  \href{https://docs.gradle.org/current/userguide/dependency_management.html\#sub:module_dependencies}{external
  module} or
  \href{https://docs.gradle.org/current/userguide/dependency_management.html\#sub:project_dependencies}{project
  dependencies}) in Node.js scripts:

\begin{verbatim}
apply plugin: "com.liferay.js.transpiler.base"
\end{verbatim}
\end{itemize}

\subsection{JS Transpiler Plugin}\label{js-transpiler-plugin}

The JS Transpiler plugin automatically applies the
\hyperref[js-transpiler-base-plugin]{\emph{JS Transpiler Base Plugin}}.

The plugin adds two tasks to your project:

Name \textbar{} Depends On \textbar{} Type \textbar{} Description
\texttt{downloadMetalCli} \textbar{} \texttt{downloadNode} \textbar{}
\texttt{DownloadNodeModuleTask} \textbar{} Downloads \texttt{metal-cli}
in the project's \texttt{node\_modules} directory. \texttt{transpileJS}
\textbar{} \texttt{downloadMetalCli},
\texttt{expandJSCompileDependencies}, \texttt{npmInstall},
\texttt{processResources} \textbar{}
\hyperref[transpilejstask]{\texttt{TranspileJSTask}} \textbar{} Builds
Metal.js code.

By default, the \texttt{downloadMetalCli} task downloads the version
1.3.1 of \texttt{metal-cli}. If the project's \texttt{package.json}
file, however, already lists the \texttt{metal-cli} package in its
\texttt{dependencies} or \texttt{devDependencies}, the
\texttt{downloadMetalCli} task is disabled.

The \texttt{transpileJS} task is automatically configured with sensible
defaults, depending on whether the
\href{https://docs.gradle.org/current/userguide/java_plugin.html}{\texttt{java}}
plugin is applied:

Property Name \textbar{} Default Value
\hyperref[sourcedir]{\texttt{sourceDir}} \textbar{} The directory
\texttt{META-INF/resources} in the first \texttt{resources} directory of
the \texttt{main} source set (by default,
\texttt{src/main/resources/META-INF/resources}). \texttt{workingDir}
\textbar{}
\texttt{"\$\{sourceSets.main.output.resourcesDir\}/META-INF/resources"}

The plugin also adds the following dependencies to tasks defined by the
\texttt{java} plugin:

Name \textbar{} Depends On \texttt{classes} \textbar{}
\texttt{transpileJS}

The plugin adds a new configuration to the project called
\texttt{soyCompile}. If one or more dependencies are added to this
configuration, they will be expanded into temporary directories and
passed to the \texttt{transpileJS} task as additional
\hyperref[soydependencies]{\texttt{soyDependencies}} values.

\subsection{JS Transpiler Base Plugin}\label{js-transpiler-base-plugin}

The JS Transpiler Base plugin automatically applies the
\href{https://github.com/liferay/liferay-portal/tree/master/modules/sdk/gradle-plugins-node}{\texttt{com.liferay.node}}
plugin.

The plugin adds a new configuration to the project called
\texttt{jsCompile}. If one or more dependencies are added to this
configuration, they will be expanded into sub-directories of the
\texttt{node\_modules} directory, with names equal to the names of the
dependencies.

The plugin also adds one task to your project:

Name \textbar{} Depends On \textbar{} Type \textbar{} Description
\texttt{expandJSCompileDependencies} \textbar{} - \textbar{}
\href{https://docs.gradle.org/current/javadoc/org/gradle/api/DefaultTask.html}{\texttt{DefaultTask}}
\textbar{} Expands the additional configured JavaScript dependencies.
The task itself does not do any work, but depends on a series of
\href{https://docs.gradle.org/current/dsl/org.gradle.api.tasks.Copy.html}{Copy}
tasks called \texttt{expandJSCompileDependency\$\{file\}}, which expand
each dependency declared in the \texttt{jsCompile} configuration into
the \texttt{node\_modules} directory.

All the tasks of type \texttt{ExecuteNpmTask} whose name starts with
\texttt{"npmRun"} are configured to depend on
\texttt{expandJSCompileDependencies}. This means that, before running
any \href{https://docs.npmjs.com/misc/scripts}{script} declared in the
\texttt{package.json} file of the project, all the \texttt{jsCompile}
dependencies will be expanded into the \texttt{node\_modules} directory.

\subsection{Tasks}\label{tasks-11}

\subsubsection{TranspileJSTask}\label{transpilejstask}

Tasks of type \texttt{TranspileJSTask} extend
\texttt{ExecuteNodeScriptTask}, so all its properties and methods, such
as \texttt{args}, \texttt{inheritProxy}, and \texttt{workingDir}, are
available. They also have the following properties set by default:

Property Name \textbar{} Default Value \texttt{scriptFile} \textbar{}
\texttt{"\$\{downloadMetalCli.moduleDir\}/index.js"}
\texttt{soySrcIncludes} \textbar{} \texttt{{[}"**/*.soy"{]}}
\texttt{srcIncludes} \textbar{}
\texttt{{[}"**/*.es.js*",\ "**/*.soy.js*"{]}}

The purpose of this task is to run the \texttt{build} command of
\texttt{metal-cli} to build Metal.js code from
\hyperref[sourcedir]{\texttt{sourceDir}} into the \texttt{workingDir}
directory.

\paragraph{Task Properties}\label{task-properties-11}

Property Name \textbar{} Type \textbar{} Default Value \textbar{}
Description \texttt{bundleFileName} \textbar{} \texttt{String}
\textbar{} \texttt{null} \textbar{} The name of the final bundle file
for formats (e.g., \emph{globals}) that create one. It sets the
\texttt{-\/-bundleFileName} argument. \texttt{globalName} \textbar{}
\texttt{String} \textbar{} \texttt{null} \textbar{} The name of the
global variable that holds exported modules. It sets the
\texttt{-\/-globalName} argument. This is only used by the
\emph{globals} format build. \texttt{moduleName} \textbar{}
\texttt{String} \textbar{} \texttt{null} \textbar{} The name of the
project that is being compiled. All built modules are stored in a folder
with this name. It sets the \texttt{-\/-moduleName} argument. This is
only used by the \emph{amd} format build. \texttt{modules} \textbar{}
\texttt{String} \textbar{} \texttt{"amd"} \textbar{} The format(s) that
the source files are built to. It sets the \texttt{-\/-format} argument.
\texttt{skipWhenEmpty} \textbar{} \texttt{boolean} \textbar{}
\texttt{true} \textbar{} Whether to disable the task and remove its
dependencies if the \hyperref[sourcefiles]{\texttt{sourceFiles}}
property does not return any file at the end of the project evaluation.
\texttt{sourceDir} \textbar{} \texttt{File} \textbar{} \texttt{null}
\textbar{} The directory that contains the files to build.
\texttt{sourceFiles} \textbar{} \texttt{FileCollection} \textbar{}
\texttt{{[}{]}} \textbar{} The Soy and JS files to compile.
\emph{(Read-only)} \texttt{sourceMaps} \textbar{} \texttt{SourceMaps}
\textbar{} \texttt{enabled} \textbar{} Whether to generate source map
files. Available values include \texttt{disabled}, \texttt{enabled}, and
\texttt{enabled\_inline}. \texttt{soyDependencies} \textbar{}
\texttt{Set\textless{}String\textgreater{}} \textbar{}
\texttt{{[}"\$\{npmInstall.workingDir\}/node\_modules/clay*/src/**/*.soy",\ "\$\{npmInstall.workingDir\}/node\_modules/metal*/src/**/*.soy"{]}}
\textbar{} The path GLOBs of Soy files that the main source files depend
on, but that should not be compiled. It sets the \texttt{-\/-soyDeps}
argument. \texttt{soySkipMetalGeneration} \textbar{} \texttt{boolean}
\textbar{} \texttt{false} \textbar{} Whether to just compile Soy files,
without adding Metal.js generated code, like the \texttt{component}
class. It sets the \texttt{-\/-soySkipMetalGeneration} argument.
\texttt{soySrcIncludes} \textbar{}
\texttt{Set\textless{}String\textgreater{}} \textbar{} \texttt{{[}{]}}
\textbar{} The path GLOBs of the Soy files to compile. It sets the
\texttt{-\/-soySrc} argument. \texttt{srcIncludes} \textbar{}
\texttt{Set\textless{}String\textgreater{}} \textbar{} \texttt{{[}{]}}
\textbar{} The path GLOBs of the JS files to compile. It sets the
\texttt{-\/-src} argument.

The properties of type \texttt{File} support any type that can be
resolved by
\href{https://docs.gradle.org/current/dsl/org.gradle.api.Project.html\#org.gradle.api.Project:file(java.css.Object)}{\texttt{project.file}}.
Moreover, it is possible to use Closures and Callables as values for the
\texttt{int} and \texttt{String} properties to defer evaluation until
task execution.

\paragraph{Task Methods}\label{task-methods-5}

Method \textbar{} Description
\texttt{TranspileJSTask\ soyDependency(Iterable\textless{}?\textgreater{}\ soyDependencies)}
\textbar{} Adds path GLOBs of Soy files that the main source files
depend on, but that should not be compiled.
\texttt{TranspileJSTask\ soyDependency(Object...\ soyDependencies)}
\textbar{} Adds path GLOBs of Soy files that the main source files
depend on, but that should not be compiled.
\texttt{TranspileJSTask\ soySrcInclude(Iterable\textless{}?\textgreater{}\ soySrcIncludes)}
\textbar{} Adds path GLOBs of Soy files to compile.
\texttt{TranspileJSTask\ soySrcInclude(Object...\ soySrcIncludes)}
\textbar{} Adds path GLOBs of Soy files to compile.
\texttt{TranspileJSTask\ srcInclude(Iterable\textless{}?\textgreater{}\ srcIncludes)}
\textbar{} Adds path GLOBs of JS files to compile.
\texttt{TranspileJSTask\ srcInclude(Object...\ srcIncludes)} \textbar{}
Adds path GLOBs of JS files to compile.

\section{JSDoc Gradle Plugin}\label{jsdoc-gradle-plugin}

The JSDoc Gradle plugin lets you run the
\href{http://usejsdoc.org/}{JSDoc} tool in order to generate
documentation for your project's JavaScript files.

The plugin has been successfully tested with Gradle 4.10.2.

\subsection{Usage}\label{usage-13}

To use the plugin, include it in your build script:

\begin{verbatim}
buildscript {
    dependencies {
        classpath group: "com.liferay", name: "com.liferay.gradle.plugins.jsdoc", version: "2.0.33"
    }

    repositories {
        maven {
            url "https://repository-cdn.liferay.com/nexus/content/groups/public"
        }
    }
}
\end{verbatim}

There are two JSDoc Gradle plugins you can apply to your project:

\begin{itemize}
\item
  Apply the \hyperref[jsdoc-plugin]{JSDoc Plugin} to generate JavaScript
  documentation for your project:

\begin{verbatim}
apply plugin: "com.liferay.jsdoc"
\end{verbatim}
\item
  Apply the \hyperref[appjsdoc-plugin]{App JSDoc Plugin} in a parent
  project to generate the JavaScript documentation as a single, combined
  HTML document for an application that spans different subprojects,
  each one representing a different component of the same application:

\begin{verbatim}
apply plugin: "com.liferay.app.jsdoc"
\end{verbatim}
\end{itemize}

Both plugins automatically apply the
\href{https://github.com/liferay/liferay-portal/tree/master/modules/sdk/gradle-plugins-node}{\texttt{com.liferay.node}}
plugin.

\subsection{JSDoc Plugin}\label{jsdoc-plugin}

The plugin adds two tasks to your project:

Name \textbar{} Depends On \textbar{} Type \textbar{} Description
\texttt{downloadJSDoc} \textbar{} \texttt{downloadNode} \textbar{}
\texttt{DownloadNodeModuleTask} \textbar{} Downloads JSDoc in the
project's \texttt{node\_modules} directory. \texttt{jsdoc} \textbar{}
\texttt{downloadJSDoc} \textbar{}
\hyperref[jsdoctask]{\texttt{JSDocTask}} \textbar{} Generates API
documentation for the project's JavaScript code.

By default, the \texttt{downloadJSDoc} task downloads version
\texttt{3.5.5} of the \texttt{jsdoc} package. If the project's
\texttt{package.json} file, however, already lists the \texttt{jsdoc}
package in its \texttt{dependencies} or \texttt{devDependencies}, the
\texttt{downloadJSDoc} task is disabled.

The \texttt{jsdoc} task is automatically configured with sensible
defaults, depending on whether the
\href{https://docs.gradle.org/current/userguide/java_plugin.html}{\texttt{java}}
plugin is applied:

Property Name \textbar{} Default Value
\hyperref[destinationdir]{\texttt{destinationDir}} \textbar{}

\textbf{If the \texttt{java} plugin is applied:}
\texttt{"\$\{project.docsDir\}/jsdoc"}

\textbf{Otherwise:} \texttt{"\$\{project.buildDir\}/jsdoc"}

\hyperref[sourcedirs]{\texttt{sourceDirs}} \textbar{} The directory
\texttt{META-INF/resources} in the first \texttt{resources} directory of
the \texttt{main} source set (by default,
\texttt{src/main/resources/META-INF/resources}).

\subsection{AppJSDoc Plugin}\label{appjsdoc-plugin}

To use the App JSDoc plugin, it is required to apply the
\texttt{com.liferay.app.jsdoc} plugin in a parent project (that is, a
project that is a common ancestor of all the subprojects representing
the various components of the app). It is also required to apply the
\hyperref[jsdoc-plugin]{\texttt{com.liferay.jsdoc}} plugin to all the
subprojects that contain JavaScript files.

The App JSDoc plugin adds three tasks to your project:

Name \textbar{} Depends On \textbar{} Type \textbar{} Description
\texttt{appJSDoc} \textbar{} \texttt{downloadJSDoc} \textbar{}
\hyperref[jsdoctask]{\texttt{JSDocTask}} \textbar{} Generates API
documentation for the app's JavaScript code. \texttt{downloadJSDoc}
\textbar{} \texttt{downloadNode} \textbar{}
\texttt{DownloadNodeModuleTask} \textbar{} Downloads JSDoc in the app's
\texttt{node\_modules} directory. \texttt{jarAppJSDoc} \textbar{}
\texttt{appJSDoc} \textbar{}
\href{https://docs.gradle.org/current/dsl/org.gradle.api.tasks.bundling.Jar.html}{\texttt{Jar}}
\textbar{} Assembles a JAR archive containing the JavaScript
documentation files for this app.

By default, the \texttt{downloadJSDoc} task downloads version
\texttt{3.5.5} of the \texttt{jsdoc} package. If the project's
\texttt{package.json} file, however, already lists the \texttt{jsdoc}
package in its \texttt{dependencies} or \texttt{devDependencies}, the
\texttt{downloadJSDoc} task is disabled.

The \texttt{appJSDoc} task is automatically configured with sensible
defaults:

Property Name \textbar{} Default Value
\hyperref[destinationdir]{\texttt{destinationDir}} \textbar{}
\texttt{\$\{project.buildDir\}/docs/jsdoc}
\hyperref[sourcedirs]{\texttt{sourceDirs}} \textbar{} The sum of all the
\texttt{jsdoc.sourceDirs} values of the subprojects.

\subsection{Project Extension}\label{project-extension-5}

The App JSDoc plugin exposes the following properties through the
extension named \texttt{appJSDocConfiguration}:

Property Name \textbar{} Type \textbar{} Default Value \textbar{}
Description \texttt{subprojects} \textbar{}
\texttt{Set\textless{}Project\textgreater{}} \textbar{}
\texttt{project.subprojects} \textbar{} The subprojects to include in
the JavaScript documentation of the app.

The same extension exposes the following methods:

Method \textbar{} Description
\texttt{AppJSDocConfigurationExtension\ subprojects(Iterable\textless{}Project\textgreater{}\ subprojects)}
\textbar{} Include additional projects in the JavaScript documentation
of the app.
\texttt{AppJSDocConfigurationExtension\ subprojects(Project...\ subprojects)}
\textbar{} Include additional projects in the JavaScript documentation
of the app.

\subsection{Tasks}\label{tasks-12}

\subsubsection{JSDocTask}\label{jsdoctask}

Tasks of type \texttt{JSDocTask} extend \texttt{ExecuteNodeScriptTask},
so all its properties and methods, such as \texttt{args},
\texttt{inheritProxy}, and \texttt{workingDir}, are available.

They also have the following properties set by default:

Property Name \textbar{} Default Value \texttt{scriptFile} \textbar{}
\texttt{"\$\{downloadJSDoc.moduleDir\}/jsdoc.js"}

\paragraph{Task Properties}\label{task-properties-12}

Property Name \textbar{} Type \textbar{} Default Value \textbar{}
Description \texttt{configuration} \textbar{}
\href{https://docs.gradle.org/current/dsl/org.gradle.api.resources.TextResource.html}{\texttt{TextResource}}
\textbar{} \texttt{null} \textbar{} The JSDoc configuration file. It
sets the \texttt{-\/-configure} argument. \texttt{destinationDir}
\textbar{} \texttt{File} \textbar{} \texttt{null} \textbar{} The
directory where the JavaScript API documentation files are saved. It
sets the \texttt{-\/-destination} argument. \texttt{packageJsonFile}
\textbar{} \texttt{File} \textbar{}
\texttt{"\$\{project.projectDir\}/package.json"} \textbar{} The path to
the project's package file. It sets the \texttt{-\/-package} argument.
\texttt{sourceDirs} \textbar{} \texttt{FileCollection} \textbar{}
\texttt{{[}{]}} \textbar{} The directories that contains the files to
process. \texttt{readmeFile} \textbar{} \texttt{File} \textbar{}
\texttt{null} \textbar{} The path to the project's README file. It sets
the \texttt{-\/-readme} argument. \texttt{tutorialsDir} \textbar{}
\texttt{File} \textbar{} \texttt{null} \textbar{} The directory in which
JSDoc should search for tutorials. It sets the \texttt{-\/-tutorials}
argument.

The properties of type \texttt{File} support any type that can be
resolved by
\href{https://docs.gradle.org/current/dsl/org.gradle.api.Project.html\#org.gradle.api.Project:file(java.css.Object)}{\texttt{project.file}}.

\section{Lang Builder Gradle Plugin}\label{lang-builder-gradle-plugin}

The Lang Builder Gradle plugin lets you run the
\href{https://github.com/liferay/liferay-portal/tree/master/modules/util/lang-builder}{Liferay
Lang Builder} tool to sort and translate the language keys in your
project.

The plugin has been successfully tested with Gradle 4.10.2.

\subsection{Usage}\label{usage-14}

To use the plugin, include it in your build script:

\begin{verbatim}
buildscript {
    dependencies {
        classpath group: "com.liferay", name: "com.liferay.gradle.plugins.lang.builder", version: "3.0.31"
    }

    repositories {
        maven {
            url "https://repository-cdn.liferay.com/nexus/content/groups/public"
        }
    }
}

apply plugin: "com.liferay.lang.builder"
\end{verbatim}

Since the plugin automatically resolves the Liferay Lang Builder library
as a dependency, you have to configure a repository that hosts the
library and its transitive dependencies. The Liferay CDN repository
hosts them all:

\begin{verbatim}
repositories {
    maven {
        url "https://repository-cdn.liferay.com/nexus/content/groups/public"
    }
}
\end{verbatim}

See
\href{/docs/7-1/tutorials/-/knowledge_base/t/automatically-generating-language-files}{this
page} on the \emph{Liferay Developer Network} for more information about
usage of the Lang Builder Gradle plugin.

\subsection{Tasks}\label{tasks-13}

The plugin adds one task to your project:

Name \textbar{} Depends On \textbar{} Type \textbar{} Description
\texttt{buildLang} \textbar{} - \textbar{}
\hyperref[buildlangtask]{\texttt{BuildLangTask}} \textbar{} Runs Liferay
Lang Builder to translate language property files.

The \texttt{buildLang} task is automatically configured with sensible
defaults, depending on whether the
\href{https://docs.gradle.org/current/userguide/java_plugin.html}{\texttt{java}}
plugin is applied:

Property Name \textbar{} Default Value
\hyperref[langdir]{\texttt{langDir}} \textbar{}

\textbf{If the \texttt{java} plugin is applied:} The directory
\texttt{content} in the first \texttt{resources} directory of the
\texttt{main} source set (by default:
\texttt{src/main/resources/content}).

\textbf{Otherwise:} \texttt{null}

\subsubsection{BuildLangTask}\label{buildlangtask}

Tasks of type \texttt{BuildLangTask} extend
\href{https://docs.gradle.org/current/dsl/org.gradle.api.tasks.JavaExec.html}{\texttt{JavaExec}},
so all its properties and methods, such as
\href{https://docs.gradle.org/current/dsl/org.gradle.api.tasks.JavaExec.html\#org.gradle.api.tasks.JavaExec:args(java.lang.Iterable)}{\texttt{args}}
and
\href{https://docs.gradle.org/current/dsl/org.gradle.api.tasks.JavaExec.html\#org.gradle.api.tasks.JavaExec:maxHeapSize}{\texttt{maxHeapSize}},
are available. They also have the following properties set by default:

Property Name \textbar{} Default Value
\href{https://docs.gradle.org/current/dsl/org.gradle.api.tasks.JavaExec.html\#org.gradle.api.tasks.JavaExec:args}{\texttt{args}}
\textbar{} Lang Builder command line arguments
\href{https://docs.gradle.org/current/dsl/org.gradle.api.tasks.JavaExec.html\#org.gradle.api.tasks.JavaExec:classpath}{\texttt{classpath}}
\textbar{}
\hyperref[liferay-lang-builder-dependency]{\texttt{project.configurations.langBuilder}}
\href{https://docs.gradle.org/current/dsl/org.gradle.api.tasks.JavaExec.html\#org.gradle.api.tasks.JavaExec:main}{\texttt{main}}
\textbar{} \texttt{"com.liferay.lang.builder.LangBuilder"}

\paragraph{Task Properties}\label{task-properties-13}

Property Name \textbar{} Type \textbar{} Default Value \textbar{}
Description \texttt{excludedLanguageIds} \textbar{}
\texttt{Set\textless{}String\textgreater{}} \textbar{}
\texttt{{[}"da",\ "de",\ "fi",\ "ja",\ "nl",\ "pt\_PT",\ "sv"{]}}
\textbar{} The language IDs to exclude in the automatic translation. It
sets the \texttt{lang.excluded.language.ids} argument. \texttt{langDir}
\textbar{} \texttt{File} \textbar{} \texttt{null} \textbar{} The
directory where the language properties files are saved. It sets the
\texttt{lang.dir} argument. \texttt{langFileName} \textbar{}
\texttt{String} \textbar{} \texttt{"Language"} \textbar{} The file name
prefix of the language properties files (e.g.,
\texttt{Language\_it.properties}). It sets the \texttt{lang.file}
argument. \texttt{plugin} \textbar{} \texttt{boolean} \textbar{}
\texttt{true} \textbar{} Whether to check for duplicate language keys
between the project and the portal. If
\texttt{portalLanguagePropertiesFile} is not set, this property has no
effect. It sets the \texttt{lang.plugin} argument.
\texttt{portalLanguagePropertiesFile} \textbar{} \texttt{File}
\textbar{} \texttt{null} \textbar{} The \texttt{Language.properties}
file of the portal. It sets the
\texttt{lang.portal.language.properties.file} argument.
\texttt{translate} \textbar{} \texttt{boolean} \textbar{} \texttt{true}
\textbar{} Whether to translate the language keys and generate a
language properties file for each locale that's supported by Liferay. It
sets the \texttt{lang.translate} argument.
\texttt{translateSubscriptionKey} \textbar{} \texttt{String} \textbar{}
\texttt{null} \textbar{} The subscription key for Microsoft Translation
integration. Subscription to the Translator Text Translation API on
Microsoft Cognitive Services is required. Basic subscriptions, up to 2
million characters a month, are free. See
\href{http://docs.microsofttranslator.com/text-translate.html}{here} for
more information. It sets the \texttt{lang.translate.subscription.key}
argument.

The properties of type \texttt{File} support any type that can be
resolved by
\href{https://docs.gradle.org/current/dsl/org.gradle.api.Project.html\#org.gradle.api.Project:file(java.lang.Object)}{\texttt{project.file}}.
Moreover, it is possible to use Closures and Callables as values for the
\texttt{String} properties, to defer evaluation until task execution.

\paragraph{Task Methods}\label{task-methods-6}

Method \textbar{} Description
\texttt{BuildLangTask\ excludedLanguageIds(Iterable\textless{}Object\textgreater{}\ excludedLanguageIds)}
\textbar{} Adds language IDs to exclude in the automatic translation.
\texttt{BuildLangTask\ excludedLanguageIds(Object...\ excludedLanguageIds)}
\textbar{} Adds language IDs to exclude in the automatic translation.

\subsection{Additional Configuration}\label{additional-configuration-7}

There are additional configurations that can help you use the Lang
Builder.

\subsubsection{Liferay Lang Builder
Dependency}\label{liferay-lang-builder-dependency}

By default, the plugin creates a configuration called
\texttt{langBuilder} and adds a dependency to the latest released
version of the Liferay Lang Builder. It is possible to override this
setting and use a specific version of the tool by manually adding a
dependency to the \texttt{langBuilder} configuration:

\begin{verbatim}
dependencies {
    langBuilder group: "com.liferay", name: "com.liferay.lang.builder", version: "1.0.29"
}
\end{verbatim}

\section{Maven Plugin Builder Gradle
Plugin}\label{maven-plugin-builder-gradle-plugin}

The Maven Plugin Builder Gradle Plugin lets you generate the
\href{https://maven.apache.org/ref/current/maven-plugin-api/plugin.html}{Maven
plugin descriptor} for any
\href{https://maven.apache.org/general.html\#What_is_a_Mojo}{Mojos}
found in your project.

The plugin has been successfully tested with Gradle 4.10.2.

\subsection{Usage}\label{usage-15}

To use the plugin, include it in your build script:

\begin{verbatim}
buildscript {
    dependencies {
        classpath group: "com.liferay", name: "com.liferay.gradle.plugins.maven.plugin.builder", version: "1.2.4"
    }

    repositories {
        maven {
            url "https://repository-cdn.liferay.com/nexus/content/groups/public"
        }
    }
}

apply plugin: "com.liferay.maven.plugin.builder"
\end{verbatim}

\subsection{Tasks}\label{tasks-14}

The plugin adds two tasks to your project:

Name \textbar{} Depends On \textbar{} Type \textbar{} Description
\texttt{buildPluginDescriptor}
\textbar{}\href{https://docs.gradle.org/current/userguide/java_plugin.html\#sec:compile}{\texttt{compileJava}},
\hyperref[writemavensettings]{\texttt{WriteMavenSettings}} \textbar{}
\hyperref[buildplugindescriptortask]{\texttt{BuildPluginDescriptorTask}}
\textbar{} Generates the Maven plugin descriptor for the project.
\texttt{WriteMavenSettings} \textbar{} - \textbar{}
\hyperref[writemavensettingstask]{\texttt{WriteMavenSettingsTask}}
\textbar{} Writes a temporary Maven settings file to be used during
subsequent Maven invocations.

The Maven Plugin Builder Plugin automatically applies the
\href{https://docs.gradle.org/current/userguide/java_plugin.html}{\texttt{java}}
plugin.

The plugin also adds the following dependencies to tasks defined by the
\href{https://docs.gradle.org/current/userguide/maven_plugin.html}{\texttt{maven}}
plugin:

Name \textbar{} Depends On \texttt{install}, \texttt{uploadArchives},
and all the other tasks of type
\href{https://docs.gradle.org/current/dsl/org.gradle.api.tasks.Upload.html}{\texttt{Upload}}
\textbar{} \texttt{buildPluginDescriptor}

The \texttt{buildPluginDescriptor} task is automatically configured with
sensible defaults:

Property Name \textbar{} Default Value
\hyperref[classesdir]{\texttt{classesDir}} \textbar{}
\texttt{sourceSets.main.output.classesDir}
\hyperref[mavenembedderclasspath]{\texttt{mavenEmbedderClasspath}}
\textbar{}
\hyperref[maven-embedder-dependency]{\texttt{configurations.mavenEmbedder}}
\hyperref[mavensettingsfile]{\texttt{mavenSettingsFile}} \textbar{}
\hyperref[outputfile]{\texttt{writeMavenSettings.outputFile}}
\hyperref[outputdir]{\texttt{outputDir}} \textbar{} The directory
\texttt{META-INF/maven} in the first \texttt{resources} directory of the
\texttt{main} source set (by default:
\texttt{src/main/resources/META-INF/maven}).
\hyperref[pomartifactid]{\texttt{pomArtifactId}} \textbar{} The bundle
symbolic name of the project inferred via the
\href{https://github.com/gradle/gradle/blob/master/subprojects/osgi/src/main/java/org/gradle/api/internal/plugins/osgi/OsgiHelper.java}{\texttt{OsgiHelper}}
class. \hyperref[pomgroupid]{\texttt{pomGroupId}} \textbar{}
\href{https://docs.gradle.org/current/dsl/org.gradle.api.Project.html\#org.gradle.api.Project:group}{\texttt{project.group}}
\hyperref[pomversion]{\texttt{pomVersion}} \textbar{}
\href{https://docs.gradle.org/current/dsl/org.gradle.api.Project.html\#org.gradle.api.Project:version}{\texttt{project.version}}
(if it ends with \texttt{"-SNAPSHOT"}, the suffix will be removed)
\hyperref[sourcedir]{\texttt{sourceDir}} \textbar{} The first
\texttt{java} directory of the \texttt{main} source set (by default:
\texttt{src/main/java}).

The plugin ensures that the \texttt{processResources} task always runs
before \texttt{buildPluginDescriptor} to let \texttt{processResources}
copy the newly generated files in the
\texttt{buildPluginDescriptor.outputDir} directory.

The \texttt{writeMavenSettings} task is also automatically configured
with sensible defaults:

Property Name \textbar{} Default Value
\hyperref[localrepositorydir]{\texttt{localRepositoryDir}} \textbar{}
\texttt{maven.repo.local} system property
\hyperref[nonproxyhosts]{\texttt{nonProxyHosts}} \textbar{}
\texttt{http.nonProxyHosts} system property
\hyperref[outputfile]{\texttt{outputFile}} \textbar{}
\texttt{"\$\{project.buildDir\}/settings.xml"}
\hyperref[proxyhost]{\texttt{proxyHost}} \textbar{}
\texttt{http.ProxyHost} or \texttt{https.proxyHost} system property
(depending on the protocol of
\hyperref[repositoryurl]{\texttt{repositoryUrl}})
\hyperref[proxypassword]{\texttt{proxyPassword}} \textbar{}
\texttt{http.ProxyPassword} or \texttt{https.proxyPassword} system
property (depending on the protocol of
\hyperref[repositoryurl]{\texttt{repositoryUrl}})
\hyperref[proxyport]{\texttt{proxyPort}} \textbar{}
\texttt{http.ProxyPort} or \texttt{https.proxyPort} system property
(depending on the protocol of
\hyperref[repositoryurl]{\texttt{repositoryUrl}})
\hyperref[proxyuser]{\texttt{proxyUser}} \textbar{}
\texttt{http.ProxyUser} or \texttt{https.proxyUser} system property
(depending on the protocol of
\hyperref[repositoryurl]{\texttt{repositoryUrl}})
\hyperref[repositoryurl]{\texttt{repositoryUrl}} \textbar{}
\texttt{repository.url} system property

If running on JDK8+, the plugin also disables the
\href{http://docs.oracle.com/javase/8/docs/technotes/tools/unix/javadoc.html\#BEJEFABE}{\emph{doclint}}
feature in all tasks of type
\href{https://docs.gradle.org/current/dsl/org.gradle.api.tasks.javadoc.Javadoc.html}{\texttt{Javadoc}}.

\subsubsection{BuildPluginDescriptorTask}\label{buildplugindescriptortask}

Tasks of type \texttt{BuildPluginDescriptorTask} work by generating a
temporary \texttt{pom.xml} file based on the project, and then invoking
the \href{http://maven.apache.org/ref/3.3.9/maven-embedder/}{Maven
Embedder} to build the Maven plugin descriptor.

It is possible to declare information for the plugin descriptor
generation using either
\href{https://maven.apache.org/plugin-tools/maven-plugin-tools-annotations/}{Java
5 Annotations} or
\href{https://maven.apache.org/plugin-tools/maven-plugin-tools-java/}{Javadoc
Tags}.

\paragraph{Task Properties}\label{task-properties-14}

Property Name \textbar{} Type \textbar{} Default Value \textbar{}
Description \texttt{classesDir} \textbar{} \texttt{File} \textbar{}
\texttt{null} \textbar{} The directory that contains the compiled
classes. It sets the value of the
\href{http://maven.apache.org/ref/3.3.9/maven-model/maven.html\#class_build}{\texttt{build.outputDirectory}}
element in the generated \texttt{pom.xml} file.
\texttt{configurationScopeMappings} \textbar{}
\texttt{Map\textless{}String,\ String\textgreater{}} \textbar{}
\texttt{{[}"compile":\ "compile",\ "provided",\ "provided"{]}}
\textbar{} The mapping between the configuration names in the Gradle
project and the
\href{https://maven.apache.org/guides/introduction/introduction-to-dependency-mechanism.html\#Dependency_Scope}{dependency
scopes} in the \texttt{pom.xml} file. It is used to add
\href{http://maven.apache.org/ref/3.3.3/maven-model/maven.html\#class_dependency}{\texttt{dependencies.dependency}}
elements in the generated \texttt{pom.xml} file.
\texttt{forcedExclusions} \textbar{}
\texttt{Set\textless{}String\textgreater{}} \textbar{} \texttt{{[}{]}}
\textbar{} The \emph{group:name:version} notation of the dependencies to
always exclude from the ones added in the \texttt{pom.xml} file. It adds
\href{http://maven.apache.org/ref/3.3.3/maven-model/maven.html\#class_exclusion}{\texttt{dependencies.dependency.exclusions.exclusion}}
elements to the generated \texttt{pom.xml} file. \texttt{goalPrefix}
\textbar{} \texttt{String} \textbar{} \texttt{null} \textbar{} The goal
prefix for the Maven plugin specified in the descriptor. It sets the
value of the
\href{https://maven.apache.org/plugin-tools/maven-plugin-plugin/examples/generate-descriptor.html}{\texttt{build.plugins.plugin.configuration.goalPrefix}}
element in the generated \texttt{pom.xml} file. \texttt{mavenDebug}
\textbar{} \texttt{boolean} \textbar{} \texttt{false} \textbar{} Whether
to invoke the Maven Embedder in debug mode.
\texttt{mavenEmbedderClasspath} \textbar{} \texttt{FileCollection}
\textbar{} \texttt{null} \textbar{} The classpath used to invoke the
Maven Embedder. \texttt{mavenEmbedderMainClassName} \textbar{}
\texttt{String} \textbar{} \texttt{"org.apache.maven.cli.MavenCli"}
\textbar{} The Maven Embedder's main class name.
\texttt{mavenPluginPluginVersion} \textbar{} \texttt{String} \textbar{}
\texttt{"3.4"} \textbar{} The version of the
\href{https://maven.apache.org/plugin-tools/maven-plugin-plugin/}{Maven
Plugin Plugin} to use to generate the plugin descriptor for the project.
\texttt{mavenSettingsFile} \textbar{} \texttt{File} \textbar{}
\texttt{null} \textbar{} The custom \texttt{settings.xml} file to use.
It sets the \texttt{-\/-settings} argument on the Maven Embedder
invocation. \texttt{outputDir} \textbar{} \texttt{File} \textbar{}
\texttt{null} \textbar{} The directory where the Maven plugin descriptor
files are saved. \texttt{pomArtifactId} \textbar{} \texttt{String}
\textbar{} \texttt{null} \textbar{} The identifier for the artifact that
is unique within the group. It sets the value of the
\href{http://maven.apache.org/ref/3.3.3/maven-model/maven.html\#class_project}{\texttt{project.artifactId}}
element in the generated \texttt{pom.xml} file. \texttt{pomGroupId}
\textbar{} \texttt{String} \textbar{} \texttt{null} \textbar{} The
universally unique identifier for the project. It sets the value of the
\href{http://maven.apache.org/ref/3.3.3/maven-model/maven.html\#class_project}{\texttt{project.groupId}}
element in the generated \texttt{pom.xml} file. \texttt{pomRepositories}
\textbar{} \texttt{Map\textless{}String,\ Object\textgreater{}}
\textbar{}
\texttt{{[}"liferay-public":\ "http://repository.liferay.com/nexus/content/groups/public"{]}}
\textbar{} The name and URL of the remote repositories. It adds
\href{http://maven.apache.org/ref/3.3.3/maven-model/maven.html\#class_repository}{\texttt{repositories.repository}}
elements in the generated \texttt{pom.xml} file. \texttt{pomVersion}
\textbar{} \texttt{String} \textbar{} \texttt{null} \textbar{} The
version of the artifact produced by this project. It sets the value of
the
\href{http://maven.apache.org/ref/3.3.3/maven-model/maven.html\#class_project}{\texttt{project.version}}
element in the generated \texttt{pom.xml} file. \texttt{sourceDir}
\textbar{} \texttt{String} \textbar{} \texttt{null} \textbar{} The
directory that contains the source files. It sets the value of the
\href{http://maven.apache.org/ref/3.3.9/maven-model/maven.html\#class_build}{\texttt{build.sourceDirectory}}
element in the generated \texttt{pom.xml} file.
\texttt{useSetterComments} \textbar{} \texttt{boolean} \textbar{}
\texttt{true} \textbar{} Whether to allow
\href{https://maven.apache.org/plugin-tools/maven-plugin-tools-java/}{Mojo
Javadoc Tags} in the setter methods of the Mojo.

The properties of type \texttt{File} support any type that can be
resolved by
\href{https://docs.gradle.org/current/dsl/org.gradle.api.Project.html\#org.gradle.api.Project:file(java.lang.Object)}{\texttt{project.file}}.
Moreover, it is possible to use Closures and Callables as values for the
\texttt{String} properties, to defer evaluation until task execution.

\subsubsection{Task Methods}\label{task-methods-7}

Method \textbar{} Description
\texttt{void\ configurationScopeMapping(String\ configurationName,\ String\ scope)}
\textbar{} Adds a mapping between a configuration name in the Gradle
project and the dependency scope in the \texttt{pom.xml} file.
\texttt{BuildPluginDescriptorTask\ forcedExclusions(Iterable\textless{}String\textgreater{}\ forcedExclusions)}
\textbar{} Adds \emph{group:name:version} notations of dependencies to
always exclude from the ones added in the \texttt{pom.xml} file.
\texttt{BuildPluginDescriptorTask\ forcedExclusions(String...\ forcedExclusions)}
\textbar{} Adds \emph{group:name:version} notations of dependencies to
always exclude from the ones added in the \texttt{pom.xml} file.
\texttt{BuildPluginDescriptorTask\ pomRepositories(Map\textless{}String,\ ?\textgreater{}\ pomRepositories}
\textbar{} Adds names and URLs of remote repositories in the
\texttt{pom.xml} file.
\texttt{BuildPluginDescriptorTask\ pomRepository(String\ id,\ Object\ url)}
\textbar{} Adds the name and URL of a remote repository in the
\texttt{pom.xml} file.

\subsubsection{WriteMavenSettingsTask}\label{writemavensettingstask}

\paragraph{Task Properties}\label{task-properties-15}

Property Name \textbar{} Type \textbar{} Default Value \textbar{}
Description \texttt{localRepositoryDir} \textbar{} \texttt{String}
\textbar{} \texttt{null} \textbar{} The directory of the system's local
repository. It sets the value of the
\href{https://maven.apache.org/settings.html\#Simple_Values}{\texttt{localRepository}}
element in the generated \texttt{settings.xml} file.
\texttt{nonProxyHosts} \textbar{} \texttt{String} \textbar{}
\texttt{null} \textbar{} The patterns of the host that should be
accessed without going through the proxy. It sets the value of the
\href{https://maven.apache.org/settings.html\#Proxies}{\texttt{proxies.proxy.nonProxyHosts}}
element in the generated \texttt{settings.xml} file. \texttt{outputFile}
\textbar{} \texttt{File} \textbar{} \texttt{null} \textbar{} The
generated \texttt{settings.xml} file. \texttt{proxyHost} \textbar{}
\texttt{String} \textbar{} \texttt{null} \textbar{} The host name or
address of the proxy server. It sets the value of the
\href{https://maven.apache.org/settings.html\#Proxies}{\texttt{proxies.proxy.host}}
element in the generated \texttt{settings.xml} file.
\texttt{proxyPassword} \textbar{} \texttt{String} \textbar{}
\texttt{null} \textbar{} The password to use to access a protected proxy
server. It sets the value of the
\href{https://maven.apache.org/settings.html\#Proxies}{\texttt{proxies.proxy.password}}
element in the generated \texttt{settings.xml} file. \texttt{proxyPort}
\textbar{} \texttt{String} \textbar{} \texttt{null} \textbar{} The port
number of the proxy server. It sets the value of the
\href{https://maven.apache.org/settings.html\#Proxies}{\texttt{proxies.proxy.port}}
element in the generated \texttt{settings.xml} file. \texttt{proxyUser}
\textbar{} \texttt{String} \textbar{} \texttt{null} \textbar{} The user
name to use to access a protected proxy server. It sets the value of the
\href{https://maven.apache.org/settings.html\#Proxies}{\texttt{proxies.proxy.username}}
element in the generated \texttt{settings.xml} file.
\texttt{repositoryUrl} \textbar{} \texttt{String} \textbar{}
\texttt{null} \textbar{} The URL of the repository
\href{https://maven.apache.org/guides/mini/guide-mirror-settings.html\#Using_A_Single_Repository}{mirror}.
It sets the value of the
\href{https://maven.apache.org/settings.html\#Mirrors}{\texttt{mirrors.mirror.url}}
element in the generated \texttt{settings.xml} file.

The properties of type \texttt{File} support any type that can be
resolved by
\href{https://docs.gradle.org/current/dsl/org.gradle.api.Project.html\#org.gradle.api.Project:file(java.lang.Object)}{\texttt{project.file}}.
Moreover, it is possible to use Closures and Callables as values for the
\texttt{String} properties, to defer evaluation until task execution.

\subsection{Additional Configuration}\label{additional-configuration-8}

There are additional configurations that can help you use the Maven
Plugin Builder.

\subsubsection{Maven Embedder
Dependency}\label{maven-embedder-dependency}

By default, the plugin creates a configuration called
\texttt{mavenEmbedder} and adds a dependency to the 3.3.9 version of the
Maven Embedder. It is possible to override this setting and use a
specific version of the tool by manually adding a dependency to the
\texttt{mavenEmbedder} configuration:

\begin{verbatim}
dependencies {
    mavenEmbedder group: "org.apache.maven", name: "maven-embedder", version: "3.3.9"
    mavenEmbedder group: "org.apache.maven.wagon", name: "wagon-http", version: "2.10"
    mavenEmbedder group: "org.eclipse.aether", name: "aether-connector-basic", version: "1.0.2.v20150114"
    mavenEmbedder group: "org.eclipse.aether", name: "aether-transport-wagon", version: "1.0.2.v20150114"
    mavenEmbedder group: "org.slf4j", name: "slf4j-simple", version: "1.7.5"
}
\end{verbatim}

\subsubsection{System Properties}\label{system-properties-2}

It is possible to set the default value of the \texttt{mavenDebug}
property for a \texttt{BuildPluginDescriptorTask} task via system
property:

\begin{itemize}
\tightlist
\item
  \texttt{-D\$\{task.name\}.maven.debug=true}
\end{itemize}

For example, run the following Bash command to invoke the Maven Embedder
in debug mode to attach a remote debugger:

\begin{verbatim}
./gradlew buildPluginDescriptor -DbuildPluginDescriptor.maven.debug=true
\end{verbatim}

\section{Node Gradle Plugin}\label{node-gradle-plugin}

The Node Gradle plugin lets you run \href{https://nodejs.org/}{Node.js}
and \href{https://www.npmjs.com/}{NPM} as part of your build.

The plugin has been successfully tested with Gradle 4.10.2.

\subsection{Usage}\label{usage-16}

To use the plugin, include it in your build script:

\begin{verbatim}
buildscript {
    dependencies {
        classpath group: "com.liferay", name: "com.liferay.gradle.plugins.node", version: "4.6.18"
    }

    repositories {
        maven {
            url "https://repository-cdn.liferay.com/nexus/content/groups/public"
        }
    }
}

apply plugin: "com.liferay.node"
\end{verbatim}

\subsection{Project Extension}\label{project-extension-6}

The Node Gradle plugin exposes the following properties through the
extension named \texttt{node}:

Property Name \textbar{} Type \textbar{} Default Value \textbar{}
Description \texttt{download} \textbar{} \texttt{boolean} \textbar{}
\texttt{true} \textbar{} Whether to download and use a local and
isolated Node.js distribution instead of the one installed in the
system. \texttt{global} \textbar{} \texttt{boolean} \textbar{}
\texttt{false} \textbar{} Whether to use a single Node.js installation
for the whole multi-project build. This reduces the time required to
unpack the Node.js distribution and the time required to download NPM
packages thanks to a shared packages cache. If \texttt{download} is
\texttt{false}, this property has no effect. \texttt{nodeDir} \textbar{}
\texttt{File} \textbar{}

\textbf{If \texttt{global} is \texttt{true}:}
\texttt{"\$\{rootProject.buildDir\}/node"}

\textbf{Otherwise:} \texttt{"\$\{project.buildDir\}/node"}

\textbar{} The directory where the Node.js distribution is unpacked. If
\texttt{download} is \texttt{false}, this property has no effect.
\texttt{nodeUrl} \textbar{} \texttt{String} \textbar{}
\texttt{"http://nodejs.org/dist/v\$\{node.nodeVersion\}/node-v\$\{node.nodeVersion\}-\$\{platform\}-x\$\{bitMode\}.\$\{extension\}"}
\textbar{} The URL of the Node.js distribution to download. If
\texttt{download} is \texttt{false}, this property has no effect.
\texttt{nodeVersion} \textbar{} \texttt{String} \textbar{}
\texttt{"5.5.0"} \textbar{} The version of the Node.js distribution to
use. If \texttt{download} is \texttt{false}, this property has no
effect. \texttt{npmArgs} \textbar{}
\texttt{List\textless{}String\textgreater{}} \textbar{} \texttt{{[}{]}}
\textbar{} The arguments added automatically to every task of type
\hyperref[executenpmtask]{\texttt{ExecuteNpmTask}}. \texttt{npmUrl}
\textbar{} \texttt{String} \textbar{}
\texttt{"https://registry.npmjs.org/npm/-/npm-\$\{node.npmVersion\}.tgz"}
\textbar{} The URL of the NPM version to download. If \texttt{download}
is \texttt{false}, this property has no effect. \texttt{npmVersion}
\textbar{} \texttt{String} \textbar{} \texttt{null} \textbar{} The
version of NPM to use. If \texttt{null}, the version of NPM embedded
inside the Node.js distribution is used. If \texttt{download} is
\texttt{false}, this property has no effect.

It is possible to override the default value of the \texttt{download}
property by setting the \texttt{nodeDownload} project property. For
example, this can be done via command line argument:

\begin{verbatim}
./gradlew -PnodeDownload=false npmInstall
\end{verbatim}

The same extension exposes the following methods:

Method \textbar{} Description
\texttt{NodeExtension\ npmArgs(Iterable\textless{}?\textgreater{}\ npmArgs)}
\textbar{} Adds arguments to automatically add to every task of type
\hyperref[executenpmtask]{\texttt{ExecuteNpmTask}}.
\texttt{NodeExtension\ npmArgs(Object...\ npmArgs)} \textbar{} Adds
arguments to automatically add to every task of type
\hyperref[executenpmtask]{\texttt{ExecuteNpmTask}}.

The properties of type \texttt{File} support any type that can be
resolved by
\href{https://docs.gradle.org/current/dsl/org.gradle.api.Project.html\#org.gradle.api.Project:file(java.css.Object)}{\texttt{project.file}}.
Moreover, it is possible to use Closures and Callables as values for
\texttt{String}, to defer evaluation until execution.

Please note that setting the \texttt{global} property of the
\texttt{node} extension via the command line is not supported. It can
only be set via Gradle script, which can be done by adding the following
code to the \texttt{build.gradle} file in the root of a project (e.g.,
Liferay Workspace):

\begin{verbatim}
allprojects {
    plugins.withId("com.liferay.node") {
        node.global = true
    }
}
\end{verbatim}

\subsection{Tasks}\label{tasks-15}

The plugin adds a series of tasks to your project:

Name \textbar{} Depends On \textbar{} Type \textbar{} Description
\texttt{cleanNPM} \textbar{} - \textbar{}
\href{https://docs.gradle.org/current/dsl/org.gradle.api.tasks.Delete.html}{\texttt{Delete}}
\textbar{} Deletes the \texttt{node\_modules} directory, the
\texttt{npm-shrinkwrap.json} file and the \texttt{package-lock.json}
files from the project, if present. \texttt{downloadNode} \textbar{} -
\textbar{} \hyperref[downloadnodetask]{\texttt{DownloadNodeTask}}
\textbar{} Downloads and unpacks the local Node.js distribution for the
project. If \texttt{node.download} is \texttt{false}, this task is
disabled. \texttt{npmInstall} \textbar{} \texttt{downloadNode}
\textbar{} \hyperref[npminstalltask]{\texttt{NpmInstallTask}} \textbar{}
Runs \texttt{npm\ install} to install the dependencies declared in the
project's \texttt{package.json} file, if present. By default, the task
is \hyperref[npminstallretries]{configured} to run \texttt{npm\ install}
two more times if it fails.
\hyperref[npmrunscript-task]{\texttt{npmRun\$\{script\}}} \textbar{}
\texttt{npmInstall} \textbar{}
\hyperref[executenpmtask]{\texttt{ExecuteNpmTask}} \textbar{} Runs the
\texttt{\$\{script\}} NPM script. \texttt{npmPackageLock} \textbar{}
\texttt{cleanNPM}, \texttt{npmInstall} \textbar{}
\href{https://docs.gradle.org/current/javadoc/org/gradle/api/DefaultTask.html}{\texttt{DefaultTask}}
\textbar{} Deletes the NPM files and runs \texttt{npm\ install} to
install the dependencies declared in the project's \texttt{package.json}
file, if present. \texttt{npmShrinkwrap} \textbar{} \texttt{cleanNPM},
\texttt{npmInstall} \textbar{}
\hyperref[npmshrinkwraptask]{\texttt{NpmShrinkwrapTask}} \textbar{}
Locks down the versions of a package's dependencies in order to control
which dependency versions are used.

\subsubsection{DownloadNodeTask}\label{downloadnodetask}

The purpose of this task is to download and unpack a Node.js
distribution.

\paragraph{Task Properties}\label{task-properties-16}

Property Name \textbar{} Type \textbar{} Default Value \textbar{}
Description \texttt{nodeDir} \textbar{} \texttt{File} \textbar{}
\texttt{null} \textbar{} The directory where the Node.js distribution is
unpacked. \texttt{nodeExeUrl} \textbar{} \texttt{String} \textbar{}
\texttt{null} \textbar{} The URL of \texttt{node.exe} to download when
on Windows. \texttt{nodeUrl} \textbar{} \texttt{String} \textbar{}
\texttt{null} \textbar{} The URL of the Node.js distribution to
download. \texttt{npmUrl} \textbar{} \texttt{String} \textbar{}
\texttt{null} \textbar{} The URL of the NPM version to download.

The properties of type \texttt{File} support any type that can be
resolved by
\href{https://docs.gradle.org/current/dsl/org.gradle.api.Project.html\#org.gradle.api.Project:file(java.css.Object)}{\texttt{project.file}}.
Moreover, it is possible to use Closures and Callables as values for the
\texttt{String} properties, to defer evaluation until task execution.

\subsubsection{ExecuteNodeTask}\label{executenodetask}

This is the base task to run Node.js in a Gradle build. All tasks of
type \texttt{ExecuteNodeTask} automatically depend on
\hyperref[downloadnode]{\texttt{downloadNode}}.

\paragraph{Task Properties}\label{task-properties-17}

Property Name \textbar{} Type \textbar{} Default Value \textbar{}
Description \texttt{args} \textbar{}
\texttt{List\textless{}Object\textgreater{}} \textbar{} \texttt{{[}{]}}
\textbar{} The arguments for the Node.js invocation. \texttt{command}
\textbar{} \texttt{String} \textbar{} \texttt{"node"} \textbar{} The
file name of the executable to invoke. \texttt{environment} \textbar{}
\texttt{Map\textless{}Object,\ Object\textgreater{}} \textbar{}
\texttt{{[}{]}} \textbar{} The environment variables for the Node.js
invocation. \texttt{inheritProxy} \textbar{} \texttt{boolean} \textbar{}
\texttt{true} \textbar{} Whether to set the \texttt{http\_proxy},
\texttt{https\_proxy}, and \texttt{no\_proxy} environment variables in
the Node.js invocation based on the values of the system properties
\texttt{https.proxyHost}, \texttt{https.proxyPort},
\texttt{https.proxyUser}, \texttt{https.proxyPassword},
\texttt{https.nonProxyHosts}, \texttt{https.proxyHost},
\texttt{https.proxyPort}, \texttt{https.proxyUser},
\texttt{https.proxyPassword}, and \texttt{https.nonProxyHosts}. If these
environment variables are already set, their values will not be
overwritten. \texttt{nodeDir} \textbar{} \texttt{File} \textbar{}

\textbf{If \hyperref[download]{\texttt{node.download}} is
\texttt{true}:} \hyperref[nodedir]{\texttt{node.nodeDir}}

\textbf{Otherwise:} \texttt{null}

\textbar{} The directory that contains the executable to invoke. If
\texttt{null}, the executable must be available in the system
\texttt{PATH}. \texttt{npmInstallRetries} \textbar{} \texttt{int}
\textbar{} \texttt{0} \textbar{} The number of times the
\texttt{node\_modules} is deleted, the NPM cached data is verified
(\texttt{npm\ cache\ verify}), and \texttt{npm\ install} is retried in
case the Node.js invocation defined by this task fails. This can help
solving corrupted \texttt{node\_modules} directories by re-downloading
the project's dependencies. \texttt{workingDir} \textbar{} \texttt{File}
\textbar{} \texttt{project.projectDir} \textbar{} The working directory
to use in the Node.js invocation.

The properties of type \texttt{File} support any type that can be
resolved by
\href{https://docs.gradle.org/current/dsl/org.gradle.api.Project.html\#org.gradle.api.Project:file(java.css.Object)}{\texttt{project.file}}.
Moreover, it is possible to use Closures and Callables as values for the
\texttt{String} properties to defer evaluation until task execution.

\paragraph{Task Methods}\label{task-methods-8}

Method \textbar{} Description
\texttt{ExecuteNodeTask\ args(Iterable\textless{}?\textgreater{}\ args)}
\textbar{} Adds arguments for the Node.js invocation.
\texttt{ExecuteNodeTask\ args(Object...\ args)} \textbar{} Adds
arguments for the Node.js invocation.
\texttt{ExecuteNodeTask\ environment(Map\textless{}?,\ ?\textgreater{}\ environment)}
\textbar{} Adds environment variables for the Node.js invocation.
\texttt{ExecuteNodeTask\ environment(Object\ key,\ Object\ value)}
\textbar{} Adds an environment variable for the Node.js invocation.

\subsubsection{ExecuteNodeScriptTask}\label{executenodescripttask}

The purpose of this task is to execute a Node.js script. Tasks of type
\texttt{ExecuteNodeScriptTask} extend
\hyperref[executenodetask]{\texttt{ExecuteNodeTask}}.

\paragraph{Task Properties}\label{task-properties-18}

Property Name \textbar{} Type \textbar{} Default Value \textbar{}
Description \texttt{scriptFile} \textbar{} \texttt{File} \textbar{}
\texttt{null} \textbar{} The Node.js script to execute.

The properties of type \texttt{File} support any type that can be
resolved by
\href{https://docs.gradle.org/current/dsl/org.gradle.api.Project.html\#org.gradle.api.Project:file(java.css.Object)}{\texttt{project.file}}.

\subsubsection{ExecuteNpmTask}\label{executenpmtask}

The purpose of this task is to execute an NPM command. Tasks of type
\texttt{ExecuteNpmTask} extend
\hyperref[executenodescripttask]{\texttt{ExecuteNodeScriptTask}} with
the following properties set by default:

Property Name \textbar{} Default Value \texttt{command} \textbar{}

\textbf{If \texttt{nodeDir} is \texttt{null}:} \texttt{"npm"}

\textbf{Otherwise:} \texttt{"node"}

\texttt{scriptFile} \textbar{}

\textbf{If \texttt{nodeDir} is \texttt{null}:} \texttt{null}

\textbf{Otherwise:}
\texttt{"\$\{nodeDir\}/lib/node\_modules/npm/bin/npm-cli.js"} or
\texttt{"\$\{nodeDir\}/node\_modules/npm/bin/npm-cli.js"} on Windows.

\paragraph{Task Properties}\label{task-properties-19}

Property Name \textbar{} Type \textbar{} Default Value \textbar{}
Description \texttt{cacheConcurrent} \textbar{} \texttt{boolean}
\textbar{}

\textbf{If \texttt{node.npmVersion} is greater than or equal to
\texttt{5.0.0}, or \texttt{node.nodeVersion} is greater than or equal to
\texttt{8.0.0}:} \texttt{true}

\textbf{Otherwise:} \texttt{false}

\textbar{} Whether to run this task concurrently, in case the version of
NPM in use supports multiple concurrent accesses to the same cache
directory. \texttt{cacheDir} \textbar{} \texttt{File} \textbar{}

\textbf{If \texttt{nodeDir} is \texttt{null}, or
\texttt{node.npmVersion} is greater than or equal to \texttt{5.0.0}, or
\texttt{node.nodeVersion} is greater than or equal to \texttt{8.0.0}:}
\texttt{null}

\textbf{Otherwise:} \texttt{"\$\{nodeDir\}/.cache"}

\textbar{} The location of NPM's cache directory. It sets the
\href{https://docs.npmjs.com/misc/config\#cache}{\texttt{-\/-cache}}
argument. Leave the property \texttt{null} to keep the default value.
\texttt{logLevel} \textbar{} \texttt{String} \textbar{} Value to mirror
the log level set in the task's
\href{https://docs.gradle.org/current/dsl/org.gradle.api.Task.html\#org.gradle.api.Task:logger}{\texttt{logger}}
object. \textbar{} The NPM log level. It sets the
\href{https://docs.npmjs.com/misc/config\#loglevel}{--loglevel}
argument. \texttt{production} \textbar{} \texttt{boolean} \textbar{}
\texttt{false} \textbar{} Whether to run in production mode during the
NPM invocation. It sets the
\href{https://docs.npmjs.com/misc/config\#production}{\texttt{-\/-production}}
argument. \texttt{progress} \textbar{} \texttt{boolean} \textbar{}
\texttt{true} \textbar{} Whether to show a progress bar during the NPM
invocation. It sets the
\href{https://docs.npmjs.com/misc/config\#progress}{\texttt{-\/-progress}}
argument. \texttt{registry} \textbar{} \texttt{String} \textbar{}
\texttt{null} \textbar{} The base URL of the NPM package registry. It
sets the
\href{https://docs.npmjs.com/misc/config\#registry}{\texttt{-\/-registry}}
argument. Leave the property \texttt{null} or empty to keep the default
value.

The properties of type \texttt{File} support any type that can be
resolved by
\href{https://docs.gradle.org/current/dsl/org.gradle.api.Project.html\#org.gradle.api.Project:file(java.css.Object)}{\texttt{project.file}}.
Moreover, it is possible to use Closures and Callables as values for the
\texttt{String} properties, to defer evaluation until task execution.

\subsubsection{DownloadNodeModuleTask}\label{downloadnodemoduletask}

The purpose of this task is to download a Node.js package. The packages
are downloaded in the \texttt{\$\{workingDir\}/node\_modules} directory,
which is equal, by default, to the \texttt{node\_modules} directory of
the project. Tasks of type \texttt{DownloadNodeModuleTask} extend
\hyperref[executenpmtask]{\texttt{ExecuteNpmTask}} in order to execute
the command
\href{https://docs.npmjs.com/cli/install}{\texttt{npm\ install\ \$\{moduleName\}@\$\{moduleVersion\}}}.

\texttt{DownloadNodeModuleTask} instances are automatically disabled if
the project's \texttt{package.json} file already lists a module with the
same name in its \texttt{dependencies} or \texttt{devDependencies}
object.

\paragraph{Task Properties}\label{task-properties-20}

Property Name \textbar{} Type \textbar{} Default Value \textbar{}
Description \texttt{moduleName} \textbar{} \texttt{String} \textbar{}
\texttt{null} \textbar{} The name of the Node.js package to download.
\texttt{moduleVersion} \textbar{} \texttt{String} \textbar{}
\texttt{null} \textbar{} The version of the Node.js package to download.

It is possible to use Closures and Callables as values for the
\texttt{String} properties, to defer evaluation until task execution.

\subsubsection{NpmInstallTask}\label{npminstalltask}

Purpose of these tasks is to install the dependencies declared in a
\texttt{package.json} file. Tasks of type \texttt{NpmInstallTask} extend
\hyperref[executenpmtask]{\texttt{ExecuteNpmTask}} in order to run the
command
\href{https://docs.npmjs.com/cli/install}{\texttt{npm\ install}}.

\texttt{NpmInstallTask} instances are automatically disabled if the
\texttt{package.json} file does not declare any dependency in the
\texttt{dependency} or \texttt{devDependencies} object.

\paragraph{Task Properties}\label{task-properties-21}

Property Name \textbar{} Type \textbar{} Default Value \textbar{}
Description \texttt{nodeModulesCacheDir} \textbar{} \texttt{File}
\textbar{} \texttt{null} \textbar{}

The directory where \texttt{node\_modules} directories are cached. By
setting this property, it is possible to cache the
\texttt{node\_modules} directory of a project and avoid unnecessary
invocations of \texttt{npm\ install}, useful especially in Continuous
Integration environments.

The \texttt{node\_modules} directory is cached based on the content of
the project's \texttt{package-lock.json} (or
\texttt{npm-shrinkwrap.json}, or \texttt{package.json} if absent).
Therefore, if \texttt{NpmInstallTask} tasks in multiple projects are
configured with the same \texttt{nodeModulesCacheDir}, and their
\texttt{package-lock.json}, \texttt{npm-shrinkwrap.json} or
\texttt{package.json} declare the same dependencies, their
\texttt{node\_modules} caches will be shared.

This feature is not available if the
\href{https://github.com/liferay/liferay-portal/tree/master/modules/sdk/gradle-plugins-cache}{\texttt{com.liferay.cache}}
plugin is applied.

\texttt{nodeModulesCacheNativeSync} \textbar{} \texttt{boolean}
\textbar{} \texttt{true} \textbar{} Whether to use \texttt{rsync} (on
Linux/macOS) or \texttt{robocopy} (on Windows) to cache and restore the
\texttt{node\_modules} directory. If \texttt{nodeModulesCacheDir} is not
set, this property has no effect. \texttt{nodeModulesDigestFile}
\textbar{} \texttt{File} \textbar{} \texttt{null} \textbar{}

If this property is set, the content of the project's
\texttt{package-lock.json} (or \texttt{npm-shrinkwrap.json}, or
\texttt{package.json} if absent) is checked with the digest from the
\texttt{node\_modules} directory. If the digests match, do nothing. If
the digests don't match, the \texttt{node\_modules} directory is deleted
before running \texttt{npm\ install}.

This feature is not available if the \texttt{com.liferay.cache} plugin
is applied or if the property \texttt{nodeModulesCacheDir} is set.

\texttt{removeShrinkwrappedUrls} \textbar{} \texttt{boolean} \textbar{}
\texttt{true} if the \hyperref[registry]{registry} property has a value,
\texttt{false} otherwise. \textbar{} Whether to temporarily remove all
the hard-coded URLs in the \texttt{from} and \texttt{resolved} fields of
the \texttt{npm-shinkwrap.json} file before invoking
\texttt{npm\ install}. This way, it is possible to force NPM to download
all dependencies from a custom registry declared in the
\hyperref[registry]{\texttt{registry}} property. \texttt{useNpmCI}
\textbar{} \texttt{boolean} \textbar{} \texttt{false} \textbar{} Whether
to run \texttt{npm\ ci} instead of \texttt{npm\ install}. If the
\texttt{package-lock.json} file does not exist, this property has no
effect.

The properties of type \texttt{File} support any type that can be
resolved by
\href{https://docs.gradle.org/current/dsl/org.gradle.api.Project.html\#org.gradle.api.Project:file(java.css.Object)}{\texttt{project.file}}.

\subsubsection{NpmShrinkwrapTask}\label{npmshrinkwraptask}

The purpose of this task is to lock down the versions of a package's
dependencies so that you can control exactly which dependency versions
are used when your package is installed. Tasks of type
\texttt{NpmShrinkwrapTask} extend
\hyperref[executenpmtask]{\texttt{ExecuteNpmTask}} to execute the
command
\href{https://docs.npmjs.com/cli/shrinkwrap}{\texttt{npm\ shrinkwrap}}.

The generated \texttt{npm-shrinkwrap.json} file is automatically sorted
and formatted, so it's easier to see the changes with the previous
version.

\texttt{NpmShrinkwrapTask} instances are automatically disabled if the
\texttt{package.json} file does not exist.

\paragraph{Task Properties}\label{task-properties-22}

Property Name \textbar{} Type \textbar{} Default Value \textbar{}
Description \texttt{excludedDependencies} \textbar{}
\texttt{List\textless{}String\textgreater{}} \textbar{} \texttt{{[}{]}}
\textbar{} The package names to exclude from the generated
\texttt{npm-shrinkwrap.json} file. \texttt{includeDevDependencies}
\textbar{} \texttt{boolean} \textbar{} \texttt{true} \textbar{} Whether
to include the package's \texttt{devDependencies}. It sets the
\href{https://docs.npmjs.com/cli/shrinkwrap\#other-notes}{\texttt{-\/-dev}}
argument.

It is possible to use Closures and Callables as values for the
\texttt{String} properties to defer evaluation until task execution.

\paragraph{Task Methods}\label{task-methods-9}

Method \textbar{} Description
\texttt{NpmShrinkwrapTask\ excludeDependencies(Iterable\textless{}?\textgreater{}\ excludedDependencies)}
\textbar{} Adds package names to exclude from the generated
\texttt{npm-shrinkwrap.json} file.
\texttt{NpmShrinkwrapTask\ excludeDependencies(Object...\ excludedDependencies)}
\textbar{} Adds package names to exclude from the generated
\texttt{npm-shrinkwrap.json} file.

\subsubsection{PublishNodeModuleTask}\label{publishnodemoduletask}

The purpose of this task is to publish a package to the
\href{https://www.npmjs.com/}{NPM registry}. Tasks of type
\texttt{PublishNodeModuleTask} extend
\hyperref[executenpmtask]{\texttt{ExecuteNpmTask}} in order to execute
the command
\href{https://docs.npmjs.com/cli/publish}{\texttt{npm\ publish}}.

These tasks generate a new temporary \texttt{package.json} file in the
directory assigned to the \hyperref[workingdir]{\texttt{workingDir}}
property; then the \texttt{npm\ publish} command is executed. If the
\texttt{package.json} file in that location does not exist, the one in
the root of the project directory (if found) is copied; otherwise, a new
file is created.

The \texttt{package.json} is then processed by adding the values
provided by the task properties, if not already present in the file
itself. It is still possible to override one or more fields of the
\texttt{package.json} file with the values provided by the task
properties by adding one or more keys (e.g., \texttt{"version"}) to the
\texttt{overriddenPackageJsonKeys} property.

\paragraph{Task Properties}\label{task-properties-23}

Property Name \textbar{} Type \textbar{} Default Value \textbar{}
Description \texttt{moduleAuthor} \textbar{} \texttt{String} \textbar{}
\texttt{null} \textbar{} The value of the
\href{https://docs.npmjs.com/files/package.json\#people-fields-author-contributors}{\texttt{author}}
field in the generated \texttt{package.json} file.
\texttt{moduleBugsUrl} \textbar{} \texttt{String} \textbar{}
\texttt{null} \textbar{} The value of the
\href{https://docs.npmjs.com/files/package.json\#bugs}{\texttt{bugs.url}}
field in the generated \texttt{package.json} file.
\texttt{moduleDescription} \textbar{} \texttt{String} \textbar{}
\texttt{project.description} \textbar{} The value of the
\href{https://docs.npmjs.com/files/package.json\#description-1}{\texttt{description}}
field in the generated \texttt{package.json} file.
\texttt{moduleKeywords} \textbar{}
\texttt{List\textless{}String\textgreater{}} \textbar{} \texttt{{[}{]}}
\textbar{} The value of the
\href{https://docs.npmjs.com/files/package.json\#keywords}{\texttt{keywords}}
field in the generated \texttt{package.json} file.
\texttt{moduleLicense} \textbar{} \texttt{String} \textbar{}
\texttt{null} \textbar{} The value of the
\href{https://docs.npmjs.com/files/package.json\#license}{\texttt{license}}
field in the generated \texttt{package.json} file. \texttt{moduleMain}
\textbar{} \texttt{String} \textbar{} \texttt{null} \textbar{} The value
of the
\href{https://docs.npmjs.com/files/package.json\#main}{\texttt{main}}
field in the generated \texttt{package.json} file. \texttt{moduleName}
\textbar{} \texttt{String} \textbar{} Name based on
\href{https://github.com/gradle/gradle/blob/master/subprojects/osgi/src/main/java/org/gradle/api/internal/plugins/osgi/OsgiHelper.java}{\texttt{osgiHelper.bundleSymbolicName}}:
for example, if \texttt{osgiHelper.bundleSymbolicName} is
\texttt{"com.liferay.gradle.plugins.node"}, the default value for the
\texttt{moduleName} property is \texttt{"liferay-gradle-plugins-node"}.
\textbar{} The value of the
\href{https://docs.npmjs.com/files/package.json\#name}{\texttt{name}}
field in the generated \texttt{package.json} file.
\texttt{moduleRepository} \textbar{} \texttt{String} \textbar{}
\texttt{null} \textbar{} The value of the
\href{https://docs.npmjs.com/files/package.json\#repository}{\texttt{repository}}
field in the generated \texttt{package.json} file.
\texttt{moduleVersion} \textbar{} \texttt{String} \textbar{}
\texttt{project.version} \textbar{} The value of the
\href{https://docs.npmjs.com/files/package.json\#version}{\texttt{version}}
field in the generated \texttt{package.json} file.
\texttt{npmEmailAddress} \textbar{} \texttt{String} \textbar{}
\texttt{null} \textbar{} The email address of the npmjs.com user that
publishes the package. \texttt{npmPassword} \textbar{} \texttt{String}
\textbar{} \texttt{null} \textbar{} The password of the npmjs.com user
that publishes the package. \texttt{npmUserName} \textbar{}
\texttt{String} \textbar{} \texttt{null} \textbar{} The name of the
npmjs.com user that publishes the package.
\texttt{overriddenPackageJsonKeys} \textbar{}
\texttt{Set\textless{}String\textgreater{}} \textbar{} \texttt{{[}{]}}
\textbar{} The field values to override in the generated
\texttt{package.json} file.

\paragraph{Task Methods}\label{task-methods-10}

\begin{longtable}[]{@{}
  >{\raggedright\arraybackslash}p{(\columnwidth - 2\tabcolsep) * \real{0.3529}}
  >{\raggedright\arraybackslash}p{(\columnwidth - 2\tabcolsep) * \real{0.6471}}@{}}
\toprule\noalign{}
\begin{minipage}[b]{\linewidth}\raggedright
Method
\end{minipage} & \begin{minipage}[b]{\linewidth}\raggedright
Description
\end{minipage} \\
\midrule\noalign{}
\endhead
\bottomrule\noalign{}
\endlastfoot
\texttt{PublishNodeModuleTask\ overriddenPackageJsonKeys(Iterable\textless{}String\textgreater{}\ overriddenPackageJsonKeys)}
& Adds field values to override in the generated \texttt{package.json}
file. \\
\texttt{PublishNodeModuleTask\ overriddenPackageJsonKeys(String...\ overriddenPackageJsonKeys)}
& Adds field values to override in the generated \texttt{package.json}
file. \\
\end{longtable}

\subsubsection{npmRun\$\{script\} Task}\label{npmrunscript-task}

For each \href{https://docs.npmjs.com/misc/scripts}{script} declared in
the \texttt{package.json} file of the project, one task
\texttt{npmRun\$\{script\}} of type
\hyperref[executenpmtask]{\texttt{ExecuteNpmTask}} is added. Each of
these tasks is automatically configured with sensible defaults:

Property Name \textbar{} Default Value \texttt{args} \textbar{}
\texttt{{[}"run-script",\ "\$\{script\}"{]}}

If the
\href{https://docs.gradle.org/current/userguide/java_plugin.html}{\texttt{java}}
plugin is applied and the \texttt{package.json} file declares a script
named \texttt{"build"}, the script is executed before the
\texttt{classes} task but after the
\href{https://docs.gradle.org/4.0/userguide/java_plugin.html\#sec:java_resources}{\texttt{processResources}}
task.

If the
\href{https://docs.gradle.org/current/javadoc/org/gradle/language/base/plugins/LifecycleBasePlugin.html}{\texttt{lifecycle-base}}
plugin is applied and the \texttt{package.json} file declares a script
named \texttt{test}, the script is executed when running the
\texttt{check} task.

\section{REST Builder Gradle Plugin}\label{rest-builder-gradle-plugin}

The REST Builder Gradle plugin lets you generate a REST layer defined in
the REST Builder \texttt{rest-config.yaml} and
\texttt{rest-openapi.yaml} files.

The plugin has been successfully tested with Gradle 4.10.2.

\subsection{Usage}\label{usage-17}

To use the plugin, include it in your build script:

\begin{verbatim}
buildscript {
    dependencies {
        classpath group: "com.liferay", name: "com.liferay.gradle.plugins.rest.builder", version: "1.0.21"
    }

    repositories {
        maven {
            url "https://repository-cdn.liferay.com/nexus/content/groups/public"
        }
    }
}

apply plugin: "com.liferay.portal.tools.rest.builder"
\end{verbatim}

The REST Builder plugin automatically applies the
\href{https://docs.gradle.org/current/userguide/java_plugin.html}{\texttt{java}}
plugin.

Since the plugin automatically resolves the
\href{https://github.com/liferay/liferay-portal/tree/master/modules/util/portal-tools-rest-builder}{Liferay
REST Builder} library as a dependency, you have to configure a
repository that hosts the library and its transitive dependencies. The
Liferay CDN repository hosts them all:

\begin{verbatim}
repositories {
    maven {
        url "https://repository-cdn.liferay.com/nexus/content/groups/public"
    }
}
\end{verbatim}

\subsection{Tasks}\label{tasks-16}

The plugin adds one task to your project:

Name \textbar{} Depends On \textbar{} Type \textbar{} Description
\texttt{buildREST} \textbar{} - \textbar{}
\hyperref[buildresttask]{\texttt{BuildRESTTask}} \textbar{} Runs the
Liferay REST Builder.

\subsubsection{BuildRESTTask}\label{buildresttask}

Tasks of type \texttt{BuildRESTTask} extend
\href{https://docs.gradle.org/current/dsl/org.gradle.api.tasks.JavaExec.html}{\texttt{JavaExec}},
so all its properties and methods, such as
\href{https://docs.gradle.org/current/dsl/org.gradle.api.tasks.JavaExec.html\#org.gradle.api.tasks.JavaExec:args(java.lang.Iterable)}{\texttt{args}}
and
\href{https://docs.gradle.org/current/dsl/org.gradle.api.tasks.JavaExec.html\#org.gradle.api.tasks.JavaExec:maxHeapSize}{\texttt{maxHeapSize}}
are available. They also have the following properties set by default:

Property Name \textbar{} Default Value
\href{https://docs.gradle.org/current/dsl/org.gradle.api.tasks.JavaExec.html\#org.gradle.api.tasks.JavaExec:args}{\texttt{args}}
\textbar{} REST Builder command line arguments
\href{https://docs.gradle.org/current/dsl/org.gradle.api.tasks.JavaExec.html\#org.gradle.api.tasks.JavaExec:classpath}{\texttt{classpath}}
\textbar{}
\hyperref[liferay-rest-builder-dependency]{\texttt{project.configurations.restBuilder}}
\href{https://docs.gradle.org/current/dsl/org.gradle.api.tasks.JavaExec.html\#org.gradle.api.tasks.JavaExec:main}{\texttt{main}}
\textbar{} \texttt{"com.liferay.portal.tools.rest.builder.RESTBuilder"}
\href{https://docs.gradle.org/current/dsl/org.gradle.api.tasks.JavaExec.html\#org.gradle.api.tasks.JavaExec:systemProperties}{\texttt{systemProperties}}
\textbar{} \texttt{{[}{]}}

\paragraph{Task Properties}\label{task-properties-24}

Property Name \textbar{} Type \textbar{} Default Value \textbar{}
Description \texttt{copyrightFile} \textbar{} \texttt{File} \textbar{}
\texttt{null} \textbar{} The file that contains the copyright header.
\texttt{restConfigDir} \textbar{} \texttt{File}
\textbar{}\texttt{\$\{project.projectDir\}} \textbar{} The directory
that contains the \texttt{rest-config.yaml} and
\texttt{rest-openapi.yaml} files.

In the typical scenario, the \texttt{rest-config.yaml} and
\texttt{rest-openapi.yaml} files are contained in the project directory
of \texttt{my-rest-app-impl}. In the \texttt{build.gradle} of the same
module, apply the \texttt{com.liferay.rest.builder} plugin.

The properties of type \texttt{File} supports any type that can be
resolved by
\href{https://docs.gradle.org/current/dsl/org.gradle.api.Project.html\#org.gradle.api.Project:file(java.lang.Object)}{\texttt{project.file}}.
Moreover, it is possible to use Closures and Callables as values for the
\texttt{String} properties, to defer evaluation until task execution.

\subsection{Additional Configuration}\label{additional-configuration-9}

There are additional configurations added to use REST Builder.

\subsubsection{Liferay REST Builder
Dependency}\label{liferay-rest-builder-dependency}

By default, the plugin creates a configuration called
\texttt{restBuilder} and adds a dependency to the latest released
version of Liferay REST Builder.

\begin{verbatim}
dependencies {
    restBuilder group: "com.liferay", name: "com.liferay.portal.tools.rest.builder", version: "1.0.22"
}
\end{verbatim}

\section{Service Builder Gradle
Plugin}\label{service-builder-gradle-plugin}

The Service Builder Gradle plugin lets you generate a service layer
defined in a
\href{/docs/7-1/tutorials/-/knowledge_base/t/what-is-service-builder}{Service
Builder} \texttt{service.xml} file.

The plugin has been successfully tested with Gradle 4.10.2.

\subsection{Usage}\label{usage-18}

To use the plugin, include it in your build script:

\begin{verbatim}
buildscript {
    dependencies {
        classpath group: "com.liferay", name: "com.liferay.gradle.plugins.service.builder", version: "2.2.46"
    }

    repositories {
        maven {
            url "https://repository-cdn.liferay.com/nexus/content/groups/public"
        }
    }
}

apply plugin: "com.liferay.portal.tools.service.builder"
\end{verbatim}

The Service Builder plugin automatically applies the
\href{https://docs.gradle.org/current/userguide/java_plugin.html}{\texttt{java}}
plugin.

Since the plugin automatically resolves the
\href{https://github.com/liferay/liferay-portal/tree/master/modules/util/portal-tools-service-builder}{Liferay
Service Builder} library as a dependency, you have to configure a
repository that hosts the library and its transitive dependencies. The
Liferay CDN repository hosts them all:

\begin{verbatim}
repositories {
    maven {
        url "https://repository-cdn.liferay.com/nexus/content/groups/public"
    }
}
\end{verbatim}

\subsection{Tasks}\label{tasks-17}

The plugin adds one task to your project:

Name \textbar{} Depends On \textbar{} Type \textbar{} Description
\texttt{buildService} \textbar{} - \textbar{}
\hyperref[buildservicetask]{\texttt{BuildServiceTask}} \textbar{} Runs
the Liferay Service Builder.

The \texttt{buildService} task is automatically configured with sensible
defaults, depending on whether the
\href{https://docs.gradle.org/current/userguide/war_plugin.html}{\texttt{war}}
plugin is applied, or whether the
\hyperref[osgimodule]{\texttt{osgiModule}} property is \texttt{true}:

Property Name \textbar{} Default Value
\hyperref[apidir]{\texttt{apiDir}} \textbar{}

\textbf{If the \texttt{war} plugin is applied:}
\texttt{\$\{project.webAppDir\}/WEB-INF/service}

\textbf{Otherwise:} \texttt{null}

\hyperref[hbmfile]{\texttt{hbmFile}} \textbar{}

\textbf{If \texttt{osgiModule} is \texttt{true}:}
\texttt{\$\{buildService.resourcesDir\}/META-INF/module-hbm.xml}

\textbf{Otherwise:}
\texttt{\$\{buildService.resourcesDir\}/META-INF/portlet-hbm.xml}

\hyperref[impldir]{\texttt{implDir}} \textbar{} The first \texttt{java}
directory of the \texttt{main} source set (by default:
\texttt{src/main/java}). \hyperref[inputfile]{\texttt{inputFile}}
\textbar{}

\textbf{If the \texttt{war} plugin is applied:}
\texttt{\$\{project.webAppDir\}/WEB-INF/service.xml}

\textbf{Otherwise:} \texttt{\$\{project.projectDir\}/service.xml}

\hyperref[modelhintsfile]{\texttt{modelHintsFile}} \textbar{} The file
\texttt{META-INF/portlet-model-hints.xml} in the first
\texttt{resources} directory of the \texttt{main} source set (by
default: \texttt{src/main/resources/META-INF/portlet-model-hints.xml}).
\hyperref[pluginname]{\texttt{pluginName}} \textbar{}

\textbf{If \texttt{osgiModule} is \texttt{true}:} \texttt{""}

\textbf{Otherwise:} \texttt{project.name}

\hyperref[pluginname]{\texttt{propsUtil}} \textbar{}

\textbf{If \texttt{osgiModule} is \texttt{true}:}
\texttt{"\$\{bundleSymbolicName\}.util.ServiceProps"}The
\texttt{bundleSymbolicName} of the project is inferred via the
\href{https://github.com/gradle/gradle/blob/master/subprojects/osgi/src/main/java/org/gradle/api/internal/plugins/osgi/OsgiHelper.java}{\texttt{OsgiHelper}}
class.

\textbf{Otherwise:} \texttt{"com.liferay.util.service.ServiceProps"}

\hyperref[resourcesdir]{\texttt{resourcesDir}} \textbar{} The first
\texttt{resources} directory of the \texttt{main} source set (by
default: \texttt{src/main/resources}).
\hyperref[springfile]{\texttt{springFile}} \textbar{}

\textbf{If \texttt{osgiModule} is \texttt{true}:} the file
\texttt{META-INF/spring/module-spring.xml} in the first
\texttt{resources} directory of the \texttt{main} source set (by
default: \texttt{src/main/resources/META-INF/spring/module-spring.xml})

\textbf{Otherwise:} the file \texttt{META-INF/portlet-spring.xml} in the
first \texttt{resources} directory of the \texttt{main} source set (by
default: \texttt{src/main/resources/META-INF/portlet-spring.xml})

\hyperref[sqldir]{\texttt{sqlDir}} \textbar{}

\textbf{If the \texttt{war} plugin is applied:}
\texttt{\$\{project.webAppDir\}/WEB-INF/sql}

\textbf{Otherwise:} The directory \texttt{META-INF/sql} in the first
\texttt{resources} directory of the \texttt{main} source set (by
default: \texttt{src/main/resources/META-INF/sql}).

In the
\href{/docs/7-1/tutorials/-/knowledge_base/t/defining-an-object-relational-map-with-service-builder}{typical
scenario} of a data-driven Liferay OSGi application split in
\texttt{myapp-app}, \texttt{myapp-service} and \texttt{myapp-web}
modules, the \texttt{service.xml} file is usually contained in the root
directory of \texttt{myapp-service}. In the \texttt{build.gradle} of the
same module, it is enough to apply the
\texttt{com.liferay.service.builder} plugin \hyperref[usage]{as
described}, and then add the following snippet to enable the use of
Liferay Service Builder:

\begin{verbatim}
buildService {
    apiDir = "../myapp-api/src/main/java"
    testDir = "../myapp-test/src/testIntegration/java"
}
\end{verbatim}

While \texttt{apiDir} is required, the \texttt{testDir} property
assignment can be left out, in which case Arquillian-based integration
test classes are generated.

\subsubsection{BuildServiceTask}\label{buildservicetask}

Tasks of type \texttt{BuildWSDDTask} extend
\href{https://docs.gradle.org/current/dsl/org.gradle.api.tasks.JavaExec.html}{\texttt{JavaExec}},
so all its properties and methods, such as
\href{https://docs.gradle.org/current/dsl/org.gradle.api.tasks.JavaExec.html\#org.gradle.api.tasks.JavaExec:args(java.lang.Iterable)}{\texttt{args}}
and
\href{https://docs.gradle.org/current/dsl/org.gradle.api.tasks.JavaExec.html\#org.gradle.api.tasks.JavaExec:maxHeapSize}{\texttt{maxHeapSize}}
are available. They also have the following properties set by default:

Property Name \textbar{} Default Value
\href{https://docs.gradle.org/current/dsl/org.gradle.api.tasks.JavaExec.html\#org.gradle.api.tasks.JavaExec:args}{\texttt{args}}
\textbar{} Service Builder command line arguments
\href{https://docs.gradle.org/current/dsl/org.gradle.api.tasks.JavaExec.html\#org.gradle.api.tasks.JavaExec:classpath}{\texttt{classpath}}
\textbar{}
\hyperref[liferay-service-builder-dependency]{\texttt{project.configurations.serviceBuilder}}
\href{https://docs.gradle.org/current/dsl/org.gradle.api.tasks.JavaExec.html\#org.gradle.api.tasks.JavaExec:main}{\texttt{main}}
\textbar{}
\texttt{"com.liferay.portal.tools.service.builder.ServiceBuilder"}
\href{https://docs.gradle.org/current/dsl/org.gradle.api.tasks.JavaExec.html\#org.gradle.api.tasks.JavaExec:systemProperties}{\texttt{systemProperties}}
\textbar{} \texttt{{[}"file.encoding":\ "UTF-8"{]}}

\paragraph{Task Properties}\label{task-properties-25}

Property Name \textbar{} Type \textbar{} Default Value \textbar{}
Description \texttt{apiDir} \textbar{} \texttt{File} \textbar{}
\texttt{null} \textbar{} A directory where the service API Java source
files are generated. It sets the \texttt{service.api.dir} argument.
\texttt{autoImportDefaultReferences} \textbar{} \texttt{boolean}
\textbar{} \texttt{true} \textbar{} Whether to automatically add default
references, like \texttt{com.liferay.portal.ClassName},
\texttt{com.liferay.portal.Resource} and
\texttt{com.liferay.portal.User}, to the services. It sets the
\texttt{service.auto.import.default.references} argument.
\texttt{autoNamespaceTables} \textbar{} \texttt{boolean} \textbar{}
\texttt{true} \textbar{} Whether to prefix table names by the namespace
specified in the \texttt{service.xml} file. It sets the
\texttt{service.auto.namespace.tables} argument.
\texttt{beanLocatorUtil} \textbar{} \texttt{String} \textbar{}
\texttt{"com.liferay.util.bean.PortletBeanLocatorUtil"} \textbar{} The
fully qualified class name of a bean locator class to use in the
generated service classes. It sets the
\texttt{service.bean.locator.util} argument. \texttt{buildNumber}
\textbar{} \texttt{long} \textbar{} \texttt{1} \textbar{} A specific
value to assign the \texttt{build.number} property in the
\texttt{service.properties} file. It sets the
\texttt{service.build.number} argument. \texttt{buildNumberIncrement}
\textbar{} \texttt{boolean} \textbar{} \texttt{true} \textbar{} Whether
to automatically increment the \texttt{build.number} property in the
\texttt{service.properties} file by one at every service generation. It
sets the \texttt{service.build.number.increment} argument.
\texttt{databaseNameMaxLength} \textbar{} \texttt{int} \textbar{}
\texttt{30} \textbar{} The upper bound for database table and column
name lengths to ensure it works on all databases. It sets the
\texttt{service.database.name.max.length} argument. \texttt{hbmFile}
\textbar{} \texttt{File} \textbar{} \texttt{null} \textbar{} A Hibernate
Mapping file to generate. It sets the \texttt{service.hbm.file}
argument. \texttt{implDir} \textbar{} \texttt{File} \textbar{}
\texttt{null} \textbar{} A directory where the service Java source files
are generated. It sets the \texttt{service.impl.dir} argument.
\texttt{inputFile} \textbar{} \texttt{File} \textbar{} \texttt{null}
\textbar{} The project's \texttt{service.xml} file. It sets the
\texttt{service.input.file} argument. \texttt{modelHintsConfigs}
\textbar{} \texttt{Set} \textbar{}
\texttt{{[}"classpath*:META-INF/portal-model-hints.xml",\ "META-INF/portal-model-hints.xml",\ "classpath*:META-INF/ext-model-hints.xml",\ "classpath*:META-INF/portlet-model-hints.xml"{]}}
\textbar{} Paths to the
\href{/docs/7-1/tutorials/-/knowledge_base/t/customizing-model-entities-with-model-hints}{model
hints} files for Liferay Service Builder to use in generating the
service layer. It sets the \texttt{service.model.hints.configs}
argument. \texttt{modelHintsFile} \textbar{} \texttt{File} \textbar{}
\texttt{null} \textbar{} A model hints file for the project. It sets the
\texttt{service.model.hints.file} argument. \texttt{osgiModule}
\textbar{} \texttt{boolean} \textbar{} \texttt{false} \textbar{} Whether
to generate the service layer for OSGi modules. It sets the
\texttt{service.osgi.module} argument. \texttt{pluginName} \textbar{}
\texttt{String} \textbar{} \texttt{null} \textbar{} If specified, a
plugin can enable additional generation features, such as \texttt{Clp}
class generation, for non-OSGi modules. It sets the
\texttt{service.plugin.name} argument. \texttt{propsUtil} \textbar{}
\texttt{String} \textbar{} \texttt{null} \textbar{} The fully qualified
class name of the service properties util class to generate. It sets the
\texttt{service.props.util} argument. \texttt{readOnlyPrefixes}
\textbar{} \texttt{Set} \textbar{}
\texttt{{[}"fetch",\ "get",\ "has",\ "is",\ "load",\ "reindex",\ "search"{]}}
\textbar{} Prefixes of methods to consider read-only. It sets the
\texttt{service.read.only.prefixes} argument.
\texttt{resourceActionsConfigs} \textbar{} \texttt{Set} \textbar{}
\texttt{{[}"META-INF/resource-actions/default.xml",\ "resource-actions/default.xml"{]}}
\textbar{} Paths to the
\href{/docs/7-1/tutorials/-/knowledge_base/t/defining-application-permissions}{resource
actions} files for Liferay Service Builder to use in generating the
service layer. It sets the \texttt{service.resource.actions.configs}
argument. \texttt{resourcesDir} \textbar{} \texttt{File} \textbar{}
\texttt{null} \textbar{} A directory where the service non-Java files
are generated. It sets the \texttt{service.resources.dir} argument.
\texttt{springFile} \textbar{} \texttt{File} \textbar{} \texttt{null}
\textbar{} A service Spring file to generate. It sets the
\texttt{service.spring.file} argument. \texttt{springNamespaces}
\textbar{} \texttt{Set} \textbar{} \texttt{{[}"beans"{]}} \textbar{}
Namespaces of Spring XML Schemas to add to the service Spring file. It
sets the \texttt{service.spring.namespaces} argument. \texttt{sqlDir}
\textbar{} \texttt{File} \textbar{} \texttt{null} \textbar{} A directory
where the SQL files are generated. It sets the \texttt{service.sql.dir}
argument. \texttt{sqlFileName} \textbar{} \texttt{String} \textbar{}
\texttt{"tables.sql"} \textbar{} A name (relative to \texttt{sqlDir})
for the file in which the SQL table creation instructions are generated.
It sets the \texttt{service.sql.file} argument.
\texttt{sqlIndexesFileName} \textbar{} \texttt{String} \textbar{}
\texttt{"indexes.sql"} \textbar{} A name (relative to \texttt{sqlDir})
for the file in which the SQL index creation instructions are generated.
It sets the \texttt{service.sql.indexes.file} argument.
\texttt{sqlSequencesFileName} \textbar{} \texttt{String} \textbar{}
\texttt{"sequences.sql"} \textbar{} A name (relative to \texttt{sqlDir})
for the file in which the SQL sequence creation instructions are
generated. It sets the \texttt{service.sql.sequences.file} argument.
\texttt{targetEntityName} \textbar{} \texttt{String} \textbar{}
\texttt{null} \textbar{} If specified, it's the name of the entity for
which Liferay Service Builder should generate the service. It sets the
\texttt{service.target.entity.name} argument. \texttt{testDir}
\textbar{} \texttt{File} \textbar{} \texttt{null} \textbar{} If
specified, it's a directory where integration test Java source files are
generated. It sets the \texttt{service.test.dir} argument.
\texttt{uadDir} \textbar{} \texttt{File} \textbar{} \texttt{null}
\textbar{} A directory where the UAD (user-associated data) Java source
files are generated. It sets the \texttt{service.uad.dir} argument.
\texttt{uadTestIntegrationDir} \textbar{} \texttt{File} \textbar{}
\texttt{null} \textbar{} A directory where integration test UAD
(user-associated data) Java source files are generated. It sets the
\texttt{service.uad.test.integration.dir} argument.

The properties of type \texttt{File} supports any type that can be
resolved by
\href{https://docs.gradle.org/current/dsl/org.gradle.api.Project.html\#org.gradle.api.Project:file(java.lang.Object)}{\texttt{project.file}}.
Moreover, it is possible to use Closures and Callables as values for the
\texttt{String} properties, to defer evaluation until task execution.

\subsection{Additional Configuration}\label{additional-configuration-10}

There are additional configurations that can help you use Service
Builder.

\subsubsection{Liferay Service Builder
Dependency}\label{liferay-service-builder-dependency}

By default, the plugin creates a configuration called
\texttt{serviceBuilder} and adds a dependency to the latest released
version of Liferay Service Builder. It is possible to override this
setting and use a specific version of the tool by manually adding a
dependency to the \texttt{serviceBuilder} configuration:

\begin{verbatim}
dependencies {
    serviceBuilder group: "com.liferay", name: "com.liferay.portal.tools.service.builder", version: "1.0.292"
}
\end{verbatim}

If you're applying the
\href{https://github.com/liferay/liferay-portal/tree/master/modules/sdk/gradle-plugins}{\texttt{com.liferay.gradle.plugins}}
or
\href{https://github.com/liferay/liferay-portal/blob/master/modules/sdk/gradle-plugins-workspace}{\texttt{com.liferay.gradle.plugins.workspace}}
plugins to your project, the Service Builder dependency is already added
to the \texttt{serviceBuilder} configuration. Therefore, if you try to
apply a customized version of Service Builder, it's not recognized; you
must override the configuration already applied.

To do this, you must customize the classpath of the
\texttt{buildService} task. If you're supplying the customized Service
Builder plugin through a module named \texttt{custom-sb-api}, you could
modify the \texttt{buildService} task like this:

\begin{verbatim}
buildService {
    apiDir = "../custom-sb-api/src/main/java"
    classpath = configurations.serviceBuilder.filter { file -> !file.name.contains("com.liferay.portal.tools.service.builder") }
}
\end{verbatim}

If you do this in conjunction with the \texttt{serviceBuilder}
dependency configuration, the custom Service Builder version is used.

\section{Source Formatter Gradle
Plugin}\label{source-formatter-gradle-plugin}

The Source Formatter Gradle plugin lets you format project files using
the
\href{https://github.com/liferay/liferay-portal/tree/master/modules/util/source-formatter}{Liferay
Source Formatter} tool.

The plugin has been successfully tested with Gradle 4.10.2.

\subsection{Usage}\label{usage-19}

To use the plugin, include it in your build script:

\begin{verbatim}
buildscript {
    dependencies {
        classpath group: "com.liferay", name: "com.liferay.gradle.plugins.source.formatter", version: "2.3.413"
    }

    repositories {
        maven {
            url "https://repository-cdn.liferay.com/nexus/content/groups/public"
        }
    }
}

apply plugin: "com.liferay.source.formatter"
\end{verbatim}

Since the plugin automatically resolves the Liferay Source Formatter
library as a dependency, you have to configure a repository that hosts
the library and its transitive dependencies. The Liferay CDN repository
hosts them all:

\begin{verbatim}
repositories {
    maven {
        url "https://repository-cdn.liferay.com/nexus/content/groups/public"
    }
}
\end{verbatim}

\subsection{Tasks}\label{tasks-18}

The plugin adds two tasks to your project:

Name \textbar{} Depends On \textbar{} Type \textbar{} Description
\texttt{checkSourceFormatting} \textbar{} - \textbar{}
\hyperref[formatsourcetask]{\texttt{FormatSourceTask}} \textbar{} Runs
the Liferay Source Formatter to check for source formatting errors.
\texttt{formatSource} \textbar{} - \textbar{}
\hyperref[formatsourcetask]{\texttt{FormatSourceTask}} \textbar{} Runs
the Liferay Source Formatter to format the project files.

If desired, it is possible to check for source formatting errors while
executing the
\href{https://docs.gradle.org/current/userguide/java_plugin.html\#N15056}{\texttt{check}}
task by adding the following dependency:

\begin{verbatim}
check {
    dependsOn checkSourceFormatting
}
\end{verbatim}

The same can be achieved by adding the following snippet to the
\texttt{build.gradle} file in the root directory of a
\href{/docs/7-1/tutorials/-/knowledge_base/t/liferay-workspace}{\emph{Liferay
Workspace}}:

\begin{verbatim}
subprojects {
    afterEvaluate {
        if (plugins.hasPlugin("base") && plugins.hasPlugin("com.liferay.source.formatter")) {
            check.dependsOn checkSourceFormatting
        }
    }
}
\end{verbatim}

The tasks \texttt{checkSourceFormatting} and \texttt{formatSource} are
automatically skipped if another task with the same name is being
executed in a parent project.

\subsubsection{FormatSourceTask}\label{formatsourcetask}

Tasks of type \texttt{FormatSourceTask} extend
\href{https://docs.gradle.org/current/dsl/org.gradle.api.tasks.JavaExec.html}{\texttt{JavaExec}},
so all its properties and methods, like
\href{https://docs.gradle.org/current/dsl/org.gradle.api.tasks.JavaExec.html\#org.gradle.api.tasks.JavaExec:args(java.lang.Iterable)}{\texttt{args}}
and
\href{https://docs.gradle.org/current/dsl/org.gradle.api.tasks.JavaExec.html\#org.gradle.api.tasks.JavaExec:maxHeapSize}{\texttt{maxHeapSize}}
are available. They also have the following properties set by default:

Property Name \textbar{} Default Value
\href{https://docs.gradle.org/current/dsl/org.gradle.api.tasks.JavaExec.html\#org.gradle.api.tasks.JavaExec:args}{\texttt{args}}
\textbar{} Source Formatter command line arguments
\href{https://docs.gradle.org/current/dsl/org.gradle.api.tasks.JavaExec.html\#org.gradle.api.tasks.JavaExec:classpath}{\texttt{classpath}}
\textbar{}
\hyperref[liferay-source-formatter-dependency]{\texttt{project.configurations.sourceFormatter}}
\href{https://docs.gradle.org/current/dsl/org.gradle.api.tasks.JavaExec.html\#org.gradle.api.tasks.JavaExec:main}{\texttt{main}}
\textbar{} \texttt{"com.liferay.source.formatter.SourceFormatter"}

\paragraph{Task Properties}\label{task-properties-26}

Property Name \textbar{} Type \textbar{} Default Value \textbar{}
Description \texttt{autoFix} \textbar{} \texttt{boolean} \textbar{}
\texttt{false} \textbar{} Whether to automatically fix source formatting
errors. It sets the \texttt{source.auto.fix} argument. \texttt{baseDir}
\textbar{} \texttt{File} \textbar{} \textbar{} The Source Formatter base
directory. It sets the \texttt{source.base.dir} argument.
\emph{(Read-only)} \texttt{baseDirName} \textbar{} \texttt{String}
\textbar{} \texttt{"./"} \textbar{} The name of the Source Formatter
base directory, relative to the project directory.
\texttt{fileExtensions} \textbar{}
\texttt{List\textless{}String\textgreater{}} \textbar{} \texttt{{[}{]}}
\textbar{} The file extensions to format. If empty, all file extensions
will be formatted. It sets the \texttt{source.file.extensions} argument.
\texttt{files} \textbar{} \texttt{List\textless{}File\textgreater{}}
\textbar{} \textbar{} The list of files to format. It sets the
\texttt{source.files} argument. \emph{(Read-only)} \texttt{fileNames}
\textbar{} \texttt{List\textless{}String\textgreater{}} \textbar{}
\texttt{null} \textbar{} The file names to format, relative to the
project directory. If \texttt{null}, all files contained in
\texttt{baseDir} will be formatted. \texttt{formatCurrentBranch}
\textbar{} \texttt{boolean} \textbar{} \texttt{false} \textbar{} Whether
to format only the files contained in \texttt{baseDir} that are added or
modified in the current Git branch. It sets the
\texttt{format.current.branch} argument. \texttt{formatLatestAuthor}
\textbar{} \texttt{boolean} \textbar{} \texttt{false} \textbar{} Whether
to format only the files contained in \texttt{baseDir} that are added or
modified in the latest Git commits of the same author. It sets the
\texttt{format.latest.author} argument. \texttt{formatLocalChanges}
\textbar{} \texttt{boolean} \textbar{} \texttt{false} \textbar{} Whether
to format only the unstaged files contained in \texttt{baseDir}. It sets
the \texttt{format.local.changes} argument.
\texttt{gitWorkingBranchName} \textbar{} \texttt{String} \textbar{}
\texttt{"master"} \textbar{} The Git working branch name. It sets the
\texttt{git.working.branch.name} argument.
\texttt{includeSubrepositories} \textbar{} \texttt{boolean} \textbar{}
\texttt{false} \textbar{} Whether to format files that are in read-only
subrepositories. It sets the \texttt{include.subrepositories} argument.
\texttt{maxLineLength} \textbar{} \texttt{int} \textbar{} \texttt{80}
\textbar{} The maximum number of characters allowed in Java files. It
sets the \texttt{max.line.length} argument. \texttt{printErrors}
\textbar{} \texttt{boolean} \textbar{} \texttt{true} \textbar{} Whether
to print formatting errors on the Standard Output stream. It sets the
\texttt{source.print.errors} argument. \texttt{processorThreadCount}
\textbar{} \texttt{int} \textbar{} \texttt{5} \textbar{} The number of
threads used by Source Formatter. It sets the
\texttt{processor.thread.count} argument. \texttt{showDebugInformation}
\textbar{} \texttt{boolean} \textbar{} \texttt{false} \textbar{} Whether
to show debug information, if present. It sets the
\texttt{show.debug.information} argument. \texttt{showDocumentation}
\textbar{} \texttt{boolean} \textbar{} \texttt{false} \textbar{} Whether
to show the documentation for the source formatting issues, if present.
It sets the \texttt{show.documentation} argument.
\texttt{showStatusUpdates} \textbar{} \texttt{boolean} \textbar{}
\texttt{false} \textbar{} Whether to show status updates during source
formatting, if present. It sets the \texttt{show.status.updates}
argument. \texttt{throwException} \textbar{} \texttt{boolean} \textbar{}
\texttt{false} \textbar{} Whether to fail the build if formatting errors
are found. It sets the \texttt{source.throw.exception} argument.

\subsection{Additional Configuration}\label{additional-configuration-11}

There are additional configurations that can help you use the Source
Formatter.

\subsubsection{Liferay Source Formatter
Dependency}\label{liferay-source-formatter-dependency}

By default, the plugin creates a configuration called
\texttt{sourceFormatter} and adds a dependency to the latest released
version of Liferay Source Formatter. It is possible to override this
setting and use a specific version of the tool by manually adding a
dependency to the \texttt{sourceFormatter} configuration:

\begin{verbatim}
dependencies {
    sourceFormatter group: "com.liferay", name: "com.liferay.source.formatter", version: "1.0.885"
}
\end{verbatim}

\subsubsection{System Properties}\label{system-properties-3}

It is possible to set the default values of the \texttt{fileExtensions},
\texttt{fileNames}, \texttt{formatCurrentBranch},
\texttt{formatLatestAuthor}, and \texttt{formatLocalChanges} properties
for a \texttt{FormatSourceTask} task via system properties:

\begin{itemize}
\tightlist
\item
  \texttt{-D\$\{task.name\}.file.extensions=java,xml}
\item
  \texttt{-D\$\{task.name\}.file.names=README.markdown,src/main/resources/hello.txt}
\item
  \texttt{-D\$\{task.name\}.format.current.branch=true}
\item
  \texttt{-D\$\{task.name\}.format.latest.author=true}
\item
  \texttt{-D\$\{task.name\}.format.local.changes=true}
\end{itemize}

For example, run the following Bash command to format only the unstaged
files in the project:

\begin{verbatim}
./gradlew formatSource -DformatSource.format.local.changes=true
\end{verbatim}

\section{Soy Gradle Plugin}\label{soy-gradle-plugin}

The Soy Gradle plugin lets you compile
\href{https://developers.google.com/closure/templates/}{Closure
Templates} into JavaScript functions. It also lets you use a custom
localization mechanism in the generated \texttt{.soy.js} files by
replacing
\href{https://developers.google.com/closure/templates/docs/translation\#closurecompiler}{\texttt{goog.getMsg}}
definitions with a different function call (e.g.,
\texttt{Liferay.Language.get}).

The plugin has been successfully tested with Gradle 4.10.2.

\subsection{Usage}\label{usage-20}

To use the plugin, include it in your build script:

\begin{verbatim}
buildscript {
    dependencies {
        classpath group: "com.liferay", name: "com.liferay.gradle.plugins.soy", version: "3.1.8"
    }

    repositories {
        maven {
            url "https://repository-cdn.liferay.com/nexus/content/groups/public"
        }
    }
}
\end{verbatim}

There are two Soy Gradle plugins you can apply to your project:

\begin{itemize}
\item
  Apply the \hyperref[soy-plugin]{\emph{Soy Plugin}} to compile Closure
  Templates into JavaScript functions:

\begin{verbatim}
apply plugin: "com.liferay.soy"
\end{verbatim}
\item
  Apply the \hyperref[soy-translation-plugin]{\emph{Soy Translation
  Plugin}} to use a custom localization mechanism in the generated
  \texttt{.soy.js} files:

\begin{verbatim}
apply plugin: "com.liferay.soy.translation"
\end{verbatim}
\end{itemize}

Since the Soy Gradle plugin automatically resolves the Soy library as a
dependency, you have to configure a repository that hosts the library
and its transitive dependencies. The Liferay CDN repository hosts them
all:

\begin{verbatim}
repositories {
    maven {
        url "https://repository-cdn.liferay.com/nexus/content/groups/public"
    }
}
\end{verbatim}

\subsection{Soy Plugin}\label{soy-plugin}

The Soy plugin adds two tasks to your project:

Name \textbar{} Depends On \textbar{} Type \textbar{} Description
\texttt{buildSoy} \textbar{} - \textbar{}
\hyperref[buildsoytask]{\texttt{BuildSoyTask}} \textbar{} Compiles
Closure Templates into JavaScript functions.
\texttt{wrapSoyAlloyTemplate} \textbar{} - \texttt{configJSModules} if
\href{https://github.com/liferay/liferay-portal/tree/master/modules/sdk/gradle-plugins-js-module-config-generator}{\texttt{com.liferay.js.module.config.generator}}
is applied - \texttt{processResources} if \texttt{java} is applied -
\texttt{transpileJS} if
\href{https://github.com/liferay/liferay-portal/tree/master/modules/sdk/gradle-plugins-js-transpiler}{\texttt{com.liferay.js.transpiler}}
is applied \textbar{}
\hyperref[wrapsoyalloytemplatetask]{\texttt{WrapSoyAlloyTemplateTask}}
\textbar{} Wraps the JavaScript functions compiled from Closure
Templates into AlloyUI modules.

The plugin also adds the following dependencies to tasks defined by the
\texttt{java} plugin:

Name \textbar{} Depends On \texttt{classes} \textbar{}
\texttt{wrapSoyAlloyTemplate}

The \texttt{buildSoy} task is automatically configured with sensible
defaults, depending on whether the
\href{https://docs.gradle.org/current/userguide/java_plugin.html}{\texttt{java}}
plugin is applied:

Property Name \textbar{} Default Value
\href{https://docs.gradle.org/current/dsl/org.gradle.api.tasks.SourceTask.html\#org.gradle.api.tasks.SourceTask:includes}{\texttt{includes}}
\textbar{} \texttt{{[}"**/*.soy"{]}}
\href{https://docs.gradle.org/current/dsl/org.gradle.api.tasks.SourceTask.html\#org.gradle.api.tasks.SourceTask:source}{\texttt{source}}
\textbar{}

\textbf{If the \texttt{java} plugin is applied:} The first
\texttt{resources} directory of the \texttt{main} source set (by
default, \texttt{src/main/resources}).

\textbf{Otherwise:} \texttt{{[}{]}}

The \texttt{wrapSoyAlloyTemplate} task is \textbf{disabled by default},
and it is automatically configured with sensible defaults, depending on
whether the \texttt{java} plugin is applied:

Property Name \textbar{} Default Value
\href{https://docs.gradle.org/current/dsl/org.gradle.api.Task.html\#org.gradle.api.Task:enabled}{\texttt{enabled}}
\textbar{} \texttt{false}
\href{https://docs.gradle.org/current/dsl/org.gradle.api.tasks.SourceTask.html\#org.gradle.api.tasks.SourceTask:includes}{\texttt{includes}}
\textbar{} \texttt{{[}"**/*.soy.js"{]}}
\href{https://docs.gradle.org/current/dsl/org.gradle.api.tasks.SourceTask.html\#org.gradle.api.tasks.SourceTask:source}{\texttt{source}}
\textbar{}

\textbf{If the \texttt{java} plugin is applied:}
\texttt{project.sourceSets.main.output.resourcesDir}

\textbf{Otherwise:} \texttt{{[}{]}}

\subsubsection{Additional
Configuration}\label{additional-configuration-12}

There are additional configurations that can help you use the Soy
library.

\paragraph{Soy Dependency}\label{soy-dependency}

By default, the plugin creates a configuration called \texttt{soy} and
adds a dependency to the \texttt{2015-04-10} version of the Soy library.
It is possible to override this setting and use a specific version of
the tool by manually adding a dependency to the \texttt{soy}
configuration:

\begin{verbatim}
dependencies {
    soy group: "com.google.template", name: "soy", version: "2015-04-10"
}
\end{verbatim}

\subsection{Soy Translation Plugin}\label{soy-translation-plugin}

The Soy Translation plugin adds one task to your project:

Name \textbar{} Depends On \textbar{} Type \textbar{} Description
\texttt{replaceSoyTranslation} \textbar{} - \texttt{configJSModules} if
\href{https://github.com/liferay/liferay-portal/tree/master/modules/sdk/gradle-plugins-js-module-config-generator}{\texttt{com.liferay.js.module.config.generator}}
is applied - \texttt{processResources} if \texttt{java} is applied -
\texttt{transpileJS} if
\href{https://github.com/liferay/liferay-portal/tree/master/modules/sdk/gradle-plugins-js-transpiler}{\texttt{com.liferay.js.transpiler}}
is applied \textbar{}
\hyperref[replacesoytranslationtask]{\texttt{ReplaceSoyTranslationTask}}
\textbar{} Replaces \texttt{goog.getMsg} definitions with
\texttt{Liferay.Language.get} calls.

The plugin also adds the following dependencies to tasks defined by the
\texttt{java} plugin:

Name \textbar{} Depends On \texttt{classes} \textbar{}
\texttt{replaceSoyTranslation}

The \texttt{replaceSoyTranslation} task is automatically configured with
sensible defaults, depending on whether the \texttt{java} plugin is
applied:

Property Name \textbar{} Default Value
\href{https://docs.gradle.org/current/dsl/org.gradle.api.tasks.SourceTask.html\#org.gradle.api.tasks.SourceTask:includes}{\texttt{includes}}
\textbar{} \texttt{{[}"**/*.soy.js"{]}}
\hyperref[replacementclosure]{\texttt{replacementClosure}} \textbar{}
Replaces \texttt{goog.getMsg} definitions with
\texttt{Liferay.Language.get} calls.
\href{https://docs.gradle.org/current/dsl/org.gradle.api.tasks.SourceTask.html\#org.gradle.api.tasks.SourceTask:source}{\texttt{source}}
\textbar{}

\textbf{If the \texttt{java} plugin is applied:}
\texttt{project.sourceSets.main.output.resourcesDir}

\textbf{Otherwise:} \texttt{{[}{]}}

\subsection{Tasks}\label{tasks-19}

\subsubsection{BuildSoyTask}\label{buildsoytask}

Tasks of type \texttt{BuildSoyTask} extend
\href{https://docs.gradle.org/current/dsl/org.gradle.api.tasks.SourceTask.html}{\texttt{SourceTask}},
so all its properties and methods, such as
\href{https://docs.gradle.org/current/dsl/org.gradle.api.tasks.SourceTask.html\#org.gradle.api.tasks.SourceTask:include(java.lang.Iterable)}{\texttt{include}}
and
\href{https://docs.gradle.org/current/dsl/org.gradle.api.tasks.SourceTask.html\#org.gradle.api.tasks.SourceTask:exclude(java.lang.Iterable)}{\texttt{exclude}},
are available.

\paragraph{Task Properties}\label{task-properties-27}

Property Name \textbar{} Type \textbar{} Default Value \textbar{}
Description \texttt{classpath} \textbar{}
\href{https://docs.gradle.org/current/javadoc/org/gradle/api/file/FileCollection.html}{\texttt{FileCollection}}
\textbar{}
\hyperref[soy-dependency]{\texttt{project.configurations.soy}}
\textbar{} The classpath for executing the
\href{https://github.com/liferay/liferay-portal/tree/master/modules/util/portal-tools-soy-builder}{Liferay
Portal Tools Soy Builder}.

\subsubsection{WrapSoyAlloyTemplateTask}\label{wrapsoyalloytemplatetask}

Tasks of type \texttt{WrapSoyAlloyTemplateTask} extend
\href{https://docs.gradle.org/current/dsl/org.gradle.api.tasks.SourceTask.html}{\texttt{SourceTask}},
so all its properties and methods, such as
\href{https://docs.gradle.org/current/dsl/org.gradle.api.tasks.SourceTask.html\#org.gradle.api.tasks.SourceTask:include(java.lang.Iterable)}{\texttt{include}}
and
\href{https://docs.gradle.org/current/dsl/org.gradle.api.tasks.SourceTask.html\#org.gradle.api.tasks.SourceTask:exclude(java.lang.Iterable)}{\texttt{exclude}},
are available.

\paragraph{Task Properties}\label{task-properties-28}

Property Name \textbar{} Type \textbar{} Default Value \textbar{}
Description \texttt{moduleName} \textbar{} \texttt{String} \textbar{}
\texttt{null} \textbar{} The name of the AlloyUI module.
\texttt{namespace} \textbar{} \texttt{String} \textbar{} \texttt{null}
\textbar{} The namespace of the Closure Templates of the project.

It is possible to use Closures and Callables as values for the
\texttt{String} properties to defer evaluation until task execution.

\subsubsection{ReplaceSoyTranslationTask}\label{replacesoytranslationtask}

The \texttt{ReplaceSoyTranslationTask} task type finds all the
\texttt{goog.getMsg} definitions in the project's files and replaces
them with a custom function call.

\begin{verbatim}
var MSG_EXTERNAL_123 = goog.getMsg('welcome-to-{$releaseInfo}', { 'releaseInfo': opt_data.releaseInfo });
\end{verbatim}

A \texttt{goog.getMsg} definition looks like the example above, and it
has the following components:

\begin{itemize}
\tightlist
\item
  \emph{variable name}: \texttt{MSG\_EXTERNAL\_123}
\item
  \emph{language key}: \texttt{welcome-to-\{\$releaseInfo\}}
\item
  \emph{arguments object}:
  \texttt{\{\ \textquotesingle{}releaseInfo\textquotesingle{}:\ opt\_data.releaseInfo\ \}}
\end{itemize}

Tasks of type \texttt{ReplaceSoyTranslationTask} extend
\href{https://docs.gradle.org/current/dsl/org.gradle.api.tasks.SourceTask.html}{\texttt{SourceTask}},
so all its properties and methods, such as
\href{https://docs.gradle.org/current/dsl/org.gradle.api.tasks.SourceTask.html\#org.gradle.api.tasks.SourceTask:include(java.lang.Iterable)}{\texttt{include}}
and
\href{https://docs.gradle.org/current/dsl/org.gradle.api.tasks.SourceTask.html\#org.gradle.api.tasks.SourceTask:exclude(java.lang.Iterable)}{\texttt{exclude}},
are available.

\paragraph{Task Properties}\label{task-properties-29}

Property Name \textbar{} Type \textbar{} Default Value \textbar{}
Description \texttt{replacementClosure} \textbar{}
\texttt{Closure\textless{}String\textgreater{}} \textbar{} \texttt{null}
\textbar{} The Closure invoked in order to get the replacement for
\texttt{goog.getMsg} definitions. The given Closure is passed the
\emph{variable name}, \emph{language key}, and \emph{arguments object}
as its parameters.

\section{Target Platform Gradle
Plugin}\label{target-platform-gradle-plugin}

The Target Platform Gradle plugin helps with building multiple projects
against a declared API target platform. Java dependencies can be managed
with Maven BOMs and OSGi modules can be resolved against an OSGi
distribution.

The plugin has been successfully tested with Gradle 4.10.2.

\subsection{Usage}\label{usage-21}

To use the plugin, include it in your build script:

\begin{verbatim}
buildscript {
    dependencies {
        classpath group: "com.liferay", name: "com.liferay.gradle.plugins.target.platform", version: "1.1.13"
    }

    repositories {
        maven {
            url "https://repository-cdn.liferay.com/nexus/content/groups/public"
        }
    }
}
\end{verbatim}

There are two Target Platform Gradle plugins you can apply to your
project:

\begin{itemize}
\item
  The \hyperref[target-platform-plugin]{\emph{Target Platform Plugin}}
  helps to configure your projects to build against an established set
  of platform artifacts, including Java and OSGi dependencies.

\begin{verbatim}
apply plugin: "com.liferay.target.platform"
\end{verbatim}
\item
  The \hyperref[target-platform-ide-plugin]{\emph{Target Platform IDE
  Plugin}} is a superset of the Target Platform Plugin (it applies the
  above plugin) and also adds IDE integration for searching and
  debugging source code in the target platform artifacts.

\begin{verbatim}
apply plugin: "com.liferay.target.platform.ide"
\end{verbatim}
\end{itemize}

Since the plugin automatically resolves target platform configurations
as dependencies, you must configure a repository that hosts these
artifacts. The Liferay CDN repository hosts them all:

\begin{verbatim}
repositories {
    maven {
        url "https://repository-cdn.liferay.com/nexus/content/groups/public"
    }
}
\end{verbatim}

\subsection{Target Platform Plugin}\label{target-platform-plugin}

The plugin applies the
\href{https://github.com/spring-gradle-plugins/dependency-management-plugin}{Spring
Dependency Management Plugin} and then adds several specific
configurations to configure the BOMs that are imported to manage Java
dependencies and the various artifacts used in resolving OSGi
dependencies. Also, a new \texttt{resolve} task is added to resolve all
OSGi requirements against a declared distribution artifact.

The plugin adds a series of configurations to your project:

Name \textbar{} Description \texttt{targetPlatformBOMs} \textbar{}
Configures all the BOMs to import as managed dependencies.
\texttt{targetPlatformBundles} \textbar{} Configures all the bundles in
addition to the distro to resolve against. \texttt{targetPlatformDistro}
\textbar{} Configures the distro JAR file to use as base for resolving
against. \texttt{targetPlatformRequirements} \textbar{} Configures the
list of JAR files to use as run requirements for resolving.

The plugin adds a task \texttt{resolve} of type
\hyperref[resolvetask]{\texttt{ResolveTask}} to your project that
performs an OSGi resolve operation using the
\texttt{targetPlatformRequirements} configuration as the basis of the
requirements. The \texttt{targetPlatformBundles} configuration is used
as a repository for the resolver to resolve requirements. Lastly, the
\texttt{targetPlatformDistro} configuration is used to provide the
\emph{distro} artifact for the resolve process. The \emph{distro} is the
artifact that provides all the OSGi capabilities of the target platform.
All of these parameters are used to create a \texttt{bndrun} file that
can be used as input into the Bndrun resolve operation.

\subsection{Target Platform IDE
Plugin}\label{target-platform-ide-plugin}

The plugin applies the \hyperref[target-platform-plugin]{Target
Platform} and the
\href{https://docs.gradle.org/current/userguide/eclipse_plugin.html}{\texttt{eclipse}}
plugins to your project, and also adds a special
\texttt{targetPlatformIDE} configuration, which is used to configure
both the \texttt{eclipse} model and \texttt{idea} plugin model in Gradle
to add all target platform artifacts to the classpath so they are
visible to both Eclipse and IntelliJ's Java Model Search (for looking up
sources to classes).

\subsection{Project Extension}\label{project-extension-7}

The Target Platform plugin exposes the following properties through the
extension named \texttt{targetPlatform}:

Property Name \textbar{} Type \textbar{} Default Value \textbar{}
Description \texttt{ignoreResolveFailures} \textbar{} \texttt{boolean}
\textbar{} \texttt{true} \textbar{} Whether to ignore resolve failures
found when executing tasks of type
\hyperref[resolvetask]{\texttt{ResolveTask}}. \texttt{subprojects}
\textbar{} \texttt{Set\textless{}Project\textgreater{}} \textbar{}
\texttt{project.subprojects} \textbar{} The subprojects to configure
with target platform support, including dependency management and the
\texttt{resolve} task.

The same extension exposes the following methods:

Method \textbar{} Description
\texttt{TargetPlatformExtension\ applyToConfiguration(Iterable\textless{}?\textgreater{}\ configurationNames)}
\textbar{} Adds additional configurations to configure the BOMs that are
imported to manage Java dependencies and the various artifacts used in
resolving OSGi dependencies.
\texttt{TargetPlatformExtension\ applyToConfiguration(Object...\ configurationNames)}
\textbar{} Adds additional configurations to configure the BOMs that are
imported to manage Java dependencies and the various artifacts used in
resolving OSGi dependencies.
\texttt{TargetPlatformExtension\ onlyIf(Closure\textless{}Boolean\textgreater{}\ onlyIfClosure)}
\textbar{} Includes a subproject in the target platform configuration if
the given closure returns \texttt{true}. The closure is evaluated at the
end of the subproject configuration phase and is passed a single
parameter: the subproject. If the closure returns \texttt{false}, the
subproject is not included in the target platform configuration
\texttt{TargetPlatformExtension\ onlyIf(Spec\textless{}Project\textgreater{}\ onlyIfSpec)}
\textbar{} Includes a subproject in the target platform configuration if
the given spec is satisfied. The spec is evaluated at the end of the
subproject configuration phase. If the spec is not satisfied, the
subproject is not included in the target platform configuration.
\texttt{TargetPlatformExtension\ resolveOnlyIf(Closure\textless{}Boolean\textgreater{}\ resolveOnlyIfClosure)}
\textbar{} Includes a subproject in the resolving process (including
both the requirements and bundles configuration) if the given closure
returns \texttt{true}. The closure is evaluated at the end of the
subproject configuration phase and is passed a single parameter: the
subproject. If the closure returns \texttt{false}, the subproject is the
resolution process.
\texttt{TargetPlatformExtension\ resolveOnlyIf(Spec\textless{}Project\textgreater{}\ resolveOnlyIfSpec)}
\textbar{} Includes a subproject in the resolving platform configuration
if the given spec is satisfied. The spec is evaluated at the end of the
subproject configuration phase. If the spec is not satisfied, the
subproject is not included in the target platform configuration.
\texttt{TargetPlatformExtension\ subprojects(Iterable\textless{}Project\textgreater{}\ subprojects)}
\textbar{} Includes additional projects to be configured with Target
Platform support.
\texttt{TargetPlatformExtension\ subprojects(Project...\ subprojects)}
\textbar{} Includes additional projects to be configured with Target
Platform support.

\subsection{Tasks}\label{tasks-20}

\subsubsection{ResolveTask}\label{resolvetask}

The purpose of this task is to resolve an OSGi module (or all OSGi
modules of subprojects) against the available
\texttt{targetPlatformBundles} and \texttt{targetPlatformDistro}
configurations. By default, the \texttt{targetPlatformBundles} are all
the artifacts created by all the subprojects. The
\texttt{targetPlatformDistro} must be set explicitly to a valid
distribution artifact. When the task is performed, a \texttt{bndrun}
file is generated using the specified \texttt{targetPlatformDistro} as
the \texttt{-distro} instruction; the \texttt{-runrequirements} are a
set of \texttt{osgi.identity} requirements for the
\texttt{targetPlatformRequirements} configuration. If the resolve
operation is able to find a valid set of \texttt{-runbundles} that match
the \texttt{-runrequirements}, then the task passes successfully (the
resolution is valid). If a set of run bundles can't be found, the
resolution has failed and the failed requirements are listed as output
of the task.

\paragraph{Task Properties}\label{task-properties-30}

Property Name \textbar{} Type \textbar{} Default Value \textbar{}
Description \texttt{bndrunFile} \textbar{} \texttt{File} \textbar{}
\texttt{null} \textbar{} If this property is specified, it is used as
the \texttt{bndrun} file to input into the resolver.
\texttt{bundlesFileCollection} \textbar{} \texttt{FileCollection}
\textbar{} All JAR files of subprojects with \texttt{jar} task
\textbar{} The input to \texttt{bndrun} resolve operation.
\texttt{distroFileCollection} \textbar{} \texttt{FileCollection}
\textbar{} \texttt{null} \textbar{} The \emph{distro} parameter for the
generated \texttt{bndrun} file. \texttt{ignoreFailures} \textbar{}
\texttt{boolean} \textbar{} \texttt{false} \textbar{} Whether the
\texttt{resolve} task should ignore failing the build for resolution
errors. \texttt{offline} \textbar{} \texttt{boolean} \textbar{}
\texttt{null} \textbar{} Whether to run the bndrun resolve operation in
offline mode. \texttt{requirementsFileCollection} \textbar{}
\texttt{FileCollection} \textbar{}

\textbf{For the root project:} All the output JAR files of the
subprojects.

\textbf{For subprojects:} The output JAR file of the subproject.

\textbar{} For each resolve operation, the requirements must be
specified in the \texttt{bndrun} file; each of the JARs in this
collection generate an \texttt{osgi.identify} requirement in the
\texttt{bndrun} file.

\subsection{Additional Configuration}\label{additional-configuration-13}

There are additional configurations that you can use to configure the
target platform.

\subsubsection{Target Platform BOMs
Dependency}\label{target-platform-boms-dependency}

The plugin creates a configuration called \texttt{targetPlatformBOMs}
with no defaults. You can use this dependency to set which BOMs to
import to configure your target platform.

\begin{verbatim}
dependencies {
    targetPlatformBOMs group: "com.liferay", name: "com.liferay.ce.portal.bom", version: "7.1.0"
    targetPlatformBOMs group: "com.liferay", name: "com.liferay.ce.portal.compile.only", version: "7.1.0"
}
\end{verbatim}

\subsubsection{Target Platform Bundles
Dependency}\label{target-platform-bundles-dependency}

The plugin creates a configuration called
\texttt{targetPlatformBundles}. It is configured with default
dependencies to all resolvable bundles in a multi-project build (e.g.,
all projects in multi-project build that have a \texttt{jar} task). This
can be used to specify additional bundles that should be added to the
set of bundles given to \texttt{resolve} task to resolve against when
checking for OSGi requirements.

\begin{verbatim}
dependencies {
    targetPlatformBundles group: "com.google.guava", name: "guava", version: "23.0"
}
\end{verbatim}

\subsubsection{Target Platform Distro
Dependency}\label{target-platform-distro-dependency}

The plugin creates a configuration called \texttt{targetPlatformDistro}.
It is has no default so you must specify which artifact you want to use
as the distribution to resolve against.

\begin{verbatim}
dependencies {
    targetPlatformDistro group: "com.liferay", name: "com.liferay.ce.portal.distro", version: "7.1.0"
}
\end{verbatim}

If you have created your own custom distro JAR that is available
locally, you can use the \texttt{files} method to add it to the
configuration.

\begin{verbatim}
dependencies {
    targetPlatformDistro files("custom-distro.jar")
}
\end{verbatim}

\subsubsection{Target Platform Requirements
Dependency}\label{target-platform-requirements-dependency}

The plugin creates a configuration called
\texttt{targetPlatformRequirements}. It is configured with default
dependencies to all resolvable bundles in a multi-project build (e.g.,
all projects in multi-project build that have a \texttt{jar} task). This
is can be used to specify additional bundles that should be added to the
set of bundles given to the \texttt{resolve} task to set as
\texttt{osgi.identity} requirements.

\begin{verbatim}
dependencies {
    targetPlatformRequirements group: "com.liferay", name: "com.liferay.other.bundle", version: "1.0"
}
\end{verbatim}

\section{Theme Builder Gradle Plugin}\label{theme-builder-gradle-plugin}

The Theme Builder Gradle plugin lets you run the
\href{https://github.com/liferay/liferay-portal/tree/master/modules/util/portal-tools-theme-builder}{Liferay
Theme Builder} tool to build the Liferay theme files in your project.

The plugin has been successfully tested with Gradle 4.10.2.

\subsection{Usage}\label{usage-22}

To use the plugin, include it in your build script:

\begin{verbatim}
buildscript {
    dependencies {
        classpath group: "com.liferay", name: "com.liferay.gradle.plugins.theme.builder", version: "2.0.7"
    }

    repositories {
        maven {
            url "https://repository-cdn.liferay.com/nexus/content/groups/public"
        }
    }
}

apply plugin: "com.liferay.portal.tools.theme.builder"
\end{verbatim}

The Theme Builder plugin automatically applies the
\href{https://docs.gradle.org/current/userguide/war_plugin.html}{\texttt{war}}
plugin. It also applies the
\href{https://github.com/liferay/liferay-portal/tree/master/modules/sdk/gradle-plugins-css-builder}{\texttt{com.liferay.css.builder}}
plugin to compile the \href{http://sass-lang.com/}{Sass} files in the
theme.

Since the plugin automatically resolves the Liferay Theme Builder
library as a dependency, you have to configure a repository that hosts
the library and its transitive dependencies. The Liferay CDN repository
hosts them all:

\begin{verbatim}
repositories {
    maven {
        url "https://repository-cdn.liferay.com/nexus/content/groups/public"
    }
}
\end{verbatim}

\subsection{Tasks}\label{tasks-21}

The plugin adds one task to your project:

Name \textbar{} Depends On \textbar{} Type \textbar{} Description
\texttt{buildTheme} \textbar{} - \textbar{}
\hyperref[buildthemetask]{\texttt{BuildThemeTask}} \textbar{} Builds the
theme files.

The plugin also adds the following dependencies to tasks defined by the
\texttt{com.liferay.css.builder} and \texttt{war} plugins:

Name \textbar{} Depends On
\href{https://github.com/liferay/liferay-portal/tree/master/modules/sdk/gradle-plugins-css-builder\#tasks}{\texttt{buildCSS}}
\textbar{} \texttt{buildTheme}
\href{https://docs.gradle.org/current/userguide/war_plugin.html\#sec:war_default_settings}{\texttt{war}}
\textbar{} \texttt{buildTheme}

The \texttt{buildCSS} dependency compiles the Sass files contained in
the directory specified by the
\hyperref[outputdir]{\texttt{buildTheme.outputDir}} property. Moreover,
the \texttt{war} task is configured as follows

\begin{itemize}
\tightlist
\item
  exclude the directory specified in the
  \hyperref[diffsdir]{\texttt{buildTheme.diffsDir}} property from the
  WAR file.
\item
  include the files contained in the
  \hyperref[outputdir]{\texttt{buildTheme.outputDir}} directory into the
  WAR file.
\item
  include only the compiled CSS files, not SCSS files, into the WAR
  file.
\end{itemize}

The \texttt{buildTheme} task is automatically configured with sensible
defaults:

Property Name \textbar{} Default Value
\hyperref[diffsdir]{\texttt{diffsDir}} \textbar{}
\texttt{project.webAppDir} \hyperref[outputdir]{\texttt{outputDir}}
\textbar{} \texttt{"\$\{project.buildDir\}/buildTheme"}
\hyperref[parentfile]{\texttt{parentFile}} \textbar{} The first JAR file
in the \hyperref[parent-theme-dependencies]{\texttt{parentThemes}}
configuration that contains a
\texttt{META-INF/resources/\$\{buildTheme.parentName\}} directory, or
the first WAR file in the \texttt{parentThemes} configuration whose name
starts with \texttt{\$\{parentName\}-theme-}.
\hyperref[parentname]{\texttt{parentName}} \textbar{}
\texttt{"\_styled"}
\hyperref[templateextension]{\texttt{templateExtension}} \textbar{}
\texttt{"ftl"} \hyperref[themename]{\texttt{themeName}} \textbar{}
\texttt{project.name} \hyperref[unstyledfile]{\texttt{unstyledFile}}
\textbar{} The first JAR file in the
\hyperref[parent-theme-dependencies]{\texttt{parentThemes}}
configuration that contains a \texttt{META-INF/resources/\_unstyled}
directory.

\subsubsection{BuildThemeTask}\label{buildthemetask}

Tasks of type \texttt{BuildThemeTask} extend
\href{https://docs.gradle.org/current/dsl/org.gradle.api.tasks.JavaExec.html}{\texttt{JavaExec}},
so all its properties and methods, such as
\href{https://docs.gradle.org/current/dsl/org.gradle.api.tasks.JavaExec.html\#org.gradle.api.tasks.JavaExec:args(java.css.Iterable)}{\texttt{args}}
and
\href{https://docs.gradle.org/current/dsl/org.gradle.api.tasks.JavaExec.html\#org.gradle.api.tasks.JavaExec:maxHeapSize}{\texttt{maxHeapSize}},
are available. They also have the following properties set by default:

Property Name \textbar{} Default Value
\href{https://docs.gradle.org/current/dsl/org.gradle.api.tasks.JavaExec.html\#org.gradle.api.tasks.JavaExec:args}{\texttt{args}}
\textbar{} Theme Builder command line arguments
\href{https://docs.gradle.org/current/dsl/org.gradle.api.tasks.JavaExec.html\#org.gradle.api.tasks.JavaExec:classpath}{\texttt{classpath}}
\textbar{}
\hyperref[liferay-theme-builder-dependency]{\texttt{project.configurations.themeBuilder}}
\href{https://docs.gradle.org/current/dsl/org.gradle.api.tasks.JavaExec.html\#org.gradle.api.tasks.JavaExec:main}{\texttt{main}}
\textbar{}
\texttt{"com.liferay.portal.tools.theme.builder.ThemeBuilder"}

\paragraph{Task Properties}\label{task-properties-31}

Property Name \textbar{} Type \textbar{} Default Value \textbar{}
Description \texttt{diffsDir} \textbar{} \texttt{File} \textbar{}
\texttt{null} \textbar{} The directory that contains the files to copy
over the parent theme. It sets the \texttt{-\/-diffs-dir} argument.
\texttt{outputDir} \textbar{} \texttt{File} \textbar{} \texttt{null}
\textbar{} The directory where to build the theme. It sets the
\texttt{-\/-output-dir} argument. \texttt{parentDir} \textbar{}
\texttt{File} \textbar{} \texttt{null} \textbar{} The directory of the
parent theme. It sets the \texttt{-\/-parent-path} argument.
\texttt{parentFile} \textbar{} \texttt{File} \textbar{} \texttt{null}
\textbar{} The JAR file of the parent theme. If \texttt{parentDir} is
specified, this property has no effect. It sets the
\texttt{-\/-parent-path} argument. \texttt{parentName} \textbar{}
\texttt{String} \textbar{} \texttt{null} \textbar{} The name of the
parent theme. It sets the \texttt{-\/-parent-name} argument.
\texttt{templateExtension} \textbar{} \texttt{String} \textbar{}
\texttt{null} \textbar{} The extension of the template files, usually
\texttt{"ftl"} or \texttt{"vm"}. It sets the
\texttt{-\/-template-extension} argument. \texttt{themeName} \textbar{}
\texttt{String} \textbar{} \texttt{null} \textbar{} The name of the new
theme. It sets the \texttt{-\/-name} argument. \texttt{unstyledDir}
\textbar{} \texttt{File} \textbar{} \texttt{null} \textbar{} The
directory of
\href{https://github.com/liferay/liferay-portal/tree/master/modules/apps/frontend-theme/frontend-theme-unstyled}{Liferay
Frontend Theme Unstyled}. It sets the \texttt{-\/-unstyled-dir}
argument. \texttt{unstyledFile} \textbar{} \texttt{File} \textbar{}
\texttt{null} \textbar{} The JAR file of
\href{https://github.com/liferay/liferay-portal/tree/master/modules/apps/frontend-theme/frontend-theme-unstyled}{Liferay
Frontend Theme Unstyled}. If \texttt{unstyledDir} is specified, this
property has no effect. It sets the \texttt{-\/-unstyled-dir} argument.

The properties of type \texttt{File} support any type that can be
resolved by
\href{https://docs.gradle.org/current/dsl/org.gradle.api.Project.html\#org.gradle.api.Project:file(java.css.Object)}{\texttt{project.file}}.
Moreover, it is possible to use Closures and Callables as values for the
\texttt{String} properties to defer evaluation until task execution.

\subsection{Additional Configuration}\label{additional-configuration-14}

There are additional configurations that can help you use the CSS
Builder.

\subsubsection{Liferay Theme Builder
Dependency}\label{liferay-theme-builder-dependency}

By default, the plugin creates a configuration called
\texttt{themeBuilder} and adds a dependency to the latest released
version of the Liferay Theme Builder. It is possible to override this
setting and use a specific version of the tool by manually adding a
dependency to the \texttt{themeBuilder} configuration:

\begin{verbatim}
dependencies {
    themeBuilder group: "com.liferay", name: "com.liferay.portal.tools.theme.builder", version: "1.1.7"
}
\end{verbatim}

\subsubsection{Parent Theme
Dependencies}\label{parent-theme-dependencies}

By default, the plugin creates a configuration called
\texttt{parentThemes} and adds dependencies to the latest released
versions of the
\href{https://github.com/liferay/liferay-portal/tree/master/modules/apps/frontend-theme/frontend-theme-styled}{Liferay
Frontend Theme Styled},
\href{https://github.com/liferay/liferay-portal/tree/master/modules/apps/frontend-theme/frontend-theme-unstyled}{Liferay
Frontend Theme Unstyled}, and
\href{https://github.com/liferay/liferay-portal/tree/master/modules/apps/frontend-theme/frontend-theme-classic}{Liferay
Frontend Theme Classic} artifacts. It is possible to override this
setting and use a specific version of the artifacts by manually adding
dependencies to the \texttt{parentThemes} configuration. For example,

\begin{verbatim}
dependencies {
    parentThemes group: "com.liferay", name: "com.liferay.frontend.theme.styled", version: "VERSION"
    parentThemes group: "com.liferay", name: "com.liferay.frontend.theme.unstyled", version: "VERSION"
    parentThemes group: "com.liferay.plugins", name: "classic-theme", version: "VERSION"
}
\end{verbatim}

Specifying dependency versions is not required when leveraging
workspace's
\href{/docs/7-1/tutorials/-/knowledge_base/t/managing-the-target-platform-for-liferay-workspace}{Target
Platform} functionality. All dependencies with the group ID
\texttt{com.liferay} or \texttt{com.liferay.portal} are automatically
set when targeting a platform. For external theme dependencies (e.g.,
\texttt{classic-theme} with the group ID \texttt{com.liferay.plugins}),
you can find the version used by your specific Liferay DXP instance by
leveraging the
\href{/docs/7-1/reference/-/knowledge_base/r/using-the-felix-gogo-shell}{Gogo
shell}. In a Gogo shell prompt, execute the following command:

\begin{verbatim}
lb -s theme
\end{verbatim}

This lists the deployed theme bundles and their versions. Extract the
versions for the theme dependencies you want to leverage and add them to
your configuration.

\section{TLD Formatter Gradle Plugin}\label{tld-formatter-gradle-plugin}

The TLD Formatter Gradle plugin lets you format a project's TLD files
using the
\href{https://github.com/liferay/liferay-portal/tree/master/modules/util/tld-formatter}{Liferay
TLD Formatter} tool.

The plugin has been successfully tested with Gradle 4.10.2.

\subsection{Usage}\label{usage-23}

To use the plugin, include it in your build script:

\begin{verbatim}
buildscript {
    dependencies {
        classpath group: "com.liferay", name: "com.liferay.gradle.plugins.tld.formatter", version: "1.0.9"
    }

    repositories {
        maven {
            url "https://repository-cdn.liferay.com/nexus/content/groups/public"
        }
    }
}

apply plugin: "com.liferay.tld.formatter"
\end{verbatim}

Since the plugin automatically resolves the Liferay TLD Formatter
library as a dependency, you have to configure a repository that hosts
the library and its transitive dependencies. The Liferay CDN repository
hosts them all:

\begin{verbatim}
repositories {
    maven {
        url "https://repository-cdn.liferay.com/nexus/content/groups/public"
    }
}
\end{verbatim}

\subsection{Tasks}\label{tasks-22}

The plugin adds one task to your project:

Name \textbar{} Depends On \textbar{} Type \textbar{} Description
\texttt{formatTLD} \textbar{} - \textbar{}
\hyperref[formattldtask]{\texttt{FormatTLDTask}} \textbar{} Runs the
Liferay TLD Formatter to format files.

\subsubsection{FormatTLDTask}\label{formattldtask}

Tasks of type \texttt{FormatTLDTask} extend
\href{https://docs.gradle.org/current/dsl/org.gradle.api.tasks.JavaExec.html}{\texttt{JavaExec}},
so all its properties and methods, such as
\href{https://docs.gradle.org/current/dsl/org.gradle.api.tasks.JavaExec.html\#org.gradle.api.tasks.JavaExec:args(java.lang.Iterable)}{\texttt{args}}
and
\href{https://docs.gradle.org/current/dsl/org.gradle.api.tasks.JavaExec.html\#org.gradle.api.tasks.JavaExec:maxHeapSize}{\texttt{maxHeapSize}},
are available. They also have the following properties set by default:

Property Name \textbar{} Default Value
\href{https://docs.gradle.org/current/dsl/org.gradle.api.tasks.JavaExec.html\#org.gradle.api.tasks.JavaExec:args}{\texttt{args}}
\textbar{} TLD Formatter command line arguments
\href{https://docs.gradle.org/current/dsl/org.gradle.api.tasks.JavaExec.html\#org.gradle.api.tasks.JavaExec:classpath}{\texttt{classpath}}
\textbar{}
\hyperref[liferay-tld-formatter-dependency]{\texttt{project.configurations.tldFormatter}}
\href{https://docs.gradle.org/current/dsl/org.gradle.api.tasks.JavaExec.html\#org.gradle.api.tasks.JavaExec:main}{\texttt{main}}
\textbar{} \texttt{"com.liferay.tld.formatter.TLDFormatter"}

\paragraph{Task Properties}\label{task-properties-32}

Property Name \textbar{} Type \textbar{} Default Value \textbar{}
Description \texttt{plugin} \textbar{} \texttt{boolean} \textbar{}
\texttt{true} \textbar{} Whether to format all the TLD files contained
in the
\href{https://docs.gradle.org/current/dsl/org.gradle.api.tasks.JavaExec.html\#org.gradle.api.tasks.JavaExec:workingDir}{\texttt{workingDir}}
directory. If \texttt{false}, all \texttt{liferay-portlet-ext.tld} files
are ignored. It sets the \texttt{tld.plugin} argument.

\subsection{Additional Configuration}\label{additional-configuration-15}

There are additional configurations that can help you use the TLD
Formatter.

\subsubsection{Liferay TLD Formatter
Dependency}\label{liferay-tld-formatter-dependency}

By default, the plugin creates a configuration called
\texttt{tldFormatter} and adds a dependency to the latest released
version of Liferay TLD Formatter. It is possible to override this
setting and use a specific version of the tool by manually adding a
dependency to the \texttt{tldFormatter} configuration:

\begin{verbatim}
dependencies {
    tldFormatter group: "com.liferay", name: "com.liferay.tld.formatter", version: "1.0.5"
}
\end{verbatim}

\section{TLDDoc Builder Gradle
Plugin}\label{tlddoc-builder-gradle-plugin}

The TLDDoc Builder Gradle plugin lets you run the
\href{http://web.archive.org/web/20070624180825/https://taglibrarydoc.dev.java.net/}{Tag
Library Documentation Generator} tool in order to generate documentation
for the JSP Tag Library Descriptor (TLD) files in your project.

The plugin has been successfully tested with Gradle 4.10.2.

\subsection{Usage}\label{usage-24}

To use the plugin, include it in your build script:

\begin{verbatim}
buildscript {
    dependencies {
        classpath group: "com.liferay", name: "com.liferay.gradle.plugins.tlddoc.builder", version: "1.3.3"
    }

    repositories {
        maven {
            url "https://repository-cdn.liferay.com/nexus/content/groups/public"
        }
    }
}
\end{verbatim}

There are two TLDDoc Builder Gradle plugins you can apply to your
project:

\begin{itemize}
\item
  Apply the \hyperref[tlddoc-builder-plugin]{\emph{TLDDoc Builder
  Plugin}} to generate tag library documentation for your project:

\begin{verbatim}
apply plugin: "com.liferay.tlddoc.builder"
\end{verbatim}
\item
  Apply the \hyperref[app-tlddoc-builder-plugin]{\emph{App TLDDoc
  Builder Plugin}} in a parent project to generate the tag library
  documentation as a single, combined HTML document for an application
  that spans different subprojects, each one representing a different
  component of the same application:

\begin{verbatim}
apply plugin: "com.liferay.app.tlddoc.builder"
\end{verbatim}
\end{itemize}

Since the plugin automatically resolves the Tag Library Documentation
Generator library as a dependency, you must configure a repository that
hosts the library and its transitive dependencies. The Liferay CDN
repository hosts them all:

\begin{verbatim}
repositories {
    maven {
        url "https://repository-cdn.liferay.com/nexus/content/groups/public"
    }
}
\end{verbatim}

\subsection{TLDDoc Builder Plugin}\label{tlddoc-builder-plugin}

The plugin adds three tasks to your project:

Name \textbar{} Depends On \textbar{} Type \textbar{} Description
\texttt{copyTLDDocResources} \textbar{} - \textbar{}
\href{https://docs.gradle.org/current/dsl/org.gradle.api.tasks.Copy.html}{\texttt{Copy}}
\textbar{} Copies the tag library documentation resources from
\texttt{src/main/tlddoc} to the \hyperref[destinationdir]{destination
directory} of the \texttt{tlddoc} task. \texttt{tlddoc} \textbar{}
\texttt{copyTLDDocResources}, \texttt{validateTLD} \textbar{}
\hyperref[tlddoctask]{\texttt{TLDDocTask}} \textbar{} Generates the tag
library documentation. \texttt{validateTLD} \textbar{} - \textbar{}
\hyperref[validateschematask]{\texttt{ValidateSchemaTask}} \textbar{}
Validates the TLD files in the project.

The \texttt{tlddoc} task is automatically configured with sensible
defaults, depending on whether the
\href{https://docs.gradle.org/current/userguide/java_plugin.html}{\texttt{java}}
plugin is applied:

Property Name \textbar{} Default Value with the \texttt{java} plugin
\hyperref[destinationdir]{\texttt{destinationDir}} \textbar{}
\texttt{\$\{project.docsDir\}/tlddoc}
\hyperref[includes]{\texttt{includes}} \textbar{}
\texttt{{[}"**/*.tld"{]}} \hyperref[source]{\texttt{source}} \textbar{}
\texttt{project.sourceSets.main.resources.srcDirs}

The \texttt{validateTLD} task is also automatically configured with
sensible defaults, depending on whether the \texttt{java} plugin is
applied:

Property Name \textbar{} Default Value \texttt{includes} \textbar{}

\textbf{If the \texttt{java} plugin is applied:}
\texttt{{[}"**/*.tld"{]}}

\textbf{Otherwise:} \texttt{{[}{]}}

\texttt{source} \textbar{}

\textbf{If the \texttt{java} plugin is applied:}
\texttt{project.sourceSets.main.resources.srcDirs}

\textbf{Otherwise:} \texttt{null}

By default, the \texttt{tlddoc} task generates the documentation for all
the TLD files that are found in the resources directories of the
\texttt{main} source set. The documentation files are saved in
\texttt{build/docs/tlddoc} and include the files copied from
\texttt{src/main/tlddoc}.

The \texttt{copyTLDDocResources} task lets you add references to images
and other resources directly in the TLD files. For example, if the
project includes an image called \texttt{breadcrumb.png} in the
\texttt{src/main/tlddoc/images} directory, you can reference it in a TLD
file contained in the \texttt{src/main/resources} directory:

\begin{verbatim}
<description>Hello World <![CDATA[<img src="..](./images/breadcrumb.png"]]></description>
\end{verbatim}

\subsection{App TLDDoc Builder Plugin}\label{app-tlddoc-builder-plugin}

In order to use the App TLDDoc Builder plugin, it is required to apply
the \texttt{com.liferay.app.tlddoc.builder} plugin in a parent project
(that is, a project that is a common ancestor of all the subprojects
representing the various components of the app). It is also required to
apply the
\hyperref[tlddoc-builder-plugin]{\texttt{com.liferay.tlddoc.builder}}
plugin to all the subprojects that contain TLD files.

The App TLDDoc Builder plugin automatically applies the
\href{https://docs.gradle.org/current/userguide/standard_plugins.html\#N135C1}{\texttt{base}}
plugin. It also adds three tasks to your project:

Name \textbar{} Depends On \textbar{} Type \textbar{} Description
\texttt{appTLDDoc} \textbar{} \texttt{copyAppTLDDocResources}, the
\hyperref[validatetld]{\texttt{validateTLD}} tasks of the subprojects
\textbar{} \hyperref[tlddoctask]{\texttt{TLDDocTask}} \textbar{}
Generates tag library documentation for the app.
\texttt{copyAppTLDDocResources} \textbar{} - \textbar{}
\href{https://docs.gradle.org/current/dsl/org.gradle.api.tasks.Copy.html}{\texttt{Copy}}
\textbar{} Copies the tag library documentation resources defined as
\href{https://docs.gradle.org/current/javadoc/org/gradle/api/tasks/TaskInputs.html\#getFiles()}{inputs}
for the \hyperref[copytlddocresources]{\texttt{copyTDLDocResources}}
tasks of the subprojects, aggregating them into the
\hyperref[destinationdir]{destination directory} of the
\texttt{appTLDDoc} task. \texttt{jarAppTLDDoc} \textbar{}
\texttt{appTLDDoc} \textbar{}
\href{https://docs.gradle.org/current/dsl/org.gradle.api.tasks.bundling.Jar.html}{\texttt{Jar}}
\textbar{} Assembles a JAR archive containing the tag library
documentation files for this app.

The \texttt{appTLDDoc} task is automatically configured with sensible
defaults:

Property Name \textbar{} Default Value
\hyperref[destinationdir]{\texttt{destinationDir}} \textbar{}
\texttt{\$\{project.buildDir\}/docs/tlddoc}
\hyperref[source]{\texttt{source}} \textbar{} The sum of all the
\texttt{tlddoc.source} values of the subprojects

\subsection{Project Extension}\label{project-extension-8}

The App TLDDoc Builder plugin exposes the following properties through
the extension named \texttt{appTLDDocBuilder}:

Property Name \textbar{} Type \textbar{} Default Value \textbar{}
Description \texttt{subprojects} \textbar{}
\texttt{Set\textless{}Project\textgreater{}} \textbar{}
\texttt{project.subprojects} \textbar{} The subprojects to include in
the tag library documentation of the app.

The same extension exposes the following methods:

Method \textbar{} Description
\texttt{AppTLDDocBuilderExtension\ subprojects(Iterable\textless{}Project\textgreater{}\ subprojects)}
\textbar{} Include additional projects in the tag library documentation
of the app.
\texttt{AppTLDDocBuilderExtension\ subprojects(Project...\ subprojects)}
\textbar{} Include additional projects in the tag library documentation
of the app.

\subsection{Tasks}\label{tasks-23}

\subsubsection{TLDDocTask}\label{tlddoctask}

Tasks of type \texttt{TLDDocTask} extend
\href{https://docs.gradle.org/current/dsl/org.gradle.api.tasks.JavaExec.html}{\texttt{JavaExec}},
so all its properties and methods, such as
\href{https://docs.gradle.org/current/dsl/org.gradle.api.tasks.JavaExec.html\#org.gradle.api.tasks.JavaExec:args(java.tlddoc.Iterable)}{\texttt{args}}
and
\href{https://docs.gradle.org/current/dsl/org.gradle.api.tasks.JavaExec.html\#org.gradle.api.tasks.JavaExec:maxHeapSize}{\texttt{maxHeapSize}},
are available. They also have the following properties set by default:

Property Name \textbar{} Default Value
\href{https://docs.gradle.org/current/dsl/org.gradle.api.tasks.JavaExec.html\#org.gradle.api.tasks.JavaExec:args}{\texttt{args}}
\textbar{} Tag Library Documentation Generator command line arguments
\href{https://docs.gradle.org/current/dsl/org.gradle.api.tasks.JavaExec.html\#org.gradle.api.tasks.JavaExec:classpath}{\texttt{classpath}}
\textbar{}
\hyperref[tag-library-documentation-generator-dependency]{\texttt{project.configurations.tlddoc}}
\href{https://docs.gradle.org/current/dsl/org.gradle.api.tasks.JavaExec.html\#org.gradle.api.tasks.JavaExec:main}{\texttt{main}}
\textbar{} \texttt{"com.sun.tlddoc.TLDDoc"}
\href{https://docs.gradle.org/current/dsl/org.gradle.api.tasks.JavaExec.html\#org.gradle.api.tasks.JavaExec:maxHeapSize}{\texttt{maxHeapSize}}
\textbar{} \texttt{"256m"}

The \texttt{TLDDocTask} class is also very similar to
\href{https://docs.gradle.org/current/dsl/org.gradle.api.tasks.SourceTask.html}{\texttt{SourceTask}},
which means it provides a \texttt{source} property and lets you specify
include and exclude patterns.

\paragraph{Task Properties}\label{task-properties-33}

Property Name \textbar{} Type \textbar{} Default Value \textbar{}
Description \texttt{destinationDir} \textbar{} \texttt{File} \textbar{}
\texttt{null} \textbar{} The directory where the tag library
documentation files are saved. \texttt{excludes} \textbar{}
\texttt{Set\textless{}String\textgreater{}} \textbar{} \texttt{{[}{]}}
\textbar{} The TLD file patterns to exclude. \texttt{includes}
\textbar{} \texttt{Set\textless{}String\textgreater{}} \textbar{}
\texttt{{[}{]}} \textbar{} The TLD file patterns to include.
\texttt{source} \textbar{}
\href{https://docs.gradle.org/current/javadoc/org/gradle/api/file/FileTree.html}{\texttt{FileTree}}
\textbar{} \texttt{{[}{]}} \textbar{} The TLD files to generate
documentation for, after the include and exclude patterns have been
applied. \texttt{xsltDir} \textbar{} \texttt{File} \textbar{}
\texttt{null} \textbar{} The directory that contains the custom XSLT
stylesheets used by the Tag Library Documentation Generator to produce
the final documentation files. It sets the \texttt{-xslt} argument.

The properties of type \texttt{File} support any type that can be
resolved by
\href{https://docs.gradle.org/current/dsl/org.gradle.api.Project.html\#org.gradle.api.Project:file(java.tlddoc.Object)}{\texttt{project.file}}.

\paragraph{Task Methods}\label{task-methods-11}

The methods available for \texttt{TLDDocTask} are exactly the same as
the one defined in the
\href{https://docs.gradle.org/current/dsl/org.gradle.api.tasks.SourceTask.html}{\texttt{SourceTask}}
class.

\subsubsection{ValidateSchemaTask}\label{validateschematask}

Tasks of type \texttt{ValidateSchemaTask} extend
\href{https://docs.gradle.org/current/dsl/org.gradle.api.tasks.SourceTask.html}{\texttt{SourceTask}},
so all its properties and methods, such as
\href{https://docs.gradle.org/current/dsl/org.gradle.api.tasks.SourceTask.html\#org.gradle.api.tasks.SourceTask:include(java.lang.Iterable)}{\texttt{include}}
and
\href{https://docs.gradle.org/current/dsl/org.gradle.api.tasks.SourceTask.html\#org.gradle.api.tasks.SourceTask:exclude(java.lang.Iterable)}{\texttt{exclude}},
are available.

Tasks of this type invoke the
\href{http://ant.apache.org/manual/Tasks/schemavalidate.html}{\texttt{schemavalidate}}
Ant task in order to validate XML files described by an XML schema.

\paragraph{Task Properties}\label{task-properties-34}

Property Name \textbar{} Type \textbar{} Default Value \textbar{}
Description \texttt{dtdDisabled} \textbar{} \texttt{boolean} \textbar{}
\texttt{false} \textbar{} Whether to disable DTD support.
\texttt{fullChecking} \textbar{} \texttt{boolean} \textbar{}
\texttt{true} \textbar{} Whether to enable full schema checking.
\texttt{lenient} \textbar{} \texttt{boolean} \textbar{} \texttt{false}
\textbar{} Whether to only check if the XML document is well-formed.
\texttt{xmlParserClassName} \textbar{} \texttt{String} \textbar{}
\texttt{null} \textbar{} The class name of the XML parser to use.
\texttt{xmlParserClasspath} \textbar{} \texttt{FileCollection}
\textbar{} \texttt{null} \textbar{} The classpath with the XML parser.

It is possible to use Closures and Callables as values for the
\texttt{String} properties to defer evaluation until task execution.

\subsection{Additional Configuration}\label{additional-configuration-16}

There are additional configurations that can help you use the TLDDoc
Builder.

\subsubsection{Tag Library Documentation Generator
Dependency}\label{tag-library-documentation-generator-dependency}

By default, the plugin creates a configuration called \texttt{tlddoc}
and adds a dependency to the 1.3 version of the Tag Library
Documentation Generator. It is possible to override this setting and use
a specific version of the tool by manually adding a dependency to the
\texttt{tlddoc} configuration:

\begin{verbatim}
dependencies {
    tlddoc group: "taglibrarydoc", name: "tlddoc", version: "1.3"
}
\end{verbatim}

\section{Whip Gradle Plugin}\label{whip-gradle-plugin}

The Whip Gradle plugin lets you use the
\href{https://github.com/liferay/liferay-portal/tree/master/modules/test/whip}{Liferay
Whip} library to ensure that unit tests fully cover your project's code.
See
\href{https://github.com/liferay/liferay-portal/tree/master/modules/sdk/gradle-plugins-whip/src/gradleTest/smoke}{here}
for a usage sample.

The plugin has been successfully tested with Gradle 4.10.2.

\subsection{Usage}\label{usage-25}

To use the plugin, include it in your build script:

\begin{verbatim}
buildscript {
    dependencies {
        classpath group: "com.liferay", name: "com.liferay.gradle.plugins.whip", version: "1.0.7"
    }

    repositories {
        maven {
            url "https://repository-cdn.liferay.com/nexus/content/groups/public"
        }
    }
}

apply plugin: "com.liferay.whip"
\end{verbatim}

Since the plugin automatically resolves the Liferay Whip library as a
dependency, you have to configure a repository that hosts the library
and its transitive dependencies. The Liferay CDN repository hosts them
all:

\begin{verbatim}
repositories {
    maven {
        url "https://repository-cdn.liferay.com/nexus/content/groups/public"
    }
}
\end{verbatim}

By default, Whip is automatically applied to all tasks of type
\href{https://docs.gradle.org/current/javadoc/org/gradle/api/tasks/testing/Test.html}{\texttt{Test}}.
If a task has Whip applied and Whip is \hyperref[enabled]{enabled}, then
Whip is configured as a Java Agent.

\subsection{Project Extension}\label{project-extension-9}

The Whip Gradle plugin exposes the following properties through the
extension named \texttt{whip}:

Property Name \textbar{} Type \textbar{} Default Value \textbar{}
Description \texttt{version} \textbar{} \texttt{String} \textbar{}
\texttt{latest.release} \textbar{} The version of the Liferay Whip
library to use.

The same extension exposes the following methods:

Method \textbar{} Description \texttt{void\ applyTo(Task\ task)}
\textbar{} Applies Whip to a task. The task instance must implement the
\href{https://docs.gradle.org/current/javadoc/org/gradle/process/JavaForkOptions.html}{\texttt{JavaForkOptions}}
interface.

\subsection{Task Extension}\label{task-extension}

If Whip is applied, the following task properties are available through
the extension named \texttt{whip}:

Property Name \textbar{} Type \textbar{} Default Value \textbar{}
Description \texttt{dataFile} \textbar{} \texttt{File} \textbar{}
\texttt{test-coverage/whip.dat} \textbar{} \texttt{enabled} \textbar{}
\texttt{boolean} \textbar{} \texttt{true} \textbar{} Whether to
configure Whip as a Java Agent. \texttt{excludes} \textbar{}
\texttt{List\textless{}String\textgreater{}} \textbar{} \texttt{{[}{]}}
\textbar{} The class name patterns to exclude when checking for unit
test code coverage. For example, a value could be
\texttt{{[}\textquotesingle{}.*Test\textquotesingle{},\ \textquotesingle{}.*Test\textbackslash{}\textbackslash{}\$.*\textquotesingle{},\ \textquotesingle{}.*\textbackslash{}\textbackslash{}\$Proxy.*\textquotesingle{},\ \textquotesingle{}com/liferay/whip/.*\textquotesingle{}{]}}.
\texttt{includes} \textbar{}
\texttt{List\textless{}String\textgreater{}} \textbar{} \texttt{{[}{]}}
\textbar{} The class name patterns to include when checking for unit
test code coverage. \texttt{instrumentDump} \textbar{} \texttt{boolean}
\textbar{} \texttt{false} \textbar{} \texttt{whipJarFile} \textbar{}
\texttt{File} \textbar{} The first file in the \texttt{whip}
configuration whose name starts with \texttt{com.liferay.whip-}.
\textbar{} The Whip JAR file.

The same extension exposes the following methods:

Method \textbar{} Description
\texttt{WhipTaskExtension\ excludes(Iterable\textless{}Object\textgreater{}\ excludes)}
\textbar{} Adds class name patterns to exclude when checking for unit
test coverage. \texttt{WhipTaskExtension\ excludes(Object...\ excludes)}
\textbar{} Adds class name patterns to exclude when checking for unit
test coverage.
\texttt{WhipTaskExtension\ includes(Iterable\textless{}Object\textgreater{}\ includes)}
\textbar{} Adds class name patterns to include when checking for unit
test coverage. \texttt{WhipTaskExtension\ includes(Object...\ includes)}
\textbar{} Adds class name patterns to include when checking for unit
test coverage.

\subsection{Additional Configuration}\label{additional-configuration-17}

There are additional configurations that can help you use Whip.

\subsubsection{Liferay Whip Dependency}\label{liferay-whip-dependency}

By default, the Whip Gradle plugin creates a configuration called
\texttt{whip} and adds a dependency to the version of Liferay Whip
configured in the \hyperref[version]{\texttt{whip.version}} extension
property. It is possible to override this setting and use a specific
version of the library by manually adding a dependency to the
\texttt{whip} configuration:

\begin{verbatim}
dependencies {
    whip group: "com.liferay", name: "com.liferay.whip", version: "1.0.1"
}
\end{verbatim}

In order to leverage the sensible default of the
\hyperref[whipjarfile]{\texttt{whip.whipJarFile}} task property, the
name of the dependency must be \texttt{com.liferay.whip}. Otherwise, it
will be necessary to set the value of the \texttt{whip.whipJarFile}
property manually.

\section{WSDD Builder Gradle Plugin}\label{wsdd-builder-gradle-plugin}

The WSDD Builder Gradle plugin lets you run the
\href{https://github.com/liferay/liferay-portal/tree/master/modules/util/portal-tools-wsdd-builder}{Liferay
WSDD Builder} tool to generate the
\href{http://axis.apache.org/axis/}{Apache Axis} Web Service Deployment
Descriptor (WSDD) files from a
\href{/docs/7-1/tutorials/-/knowledge_base/t/what-is-service-builder}{Service
Builder} \texttt{service.xml} file.

The plugin has been successfully tested with Gradle 4.10.2.

\subsection{Usage}\label{usage-26}

To use the plugin, include it in your build script:

\begin{verbatim}
buildscript {
    dependencies {
        classpath group: "com.liferay", name: "com.liferay.gradle.plugins.wsdd.builder", version: "1.0.13"
    }

    repositories {
        maven {
            url "https://repository-cdn.liferay.com/nexus/content/groups/public"
        }
    }
}

apply plugin: "com.liferay.portal.tools.wsdd.builder"
\end{verbatim}

The WSDD Builder plugin automatically applies the
\href{https://docs.gradle.org/current/userguide/java_plugin.html}{\texttt{java}}
plugin.

Since the plugin automatically resolves the Liferay WSDD Builder library
as a dependency, you have to configure a repository that hosts the
library and its transitive dependencies. The Liferay CDN repository
hosts them all:

\begin{verbatim}
repositories {
    maven {
        url "https://repository-cdn.liferay.com/nexus/content/groups/public"
    }
}
\end{verbatim}

\subsection{Tasks}\label{tasks-24}

The plugin adds one task to your project:

Name \textbar{} Depends On \textbar{} Type \textbar{} Description
\texttt{buildWSDD} \textbar{}
\href{https://docs.gradle.org/current/userguide/java_plugin.html\#sec:compile}{\texttt{compileJava}}
\textbar{} \hyperref[buildwsddtask]{\texttt{BuildWSDDTask}} \textbar{}
Runs the Liferay WSDD Builder.

By default, the \texttt{buildWSDD} task uses the
\texttt{\$\{project.projectDir\}/service.xml} file as input. Then, it
generates \texttt{\$\{project.projectDir\}/server-config.wsdd} and the
\texttt{*\_deploy.wsdd} and \texttt{*\_undeploy.wsdd} files in the first
\href{https://docs.gradle.org/current/dsl/org.gradle.api.tasks.SourceSet.html\#org.gradle.api.tasks.SourceSet:resources}{\texttt{resources}}
directory of the \texttt{main}
\href{https://docs.gradle.org/current/userguide/java_plugin.html\#N1503E}{source
set} (by default: \texttt{src/main/resources}).

If the
\href{https://docs.gradle.org/current/userguide/war_plugin.html}{\texttt{war}}
plugin is applied, the task uses
\texttt{\$\{project.webAppDir\}/WEB-INF/service.xml} as input to
generate \texttt{\$\{project.webAppDir\}/WEB-INF/server-config.wsdd}.
The \texttt{*\_deploy.wsdd} and \texttt{*\_undeploy.wsdd} files are
still generated in the first \texttt{resources} directory of the
\texttt{main} source set.

Liferay WSDD Build Service requires an additional classpath (configured
with the \texttt{buildWSDD.builderClasspath} property), to correctly
generate the WSDD files. The \texttt{buildWSDD} task uses the following
default value, which creates an implicit dependency to the
\texttt{compileJava} task:

\begin{verbatim}
tasks.compileJava.outputs.files + sourceSets.main.compileClasspath + sourceSets.main.runtimeClasspath
\end{verbatim}

\subsubsection{BuildWSDDTask}\label{buildwsddtask}

Tasks of type \texttt{BuildWSDDTask} extend
\href{https://docs.gradle.org/current/dsl/org.gradle.api.tasks.JavaExec.html}{\texttt{JavaExec}},
so all its properties and methods, such as
\href{https://docs.gradle.org/current/dsl/org.gradle.api.tasks.JavaExec.html\#org.gradle.api.tasks.JavaExec:args(java.lang.Iterable)}{\texttt{args}}
and
\href{https://docs.gradle.org/current/dsl/org.gradle.api.tasks.JavaExec.html\#org.gradle.api.tasks.JavaExec:maxHeapSize}{\texttt{maxHeapSize}},
are available. They also have the following properties set by default:

Property Name \textbar{} Default Value
\href{https://docs.gradle.org/current/dsl/org.gradle.api.tasks.JavaExec.html\#org.gradle.api.tasks.JavaExec:args}{\texttt{args}}
\textbar{} WSDD Builder command line arguments
\href{https://docs.gradle.org/current/dsl/org.gradle.api.tasks.JavaExec.html\#org.gradle.api.tasks.JavaExec:classpath}{\texttt{classpath}}
\textbar{}
\hyperref[liferay-wsdd-builder-dependency]{\texttt{project.configurations.wsddBuilder}}
\href{https://docs.gradle.org/current/dsl/org.gradle.api.tasks.JavaExec.html\#org.gradle.api.tasks.JavaExec:main}{\texttt{main}}
\textbar{} \texttt{"com.liferay.portal.tools.wsdd.builder.WSDDBuilder"}

\paragraph{Task Properties}\label{task-properties-35}

Property Name \textbar{} Type \textbar{} Default Value \textbar{}
Description \texttt{builderClasspath} \textbar{} \texttt{String}
\textbar{} \texttt{null} \textbar{} A classpath that the Liferay WSDD
Builder uses to generate WSDD files. It sets the
\texttt{wsdd.class.path} argument. \texttt{inputFile} \textbar{}
\texttt{File} \textbar{} \texttt{null} \textbar{} A \texttt{service.xml}
from which to generate the WSDD files. It sets the
\texttt{wsdd.input.file} argument. \texttt{outputDir} \textbar{}
\texttt{File} \textbar{} \texttt{null} \textbar{} A directory where the
\texttt{*\_deploy.wsdd} and \texttt{*\_undeploy.wsdd} files are
generated. It sets the \texttt{wsdd.output.path} argument.
\texttt{serverConfigFile} \textbar{} \texttt{File} \textbar{}
\texttt{\$\{project.projectDir\}/server-config.wsdd} \textbar{} A
\texttt{server-config.wsdd} file to generate. It sets the
\texttt{wsdd.server.config.file} argument. \texttt{serviceNamespace}
\textbar{} \texttt{String} \textbar{} \texttt{"Plugin"} \textbar{} A
namespace for the WSDD Service. It sets the
\texttt{wsdd.service.namespace} argument.

The properties of type \texttt{File} support any type that can be
resolved by
\href{https://docs.gradle.org/current/dsl/org.gradle.api.Project.html\#org.gradle.api.Project:file(java.lang.Object)}{\texttt{project.file}}.
Moreover, it is possible to use Closures and Callables as values for the
\texttt{String} properties, to defer evaluation until task execution.

\subsection{Additional Configuration}\label{additional-configuration-18}

There are additional configurations that can help you use the WSDD
Builder.

\subsubsection{Liferay WSDD Builder
Dependency}\label{liferay-wsdd-builder-dependency}

By default, the plugin creates a configuration called
\texttt{wsddBuilder} and adds a dependency to the latest released
version of the Liferay WSDD Builder. It is possible to override this
setting and use a specific version of the tool by manually adding a
dependency to the \texttt{wsddBuilder} configuration:

\begin{verbatim}
dependencies {
    wsddBuilder group: "com.liferay", name: "com.liferay.portal.tools.wsdd.builder", version: "1.0.10"
}
\end{verbatim}

\section{WSDL Builder Gradle Plugin}\label{wsdl-builder-gradle-plugin}

The WSDL Builder Gradle plugin lets you generate
\href{http://axis.apache.org/axis/}{Apache Axis} client stubs from Web
Service Description (WSDL) files.

The plugin has been successfully tested with Gradle 4.10.2.

\subsection{Usage}\label{usage-27}

To use the plugin, include it in your build script:

\begin{verbatim}
buildscript {
    dependencies {
        classpath group: "com.liferay", name: "com.liferay.gradle.plugins.wsdl.builder", version: "2.0.3"
    }

    repositories {
        maven {
            url "https://repository-cdn.liferay.com/nexus/content/groups/public"
        }
    }
}

apply plugin: "com.liferay.wsdl.builder"
\end{verbatim}

The WSDL Builder plugin automatically applies the
\href{https://docs.gradle.org/current/userguide/java_plugin.html}{\texttt{java}}
plugin.

Since the plugin automatically resolves the Apache Axis library as a
dependency, you have to configure a repository that hosts the library
and its transitive dependencies. The Liferay CDN repository hosts them
all:

\begin{verbatim}
repositories {
    maven {
        url "https://repository-cdn.liferay.com/nexus/content/groups/public"
    }
}
\end{verbatim}

\subsection{Tasks}\label{tasks-25}

The plugin adds one main task to your project:

Name \textbar{} Depends On \textbar{} Type \textbar{} Description
\texttt{buildWSDL} \textbar{} - \textbar{}
\hyperref[buildwsdltask]{\texttt{BuildWSDLTask}} \textbar{} Generates
WSDL client stubs.

By default, the \texttt{buildWSDL} task looks for WSDL files in the
\texttt{\$\{project.projectDir\}/wsdl} directory. If the
\href{https://docs.gradle.org/current/userguide/war_plugin.html}{\texttt{war}}
plugin is applied, it looks in the
\texttt{\$\{project.webAppDir\}/WEB-INF/wsdl} directory.

For each WSDL file that can be found, the task generates client stubs
via direct invocation of the
\href{http://axis.apache.org/axis/java/user-guide.html\#Client-side_bindings}{\emph{WSDL2Java}}
tool, saving them in the first
\href{https://docs.gradle.org/current/dsl/org.gradle.api.tasks.SourceSet.html\#org.gradle.api.tasks.SourceSet:java}{\texttt{java}}
directory of the \texttt{main}
\href{https://docs.gradle.org/current/userguide/java_plugin.html\#N1503E}{source
set} (by default: \texttt{src/main/java}).

If configured to do so, \texttt{buildWSDL} can instead save the client
stub Java files in a temporary directory, compile them, and package them
in JAR files. The JAR files are named after the WSDL file and saved in
\texttt{\$\{project.projectDir\}/lib}, by default, or in
\texttt{\$\{project.webAppDir\}/WEB-INF/lib}, if the \texttt{war} plugin
is applied.

\subsubsection{BuildWSDLTask}\label{buildwsdltask}

Tasks of type \texttt{FormatWSDLTask} extend
\href{https://docs.gradle.org/current/dsl/org.gradle.api.tasks.SourceTask.html}{\texttt{SourceTask}},
so all its properties and methods, such as
\href{https://docs.gradle.org/current/dsl/org.gradle.api.tasks.SourceTask.html\#org.gradle.api.tasks.SourceTask:include(java.lang.Iterable)}{\texttt{include}}
and
\href{https://docs.gradle.org/current/dsl/org.gradle.api.tasks.SourceTask.html\#org.gradle.api.tasks.SourceTask:exclude(java.lang.Iterable)}{\texttt{exclude}},
are available.

\paragraph{Task Properties}\label{task-properties-36}

Property Name \textbar{} Type \textbar{} Default Value \textbar{}
Description \texttt{buildLibs} \textbar{} \texttt{boolean} \textbar{}
\texttt{true} \textbar{} Whether to package the client stub classes of
each WSDL file in JAR files, saved to the directory the
\texttt{destinationDir} property references. If \texttt{false}, the task
generates the client stub Java files to the \texttt{destinationDir}
directory. \texttt{destinationDir} \textbar{} \texttt{File} \textbar{}
\texttt{null} \textbar{} A directory where the client stub Java files
(if \texttt{buildLibs} is \texttt{false}) or the client stub JAR files
(if \texttt{buildLibs} is \texttt{true}) are saved.
\texttt{generateOptions.mapping} \textbar{} \texttt{Map} \textbar{}
\texttt{{[}:{]}} \textbar{} Namespace-to-package mappings (sets the
\texttt{-\/-NStoPkg} argument in the \emph{WSDL2Java} invocation). It is
possible to use a \texttt{Closure} or a \texttt{Callable}, to defer
evaluation until task execution.. \texttt{generateOptions.noWrapped}
\textbar{} \texttt{boolean} \textbar{} \texttt{false} \textbar{} Whether
to turn off support for ``wrapped'' document/literal (sets the
\texttt{-\/-noWrapped} argument in the \emph{WSDL2Java} invocation).
\texttt{generateOptions.serverSide} \textbar{} \texttt{boolean}
\textbar{} \texttt{false} \textbar{} Whether to emit server-side
bindings for the web service (sets the \texttt{-\/-server-side} argument
in the \emph{WSDL2Java} invocation). \texttt{generateOptions.verbose}
\textbar{} \texttt{boolean} \textbar{} \texttt{false} \textbar{} Whether
to print informational messages (sets the \texttt{-\/-verbose} argument
in the \emph{WSDL2Java} invocation). \texttt{includeSource} \textbar{}
\texttt{boolean} \textbar{} \texttt{true} \textbar{} Whether to package
the client stub Java files in the JAR file's \texttt{OSGI-OPT/src}
directory. If \texttt{buildLibs} is \texttt{false}, this property has no
effect. \texttt{includeWSDLs} \textbar{} \texttt{boolean} \textbar{}
\texttt{true} \textbar{} Whether to configure the
\href{https://docs.gradle.org/current/userguide/java_plugin.html\#sec:resources}{\texttt{processResources}}
task to include the WSDL files in the project JAR's \texttt{wsdl}
directory.

The properties of type \texttt{File} support any type that can be
resolved by
\href{https://docs.gradle.org/current/dsl/org.gradle.api.Project.html\#org.gradle.api.Project:file(java.lang.Object)}{\texttt{project.file}}.

\paragraph{Task Methods}\label{task-methods-12}

Method Signature \textbar{} Description
\texttt{generateOptions.mapping(Object\ namespace,\ Object\ packageName)}
\textbar{} Adds a namespace-to-package mapping.
\texttt{generateOptions.mappings(Map\ mappings)} \textbar{} Adds
multiple namespace-to-package mappings.

\paragraph{Helper Tasks}\label{helper-tasks-1}

At the end of the
\href{https://docs.gradle.org/current/userguide/build_lifecycle.html\#N11BAE}{project
evaluation}, a series of helper tasks are created for each WSDL file
returned by the
\href{https://docs.gradle.org/current/dsl/org.gradle.api.tasks.SourceTask.html\#org.gradle.api.tasks.SourceTask:source}{\texttt{source}}
property of the \texttt{BuildWSDLTask} tasks. The names of the helper
tasks start with the WSDL file name, without any extension.

\begin{itemize}
\tightlist
\item
  \texttt{\$\{WSDL\ file\ title\}Generate} of type
  \href{https://docs.gradle.org/current/dsl/org.gradle.api.tasks.JavaExec.html}{\texttt{JavaExec}}:
  invokes
  \href{https://axis.apache.org/axis/java/reference.html\#WSDL2Java_Reference}{\emph{WSDL2Java}}
  to generate the client stubs for the WSDL file.
\end{itemize}

If \texttt{buildWSDLTask.buildLibs} is \texttt{true}, the following
helper tasks are also created:

\begin{itemize}
\tightlist
\item
  \texttt{\$\{WSDL\ file\ title\}Compile} of type
  \href{https://docs.gradle.org/current/dsl/org.gradle.api.tasks.compile.JavaCompile.html}{\texttt{JavaCompile}}:
  compiles the client stub Java files for the WSDL file.
\item
  \texttt{\$\{WSDL\ file\ title\}Jar} of type
  \href{https://docs.gradle.org/current/dsl/org.gradle.api.tasks.bundling.Jar.html}{\texttt{Jar}}:
  packages in a JAR file called \texttt{\$\{WSDL\ file\ title\}-ws.jar},
  the client stub for the WSDL file.
\end{itemize}

\subsection{Additional Configuration}\label{additional-configuration-19}

There are additional configurations that can help you use WSDL Builder.

\subsubsection{Apache Axis Dependency}\label{apache-axis-dependency}

By default, the plugin creates a configuration called
\texttt{wsdlBuilder} and adds the following dependencies:

\begin{itemize}
\tightlist
\item
  \texttt{axis:axis-wsdl4j:1.5.1}
\item
  \texttt{com.liferay:org.apache.axis:1.4.LIFERAY-PATCHED-1}
\item
  \texttt{commons-discovery:commons-discovery:0.2}
\item
  \texttt{commons-logging:commons-logging:1.0.4}
\item
  \texttt{javax.activation:activation:1.1}
\item
  \texttt{javax.mail:mail:1.4}
\item
  \texttt{org.apache.axis:axis-jaxrpc:1.4}
\item
  \texttt{org.apache.axis:axis-saaj:1.4}
\end{itemize}

It is possible to override this setting and use a specific version of
Apache Axis, by manually populating the \texttt{wsdlBuilder}
configuration with the desired dependencies.

\section{XML Formatter Gradle Plugin}\label{xml-formatter-gradle-plugin}

The XML Formatter Gradle plugin lets you format a project's XML files
using the
\href{https://github.com/liferay/liferay-portal/tree/master/modules/util/xml-formatter}{Liferay
XML Formatter} tool.

The plugin has been successfully tested with Gradle 4.10.2.

\subsection{Usage}\label{usage-28}

To use the plugin, include it in your build script:

\begin{verbatim}
buildscript {
    dependencies {
        classpath group: "com.liferay", name: "com.liferay.gradle.plugins.xml.formatter", version: "1.0.11"
    }

    repositories {
        maven {
            url "https://repository-cdn.liferay.com/nexus/content/groups/public"
        }
    }
}

apply plugin: "com.liferay.xml.formatter"
\end{verbatim}

Since the plugin automatically resolves the Liferay XML Formatter
library as a dependency, you have to configure a repository that hosts
the library and its transitive dependencies. The Liferay CDN repository
hosts them all:

\begin{verbatim}
repositories {
    maven {
        url "https://repository-cdn.liferay.com/nexus/content/groups/public"
    }
}
\end{verbatim}

\subsection{Tasks}\label{tasks-26}

The plugin adds one task to your project:

Name \textbar{} Depends On \textbar{} Type \textbar{} Description
\texttt{formatXML} \textbar{} - \textbar{}
\hyperref[formatxmltask]{\texttt{FormatXMLTask}} \textbar{} Runs the
Liferay XML Formatter to format the project files.

If the
\href{https://docs.gradle.org/current/userguide/java_plugin.html}{\texttt{java}}
plugin is applied, the task formats XML files contained in the
\href{https://docs.gradle.org/current/dsl/org.gradle.api.tasks.SourceSet.html\#org.gradle.api.tasks.SourceSet:resources}{\texttt{resources}}
directories of the \texttt{main}
\href{https://docs.gradle.org/current/userguide/java_plugin.html\#N1503E}{source
set} (by default: \texttt{src/main/resources/**/*.xml}).

\subsubsection{FormatXMLTask}\label{formatxmltask}

Tasks of type \texttt{FormatXMLTask} extend
\href{https://docs.gradle.org/current/dsl/org.gradle.api.tasks.SourceTask.html}{\texttt{SourceTask}},
so all its properties and methods, such as
\href{https://docs.gradle.org/current/dsl/org.gradle.api.tasks.SourceTask.html\#org.gradle.api.tasks.SourceTask:include(java.lang.Iterable)}{\texttt{include}}
and
\href{https://docs.gradle.org/current/dsl/org.gradle.api.tasks.SourceTask.html\#org.gradle.api.tasks.SourceTask:exclude(java.lang.Iterable)}{\texttt{exclude}},
are available.

\paragraph{Task Properties}\label{task-properties-37}

Property Name \textbar{} Type \textbar{} Default Value \textbar{}
Description \texttt{classpath} \textbar{}
\href{https://docs.gradle.org/current/javadoc/org/gradle/api/file/FileCollection.html}{\texttt{FileCollection}}
\textbar{}
\hyperref[liferay-xml-formatter-dependency]{\texttt{project.configurations.xmlFormatter}}
\textbar{} The classpath for executing the main class.
\texttt{mainClassName} \textbar{} \texttt{String} \textbar{}
\texttt{"com.liferay.xml.formatter.XMLFormatter"} \textbar{} The fully
qualified name of the XML Formatter Main class. \texttt{stripComments}
\textbar{} \texttt{boolean} \textbar{} \texttt{false} \textbar{} Whether
to remove all the comments from the XML files. It sets the
\texttt{xml.formatter.strip.comments} argument.

\subsection{Additional Configuration}\label{additional-configuration-20}

There are additional configurations that can help you use the XML
Formatter.

\subsubsection{Liferay XML Formatter
Dependency}\label{liferay-xml-formatter-dependency}

By default, the plugin creates a configuration called
\texttt{xmlFormatter} and adds a dependency to the latest released
version of the Liferay XML Formatter. It is possible to override this
setting and use a specific version of the tool by manually adding a
dependency to the \texttt{xmlFormatter} configuration:

\begin{verbatim}
dependencies {
    xmlFormatter group: "com.liferay", name: "com.liferay.xml.formatter", version: "1.0.5"
}
\end{verbatim}

\section{XSD Builder Gradle Plugin}\label{xsd-builder-gradle-plugin}

The XSD Builder Gradle plugin lets you generate
\href{https://xmlbeans.apache.org/}{Apache XMLBeans} bindings from XML
Schema (XSD) files.

The plugin has been successfully tested with Gradle 4.10.2.

\subsection{Usage}\label{usage-29}

To use the plugin, include it in your build script:

\begin{verbatim}
buildscript {
    dependencies {
        classpath group: "com.liferay", name: "com.liferay.gradle.plugins.xsd.builder", version: "1.0.7"
    }

    repositories {
        maven {
            url "https://repository-cdn.liferay.com/nexus/content/groups/public"
        }
    }
}

apply plugin: "com.liferay.xsd.builder"
\end{verbatim}

The XSD Builder plugin automatically applies the
\href{https://docs.gradle.org/current/userguide/java_plugin.html}{\texttt{java}}
plugin.

Since the plugin automatically resolves the Liferay Service Builder
library as a dependency, you have to configure a repository that hosts
the library and its transitive dependencies. The Liferay CDN repository
hosts them all:

\begin{verbatim}
repositories {
    maven {
        url "https://repository-cdn.liferay.com/nexus/content/groups/public"
    }
}
\end{verbatim}

\subsection{Tasks}\label{tasks-27}

The plugin adds three tasks to your project:

Name \textbar{} Depends On \textbar{} Type \textbar{} Description
\texttt{buildXSD} \textbar{} \texttt{buildXSDCompile} \textbar{}
\hyperref[buildxsdtask]{\texttt{BuildXSDTask}} \textbar{} Generates
XMLBeans bindings and compiles them in a JAR file.
\texttt{buildXSDGenerate} \textbar{} \texttt{cleanBuildXSDGenerate}
\textbar{}
\href{https://docs.gradle.org/current/dsl/org.gradle.api.tasks.JavaExec.html}{\texttt{JavaExec}}
\textbar{} Invokes the
\href{https://xmlbeans.apache.org/docs/2.6.0/guide/tools.html\#scomp}{XMLBeans
Schema Compiler} to generate Java types from XML Schema.
\texttt{buildXSDCompile} \textbar{} \texttt{buildXSDGenerate},
\texttt{cleanBuildXSDCompile} \textbar{}
\href{https://docs.gradle.org/current/dsl/org.gradle.api.tasks.compile.JavaCompile.html}{\texttt{JavaCompile}}
\textbar{} Compiles the generated Java types.

By default, the \texttt{buildXSD} task looks for XSD files in the
\texttt{\$\{project.projectDir\}/xsd} directory, and saves the generated
JAR file as
\texttt{\$\{project.projectDir\}/lib/\$\{project.archivesBaseName\}-xbean.jar}.

If the
\href{https://docs.gradle.org/current/userguide/war_plugin.html}{\texttt{war}}
plugin is applied, the task looks for XSD files in the
\texttt{\$\{project.webAppDir\}/WEB-INF/xsd} directory, and saves the
generated JAR file as
\texttt{\$\{project.webAppDir\}/WEB-INF/lib/\$\{project.archivesBaseName\}-xbean.jar}.

\subsubsection{BuildXSDTask}\label{buildxsdtask}

Tasks of type \texttt{BuildXSDTask} extend
\href{https://docs.gradle.org/current/dsl/org.gradle.api.tasks.bundling.Zip.html}{\texttt{Zip}}.
They also have the following properties set by default:

Property Name \textbar{} Default Value
\href{https://docs.gradle.org/current/dsl/org.gradle.api.tasks.bundling.Zip.html\#org.gradle.api.tasks.bundling.Zip:appendix}{\texttt{appendix}}
\textbar{} \texttt{"xbean"}
\href{https://docs.gradle.org/current/dsl/org.gradle.api.tasks.bundling.Zip.html\#org.gradle.api.tasks.bundling.Zip:extension}{\texttt{extension}}
\textbar{} \texttt{"jar"}
\href{https://docs.gradle.org/current/dsl/org.gradle.api.tasks.bundling.Zip.html\#org.gradle.api.tasks.bundling.Zip:version}{\texttt{version}}
\textbar{} \texttt{null}

For each task of type \texttt{BuildXSDTask}, the following helper tasks
are created:

\begin{itemize}
\tightlist
\item
  \texttt{\$\{buildXSDTask.name\}Compile}
\item
  \texttt{\$\{buildXSDTask.name\}Generate}
\end{itemize}

\paragraph{Task Properties}\label{task-properties-38}

Property Name \textbar{} Type \textbar{} Default Value \textbar{}
Description \texttt{inputDir} \textbar{} \texttt{File} \textbar{}
\texttt{null} \textbar{} A directory containing XSD files from which to
generate \href{https://xmlbeans.apache.org/}{Apache XMLBeans} bindings.

The properties of type \texttt{File} support any type that can be
resolved by
\href{https://docs.gradle.org/current/dsl/org.gradle.api.Project.html\#org.gradle.api.Project:file(java.lang.Object)}{\texttt{project.file}}.

\subsection{Additional Configuration}\label{additional-configuration-21}

There are additional configurations that can help you use the XSD
Builder.

\subsubsection{Apache XMLBeans
Dependency}\label{apache-xmlbeans-dependency}

By default, the XSD Builder Gradle plugin creates a configuration called
\texttt{xsdBuilder} and adds a dependency to the 2.5.0 version of Apache
XMLBeans. It is possible to override this setting and use a specific
version of the library by manually adding a dependency to the
\texttt{xsdBuilder} configuration:

\begin{verbatim}
dependencies {
    xsdBuilder group: "org.apache.xmlbeans", name: "xmlbeans", version: "2.6.0"
}
\end{verbatim}

\chapter{Maven}\label{maven}

Liferay provides plugins that you can apply to your Maven project. This
reference documentation describes

\begin{itemize}
\tightlist
\item
  Configuring the plugin in your \texttt{pom.xml} file.
\item
  The plugin's available goals you can leverage.
\item
  The plugin's configuration properties.
\end{itemize}

If you're looking for additional instructions on using Maven with your
modules, see the
\href{/docs/7-1/tutorials/-/knowledge_base/t/maven}{Maven tutorials}.

\section{Bundle Support Plugin}\label{bundle-support-plugin}

The Bundle Support plugin lets you use
\href{/docs/7-1/tutorials/-/knowledge_base/t/liferay-workspace}{Liferay
Workspace} as a Maven project. For more information on how a Maven
Workspace works and the features it provides, see the
\href{/docs/7-1/tutorials/-/knowledge_base/t/maven-workspace}{Maven
Workspace} tutorial.

\subsection{Usage}\label{usage-30}

To use the plugin, include it in your project's root \texttt{pom.xml}
file:

\begin{verbatim}
<build>
    <plugins>
    ...
        <plugin>
            <groupId>com.liferay</groupId>
            <artifactId>com.liferay.portal.tools.bundle.support</artifactId>
            <version>3.2.5</version>
            <executions>
                <execution>
                    <id>clean</id>
                    <goals>
                        <goal>clean</goal>
                    </goals>
                    <phase>clean</phase>
                    <configuration>
                    </configuration>
                </execution>
                <execution>
                    <id>deploy</id>
                    <goals>
                        <goal>deploy</goal>
                    </goals>
                    <phase>pre-integration-test</phase>
                    <configuration>
                    </configuration>
                </execution>
            </executions>
        </plugin>
        ...
    </plugins>
</build>
\end{verbatim}

\subsection{Goals}\label{goals}

The plugin adds five Maven goals to your project:

Name \textbar{} Description
\hyperref[clean-goals-available-parameters]{bundle-support:clean}
\textbar{} Deletes a file from the \texttt{deploy} directory of a
Liferay bundle.
\hyperref[create-token-goals-available-parameters]{bundle-support:create-token}
\textbar{} Creates a token used to validate your user credentials when
downloading a DXP bundle.
\hyperref[deploy-goals-available-parameters]{bundle-support:deploy}
\textbar{} Deploys the Maven project to the specified Liferay DXP
bundle. \hyperref[dist-goals-available-parameters]{bundle-support:dist}
\textbar{} Creates a distributable Liferay DXP bundle archive file
(e.g., ZIP).
\hyperref[init-goals-available-parameters]{bundle-support:init}
\textbar{} Downloads and installs the specified Liferay DXP version.

\subsection{clean Goal's Available
Parameters}\label{clean-goals-available-parameters}

You can set the following parameters in the \texttt{clean} execution's
\texttt{\textless{}configuration\textgreater{}} section of the POM:

Parameter Name \textbar{} Type \textbar{} Default Value \textbar{}
Description \texttt{liferayHome} \textbar{} \texttt{String} \textbar{}
\texttt{bundles} \textbar{} The directory where your Liferay DXP
instance resides. This can be specified from the command line as
\texttt{-DliferayHome=}. \texttt{fileName} \textbar{} \texttt{String}
\textbar{} \texttt{\$\{project.artifactId\}.\$\{project.packaging\}}
\textbar{} The name of the file to delete from your bundle.

\subsection{create-token Goal's Available
Parameters}\label{create-token-goals-available-parameters}

You can change the default parameter values of the \texttt{create-token}
goal by creating an \texttt{\textless{}execution\textgreater{}} section
containing \texttt{\textless{}configuration\textgreater{}} tags. For
example,

\begin{verbatim}
<execution>
    <id>create-token</id>
    <goals>
        <goal>create-token</goal>
    </goals>
    <configuration>
    </configuration>
</execution>
\end{verbatim}

You can set the following parameters in the \texttt{create-token}
execution's \texttt{\textless{}configuration\textgreater{}} section of
the POM:

Parameter Name \textbar{} Type \textbar{} Default Value \textbar{}
Description \texttt{emailAddress} \textbar{} \texttt{String} \textbar{}
\texttt{null} \textbar{} The email address to use when downloading a DXP
bundle. This email address must match the one registered for your DXP
subscription. \texttt{force} \textbar{} \texttt{boolean} \textbar{}
\texttt{false} \textbar{} Whether to override the existing token with a
newly generated one. \texttt{password} \textbar{} \texttt{String}
\textbar{} \texttt{null} \textbar{} The password to use when downloading
a DXP bundle. This password must match the one registered for your DXP
subscription. \texttt{passwordFile} \textbar{} \texttt{File} \textbar{}
\texttt{null} \textbar{} The file to hold your password used when
downloading a DXP bundle. \texttt{tokenFile} \textbar{} \texttt{File}
\textbar{} \texttt{\$\{user.home\}/.liferay/token} \textbar{} The file
to hold the Liferay bundle authentication token. \texttt{tokenUrl}
\textbar{} \texttt{URL} \textbar{}
\texttt{https://releases-cdn.liferay.com/portal/7.1.0-b3/liferay-ce-portal-tomcat-7.1-b3-20180611140920623.zip}
\textbar{} The URL pointing to the bundle Zip to download.

After executing the \texttt{create-token} goal, you're prompted for your
email address and password, both of which are used to generate your
token. It's recommended to configure your email and password from the
command line rather than specifying them in your POM file.

\subsection{deploy Goal's Available
Parameters}\label{deploy-goals-available-parameters}

You can set the following parameters in the \texttt{deploy} execution's
\texttt{\textless{}configuration\textgreater{}} section of the POM:

Parameter Name \textbar{} Type \textbar{} Default Value \textbar{}
Description \texttt{liferayHome} \textbar{} \texttt{String} \textbar{}
\texttt{bundles} \textbar{} The directory where your Liferay DXP
instance resides. This can be specified from the command line as
\texttt{-DliferayHome=}. \texttt{deployFile} \textbar{} \texttt{File}
\textbar{}
\texttt{\$\{project.build.directory\}/\$\{project.build.finalName\}.\$\{project.packaging\}}
\textbar{} The packaged file (e.g., JAR) to deploy to the Liferay
bundle. \texttt{outputFileName} \textbar{} \texttt{String} \textbar{}
\texttt{\$\{project.artifactId\}.\$\{project.packaging\}} \textbar{} The
name of the output file.

\subsection{dist Goal's Available
Parameters}\label{dist-goals-available-parameters}

You can change the default parameter values of the \texttt{dist} goal by
creating an \texttt{\textless{}execution\textgreater{}} section
containing \texttt{\textless{}configuration\textgreater{}} tags. For
example,

\begin{verbatim}
<execution>
    <id>dist</id>
    <goals>
        <goal>dist</goal>
    </goals>
    <configuration>
    </configuration>
</execution>
\end{verbatim}

You can set the following parameters in the \texttt{dist} execution's
\texttt{\textless{}configuration\textgreater{}} section of the POM:

Parameter Name \textbar{} Type \textbar{} Default Value \textbar{}
Description \texttt{liferayHome} \textbar{} \texttt{String} \textbar{}
\texttt{bundles} \textbar{} The directory where your Liferay DXP
instance resides. This can be specified from the command line as
\texttt{-DliferayHome=}. \texttt{archiveFileName} \textbar{}
\texttt{String} \textbar{} \texttt{null} \textbar{} The name for the
generated archive file. \texttt{cacheDir} \textbar{} \texttt{File}
\textbar{} \texttt{\$\{user.home\}/.liferay/bundles} \textbar{} The
directory where the downloaded bundle Zip files are stored.
\texttt{configs} \textbar{} \texttt{String} \textbar{} \texttt{configs}
\textbar{} The directory that contains the configuration files.
\texttt{deployFile} \textbar{} \texttt{File}
\textbar{}\texttt{\$\{project.build.directory\}/\$\{project.build.finalName\}.\$\{project.packaging\}}
\textbar{} The packaged file (e.g., JAR) to deploy to the Liferay
bundle. \texttt{environment} \textbar{} \texttt{String} \textbar{}
\texttt{\$\{liferay.workspace.environment\}} \textbar{} The environment
of your Liferay home deployment. (e.g., \texttt{common}, \texttt{dev},
\texttt{local}, \texttt{prod}, and \texttt{uat}). \texttt{format}
\textbar{} \texttt{String} \textbar{} \texttt{zip} \textbar{} The format
type to use when packaging the Liferay bundle as an archive.
\texttt{includeFolder} \textbar{} \texttt{boolean} \textbar{}
\texttt{true} \textbar{} Whether to add a parent folder to the archive.
\texttt{outputFileName} \textbar{} \texttt{String} \textbar{}
\texttt{\$\{project.artifactId\}.\$\{project.packaging\}} \textbar{} The
path to the archive file. \texttt{password} \textbar{} \texttt{String}
\textbar{} \texttt{null} \textbar{} The password if your Liferay
bundle's URL requires authentication. \texttt{stripComponents}
\textbar{} \texttt{int} \textbar{} \texttt{1} \textbar{} The number of
directories to strip when expanding your bundle. \texttt{token}
\textbar{} \texttt{boolean} \textbar{} \texttt{false} \textbar{} Whether
to use a token to download a Liferay DXP bundle. This should be set to
\texttt{true} when downloading a DXP bundle. \texttt{tokenFile}
\textbar{} \texttt{File} \textbar{}
\texttt{\$\{user.home\}/.liferay/token} \textbar{} The file to hold the
Liferay bundle authentication token. \texttt{url} \textbar{}
\texttt{URL} \textbar{} \texttt{\$\{liferay.workspace.bundle.url\}}
\textbar{} The URL of the Liferay bundle to expand. \texttt{userName}
\textbar{} \texttt{String} \textbar{} \texttt{null} \textbar{} The user
name if your Liferay bundle's URL requires authentication.

\subsection{init Goal's Available
Parameters}\label{init-goals-available-parameters}

You can change the default parameter values of the \texttt{init} goal by
creating an \texttt{\textless{}execution\textgreater{}} section
containing \texttt{\textless{}configuration\textgreater{}} tags. For
example,

\begin{verbatim}
<execution>
    <id>init</id>
    <goals>
        <goal>init</goal>
    </goals>
    <configuration>
    </configuration>
</execution>
\end{verbatim}

You can set the following parameters in the \texttt{init} execution's
\texttt{\textless{}configuration\textgreater{}} section of the POM:

Parameter Name \textbar{} Type \textbar{} Default Value \textbar{}
Description \texttt{liferayHome} \textbar{} \texttt{String} \textbar{}
\texttt{bundles} \textbar{} The directory where your Liferay DXP
instance resides. This can be specified from the command line as
\texttt{-DliferayHome=}. \texttt{cacheDir} \textbar{} \texttt{File}
\textbar{} \texttt{\$\{user.home\}/.liferay/bundles} \textbar{} The
directory where the downloaded bundle Zip files are stored.
\texttt{configs} \textbar{} \texttt{String} \textbar{} \texttt{configs}
\textbar{} The directory that contains the configuration files.
\texttt{environment} \textbar{} \texttt{String} \textbar{}
\texttt{\$\{liferay.workspace.environment\}} \textbar{} The environment
with the settings appropriate for current development (e.g.,
\texttt{common}, \texttt{dev}, \texttt{local}, \texttt{prod}, and
\texttt{uat}). \texttt{password} \textbar{} \texttt{String} \textbar{}
\texttt{null} \textbar{} The password if your Liferay bundle's URL
requires authentication. \texttt{stripComponents} \textbar{}
\texttt{int} \textbar{} \texttt{1} \textbar{} The number of directories
to strip when expanding your bundle. \texttt{token} \textbar{}
\texttt{boolean} \textbar{} \texttt{false} \textbar{} Whether to use a
token to download a Liferay DXP bundle. This should be set to
\texttt{true} when downloading a DXP bundle. \texttt{tokenFile}
\textbar{} \texttt{File} \textbar{}
\texttt{\$\{user.home\}/.liferay/token} \textbar{} The file to hold the
Liferay bundle authentication token. \texttt{url} \textbar{}
\texttt{URL} \textbar{} \texttt{\$\{liferay.workspace.bundle.url\}}
\textbar{} The URL of the Liferay bundle to expand. \texttt{userName}
\textbar{} \texttt{String} \textbar{} \texttt{null} \textbar{} The user
name if your Liferay bundle's URL requires authentication.

\section{CSS Builder Plugin}\label{css-builder-plugin}

The CSS Builder plugin lets you compile
\href{http://sass-lang.com/}{Sass} files in your project.

\subsection{Usage}\label{usage-31}

To use the plugin, include it in your project's root \texttt{pom.xml}
file:

\begin{verbatim}
<build>
    <plugins>
    ...
        <plugin>
            <groupId>com.liferay</groupId>
            <artifactId>com.liferay.css.builder</artifactId>
            <version>3.0.0</version>
            <executions>
                <execution>
                    <id>default-build</id>
                    <phase>compile</phase>
                    <goals>
                        <goal>build</goal>
                    </goals>
                </execution>
            </executions>
            <configuration>
            </configuration>
        </plugin>
    ...
    </plugins>
</build>
\end{verbatim}

You can view an example POM containing the CSS Builder configuration
\href{https://github.com/liferay/liferay-portal/blob/master/modules/util/css-builder/samples/pom.xml}{here}.

\subsection{Goals}\label{goals-1}

The plugin adds one Maven goal to your project:

Name \textbar{} Description \texttt{css-builder:build} \textbar{}
Compiles the Sass files in the project.

\subsection{Available Parameters}\label{available-parameters}

You can set the following parameters in the
\texttt{\textless{}configuration\textgreater{}} section of the POM:

Parameter Name \textbar{} Type \textbar{} Default Value \textbar{}
Description \texttt{appendCssImportTimestamps} \textbar{}
\texttt{boolean} \textbar{} \texttt{true} \textbar{} Whether to append
the current timestamp to the URLs in the \texttt{@import} CSS at-rules.
\texttt{baseDir} \textbar{} \texttt{File} \textbar{}
\texttt{"src/META-INF/resources"} \textbar{} The base directory that
contains the SCSS files to compile. \texttt{dirNames} \textbar{}
\texttt{List\textless{}String\textgreater{}} \textbar{}
\texttt{{[}"/"{]}} \textbar{} The name of the directories, relative to
\hyperref[basedir]{\texttt{baseDir}}, which contain the SCSS files to
compile. \texttt{generateSourceMap} \textbar{} \texttt{boolean}
\textbar{} \texttt{false} \textbar{} Whether to generate
\href{https://developers.google.com/web/tools/chrome-devtools/debug/readability/source-maps}{source
maps} for easier debugging. \texttt{importDir} \textbar{} \texttt{File}
\textbar{} \texttt{null} \textbar{} The \texttt{META-INF/resources}
directory of the
\href{https://github.com/liferay/liferay-portal/tree/master/modules/apps/frontend-css/frontend-css-common}{Liferay
Frontend Common CSS} artifact. This is required in order to make
\href{http://bourbon.io}{Bourbon} and other CSS libraries available to
the compilation. \texttt{outputDirName} \textbar{} \texttt{String}
\textbar{} \texttt{".sass-cache/"} \textbar{} The name of the
sub-directories where the SCSS files are compiled to. For each directory
that contains SCSS files, a sub-directory with this name is created.
\texttt{precision} \textbar{} \texttt{int} \textbar{} \texttt{9}
\textbar{} The numeric precision of numbers in Sass.
\texttt{rtlExcludedPathRegexps} \textbar{}
\texttt{List\textless{}String\textgreater{}} \textbar{} \textbar{} The
SCSS file patterns to exclude when converting for right-to-left (RTL)
support. \texttt{sassCompilerClassName} \textbar{} \texttt{String}
\textbar{} \texttt{"jni"} \textbar{} The type of Sass compiler to use.
Supported values are \texttt{"jni"} and \texttt{"ruby"}. The Ruby Sass
compiler requires \texttt{com.liferay.sass.compiler.ruby.jar},
\texttt{com.liferay.ruby.gems.jar}, and \texttt{jruby-complete.jar} to
be added to the classpath.

You can also manage the \texttt{com.liferay.frontend.css.common} default
theme dependency provided by the CSS Builder in your \texttt{pom.xml}.
This can be modified by adding it as a project dependency:

\begin{verbatim}
<project>
    ...
    <dependencies>
        <dependency>
            <groupId>com.liferay</groupId>
            <artifactId>com.liferay.frontend.css.common</artifactId>
            <version>3.0.1</version>
            <scope>provided</scope>
        </dependency>
        ...
    </dependencies>
</project>
\end{verbatim}

There are additional Liferay theme-related dependencies you can manage
this way that are provided by the Theme Builder. See
\href{/docs/7-1/reference/-/knowledge_base/r/theme-builder-plugin}{this
section} for more information.

\section{DB Support Plugin}\label{db-support-plugin}

The DB Support plugin lets you run the Liferay DB Support tool to
execute certain actions on a local Liferay DXP database. The following
actions are available:

\begin{itemize}
\tightlist
\item
  Cleans the Liferay database from the Service Builder tables and rows
  of a module.
\end{itemize}

\subsection{Usage}\label{usage-32}

To use the plugin, include it in your project's \texttt{pom.xml} file:

\begin{verbatim}
<build>
    <plugins>
    ...
        <plugin>
            <groupId>com.liferay</groupId>
            <artifactId>com.liferay.portal.tools.db.support</artifactId>
            <version>1.0.6</version>
            <configuration>
            </configuration>
            <dependencies>
                <dependency>
                    <groupId>org.hsqldb</groupId>
                    <artifactId>hsqldb</artifactId>
                    <version>2.4.0</version>
                </dependency>
            </dependencies>
        </plugin>
    ...
    </plugins>
</build>
\end{verbatim}

Also notice the configured plugin dependency. You must configure the
JDBC driver used by your Liferay DXP bundle so the DB Support plugin can
properly manage your database. Replace the HSQLDB driver listed above
with your custom database's JDBC driver.

\subsection{Goals}\label{goals-2}

The plugin adds one Maven goal to your project:

Name \textbar{} Description \texttt{db-support:clean-service-builder}
\textbar{} Cleans the Liferay DXP database from the Service Builder
tables and rows of a module.

\subsection{Available Parameters}\label{available-parameters-1}

You can set the following parameters in the
\texttt{\textless{}configuration\textgreater{}} section of the POM:

Parameter Name \textbar{} Type \textbar{} Default Value \textbar{}
Description \texttt{password} \textbar{} \texttt{String} \textbar{}
\texttt{jdbc.default.password} \textbar{} The user password for
connecting to the Liferay DXP database. \texttt{propertiesFile}
\textbar{} \texttt{File} \textbar{} \texttt{null} \textbar{} The
\texttt{portal-ext.properties} file which contains the JDBC settings for
connecting to the Liferay DXP database. \texttt{serviceXmlFile}
\textbar{} \texttt{File} \textbar{} \texttt{null} \textbar{} The
\texttt{service.xml} file of the module. \texttt{servletContextName}
\textbar{} \texttt{String} \textbar{} \texttt{null} \textbar{} The
servlet context name (usually the value of the
\texttt{Bundle-Symbolic-Name} manifest header) of the module.
\texttt{url} \textbar{} \texttt{String} \textbar{}
\texttt{jdbc.default.url} \textbar{} The JDBC URL for connecting to the
Liferay DXP database. \texttt{userName} \textbar{} \texttt{String}
\textbar{} \texttt{jdbc.default.username} \textbar{} The user name for
connecting to the Liferay DXP database.

\section{Deployment Helper Plugin}\label{deployment-helper-plugin}

The Deployment Helper plugin lets you create a cluster deployable WAR
from your OSGi artifacts.

\subsection{Usage}\label{usage-33}

To use the plugin, include it in your project's root \texttt{pom.xml}
file:

\begin{verbatim}
<build>
    <plugins>
    ...
        <plugin>
            <groupId>com.liferay</groupId>
            <artifactId>com.liferay.deployment.helper</artifactId>
            <version>1.0.4</version>
            <configuration>
            </configuration>
        </plugin>
    ...
    </plugins>
</build>
\end{verbatim}

You can view an example POM containing the Deployment Helper
configuration
\href{https://github.com/liferay/liferay-portal/blob/master/modules/util/deployment-helper/samples/pom.xml}{here}.

\subsection{Goals}\label{goals-3}

The plugin adds one Maven goal to your project:

Name \textbar{} Description \texttt{deployment-helper:build} \textbar{}
Builds a WAR which contains one or more files that are copied once the
WAR is deployed.

\subsection{Available Parameters}\label{available-parameters-2}

You can set the following parameters in the
\texttt{\textless{}configuration\textgreater{}} section of the POM:

Parameter Name \textbar{} Type \textbar{} Default Value \textbar{}
Description \texttt{deploymentFileNames} \textbar{} \texttt{String}
\textbar{} \texttt{null} \textbar{} The files or directories to include
in the WAR and copy once the WAR is deployed. If a directory is added to
this collection, all the JAR files contained in the directory are
included in the WAR. \texttt{deploymentPath} \textbar{} \texttt{String}
\textbar{} \texttt{null} \textbar{} The directory to which the included
files are copied. \texttt{outputFileName} \textbar{} \texttt{String}
\textbar{} \texttt{null} \textbar{} The WAR file to build.

\section{Javadoc Formatter Plugin}\label{javadoc-formatter-plugin}

The Javadoc Formatter plugin lets you format project Javadoc comments.
The tool lets you generate:

\begin{itemize}
\tightlist
\item
  Default
  \href{http://www.oracle.com/technetwork/java/javase/documentation/index-137868.html\#@author}{\texttt{@author}}
  tags to all classes.
\item
  Comment stubs to classes, fields, and methods.
\item
  Missing
  \href{https://docs.oracle.com/javase/8/docs/api/java/lang/Override.html}{\texttt{@Override}}
  annotations.
\item
  An XML representation of the Javadoc comments, which can be used by
  tools in order to index the Javadocs of the project.
\end{itemize}

\subsection{Usage}\label{usage-34}

To use the plugin, include it in your project's root \texttt{pom.xml}
file:

\begin{verbatim}
<build>
    <plugins>
    ...
        <plugin>
            <groupId>com.liferay</groupId>
            <artifactId>com.liferay.javadoc.formatter</artifactId>
            <version>1.0.32</version>
            <configuration>
            </configuration>
        </plugin>
    ...
    </plugins>
</build>
\end{verbatim}

You can view an example POM containing the Javadoc Formatter
configuration
\href{https://github.com/liferay/liferay-portal/blob/master/modules/util/javadoc-formatter/samples/pom.xml}{here}.

\subsection{Goals}\label{goals-4}

The plugin adds one Maven goal to your project:

Name \textbar{} Description \texttt{javadoc-formatter:format} \textbar{}
Runs the Liferay Javadoc Formatter to format files.

\subsection{Available Parameters}\label{available-parameters-3}

You can set the following parameters in the
\texttt{\textless{}configuration\textgreater{}} section of the POM:

Parameter Name \textbar{} Type \textbar{} Default Value \textbar{}
Description \texttt{author} \textbar{} \texttt{String} \textbar{}
\texttt{"Brian\ Wing\ Shun\ Chan"} \textbar{} The value of the
\texttt{@author} tag to add at class level if missing.
\texttt{generateXml} \textbar{} \texttt{boolean} \textbar{}
\texttt{false} \textbar{} Whether to generate a XML representation of
the Javadoc comments. The XML files are generated in the
\texttt{src/main/resources} directory only if the Java files are
contained in \texttt{src/main/java}. \texttt{initializeMissingJavadocs}
\textbar{} \texttt{boolean} \textbar{} \texttt{false} \textbar{} Whether
to add comment stubs at the class, field, and method levels. If
\texttt{false}, only the class-level \texttt{@author} is added.
\texttt{inputDirName} \textbar{} \texttt{String} \textbar{}
\texttt{"./"} \textbar{} The root directory to begin searching for Java
files to format. \texttt{limits} \textbar{} \texttt{String{[}{]}}
\textbar{} \texttt{{[}{]}} \textbar{} The Java file name patterns,
relative to the working directory, to include when formatting Javadoc
comments. The patterns must be specified without the \texttt{.java} file
type suffix. If empty, all Java files are formatted.
\texttt{outputFilePrefix} \textbar{} \texttt{String} \textbar{}
\texttt{"javadocs"} \textbar{} The file name prefix of the XML
representation of the Javadoc comments. If \texttt{generateXML} is
\texttt{false}, this property is not used. \texttt{updateJavadocs}
\textbar{} \texttt{boolean} \textbar{} \texttt{false} \textbar{} Whether
to fix existing comment blocks by adding missing tags.

\section{Lang Builder Plugin}\label{lang-builder-plugin}

The Lang Builder plugin lets you sort and translate the language keys in
your project.

\subsection{Usage}\label{usage-35}

To use the plugin, include it in your project's root \texttt{pom.xml}
file:

\begin{verbatim}
<build>
    <plugins>
    ...
        <plugin>
            <groupId>com.liferay</groupId>
            <artifactId>com.liferay.lang.builder</artifactId>
            <version>1.0.31</version>
            <configuration>
            </configuration>
        </plugin>
    ...
    </plugins>
</build>
\end{verbatim}

You can view an example POM containing the Lang Builder configuration
\href{https://github.com/liferay/liferay-portal/blob/master/modules/util/lang-builder/samples/pom.xml}{here}.

\subsection{Goals}\label{goals-5}

The plugin adds one Maven goal to your project:

Name \textbar{} Description \texttt{lang-builder:build} \textbar{} Runs
Liferay Lang Builder to translate language property files.

\subsection{Available Parameters}\label{available-parameters-4}

You can set the following parameters in the
\texttt{\textless{}configuration\textgreater{}} section of the POM:

Parameter Name \textbar{} Type \textbar{} Default Value \textbar{}
Description \texttt{excludedLanguageIds} \textbar{}
\texttt{String{[}{]}} \textbar{}
\texttt{\{"da",\ "de",\ "fi",\ "ja",\ "nl",\ "pt\_PT",\ "sv"\}}
\textbar{} The language IDs to exclude in the automatic translation.
\texttt{langDirName} \textbar{} \texttt{String} \textbar{}
\texttt{"src/content"} \textbar{} The directory where the language
properties files are saved. \texttt{langFileName} \textbar{}
\texttt{String} \textbar{} \texttt{"Language"} \textbar{} The file name
prefix of the language properties files (e.g.,
\texttt{Language\_it.properties}). \texttt{plugin} \textbar{}
\texttt{boolean} \textbar{} \texttt{true} \textbar{} Whether to check
for duplicate language keys between the project and the portal.
\texttt{portalLanguagePropertiesFileName} \textbar{} \texttt{String}
\textbar{} \texttt{null} \textbar{} The \texttt{Language.properties}
file of the portal. \texttt{translate} \textbar{} \texttt{boolean}
\textbar{} \texttt{true} \textbar{} Whether to translate the language
keys and generate a language properties file for each locale that's
supported by Liferay DXP. \texttt{translateSubscriptionKey} \textbar{}
\texttt{String} \textbar{} \texttt{null} \textbar{} The subscription key
for Microsoft Translation integration. Subscription to the Translator
Text Translation API on Microsoft Cognitive Services is required. Basic
subscriptions, up to 2 million characters a month, are free.

\section{REST Builder Plugin}\label{rest-builder-plugin}

The REST Builder plugin lets you generate a REST layer defined in the
REST Builder \texttt{rest-config.yaml} and \texttt{rest-openapi.yaml}
files.

\subsection{Usage}\label{usage-36}

To use the plugin, include it in your project's root \texttt{pom.xml}
file:

\begin{verbatim}
<build>
    <plugins>
    ...
        <plugin>
            <groupId>com.liferay</groupId>
            <artifactId>com.liferay.portal.tools.rest.builder</artifactId>
            <version>1.0.22</version>
            <configuration>
            </configuration>
        </plugin>
    ...
    </plugins>
</build>
\end{verbatim}

You can view an example POM containing the REST Builder configuration
\href{https://github.com/liferay/liferay-portal/blob/master/modules/util/portal-tools-rest-builder/samples/pom.xml}{here}.

\subsection{Goals}\label{goals-6}

The plugin adds one Maven goal to your project:

Name \textbar{} Description \texttt{rest-builder:build} \textbar{} Runs
the Liferay REST Builder.

\subsection{Available Parameters}\label{available-parameters-5}

You can set the following parameters in the
\texttt{\textless{}configuration\textgreater{}} section of the POM:

Parameter Name \textbar{} Type \textbar{} Default Value \textbar{}
Description \texttt{copyrightFile} \textbar{} \texttt{File} \textbar{}
\texttt{null} \textbar{} The file that contains the copyright header.
\texttt{restConfigDir} \textbar{} \texttt{File} \textbar{}
\texttt{\$\{project.projectDir\}} \textbar{} The directory that contains
the \texttt{rest-config.yaml} and \texttt{rest-openapi.yaml} files.

\section{Service Builder Plugin}\label{service-builder-plugin}

The Service Builder plugin lets you generate a service layer defined in
a
\href{/docs/7-1/tutorials/-/knowledge_base/t/what-is-service-builder}{Service
Builder} \texttt{service.xml} file. Visit the
\href{/docs/7-1/tutorials/-/knowledge_base/t/using-service-builder-in-a-maven-project}{Using
Service Builder in a Maven Project} tutorial to learn more about
applying Service Builder to your Maven project.

\subsection{Usage}\label{usage-37}

To use the plugin, include it in your project's root \texttt{pom.xml}
file:

\begin{verbatim}
<build>
    <plugins>
    ...
        <plugin>
            <groupId>com.liferay</groupId>
            <artifactId>com.liferay.portal.tools.service.builder</artifactId>
            <version>1.0.292</version>
            <configuration>
            </configuration>
        </plugin>
    ...
    </plugins>
</build>
\end{verbatim}

You can view an example POM containing the Service Builder configuration
\href{https://github.com/liferay/liferay-portal/blob/master/modules/util/portal-tools-service-builder/samples/pom.xml}{here}.

\subsection{Goals}\label{goals-7}

The plugin adds one Maven goal to your project:

Name \textbar{} Description \texttt{service-builder:build} \textbar{}
Runs the Liferay Service Builder.

\subsection{Available Parameters}\label{available-parameters-6}

You can set the following parameters in the
\texttt{\textless{}configuration\textgreater{}} section of the POM:

Parameter Name \textbar{} Type \textbar{} Default Value \textbar{}
Description \texttt{apiDirName} \textbar{} \texttt{String} \textbar{}
\texttt{"../portal-kernel/src"} \textbar{} A directory where the service
API Java source files are generated.
\texttt{autoImportDefaultReferences} \textbar{} \texttt{boolean}
\textbar{} \texttt{true} \textbar{} Whether to automatically add default
references, like \texttt{com.liferay.portal.ClassName},
\texttt{com.liferay.portal.Resource} and
\texttt{com.liferay.portal.User}, to the services.
\texttt{autoNamespaceTables} \textbar{} \texttt{boolean} \textbar{}
\texttt{null} \textbar{} Whether to prefix table names by the namespace
specified in the \texttt{service.xml} file. \texttt{beanLocatorUtil}
\textbar{} \texttt{String} \textbar{}
\texttt{"com.liferay.portal.kernel.bean.PortalBeanLocatorUtil"}
\textbar{} The fully qualified class name of a bean locator class to use
in the generated service classes. \texttt{buildNumber} \textbar{}
\texttt{long} \textbar{} \texttt{1} \textbar{} A specific value to
assign the \texttt{build.number} property in the
\texttt{service.properties} file. \texttt{buildNumberIncrement}
\textbar{} \texttt{boolean} \textbar{} \texttt{true} \textbar{} Whether
to automatically increment the \texttt{build.number} property in the
\texttt{service.properties} file by one at every service generation.
\texttt{databaseNameMaxLength} \textbar{} \texttt{int} \textbar{}
\texttt{30} \textbar{} The upper bound for database table and column
name lengths to ensure it works on all databases. \texttt{hbmFileName}
\textbar{} \texttt{String} \textbar{}
\texttt{"src/META-INF/portal-hbm.xml"} \textbar{} A Hibernate Mapping
file to generate. \texttt{implDirName} \textbar{} \texttt{String}
\textbar{} \texttt{"src"} \textbar{} A directory where the service Java
source files are generated. \texttt{inputFileName} \textbar{}
\texttt{String} \textbar{} \texttt{"service.xml"} \textbar{} The
project's \texttt{service.xml} file. \texttt{modelHintsConfigs}
\textbar{} \texttt{String} \textbar{}
\texttt{"classpath*:META-INF/portal-model-hints.xml,\ META-INF/portal-model-hints.xml,\ classpath*:META-INF/ext-model-hints.xml,\ classpath*:META-INF/portlet-model-hints.xml"}
\textbar{} Paths to the
\href{/docs/7-1/tutorials/-/knowledge_base/t/customizing-model-entities-with-model-hints}{model
hints} files for Liferay Service Builder to use in generating the
service layer. \texttt{modelHintsFileName} \textbar{} \texttt{String}
\textbar{} \texttt{"src/META-INF/portal-model-hints.xml"} \textbar{} A
model hints file for the project. \texttt{osgiModule} \textbar{}
\texttt{boolean} \textbar{} \texttt{null} \textbar{} Whether to generate
the service layer for OSGi modules. \texttt{pluginName} \textbar{}
\texttt{String} \textbar{} \texttt{null} \textbar{} If specified, a
plugin can enable additional generation features, such as \texttt{Clp}
class generation, for non-OSGi modules. \texttt{propsUtil} \textbar{}
\texttt{String} \textbar{} \texttt{"com.liferay.portal.util.PropsUtil"}
\textbar{} The fully qualified class name of the service properties util
class to generate. \texttt{readOnlyPrefixes} \textbar{} \texttt{String}
\textbar{} \texttt{"fetch,\ get,\ has,\ is,\ load,\ reindex,\ search"}
\textbar{} Prefixes of methods to consider read-only.
\texttt{resourceActionsConfigs} \textbar{} \texttt{String} \textbar{}
\texttt{"META-INF/resource-actions/default.xml,\ resource-actions/default.xml"}
\textbar{} Paths to the
\href{/docs/7-1/tutorials/-/knowledge_base/t/defining-application-permissions}{resource
actions} files for Liferay Service Builder to use in generating the
service layer. \texttt{resourcesDirName} \textbar{} \texttt{String}
\textbar{} \texttt{"src"} \textbar{} A directory where the service
non-Java files are generated. \texttt{springFileName} \textbar{}
\texttt{String} \textbar{} \texttt{"src/META-INF/portal-spring.xml"}
\textbar{} A service Spring file to generate. \texttt{springNamespaces}
\textbar{} \texttt{String} \textbar{} \texttt{"beans"} \textbar{}
Namespaces of Spring XML Schemas to add to the service Spring file.
\texttt{sqlDirName} \textbar{} \texttt{String} \textbar{}
\texttt{"../sql"} \textbar{} A directory where the SQL files are
generated. \texttt{sqlFileName} \textbar{} \texttt{String} \textbar{}
\texttt{"portal-tables.sql"} \textbar{} A name (relative to
\texttt{sqlDir}) for the file in which the SQL table creation
instructions are generated. \texttt{sqlIndexesFileName} \textbar{}
\texttt{String} \textbar{} \texttt{"indexes.sql"} \textbar{} A name
(relative to \texttt{sqlDir}) for the file in which the SQL index
creation instructions are generated. \texttt{sqlSequencesFileName}
\textbar{} \texttt{String} \textbar{} \texttt{"sequences.sql"}
\textbar{} A name (relative to \texttt{sqlDir}) for the file in which
the SQL sequence creation instructions are generated.
\texttt{targetEntityName} \textbar{} \texttt{String} \textbar{}
\texttt{null} \textbar{} If specified, it's the name of the entity for
which Liferay Service Builder should generate the service.
\texttt{testDirName} \textbar{} \texttt{String} \textbar{}
\texttt{"test/integration"} \textbar{} If specified, it's a directory
where integration test Java source files are generated.

\section{Source Formatter Plugin}\label{source-formatter-plugin}

The Source Formatter plugin formats project files according to Liferay's
source formatting standards. For more documentation on Source Formatter
specific functionality, visit the tool's
\href{https://github.com/liferay/liferay-portal/tree/master/modules/util/source-formatter/documentation}{documentation}
folder.

\subsection{Usage}\label{usage-38}

To use the plugin, include it in your project's root \texttt{pom.xml}
file:

\begin{verbatim}
<build>
    <plugins>
    ...
        <plugin>
            <groupId>com.liferay</groupId>
            <artifactId>com.liferay.source.formatter</artifactId>
            <version>1.0.885</version>
            <executions>
                <execution>
                    <phase>process-sources</phase>
                    <goals>
                        <goal>format</goal>
                    </goals>
                </execution>
            </executions>
            <configuration>
            </configuration>
        </plugin>
    ...
    </plugins>
</build>
\end{verbatim}

You can view an example POM containing the Source Formatter
configuration
\href{https://github.com/liferay/liferay-portal/blob/master/modules/util/source-formatter/samples/pom.xml}{here}.

\subsection{Goals}\label{goals-8}

The plugin adds one Maven goal to your project:

Name \textbar{} Description \texttt{source-formatter:format} \textbar{}
Runs the Liferay Source Formatter to format source formatting errors.

\subsection{Available Parameters}\label{available-parameters-7}

You can set the following parameters in the
\texttt{\textless{}configuration\textgreater{}} section of the POM:

Parameter Name \textbar{} Type \textbar{} Default Value \textbar{}
Description \texttt{autoFix} \textbar{} \texttt{boolean} \textbar{}
\texttt{true} \textbar{} Whether to automatically fix source formatting
errors. \texttt{baseDir} \textbar{} \texttt{String} \textbar{}
\texttt{"./"} \textbar{} The Source Formatter base directory.
\emph{(Read-only)} \texttt{fileNames} \textbar{} \texttt{String{[}{]}}
\textbar{} \texttt{null} \textbar{} The file names to format, relative
to the project directory. If \texttt{null}, all files contained in
\texttt{baseDir} will be formatted. \texttt{formatCurrentBranch}
\textbar{} \texttt{boolean} \textbar{} \texttt{false} \textbar{} Whether
to format only the files contained in \texttt{baseDir} that are added or
modified in the current Git branch. \texttt{formatLatestAuthor}
\textbar{} \texttt{boolean} \textbar{} \texttt{false} \textbar{} Whether
to format only the files contained in \texttt{baseDir} that are added or
modified in the latest Git commits of the same author.
\texttt{formatLocalChanges} \textbar{} \texttt{boolean} \textbar{}
\texttt{false} \textbar{} Whether to format only the unstaged files
contained in \texttt{baseDir}. \texttt{gitWorkingBranchName} \textbar{}
\texttt{String} \textbar{} \texttt{"master"} \textbar{} The Git working
branch name. \texttt{includeSubrepositories} \textbar{} \texttt{boolean}
\textbar{} \texttt{false} \textbar{} Whether to format files that are in
read-only subrepositories. \texttt{maxLineLength} \textbar{}
\texttt{int} \textbar{} \texttt{80} \textbar{} The maximum number of
characters allowed in Java files. \texttt{printErrors} \textbar{}
\texttt{boolean} \textbar{} \texttt{true} \textbar{} Whether to print
formatting errors on the Standard Output stream.
\texttt{processorThreadCount} \textbar{} \texttt{int} \textbar{}
\texttt{5} \textbar{} The number of threads used by Source Formatter.
\texttt{showDocumentation} \textbar{} \texttt{boolean} \textbar{}
\texttt{false} \textbar{} Whether to show the documentation for the
source formatting issues, if present. \texttt{throwException} \textbar{}
\texttt{boolean} \textbar{} \texttt{false} \textbar{} Whether to fail
the build if formatting errors are found.

\section{Theme Builder Plugin}\label{theme-builder-plugin}

The Theme Builder plugin lets you build Liferay theme files in your
project. Visit the
\href{/docs/7-1/tutorials/-/knowledge_base/t/building-themes-in-a-maven-project}{Building
Themes in a Maven Project} tutorial to learn more about applying Theme
Builder to your Maven project.

\subsection{Usage}\label{usage-39}

To use the plugin, include it in your project's root \texttt{pom.xml}
file:

\begin{verbatim}
<build>
    <plugins>
    ...
        <plugin>
            <groupId>com.liferay</groupId>
            <artifactId>com.liferay.portal.tools.theme.builder</artifactId>
            <version>1.1.7</version>
            <executions>
                <execution>
                    <phase>generate-resources</phase>
                    <goals>
                        <goal>build</goal>
                    </goals>
                    <configuration>
                    </configuration>
                </execution>
            </executions>
        </plugin>
        ...
    </plugins>
</build>
\end{verbatim}

You can view an example POM containing the Theme Builder configuration
\href{https://github.com/liferay/liferay-portal/blob/master/modules/util/portal-tools-theme-builder/samples/pom.xml}{here}.

\subsection{Goals}\label{goals-9}

The plugin adds one Maven goal to your project:

Name \textbar{} Description \texttt{theme-builder:build} \textbar{}
Builds the theme files.

\subsection{Available Parameters}\label{available-parameters-8}

You can set the following parameters in the
\texttt{\textless{}configuration\textgreater{}} section of the POM:

Parameter Name \textbar{} Type \textbar{} Default Value \textbar{}
Description \texttt{diffsDir} \textbar{} \texttt{File} \textbar{}
\texttt{\$\{maven.war.src\}} \textbar{} The directory that contains the
files to copy over the parent theme. \texttt{name} \textbar{}
\texttt{String} \textbar{} \texttt{\$\{project.artifactId\}} \textbar{}
The name of the new theme. \texttt{outputDir} \textbar{} \texttt{File}
\textbar{}
\texttt{\$\{project.build.directory\}/\$\{project.build.finalName\}}
\textbar{} The directory where to build the theme. \texttt{parentDir}
\textbar{} \texttt{File} \textbar{} \texttt{null} \textbar{} The
directory of the parent theme. \texttt{parentName} \textbar{}
\texttt{String} \textbar{} \texttt{null} \textbar{} The name of the
parent theme. \texttt{templateExtension} \textbar{} \texttt{String}
\textbar{} \texttt{"ftl"} \textbar{} The extension of the template
files, usually \texttt{"ftl"} or \texttt{"vm"}. \texttt{unstyledDir}
\textbar{} \texttt{File} \textbar{} \texttt{null} \textbar{} The
directory of
\href{https://github.com/liferay/liferay-portal/tree/master/modules/apps/frontend-theme/frontend-theme-unstyled}{Liferay
Frontend Theme Unstyled}.

You can also manage the \texttt{com.liferay.frontend.theme.styled} and
\texttt{com.liferay.frontend.theme.unstyled} default theme dependencies
provided by the Theme Builder in your \texttt{pom.xml}. They can be
modified by adding them as project dependencies:

\begin{verbatim}
<project>
    ...
    <dependencies>
        ...
        <dependency>
            <groupId>com.liferay</groupId>
            <artifactId>com.liferay.frontend.theme.styled</artifactId>
            <version>3.0.4</version>
            <scope>provided</scope>
        </dependency>
        <dependency>
            <groupId>com.liferay</groupId>
            <artifactId>com.liferay.frontend.theme.unstyled</artifactId>
            <version>3.0.4</version>
            <scope>provided</scope>
        </dependency>
    </dependencies>
</project>
\end{verbatim}

There is an additional Liferay theme-related dependency you can manage
this way that's provided by the CSS Builder. See
\href{/docs/7-1/reference/-/knowledge_base/r/css-builder-plugin}{this
section} for more information.

\section{TLD Formatter Plugin}\label{tld-formatter-plugin}

The TLD Formatter plugin lets you format a project's TLD files.

\subsection{Usage}\label{usage-40}

To use the plugin, include it in your project's root \texttt{pom.xml}
file:

\begin{verbatim}
<build>
    <plugins>
    ...
        <plugin>
            <groupId>com.liferay</groupId>
            <artifactId>com.liferay.tld.formatter</artifactId>
            <version>1.0.5</version>
            <configuration>
            </configuration>
        </plugin>
    ...
    </plugins>
</build>
\end{verbatim}

You can view an example POM containing the TLD Formatter configuration
\href{https://github.com/liferay/liferay-portal/blob/master/modules/util/tld-formatter/samples/pom.xml}{here}.

\subsection{Goals}\label{goals-10}

The plugin adds one Maven goal to your project:

Name \textbar{} Description \texttt{tld-formatter:format} \textbar{}
Runs the Liferay TLD Formatter to format files.

\subsection{Available Parameters}\label{available-parameters-9}

You can set the following parameters in the
\texttt{\textless{}configuration\textgreater{}} section of the POM:

Parameter Name \textbar{} Type \textbar{} Default Value \textbar{}
Description \texttt{baseDirName} \textbar{} \texttt{String} \textbar{}
\texttt{"./"} \textbar{} The base directory to begin searching for TLD
files to format. \texttt{plugin} \textbar{} \texttt{boolean} \textbar{}
\texttt{true} \textbar{} Whether to format all the TLD files contained
in the working directory. If \texttt{false}, all
\texttt{liferay-portlet-ext.tld} files are ignored.

\section{WSDD Builder Plugin}\label{wsdd-builder-plugin}

The WSDD Builder plugin lets you generate the
\href{http://axis.apache.org/axis/}{Apache Axis} Web Service Deployment
Descriptor (WSDD) files from a
\href{/docs/7-1/tutorials/-/knowledge_base/t/what-is-service-builder}{Service
Builder} \texttt{service.xml} file.

\subsection{Usage}\label{usage-41}

To use the plugin, include it in your project's root \texttt{pom.xml}
file:

\begin{verbatim}
<build>
    <plugins>
    ...
        <plugin>
            <groupId>com.liferay</groupId>
            <artifactId>com.liferay.portal.tools.wsdd.builder</artifactId>
            <version>1.0.10</version>
            <configuration>
            </configuration>
        </plugin>
    ...
    </plugins>
</build>
\end{verbatim}

You can view an example POM containing the WSDD Builder configuration
\href{https://github.com/liferay/liferay-portal/blob/master/modules/util/portal-tools-wsdd-builder/samples/pom.xml}{here}.

\subsection{Goals}\label{goals-11}

The plugin adds one Maven goal to your project:

Name \textbar{} Description \texttt{wsdd-builder:build} \textbar{} Runs
the Liferay WSDD Builder to generate the WSDD files.

\subsection{Available Parameters}\label{available-parameters-10}

You can set the following parameters in the
\texttt{\textless{}configuration\textgreater{}} section of the POM:

Parameter Name \textbar{} Type \textbar{} Default Value \textbar{}
Description \texttt{classPath} \textbar{} \texttt{String} \textbar{}
\texttt{null} \textbar{} The classpath that the Liferay WSDD Builder
uses to generate WSDD files. \texttt{inputFileName} \textbar{}
\texttt{String} \textbar{} \texttt{"service.xml"} \textbar{} The file
from which to generate the WSDD files. \texttt{outputDirName} \textbar{}
\texttt{String} \textbar{} \texttt{"src"} \textbar{} The directory where
the \texttt{*\_deploy.wsdd} and \texttt{*\_undeploy.wsdd} files are
generated. \texttt{serverConfigFileName} \textbar{} \texttt{String}
\textbar{} \texttt{"server-config.wsdd"} \textbar{} The file to
generate. \texttt{serviceNamespace} \textbar{} \texttt{String}
\textbar{} \texttt{"Plugin"} \textbar{} The namespace for the WSDD
Service.

\section{XML Formatter Plugin}\label{xml-formatter-plugin}

The XML Formatter plugin lets you format a project's XML files.

\subsection{Usage}\label{usage-42}

To use the plugin, include it in your project's root \texttt{pom.xml}
file:

\begin{verbatim}
<build>
    <plugins>
    ...
        <plugin>
            <groupId>com.liferay</groupId>
            <artifactId>com.liferay.xml.formatter</artifactId>
            <version>1.0.5</version>
            <configuration>
            </configuration>
        </plugin>
    ...
    </plugins>
</build>
\end{verbatim}

You can view an example POM containing the XML Formatter configuration
\href{https://github.com/liferay/liferay-portal/blob/master/modules/util/xml-formatter/samples/pom.xml}{here}.

\subsection{Goals}\label{goals-12}

The plugin adds one Maven goal to your project:

Name \textbar{} Description \texttt{xml-formatter:format} \textbar{}
Runs the Liferay XML Formatter to format the project files.

\subsection{Available Parameters}\label{available-parameters-11}

You can set the following parameters in the
\texttt{\textless{}configuration\textgreater{}} section of the POM:

Parameter Name \textbar{} Type \textbar{} Default Value \textbar{}
Description \texttt{fileName} \textbar{} \texttt{String} \textbar{}
\texttt{null} \textbar{} The XML file to format. This plugin only lets
you format one XML file at a time. \texttt{stripComments} \textbar{}
\texttt{boolean} \textbar{} \texttt{false} \textbar{} Whether to remove
all the comments from the XML file.

\section{Content Targeting Report
Template}\label{content-targeting-report-template}

In this article, you'll learn how to create a Liferay Content Targeting
Report application as a Liferay module. To create a Content Targeting
Report via the command line using Blade CLI or Maven, use one of the
commands with the following parameters:

\begin{verbatim}
blade create -t content-targeting-report -p [package name] -c [class name] [project name]
\end{verbatim}

or

\begin{verbatim}
mvn archetype:generate \
    -DarchetypeGroupId=com.liferay \
    -DarchetypeArtifactId=com.liferay.project.templates.content.targeting.report \
    -DartifactId=[projectName] \
    -Dpackage=[packageName] \
    -DclassName=[className] \
    -DliferayVersion=7.1
\end{verbatim}

You can also insert the \texttt{-b\ maven} parameter in the Blade
command to generate a Maven project using Blade CLI.

The template for this kind of project is
\texttt{content-targeting-report}. To create a report project called
\texttt{hits-by-country} with a package prefix of \texttt{com.liferay}
and a class name of \texttt{HitsByCountryReport}, use this command:

\begin{verbatim}
blade create -t content-targeting-report -p com.liferay -c HitsByCountry hits-by-country
\end{verbatim}

or

\begin{verbatim}
mvn archetype:generate \
    -DarchetypeGroupId=com.liferay \
    -DarchetypeArtifactId=com.liferay.project.templates.content.targeting.report \
    -DgroupId=com.liferay \
    -DartifactId=hits-by-country \
    -Dpackage=com.liferay \
    -Dversion=1.0 \
    -DclassName=HitsByCountry \
    -Dauthor=Joe Bloggs \
    -DliferayVersion=7.1
\end{verbatim}

The command above creates a Content Targeting Rule project named
\texttt{hits-by-country} in the current folder. In the class, you're
creating a service of type
\texttt{com.liferay.content.targeting.api.model.Report} and extending
the \texttt{com.liferay.content.targeting.api.model.BaseJSPReport}
class. Here, \emph{service} means an OSGi service, not a Liferay API.
Another way to say \emph{service type} is to say \emph{component type}.

After running the command above, your project's folder structure looks
like this:

\begin{itemize}
\tightlist
\item
  \texttt{hits-by-country}

  \begin{itemize}
  \tightlist
  \item
    \texttt{gradle}

    \begin{itemize}
    \tightlist
    \item
      \texttt{wrapper}

      \begin{itemize}
      \tightlist
      \item
        \texttt{gradle-wrapper.jar}
      \item
        \texttt{gradle-wrapper.properties}
      \end{itemize}
    \end{itemize}
  \item
    \texttt{src}

    \begin{itemize}
    \tightlist
    \item
      \texttt{main}

      \begin{itemize}
      \tightlist
      \item
        \texttt{java}

        \begin{itemize}
        \tightlist
        \item
          \texttt{com/liferay/content/targeting/report}

          \begin{itemize}
          \tightlist
          \item
            \texttt{HitsByCountryReport.java}
          \end{itemize}
        \end{itemize}
      \item
        \texttt{resources}

        \begin{itemize}
        \tightlist
        \item
          \texttt{content}

          \begin{itemize}
          \tightlist
          \item
            \texttt{Language.properties}
          \end{itemize}
        \item
          \texttt{META-INF}

          \begin{itemize}
          \tightlist
          \item
            \texttt{resources}

            \begin{itemize}
            \tightlist
            \item
              \texttt{edit.jsp}
            \item
              \texttt{view.jsp}
            \end{itemize}
          \end{itemize}
        \end{itemize}
      \end{itemize}
    \end{itemize}
  \item
    \texttt{bnd.bnd}
  \item
    \texttt{build.gradle}
  \item
    \texttt{gradlew}
  \end{itemize}
\end{itemize}

The Maven-generated project includes a \texttt{pom.xml} file and does
not include the Gradle-specific files, but otherwise, appears exactly
the same.

The generated module is a working application and is deployable to a
Liferay DXP instance. To build upon the generated app, modify the
project by adding logic and additional files to the folders outlined
above.

\section{Content Targeting Rule
Template}\label{content-targeting-rule-template}

In this article, you'll learn how to create a Liferay Content Targeting
Rule application as a Liferay module. To create a Content Targeting Rule
via the command line using Blade CLI or Maven, use one of the commands
with the following parameters:

\begin{verbatim}
blade create -t content-targeting-rule -p [package name] -c [class name] [project name]
\end{verbatim}

or

\begin{verbatim}
mvn archetype:generate \
    -DarchetypeGroupId=com.liferay \
    -DarchetypeArtifactId=com.liferay.project.templates.content.targeting.rule \
    -DartifactId=[projectName] \
    -Dpackage=[packageName] \
    -DclassName=[className] \
    -DliferayVersion=7.1
\end{verbatim}

You can also insert the \texttt{-b\ maven} parameter in the Blade
command to generate a Maven project using Blade CLI.

The template for this kind of project is
\texttt{content-targeting-rule}. To create a rule project called
\texttt{weather} with a package prefix of \texttt{com.liferay} and a
class name of \texttt{Weather}, use this command:

\begin{verbatim}
blade create -t content-targeting-rule -p com.liferay -c Weather weather
\end{verbatim}

or

\begin{verbatim}
mvn archetype:generate \
    -DarchetypeGroupId=com.liferay \
    -DarchetypeArtifactId=com.liferay.project.templates.content.targeting.rule \
    -DgroupId=com.liferay \
    -DartifactId=weather \
    -Dpackage=com.liferay \
    -Dversion=1.0 \
    -DclassName=Weather \
    -Dauthor=Joe Bloggs \
    -DliferayVersion=7.1
\end{verbatim}

The command above creates a Content Targeting Rule project named
\texttt{weather} in the current folder. In the class, you're creating a
service of type \texttt{com.liferay.content.targeting.api.model.Rule}
and extending the
\texttt{com.liferay.content.targeting.api.model.BaseJSPRule} class.
Here, \emph{service} means an OSGi service, not a Liferay API. Another
way to say \emph{service type} is to say \emph{component type}.

After running the command above, your project's folder structure looks
like this:

\begin{itemize}
\tightlist
\item
  \texttt{weather}

  \begin{itemize}
  \tightlist
  \item
    \texttt{gradle}

    \begin{itemize}
    \tightlist
    \item
      \texttt{wrapper}

      \begin{itemize}
      \tightlist
      \item
        \texttt{gradle-wrapper.jar}
      \item
        \texttt{gradle-wrapper.properties}
      \end{itemize}
    \end{itemize}
  \item
    \texttt{src}

    \begin{itemize}
    \tightlist
    \item
      \texttt{main}

      \begin{itemize}
      \tightlist
      \item
        \texttt{java}

        \begin{itemize}
        \tightlist
        \item
          \texttt{com/liferay/content/targeting/rule}

          \begin{itemize}
          \tightlist
          \item
            \texttt{WeatherRule.java}
          \end{itemize}
        \end{itemize}
      \item
        \texttt{resources}

        \begin{itemize}
        \tightlist
        \item
          \texttt{content}

          \begin{itemize}
          \tightlist
          \item
            \texttt{Language.properties}
          \end{itemize}
        \item
          \texttt{META-INF}

          \begin{itemize}
          \tightlist
          \item
            \texttt{resources}

            \begin{itemize}
            \tightlist
            \item
              \texttt{view.jsp}
            \end{itemize}
          \end{itemize}
        \end{itemize}
      \end{itemize}
    \end{itemize}
  \item
    \texttt{bnd.bnd}
  \item
    \texttt{build.gradle}
  \item
    \texttt{gradlew}
  \end{itemize}
\end{itemize}

The Maven-generated project includes a \texttt{pom.xml} file and does
not include the Gradle-specific files, but otherwise, appears exactly
the same.

The generated module is a working application and is deployable to a
Liferay DXP instance. To build upon the generated app, modify the
project by adding logic and additional files to the folders outlined
above.

\section{Content Targeting Tracking Action
Template}\label{content-targeting-tracking-action-template}

In this article, you'll learn how to create a Liferay Content Targeting
Tracking Action application as a Liferay module. To create a Content
Targeting Tracking Action via the command line using Blade CLI or Maven,
use one of the commands with the following parameters:

\begin{verbatim}
blade create -t content-targeting-tracking-action -p [package name] -c [class name] [project name]
\end{verbatim}

or

\begin{verbatim}
mvn archetype:generate \
    -DarchetypeGroupId=com.liferay \
    -DarchetypeArtifactId=com.liferay.project.templates.content.targeting.tracking.action \
    -DartifactId=[projectName] \
    -Dpackage=[packageName] \
    -DclassName=[className] \
    -DliferayVersion=7.1
    
\end{verbatim}

You can also insert the \texttt{-b\ maven} parameter in the Blade
command to generate a Maven project using Blade CLI.

The template for this kind of project is
\texttt{content-targeting-tracking-action}. To create a tracking action
project called \texttt{newsletter} with a package prefix of
\texttt{com.liferay} and a class name of \texttt{Newsletter}, use this
command:

\begin{verbatim}
blade create -t content-targeting-tracking-action -p com.liferay -c Newsletter newsletter
\end{verbatim}

or

\begin{verbatim}
mvn archetype:generate \
    -DarchetypeGroupId=com.liferay \
    -DarchetypeArtifactId=com.liferay.project.templates.content.targeting.tracking.action \
    -DgroupId=com.liferay \
    -DartifactId=newsletter \
    -Dpackage=com.liferay \
    -Dversion=1.0 \
    -DclassName=Newsletter \
    -Dauthor=Joe Bloggs \
    -DliferayVersion=7.1
    
\end{verbatim}

The command above creates a Content Targeting Tracking Action project
named \texttt{newsletter} in the current folder. In the class, you're
creating a service of type
\texttt{com.liferay.content.targeting.api.model.TrackingAction} and
extending the
\texttt{com.liferay.content.targeting.api.model.BaseJSPTrackingAction}
class. Here, \emph{service} means an OSGi service, not a Liferay API.
Another way to say \emph{service type} is to say \emph{component type}.

After running the command above, your project's folder structure looks
like this:

\begin{itemize}
\tightlist
\item
  \texttt{newsletter}

  \begin{itemize}
  \tightlist
  \item
    \texttt{gradle}

    \begin{itemize}
    \tightlist
    \item
      \texttt{wrapper}

      \begin{itemize}
      \tightlist
      \item
        \texttt{gradle-wrapper.jar}
      \item
        \texttt{gradle-wrapper.properties}
      \end{itemize}
    \end{itemize}
  \item
    \texttt{src}

    \begin{itemize}
    \tightlist
    \item
      \texttt{main}

      \begin{itemize}
      \tightlist
      \item
        \texttt{java}

        \begin{itemize}
        \tightlist
        \item
          \texttt{com/liferay/content/targeting/tracking/action}

          \begin{itemize}
          \tightlist
          \item
            \texttt{NewsletterTrackingAction.java}
          \end{itemize}
        \end{itemize}
      \item
        \texttt{resources}

        \begin{itemize}
        \tightlist
        \item
          \texttt{content}

          \begin{itemize}
          \tightlist
          \item
            \texttt{Language.properties}
          \end{itemize}
        \item
          \texttt{META-INF}

          \begin{itemize}
          \tightlist
          \item
            \texttt{resources}

            \begin{itemize}
            \tightlist
            \item
              \texttt{view.jsp}
            \end{itemize}
          \end{itemize}
        \end{itemize}
      \end{itemize}
    \end{itemize}
  \item
    \texttt{bnd.bnd}
  \item
    \texttt{build.gradle}
  \item
    \texttt{gradlew}
  \end{itemize}
\end{itemize}

The Maven-generated project includes a \texttt{pom.xml} file and does
not include the Gradle-specific files, but otherwise, appears exactly
the same.

The generated module is a working application and is deployable to a
Liferay DXP instance. To build upon the generated app, modify the
project by adding logic and additional files to the folders outlined
above.

\chapter{Sample Projects}\label{sample-projects}

\noindent\hrulefill

\textbf{Note:} This section of articles does not provide documentation
for \emph{all} sample projects residing in the
\texttt{liferay-blade-samples} repo. The documentation for these samples
is in progress and will grow over time.

\noindent\hrulefill

Liferay provides sample projects that target different integration
points in Liferay DXP. These projects reside in the
\href{https://github.com/liferay/liferay-blade-samples}{liferay-blade-samples}
Github repository and can be easily copy/pasted to your local
environment. The sample projects are grouped into three different parent
folders based on the build tools used to generate them:

\begin{itemize}
\tightlist
\item
  \texttt{gradle}
\item
  \texttt{liferay-workspace}
\item
  \texttt{maven}
\end{itemize}

\noindent\hrulefill

\textbf{Note:} The Liferay Workspace folder stores WAR-type samples in a
separate folder named
\href{https://github.com/liferay/liferay-blade-samples/tree/7.1/liferay-workspace/wars}{wars}.
The Gradle and Maven tool folders mix WAR samples with the other sample
types (apps, extensions, etc.).

\noindent\hrulefill

For more information on these sample projects, visit the
\href{/docs/7-1/tutorials/-/knowledge_base/t/liferay-sample-projects}{Liferay
Sample Projects} tutorial.

\chapter{Apps}\label{apps}

This section focuses on Liferay sample applications. You can view these
sample apps by visiting the \texttt{apps} folder corresponding to your
preferred build tool:

\begin{itemize}
\tightlist
\item
  \href{https://github.com/liferay/liferay-blade-samples/tree/7.1/gradle/apps}{Gradle
  sample apps}
\item
  \href{https://github.com/liferay/liferay-blade-samples/tree/7.1/liferay-workspace/apps}{Liferay
  Workspace sample apps}
\item
  \href{https://github.com/liferay/liferay-blade-samples/tree/7.1/maven/apps}{Maven
  sample apps}
\end{itemize}

Visit a particular sample page to learn more!

\chapter{npm Samples}\label{npm-samples}

This section focuses on Liferay npm sample portlets built with Gradle.
You can view these samples by visiting the \texttt{apps/npm} folder
corresponding to your preferred build tool:

\begin{itemize}
\tightlist
\item
  \href{https://github.com/liferay/liferay-blade-samples/tree/7.1/gradle/apps/npm}{Gradle
  sample apps}
\item
  \href{https://github.com/liferay/liferay-blade-samples/tree/7.1/liferay-workspace/apps/npm}{Liferay
  Workspace sample apps}
\end{itemize}

\noindent\hrulefill

\textbf{Note:} When building the npm samples, an error can occur caused
by the limit of open files allowed by your operating system. Consult
your operating system vendor's documentation to learn how to configure
the maximum number of open files for your OS.

\noindent\hrulefill

The following npm samples are documented:

\begin{itemize}
\tightlist
\item
  \href{/docs/7-1/reference/-/knowledge_base/r/angular-npm-portlet}{Angular
  npm Portlet}
\item
  \href{/docs/7-1/reference/-/knowledge_base/r/angular-npm-deduplication-sample}{Angular
  npm Deduplication Portlet}
\item
  \href{/docs/7-1/reference/-/knowledge_base/r/billboard-js-npm-portlet}{Billboard.js
  npm Portlet}
\item
  \href{/docs/7-1/reference/-/knowledge_base/r/jquery-npm-portlet}{jQuery
  npm Portlet}
\item
  \href{/docs/7-1/reference/-/knowledge_base/r/metal-js-npm-portlet}{Metal.js
  npm Portlet}
\item
  \href{/docs/7-1/reference/-/knowledge_base/r/react-npm-portlet}{React
  npm Portlet}
\item
  \href{/docs/7-1/reference/-/knowledge_base/r/simple-npm-portlet}{Simple
  npm Portlet}
\item
  \href{/docs/7-1/reference/-/knowledge_base/r/vue-js-npm-portlet}{Vue.js
  npm Portlet}
\end{itemize}

Visit a particular sample page to learn more!

\section{Angular 6 npm Portlet}\label{angular-6-npm-portlet}

The Angular 6 npm Portlet sample provides a portlet that uses the
\href{https://angular.io/}{Angular} framework (version 6) to render its
output.

\begin{figure}
\centering
\includegraphics{./images/angular-6-npm-sample.png}
\caption{Type custom text in the field and watch it instantaneously
displayed in the portlet.}
\end{figure}

This portlet showcases Angular's newest version and how to leverage it
in Liferay DXP. See
\href{https://blog.angular.io/version-6-of-angular-now-available-cc56b0efa7a4}{this
article} for more information on what's new with Angular 6.

\subsection{What API(s) and/or code components does this sample
highlight?}\label{what-apis-andor-code-components-does-this-sample-highlight}

This sample leverages the
\href{/docs/7-0/tutorials/-/knowledge_base/t/using-npm-in-your-portlets}{npm
development workflow support}.

\subsection{How does this sample leverage the API(s) and/or code
component?}\label{how-does-this-sample-leverage-the-apis-andor-code-component}

This sample uses the \href{https://www.npmjs.com/}{npm registry} to
download project dependencies and uses the
\href{https://github.com/liferay/liferay-npm-build-tools/tree/master/packages/liferay-npm-bundler}{liferay-npm-bundler
tool} to bundle the project dependencies inside the OSGi bundle JAR
file.

To accomplish the bundling, the project's build process relies on a
\texttt{build} script inside its \texttt{package.json} file:

\begin{verbatim}
"scripts": {
    "build": "tsc && liferay-npm-bundler",
    "tsc": "tsc"
},
\end{verbatim}

\subsection{Where Is This Sample?}\label{where-is-this-sample}

This sample is built with the following build tool:

\begin{itemize}
\tightlist
\item
  \href{https://github.com/liferay/liferay-blade-samples/tree/7.1/gradle/apps/npm/angular-npm-portlet}{Gradle}
\end{itemize}

\section{Angular npm Deduplication
Sample}\label{angular-npm-deduplication-sample}

The Angular npm Deduplication sample provides a portlet that uses the
\href{https://angular.io/}{Angular} framework to render its output.

\begin{figure}
\centering
\includegraphics{./images/angular-npm-deduplication-sample.png}
\caption{Type custom text in the field and watch it instantaneously
displayed in the portlet..}
\end{figure}

This is done by providing a deduplicated instance of the Angular
framework as an OSGi bundle and then leveraging it from a sample
portlet.

\subsection{What API(s) and/or code components does this sample
highlight?}\label{what-apis-andor-code-components-does-this-sample-highlight-1}

This sample is broken into two modules:

\begin{itemize}
\tightlist
\item
  \texttt{angular-consumer-portlet}
\item
  \texttt{angular-provider}
\end{itemize}

The Angular Provider sample generates an OSGi bundle that provides a
deduplicated instance of the \href{https://angular.io/}{Angular}
framework that portlets can share when rendering their output. The
Angular Consumer portlet uses the deduplicated instance of the Angular
framework.

\textbf{Note:} Both modules must be deployed to the server for this
sample to run.

This sample leverages the
\href{/docs/7-1/tutorials/-/knowledge_base/t/using-npm-in-your-portlets}{npm
development workflow support}.

\subsection{How does this sample leverage the API(s) and/or code
component?}\label{how-does-this-sample-leverage-the-apis-andor-code-component-1}

This sample uses the \href{https://www.npmjs.com/}{npm registry} to
download project dependencies and uses the
\href{https://github.com/liferay/liferay-npm-build-tools/tree/master/packages/liferay-npm-bundler}{liferay-npm-bundler
tool} to bundle the project dependencies inside the OSGi bundle JAR
file.

To accomplish the bundling, the project's build process relies on a
\texttt{build} script inside its \texttt{package.json} file:

\begin{verbatim}
"scripts": {
    "build": "tsc && liferay-npm-bundler"
},
\end{verbatim}

\subsection{Where Is This Sample?}\label{where-is-this-sample-1}

This sample is built with the following build tool:

\begin{itemize}
\tightlist
\item
  \href{https://github.com/liferay/liferay-blade-samples/tree/7.1/gradle/apps/npm-deduplication-portlets/angular}{Gradle}
\end{itemize}

\section{Angular npm Portlet}\label{angular-npm-portlet}

The Angular npm Portlet sample provides a portlet that uses the
\href{https://angular.io/}{Angular} framework to render its output.

\begin{figure}
\centering
\includegraphics{./images/angular-npm-sample.png}
\caption{Type custom text in the field and watch it instantaneously
displayed in the portlet.}
\end{figure}

This portlet showcases Angular's speed and performance when rendering a
user interface.

\subsection{What API(s) and/or code components does this sample
highlight?}\label{what-apis-andor-code-components-does-this-sample-highlight-2}

This sample leverages the
\href{/docs/7-1/tutorials/-/knowledge_base/t/using-npm-in-your-portlets}{npm
development workflow support}.

\subsection{How does this sample leverage the API(s) and/or code
component?}\label{how-does-this-sample-leverage-the-apis-andor-code-component-2}

This sample uses the \href{https://www.npmjs.com/}{npm registry} to
download project dependencies and uses the
\href{https://github.com/liferay/liferay-npm-build-tools/tree/master/packages/liferay-npm-bundler}{liferay-npm-bundler
tool} to bundle the project dependencies inside the OSGi bundle JAR
file.

To accomplish the bundling, the project's build process relies on a
\texttt{build} script inside its \texttt{package.json} file:

\begin{verbatim}
"scripts": {
    "build": "tsc && liferay-npm-bundler"
},
\end{verbatim}

\subsection{Where Is This Sample?}\label{where-is-this-sample-2}

There are two different versions of this sample, each built with a
different build tool:

\begin{itemize}
\tightlist
\item
  \href{https://github.com/liferay/liferay-blade-samples/tree/7.1/gradle/apps/npm/angular-npm-portlet}{Gradle}
\item
  \href{https://github.com/liferay/liferay-blade-samples/tree/7.1/liferay-workspace/apps/npm/angular-npm-portlet}{Liferay
  Workspace}
\end{itemize}

\section{Billboard.js npm Portlet}\label{billboard.js-npm-portlet}

The Billboard.js npm Portlet sample provides a portlet that uses the
\href{https://naver.github.io/billboard.js/}{Billboard.js} framework to
render its output.

\begin{figure}
\centering
\includegraphics{./images/billboardjs-npm-sample.png}
\caption{The Billboard.js npm Portlet shows off some nice looking graphs
using Billboard.js.}
\end{figure}

This portlet showcases the power of graphing by displaying a set of
default charts and a more advanced custom chart. These are all built
using Billboard.js.

\subsection{What API(s) and/or code components does this sample
highlight?}\label{what-apis-andor-code-components-does-this-sample-highlight-3}

This sample leverages the
\href{/docs/7-1/tutorials/-/knowledge_base/t/using-npm-in-your-portlets}{npm
development workflow support}.

\subsection{How does this sample leverage the API(s) and/or code
component?}\label{how-does-this-sample-leverage-the-apis-andor-code-component-3}

This sample uses the \href{https://www.npmjs.com/}{npm registry} to
download project dependencies and uses the
\href{https://github.com/liferay/liferay-npm-build-tools/tree/master/packages/liferay-npm-bundler}{liferay-npm-bundler
tool} to bundle the project dependencies inside the OSGi bundle JAR
file.

To accomplish the bundling, the project's build process relies on a
\texttt{build} script inside its \texttt{package.json} file:

\begin{verbatim}
"scripts": {
    "build": "babel --source-maps -d build/resources/main/META-INF/resources src/main/resources/META-INF/resources && liferay-npm-bundler"
},
\end{verbatim}

\subsection{Where Is This Sample?}\label{where-is-this-sample-3}

There are two different versions of this sample, each built with a
different build tool:

\begin{itemize}
\tightlist
\item
  \href{https://github.com/liferay/liferay-blade-samples/tree/7.1/gradle/apps/npm/billboardjs-npm-portlet}{Gradle}
\item
  \href{https://github.com/liferay/liferay-blade-samples/tree/7.1/liferay-workspace/apps/npm/billboardjs-npm-portlet}{Liferay
  Workspace}
\end{itemize}

\section{jQuery npm Portlet}\label{jquery-npm-portlet}

The jQuery npm Portlet sample provides a portlet that uses the
\href{https://jquery.com/}{jQuery} framework to render its output.

\begin{figure}
\centering
\includegraphics{./images/jquery-npm-sample.png}
\caption{Clicking on the portlet's hand symbol displays a message.}
\end{figure}

This portlet showcases the fast HTML document traversal jQuery offers.

\subsection{What API(s) and/or code components does this sample
highlight?}\label{what-apis-andor-code-components-does-this-sample-highlight-4}

This sample leverages the
\href{/docs/7-1/tutorials/-/knowledge_base/t/using-npm-in-your-portlets}{npm
development workflow support}.

\subsection{How does this sample leverage the API(s) and/or code
component?}\label{how-does-this-sample-leverage-the-apis-andor-code-component-4}

This sample uses the \href{https://www.npmjs.com/}{npm registry} to
download project dependencies and uses the
\href{https://github.com/liferay/liferay-npm-build-tools/tree/master/packages/liferay-npm-bundler}{liferay-npm-bundler
tool} to bundle the project dependencies inside the OSGi bundle JAR
file.

To accomplish the bundling, the project's build process relies on a
\texttt{build} script inside its \texttt{package.json} file:

\begin{verbatim}
"scripts": {
    "build": "babel --source-maps -d build/resources/main/META-INF/resources src/main/resources/META-INF/resources && liferay-npm-bundler"
},
\end{verbatim}

\subsection{Where Is This Sample?}\label{where-is-this-sample-4}

There are two different versions of this sample, each built with a
different build tool:

\begin{itemize}
\tightlist
\item
  \href{https://github.com/liferay/liferay-blade-samples/tree/7.1/gradle/apps/npm/jquery-npm-portlet}{Gradle}
\item
  \href{https://github.com/liferay/liferay-blade-samples/tree/7.1/liferay-workspace/apps/npm/jquery-npm-portlet}{Liferay
  Workspace}
\end{itemize}

\section{Metal.js npm Portlet}\label{metal.js-npm-portlet}

The Metal.js npm Portlet sample provides a portlet that uses the
\href{https://metaljs.com/}{Metal.js} framework to render its output.

\begin{figure}
\centering
\includegraphics{./images/metaljs-npm-sample.png}
\caption{Clicking the button returns displays a dialog window.}
\end{figure}

This portlet displays a Metal.js based dialog that has been rendered
using SOY templates.

\subsection{What API(s) and/or code components does this sample
highlight?}\label{what-apis-andor-code-components-does-this-sample-highlight-5}

This sample leverages the
\href{/docs/7-1/tutorials/-/knowledge_base/t/using-npm-in-your-portlets}{npm
development workflow support}.

\subsection{How does this sample leverage the API(s) and/or code
component?}\label{how-does-this-sample-leverage-the-apis-andor-code-component-5}

This sample uses the \href{https://www.npmjs.com/}{npm registry} to
download project dependencies and uses the
\href{https://github.com/liferay/liferay-npm-build-tools/tree/master/packages/liferay-npm-bundler}{liferay-npm-bundler
tool} to bundle the project dependencies inside the OSGi bundle JAR
file.

To accomplish the bundling, the project's build process relies on a
\texttt{build} script inside its \texttt{package.json} file:

\begin{verbatim}
"scripts": {
    "build": "metalsoy && babel --source-maps -d build/resources/main/META-INF/resources src/main/resources/META-INF/resources && liferay-npm-bundler"
},
\end{verbatim}

\subsection{Where Is This Sample?}\label{where-is-this-sample-5}

There are two different versions of this sample, each built with a
different build tool:

\begin{itemize}
\tightlist
\item
  \href{https://github.com/liferay/liferay-blade-samples/tree/7.1/gradle/apps/npm/metaljs-npm-portlet}{Gradle}
\item
  \href{https://github.com/liferay/liferay-blade-samples/tree/7.1/liferay-workspace/apps/npm/metaljs-npm-portlet}{Liferay
  Workspace}
\end{itemize}

\section{React npm Portlet}\label{react-npm-portlet}

The React npm Portlet sample provides a portlet that uses the
\href{https://reactjs.org/}{React} framework to render its output.

\begin{figure}
\centering
\includegraphics{./images/react-npm-sample.png}
\caption{You can play the game Tic-tac-toe with this sample portlet.}
\end{figure}

This portlet showcases the how efficiently React can render components
based on user interaction.

\subsection{What API(s) and/or code components does this sample
highlight?}\label{what-apis-andor-code-components-does-this-sample-highlight-6}

This sample leverages the
\href{/docs/7-1/tutorials/-/knowledge_base/t/using-npm-in-your-portlets}{npm
development workflow support}.

\subsection{How does this sample leverage the API(s) and/or code
component?}\label{how-does-this-sample-leverage-the-apis-andor-code-component-6}

This sample uses the \href{https://www.npmjs.com/}{npm registry} to
download project dependencies and uses the
\href{https://github.com/liferay/liferay-npm-build-tools/tree/master/packages/liferay-npm-bundler}{liferay-npm-bundler
tool} to bundle the project dependencies inside the OSGi bundle JAR
file.

To accomplish the bundling, the project's build process relies on a
\texttt{build} script inside its \texttt{package.json} file:

\begin{verbatim}
"scripts": {
    "build": "babel --source-maps -d build/resources/main/META-INF/resources src/main/resources/META-INF/resources && liferay-npm-bundler"
},
\end{verbatim}

\subsection{Where Is This Sample?}\label{where-is-this-sample-6}

There are two different versions of this sample, each built with a
different build tool:

\begin{itemize}
\tightlist
\item
  \href{https://github.com/liferay/liferay-blade-samples/tree/7.1/gradle/apps/npm/react-npm-portlet}{Gradle}
\item
  \href{https://github.com/liferay/liferay-blade-samples/tree/7.1/liferay-workspace/apps/npm/react-npm-portlet}{Liferay
  Workspace}
\end{itemize}

\section{Simple npm Portlet}\label{simple-npm-portlet}

The Simple npm Portlet sample provides a portlet that uses the
\href{https://www.npmjs.com/package/isarray}{isarray npm package} when
rendering its output.

\begin{figure}
\centering
\includegraphics{./images/simple-npm-sample.png}
\caption{The portlet's status and actions are displayed as output.}
\end{figure}

\subsection{What API(s) and/or code components does this sample
highlight?}\label{what-apis-andor-code-components-does-this-sample-highlight-7}

This sample leverages the
\href{/docs/7-1/tutorials/-/knowledge_base/t/using-npm-in-your-portlets}{npm
development workflow support}.

\subsection{How does this sample leverage the API(s) and/or code
component?}\label{how-does-this-sample-leverage-the-apis-andor-code-component-7}

This sample uses the \href{https://www.npmjs.com/}{npm registry} to
download project dependencies and uses the
\href{https://github.com/liferay/liferay-npm-build-tools/tree/master/packages/liferay-npm-bundler}{liferay-npm-bundler
tool} to bundle the project dependencies inside the OSGi bundle JAR
file.

To accomplish the bundling, the project's build process relies on a
\texttt{build} script inside its \texttt{package.json} file:

\begin{verbatim}
"scripts": {
    "build": "babel --source-maps -d build/resources/main/META-INF/resources src/main/resources/META-INF/resources && liferay-npm-bundler"
},
\end{verbatim}

\subsection{Where Is This Sample?}\label{where-is-this-sample-7}

There are two different versions of this sample, each built with a
different build tool:

\begin{itemize}
\tightlist
\item
  \href{https://github.com/liferay/liferay-blade-samples/tree/7.1/gradle/apps/npm/simple-npm-portlet}{Gradle}
\item
  \href{https://github.com/liferay/liferay-blade-samples/tree/7.1/liferay-workspace/apps/npm/simple-npm-portlet}{Liferay
  Workspace}
\end{itemize}

\section{Vue.js npm Portlet}\label{vue.js-npm-portlet}

The Vue.js npm Portlet sample provides a portlet that uses the
\href{https://vuejs.org/}{Vue.js} framework to render its output.

\begin{figure}
\centering
\includegraphics{./images/vuejs-npm-sample.png}
\caption{Clicking the portlet's button reverses the message.}
\end{figure}

This portlet showcases Vue.js's speed and performance when rendering a
user interface.

\subsection{What API(s) and/or code components does this sample
highlight?}\label{what-apis-andor-code-components-does-this-sample-highlight-8}

This sample leverages the
\href{/docs/7-1/tutorials/-/knowledge_base/t/using-npm-in-your-portlets}{npm
development workflow support}.

\subsection{How does this sample leverage the API(s) and/or code
component?}\label{how-does-this-sample-leverage-the-apis-andor-code-component-8}

This sample uses the \href{https://www.npmjs.com/}{npm registry} to
download project dependencies and uses the
\href{https://github.com/liferay/liferay-npm-build-tools/tree/master/packages/liferay-npm-bundler}{liferay-npm-bundler
tool} to bundle the project dependencies inside the OSGi bundle JAR
file.

To accomplish the bundling, the project's build process relies on a
\texttt{build} script inside its \texttt{package.json} file:

\begin{verbatim}
"scripts": {
    "build": "babel --source-maps -d build/resources/main/META-INF/resources src/main/resources/META-INF/resources && liferay-npm-bundler"
},
\end{verbatim}

\subsection{Where Is This Sample?}\label{where-is-this-sample-8}

There are two different versions of this sample, each built with a
different build tool:

\begin{itemize}
\tightlist
\item
  \href{https://github.com/liferay/liferay-blade-samples/tree/7.1/gradle/apps/npm/vuejs-npm-portlet}{Gradle}
\item
  \href{https://github.com/liferay/liferay-blade-samples/tree/7.1/liferay-workspace/apps/npm/vuejs-npm-portlet}{Liferay
  Workspace}
\end{itemize}

\chapter{Service Builder Samples}\label{service-builder-samples}

This section focuses on Liferay Service Builder sample projects built
with various build tools. You can view these samples by visiting the
\texttt{apps/service-builder} folder corresponding to your preferred
build tool:

\begin{itemize}
\tightlist
\item
  \href{https://github.com/liferay/liferay-blade-samples/tree/7.1/gradle/apps/service-builder}{Gradle
  Service Builder sample apps}
\item
  \href{https://github.com/liferay/liferay-blade-samples/tree/7.1/liferay-workspace/apps/service-builder}{Liferay
  Service Builder Workspace sample apps}
\item
  \href{https://github.com/liferay/liferay-blade-samples/tree/7.1/maven/apps/service-builder}{Maven
  Service Builder sample apps}
\end{itemize}

The following Service Builder samples are documented:

\begin{itemize}
\tightlist
\item
  \href{/docs/7-1/reference/-/knowledge_base/r/service-builder-application-demonstrating-actionable-dynamic-query}{Service
  Builder application demonstrating Actionable Dynamic Query}
\item
  \href{/docs/7-1/reference/-/knowledge_base/r/service-builder-application-using-external-database-via-jdbc}{Service
  Builder application with JDBC connection}
\item
  \href{/docs/7-1/reference/-/knowledge_base/r/service-builder-application-using-external-database-via-jndi}{Service
  Builder application with JNDI connection}
\end{itemize}

Visit a particular sample page to learn more!

\section{Service Builder Application Demonstrating Actionable Dynamic
Query}\label{service-builder-application-demonstrating-actionable-dynamic-query}

This sample is similar to the
\href{https://github.com/liferay/liferay-blade-samples/tree/7.1/gradle/apps/service-builder/basic}{\texttt{basic}
Service Builder sample}, which lets you perform CRUD (create, read,
update, delete) operations on service builder entities. The distinctive
feature of the Service Builder Actionable Dynamic Query (ADQ) sample is
that it also lets you perform a mass update on all existing service
builder entities.

\begin{figure}
\centering
\includegraphics{./images/adq-sample.png}
\caption{This sample provides options to add entities and perform a mass
update.}
\end{figure}

To see the ADQ Service Builder sample in action, complete the following
steps:

\begin{enumerate}
\def\labelenumi{\arabic{enumi}.}
\item
  Add the sample to a page by navigating to \emph{Add}
  (\includegraphics{./images/icon-control-menu-add.png}) →
  \emph{Widgets} → \emph{Sample} and dragging it to the page.
\item
  Select the app's \emph{Add} button and add an entity. Do this several
  times to create multiple entities.
\item
  Click the \emph{Mass Update} button and click \emph{Save} to invoke
  the update.

  \begin{figure}
  \centering
  \includegraphics{./images/adq-sample-mass-update.png}
  \caption{Clicking the \emph{Save} button executes the mass update.}
  \end{figure}

  After invoking the update, each entity's \texttt{field3} value (whose
  value is less than 100) is incremented.
\end{enumerate}

You've leveraged the actionable dynamic query API in your sample!

\subsection{What API(s) and/or code components does this sample
highlight?}\label{what-apis-andor-code-components-does-this-sample-highlight-9}

This sample demonstrates Liferay DXP's actionable dynamic query API.
Specifically, it demonstrates how to create an ADQ, add criteria to an
ADQ, specify an action for the ADQ, and execute the ADQ.

\subsection{How does this sample leverage the API(s) and/or code
component?}\label{how-does-this-sample-leverage-the-apis-andor-code-component-9}

An action request is sent to the \texttt{JSPPortlet} with a \texttt{cmd}
request parameter. When the \texttt{JSPPortlet}'s \texttt{processAction}
method processes the request, the value of the \texttt{cmd} parameter is
parsed and then the portlet's \texttt{massUpdate} method is invoked. The
\texttt{massUpdate} method, in turn, invokes the \texttt{massUpdate}
method defined in the \texttt{adq-service} module's
\texttt{BarLocalServiceImpl}. This is where the sample leverages the
actionable dynamic query API:

\begin{verbatim}
public void massUpdate() {
    ActionableDynamicQuery adq = getActionableDynamicQuery();

    adq.setAddCriteriaMethod(
        new ActionableDynamicQuery.AddCriteriaMethod() {

            @Override
            public void addCriteria(DynamicQuery dynamicQuery) {
                dynamicQuery.add(RestrictionsFactoryUtil.lt("field3", 100));
            }

        });

    adq.setPerformActionMethod(
        new ActionableDynamicQuery.PerformActionMethod<Bar>() {

            @Override
            public void performAction(Bar bar) {
                int field3 = bar.getField3();

                field3++;
                bar.setField3(field3);

                updateBar(bar);
            }

        });

    try {
        adq.performActions();
    }
    catch (Exception e) {
        e.printStackTrace();
    }
}
\end{verbatim}

For more information on the actionable dynamic query API, visit its
dedicated
\href{/docs/7-0/tutorials/-/knowledge_base/t/dynamic-query\#actionable-dynamic-queries}{tutorial}.

\section{Service Builder Application Using External Database via
JDBC}\label{service-builder-application-using-external-database-via-jdbc}

This sample demonstrates how to connect a Liferay Service Builder
application to an external database via a JDBC connection. Here, an
external database means any database other than Liferay DXP's database.
For this sample to work correctly, you must prepare such an external
database and configure Liferay DXP to use it. Follow the steps below to
make the required preparations before deploying the application.

\begin{enumerate}
\def\labelenumi{\arabic{enumi}.}
\item
  Create the external database to which your Service Builder application
  will connect. For example, create a MariaDB database called
  \texttt{external}. Add a table to this database called
  \texttt{country} with a \texttt{BIGINT} column called \texttt{Id} and
  a \texttt{VARCHAR(255)} column called \texttt{Name}. Add at least one
  record to this table. Here are the MariaDB commands to accomplish
  this:

\begin{verbatim}
create database external character set utf8;

use external;

create table country(id bigint not null primary key, name varchar(255));

insert into country(id, name) values(1, 'Australia');
\end{verbatim}

  Make sure that your database commands were successful: Running
  \texttt{select\ *\ from\ country;} should return the record you added.
\item
  Create a \texttt{portal-ext.properties} file in your Liferay DXP
  instance's \texttt{{[}LIFERAY\_HOME{]}} folder (this folder should be
  marked by the presence of a \texttt{.liferay-home} file). In your
  \texttt{portal-ext.properties} file, define the details of your JDBC
  data source connection:

\begin{verbatim}
jdbc.ext.driverClassName=org.mariadb.jdbc.Driver
jdbc.ext.password=userpassword
jdbc.ext.url=jdbc:mariadb://localhost/external?useUnicode=true&characterEncoding=UTF-8&useFastDateParsing=false
jdbc.ext.username=yourusername
\end{verbatim}

  Note that Liferay DXP's primary data source is specified by the
  \texttt{jdbc.default} prefix. These details are often specified in a
  \texttt{portal-setup-wizard.properties} file. Here, we've chosen to
  use the \texttt{jdbc.ext} prefix for our alternate data source.
\item
  Create a
  \texttt{com.liferay.blade.samples.jdbcservicebuilder.service-log4j-ext.xml}
  in your Liferay instance's \texttt{{[}LIFERAY\_HOME{]}/osgi/log4}
  folder. Create this folder if it doesn't yet exist. Add this content
  to the XML file that you created:

\begin{verbatim}
<?xml version="1.0"?>
<!DOCTYPE log4j:configuration SYSTEM "log4j.dtd">

<log4j:configuration xmlns:log4j="http://jakarta.apache.org/log4j/">
    <category name="com.liferay.blade.samples.jdbcservicebuilder.service.impl">
        <priority value="INFO" />
    </category>
</log4j:configuration>
\end{verbatim}

  This XML file defines the log level for the classes in the
  \texttt{com.liferay.blade.samples.jdbcservicebuilder.service.impl}
  package. The
  \texttt{com.liferay.blade.samples.jdbcservicebuilder.service.impl.CountryLocalServiceImpl}
  is the class that will produce log messages when the sample portlet is
  viewed.
\end{enumerate}

Now your sample is ready for deployment! Make sure to build and deploy
each of the three modules that comprise the sample application:

\begin{itemize}
\tightlist
\item
  \texttt{jdbc-api}
\item
  \texttt{jdbc-service}
\item
  \texttt{jdbc-web}
\end{itemize}

After these modules have been deployed, add the \texttt{-web} portlet to
a Liferay DXP page.

\begin{figure}
\centering
\includegraphics{./images/jdbc-sb-sample.png}
\caption{This sample prints out the values previously inputted into the
database.}
\end{figure}

A sample table is printed in the portlet's view, representing the info
inputted into the database.

\subsection{What API(s) and/or code components does this sample
highlight?}\label{what-apis-andor-code-components-does-this-sample-highlight-10}

The sample configures the data source using Spring Beans and
demonstrates two ways to access data from an external database defined
by a JDBC connection:

\begin{itemize}
\tightlist
\item
  extract data directly from the raw data source by explicitly
  specifying a SQL query.
\item
  read data using the helper methods that Service Builder generates in
  your application's persistence layer.
\end{itemize}

\subsection{How does this sample leverage the API(s) and/or code
component?}\label{how-does-this-sample-leverage-the-apis-andor-code-component-10}

Once you've added the \texttt{-web} portlet to a page, the
\texttt{CountryLocalService.useJDBC} method is invoked. This method
accesses the database defined by the JDBC connection you specified and
logs information about the rows in the \texttt{country} table to Liferay
DXP's log.

\subsubsection{Configuring the Data
Source}\label{configuring-the-data-source}

The \texttt{-service} module's
\texttt{src/main/resources/META-INF/spring/ext-spring.xml} file
configures the external data source connection and applies the alias
\texttt{extDataSource} to the data source. The \texttt{service.xml} file
\texttt{entity} element specifies the data source via the attribute
assignment \texttt{data-source="extDataSource"}. The
\texttt{ext-spring.xml} and \texttt{service.xml} files demonstrate the
configuration steps explained in
\href{/docs/7-1/tutorials/-/knowledge_base/t/connecting-service-builder-to-external-databases}{Connecting
the Data Source Using Spring Beans}.

\subsubsection{Accessing Data}\label{accessing-data}

The first way of accessing data from the external database is to extract
it directly from the raw data source by explicitly specifying a SQL
query. This technique is demonstrated by the
\texttt{CountryLocalServiceImpl.useJDBC} method. That method obtains the
Spring-defined data source that's injected into the
\texttt{countryPersistence} bean, opens a new connection, and reads data
from the data source. This is the technique used by the sample
application to write the data to Liferay DXP's log.

The second way of accessing data from the external database is to read
data using the helper methods that Service Builder generates in your
application's persistence layer. This technique is demonstrated by the
\texttt{UseJDBC.getCountries} method which first obtains an instance of
the \texttt{CountryLocalService} OSGi service and then invokes
\texttt{countryLocalService.getCountries}. The
\texttt{countryLocalService.getCountries} and
\texttt{countryLocalService.getCountriesCount} methods are two examples
of the persistence layer helper methods that Service Builder generates.
This is the technique used by the sample application to actually display
the data. The portlet's \texttt{view.jsp} uses the
\texttt{\textless{}search-container\textgreater{}} JSP tag to display a
list of results. The results are obtained by the
\texttt{UseJDBC.getCountries} method mentioned above.

\section{Service Builder Application Using External Database via
JNDI}\label{service-builder-application-using-external-database-via-jndi}

This sample demonstrates how to connect a Liferay Service Builder
application to an external database via a JNDI connection. Here, an
external database means any database other than Liferay DXP's database.
For this sample to work correctly, you must prepare such an external
database and configure Liferay DXP to use it. Follow the steps below to
make the required preparations before deploying the application.

\begin{enumerate}
\def\labelenumi{\arabic{enumi}.}
\item
  Create the external database to which your Service Builder application
  will connect. For example, create a MariaDB database called
  \texttt{external}. Add a table to this database called \texttt{region}
  with a \texttt{BIGINT} column called \texttt{Id} and a
  \texttt{VARCHAR(255)} column called \texttt{Name}. Add at least one
  record to this table. Here are the MariaDB commands to accomplish
  this:

\begin{verbatim}
create database external character set utf8;

use external;

create table region(id bigint not null primary key, name varchar(255));

insert into region(id, name) values(1, 'Tasmania');
\end{verbatim}

  Make sure that your database commands were successful: Running
  \texttt{select\ *\ from\ region;} should return the record you added.
\item
  Now you need to define a JNDI connection to your database. The way
  this is done depends on your application server. Here we demonstrate
  how to specify the JNDI connection for Tomcat. First, open your
  \texttt{{[}LIFERAY\_HOME{]}/tomcat-9.0.6/conf/server.xml} file and add
  this resource element inside of the
  \texttt{\textless{}GlobalNamingResources\textgreater{}} element:

\begin{verbatim}
<Resource
    name="jdbc/externalDataSource"
    auth="Container"
    type="javax.sql.DataSource"
    factory="org.apache.tomcat.jdbc.pool.DataSourceFactory"
    driverClassName="org.mariadb.jdbc.Driver"
    url="jdbc:mariadb://localhost/external"
    username="yourusername"
    password="yourpassword"
    maxActive="20"
    maxIdle="5"
    maxWait="10000"
/>
\end{verbatim}

  Replace the specified username and password with the correct values
  for your database.
\item
  Open your \texttt{{[}LIFERAY\_HOME{]}/tomcat-9.0.6/conf/context.xml}
  file and add this resource link element inside of the
  \texttt{\textless{}Context\textgreater{}} element:

\begin{verbatim}
<ResourceLink name="jdbc/externalDataSource" global="jdbc/externalDataSource" type="javax.sql.DataSource"/>
\end{verbatim}

  Now your data source is defined at Tomcat's scope.
\item
  Create a
  \texttt{com.liferay.blade.samples.jndiservicebuilder.service-log4j-ext.xml}
  in your Liferay DXP instance's \texttt{{[}LIFERAY\_HOME{]}/osgi/log4}
  folder. Create this folder if it doesn't yet exist. Add this content
  to the XML file that you created:

\begin{verbatim}
<?xml version="1.0"?>
<!DOCTYPE log4j:configuration SYSTEM "log4j.dtd">

<log4j:configuration xmlns:log4j="http://jakarta.apache.org/log4j/">
    <category name="com.liferay.blade.samples.jndiservicebuilder.service.impl">
        <priority value="INFO" />
    </category>
</log4j:configuration>
\end{verbatim}

  This XML file defines the log level for the classes in the
  \texttt{com.liferay.blade.samples.jndiservicebuilder.service.impl}
  package. The
  \texttt{com.liferay.blade.samples.jndiservicebuilder.service.impl.RegionLocalServiceImpl}
  is the class that will produce log messages when the sample portlet is
  viewed.
\end{enumerate}

Now your sample is ready for deployment! Make sure to build and deploy
each of the three modules that comprise the sample application:

\begin{itemize}
\tightlist
\item
  \texttt{jndi-api}
\item
  \texttt{jndi-service}
\item
  \texttt{jndi-web}
\end{itemize}

After these modules have been deployed, add the \texttt{jndi-web}
portlet to a Liferay DXP page.

\begin{figure}
\centering
\includegraphics{./images/jndi-sb-sample.png}
\caption{This sample prints out the values previously inputted into the
database.}
\end{figure}

A sample table is printed in the portlet's view, representing the info
inputted into the database.

\subsection{What API(s) and/or code components does this sample
highlight?}\label{what-apis-andor-code-components-does-this-sample-highlight-11}

This sample demonstrates two ways to access data from an external
database defined by a JNDI connection:

\begin{itemize}
\tightlist
\item
  extract data directly from the raw data source by explicitly
  specifying a SQL query.
\item
  read data using the helper methods that Service Builder generates in
  your application's persistence layer.
\end{itemize}

\subsection{How does this sample leverage the API(s) and/or code
component?}\label{how-does-this-sample-leverage-the-apis-andor-code-component-11}

Once you've added the \texttt{jndi-web} portlet to a page, the
\texttt{RegionLocalServiceUtil.useJNDI} method is invoked. This method
accesses the database defined by the JNDI connection you specified and
logs information about the rows in the \texttt{region} table to Liferay
DXP's log.

The first way of accessing data from the external database is to extract
data directly from the raw data source by explicitly specifying a SQL
query. This technique is demonstrated by the
\texttt{RegionLocalServiceImpl.useJNDI} method. That method obtains the
Spring-defined data source that's injected into the
\texttt{regionPersistence} bean, opens a new connection, and reads data
from the data source. This is the technique used by the sample
application to write the data to Liferay DXP's log.

The second way of accessing data from the external database is to read
data using the helper methods that Service Builder generates in your
application's persistence layer. This technique is demonstrated by the
\texttt{UseJNDI.getRegions} method which first obtains an instance of
the \texttt{RegionLocalService} OSGi service and then invokes
\texttt{regionLocalService.getRegions}. The
\texttt{regionLocalService.getRegions} and
\texttt{regionLocalService.getRegionsCount} methods are two examples of
the persistence layer helper methods that Service Builder generates.
This is the technique used by the sample application to actually display
the data. The portlet's \texttt{view.jsp} uses the
\texttt{\textless{}search-container\textgreater{}} JSP tag to display a
list of results. The results are obtained by the
\texttt{UseJNDI.getRegions} method mentioned above.

\section{Greedy Policy Option
Application}\label{greedy-policy-option-application}

The Greedy Policy Option sample provides two portlets that can be added
to a Liferay DXP page: Greedy Portlet and Reluctant Portlet.

\begin{figure}
\centering
\includegraphics{./images/greedy-policy-app.png}
\caption{The Greedy Policy Option app provides two portlets that only
print text. You'll dive deeper later to discover their interesting
capabilities involving services.}
\end{figure}

These two portlets do not provide anything useful out-of-the-box. They
are, however, very effective at demonstrating the ability to reference
services using greedy and reluctant policy options. You'll learn how to
do this later.

\subsection{What API(s) and/or code components does this sample
highlight?}\label{what-apis-andor-code-components-does-this-sample-highlight-12}

This sample provides two modules referencing services using greedy and
reluctant policy options.

\begin{itemize}
\item
  \texttt{service-reference}: Provides an OSGi service interface called
  \texttt{SomeService}, a default implementation of it, and portlets
  that refer to service instances. One portlet refers to new higher
  ranked instances of the service automatically. The other portlet is
  reluctant to use new higher ranked instances and continues to use its
  bound service. The reluctant portlet can, however, be configured
  dynamically to use other service instances.
\item
  \texttt{higher-ranked-service}: Has a higher ranked
  \texttt{SomeService} implementation.
\end{itemize}

Here are each module's file structures:

\begin{itemize}
\tightlist
\item
  \texttt{service-reference/}

  \begin{itemize}
  \tightlist
  \item
    \texttt{bnd.bnd}
  \item
    \texttt{configs/}

    \begin{itemize}
    \tightlist
    \item
      \texttt{com.liferay.blade.reluctant.vs.greedy.portlet.portlet.ReluctantPortlet.config}
      → \texttt{ReluctantPortlet} configuration file
    \end{itemize}
  \item
    \texttt{src/main/java/com/liferay/blade/reluctant/vs/greedy/portlet/}

    \begin{itemize}
    \tightlist
    \item
      \texttt{api/}

      \begin{itemize}
      \tightlist
      \item
        \texttt{SomeService.java} → Service interface
      \end{itemize}
    \item
      \texttt{constants/}

      \begin{itemize}
      \tightlist
      \item
        \texttt{ReluctantPortletVsGreedyPortletKeys.java} → Portlet
        constants
      \end{itemize}
    \item
      \texttt{portlet/}

      \begin{itemize}
      \tightlist
      \item
        \texttt{DefaultSomeService.java} → Zero ranked service
        implementation
      \item
        \texttt{GreedyPortlet.java} → Refers to \texttt{SomeService}
        using a greedy service policy option
      \item
        \texttt{ReluctantPortletPortlet.java} → Refers to
        \texttt{SomeService} using a reluctant service policy option by
        default.
      \end{itemize}
    \end{itemize}
  \end{itemize}
\item
  \texttt{higher-ranked-service/}

  \begin{itemize}
  \tightlist
  \item
    \texttt{bnd.bnd}
  \item
    \texttt{src/main/java/com/liferay/blade/reluctant/vs/greedy/svc/HigherRankedService.java}
    → Service implementation with service ranking value of \texttt{100}
  \end{itemize}
\end{itemize}

\subsection{How does this sample leverage the API(s) and/or code
component?}\label{how-does-this-sample-leverage-the-apis-andor-code-component-12}

Here are the things you can learn using the sample modules:

\begin{enumerate}
\def\labelenumi{\arabic{enumi}.}
\item
  \hyperref[binding-a-newly-deployed-components-service-reference-to-the-highest-ranking-service-instance-thats-available-initially]{Binding
  a component's service reference to the highest ranked service instance
  that's available initially.}
\item
  \hyperref[deploying-a-module-with-a-higher-ranked-service-instance-for-binding-to-greedy-references-immediately]{Deploying
  a module with a higher ranked service instance for binding to greedy
  references immediately.}
\item
  \hyperref[configuring-a-component-to-reference-a-different-service-instance-dynamically]{Configuring
  a component to reference a different service instance dynamically.}
\end{enumerate}

Let's walk through the demonstration.

\subsubsection{Binding a newly deployed component's service reference to
the highest ranking service instance that's available
initially}\label{binding-a-newly-deployed-components-service-reference-to-the-highest-ranking-service-instance-thats-available-initially}

On deploying a component that references a service, it binds to the
highest ranking service instance that matches its target filter (if
specified).

The portlet classes refer to instances of interface
\texttt{SomeService}. The \texttt{doSomething} method returns a
\texttt{String}.

\begin{verbatim}
public interface SomeService {

    public String doSomething();

}
\end{verbatim}

Class \texttt{DefaultSomeService} implements \texttt{SomeService}. Its
\texttt{doSomething} method returns the \texttt{String} ``I am
Default!''.

\begin{verbatim}
@Component
public class DefaultSomeService implements SomeService {

    @Override
    public String doSomething() {
        return "I am Default!";
    }

}
\end{verbatim}

When module's portlets refer to \texttt{DefaultSomeService}, they
display the \texttt{String} ``I am Default!''.

The \texttt{ReluctantPortlet} class's \texttt{SomeService} reference's
policy option is the default: static and reluctant. This policy option
keeps the reference bound to its current service instance unless that
instance stops or the reference is reconfigured to refer to a different
service instance.

\begin{verbatim}
@Component(
   immediate = true,
   property = {
       "com.liferay.portlet.display-category=category.sample",
       "com.liferay.portlet.instanceable=true",
       "javax.portlet.display-name=Reluctant Portlet",
       "javax.portlet.init-param.template-path=/",
       "javax.portlet.init-param.view-template=/view.jsp",
       "javax.portlet.name=" + ReluctantVsGreedyPortletKeys.Reluctant,
       "javax.portlet.resource-bundle=content.Language",
       "javax.portlet.security-role-ref=power-user,user"
   },
   service = Portlet.class
)
public class ReluctantPortlet extends MVCPortlet {

   @Override
   public void doView(
           RenderRequest renderRequest, RenderResponse renderResponse)
       throws IOException, PortletException {

       renderRequest.setAttribute("doSomething", _someService.doSomething());

       super.doView(renderRequest, renderResponse);
   }

   @Reference
   private SomeService _someService;

}
\end{verbatim}

The \texttt{ReluctantPortlet}'s method \texttt{doView} sets render
request attribute \texttt{doSomething} to the value returned from the
\texttt{SomeService} instance's \texttt{doSomething} method (e.g.,
\texttt{DefaultService} returns ``I am default!'').

The \texttt{GreedyPortlet} class is similar to
\texttt{ReluctantPortlet}, except its \texttt{SomeService} reference's
policy option is static and greedy (i.e.,
\texttt{ReferencePolicyOption.GREEDY}).

\begin{verbatim}
public class GreedyPortlet extends MVCPortlet {

    @Override
    public void doView(
            RenderRequest renderRequest, RenderResponse renderResponse)
        throws IOException, PortletException {

        renderRequest.setAttribute("doSomething", _someService.doSomething());

        super.doView(renderRequest, renderResponse);
    }

    @Reference (policyOption = ReferencePolicyOption.GREEDY)
    private SomeService _someService;

}
\end{verbatim}

The greedy policy option lets the component switch to using a higher
ranked \texttt{SomeService} instance if one becomes active in the
system. The section
\hyperref[deploying-a-module-with-a-higher-ranked-service-instance-for-binding-to-greedy-references-immediately]{\emph{Deploying
a module with a higher ranked service instance for binding to greedy
references immediately}} demonstrates this portlet switching to a higher
ranked service.

It's time to see this module's portlets and service in action.

\begin{enumerate}
\def\labelenumi{\arabic{enumi}.}
\item
  Stop module \texttt{higher-ranked-service} if it's active.
\item
  Deploy the \texttt{service-reference} module.
\item
  Add the \emph{Reluctant Portlet} from the \emph{Add} →
  \emph{Applications} → \emph{Sample} category to a site page.

  The portlet displays the message ``SomeService says I am
  default!''--whose latter part comes from the render request attribute
  set by the \texttt{DefaultService} instance.

  \begin{figure}
  \centering
  \includegraphics{./images/reluctant-portlet-using-default.png}
  \caption{\emph{Reluctant Portlet} displays the message ``SomeService
  says I am default!''}
  \end{figure}
\item
  Add the \emph{Greedy Portlet} from the \emph{Add} →
  \emph{Applications} → \emph{Sample} category to a site page.

  The portlet displays the message ``SomeService says I am better, use
  me!''. Both portlets are referencing a \texttt{DefaultService}
  instance.

  \begin{figure}
  \centering
  \includegraphics{./images/greedy-portlet-using-default.png}
  \caption{\emph{Greedy Portlet} displays the message ``SomeService says
  I am better, use me!''}
  \end{figure}
\end{enumerate}

Since \texttt{DefaultService} is the only active \texttt{SomeService}
instance in the system, the portlets refer to it for their
\texttt{SomeService} fields.

The \texttt{DefaultService} and portlets \emph{Reluctant Portlet} and
\emph{Greedy Portlet} are active. Let's activate a higher ranked
\texttt{SomeService} instance and see how the portlets react to it.

\subsubsection{Deploying a module with a higher ranked service instance
for binding to greedy references
immediately}\label{deploying-a-module-with-a-higher-ranked-service-instance-for-binding-to-greedy-references-immediately}

Module \texttt{higher-ranked-service} provides a \texttt{SomeService}
implementation called \texttt{HigherRankedService}.
\texttt{HigherRankedService}'s service ranking is \texttt{100}--that's
\texttt{100} more than \texttt{DefaultService}'s ranking \texttt{0}. Its
\texttt{doSomething} method returns the \texttt{String} ``I am better,
use me!''.

\begin{enumerate}
\def\labelenumi{\arabic{enumi}.}
\tightlist
\item
  Deploy the \texttt{higher-ranked-service} module.
\item
  Refresh your page that has the portlets \emph{Reluctant Portlet} and
  \emph{Greedy Portlet}.
\end{enumerate}

\emph{Reluctant Portlet} continues displaying message ``SomeService says
I am better, use me!''. It's ``reluctant'' to unbind from the
\texttt{DefaultService} instance and bind to the newly activated
\texttt{HigherRankedService} service.

\emph{Greedy Portlet} displays a new message ``SomeService says I am
better, use me!''. The part of the message ``I am better, use me!''
comes from the \texttt{HigherRankedService} instance to which it refers.

\begin{figure}
\centering
\includegraphics{./images/greedy-portlet-using-higher-ranked-service.png}
\caption{The \emph{Greedy Portlet} is using a
\texttt{HigherRankedService} instance}
\end{figure}

Next, learn how to bind the \emph{Reluctant Portlet} to a
\texttt{HigherRankedService} instance.

\subsubsection{Configuring a component to reference a different service
instance
dynamically}\label{configuring-a-component-to-reference-a-different-service-instance-dynamically}

The \emph{Reluctant Portlet} is currently bound to a
\texttt{DefaultService} instance. It's ``reluctant'' to unbind from it
and bind to a different service. OSGi Configuration Administration lets
you reconfigure service references to filter on and bind to different
service instances.

The \texttt{service-reference} module's configuration files and
\texttt{com.liferay.blade.reluctant.vs.greedy.portlet.portlet.ReluctantPortlet.config}
and
\texttt{com.liferay.blade.reluctant.vs.greedy.portlet.portlet.ReluctantPortlet.cfg}
configure the \texttt{ReluctantPortlet} component to use a
\texttt{HigherRankedService} instance.

\begin{verbatim}
_someService.target=(component.name=com.liferay.blade.reluctant.vs.greedy.service.HigherRankedService)
\end{verbatim}

The service configuration filters on a service whose
\texttt{component.name} is
\texttt{com.liferay.blade.reluctant.vs.greedy.service.HigherRankedService}.

Note: For deploying to 7.0, use file with suffix \texttt{.config}. For
earlier versions (i.e., Liferay DXP 7.0 Fix Packs earlier than Fix Pack
8 and Liferay CE Portal 7.0 GA3 or earlier), use the file with suffix
\texttt{.cfg}.

Here are the steps to reconfigure \texttt{ReluctantPortlet} to use
\texttt{HigherRankedService}:

\begin{enumerate}
\def\labelenumi{\arabic{enumi}.}
\tightlist
\item
  Copy the configuration file to
  \texttt{{[}Liferay-Home{]}/osgi/configs}.
\item
  Refresh your browser.
\end{enumerate}

\emph{Reluctant Portlet} displays a new message ``SomeService says I am
better, use me!''.

\begin{figure}
\centering
\includegraphics{./images/reluctant-portlet-using-higher-ranked-service.png}
\caption{\emph{Reluctant Portlet} is using the
\texttt{HigherRankedService} instance instead of a
\texttt{DefaultService} instance.}
\end{figure}

\emph{Reluctant Portlet} is using \texttt{HigherRankedService} instance
instead of a \texttt{DefaultService} instance. You've configured
\emph{Reluctant Portlet} to use a \texttt{HigherRankedService} instance!

\subsection{Where Is This Sample?}\label{where-is-this-sample-9}

There are three different versions of this sample, each built with a
different build tool:

\begin{itemize}
\tightlist
\item
  \href{https://github.com/liferay/liferay-blade-samples/tree/7.1/gradle/apps/greedy-policy-option-portlet}{Gradle}
\item
  \href{https://github.com/liferay/liferay-blade-samples/tree/7.1/liferay-workspace/apps/greedy-policy-option-portlet}{Liferay
  Workspace}
\item
  \href{https://github.com/liferay/liferay-blade-samples/tree/7.1/maven/apps/greedy-policy-option-portlet}{Maven}
\end{itemize}

\section{Kotlin Portlet}\label{kotlin-portlet}

The Kotlin Portlet sample provides an input form that accepts a name.
Once submitting a name, the portlet renders a greeting message.

\begin{figure}
\centering
\includegraphics{./images/kotlin-portlet.png}
\caption{After saving the inputted name, it's displayed as a greeting on
the portlet page.}
\end{figure}

\subsection{What API(s) and/or code components does this sample
highlight?}\label{what-apis-andor-code-components-does-this-sample-highlight-13}

This sample highlights the use of the
\href{https://kotlinlang.org/}{Kotlin} programming language in
conjunction with Liferay's MVC framework. Specifically, this sample
leverages the
\href{https://docs.liferay.com/dxp/portal/7.1-latest/javadocs/portal-kernel/com/liferay/portal/kernel/portlet/bridges/mvc/MVCActionCommand.html}{MVCActionCommand}
interface.

\subsection{How does this sample leverage the API(s) and/or code
component?}\label{how-does-this-sample-leverage-the-apis-andor-code-component-13}

This sample uses the
\href{/docs/7-1/tutorials/-/knowledge_base/t/mvc-action-command}{MVC
Action Command}'s \texttt{processAction(...)} method to process the
inputted text (i.e., name). The text is set as an attribute in the
\texttt{KotlinGreeterActionCommandKt.kt} class using an
\texttt{ActionRequest} and then is retrieved in the JSP using a
\texttt{RenderRequest}.

\subsection{Where Is This Sample?}\label{where-is-this-sample-10}

This sample is built with the following build tools:

\begin{itemize}
\tightlist
\item
  \href{https://github.com/liferay/liferay-blade-samples/tree/7.1/gradle/apps/kotlin-portlet}{Gradle}
\item
  \href{https://github.com/liferay/liferay-blade-samples/tree/7.1/liferay-workspace/apps/kotlin-portlet}{Liferay
  Workspace}
\end{itemize}

\section{Shared Language Keys}\label{shared-language-keys}

The Shared Language Keys sample provides a JSP portlet that displays
language keys.

\begin{figure}
\centering
\includegraphics{./images/language-web-portlet.png}
\caption{The sample JSP portlet displays three language keys.}
\end{figure}

The language keys displayed in the portlet come from two different
modules.

\subsection{What API(s) and/or code components does this sample
highlight?}\label{what-apis-andor-code-components-does-this-sample-highlight-14}

This sample is broken into two modules:

\begin{itemize}
\tightlist
\item
  \texttt{language}
\item
  \texttt{language-web}
\end{itemize}

The \texttt{language-web} module provides a JSP portlet with unique
language keys that it displays. The \texttt{language} module provides a
resource module which only holds language keys. Its sole purpose is to
share language keys with the JSP portlet provided in
\texttt{language-web}. This sample conveys Liferay's recommended
approach to sharing language keys through OSGi services.

\subsection{How does this sample leverage the API(s) and/or code
component?}\label{how-does-this-sample-leverage-the-apis-andor-code-component-14}

You must deploy both \texttt{language-web} and \texttt{language} modules
to simulate this sample's targeted demonstration.

First, note the language keys provided by each module:

\begin{itemize}
\tightlist
\item
  \texttt{language-web}

  \begin{itemize}
  \tightlist
  \item
    \texttt{blade\_language\_web\_LanguageWebPortlet.caption=Hello\ from\ BLADE\ Language\ Web!}
  \item
    \texttt{blade\_language\_web\_override\_LanguageWebPortlet.caption=I\ have\ overridden\ the\ key\ from\ BLADE\ Language\ Module!}
  \end{itemize}
\item
  \texttt{language}

  \begin{itemize}
  \tightlist
  \item
    \texttt{blade\_language\_LanguageWebPortlet.caption=Hello\ from\ the\ BLADE\ Language\ Module!}
  \item
    \texttt{blade\_language\_web\_override\_LanguageWebPortlet.caption=Hello\ from\ the\ BLADE\ Language\ Module\ but\ you\ won\textquotesingle{}t\ see\ me!}
  \end{itemize}
\end{itemize}

When you place the sample BLADE Language Web portlet on a Liferay DXP
page, you're presented with three language keys:

\begin{figure}
\centering
\includegraphics{./images/shared-language-keys.png}
\caption{The Language Web portlet displays three phrases, two of which
are shared from a different module.}
\end{figure}

The first message is provided by the \texttt{language-web} module. The
second message is from the \texttt{language} module. The third message
is provided by both modules; as you can see, the \texttt{language-web}'s
message is used, overriding the \texttt{language} module's identically
named language key.

This sample shows what takes precedence when displaying language keys.
The order for this example goes

\begin{enumerate}
\def\labelenumi{\arabic{enumi}.}
\tightlist
\item
  \texttt{language-web} module language keys
\item
  \texttt{language} module language keys
\item
  Liferay DXP language keys
\end{enumerate}

So how does sharing language keys work?

By default, the \texttt{ResourceBundleLoaderAnalyzerPlugin} expands
modules with \texttt{/content/Language.properties} files to add provided
capabilities:

\begin{itemize}
\tightlist
\item
  \texttt{bundle.symbolic.name}
\item
  \texttt{resource.bundle.base.name}
\end{itemize}

Then the deployed \texttt{LanguageExtender} scans modules with those
capabilities to automatically register an associated
\texttt{ResourceBundleLoader}.

You can leverage this functionality to use keys from common language
modules by republishing an aggregate \texttt{ResourceBundleLoader}. This
can be done two ways:

\begin{enumerate}
\def\labelenumi{\arabic{enumi}.}
\item
  Via Components

  You can get a reference to the registered service in your components
  as detailed in the
  \href{/docs/7-1/tutorials/-/knowledge_base/t/overriding-a-modules-language-keys}{Overriding
  a Module's Language Keys} tutorial. The main disadvantage of this
  approach is that it forces you to provide a specific implementation of
  the \texttt{ResourceBundleLoader}, making it harder to modularize in
  the future.
\item
  Via Provide Capability

  The same \texttt{LanguageExtender} that registers the services
  supports an extended syntax that lets you register an aggregate of a
  collection of bundles:

\begin{verbatim}
 -liferay-aggregate-resource-bundles: \
     blade.language
\end{verbatim}

  This approach has the advantage of easier extensibility. When language
  keys change, only the common language modules must be built and
  redeployed for the modules referencing them to recognize their
  updates.
\end{enumerate}

For more information on sharing language keys, visit the
\href{/docs/7-1/tutorials/-/knowledge_base/t/internationalization}{Internationalization}
tutorials.

\subsection{Where Is This Sample?}\label{where-is-this-sample-11}

There are three different versions of this sample, each built with a
different build tool:

\begin{itemize}
\tightlist
\item
  \href{https://github.com/liferay/liferay-blade-samples/tree/7.1/gradle/apps/shared-language-keys}{Gradle}
\item
  \href{https://github.com/liferay/liferay-blade-samples/tree/7.1/liferay-workspace/apps/shared-language-keys}{Liferay
  Workspace}
\item
  \href{https://github.com/liferay/liferay-blade-samples/tree/7.1/maven/apps/shared-language-keys}{Maven}
\end{itemize}

\section{Simulation Panel App}\label{simulation-panel-app}

The Simulation Panel App provides new functionality in Liferay DXP's
Simulation Menu. When deploying this sample with no customizations, the
\emph{Simulation Sample} feature is provided in the Simulation Menu with
four options.

\subsection{What API(s) and/or code components does this sample
highlight?}\label{what-apis-andor-code-components-does-this-sample-highlight-15}

This sample leverages the
\href{https://docs.liferay.com/dxp/apps/web-experience/latest/javadocs/com/liferay/application/list/PanelApp.html}{PanelApp}
API.

\subsection{How does this sample leverage the API(s) and/or code
component?}\label{how-does-this-sample-leverage-the-apis-andor-code-component-15}

This sample leverages the \texttt{PanelApp} interface as an OSGi service
via the \texttt{@Component} annotation:

\begin{verbatim}
@Component(
    immediate = true,
    property = {
        "panel.app.order:Integer=500",
        "panel.category.key=" + SimulationPanelCategory.SIMULATION
    },
    service = PanelApp.class
)
\end{verbatim}

There are also two properties provided via the \texttt{@Component}
annotation:

\begin{itemize}
\tightlist
\item
  \texttt{panel.app.order}: the order in which the panel app is
  displayed among other panel apps in the chosen category. Entries are
  ordered from top to bottom. For example, an entry with order
  \texttt{1} will be listed above an entry with order \texttt{2}. If the
  order is not specified, it's chosen at random based on which service
  was registered first in the OSGi container.
\item
  \texttt{panel.category.key}: the host panel category for your panel
  app, which should be the Simulation Menu category.
\end{itemize}

The simulation panel app extends the
\href{https://docs.liferay.com/ce/apps/web-experience/latest/javadocs/com/liferay/application/list/BaseJSPPanelApp.html}{BaseJSPPanelApp},
which provides a skeletal implementation of the
\href{https://docs.liferay.com/ce/apps/web-experience/latest/javadocs/com/liferay/application/list/PanelApp.html}{PanelApp}
interface with JSP support. JSPs, however, are not the only way to
provide frontend functionality to your panel categories/apps. You can
create your own class implementing \texttt{PanelApp} to use other
technologies, such as FreeMarker.

To learn more about Liferay Portal's product navigation using panel
categories and panel apps, see the
\href{/docs/7-1/tutorials/-/knowledge_base/t/customizing-the-product-menu}{Customizing
the Product Menu} tutorial. For more information on extending the
Simulation Menu, see the
\href{/docs/7-1/tutorials/-/knowledge_base/t/extending-the-simulation-menu}{Extending
the Simulation Menu} tutorial.

\subsection{Where Is This Sample?}\label{where-is-this-sample-12}

There are three different versions of this sample, each built with a
different build tool:

\begin{itemize}
\tightlist
\item
  \href{https://github.com/liferay/liferay-blade-samples/tree/7.1/gradle/apps/simulation-panel-app}{Gradle}
\item
  \href{https://github.com/liferay/liferay-blade-samples/tree/7.1/liferay-workspace/apps/simulation-panel-app}{Liferay
  Workspace}
\item
  \href{https://github.com/liferay/liferay-blade-samples/tree/7.1/maven/apps/simulation-panel-app}{Maven}
\end{itemize}

\section{Spring MVC Portlet}\label{spring-mvc-portlet}

The Spring MVC portlet provides a way to add various different fields
into the database and display them in a table. This project is a Spring
MVC based portlet WAR that implements the same functionality as the
\texttt{apps/service-builder/basic-web} sample project. It manages JSP
pages for display, uses a Spring-annotated portlet class, and invokes
the \texttt{apps/service-builder/basic-api} module to call services.

\noindent\hrulefill

\textbf{Note:} If you're planning to package this sample using Maven,
you must complete a few additional steps to avoid build errors. This
sample relies on the \texttt{service-builder/basic-api} module. Since
the \texttt{basic-api} bundle is not available on Liferay's CDN repo or
Maven Central, this sample can not reference it, resulting in build
failures.

To satisfy this dependency, you must install the bundle dependency to
your local \texttt{\textasciitilde{}/.m2\ repo}, along with the parent
BND plugin and root Maven project. Here are the steps to accomplish
this:

\begin{enumerate}
\def\labelenumi{\arabic{enumi}.}
\tightlist
\item
  Run \texttt{mvn\ clean\ install} on
  \texttt{maven/apps/service-builder/basic-api}.
\item
  Run \texttt{mvn\ clean\ install} on
  \texttt{maven/parent.bnd.bundle.plugin}.
\item
  Run \texttt{mvn\ clean\ install\ -N} in the root
  \texttt{liferay-blade-samples/maven} folder.
\end{enumerate}

Now you can build this sample successfully.

\noindent\hrulefill

\begin{figure}
\centering
\includegraphics{./images/spring-mvc-portlet.png}
\caption{Click \emph{Add} and fill out the sample fields to generate a
custom entry in the portlet's table.}
\end{figure}

Unlike the \texttt{service-builder/basic-web} module, Spring MVC
portlets must be delivered as portlet WAR projects. This project builds
to a WAR file but leverages all of the Liferay Workspace tools and
Gradle to build the WAR. You must build and deploy the
\texttt{service-builder/basic-api} and
\texttt{service-builder/basic-service} modules for this sample to work
properly. For more information on using Spring MVC portlets in Liferay
DXP, visit the
\href{/docs/7-1/tutorials/-/knowledge_base/t/spring-mvc}{Spring MVC}
tutorial.

\subsection{What API(s) and/or code components does this sample
highlight?}\label{what-apis-andor-code-components-does-this-sample-highlight-16}

This sample demonstrates a Liferay DXP portlet built using the
\href{https://docs.spring.io/spring/docs/current/spring-framework-reference/html/mvc.html}{Spring
Web MVC framework}.

\subsection{How does this sample leverage the API(s) and/or code
component?}\label{how-does-this-sample-leverage-the-apis-andor-code-component-16}

You can easily modify this sample by customizing its
\texttt{SpringMVCPortletViewController} Java class or any of its JSPs
stored in the \texttt{src/main/webapp/WEB-INF/jsp} folder. For more
information on customizing this sample, see the Javadoc listed in this
sample's \texttt{SpringMVCPortletViewController} Java class.

\subsection{Where Is This Sample?}\label{where-is-this-sample-13}

There are three different versions of this sample, each built with a
different build tool:

\begin{itemize}
\tightlist
\item
  \href{https://github.com/liferay/liferay-blade-samples/tree/7.1/gradle/apps/springmvc-portlet}{Gradle}
\item
  \href{https://github.com/liferay/liferay-blade-samples/tree/7.1/liferay-workspace/wars/springmvc-portlet}{Liferay
  Workspace}
\item
  \href{https://github.com/liferay/liferay-blade-samples/tree/7.1/maven/apps/springmvc-portlet}{Maven}
\end{itemize}

\chapter{Extensions}\label{extensions}

This section focuses on Liferay sample extensions. You can view these
sample extensions by visiting the \texttt{extensions} folder
corresponding to your preferred build tool:

\begin{itemize}
\tightlist
\item
  \href{https://github.com/liferay/liferay-blade-samples/tree/7.1/gradle/extensions}{Gradle
  sample extensions}
\item
  \href{https://github.com/liferay/liferay-blade-samples/tree/7.1/liferay-workspace/extensions}{Liferay
  Workspace sample extensions}
\item
  \href{https://github.com/liferay/liferay-blade-samples/tree/7.1/maven/extensions}{Maven
  sample extensions}
\end{itemize}

Visit a particular sample page to learn more!

\section{Control Menu Entry}\label{control-menu-entry}

The Control Menu Entry sample provides a customizable button that is
added to Liferay Portal's default Control Menu. When deploying this
sample with no customizations, an additional button is added to the User
(right side) portion of the Control Menu.

\begin{figure}
\centering
\includegraphics{./images/controlmenuentry.png}
\caption{The User area of the Control Menu is provided an additional
link button when the Control Menu Entry sample is deployed to Liferay
DXP.}
\end{figure}

The button navigates the user to Liferay's website:
https://www.liferay.com.

\subsection{What API(s) and/or code components does this sample
highlight?}\label{what-apis-andor-code-components-does-this-sample-highlight-17}

This sample leverages the
\href{https://docs.liferay.com/dxp/apps/web-experience/latest/javadocs/com/liferay/product/navigation/control/menu/ProductNavigationControlMenuEntry.html}{ProductNavigationControlMenuEntry}
API.

\subsection{How does this sample leverage the API(s) and/or code
component?}\label{how-does-this-sample-leverage-the-apis-andor-code-component-17}

This sample first leverages the
\texttt{ProductNavigationControlMenuEntry} interface as an OSGi service
via the \texttt{@Component} annotation:

\begin{verbatim}
@Component(
    immediate = true,
    property = {
        "product.navigation.control.menu.category.key=" + ProductNavigationControlMenuCategoryKeys.USER,
        "product.navigation.control.menu.entry.order:Integer=1"
    },
    service = ProductNavigationControlMenuEntry.class
)
\end{verbatim}

There are also two properties provided via the \texttt{@Component}
annotation:

\begin{itemize}
\tightlist
\item
  \texttt{product.navigation.control.menu.category.key}: the category in
  which your entry should reside. The default Control Menu provides
  three categories: \emph{SITES} (left portion), \emph{TOOLS} (middle
  portion), and \emph{USER} (right portion).
\item
  \texttt{product.navigation.control.menu.entry.order:Integer}: the
  order in which your entry will be displayed in the category. Entries
  are ordered from left to right. For example, an entry with order
  \texttt{1} will be listed to the left of an entry with order
  \texttt{2}. If the order is not specified, it's chosen at random based
  on which service was registered first in the OSGi container.
\end{itemize}

This sample also implements the
\texttt{ProductNavigationControlMenuEntry} interface. The following
methods are implemented:

\begin{itemize}
\tightlist
\item
  \texttt{getIcon(HttpServletRequest)}
\item
  \texttt{getLabel(Locale)}
\item
  \texttt{getURL(HttpServletRequest)}
\item
  \texttt{isShow(HttpServletRequest)}
\end{itemize}

Refer to this sample's \texttt{BladeProductNavigationControlMenuEntry}
class for Javadocs describing these methods. For more information on how
to customize Liferay Portal's Control Menu, visit the
\href{/docs/7-1/tutorials/-/knowledge_base/t/customizing-the-control-menu}{Customizing
the Control Menu} tutorial.

\subsection{Where Is This Sample?}\label{where-is-this-sample-14}

There are three different versions of this sample, each built with a
different build tool:

\begin{itemize}
\tightlist
\item
  \href{https://github.com/liferay/liferay-blade-samples/tree/7.1/gradle/extensions/control-menu-entry}{Gradle}
\item
  \href{https://github.com/liferay/liferay-blade-samples/tree/7.1/liferay-workspace/extensions/control-menu-entry}{Liferay
  Workspace}
\item
  \href{https://github.com/liferay/liferay-blade-samples/tree/7.1/maven/extensions/control-menu-entry}{Maven}
\end{itemize}

\section{Document Action}\label{document-action}

The Document Action sample shows how to add a context menu option to an
entry in the Documents and Media portlet. When deploying this sample
with no customizations, an additional menu option is available in the
Documents and Media Admin portlet and the Documents and Media portlet.
This sample creates a \emph{Blade Basic Info} option that displays basic
information about the entry (e.g., file name, type, version, etc.). For
example, the Admin portlet provides the new option as illustrated in the
images below:

\begin{figure}
\centering
\includegraphics{./images/documents-and-media-admin-portlet.png}
\caption{The new \emph{Blade Basic Info} option is available from the
entry's Options menu.}
\end{figure}

\begin{figure}
\centering
\includegraphics{./images/documents-and-media-admin-portlet-detail.png}
\caption{The new option is also available from the portlet's Document
Details.}
\end{figure}

Likewise, the Documents and Media portlet provides the same option after
selecting \emph{Show Actions} from the portlet's Configuration menu.

\begin{figure}
\centering
\includegraphics{./images/documents-and-media-portlet.png}
\caption{You can access the new \emph{Blade Basic Info} option from the
Documents and Media portlet added to a page.}
\end{figure}

\begin{figure}
\centering
\includegraphics{./images/documents-and-media-portlet-detail.png}
\caption{The Documents And Media portlet provides the option from its
Document Detail too.}
\end{figure}

\subsection{What API(s) and/or code components does this sample
highlight?}\label{what-apis-andor-code-components-does-this-sample-highlight-18}

This sample leverages the
\href{https://docs.liferay.com/dxp/portal/7.1-latest/javadocs/portal-kernel/com/liferay/portal/kernel/portlet/configuration/icon/PortletConfigurationIcon.html}{PortletConfigurationIcon}
API.

\subsection{How does this sample leverage the API(s) and/or code
component?}\label{how-does-this-sample-leverage-the-apis-andor-code-component-18}

There are four Java classes used in this sample:

\begin{itemize}
\tightlist
\item
  \texttt{BladeActionConfigurationIcon}: Adds the new context menu
  option to the Document Detail screen options
  (\includegraphics{./images/icon-options.png}) (top right corner) of
  the Documents and Media Admin portlet. See the
  \href{/docs/7-0/tutorials/-/knowledge_base/t/configuring-your-admin-apps-actions-menu}{Configuring
  Your Admin App's Actions Menu} tutorial for more details.
\item
  \texttt{BladeActionDisplayContext}: Adds the Display Context for the
  document action. More about Display Contexts are described later.
\item
  \texttt{BladeActionDisplayContextFactory}: Adds the Display Context
  factory for the document action.
\item
  \texttt{BladeDocumentActionPortlet}: Provides the portlet class, which
  extends the
  \href{https://portals.apache.org/pluto/portlet-2.0-apidocs/javax/portlet/GenericPortlet.html}{GenericPortlet}.
  This class generates what is shown when the context menu option is
  selected.
\end{itemize}

A Display Context is a Java class that controls access to a portlet
screen's UI elements. For example, the Document Library would use
Display Contexts to provide its screens all their UI elements. It would
use one Display Context for its document edit screen, another for its
document view screen, etc. A portlet ideally uses a different Display
Context for each of its screens.

A screen's JSP calls on the Display Context (DC) to get elements to
render and to decide whether to render certain types of elements. Some
of the DC methods return a collection of UI elements (e.g., a menu
object of menu items), while other DC methods return booleans that
determine whether to show particular element types. The DC decides which
objects to display, while the JSP organizes the rendered objects and
implements the screen's look and feel. You don't have to decide which
elements to display in your JSP; simply call the DC methods to populate
UI components with objects to render.

To customize or extend a portlet screen that uses a DC, you can extend
the DC and override the methods that control access to the elements that
interest you. For example, you can turn off displaying certain types of
elements (e.g., actions) by overriding the DC method that makes that
decision. You can add new custom elements (e.g., new actions) or remove
existing elements (e.g., a delete action) from a collection of elements
a DC method returns. The beauty of customizing via a DC is that you
don't have to modify the JSP. You only modify the particular methods
that are related to the UI customization goals. And JSP updates won't
break the DC customizations. Replacing a JSP, on the other hand, can
lead to missing an important JSP modification that a new Liferay version
introduces.

As you create custom portlets, you may want to implement DCs. You can
benefit from the separation of concerns that DCs provide and customers
can extend your portlet DCs to specify which UI elements to display. And
they don't need to worry about missing out on the updates you make to
the JSPs.

\subsection{Where Is This Sample?}\label{where-is-this-sample-15}

There are three different versions of this sample, each built with a
different build tool:

\begin{itemize}
\tightlist
\item
  \href{https://github.com/liferay/liferay-blade-samples/tree/7.1/gradle/extensions/document-action}{Gradle}
\item
  \href{https://github.com/liferay/liferay-blade-samples/tree/7.1/liferay-workspace/extensions/document-action}{Liferay
  Workspace}
\item
  \href{https://github.com/liferay/liferay-blade-samples/tree/7.1/maven/extensions/document-action}{Maven}
\end{itemize}

\section{Gogo Shell Command}\label{gogo-shell-command}

The Gogo Shell Command sample demonstrates adding a custom command to
Liferay DXP's Gogo shell environment. All Liferay DXP installations have
a Gogo shell environment, which lets system administrators interact with
Liferay DXP's module framework on a local server machine.

This example adds a new custom Gogo shell command called
\texttt{usercount} under the \texttt{blade} scope. It prints out the
number of registered users on your Liferay DXP installation.

To test this sample, follow the instructions below:

\begin{enumerate}
\def\labelenumi{\arabic{enumi}.}
\item
  Start a Liferay DXP installation.
\item
  Navigate to the Control Panel → \emph{Configuration} → \emph{Gogo
  Shell}.
\item
  Execute \texttt{help} to view all the available commands. The sample
  Gogo shell command is listed.

  \begin{figure}
  \centering
  \includegraphics{./images/gogo-shell-1.png}
  \caption{The sample Gogo shell command is listed with all the
  available commands.}
  \end{figure}
\item
  Execute \texttt{usercount} to execute the new custom command. The
  number of users on your running Liferay Portal installation is
  printed.

  \begin{figure}
  \centering
  \includegraphics{./images/gogo-shell-2.png}
  \caption{The outcome of executing the \texttt{usercount} command.}
  \end{figure}
\end{enumerate}

\subsection{What API(s) and/or code components does this sample
highlight?}\label{what-apis-andor-code-components-does-this-sample-highlight-19}

This sample demonstrates creating a new Gogo shell command by leveraging
\texttt{osgi.command.*} properties in a Java class.

\subsection{How does this sample leverage the API(s) and/or code
component?}\label{how-does-this-sample-leverage-the-apis-andor-code-component-19}

To add this new Gogo shell command, you must implement the logic in a
Java class with the following two properties:

\begin{itemize}
\tightlist
\item
  \texttt{osgi.command.function}: the command's name, which must match
  the method name in the registered service implementation.
\item
  \texttt{osgi.command.scope}: the general scope or namespace for the
  command.
\end{itemize}

These properties are set in your class's \texttt{@Component} annotation
like this:

\begin{verbatim}
@Component(
    property = {"osgi.command.function=usercount", "osgi.command.scope=blade"},
    service = Object.class
)
\end{verbatim}

The logic for the \texttt{usercount} command is specified in the method
with the same name:

\begin{verbatim}
public void usercount() {
    System.out.println(
        "# of users: " + getUserLocalService().getUsersCount());
}
\end{verbatim}

This method uses \emph{Declarative Services} to get a reference for the
\texttt{UserLocalService} to invoke the \texttt{getUsersCount} method.
This lets you find the number of users currently in the system.

For more information on using the Gogo shell, see the
\href{/docs/7-1/reference/-/knowledge_base/r/using-the-felix-gogo-shell}{Using
the Felix Gogo Shell} tutorial.

\subsection{Where Is This Sample?}\label{where-is-this-sample-16}

There are three different versions of this sample, each built with a
different build tool:

\begin{itemize}
\tightlist
\item
  \href{https://github.com/liferay/liferay-blade-samples/tree/7.1/gradle/extensions/gogo}{Gradle}
\item
  \href{https://github.com/liferay/liferay-blade-samples/tree/7.1/liferay-workspace/extensions/gogo}{Liferay
  Workspace}
\item
  \href{https://github.com/liferay/liferay-blade-samples/tree/7.1/maven/extensions/gogo}{Maven}
\end{itemize}

\section{Index Settings Contributor}\label{index-settings-contributor}

The Index Settings Contributor sample demonstrates how to add a custom
type mapping to Liferay DXP. You can demo this sample by completing the
following steps:

\begin{enumerate}
\def\labelenumi{\arabic{enumi}.}
\item
  Navigate to the \emph{Control Panel} → \emph{Configuration} →
  \emph{Search} menu.
\item
  Click \emph{Execute} for the \emph{Reindex all search indexes} action.

  All properties defined in your \texttt{.json} file are added to
  Liferay DXP's search engine. This sample adds the following index
  properties:

  \begin{itemize}
  \tightlist
  \item
    \texttt{sampleDate}
  \item
    \texttt{sampleDouble}
  \item
    \texttt{sampleLong}
  \item
    \texttt{sampleText}
  \end{itemize}

  You'll verify this next.
\item
  Find your Liferay DXP's instance ID. This can be found in the
  \emph{Control Panel} → \emph{Configuration} → \emph{Virtual Instances}
  menu.
\item
  Navigate to the following URL:

\begin{verbatim}
http://localhost:9200/liferay-[INSTANCE_ID]/_mapping/LiferayDocumentType?pretty
\end{verbatim}

  Be sure to insert your instance ID into the URL.
\item
  Verify the added properties are listed.

  \begin{figure}
  \centering
  \includegraphics{./images/index-settings-contributor.png}
  \caption{This sample added four new index properties.}
  \end{figure}
\end{enumerate}

\subsection{What API(s) and/or code components does this sample
highlight?}\label{what-apis-andor-code-components-does-this-sample-highlight-20}

This sample leverages the
\href{https://docs.liferay.com/dxp/apps/foundation/latest/javadocs/com/liferay/portal/search/elasticsearch/settings/IndexSettingsContributor.html}{IndexSettingsContributor}
API.

\subsection{How does this sample leverage the API(s) and/or code
component?}\label{how-does-this-sample-leverage-the-apis-andor-code-component-20}

Liferay's search engine provides an API to define custom mappings. To
use it, follow these fundamental steps:

\begin{enumerate}
\def\labelenumi{\arabic{enumi}.}
\item
  Define the new mapping. In this sample, the mapping is defined in the
  \texttt{META-INF/mappings/resources/index-type-mappings.json} file.
  Notice that the default document for Liferay DXP is called
  \texttt{LiferayDocumentType}. The mapping's features can be found in
  \href{https://www.elastic.co/guide/en/elasticsearch/reference/current/mapping.html}{Elasticsearch's
  docs}.
\item
  Inject the mapping into Elasticsearch. The
  \texttt{IndexSettingsContributor} class' components are invoked during
  the reindexing stage and receive a \texttt{TypeMappingsHelper} as a
  hook to add new mappings.
\end{enumerate}

This sample has two classes:

\begin{itemize}
\item
  \texttt{ResourceUtil}: reads the \texttt{.json} file.
\item
  \texttt{IndexSettingsContributor}: allows the addition of type
  mappings on Liferay DXP's search engine.
\end{itemize}

The \texttt{IndexSettingsContributor}'s \texttt{contribute} method adds
the type mappings:

\begin{verbatim}
@Override
public void contribute(
    String indexName, TypeMappingsHelper typeMappingsHelper) {
    try {
        String mappings = ResourceUtil.readResouceAsString(
            "META-INF/resources/mappings/index-type-mappings.json");

        typeMappingsHelper.addTypeMappings(indexName, mappings);
    }
    catch (Exception e) {
        e.printStackTrace();
    }
}
\end{verbatim}

For the \texttt{ResourceUtil.readResouceAsString} parameter, you should
pass the path for the \texttt{.json} file that contains the properties
to be added.

Also, it is important to highlight the
\texttt{IndexSettingsContributor}'s \texttt{@Component} annotation that
registers a new service to the OSGi container:

\begin{verbatim}
@Component(
    immediate = true,
    service = com.liferay.portal.search.elasticsearch6.settings.IndexSettingsContributor.class
)
\end{verbatim}

This sample demonstrates the essentials needed to contribute your own
index settings.

\subsection{Where Is This Sample?}\label{where-is-this-sample-17}

There are three different versions of this sample, each built with a
different build tool:

\begin{itemize}
\tightlist
\item
  \href{https://github.com/liferay/liferay-blade-samples/blob/7.1/gradle/extensions/index-settings-contributor}{Gradle}
\item
  \href{https://github.com/liferay/liferay-blade-samples/blob/7.1/liferay-workspace/extensions/index-settings-contributor}{Liferay
  Workspace}
\item
  \href{https://github.com/liferay/liferay-blade-samples/blob/7.1/maven/extensions/index-settings-contributor}{Maven}
\end{itemize}

\section{Indexer Post Processor}\label{indexer-post-processor}

The Indexer Post Processor sample demonstrates using the
\texttt{IndexerPostProcessor} interface, which is provided to customize
search queries and documents before they're sent to the search engine,
and/or customize result summaries when they're returned to end users.
This basic demonstration prints a message in the log when one of the
\texttt{*IndexerPostProcessor} methods is called.

To see this sample's messages in Liferay DXP's log, you must add a
logging category to the portal. Navigate to \emph{Control Panel} →
\emph{Configuration} → \emph{Server Administration} and click on
\emph{Log Levels} → \emph{Add Category}. Then fill out the form:

\begin{itemize}
\tightlist
\item
  \emph{Logger Name}:
  \texttt{com.liferay.blade.samples.indexerpostprocessor}
\item
  \emph{Log Level}: \texttt{INFO}
\end{itemize}

Once you save the new logging category, you can witness the sample
indexer post processor in action. For example, you can test the sample's
\texttt{BlogsIndexerPostProcessor} implementation by creating a blog
entry. When you publish the blog, the following message is logged in the
console:

\begin{verbatim}
18:27:30,737 INFO  [http-nio-8080-exec-8][BlogsIndexerPostProcessor:76] postProcessDocument
\end{verbatim}

\subsection{What API(s) and/or code components does this sample
highlight?}\label{what-apis-andor-code-components-does-this-sample-highlight-21}

This sample leverages the
\href{https://docs.liferay.com/dxp/portal/7.1-latest/javadocs/portal-kernel/com/liferay/portal/kernel/search/IndexerPostProcessor.html}{IndexerPostProcessor}
API.

\subsection{How does this sample leverage the API(s) and/or code
component?}\label{how-does-this-sample-leverage-the-apis-andor-code-component-21}

This sample contains four implementations of the
\texttt{IndexerPostProcessor} interface:

\begin{itemize}
\tightlist
\item
  \texttt{BlogsIndexerPostProcessor}
\item
  \texttt{MultipleEntityIndexerPostProcessor}
\item
  \texttt{MultipleIndexerPostProcessor}
\item
  \texttt{UserEntityIndexerPostProcessor}
\end{itemize}

All these classes leverage the interface as an OSGi service via the
\texttt{@Component} annotation. For example, the \texttt{@Component}
annotation of the \texttt{UserEntityIndexerPostProcessor} looks like
this:

\begin{verbatim}
@Component(
    immediate = true,
    property = {
        "indexer.class.name=com.liferay.portal.kernel.model.User",
        "indexer.class.name=com.liferay.portal.kernel.model.UserGroup"
    },
    service = IndexerPostProcessor.class
)
\end{verbatim}

There's one property type provided via the \texttt{@Component}
annotation:

\begin{itemize}
\tightlist
\item
  \texttt{indexer.class.name}: the fully qualified class name of the
  indexed entity or an \texttt{Indexer} class itself.
\end{itemize}

This sample's implementations of the \texttt{IndexerPostProcessor}
interface override the following methods:

\begin{itemize}
\tightlist
\item
  \texttt{postProcessContextBooleanFilter}
\item
  \texttt{postProcessContextQuery}
\item
  \texttt{postProcessDocument}
\item
  \texttt{postProcessFullQuery}
\item
  \texttt{postProcessSearchQuery(BooleanQuery,\ BooleanFilter)}
\item
  \texttt{postProcessSearchQuery(BooleanQuery,\ SearchContext)}
\item
  \texttt{postProcessSummary}
\end{itemize}

For more information on Liferay's Search API, refer to the
\href{/docs/7-0/tutorials/-/knowledge_base/t/introduction-to-liferay-search}{Introduction
to Liferay Search} tutorial.

\subsection{Where Is This Sample?}\label{where-is-this-sample-18}

There are three different versions of this sample, each built with a
different build tool:

\begin{itemize}
\tightlist
\item
  \href{https://github.com/liferay/liferay-blade-samples/tree/7.1/gradle/extensions/indexer-post-processor}{Gradle}
\item
  \href{https://github.com/liferay/liferay-blade-samples/tree/7.1/liferay-workspace/extensions/indexer-post-processor}{Liferay
  Workspace}
\item
  \href{https://github.com/liferay/liferay-blade-samples/tree/7.1/maven/extensions/indexer-post-processor}{Maven}
\end{itemize}

\section{Model Listener}\label{model-listener}

The Model Listener sample demonstrates adding a custom model listener to
a Liferay Portal out-of-the-box entity. When deploying this sample with
no customizations, a custom model listener is added to the portal's
layouts, listening for \texttt{onBeforeCreate} events. This means that
any page creation will trigger this listener, which will execute before
the new page is created.

For example, if a new page is added with the name \emph{My Test Page},
the following message is printed to the console:

\begin{figure}
\centering
\includegraphics{./images/model-listener-1.png}
\caption{The sample model listener's message in the console.}
\end{figure}

You can also verify that the model listener sample was executed by
navigating to the new page's \emph{Options} → \emph{Configure Page} →
\emph{SEO} option. The HTML Title field looks like this:

\begin{figure}
\centering
\includegraphics{./images/model-listener-2.png}
\caption{The page's HTML title updated by the model listener sample.}
\end{figure}

\subsection{What API(s) and/or code components does this sample
highlight?}\label{what-apis-andor-code-components-does-this-sample-highlight-22}

This sample leverages the
\href{https://docs.liferay.com/dxp/portal/7.1-latest/javadocs/portal-kernel/com/liferay/portal/kernel/model/ModelListener.html}{ModelListener}
API.

\subsection{How does this sample leverage the API(s) and/or code
component?}\label{how-does-this-sample-leverage-the-apis-andor-code-component-22}

Model Listeners are used to listen for persistence events on models and
take actions as a result of those events. Actions can be executed on an
entity's database table before or after a \texttt{create},
\texttt{remove}, \texttt{update}, \texttt{addAssociation}, or
\texttt{removeAssociation} event. It's possible to have more than one
model listener on a single model too; the execution order is not
guaranteed.

There are two steps to create a new model listener:

\begin{itemize}
\tightlist
\item
  Implement a Model Listener class
\item
  Register the new service in Liferay's OSGi runtime
\end{itemize}

This sample adds the model listener logic in a new Java class named
\texttt{CustomLayoutListener} that extends
\href{https://docs.liferay.com/dxp/portal/7.1-latest/javadocs/portal-kernel/com/liferay/portal/kernel/model/BaseModelListener.html}{BaseModelListener}.

\begin{verbatim}
public class CustomLayoutListener extends BaseModelListener<Layout> {

    @Override
    public void onBeforeCreate(Layout model) throws ModelListenerException {
        System.out.println(
            "About to create layout: " + model.getNameCurrentValue());

        model.setTitle("Title generated by model listener!");
    }

}
\end{verbatim}

Important things to note in this code snippet are

\begin{itemize}
\tightlist
\item
  The entity to be targeted by this model listener is specified as the
  parameterized type (e.g., \texttt{Layout}).
\item
  The overridden methods dictate the type of event(s) that are listened
  for (e.g., \texttt{onBeforeCreate}); they also trigger the logic
  execution.
\end{itemize}

The final step is registering the service in Liferay's OSGi runtime,
which is accomplished by the following annotation (if using Declarative
Services):

\begin{verbatim}
@Component(immediate = true, service = ModelListener.class)
\end{verbatim}

For more information on model listeners, see the
\href{/docs/7-1/tutorials/-/knowledge_base/t/model-listeners}{Creating
Model Listeners} tutorial.

\subsection{Where Is This Sample?}\label{where-is-this-sample-19}

There are three different versions of this sample, each built with a
different build tool:

\begin{itemize}
\tightlist
\item
  \href{https://github.com/liferay/liferay-blade-samples/tree/7.1/gradle/extensions/model-listener}{Gradle}
\item
  \href{https://github.com/liferay/liferay-blade-samples/tree/7.1/liferay-workspace/extensions/model-listener}{Liferay
  Workspace}
\item
  \href{https://github.com/liferay/liferay-blade-samples/tree/7.1/maven/extensions/model-listener}{Maven}
\end{itemize}

\section{Screen Name Validator}\label{screen-name-validator}

The Screen Name Validator sample provides a way to validate a user's
inputted screen name. During validation, the screen name is tested
client-side and server-side.

This sample checks if a user's screen name contains reserved words that
are configured in the \emph{Control Panel} → \emph{Configuration} →
\emph{System Settings} → \emph{Foundation} → \emph{ScreenName Validator}
menu. The default values for the screen name validator's reserved words
are \emph{admin} and \emph{user}.

\begin{figure}
\centering
\includegraphics{./images/screenname-validator-config.png}
\caption{Enter reserved words for the screen name validator.}
\end{figure}

You can test this sample by following the following steps:

\begin{enumerate}
\def\labelenumi{\arabic{enumi}.}
\tightlist
\item
  Deploy the Screen Name Validator to your portal installation.
\item
  Navigate to the \emph{Control Panel} → \emph{Users} → \emph{Users and
  Organizations} menu.
\item
  Create a new user by selecting the \emph{Add User}
  (\includegraphics{./images/icon-add.png}) button.
\item
  Adding a screen name that contains the word \emph{admin} or
  \emph{user}.
\end{enumerate}

\begin{figure}
\centering
\includegraphics{./images/screenname-validator-test.png}
\caption{The error message displays when inputting a reserved word for
the screen name.}
\end{figure}

\subsection{What API(s) and/or code components does this sample
highlight?}\label{what-apis-andor-code-components-does-this-sample-highlight-23}

This sample leverages the
\href{https://docs.liferay.com/dxp/portal/7.1-latest/javadocs/portal-kernel/com/liferay/portal/kernel/security/auth/ScreenNameValidator.html}{ScreenNameValidator}
API.

\subsection{How does this sample leverage the API(s) and/or code
component?}\label{how-does-this-sample-leverage-the-apis-andor-code-component-23}

To customize this sample, modify its
\texttt{com.liferay.blade.samples.screenname.validator.internal.CustomScreenNameValidator}
class.

You can also customize this sample's configuration by adding more
properties in its
\texttt{com.liferay.blade.samples.screenname.validator.CustomScreenNameConfiguration}
class.

For more information on customizing the Validation sample to fit your
needs, see the Javadoc provided in this sample's Java classes.

\subsection{Where Is This Sample?}\label{where-is-this-sample-20}

There are three different versions of this sample, each built with a
different build tool:

\begin{itemize}
\tightlist
\item
  \href{https://github.com/liferay/liferay-blade-samples/tree/7.1/gradle/extensions/screen-name-validator}{Gradle}
\item
  \href{https://github.com/liferay/liferay-blade-samples/tree/7.1/liferay-workspace/extensions/screen-name-validator}{Liferay
  Workspace}
\item
  \href{https://github.com/liferay/liferay-blade-samples/tree/7.1/maven/extensions/screen-name-validator}{Maven}
\end{itemize}

\section{Servlet}\label{servlet}

The Servlet sample provides an OSGi Whiteboard Servlet in Liferay DXP.
When deploying this sample and configuring the servlet, a \emph{Hello
World} message is displayed when accessing the servlet page URL. Log
info is also outputted to your console.

\begin{figure}
\centering
\includegraphics{./images/servlet-sample.png}
\caption{The servlet displays \emph{Hello World} from the configured
servlet page URL.}
\end{figure}

\begin{figure}
\centering
\includegraphics{./images/servlet-sample-log.png}
\caption{The servlet also logs info in the console.}
\end{figure}

To configure the servlet in Liferay DXP, complete the following steps:

\begin{enumerate}
\def\labelenumi{\arabic{enumi}.}
\item
  Navigate to the \emph{Control Panel} → \emph{Configuration} →
  \emph{Server Administration} → \emph{Log Levels}.
\item
  Select \emph{Add Category}.
\item
  Insert \emph{com.liferay.blade.samples.servlet.BladeServlet} for the
  Logger Name and \emph{INFO} for the Log Level.
\item
  Navigate to the http://localhost:8080/o/blade/servlet URL.
\end{enumerate}

\subsection{What API(s) and/or code components does this sample
highlight?}\label{what-apis-andor-code-components-does-this-sample-highlight-24}

This sample leverages the
\href{https://tomcat.apache.org/tomcat-5.5-doc/servletapi/javax/servlet/http/HttpServlet.html}{HttpServlet}
API.

\subsection{How does this sample leverage the API(s) and/or code
component?}\label{how-does-this-sample-leverage-the-apis-andor-code-component-24}

To customize this sample, modify its
\texttt{com.liferay.blade.samples.servlet.BladeServlet} class. This
class extends the \texttt{HttpServlet} class. Creating your own servlet
for Liferay DXP is useful when you need to implement servlet actions.
For example, if you wanted to implement the CMIS server by yourself with
\href{https://chemistry.apache.org/}{Apache Chemistry}, you would need
to implement your own servlet, managing requests at a low level.

\subsection{Where Is This Sample?}\label{where-is-this-sample-21}

There are three different versions of this sample, each built with a
different build tool:

\begin{itemize}
\tightlist
\item
  \href{https://github.com/liferay/liferay-blade-samples/tree/7.1/gradle/extensions/servlet}{Gradle}
\item
  \href{https://github.com/liferay/liferay-blade-samples/tree/7.1/liferay-workspace/extensions/servlet}{Liferay
  Workspace}
\item
  \href{https://github.com/liferay/liferay-blade-samples/tree/7.1/maven/extensions/servlet}{Maven}
\end{itemize}

\chapter{Overrides}\label{overrides}

This section focuses on Liferay sample overrides. You can view these
sample overrides by visiting the \texttt{overrides} folder corresponding
to your preferred build tool:

\begin{itemize}
\tightlist
\item
  \href{https://github.com/liferay/liferay-blade-samples/tree/7.1/gradle/overrides}{Gradle
  sample overrides}
\item
  \href{https://github.com/liferay/liferay-blade-samples/tree/7.1/liferay-workspace/overrides}{Liferay
  Workspace sample overrides}
\item
  \href{https://github.com/liferay/liferay-blade-samples/tree/7.1/maven/overrides}{Maven
  sample overrides}
\end{itemize}

Visit a particular sample page to learn more!

\section{Module JSP Override}\label{module-jsp-override}

The Module JSP Override sample conveys Liferay's recommended approach to
override an application's JSP by leveraging OSGi fragment modules. This
example overrides the default \texttt{login.jsp} file in the
\texttt{com.liferay.login.web} bundle by adding the red text
\emph{changed} to the Sign In form.

\begin{figure}
\centering
\includegraphics{./images/hook-jsp.png}
\caption{The customized Sign In form with the new \emph{changed} text.}
\end{figure}

\subsection{What API(s) and/or code components does this sample
highlight?}\label{what-apis-andor-code-components-does-this-sample-highlight-25}

This sample demonstrates how to create a fragment host module and
configure it to override an existing module's JSP.

\subsection{How does this sample leverage the API(s) and/or code
component?}\label{how-does-this-sample-leverage-the-apis-andor-code-component-25}

You can create your own JSP override by

\begin{itemize}
\tightlist
\item
  Declaring the fragment host.
\item
  Providing the JSP that will override the original one.
\end{itemize}

To properly declare the fragment host in the \texttt{bnd.bnd} file, you
must specify the host module's (where the original JSP is located)
Bundle Symbolic Name and the host module's exact version to which the
fragment belongs. In this example, this is configured like this:

\begin{verbatim}
Fragment-Host: com.liferay.login.web;bundle-version="1.0.0"
\end{verbatim}

Then you must provide the new JSP intended to override the original one.
Be sure to mimic the host module's folder structure when overriding its
JAR. For this example, since the original JSP is in the folder
\texttt{/META-INF/resources/login.jsp}, the new JSP file resides in the
folder \texttt{src/main/resources/META-INF/resources/login.jsp}.

If needed, you can also target the original JSP following one of the two
possible naming conventions: \texttt{original} or \texttt{portal}. This
pattern looks like

\begin{verbatim}
<liferay-util:include
    page="/login.original.jsp"
    servletContext="<%= application %>"
/>
\end{verbatim}

or

\begin{verbatim}
<liferay-util:include
    page="/login.portal.jsp"
    servletContext="<%= application %>"
/>
\end{verbatim}

This approach can be used to override any application JSP (i.e., JSPs
residing in a module). You can also add new JSPs to an existing module
with this technique. If you need to override a core JSP, see the
\href{/docs/7-1/tutorials/-/knowledge_base/t/jsp-overrides-using-custom-jsp-bag}{JSP
Overrides Using Custom JSP Bag} tutorial.

For more information on other ways to customize JSPs, see the
\href{/docs/7-1/tutorials/-/knowledge_base/t/customizing-jsps}{Customizing
JSPs} tutorial.

\subsection{Where Is This Sample?}\label{where-is-this-sample-22}

There are three different versions of this sample, each built with a
different build tool:

\begin{itemize}
\tightlist
\item
  \href{https://github.com/liferay/liferay-blade-samples/tree/7.1/gradle/overrides/module-jsp-override}{Gradle}
\item
  \href{https://github.com/liferay/liferay-blade-samples/tree/7.1/liferay-workspace/overrides/module-jsp-override}{Liferay
  Workspace}
\item
  \href{https://github.com/liferay/liferay-blade-samples/tree/7.1/maven/overrides/module-jsp-override}{Maven}
\end{itemize}

\section{Resource Bundle Override}\label{resource-bundle-override}

This example overrides the default
\texttt{javax.portlet.title.com\_liferay\_login\_web\_portlet\_LoginPortlet}
language key for Liferay DXP's default Login portlet. After deploying
this sample to Liferay DXP, the Login portlet's \emph{Sign In} title is
modified to display \emph{Login Portlet Override}.

\begin{figure}
\centering
\includegraphics{./images/hook-resourcebundle.png}
\caption{The customized Login portlet displays the new language key.}
\end{figure}

For reference, the Login portlet's language keys are stored in the
\href{https://github.com/liferay/liferay-portal}{liferay-portal} Github
repo's \texttt{modules/apps/login/login-web/src/main/resources/content}
folder.

\subsection{What API(s) and/or code components does this sample
highlight?}\label{what-apis-andor-code-components-does-this-sample-highlight-26}

This sample leverages the
\href{https://bnd.bndtools.org/chapters/220-contracts.html}{\texttt{Provide-Capability}}
OSGi manifest header.

\subsection{How does this sample leverage the API(s) and/or code
component?}\label{how-does-this-sample-leverage-the-apis-andor-code-component-26}

This sample conveys the recommended approach to override a portlet's
language keys file for any module that is deployed to Liferay DXP's OSGi
runtime (not applicable to Liferay DXP's core language keys).

The steps to override a portlet's language keys are

\begin{itemize}
\tightlist
\item
  Provide the new language keys that will override the original ones.
\item
  Prioritize the new module's resource bundle.
\end{itemize}

This sample's \texttt{src/main/resources/content} folder holds the
language properties file to override. Since this example's goal is to
override only the English keys, the \texttt{Language\_en.properties} is
added. You can add more language properties files for additional
language key locales you want to override (e.g.,
\texttt{Language\_en.properties} for Spanish).

Once your language keys are in place, you must use OSGi manifest headers
to specify your custom language keys are for the target module. To
compliment the target module's resource bundle, you must aggregate your
resource bundle with the target module's resource bundle. This is done
by ranking your module first to prioritize its resource bundle over the
target module resource bundle. See this sample's \texttt{bnd.bnd} as an
example for setting the \texttt{Provide-Capability} OSGi header:

\begin{verbatim}
Provide-Capability:\
    liferay.resource.bundle;\
        resource.bundle.base.name="content.Language",\
    liferay.resource.bundle;\
        bundle.symbolic.name=com.liferay.login.web;\
        resource.bundle.aggregate:String="(bundle.symbolic.name=com.liferay.blade.login.web.resource.bundle.override),(bundle.symbolic.name=com.liferay.login.web)";\
        resource.bundle.base.name="content.Language";\
        service.ranking:Long="2";\
        servlet.context.name=login-web
\end{verbatim}

For more information on the \texttt{Provide-Capability} header and its
parts, see the
\href{/docs/7-1/tutorials/-/knowledge_base/t/overriding-a-modules-language-keys\#prioritize-your-modules-resource-bundle}{Prioritze
Your Module's Resource Bundle} section.

This approach can be used to override any portlet's language keys (i.e.,
\texttt{language.properties} files that are inside a module deployed to
Liferay DXP's OSGi runtime). If you need to override Liferay DXP's core
language keys, see the
\href{/docs/7-1/tutorials/-/knowledge_base/t/overriding-global-language-keys}{Overriding
Global Language Keys} tutorial.

For more information on using a resource bundle to override a module's
language keys, see the
\href{/docs/7-1/tutorials/-/knowledge_base/t/overriding-a-modules-language-keys}{Overriding
a Module's Language Keys} tutorial.

\subsection{Where Is This Sample?}\label{where-is-this-sample-23}

There are three different versions of this sample, each built with a
different build tool:

\begin{itemize}
\tightlist
\item
  \href{https://github.com/liferay/liferay-blade-samples/tree/7.1/gradle/overrides/login-web-resource-bundle-override}{Gradle}
\item
  \href{https://github.com/liferay/liferay-blade-samples/tree/7.1/liferay-workspace/overrides/login-web-resource-bundle-override}{Liferay
  Workspace}
\item
  \href{https://github.com/liferay/liferay-blade-samples/tree/7.1/maven/overrides/login-web-resource-bundle-override}{Maven}
\end{itemize}

\chapter{Themes}\label{themes-23}

This section focuses on Liferay sample themes. You can view these sample
themes by visiting the \texttt{themes} folder corresponding to your
preferred build tool:

\begin{itemize}
\tightlist
\item
  \href{https://github.com/liferay/liferay-blade-samples/tree/7.1/gradle/themes}{Gradle
  sample themes}
\item
  \href{https://github.com/liferay/liferay-blade-samples/tree/7.1/liferay-workspace/themes}{Liferay
  Workspace sample themes}
\item
  \href{https://github.com/liferay/liferay-blade-samples/tree/7.1/maven/themes}{Maven
  sample themes}
\end{itemize}

Visit a particular sample page to learn more!

\section{Simple Theme}\label{simple-theme}

The Simple Theme sample provides the base files for a theme, using the
\href{/docs/7-1/reference/-/knowledge_base/r/theme-builder-gradle-plugin}{Theme
Builder Gradle plugin}. When deploying this sample with no
customizations, a theme based off of the \texttt{\_styled} base theme is
created.

\begin{figure}
\centering
\includegraphics{./images/theme.png}
\caption{A theme based off of the Styled base theme is created when the
Theme Blade sample is deployed to Liferay Portal.}
\end{figure}

For more information on themes, visit the
\href{/docs/7-1/tutorials/-/knowledge_base/t/introduction-to-themes}{Introduction
to Themes} tutorial.

\subsection{What API(s) and/or code components does this sample
highlight?}\label{what-apis-andor-code-components-does-this-sample-highlight-27}

This sample demonstrates a way to create a simple theme in Liferay DXP.

\subsection{How does this sample leverage the API(s) and/or code
component?}\label{how-does-this-sample-leverage-the-apis-andor-code-component-27}

To modify this sample, add the \texttt{images}, \texttt{js}, or
\texttt{templates} folder, along with your modified files, to the
\texttt{src/main/webapp} folder. The sample already provides the
\texttt{src/main/resources/resources-importer},
\texttt{src/main/webapp/WEB-INF}, and \texttt{src/main/webapp/css}
folders for you. Add your style modifications to the provided
\texttt{css/\_custom.scss} file. For a complete explanation of a theme's
files, see the
\href{/docs/7-1/reference/-/knowledge_base/r/theme-reference-guide}{Theme
Reference Guide}.

\subsection{Where Is This Sample?}\label{where-is-this-sample-24}

There are three different versions of this sample, each built with a
different build tool:

\begin{itemize}
\tightlist
\item
  \href{https://github.com/liferay/liferay-blade-samples/tree/7.1/gradle/themes/simple-theme}{Gradle}
\item
  \href{https://github.com/liferay/liferay-blade-samples/tree/7.1/liferay-workspace/wars/simple-theme}{Liferay
  Workspace}
\item
  \href{https://github.com/liferay/liferay-blade-samples/tree/7.1/maven/themes/simple-theme}{Maven}
\end{itemize}

\section{Template Context
Contributor}\label{template-context-contributor}

The Template Context Contributor sample injects a new variable into
Liferay DXP's theme context. When deploying this sample with no
customizations, you can use the \texttt{\$\{sample\_text\}} variable
from any theme.

\subsection{What API(s) and/or code components does this sample
highlight?}\label{what-apis-andor-code-components-does-this-sample-highlight-28}

Many developers prefer using templating frameworks like FreeMarker and
Velocity, but don't have access to the common objects offered to those
working with JSPs. Context contributors allow non-JSP developers an easy
way to inject variables into their Liferay templates.

This sample leverages the
\href{https://docs.liferay.com/dxp/portal/7.1-latest/javadocs/portal-kernel/com/liferay/portal/kernel/template/TemplateContextContributor.html}{TemplateContextContributor}
API.

\subsection{How does this sample leverage the API(s) and/or code
component?}\label{how-does-this-sample-leverage-the-apis-andor-code-component-28}

You can easily modify this sample by customizing its
\texttt{BladeTemplateContextContributor.java} Java class. For example,
the default context contributor sample provides the
\texttt{\$\{sample\_text\}} variable by injecting it into Liferay's
\texttt{contextObjects}, which is a map provided by default to offer
common variables to non-JSP template developers. You can easily inject
your own variables into the \texttt{contextObjects} map usable by any
theme deployed to Liferay DXP.

Are you working with templates that aren't themes (e.g., ADTs, DDM
templates, etc.)? You can change the context in which your variables are
injected by modifying the \texttt{property} attribute in the
\texttt{@Component} annotation. If you want your variable available for
all templates, change it to

\begin{verbatim}
property = {"type=" + TemplateContextContributor.TYPE_GLOBAL}
\end{verbatim}

For more information on customizing the Template Context Contributor
sample to fit your needs, see the Javadoc listed in this sample's
\texttt{com.liferay.blade.samples.theme.contributorBladeTemplateContextContributor}
class. For more information on context contributors and how to create
them in Liferay DXP, visit the
\href{/docs/7-1/tutorials/-/knowledge_base/t/injecting-additional-context-variables-into-your-templates}{Context
Contributors} tutorial.

\subsection{Where Is This Sample?}\label{where-is-this-sample-25}

There are three different versions of this sample, each built with a
different build tool:

\begin{itemize}
\tightlist
\item
  \href{https://github.com/liferay/liferay-blade-samples/tree/7.1/gradle/themes/template-context-contributor}{Gradle}
\item
  \href{https://github.com/liferay/liferay-blade-samples/tree/7.1/liferay-workspace/themes/template-context-contributor}{Liferay
  Workspace}
\item
  \href{https://github.com/liferay/liferay-blade-samples/tree/7.1/maven/themes/template-context-contributor}{Maven}
\end{itemize}

\section{Theme Contributor}\label{theme-contributor}

The Theme Contributor sample contributes updates to the UI of the theme
body, Control Menu, Product Menu, and Simulation Panel. When deploying
this sample with no customizations, the colors of the theme and
aforementioned menus are updated.

\begin{figure}
\centering
\includegraphics{./images/theme-contributor-yellow.png}
\caption{Your Liferay DXP pages and menu fonts now have a yellow tint.}
\end{figure}

Also, there's a simple JavaScript update that is provided, which logs a
message to the browser's console window that states \emph{Hello Blade
Theme Contributor!}.

\begin{figure}
\centering
\includegraphics{./images/theme-contributor-console-output.png}
\caption{The message is printed to your browser's console window using
JavaScript.}
\end{figure}

\subsection{What API(s) and/or code components does this sample
highlight?}\label{what-apis-andor-code-components-does-this-sample-highlight-29}

This sample demonstrates a way to contribute updates to a Liferay DXP
theme. Theme Contributors let you package UI resources (e.g., CSS and
JS) independent of a theme to include on a Liferay DXP page.

\subsection{How does this sample leverage the API(s) and/or code
component?}\label{how-does-this-sample-leverage-the-apis-andor-code-component-29}

To modify this sample, replace the corresponding JS or SCSS file with
the JavaScript or styles that you want, or add your own JS or SCSS
files. For example, this sample provides an update to the Control Menu's
\texttt{background-color} in its
\texttt{src/main/resources/META-INF/resources/css/blade.theme.contributor/\_control\_menu.scss}
file:

\begin{verbatim}
body {
        .control-menu {
                background-color: darkkhaki;
        }
}
\end{verbatim}

All of the SCSS files used in this sample are imported into the main
\texttt{blade.theme.contributor.scss} file:

\begin{verbatim}
@import "bourbon";
@import "mixins";

@import "blade.theme.contributor/body";
@import "blade.theme.contributor/control_menu";
@import "blade.theme.contributor/product_menu";
@import "blade.theme.contributor/simulation_panel";
\end{verbatim}

If you add your own \texttt{SCSS} files, you must add them to the list
of imports in the \texttt{blade.theme.contributor.scss} file.

Likewise, the sample \texttt{blade.theme.contributor.js} logs a message
to your browser's console window using the following JS logic:

\begin{verbatim}
console.log('Hello Blade Theme Contributor!');
\end{verbatim}

For more information on Theme Contributors, visit the
\href{/docs/7-1/tutorials/-/knowledge_base/t/packaging-independent-ui-resources-for-your-site}{Theme
Contributors} tutorial.

\subsection{Where Is This Sample?}\label{where-is-this-sample-26}

There are three different versions of this sample, each built with a
different build tool:

\begin{itemize}
\tightlist
\item
  \href{https://github.com/liferay/liferay-blade-samples/tree/7.1/gradle/themes/theme-contributor}{Gradle}
\item
  \href{https://github.com/liferay/liferay-blade-samples/tree/7.1/liferay-workspace/themes/theme-contributor}{Liferay
  Workspace}
\item
  \href{https://github.com/liferay/liferay-blade-samples/tree/7.1/maven/themes/theme-contributor}{Maven}
\end{itemize}

\chapter{Ext}\label{ext}

This section focuses on Liferay Ext modules. You can view these sample
apps by visiting the \texttt{ext} folder corresponding to your preferred
build tool:

\begin{itemize}
\tightlist
\item
  \href{https://github.com/liferay/liferay-blade-samples/tree/7.1/gradle/ext}{Gradle
  sample apps}
\item
  \href{https://github.com/liferay/liferay-blade-samples/tree/7.1/liferay-workspace/ext}{Liferay
  Workspace sample apps}
\end{itemize}

Visit the sample page to learn more!

\section{Login Web Ext}\label{login-web-ext}

The Login Ext Module sample demonstrates how to customize a default
Liferay module's source code. This example replaces the default
\texttt{login.jsp} file in the \texttt{com.liferay.login.web} bundle by
adding the text \emph{Hello from com.liferay.login.web.ext module! 2 + 2
= 4} to the Sign In form.

\begin{figure}
\centering
\includegraphics{./images/login-ext.png}
\caption{The Login Ext module customizes the original Login module.}
\end{figure}

It also prints the following text to the console when you select
\emph{Forgot Password} from the Sign In form:

\begin{verbatim}
In com.liferay.login.web.internal.portlet.action.ForgotPasswordMVCRenderCommand render
\end{verbatim}

Before deploying the sample, you must stop the original bundle you
intend to override. This is because the Ext sample's generated JAR
includes the original bundle source plus your modified source files.
Follow the instructions below to do this:

\begin{enumerate}
\def\labelenumi{\arabic{enumi}.}
\item
  Connect to your portal instance using
  \href{/docs/7-1/reference/-/knowledge_base/r/using-the-felix-gogo-shell}{Gogo
  Shell}.
\item
  Search for the bundle ID of the original bundle to override. To find
  the \texttt{com.liferay.login.web} bundle, execute this command:

\begin{verbatim}
lb -s | grep com.liferay.login.web
\end{verbatim}

  This returns output similar to this:

\begin{verbatim}
423|Active   |   10|com.liferay.login.web (3.0.4)
\end{verbatim}

  Make note of the ID (e.g., \texttt{423}).
\item
  Stop the bundle:

\begin{verbatim}
stop 423
\end{verbatim}
\end{enumerate}

Once the original bundle is stopped, deploy the Ext module. Note that
you cannot leverage Blade or Gradle's \texttt{deploy} command to do
this. The \texttt{deploy} command deploys the module to the
\texttt{osgi\textbackslash{}marketplace\textbackslash{}override} folder
by default, which does not configure Ext modules properly for usage. You
should build and copy the Ext module's JAR to the \texttt{deploy} folder
manually, or leverage Liferay Dev Studio's
\href{/docs/7-1/tutorials/-/knowledge_base/t/deploying-projects-with-liferay-ide}{drag-and-drop
deployment} feature.

\subsection{What API(s) and/or code components does this sample
highlight?}\label{what-apis-andor-code-components-does-this-sample-highlight-30}

This sample demonstrates how to create an Ext module and configure it to
replace a default module bundle.

\subsection{How does this sample leverage the API(s) and/or code
component?}\label{how-does-this-sample-leverage-the-apis-andor-code-component-30}

You can create your own Ext module project by

\begin{itemize}
\tightlist
\item
  Declaring the original module name and version.
\item
  Providing the source code that will replace the original.
\end{itemize}

To declare the original module in the \texttt{build.gradle} file
properly (only supports Gradle), you must specify the original module's
Bundle Symbolic Name and the original module's exact version. In this
example, this is configured like this:

\begin{verbatim}
originalModule group: "com.liferay", name: "com.liferay.login.web", version: "3.0.4"
\end{verbatim}

If you're leveraging
\href{/docs/7-1/tutorials/-/knowledge_base/t/liferay-workspace}{Liferay
Workspace}, you should put your Ext module project in the \texttt{/ext}
folder (default); you can specify a different Ext folder name in
workspace's \texttt{gradle.properties} by adding

\begin{verbatim}
liferay.workspace.ext.dir=EXT_DIR
\end{verbatim}

If you are developing an Ext module project in standalone mode (not
associated with Liferay Workspace), you must declare the Ext Gradle
plugin in your \texttt{build.gradle}:

\begin{verbatim}
apply plugin: 'com.liferay.osgi.ext.plugin'
\end{verbatim}

Then you must provide your own code intended to replace the original
one. \textbf{Be sure to mimic the original module's folder structure
when overriding its JAR.}

The following file types can be overlaid with an Ext module:

\begin{itemize}
\tightlist
\item
  CSS
\item
  Java
\item
  JavaScript
\item
  Language files (\texttt{Language.properties})
\item
  Scss
\item
  Soy
\item
  etc.
\end{itemize}

The
\href{https://github.com/liferay/liferay-portal/blob/master/modules/sdk/gradle-plugins/src/main/java/com/liferay/gradle/plugins/LiferayOSGiExtPlugin.java}{Ext
Gradle Plugin} helps compile your code into the JAR. For example,
\texttt{.scss} files are compiled into \texttt{.css} files, which are
included in your module's JAR file artifact. This is done by the
\texttt{buildCSS} task.

\subsection{Where Is This Sample?}\label{where-is-this-sample-27}

There are two different versions of this sample, each built with a
different build tool:

\begin{itemize}
\tightlist
\item
  \href{https://github.com/liferay/liferay-blade-samples/tree/7.1/gradle/ext/login-web-ext}{Gradle}
\item
  \href{https://github.com/liferay/liferay-blade-samples/tree/7.1/liferay-workspace/ext/login-web-ext}{Liferay
  Workspace}
\end{itemize}

\section{Felix Gogo Shell}\label{felix-gogo-shell}

To interact with Liferay DXP's module framework, you can leverage the
Gogo shell portlet. You can access this portlet in the Control Panel →
\emph{Configuration} → \emph{Gogo Shell}.

\noindent\hrulefill

\textbf{Note:} You can also interact with Liferay DXP's module framework
via a local telnet session. To do this, you must have
\href{/docs/7-1/tutorials/-/knowledge_base/t/using-developer-mode-with-themes\#setting-developer-mode-for-your-server-using-portal-developerproperties}{Developer
Mode enabled}.

To open the Gogo shell via telnet, execute the following command:

\begin{verbatim}
 telnet localhost 11311
\end{verbatim}

Running this command requires a local running instance of Liferay DXP
and your machine's telnet command line utilities enabled.

To disconnect the session, execute the \texttt{disconnect} command.
Avoid using the following commands, which stop the OSGi framework:

\begin{itemize}
\tightlist
\item
  \texttt{close}
\item
  \texttt{exit}
\item
  \texttt{shutdown}
\end{itemize}

If you have
\href{/docs/7-1/tutorials/-/knowledge_base/t/blade-cli}{Blade CLI}
installed and the telnet capability enabled, you can run the Gogo shell
via Blade command too:

\begin{verbatim}
 blade sh <gogoShellCommand>
\end{verbatim}

\noindent\hrulefill

Here are some useful Gogo shell commands:

\texttt{b\ {[}BUNDLE\_ID{]}}: lists information about a specific bundle
including the bundle's symbolic name, bundle ID, data root, registered
(provided) and used services, imported and exported packages, and more

\texttt{diag\ {[}BUNDLE\_ID{]}}: lists information about why the
specified bundle is not working (e.g., unresolved dependencies, etc.)

\texttt{headers\ {[}BUNDLE\_ID{]}}: lists metadata about the bundle from
the bundle's \texttt{MANIFEST.MF} file

\texttt{help}: lists all the available Gogo shell commands. Notice that
each command has two parts to its name, separated by a colon. For
example, the full name of the \texttt{help} command is
\texttt{felix:help}. The first part is the command scope while the
second part is the command function. The scope allows commands with the
same name to be disambiguated. E.g., scope allows the
\texttt{felix:refresh} command to be distinguished from the
\texttt{equinox:refresh} command.

\texttt{help\ {[}COMMAND\_NAME{]}}: lists information about a specific
command including a description of the command, the scope of the
command, and information about any flags or parameters that can be
supplied when invoking the command.

\texttt{inspect\ capability\ service\ {[}BUNDLE\_ID{]}}: lists services
exposed by a bundle

\texttt{install\ {[}PATH\_TO\_JAR\_FILE{]}}: installs the specified
bundle into Liferay's module framework

\texttt{lb}: lists all of the bundles installed in Liferay's module
framework. Use the \texttt{-s} flag to list the bundles using the
bundles' symbolic names.

\texttt{packages\ {[}PACKAGE\_NAME{]}}: lists all of the named package's
dependencies

\texttt{scr:list}: lists all of the components registered in the module
framework (\emph{scr} stands for service component runtime)

\texttt{scr:info\ {[}COMPONENT\_NAME{]}}: lists information about a
specific component including the component's description, services,
properties, configuration, references, and more.

\texttt{services}: lists all of the services that have been registered
in Liferay's module framework

\texttt{start\ {[}BUNDLE\_ID{]}}: starts the specified bundle

\texttt{stop\ {[}BUNDLE\_ID{]}}: stops the specified bundle

\texttt{system:getproperties}: lists all of the system properties

\texttt{uninstall\ {[}BUNDLE\_ID{]}}: uninstalls the specified bundle
from Liferay's module framework. This does not remove the specified
bundle from Liferay's module framework; it's hidden from Gogo's
\texttt{lb} command, but is still present. Adding a new version of the
uninstalled bundle, therefore, will not reinstall it; it will update the
currently hidden uninstalled version. To remove a bundle from Liferay's
module framework permanently, manually delete it from the
\texttt{LIFERAY\_HOME/osgi} folder. For more information on the
\texttt{uninstall} command, see OSGi's
\href{https://osgi.org/javadoc/r6/core/org/osgi/framework/Bundle.html\#uninstall()}{uninstall}
documentation.

For more information about the Gogo shell, visit
\href{http://felix.apache.org/documentation/subprojects/apache-felix-gogo.html}{Apache's
official documentation}.

\chapter{Liferay Faces}\label{liferay-faces}

Liferay Faces is an umbrella project that provides support for the
JavaServer™ Faces (JSF) standard within Liferay Portal. It encompasses
the following projects:

\begin{itemize}
\tightlist
\item
  \href{/docs/7-1/reference/-/knowledge_base/r/understanding-liferay-faces-bridge}{Liferay
  Faces Bridge} enables you to deploy JSF web apps as portlets without
  writing portlet-specific Java code. It also contains innovative
  features that make it possible to leverage the power of JSF 2.x inside
  a portlet application. Liferay Faces Bridge implements the JSR 329
  Portlet Bridge Standard.
\item
  \href{/docs/7-1/reference/-/knowledge_base/r/understanding-liferay-faces-alloy}{Liferay
  Faces Alloy} enables you to use AlloyUI components in a way that is
  consistent with JSF development.
\item
  \href{/docs/7-1/reference/-/knowledge_base/r/understanding-liferay-faces-portal}{Liferay
  Faces Portal} enables you to leverage Liferay-specific utilities and
  UI components in JSF portlets.
\end{itemize}

In this section of reference documentation, you'll learn more about each
of these projects. You'll also learn about the Liferay Faces version
scheme.

\section{Liferay Faces Version
Scheme}\label{liferay-faces-version-scheme}

In this article, you'll learn which Liferay Faces artifacts should be
used with your portlet and explore the Liferay Faces versioning scheme
by discovering what each component of a version means. Once you have the
versioning scheme mastered, you can view several example configurations.

\subsection{Using The Liferay Faces Archetype
Portlet}\label{using-the-liferay-faces-archetype-portlet}

The \href{http://liferayfaces.org}{Liferay Faces Archetype portlet} can
be used to determine the Liferay Faces artifacts and versions that you
must include in your portlet. Select your preferred Liferay Portal
version, JSF version, component suite (optional), and build tool, and
the portlet will provide you with both a command to generate your
portlet from a Maven archetype and a list of dependencies that can be
copied into your build files. In the next section, you'll be provided
with compatibility information about each version of the Liferay Faces
artifacts.

\subsection{Liferay Faces Alloy}\label{liferay-faces-alloy}

Provides a suite of JSF components that utilize
\href{http://alloyui.com/}{AlloyUI}.

\noindent\hrulefill

Branch\textbar Example Artifact\textbar AlloyUI\textbar JSF
API\textbar Additional Info\textbar{}
\href{https://github.com/liferay/liferay-faces-alloy/tree/master}{master
(3.x)}\textbar com.liferay.faces.alloy-3.0.1.jar\textbar3.0.x\textbar2.2+\textbar{}\emph{AlloyUI
3.0.x is the version that comes bundled with Liferay Portal
7.0+.}\textbar{}
\href{https://github.com/liferay/liferay-faces-alloy/tree/2.x}{2.x}\textbar com.liferay.faces.alloy-2.0.1.jar\textbar2.0.x\textbar2.1+\textbar{}\emph{AlloyUI
2.0.x is the version that comes bundled with Liferay Portal
6.2.}\textbar{}
\href{https://github.com/liferay/liferay-faces-alloy/tree/1.x}{1.x}\textbar com.liferay.faces.alloy-1.0.1.jar\textbar2.0.x\textbar1.2\textbar{}\emph{AlloyUI
2.0.x is the version that comes bundled with Liferay Portal
6.2.}\textbar{}

\noindent\hrulefill

\subsection{Liferay Faces Bridge}\label{liferay-faces-bridge}

Provides the ability to deploy JSF web applications as portlets within
\href{https://portals.apache.org/pluto/}{Apache Pluto}, the reference
implementation for JSR 286 (Portlet 2.0) and JSR 362 (Portlet 3.0).

\noindent\hrulefill

Branch\textbar Example Artifacts\textbar Portlet API\textbar JSF
API\textbar JCP Specification\textbar Additional Info\textbar{} API:
\href{https://github.com/liferay/liferay-faces-bridge-api/tree/5.x}{5.x}IMPL:
\href{https://github.com/liferay/liferay-faces-bridge-impl/tree/5.x}{5.x}\textbar com.liferay.faces.bridge.api-5.0.0.jarcom.liferay.faces.bridge.impl-5.0.0.jar\textbar3.0\textbar2.2\textbar{}\href{https://www.jcp.org/en/jsr/detail?id=378}{JSR
378}\textbar{}\emph{The Expert Group began work in September 2015 and
the Specification is currently under development.}\textbar{} API:
\href{https://github.com/liferay/liferay-faces-bridge-api/tree/4.x}{4.x}IMPL:
\href{https://github.com/liferay/liferay-faces-bridge-impl/tree/4.x}{4.x}\textbar com.liferay.faces.bridge.api-4.1.0.jarcom.liferay.faces.bridge.impl-4.0.0.jar\textbar2.0\textbar2.2\textbar{}\href{https://www.jcp.org/en/jsr/detail?id=329}{JSR
329}\textbar{}\emph{Includes non-standard bridge extensions for JSF
2.2.}\textbar{} API:
\href{https://github.com/liferay/liferay-faces-bridge-api/tree/3.x}{3.x}IMPL:
\href{https://github.com/liferay/liferay-faces-bridge-impl/tree/3.x}{3.x}\textbar com.liferay.faces.bridge.api-3.1.0.jarcom.liferay.faces.bridge.impl-3.0.0.jar\textbar2.0\textbar2.1\textbar{}\href{https://www.jcp.org/en/jsr/detail?id=329}{JSR
329}\textbar{}\emph{Includes non-standard bridge extensions for JSF
2.1.}\textbar{} API:
\href{https://github.com/liferay/liferay-faces-bridge-api/tree/2.x}{2.x}IMPL:
\href{https://github.com/liferay/liferay-faces-bridge-impl/tree/2.x}{2.x}\textbar com.liferay.faces.bridge.api-2.1.0.jarcom.liferay.faces.bridge.impl-2.0.0.jar\textbar2.0\textbar1.2\textbar{}\href{https://www.jcp.org/en/jsr/detail?id=329}{JSR
329} (MR1)\textbar{}\emph{Includes support for Maintenance Release 1
(MR1).}\textbar{}
1.x\textbar N/A\textbar1.0\textbar1.2\textbar{}\href{https://www.jcp.org/en/jsr/detail?id=301}{JSR
301}\textbar{}\emph{N/A (Not Applicable) since Liferay Faces Bridge has
never implemented JSR 301.}\textbar{}

\noindent\hrulefill

\subsection{Liferay Faces Bridge Ext}\label{liferay-faces-bridge-ext}

Extension to Liferay Faces Bridge that provides compatibility with
\href{https://liferay.dev/-/portal}{Liferay Portal} and also takes
advantage of Liferay-specific features such as friendly URLs.

\noindent\hrulefill

Branch \textbar Example Artifact \textbar~~Liferay Portal
API~~\textbar~~Bridge API~~\textbar~~Portlet API~~\textbar JSF
API\textbar{}
\href{https://github.com/liferay/liferay-faces-bridge-ext/tree/master}{8.x}\textbar com.liferay.faces.bridge.ext-8.0.0.jar\textbar7.3.0+\textbar5.x\textbar3.0\textbar2.3\textbar{}
\href{https://github.com/liferay/liferay-faces-bridge-ext/tree/7.x}{7.x}\textbar com.liferay.faces.bridge.ext-7.0.0.jar\textbar7.3.0+\textbar5.x\textbar3.0\textbar2.2\textbar{}
\href{https://github.com/liferay/liferay-faces-bridge-ext/tree/6.x}{6.x}\textbar com.liferay.faces.bridge.ext-6.0.0.jar\textbar7.3.0+\textbar4.x\textbar2.0\textbar2.2\textbar{}
\href{https://github.com/liferay/liferay-faces-bridge-ext/tree/5.x}{5.x}\textbar com.liferay.faces.bridge.ext-5.0.4.jar\textbar7.0.x/7.1.x/7.2.x\textbar4.x\textbar2.0\textbar2.2\textbar{}
\href{https://github.com/liferay/liferay-faces-bridge-ext/tree/4.x}{4.x}\textbar UNUSED\textbar N/A\textbar N/A\textbar N/A\textbar N/A\textbar{}
\href{https://github.com/liferay/liferay-faces-bridge-ext/tree/3.x}{3.x}\textbar com.liferay.faces.bridge.ext-3.0.1.jar\textbar6.2.x\textbar4.x\textbar2.0\textbar2.2\textbar{}
\href{https://github.com/liferay/liferay-faces-bridge-ext/tree/2.x}{2.x}\textbar com.liferay.faces.bridge.ext-2.0.1.jar\textbar6.2.x\textbar3.x\textbar2.0\textbar2.1\textbar{}
\href{https://github.com/liferay/liferay-faces-bridge-ext/tree/1.x}{1.x}\textbar com.liferay.faces.bridge.ext-1.0.1.jar\textbar6.2.x\textbar2.x\textbar2.0\textbar1.2\textbar{}

\noindent\hrulefill

\subsection{Liferay Faces Portal}\label{liferay-faces-portal}

Provides a suite of JSF components that are based on the JSP tags
provided by \href{https://liferay.dev/-/portal}{Liferay Portal}.

\noindent\hrulefill

Branch\textbar Example Artifact\textbar Liferay Portal
API~~\textbar~~JSF API\textbar{}
\href{https://github.com/liferay/liferay-faces-portal/tree/3.x}{3.x}\textbar com.liferay.faces.portal-3.0.1.jar\textbar7.0.x+\textbar2.2+\textbar{}
\href{https://github.com/liferay/liferay-faces-portal/tree/2.x}{2.x}\textbar com.liferay.faces.portal-2.0.1.jar\textbar6.2.x\textbar2.1+\textbar{}
\href{https://github.com/liferay/liferay-faces-portal/tree/1.x}{1.x}\textbar com.liferay.faces.portal-1.0.1.jar\textbar6.2.x\textbar1.2\textbar{}

\noindent\hrulefill

\subsection{Liferay Faces Util}\label{liferay-faces-util}

Library that contains general purpose JSF utilities to support many of
the sub-projects that comprise Liferay Faces.

\noindent\hrulefill

Branch\textbar Example Artifact\textbar~~JSF API\textbar{}
\href{https://github.com/liferay/liferay-faces-util/tree/4.x}{4.x}\textbar com.liferay.faces.util-3.1.0.jar\textbar2.3\textbar{}
\href{https://github.com/liferay/liferay-faces-util/tree/3.x}{3.x}\textbar com.liferay.faces.util-3.1.0.jar\textbar2.2\textbar{}
\href{https://github.com/liferay/liferay-faces-util/tree/2.x}{2.x}\textbar com.liferay.faces.util-2.1.0.jar\textbar2.1\textbar{}
\href{https://github.com/liferay/liferay-faces-util/tree/1.x}{1.x}\textbar com.liferay.faces.util-1.1.0.jar\textbar1.2\textbar{}

\noindent\hrulefill

Now that you know all about the Liferay Faces versioning scheme, you may
be curious as to how these components interact with each other. Refer to
the following figure to view the Liferay Faces dependency diagram.

\begin{figure}
\centering
\includegraphics{./images/liferay-faces-dependency-diagram.png}
\caption{The Liferay Faces dependency diagram helps visualize how
components interact and depend on each other.}
\end{figure}

Next, you can view some example configurations to see the new versioning
scheme in action.

\section{Understanding Liferay Faces
Bridge}\label{understanding-liferay-faces-bridge}

The Liferay Faces Bridge enables you to deploy JSF web apps as portlets
without writing portlet-specific code. It also contains innovative
features that make it possible to leverage the power of JSF 2.x inside a
portlet application.

Liferay Faces Bridge is distributed in a \texttt{.jar} file. You can add
Liferay Faces Bridge as a dependency to your portlet projects, in order
to deploy your JSF web applications as portlets within JSR 286 (Portlet
2.0) compliant portlet containers, like Liferay Portal 5.2, 6.0, 6.1,
6.2, and 7.0.

The Liferay Faces Bridge project home page can be found
\href{https://community.liferay.com/-/faces}{here}.

To fully understand Liferay Faces Bridge, you must first understand the
portlet bridge standard. Because the Portlet 1.0 and JSF 1.0 specs were
being created at essentially the same time, the Expert Group (EG) for
the JSF specification constructed the JSF framework to be compliant with
portlets. For example, the
\href{https://javaee.github.io/javaee-spec/javadocs/javax/faces/context/ExternalContext.html\#getRequest--}{ExternalContext.getRequest()}
method returns an \texttt{Object} instead of an
\href{https://javaee.github.io/javaee-spec/javadocs/javax/servlet/http/HttpServletRequest.html}{javax.servlet.http.HttpServletRequest}.
When this method is used in a portal, the \texttt{Object} can be cast to
a
\href{http://portals.apache.org/pluto/portlet-2.0-apidocs/javax/portlet/PortletRequest.html}{javax.portlet.PortletRequest}.
Despite the EG's consciousness of portlet compatibility within the
design of JSF, the gap between the portlet and JSF lifecycles had to be
bridged.

Portlet bridge standards and implementations evolved over time.

Starting in 2004, several different JSF portlet bridge implementations
were developed in order to provide JSF developers with the ability to
deploy their JSF web apps as portlets. In 2006, the JCP formed the
Portlet Bridge 1.0 (\href{http://www.jcp.org/en/jsr/detail?id=301}{JSR
301}) EG in order to define a standard bridge API, as well as detailed
requirements for bridge implementations. JSR 301 was released in 2010,
targeting Portlet 1.0 and JSF 1.2.

When the Portlet 2.0 (\href{http://www.jcp.org/en/jsr/detail?id=286}{JSR
286}) standard was released in 2008, it became necessary for the JCP to
form the Portlet Bridge 2.0
(\href{http://www.jcp.org/en/jsr/detail?id=329}{JSR 329}) EG. JSR 329
was also released in 2010, targeting Portlet 2.0 and JSF 1.2.

After the \href{http://www.jcp.org/en/jsr/detail?id=314}{JSR 314} EG
released JSF 2.0 in 2009 and JSF 2.1 in 2010, it became evident that a
Portlet Bridge 3.0 standard would be beneficial. In 2015 the JCP formed
\href{http://www.jcp.org/en/jsr/detail?id=378}{JSR 378}) which is
defining a bridge for Portlet 3.0 and JSF 2.2. There are also variants
of \emph{Liferay Faces Bridge} that support Portlet 2.0 and JSF
1.2/2.1/2.2.

Liferay Faces Bridge is the Reference Implementation (RI) of the Portlet
Bridge Standard. It also contains innovative features that make it
possible to leverage the power of JSF 2.x inside a portlet application.

Now that you're familiar with some of the history of the Portlet Bridge
standards, you'll learn about the responsibilities required of the
portlet bridge.

A JSF portlet bridge aligns the correct phases of the JSF lifecycle with
each phase of the portlet lifecycle. For instance, if a browser sends an
HTTP GET request to a portal page with a JSF portlet in it, the
\texttt{RENDER\_PHASE} is perfomed in the portlet's lifecycle. The JSF
portlet bridge then initiates the \texttt{RESTORE\_VIEW} and
\texttt{RENDER\_RESPONSE} phases in the JSF lifecycle. Likewise, when an
HTTP POST is executed on a portlet and the portlet enters the
\texttt{ACTION\_PHASE}, then the full JSF lifecycle is initiated by the
bridge.

\begin{figure}
\centering
\includegraphics{./images/lifecycle-bridge.png}
\caption{The different phases of the JSF Lifecycle are executed
depending on which phase of the Portlet lifecycle is being executed.}
\end{figure}

Besides ensuring that the two lifecycles connect correctly, the JSF
portlet bridge also acts as a mediator between the portal URL generator
and JSF navigation rules. JSF portlet bridges ensure that URLs created
by the portal comply with JSF navigation rules, so that a JSF portlet is
able to switch to different views.

The JSR 329/378 standards defines several configuration options prefixed
with the \texttt{javax.portlet.faces} namespace. Liferay Faces Bridge
defines additional implementation-specific options prefixed with the
\texttt{com.liferay.faces.bridge} namespace.

Liferay Faces Bridge is an essential part of the JSF development process
for Liferay DXP. Visit the
\href{/docs/7-1/tutorials/-/knowledge_base/t/jsf-portlets-with-liferay-faces}{JSF
Portlets with Liferay Faces} section of tutorials for more information
on JSF development for Liferay DXP.

\subsection{Related Topics}\label{related-topics}

\href{/docs/7-1/reference/-/knowledge_base/r/understanding-liferay-faces-alloy}{Understanding
Liferay Faces Alloy}

\href{/docs/7-1/reference/-/knowledge_base/r/understanding-liferay-faces-portal}{Understanding
Liferay Faces Portal}

\href{/docs/7-1/tutorials/-/knowledge_base/t/what-is-service-builder}{What
is Service Builder?}

\section{Understanding Liferay Faces
Alloy}\label{understanding-liferay-faces-alloy}

Liferay Faces Alloy is distributed in a \texttt{.jar} file. You can add
Liferay Faces Alloy as a dependency to your portlet projects, in order
to use AlloyUI in a way that is consistent with JSF development.

During the creation of a JSF portlet in Liferay IDE/Developer Studio,
you have the option of choosing the portlet's JSF Component Suite. The
options include \emph{JSF standard},
\href{http://www.icesoft.org/java/projects/ICEfaces/overview.jsf}{\emph{ICEfaces}},
\href{http://primefaces.org/}{\emph{PrimeFaces}},
\href{http://richfaces.jboss.org/}{\emph{RichFaces}}, and \emph{Liferay
Faces Alloy}.

If you selected the Liferay Faces Alloy JSF Component Suite during your
portlet's setup, the \texttt{.jar} file is included in your portlet
project.

The Liferay Faces Alloy project provides a set of UI components that
utilize AlloyUI. For example, a brief list of some of the supported
\texttt{aui:} tags are listed below:

\begin{itemize}
\tightlist
\item
  Input: \texttt{alloy:inputText}, \texttt{alloy:inputDate},
  \texttt{alloy:inputFile}
\item
  Panel: \texttt{alloy:accordion}, \texttt{alloy:column},
  \texttt{alloy:fieldset}, \texttt{alloy:row}
\item
  Select: \texttt{alloy:selectOneMenu}, \texttt{alloy:selectOneRadio},
  \texttt{alloy:selectStarRating}
\end{itemize}

If you want to utilize Liferay's AlloyUI technology based on YUI3, you
must include the Liferay Faces Alloy \texttt{.jar} file in your JSF
portlet project. If you selected \emph{Liferay Faces Alloy} during your
JSF portlet's setup, you have Liferay Faces Alloy preconfigured in your
project, so you're automatically able to use the \texttt{alloy:} tags.

As you can see, it's extremely easy to configure your JSF application to
use Liferay's AlloyUI tags.

\subsection{Related Topics}\label{related-topics-1}

\href{/docs/7-1/tutorials/-/knowledge_base/t/creating-a-jsf-project-manually}{Creating
a JSF Project Manually}

\href{/docs/7-1/reference/-/knowledge_base/r/understanding-liferay-faces-bridge}{Understanding
Liferay Faces Bridge}

\href{/docs/7-1/reference/-/knowledge_base/r/understanding-liferay-faces-portal}{Understanding
Liferay Faces Portal}

\section{Understanding Liferay Faces
Portal}\label{understanding-liferay-faces-portal}

\emph{Liferay Faces Portal} is distributed in a \texttt{.jar} file. You
can add Liferay Faces Portal as a dependency for your portlet projects
to use its Liferay-specific utilities and UI components. When Liferay
Faces Portal is included in a JSF portlet project, the
\texttt{com.liferay.faces.portal.{[}version{]}.jar} file resides in the
portlet's library.

\begin{figure}
\centering
\includegraphics{./images/jsf-jars-package-explorer.png}
\caption{The required \texttt{.jar} files are downloaded for your JSF
portlet based on the JSF UI Component Suite you configured.}
\end{figure}

Some of the features included in Liferay Faces Portal are:

\begin{itemize}
\item
  Utilities: Provides the \texttt{LiferayPortletHelperUtil} which
  contains a variety Portlet-API and Liferay-specific convenience
  methods.
\item
  JSF Components: Provides a set of JSF equivalents for popular Liferay
  DXP JSP tags (not exhaustive):

  \begin{itemize}
  \tightlist
  \item
    \texttt{liferay-ui:captcha} → \texttt{portal:captcha}
  \item
    \texttt{liferay-ui:input-editor} → \texttt{portal:inputRichText}
  \item
    \texttt{liferay-ui:search} → \texttt{portal:inputSearch}
  \item
    \texttt{liferay-ui:header} → \texttt{portal:header}
  \item
    \texttt{aui:nav} → \texttt{portal:nav}
  \item
    \texttt{aui:nav-item} → \texttt{portal:navItem}
  \item
    \texttt{aui:nav-bar} → \texttt{portal:navBar}
  \item
    \texttt{liferay-security:permissionsURL} →
    \texttt{portal:permissionsURL}
  \item
    \texttt{liferay-portlet:runtime} → \texttt{portal:runtime}
  \end{itemize}

  For more information, visit
  \url{https://liferayfaces.org/web/guest/portal-showcase}.
\item
  Expression Language: Adds a set of EL keywords such as
  \texttt{liferay} for getting Liferay-specific info, and \texttt{i18n}
  for integration with out-of-the-box Liferay internationalized
  messages.
\end{itemize}

Great! You now have an understanding of what Liferay Faces Portal is,
and what it accomplishes in your JSF application.

\subsection{Related Topics}\label{related-topics-2}

\href{/docs/7-1/tutorials/-/knowledge_base/t/creating-a-jsf-project-manually}{Creating
a JSF Project Manually}

\href{/docs/7-1/reference/-/knowledge_base/r/understanding-liferay-faces-bridge}{Understanding
Liferay Faces Bridge}

\href{/docs/7-1/reference/-/knowledge_base/r/understanding-liferay-faces-alloy}{Understanding
Liferay Faces Alloy}

\section{Page Fragments}\label{page-fragments}

Page Fragments are templates made up of CSS, HTML, and JavaScript used
to build Content Pages. The HTML, CSS, and JavaScript are all completely
standard, just like anywhere else on the web, but are also enhanced with
Liferay-specific features. The articles in this section provide
additional information about the Liferay-specific features of Page
Fragments.

\section{Embedding Widgets in Page
Fragments}\label{embedding-widgets-in-page-fragments}

You can embed both a selection of Liferay widgets and your own custom
widgets in Page Fragments. For a more information on embedding custom
widgets, see the
\href{/docs/7-1/tutorials/-/knowledge_base/t/fragment-specific-tags}{Fragment
Specific Tags} tutorial.

\subsection{Embedding Liferay Widgets}\label{embedding-liferay-widgets}

Many Liferay widgets can be embedded in Page Fragments. Each embeddable
widget has a specific tag for use in fragments which looks like
\texttt{\textless{}lfr-widget-{[}widget-name{]}\textgreater{}}. When you
embed the widget, the complete opening and closing tags must be used
like this:

\begin{verbatim}
<lfr-widget-[widget-name]>
</lfr-widget-[widget-name]>
\end{verbatim}

Here is the full list of widgets that can be embedded:

\noindent\hrulefill

\begin{longtable}[]{@{}ll@{}}
\toprule\noalign{}
Widget Name & Tag \\
\midrule\noalign{}
\endhead
\bottomrule\noalign{}
\endlastfoot
DDL Display &
\texttt{\textless{}lfr-widget-dynamic-data-list\textgreater{}} \\
Form & \texttt{\textless{}lfr-widget-form\textgreater{}} \\
Asset Publisher &
\texttt{\textless{}lfr-widget-asset-list\textgreater{}} \\
Breadcrumb & \texttt{\textless{}lfr-widget-breadcrumb\textgreater{}} \\
Categories Navigation &
\texttt{\textless{}lfr-widget-categories-nav\textgreater{}} \\
Flash & \texttt{\textless{}lfr-widget-flash\textgreater{}} \\
Media Gallery &
\texttt{\textless{}lfr-widget-media-gallery\textgreater{}} \\
Navigation Menu & \texttt{\textless{}lfr-widget-nav\textgreater{}} \\
Polls Display & \texttt{\textless{}lfr-widget-polls\textgreater{}} \\
Related Assets &
\texttt{\textless{}lfr-widget-related-assets\textgreater{}} \\
Site Map & \texttt{\textless{}lfr-widget-site-map\textgreater{}} \\
Tag Cloud & \texttt{\textless{}lfr-widget-tag-cloud\textgreater{}} \\
Tags Navigation &
\texttt{\textless{}lfr-widget-tags-nav\textgreater{}} \\
Web Content Display &
\texttt{\textless{}lfr-widget-web-content\textgreater{}} \\
Rss Publisher (Deprecated) &
\texttt{\textless{}lfr-widget-rss\textgreater{}} \\
Iframe & \texttt{\textless{}lfr-widget-iframe\textgreater{}} \\
\end{longtable}

\noindent\hrulefill

\section{JSON Web Services Invocation
Examples}\label{json-web-services-invocation-examples}

This tutorial provides examples of invoking Liferay's JSON web services
via JavaScript, URL, and \href{http://curl.haxx.se}{cURL}. To illustrate
the differences between these, the same two use cases (getting a user
and adding a user) are shown in each example. This tutorial also
includes an example of using JavaScript to invoke Liferay's JSON web
services from a portlet.

\subsection{Get User JSON Web Service Invocation via
JavaScript}\label{get-user-json-web-service-invocation-via-javascript}

Among the JavaScript objects that Liferay creates for each page is a
\texttt{Liferay} object. This object includes a \texttt{Service}
function that you can use to invoke Liferay's API.

Examine the following JSON web service invocation, written in
JavaScript:

\begin{verbatim}
<script type="text/javascript">

  Liferay.Service(
      '/user/get-user-by-email-address',
      {
          companyId: Liferay.ThemeDisplay.getCompanyId(),
          emailAddress: 'test@example.com'
      },
      function(obj) {
          console.log(obj);
      }
  );

</script>
\end{verbatim}

When you run this example, the test@example.com user (JSON object) is
returned. You can examine the returned object from your browser console.

The \texttt{Liferay.Service(...)} function takes three arguments:

\begin{enumerate}
\def\labelenumi{\arabic{enumi}.}
\item
  A string representing the service to invoke
\item
  A parameters object
\item
  A callback function
\end{enumerate}

The callback function takes the result of the service invocation as an
argument.

\subsection{Add User JSON Web Service Invocation via
JavaScript}\label{add-user-json-web-service-invocation-via-javascript}

Here's an example JSON web service invocation, also written in
JavaScript, that adds a new user. It requires many more parameters than
the one for retrieving a user!

\begin{verbatim}
Liferay.Service(
    '/user/add-user',
    {
        companyId: Liferay.ThemeDisplay.getCompanyId(),
        autoPassword: false,
        password1: 'test',
        password2: 'test',
        autoScreenName: false,
        screenName: 'joe.bloggs',
        emailAddress: 'joe.bloggs@example.com',
        facebookId: 0,
        openId: '',
        locale: 'en_US',
        firstName: 'Joe',
        middleName: 'T',
        lastName: 'Bloggs',
        prefixId: 0,
        suffixId: 0,
        male: true,
        birthdayMonth: 1,
        birthdayDay: 1,
        birthdayYear: 1970,
        jobTitle: 'Tester',
        groupIds: null,
        organizationIds: null,
        roleIds: null,
        userGroupIds: null,
        sendEmail: false,
        serviceContext: {assetTagNames: ['test']}
    },
    function(obj) {
        console.log(obj);
    }
);
\end{verbatim}

The \texttt{serviceContext} object assigns the \texttt{test} tag to the
newly created user. Note that you can use JSON syntax to supply values
for objects and arrays. For example, to supply a value for the
\texttt{serviceContext} object, you use curly brackets: \texttt{\{} and
\texttt{\}}. To supply a value for the \texttt{assetTagNames} array, you
use square brackets: \texttt{{[}} and \texttt{{]}}. Thus, the line
\texttt{serviceContext:\ \{assetTagNames:\ {[}\textquotesingle{}test\textquotesingle{}{]}\}}
indicates that \texttt{serviceContext} is an object containing an array
named \texttt{assetTagNames}, which contains the string \texttt{test}.

\subsection{Invoking JSON Web Services via JavaScript in an
Application}\label{invoking-json-web-services-via-javascript-in-an-application}

You can adapt the example from the previous section for use in a Liferay
app. For example, the JSP page below creates a form that lets the user
specify a first name, middle name, last name, screen name, and email
address. When the user clicks the \emph{Add User} button, the app uses
these values to create a new user.

\begin{verbatim}
<%@ taglib uri="http://alloy.liferay.com/tld/aui" prefix="aui" %>
<%@ taglib uri="http://java.sun.com/portlet_2_0" prefix="portlet" %>

<portlet:defineObjects />

<portlet:renderURL var="successURL">
    <portlet:param name="mvcPath" value="/success.jsp"/>
</portlet:renderURL>

<portlet:renderURL var="failureURL">
    <portlet:param name="mvcPath" value="/failure.jsp"/>
</portlet:renderURL>

<aui:form method="GET" name="<portlet:namespace />fm">
    <aui:fieldset>
        <aui:input label="First Name" name="first-name"></aui:input>
        <aui:input label="Middle Name" name="middle-name"></aui:input>
        <aui:input label="Last Name" name="last-name"></aui:input>
        <aui:input label="Screen Name" name="screen-name"></aui:input>
        <aui:input label="Email Address" name="email-address"></aui:input>
    </aui:fieldset>

        <p>Click the button below to add a new user by invoking Liferay's JSON web services.</p>

        <aui:button-row>
            <aui:button id="add-user" value="Add User">
            </aui:button>
        </aui:button-row>
</aui:form>

<aui:script use="node, event">
var addUserButton = A.one('#add-user');

var firstNameNode = A.one('#<portlet:namespace />first-name');
var middleNameNode = A.one('#<portlet:namespace />middle-name');
var lastNameNode = A.one('#<portlet:namespace />last-name');
var screenNameNode = A.one('#<portlet:namespace />screen-name');
var emailAddressNode = A.one('#<portlet:namespace />email-address');

addUserButton.on('click', function(event) {
        var firstName = firstNameNode.get('value');
        var middleName = middleNameNode.get('value');
        var lastName = lastNameNode.get('value');
        var screenName = screenNameNode.get('value');
        var emailAddress = emailAddressNode.get('value');

    var user = Liferay.Service(
        '/user/add-user',
        {
            companyId: Liferay.ThemeDisplay.getCompanyId(),
            autoPassword: false,
            password1: 'test',
            password2: 'test',
            autoScreenName: false,
            screenName: screenName,
            emailAddress: emailAddress,
            facebookId: 0,
            openId: '',
            locale: 'en_US',
            firstName: firstName,
            middleName: middleName,
            lastName: lastName,
            prefixId: 0,
            suffixId: 0,
            male: true,
            birthdayMonth: 1,
            birthdayDay: 1,
            birthdayYear: 1970,
            jobTitle: 'Tester',
            groupIds: null,
            organizationIds: null,
            roleIds: null,
            userGroupIds: null,
            sendEmail: false,
            serviceContext: {assetTagNames: ['test']}
        },
        function(obj) {
            console.log(obj);

            if (obj.hasOwnProperty('createDate')) {
                window.open('<%= successURL %>', '_self');
            }
            else {
                window.open('<%= failureURL %>', '_self');
            }
        }
    );
});
</aui:script>
\end{verbatim}

\subsection{Get User JSON Web Service Invocation via
URL}\label{get-user-json-web-service-invocation-via-url}

Here's a simple JSON web service invocation via URL that returns the
user with the specified email address:

\begin{verbatim}
http://localhost:8080/api/jsonws/user/get-user-by-email-address/company-id/20154/email-address/test%40liferay.com?p_auth=[value]
\end{verbatim}

This web service invocation returns the test@example.com user. After
invoking a service via Liferay's JSONWS API page, the URL provided when
you click on the \emph{URL Example} tab omits the \texttt{p\_auth} URL
query parameter. It's assumed that you'll add this parameter yourself.
Remember that you must be logged in as a user with the required
permission to invoke a web service. To find the \texttt{p\_auth} token
that corresponds to your session, see the
\href{/docs/7-1/tutorials/-/knowledge_base/t/invoking-json-web-services}{Invoking
JSON Web Services} tutorial.

As explained in that tutorial, you can supply parameters as either URL
path parameters or URL query parameters. In the preceding example, the
company ID and email address are supplied as URL path parameters. Here's
an equivalent example using URL query parameters:

\begin{verbatim}
http://localhost:8080/api/jsonws/user/get-user-by-email-address?companyId=20154&emailAddress=test@example.com&p_auth=[value]
\end{verbatim}

Next, you'll consider an example that requires many more parameters!

\subsection{Add User JSON Web Service Invocation via
URL}\label{add-user-json-web-service-invocation-via-url}

Here's an example JSON web service invocation via URL that adds a new
user with the specified attributes:

\begin{verbatim}
http://localhost:8080/api/jsonws/user/add-user/company-id/20154/auto-password/false/password1/test/password2/test/auto-screen-name/false/screen-name/joe.bloggs/email-address/joe.bloggs%40liferay.com/facebook-id/0/-open-id/locale/en_US/first-name/Joe/middle-name/T/last-name/Bloggs/prefix-id/0/suffix-id/0/male/true/birthday-month/1/birthday-day/1/birthday-year/1970/job-title/Tester/-group-ids/-organization-ids/-role-ids/-user-group-ids/send-email/false?p_auth=[value]
\end{verbatim}

And here's the same example using URL query parameters instead of URL
path parameters:

\begin{verbatim}
http://localhost:8080/api/jsonws/user/add-user?companyId=20154&autoPassword=false&password1=test&password2=test&autoScreenName=false&screenName=joe.bloggs&emailAddress=joe.bloggs@example.com&facebookId=0&-openId&locale=en_US&firstName=Joe&middleName=T&lastName=Bloggs&prefixId=0&suffixId=0&male=true&birthdayMonth=1&birthdayDay=1&birthdayYear=1970&jobTitle=Tester&-groupIds&-organizationIds&-roleIds&-userGroupIds&sendEmail=false&p_auth=[value]
\end{verbatim}

\subsection{Get User JSON Web Service Invocation via
cURL}\label{get-user-json-web-service-invocation-via-curl}

Here's an example JSON web service invocation via the cURL tool that
returns the user with the specified email address:

\begin{verbatim}
curl http://localhost:8080/api/jsonws/user/get-user-by-email-address \
  -u test@example.com:test \
  -d companyId=20154 \
  -d emailAddress='test@example.com'
\end{verbatim}

Note that cURL is a command line tool. You can execute this command from
a terminal window.

\subsection{Add User JSON Web Service Invocation via
cURL}\label{add-user-json-web-service-invocation-via-curl}

Here's an example JSON web service invocation via the cURL tool that
adds the user with the specified attributes:

\begin{verbatim}
curl http://localhost:8080/api/jsonws/user/add-user \
  -u test@example.com:test \
  -d companyId=20154 \
  -d autoPassword=false \
  -d password1='test' \
  -d password2='test' \
  -d autoScreenName=false \
  -d screenName='joe.bloggs' \
  -d emailAddress='joe.bloggs@example.com' \
  -d facebookId=0 \
  -d openId='0' \
  -d locale=en_US \
  -d firstName='Joe' \
  -d middleName='T' \
  -d lastName='Bloggs' \
  -d prefixId=0 \
  -d suffixId=0 \
  -d male=true \
  -d birthdayMonth=1 \
  -d birthdayDay=1 \
  -d birthdayYear=1970 \
  -d jobTitle='Tester' \
  -d groupIds= \
  -d organizationIds= \
  -d roleIds= \
  -d userGroupIds= \
  -d sendEmail=false
\end{verbatim}

Great! Now you've seen how to invoke Liferay's JSON web services from
JavaScript, URL, and cURL.

\subsection{Related Topics}\label{related-topics-3}

\href{/docs/7-1/tutorials/-/knowledge_base/t/invoking-json-web-services}{Invoking
JSON Web Services}

\href{/docs/7-1/tutorials/-/knowledge_base/t/json-web-services-invoker}{JSON
Web Services Invoker}

\href{/docs/7-1/tutorials/-/knowledge_base/t/invoking-remote-services}{Invoking
Remote Services}

\chapter{Customizing Core Functionality with
Ext}\label{customizing-core-functionality-with-ext}

\noindent\hrulefill

\textbf{Ext plugins are deprecated for 7.0 and should only be used if
absolutely necessary.}

The following app servers should be used for Ext plugin development in
Liferay DXP:

\begin{itemize}
\tightlist
\item
  Tomcat 9.0
\end{itemize}

In most cases, Ext plugins are not necessary. There are, however,
certain cases that require the use of an Ext plugin. Liferay only
supports the following Ext plugin use cases:

\begin{itemize}
\tightlist
\item
  Providing custom implementations for any beans declared in Liferay
  DXP's Spring files (when possible, use
  \href{/docs/7-1/tutorials/-/knowledge_base/t/customizing-liferay-services-service-wrappers}{service
  wrappers} instead of an Ext plugin). 7.0 removed many beans, so make
  sure your overridden beans are still relevant if converting your
  legacy Ext plugin
  (\href{/docs/7-1/reference/-/knowledge_base/r/extending-core-classes-using-spring-with-ext-plugins}{how
  to}).
\item
  Overwriting a class in a 7.0 core JAR. For a list of core JARs, see
  the
  \href{/docs/7-1/tutorials/-/knowledge_base/t/configuring-dependencies\#finding-core-artifacts}{Finding
  Core Artifacts} section
  (\href{/docs/7-1/reference/-/knowledge_base/r/overriding-core-classes-with-ext-plugins}{how
  to}).
\item
  Modifying Liferay DXP's \texttt{web.xml} file
  (\href{/docs/7-1/reference/-/knowledge_base/r/modifying-the-web-xml-with-ext-plugins}{how
  to}).
\item
  Adding to Liferay DXP's \texttt{web.xml} file
  (\href{/docs/7-1/reference/-/knowledge_base/r/adding-to-the-web-xml-with-ext-plugins}{how
  to}).
\end{itemize}

\textbf{Note:} In previous versions of Liferay Portal, you needed an Ext
plugin to specify classes as portal property values (e.g.,
\texttt{global.starup.events.my.custom.MyStartupAction}), since the
custom class had to be added to the portal class loader. This is no
longer the case in 7.0 since all lifecycle events can use OSGi services
with no need to edit these legacy properties.

\noindent\hrulefill

Ext plugins are used to customize Liferay DXP's core functionality. You
can learn more about what the core encompasses in the
\href{/docs/7-1/tutorials/-/knowledge_base/t/configuring-dependencies\#finding-core-artifacts}{Finding
Core Artifacts} article section. In this section, you'll learn how to

\begin{itemize}
\tightlist
\item
  \href{/docs/7-1/reference/-/knowledge_base/r/creating-an-ext-plugin}{Create
  an Ext plugin}
\item
  \href{/docs/7-1/reference/-/knowledge_base/r/developing-an-ext-plugin}{Develop
  an Ext plugin}
\item
  \href{/docs/7-1/reference/-/knowledge_base/r/deploying-an-ext-plugin}{Deploy
  an Ext plugin}
\item
  \href{/docs/7-1/reference/-/knowledge_base/r/redeploying-an-ext-plugin}{Redeploy
  an Ext plugin}
\end{itemize}

You can also dive into the
\href{/docs/7-1/reference/-/knowledge_base/r/anatomy-of-an-ext-plugin}{Anatomy
of an Ext Plugin} to familiarize yourself with its structure.

You'll start by creating an Ext plugin.

\section{Extending Core Classes Using Spring with Ext
Plugins}\label{extending-core-classes-using-spring-with-ext-plugins}

A supported use case for using Ext plugins in Liferay DXP is extending
its core classes (e.g., \texttt{portal-impl}, \texttt{portal-kernel},
etc.) using Spring. You can reference the
\href{/docs/7-1/tutorials/-/knowledge_base/t/configuring-dependencies\#finding-core-artifacts}{Finding
Core Artifacts} section for help distinguishing core classes. Make sure
you've reviewed the generalized
\href{/docs/7-1/reference/-/knowledge_base/r/customizing-core-functionality-with-ext}{Customization
with Ext Plugins} section before creating an Ext plugin.

As an example, you'll create a sample Ext plugin that extends the
\href{https://docs.liferay.com/ce/portal/7.1-latest/javadocs/portal-impl/com/liferay/portal/util/PortalImpl.html}{PortalImpl}
core class residing in the \texttt{portal-impl.jar}. You'll override the
\texttt{PortalImpl.getComputerName()} method via Spring bean, which
returns your server's node name. The Ext plugin will override this
method and modify the server's returned node name.

\begin{enumerate}
\def\labelenumi{\arabic{enumi}.}
\item
  Navigate to your Liferay Workspace's root folder and run the following
  command:

\begin{verbatim}
blade create -t war-core-ext portal-impl-extend-spring-ext
\end{verbatim}

  Your Ext plugin is generated and now resides in the workspace's
  \texttt{/ext} folder with the name you assigned.
\item
  Displaying the server node name in your Liferay DXP installation is
  set to \texttt{false} by default. You'll need to enable this property.
  To do this, navigate into your Liferay bundle's root folder and create
  a \texttt{portal-ext.properties} file. In that file, insert the
  following property:

\begin{verbatim}
web.server.display.node=true
\end{verbatim}

  Now your server's node name will be displayed once your Liferay bundle
  is restarted.
\item
  In the \texttt{/extImpl/java} folder, create the folder structure
  representing the package name you want your new class to reside in
  (e.g., \texttt{com/liferay/portal/util}). Then create your new Java
  class:

\begin{verbatim}
package com.liferay.portal.util;

public class SamplePortalImpl extends PortalImpl {

    @Override
    public String getComputerName() {
        return "SAMPLE_EXT_INSTALLED_" + super.getComputerName();
    }

}
\end{verbatim}
\end{enumerate}

The method defined in the extension class overrides the
\texttt{PortalImpl.getComputerName()} method. The
\texttt{"SAMPLE\_EXT\_INSTALLED\_"} String is now prefixed to your
server's node name.

\begin{enumerate}
\def\labelenumi{\arabic{enumi}.}
\setcounter{enumi}{3}
\item
  In your Ext plugin's \texttt{/extImpl/resources} folder, create a
  \texttt{META-INF/ext-spring.xml} file. In this file, insert the
  following code:

\begin{verbatim}
<?xml version="1.0"?>

<beans
    default-destroy-method="destroy"
    default-init-method="afterPropertiesSet"
    xmlns="http://www.springframework.org/schema/beans"
    xmlns:xsi="http://www.w3.org/2001/XMLSchema-instance"
    xsi:schemaLocation="http://www.springframework.org/schema/beans http://www.springframework.org/schema/beans/spring-beans-3.0.xsd"
>

    <bean class="com.liferay.portal.util.SamplePortalImpl" id="com.liferay.portal.util.PortalImpl" />
</beans>
\end{verbatim}
\end{enumerate}

Since you plan on modifying a core service class, you can inject its
extension class via a Spring bean. This will ensure your new class is
recognized. Assign your extension class's fully defined class name
(e.g., \texttt{com.liferay.portal.util.SamplePortalImpl}) to the bean
tag's \texttt{class} attribute and the fully defined original class name
(e.g., \texttt{com.liferay.portal.util.PortalImpl}) to the bean tag's
\texttt{id} attribute.

When your Ext plugin is deployed, your new service (e.g.,
\texttt{SamplePortalImpl}) will extend the core \texttt{PortalImpl}
class.

Awesome! You've created an Ext plugin that extends a core class in
Liferay DXP! Follow the instructions in the
\href{/docs/7-1/reference/-/knowledge_base/r/deploying-an-ext-plugin}{Deploy
the Plugin} article to deploy it to your server.

\section{Overriding Core Classes with Ext
Plugins}\label{overriding-core-classes-with-ext-plugins}

A supported use case for using Ext plugins in Liferay DXP is overriding
its core classes (e.g., \texttt{portal-impl}, \texttt{portal-kernel},
etc.). You can reference the
\href{/docs/7-1/tutorials/-/knowledge_base/t/configuring-dependencies\#finding-core-artifacts}{Finding
Core Artifacts} section for help distinguishing core classes. Make sure
you've reviewed the generalized
\href{/docs/7-1/reference/-/knowledge_base/r/customizing-core-functionality-with-ext}{Customization
with Ext Plugins} section before creating an Ext plugin.

As an example, you'll create a sample Ext plugin that overwrites the
\href{https://docs.liferay.com/ce/portal/7.1-latest/javadocs/portal-impl/com/liferay/portal/util/PortalImpl.html}{PortalImpl}
core class residing in the \texttt{portal-impl.jar}. You'll edit the
\texttt{PortalImpl.getComputerName()} method, which returns your
server's node name. The Ext plugin will override the entire
\texttt{PortalImpl} class, adding the method modifying the server's
returned node name.

\begin{enumerate}
\def\labelenumi{\arabic{enumi}.}
\item
  Navigate to your Liferay Workspace's root folder and run the following
  command:

\begin{verbatim}
blade create -t war-core-ext portal-impl-override
\end{verbatim}

  Your Ext plugin is generated and now resides in the workspace's
  \texttt{/ext} folder with the name you assigned.
\item
  Displaying the server node name in your Liferay DXP installation is
  set to \texttt{false} by default. You'll need to enable this property.
  To do this, navigate into your Liferay bundle's root folder and create
  a \texttt{portal-ext.properties} file. In that file, insert the
  following property:

\begin{verbatim}
web.server.display.node=true
\end{verbatim}

  Now your server's node name will be displayed once your Liferay bundle
  is restarted.
\item
  In the \texttt{/extImpl/java} folder, create the folder structure
  matching the class's folder structure you'd like to override (e.g.,
  \texttt{com/liferay/portal/util}). Then create the new Java class that
  will override the existing core class; your new class must have the
  same name as the original.
\item
  Copy all of the original class's (e.g., \texttt{PortalImpl}) logic
  into your new class. Then modify the method you want to customize. For
  this example, you want to edit the \texttt{getComputerName()} method.
  Therefore, replace it with the method below:

\begin{verbatim}
@Override
public String getComputerName() {
    return "sample_portalimpl_ext_installed_successfully_" + _computerName;
}
\end{verbatim}

  The method defined in the new class overrides the
  \texttt{PortalImpl.getComputerName()} method. The
  \texttt{sample\_portalimpl\_ext\_installed\_successfully\_} String is
  now prefixed to your server's node name.
\end{enumerate}

When your Ext plugin is deployed, your new Java class will override the
core \texttt{PortalImpl} class.

Awesome! You've created an Ext plugin that overrides a core class in
Liferay DXP! Follow the instructions in the
\href{/docs/7-1/reference/-/knowledge_base/r/deploying-an-ext-plugin}{Deploy
the Plugin} article to deploy it to your server.

\section{Adding to the web.xml with Ext
Plugins}\label{adding-to-the-web.xml-with-ext-plugins}

A supported use case for using Ext Plugins in Liferay DXP is adding
additional functionality to its \texttt{web.xml} file. Before beginning,
make sure you've reviewed the generalized
\href{/docs/7-1/reference/-/knowledge_base/r/customizing-core-functionality-with-ext}{Customization
with Ext Plugins} section.

As an example, you'll create a sample Ext plugin that adds to your
Liferay DXP's existing \texttt{web.xml} file (e.g., in the
\texttt{/tomcat-{[}version{]}/webapps/ROOT/WEB-INF} folder). You'll add
a new printout in the console during startup.

\begin{enumerate}
\def\labelenumi{\arabic{enumi}.}
\item
  Navigate to your Liferay Workspace's root folder and run the following
  command:

\begin{verbatim}
blade create -t war-core-ext add-printout
\end{verbatim}

  Your Ext plugin is generated and now resides in the workspace's
  \texttt{/ext} folder with the name you assigned.
\item
  For your Liferay DXP installation to recognize new functionality in
  the \texttt{web.xml}, you must create a class that implements the
  \href{https://javaee.github.io/javaee-spec/javadocs/javax/servlet/ServletContextListener.html}{ServletContextListener}
  interface. This class will initialize a servlet context event for
  which you'll add your new functionality. In the \texttt{extImpl/java}
  folder, create the folder structure representing the package name you
  want your new class to reside in (e.g.,
  \texttt{com/liferay/portal/servlet/context}). Then create your new
  Java class:

\begin{verbatim}
package com.liferay.portal.servlet.context;

import javax.servlet.ServletContextEvent;
import javax.servlet.ServletContextListener;

public class ExtAddEntryWebXmlPortalContextLoaderListener
        implements ServletContextListener {

    public void contextDestroyed(ServletContextEvent servletContextEvent) {
    }

    public void contextInitialized(ServletContextEvent servletContextEvent) {
        System.out.println("EXT_ADD_ENTRY_WEBXML_INSTALLED_SUCCESSFULLY");
    }

}
\end{verbatim}

  The above class includes two methods that initialize and destroy your
  servlet context event. Be sure to add the new \texttt{web.xml}'s
  functionality when the portal context is initializing. To add a
  printout verifying the Ext plugins installation, a simple print
  statement was defined in the \texttt{contextInitialized(...)} method:

\begin{verbatim}
System.out.println("EXT_ADD_ENTRY_WEBXML_INSTALLED_SUCCESSFULLY");
\end{verbatim}
\item
  Now that you've defined a servlet context event, you should add a
  listener to your \texttt{web.xml} that listens for it. In the
  \texttt{ext-web/docroot/WEB-INF} folder, open the \texttt{web.xml}
  file, which was generated for you by default.
\item
  Add the following tag between the tags:

\begin{verbatim}
<listener>
    <listener-class>com.liferay.portal.servlet.context.ExtAddEntryWebXmlPortalContextLoaderListener</listener-class>
</listener>
\end{verbatim}
\end{enumerate}

Excellent! Now when your Ext plugin is deployed, your Liferay DXP
installation will create a \texttt{ServletContextListener} instance,
which will initialize a custom servlet context event. This event will be
recognized by the \texttt{web.xml} file, which will add the new
functionality to your Liferay DXP installation. Follow the instructions
in the
\href{/docs/7-1/reference/-/knowledge_base/r/deploying-an-ext-plugin}{Deploy
the Plugin} article for help deploying the Ext plugin to your server.

\section{Modifying the web.xml with Ext
Plugins}\label{modifying-the-web.xml-with-ext-plugins}

A supported use case for using Ext Plugins in Liferay DXP is modifying
its \texttt{web.xml} file. Before beginning, make sure you've reviewed
the generalized
\href{/docs/7-1/reference/-/knowledge_base/r/customizing-core-functionality-with-ext}{Customization
with Ext Plugins} section.

As an example, you'll create a sample Ext plugin that modifies Liferay
DXP's existing \texttt{web.xml\ file} (e.g., in the
\texttt{/tomcat-{[}version{]}/webapps/ROOT/WEB-INF} folder). You'll
modify the session timeout configuration, which is set to 30 (minutes)
by default:

\begin{verbatim}
<session-config>
    <session-timeout>30</session-timeout>
    <cookie-config>
        <http-only>true</http-only>
    </cookie-config>
</session-config>
\end{verbatim}

The Ext plugin will update the session timeout to one minute.

\begin{enumerate}
\def\labelenumi{\arabic{enumi}.}
\item
  Navigate into your Liferay Workspace's \texttt{/ext} folder and run
  the following command:

\begin{verbatim}
blade create -t war-core-ext modify-session-timeout
\end{verbatim}

  Your Ext plugin is generated and now resides in the workspace's
  \texttt{/ext} folder with the name you assigned.
\item
  In the \texttt{ext-web/docroot/WEB-INF} folder, open the
  \texttt{web.xml} file, which was generated for you by default.
\item
  Insert the following logic between the
  \texttt{\textless{}web-app\textgreater{}} tags:

\begin{verbatim}
 <session-config>
     <session-timeout>1</session-timeout>
     <cookie-config>
         <http-only>true</http-only>
     </cookie-config>
 </session-config>
\end{verbatim}
\end{enumerate}

Notice that the \texttt{\textless{}session-timeout\textgreater{}} tag
has been updated to \texttt{1}.

That's it! Now when your Ext plugin is deployed, your Liferay DXP
installation will timeout after one minute of inactivity. Follow the
instructions in the
\href{/docs/7-1/reference/-/knowledge_base/r/deploying-an-ext-plugin}{Deploy
the Plugin} article for help deploying the Ext plugin to your server.

\section{Item Selector Criterion and Return
Types}\label{item-selector-criterion-and-return-types}

Liferay DXP bundles have apps and app suites containing
\href{https://docs.liferay.com/dxp/apps/collaboration/latest/javadocs/com/liferay/item/selector/ItemSelectorCriterion.html}{\texttt{ItemSelectorCriterion}
classes} and
\href{https://docs.liferay.com/dxp/apps/collaboration/latest/javadocs/com/liferay/item/selector/ItemSelectorReturnType.html}{\texttt{ItemSelectorReturnType}
classes} developers can use.

\subsection{Item Selector Criterion
Classes}\label{item-selector-criterion-classes}

\textbf{Collaboration App Suite Modules:}

\begin{itemize}
\item
  \texttt{com.liferay.item.selector.criteria.api}:

  \begin{itemize}
  \item
    \href{https://docs.liferay.com/dxp/apps/collaboration/latest/javadocs/com/liferay/item/selector/criteria/image/criterion/ImageItemSelectorCriterion.html}{ImageItemSelectorCriterion}:
    Image file entity type.
  \item
    \href{https://docs.liferay.com/dxp/apps/collaboration/latest/javadocs/com/liferay/item/selector/criteria/audio/criterion/AudioItemSelectorCriterion.html}{AudioItemSelectorCriterion}:
    Audio file entity type.
  \item
    \href{https://docs.liferay.com/dxp/apps/collaboration/latest/javadocs/com/liferay/item/selector/criteria/criteria/file/criterion/FileItemSelectorCriterion.html}{FileItemSelectorCriterion}:
    Document Library file entity type.
  \item
    \href{https://docs.liferay.com/dxp/apps/collaboration/latest/javadocs/com/liferay/item/selector/criteria/upload/criterion/UploadItemSelectorCriterion.html}{UploadItemSelectorCriterion}:
    Uploadable file entity type.
  \item
    \href{https://docs.liferay.com/dxp/apps/collaboration/latest/javadocs/com/liferay/item/selector/criteria/url/criterion/URLItemSelectorCriterion.html}{URLItemSelectorCriterion}:
    URL entity type.
  \item
    \href{https://docs.liferay.com/dxp/apps/collaboration/latest/javadocs/com/liferay/item/selector/criteria/video/criterion/VideoItemSelectorCriterion.html}{VideoItemSelectorCriterion}:
    Video file entity type.
  \end{itemize}
\item
  \href{https://docs.liferay.com/dxp/apps/collaboration/latest/javadocs/com/liferay/wiki/item/selector/criterion/package-summary.html}{\texttt{com.liferay.wiki.api}}
  has wiki criterion.
\end{itemize}

\textbf{Web Experience App Suite Modules:}

\begin{itemize}
\item
  \href{https://docs.liferay.com/dxp/apps/web-experience/latest/javadocs/com/liferay/site/item/selector/criterion/package-summary.html}{\texttt{com.liferay.site.item.selector.api}}
  has site criterion.
\item
  \href{https://docs.liferay.com/dxp/apps/web-experience/latest/javadocs/com/liferay/layout/item/selector/criterion/package-summary.html}{\texttt{com.liferay.layout.item.selector.api}}
  has layout criterion.
\item
  \href{https://docs.liferay.com/dxp/apps/web-experience/latest/javadocs/com/liferay/journal/item/selector/criterion/package-summary.html}{\texttt{com.liferay.journal.item.selector.api}}
  has web content criterion.
\end{itemize}

If there's no criterion class for your entity, you can create your own
\href{https://docs.liferay.com/dxp/apps/collaboration/latest/javadocs/com/liferay/item/selector/ItemSelectorCriterion.html}{\texttt{ItemSelectorCriterion}
class} (tutorial coming soon).

\subsection{Item Selector Return Type
Classes}\label{item-selector-return-type-classes}

The Liferay Collaboration app suite's
\href{https://docs.liferay.com/dxp/apps/collaboration/latest/javadocs/com/liferay/item/selector/criteria/package-summary.html}{\texttt{com.liferay.item.selector.criteria.api}
module} includes the following return types:

\begin{itemize}
\item
  \href{https://docs.liferay.com/dxp/apps/collaboration/latest/javadocs/com/liferay/item/selector/criteria/Base64ItemSelectorReturnType.html}{Base64ItemSelectorReturnType}:
  Base64 encoding of the entity as a \texttt{String}.
\item
  \href{https://docs.liferay.com/dxp/apps/collaboration/latest/javadocs/com/liferay/item/selector/criteria/FileEntryItemSelectorReturnType.html}{FileEntryItemSelectorReturnType}:
  File entry information as a JSON object.
\item
  \href{https://docs.liferay.com/dxp/apps/collaboration/latest/javadocs/com/liferay/item/selector/criteria/URLItemSelectorReturnType.html}{URLItemSelectorReturnType}:
  URL of the entity as a \texttt{String}.
\item
  \href{https://docs.liferay.com/dxp/apps/collaboration/latest/javadocs/com/liferay/item/selector/criteria/UUIDItemSelectorReturnType.html}{UUIDItemSelectorReturnType}:
  Universally Unique Identifier (UUID) of the entity as a
  \texttt{String}.
\end{itemize}

If there's no return type class that meets your needs, you can implement
your own
\href{https://docs.liferay.com/dxp/apps/collaboration/latest/javadocs/com/liferay/item/selector/ItemSelectorReturnType.html}{\texttt{ItemSelectorReturnType}
class} (tutorial coming soon).

\section{Breaking Changes}\label{breaking-changes}

This document presents a chronological list of changes that break
existing functionality, APIs, or contracts with third party Liferay
developers or users. We try our best to minimize these disruptions, but
sometimes they are unavoidable.

Here are some of the types of changes documented in this file:

\begin{itemize}
\tightlist
\item
  Functionality that is removed or replaced
\item
  API incompatibilities: Changes to public Java or JavaScript APIs
\item
  Changes to context variables available to templates
\item
  Changes in CSS classes available to Liferay themes and portlets
\item
  Configuration changes: Changes in configuration files, like
  \texttt{portal.properties}, \texttt{system.properties}, etc.
\item
  Execution requirements: Java version, Java EE Version, browser
  versions, etc.
\item
  Deprecations or end of support: For example, warning that a certain
  feature or API will be dropped in an upcoming version.
\end{itemize}

\subsection{Breaking Changes List}\label{breaking-changes-list}

\subsubsection{Standardized Data Attribute Names Passed into
Selectors}\label{standardized-data-attribute-names-passed-into-selectors}

\begin{itemize}
\tightlist
\item
  \textbf{Date:} 2016-Oct-26
\item
  \textbf{JIRA Ticket:}
  \href{https://issues.liferay.com/browse/LPS-66646}{LPS-66646}
\end{itemize}

\paragraph{What changed?}\label{what-changed}

The data attributes passed into the event when someone uses a selector
(e.g., asset selector, document selector, file selector, role selector,
site selector, user group selector, etc.) have been standardized from
being selector specific (e.g., \texttt{groupid},
\texttt{groupdescriptivename}, \texttt{teamid}, \texttt{teamname}, etc.)
to being more generic (e.g., \texttt{entityid} and \texttt{entityname}).

\paragraph{Who is affected?}\label{who-is-affected}

This affects anyone passing selector specific data attributes to a
selector.

\paragraph{How should I update my
code?}\label{how-should-i-update-my-code}

Instead of using selector specific data attributes, you should change
your data attributes to use \texttt{entityid} and \texttt{entityname}.

\textbf{Example}

Old way:

\begin{verbatim}
<portlet:namespace />selectFileEntryType(event.fileentrytypeid, event.fileentrytypename);
\end{verbatim}

New way:

\begin{verbatim}
<portlet:namespace />selectFileEntryType(event.entityid, event.entityname);
\end{verbatim}

Old way:

\begin{verbatim}
data.put("roleid", role.getRoleId());
data.put("roletitle", role.getTitle(locale));
\end{verbatim}

New way:

\begin{verbatim}
data.put("entityid", role.getRoleId());
data.put("entityname", role.getTitle(locale));
\end{verbatim}

\paragraph{Why was this change made?}\label{why-was-this-change-made}

This change was made to standardize the data attribute names and allow
utility methods to accept standardized event parameters.

\subsubsection{Removed URL Parameters p\_p\_col\_id, p\_p\_col\_pos, and
p\_p\_col\_count from Every Portlet
URL.}\label{removed-url-parameters-p_p_col_id-p_p_col_pos-and-p_p_col_count-from-every-portlet-url.}

\begin{itemize}
\tightlist
\item
  \textbf{Date:} 2016-Dec-12
\item
  \textbf{JIRA Ticket:}
  \href{https://issues.liferay.com/browse/LPS-69482}{LPS-69482}
\end{itemize}

\paragraph{What changed?}\label{what-changed-1}

The parameters \texttt{p\_p\_col\_count}, \texttt{p\_p\_col\_id}, and
\texttt{p\_p\_col\_pos} are no longer present in every portlet URL.

\paragraph{Who is affected?}\label{who-is-affected-1}

This affects developers who are reading these parameters in their custom
code.

\paragraph{How should I update my
code?}\label{how-should-i-update-my-code-1}

You can no longer obtain these parameters from the portlet URL. If you
need to read them, you should do it from \texttt{PortletDisplay}.

\begin{itemize}
\tightlist
\item
  The parameter \texttt{p\_p\_col\_count} can be obtained via the
  \texttt{portletDisplay.getColumnCount()} method.
\item
  The parameter \texttt{p\_p\_col\_id} can be obtained via the
  \texttt{portletDisplay.getColumnId()} method.
\item
  The parameter \texttt{p\_p\_col\_pos} can be obtained via the
  \texttt{portletDisplay.getColumnPos()} method.
\end{itemize}

\paragraph{Why was this change made?}\label{why-was-this-change-made-1}

This change simplifies portlet URLs so they only contain the required
parameters. This was done as a preliminary step of a bigger story to
create portlet URLs without passing the request as a necessary
parameter.

\subsubsection{Moved Users File Uploads Portlet Properties to OSGi
Configuration}\label{moved-users-file-uploads-portlet-properties-to-osgi-configuration}

\begin{itemize}
\tightlist
\item
  \textbf{Date:} 2017-Feb-06
\item
  \textbf{JIRA Ticket:}
  \href{https://issues.liferay.com/browse/LPS-69211}{LPS-69211}
\end{itemize}

\paragraph{What changed?}\label{what-changed-2}

The Users File Uploads portlet properties have been moved from Server
Administration to an OSGi configuration named
\texttt{UserFileUploadsConfiguration.java} in the
\texttt{users-admin-api} module.

\paragraph{Who is affected?}\label{who-is-affected-2}

This affects anyone using the following portlet properties:

\begin{itemize}
\tightlist
\item
  \texttt{users.image.check.token}
\item
  \texttt{users.image.max.size}
\item
  \texttt{users.image.max.height}
\item
  \texttt{users.image.max.width}
\end{itemize}

\paragraph{How should I update my
code?}\label{how-should-i-update-my-code-2}

Instead of overriding the \texttt{portal.properties} file, you can
manage the properties from Portal's configuration administrator. This
can be accessed by navigating to Liferay Portal's \emph{Control Panel} →
\emph{Configuration} → \emph{System Settings} → \emph{Foundation} →
\emph{User Images} and editing the settings there.

If you would like to include the new configuration in your application,
follow the instructions for
\href{https://dev.liferay.com/develop/tutorials/-/knowledge_base/7-0/making-your-applications-configurable}{making
your applications configurable in Liferay 7.0}.

\paragraph{Why was this change made?}\label{why-was-this-change-made-2}

This change was made as part of the modularization efforts to ease
portal configuration changes.

\subsubsection{Moved CAPTCHA Portal Properties to OSGi
Configuration}\label{moved-captcha-portal-properties-to-osgi-configuration}

\begin{itemize}
\tightlist
\item
  \textbf{Date:} 2017-Feb-13
\item
  \textbf{JIRA Ticket:}
  \href{https://issues.liferay.com/browse/LPS-67830}{LPS-67830}
\end{itemize}

\paragraph{What changed?}\label{what-changed-3}

The CAPTCHA properties have been moved from \texttt{portal.properties}
and Server Administration to an OSGi configuration named
\texttt{CaptchaConfiguration.java} in the \texttt{captcha-api} module.

\paragraph{Who is affected?}\label{who-is-affected-3}

This affects anyone using the following portal properties:

\begin{itemize}
\tightlist
\item
  \texttt{captcha.max.challenges}
\item
  \texttt{captcha.check.portal.create\_account}
\item
  \texttt{captcha.check.portal.send\_password}
\item
  \texttt{captcha.check.portlet.message\_boards.edit\_category}
\item
  \texttt{captcha.check.portlet.message\_boards.edit\_message}
\item
  \texttt{captcha.engine.impl}
\item
  \texttt{captcha.engine.recaptcha.key.private}
\item
  \texttt{captcha.engine.recaptcha.key.public}
\item
  \texttt{captcha.engine.recaptcha.url.script}
\item
  \texttt{captcha.engine.recaptcha.url.noscript}
\item
  \texttt{captcha.engine.recaptcha.url.verify}
\item
  \texttt{captcha.engine.simplecaptcha.height}
\item
  \texttt{captcha.engine.simplecaptcha.width}
\item
  \texttt{captcha.engine.simplecaptcha.background.producers}
\item
  \texttt{captcha.engine.simplecaptcha.gimpy.renderers}
\item
  \texttt{captcha.engine.simplecaptcha.noise.producers}
\item
  \texttt{captcha.engine.simplecaptcha.text.producers}
\item
  \texttt{captcha.engine.simplecaptcha.word.renderers}
\end{itemize}

\paragraph{How should I update my
code?}\label{how-should-i-update-my-code-3}

Instead of overriding the \texttt{portal.properties} file, you can
manage the properties from Portal's configuration administrator. This
can be accessed by navigating to Liferay Portal's \emph{Control Panel} →
\emph{Configuration} → \emph{System Settings} → \emph{Captcha} and
editing the settings there.

If you would like to include the new configuration in your application,
follow the instructions for
\href{https://dev.liferay.com/develop/tutorials/-/knowledge_base/7-0/making-your-applications-configurable}{making
your applications configurable in Liferay 7.0}.

\paragraph{Why was this change made?}\label{why-was-this-change-made-3}

This change was made as part of the modularization efforts to ease
portal configuration changes.

\subsubsection{Moved OpenOffice Properties to OSGi
Configuration}\label{moved-openoffice-properties-to-osgi-configuration}

\begin{itemize}
\tightlist
\item
  \textbf{Date:} 2017-Mar-24
\item
  \textbf{JIRA Ticket:}
  \href{https://issues.liferay.com/browse/LPS-71382}{LPS-71382}
\end{itemize}

\paragraph{What changed?}\label{what-changed-4}

The OpenOffice properties have been moved from Server Administration to
an OSGi configuration named \texttt{OpenOfficeConfiguration.java} in the
\texttt{document-library-document-conversion} module.

\paragraph{Who is affected?}\label{who-is-affected-4}

This affects anyone using the following portal properties:

\begin{itemize}
\tightlist
\item
  \texttt{openoffice.cache.enabled}
\item
  \texttt{openoffice.server.enabled}
\item
  \texttt{openoffice.server.host}
\item
  \texttt{openoffice.server.port}
\end{itemize}

\paragraph{How should I update my
code?}\label{how-should-i-update-my-code-4}

Instead of overriding the \texttt{portal.properties} file, you can
manage the properties from Portal's configuration administrator. This
can be accessed by navigating to Liferay Portal's \emph{Control Panel} →
\emph{Configuration} → \emph{System Settings} → \emph{Other} →
\emph{OpenOffice Integration} and editing the settings there.

If you would like to include the new configuration in your application,
follow the instructions for
\href{https://dev.liferay.com/develop/tutorials/-/knowledge_base/7-0/making-your-applications-configurable}{making
your applications configurable in Liferay 7.0}.

\paragraph{Why was this change made?}\label{why-was-this-change-made-4}

This change was made as part of the modularization efforts to ease
portal configuration changes.

\subsubsection{No More Exceptions Are Thrown When a DDMStructure Is
Fetched}\label{no-more-exceptions-are-thrown-when-a-ddmstructure-is-fetched}

\begin{itemize}
\tightlist
\item
  \textbf{Date:} 2017-Mar-31
\item
  \textbf{JIRA Ticket:}
  \href{https://issues.liferay.com/browse/LPS-52675}{LPS-52675}
\end{itemize}

\paragraph{What changed?}\label{what-changed-5}

The following methods no longer throw \texttt{PortalException}:

\begin{verbatim}
public DDMStructure fetchStructure(
  long groupId, long classNameId, String structureKey,
  boolean includeAncestorStructures)

public DDMStructure fetchStructureByUuidAndGroupId(
  String uuid, long groupId, boolean includeAncestorStructures)
\end{verbatim}

\paragraph{Who is affected?}\label{who-is-affected-5}

This affects anyone using these methods.

\paragraph{How should I update my
code?}\label{how-should-i-update-my-code-5}

Keep using these methods, but be aware that they don't throw exceptions.

\paragraph{Why was this change made?}\label{why-was-this-change-made-5}

Since the current method implementations don't generate exceptions,
there's no need for the methods to declare throwing a
\texttt{PortalException}.

\subsubsection{Removed Indexation of Fields ratings and
viewCount}\label{removed-indexation-of-fields-ratings-and-viewcount}

\begin{itemize}
\tightlist
\item
  \textbf{Date:} 2017-May-16
\item
  \textbf{JIRA Ticket:}
  \href{https://issues.liferay.com/browse/LPS-70724}{LPS-70724}
\end{itemize}

\paragraph{What changed?}\label{what-changed-6}

The fields \texttt{ratings} and \texttt{viewCount} are no longer indexed
in the \texttt{BaseIndexer} class for \texttt{AssetEntry} objects.

\paragraph{Who is affected?}\label{who-is-affected-6}

This affects any search-related custom code where the \texttt{ratings}
and \texttt{viewCount} fields are used in queries.

\paragraph{How should I update my
code?}\label{how-should-i-update-my-code-6}

To adapt to these changes, consider several alternatives:

\begin{itemize}
\tightlist
\item
  Use the Highest Rated Assets and Most Viewed Assets Liferay portlets.
\item
  Replace the index query with a database query.
\item
  Implement an \texttt{IndexerPostProcessor} to index these fields in
  certain documents.
\end{itemize}

\paragraph{Why was this change made?}\label{why-was-this-change-made-6}

Keeping the Ratings and View Count options in the search index in sync
with the database has a negative impact on normal operations due to the
significantly increased number of index Write requests causing
throughput issues and, therefore, performance degradation.

In addition, the view count is not always up-to-date in the database.
This behavior is controlled by the \emph{Buffered Increment} mechanism.
You can find more information about this in the
\texttt{portal.properties} file.

\subsubsection{Moved Upload Servlet Request Portal Properties to OSGi
Configuration}\label{moved-upload-servlet-request-portal-properties-to-osgi-configuration}

\begin{itemize}
\tightlist
\item
  \textbf{Date:} 2017-May-30
\item
  \textbf{JIRA Ticket:}
  \href{https://issues.liferay.com/browse/LPS-69102}{LPS-69102}
\end{itemize}

\paragraph{What changed?}\label{what-changed-7}

The Upload Servlet Request properties have been moved from the
\texttt{portal.properties} file and Server Administration to an OSGi
configuration named \texttt{UploadServletRequestConfiguration.java} in
the \texttt{portal-upload} module.

\paragraph{Who is affected?}\label{who-is-affected-7}

This affects anyone using the following portal properties:

\begin{itemize}
\tightlist
\item
  \texttt{com.liferay.portal.upload.UploadServletRequestImpl.max.size}
\item
  \texttt{com.liferay.portal.upload.UploadServletRequestImpl.temp.dir}
\end{itemize}

\paragraph{How should I update my
code?}\label{how-should-i-update-my-code-7}

Instead of overriding the \texttt{portal.properties} file, you can
manage the properties from Portal's configuration administrator. This
can be accessed by navigating to Liferay Portal's \emph{Control Panel} →
\emph{Configuration} → \emph{System Settings} → \emph{Upload Servlet
Request} and editing the settings there.

If you would like to include the new configuration in your application,
follow the instructions for
\href{https://dev.liferay.com/develop/tutorials/-/knowledge_base/7-0/making-your-applications-configurable}{making
your applications configurable in Liferay 7.0}.

\paragraph{Why was this change made?}\label{why-was-this-change-made-7}

This change was made as part of the modularization efforts to ease
portal configuration changes.

\subsubsection{Moved Three DL File Properties to OSGi
Configuration}\label{moved-three-dl-file-properties-to-osgi-configuration}

\begin{itemize}
\tightlist
\item
  \textbf{Date:} 2017-Aug-01
\item
  \textbf{JIRA Ticket:}
  \href{https://issues.liferay.com/browse/LPS-69208}{LPS-69208}
\end{itemize}

\paragraph{What changed?}\label{what-changed-8}

Two DL File properties have been moved from Server Administration to the
OSGi configuration \texttt{DLConfiguration}, and one to
\texttt{DLFileEntryConfiguration}. Both configurations are located in
the \texttt{document-library-api} module.

\paragraph{Who is affected?}\label{who-is-affected-8}

This affects anyone who is using the following portal properties:

\begin{itemize}
\tightlist
\item
  \texttt{dl.file.entry.previewable.processor.max.size}
\item
  \texttt{dl.file.extensions}
\item
  \texttt{dl.file.max.size}
\end{itemize}

\paragraph{How should I update my
code?}\label{how-should-i-update-my-code-8}

Instead of overriding the \texttt{portal.properties} file, you can
manage the properties from Portal's configuration administrator. This
can be accessed by navigating to Portal's \emph{Control Panel} →
\emph{Configuration} → \emph{System Settings} → \emph{Collaboration} →
\emph{Documents \& Media Service} or \emph{Documents \& Media File
Entries} and editing the settings there.

If you would like to include the new configuration in your application,
follow the instructions for
\href{https://dev.liferay.com/develop/tutorials/-/knowledge_base/7-0/making-your-applications-configurable}{making
your applications configurable}.

\paragraph{Why was this change made?}\label{why-was-this-change-made-8}

This change was made as part of the modularization efforts to ease
portal configuration changes.

\subsubsection{Removed the soyutils
Module}\label{removed-the-soyutils-module}

\begin{itemize}
\tightlist
\item
  \textbf{Date:} 2017-Aug-28
\item
  \textbf{JIRA Ticket:}
  \href{https://issues.liferay.com/browse/LPS-69102}{LPS-69102}
\end{itemize}

\paragraph{What changed?}\label{what-changed-9}

The module \texttt{frontend-js-soyutils-web} is no longer available.

\paragraph{Who is affected?}\label{who-is-affected-9}

This affects anyone using the \texttt{soyutils} module.

\paragraph{How should I update my
code?}\label{how-should-i-update-my-code-9}

In the rare case that a component is affected, it is recommended that
the code is migrated to use the \texttt{metal-soy} module instead. You
can do this by extending the \texttt{Metal.js} provided
\texttt{Component} classes.

\paragraph{Why was this change made?}\label{why-was-this-change-made-9}

The removed module exposed a legacy version of \texttt{soyutils}. This
caused interoperability issues between applications using different
versions of the Closure Template library.

\subsubsection{Converted liferay-ui Tags to Module-Specific
Tags}\label{converted-liferay-ui-tags-to-module-specific-tags}

\begin{itemize}
\tightlist
\item
  \textbf{Date:} 2017-Aug-28
\item
  \textbf{JIRA Ticket:}
  \href{https://issues.liferay.com/browse/LPS-74331}{LPS-74331}
\end{itemize}

\paragraph{What changed?}\label{what-changed-10}

Several \texttt{liferay-ui} taglibs have been moved from the Portal's
kernel into OSGi modules, resulting in taglib names being changed. The
updated names are listed below:

\begin{itemize}
\tightlist
\item
  \texttt{liferay-ui:asset-add-button} →
  \texttt{liferay-asset:asset-add-button}
\item
  \texttt{liferay-ui:asset-addon-entry-display} →
  \texttt{liferay-asset:asset-addon-entry-display}
\item
  \texttt{liferay-ui:asset-addon-entry-selector} →
  \texttt{liferay-asset:asset-addon-entry-selector}
\item
  \texttt{liferay-ui:asset-categories-available} →
  \texttt{liferay-asset:asset-categories-available}
\item
  \texttt{liferay-ui:asset-categories-error} →
  \texttt{liferay-asset:asset-categories-error}
\item
  \texttt{liferay-ui:asset-display} →
  \texttt{liferay-asset:asset-display}
\item
  \texttt{liferay-ui:asset-links} → \texttt{liferay-asset:asset-links}
\item
  \texttt{liferay-ui:asset-metadata} →
  \texttt{liferay-asset:asset-metadata}
\item
  \texttt{liferay-ui:asset-tags-available} →
  \texttt{liferay-asset:asset-tags-available}
\item
  \texttt{liferay-ui:asset-tags-error} →
  \texttt{liferay-asset:asset-tags-error}
\item
  \texttt{liferay-ui:asset-tags-navigation} →
  \texttt{liferay-asset:asset-tags-navigation}
\item
  \texttt{liferay-ui:input-asset-links} →
  \texttt{liferay-asset:input-asset-links}
\item
  \texttt{liferay-ui:journal-content-search} → Removed
  (\texttt{journal-content-search-web} was deprecated)
\item
  \texttt{liferay-ui:restore-entry} → Removed
\item
  \texttt{liferay-ui:rss} →\texttt{liferay-rss:rss}
\item
  \texttt{liferay-ui:rss-settings} → \texttt{liferay-rss:rss-settings}
\end{itemize}

\paragraph{Who is affected?}\label{who-is-affected-10}

This affects anyone who is using the taglibs listed above.

\paragraph{How should I update my
code?}\label{how-should-i-update-my-code-10}

You must update your \texttt{liferay-ui} tags to use the new names. If
you prefer keeping the old names temporarily, you can rely on the
compatibility layer offered by Liferay. To set this, add the
\texttt{com.liferay.portal.web.compat} dependency to your project's
build file.

Use the updated tag names as soon as you're able, as this compatibility
layer is deprecated and will not be available for future releases.

\paragraph{Why was this change made?}\label{why-was-this-change-made-10}

This change was made to categorize taglibs properly by moving them to
their respective OSGi modules.

\subsubsection{Changed Default Value for Browser Cache
Properties}\label{changed-default-value-for-browser-cache-properties}

\begin{itemize}
\tightlist
\item
  \textbf{Date:} 2017-Sep-05
\item
  \textbf{JIRA Ticket:}
  \href{https://issues.liferay.com/browse/LPS-74452}{LPS-74452}
\end{itemize}

\paragraph{What changed?}\label{what-changed-11}

The default values for the portal properties
\texttt{browser.cache.disabled} and
\texttt{browser.cache.signed.in.disabled} were changed to \texttt{true}.

\paragraph{Who is affected?}\label{who-is-affected-11}

This affects anyone relying on proxies and load balancers to cache HTML
content.

\paragraph{How should I update my
code?}\label{how-should-i-update-my-code-11}

You should set both properties \texttt{browser.cache.disabled} and
\texttt{browser.cache.signed.in.disabled} to \texttt{false}, as
documented in \texttt{portal-legacy-7.0.properties}.

\paragraph{Why was this change made?}\label{why-was-this-change-made-11}

The load balancer and web proxy's behavior when Cache-Control headers
are missing is not defined. In the past, many preferred to not cache the
content for correctness; however, it is now common to cache the content
for performance.

When an aggressive caching load balancer or web proxy appears in the
network architecture, the default value may result in security problems
such as personalized content being mistakenly shared, including names or
other personally identifiable information. As Liferay shifts towards use
cases providing personalized experiences, this is becoming a serious
problem.

While this is ultimately a load balancer or web proxy configuration
issue, it is perceived as a Liferay issue because it is Liferay content
being cached, and is viewed negatively because leaking sensitive
information in a production environment is a very serious security
issue.

A value of \texttt{true} will improve a portal administrator's
experience, and a value of \texttt{false} can be considered during
performance tuning, if needed.

\subsubsection{Users Can Have Numeric Screen Names with No Limitations,
and Sites Can No Longer Have Numeric Friendly
URLs}\label{users-can-have-numeric-screen-names-with-no-limitations-and-sites-can-no-longer-have-numeric-friendly-urls}

\begin{itemize}
\tightlist
\item
  \textbf{Date:} 2017-Oct-10
\item
  \textbf{JIRA Ticket:}
  \href{https://issues.liferay.com/browse/LPS-66460}{LPS-66460}
\end{itemize}

\paragraph{What changed?}\label{what-changed-12}

\begin{itemize}
\tightlist
\item
  The portal property \texttt{users.screen.name.allow.numeric} now
  defaults to \texttt{true}.
\item
  Numeric screen names are no longer limited by whether they correspond
  to an existing group ID.
\item
  Sites can no longer set their group ID as their friendly URL.
\item
  Sites can no longer be implicitly accessed by using their group ID in
  the URL (this used to be available automatically, even if it wasn't
  set that way).
\item
  If the friendly URL of a site is already set to the group ID, it will
  continue to work as normal, but you will be forced to change it if you
  update the site in the Site Settings portlet.
\item
  If a site is updated and no friendly URL is provided, it will default
  to \texttt{/group-\textless{}groupId\textgreater{}}. If that
  duplicates another friendly URL, the friendly URL will be incremented
  until a unique friendly URL is found (e.g.,
  \texttt{/group-\textless{}groupId\textgreater{}-1}).
\item
  The default friendly URL for new sites has \textbf{not} changed.
\end{itemize}

\paragraph{Who is affected?}\label{who-is-affected-12}

This affects anyone who

\begin{itemize}
\tightlist
\item
  has set the friendly URL of their site to the group ID.
\item
  uses a group ID to navigate or direct to a site.
\end{itemize}

\paragraph{How should I update my
code?}\label{how-should-i-update-my-code-12}

No code updates should be required, but if you fall under one of the
scenarios in the previous section, you should consider the following
changes:

\begin{itemize}
\tightlist
\item
  If you have set the friendly URL of a site to its group ID, you should
  update the friendly URL of that site to something else. A site
  administrator can do this through the Site Settings portlet.
\item
  If you have hard-coded the group ID in any links, you must change them
  to match the updated friendly URL.
\end{itemize}

\paragraph{Why was this change made?}\label{why-was-this-change-made-12}

There were common complaints from customers who used LDAP to import
users --- if users were given a numeric screen name during import, some
imports would fail because those screen names conflicted with an
existing group ID.

This was because a site's group ID could be used as its friendly URL,
while a user's screen name is used as the friendly URL to their personal
site. This could introduce a routing conflict, so numeric screen names
were disallowed if they conflicted with an existing group ID.

By removing sites' ability to use their group ID as their friendly URL,
the possible conflict with numeric screen names is expunged, allowing
users to have any number as their screen name. This makes it much less
likely for LDAP imports to fail when using numeric screen names for
imported users.

Since LDAP import is more commonly used than a site using the group ID
as its friendly URL, the less useful feature was removed to stabilize
the more common one.

\subsubsection{Removed Support for Velocity in
Themes}\label{removed-support-for-velocity-in-themes}

\begin{itemize}
\tightlist
\item
  \textbf{Date:} 2017-Oct-19
\item
  \textbf{JIRA Ticket:}
  \href{https://issues.liferay.com/browse/LPS-74379}{LPS-74379}
\end{itemize}

\paragraph{What changed?}\label{what-changed-13}

\begin{itemize}
\tightlist
\item
  Themes can no longer use Velocity for templates.
\item
  Some helper methods have been removed from the public APIs
  \texttt{com.liferay.portal.kernel.util.ThemeHelper} and
  \texttt{com.liferay.taglib.util.ThemeUtil}.
\end{itemize}

\paragraph{Who is affected?}\label{who-is-affected-13}

This affects anyone who has themes using Velocity templates or is using
the removed methods.

\paragraph{How should I update my
code?}\label{how-should-i-update-my-code-13}

If you have a theme using Velocity, consider migrating it to FreeMarker
for better maintenance and improved security.

If you are using the removed methods, consider using the
\texttt{com.liferay.portal.kernel.template.Template} functionality
directly to process templates.

\paragraph{Why was this change made?}\label{why-was-this-change-made-13}

Velocity was deprecated in Liferay Portal 7.0 and the recommendation was
to migrate to FreeMarker. Also, Velocity has had no new releases for a
long time.

The removal of Velocity support for Liferay Portal 7.1 themes allows for
an increased focus on existing and new template engines.

\subsubsection{Moved Organization Type Properties to OSGi
Configuration}\label{moved-organization-type-properties-to-osgi-configuration}

\begin{itemize}
\tightlist
\item
  \textbf{Date:} 2018-Jan-19
\item
  \textbf{JIRA Ticket:}
  \href{https://issues.liferay.com/browse/LPS-77183}{LPS-77183}
\end{itemize}

\paragraph{What changed?}\label{what-changed-14}

The organization type properties have been moved from
\texttt{portal.properties} to an OSGi configuration named
\texttt{OrganizationsTypesConfiguration.java} in the
\texttt{users-admin-api} module.

\paragraph{Who is affected?}\label{who-is-affected-14}

This affects anyone using the following portal properties:

\begin{itemize}
\tightlist
\item
  \texttt{organizations.types}
\item
  \texttt{organizations.rootable}
\item
  \texttt{organizations.children.types}
\item
  \texttt{organizations.country.enabled}
\item
  \texttt{organizations.country.required}
\end{itemize}

\paragraph{How should I update my
code?}\label{how-should-i-update-my-code-14}

Instead of overriding the \texttt{portal.properties} file, you can
manage the properties from Portal's configuration administrator. This
can be accessed by navigating to Liferay Portal's \emph{Control Panel} →
\emph{Configuration} → \emph{System Settings} → \emph{Foundation} →
\emph{Organization Type} and editing the settings there.

If you would like to include the new configuration in your application,
follow the instructions for
\href{https://dev.liferay.com/develop/tutorials/-/knowledge_base/7-0/making-your-applications-configurable}{making
your applications configurable}.

\paragraph{Why was this change made?}\label{why-was-this-change-made-14}

This change was made as part of the modularization efforts to ease
portal configuration changes.

\subsubsection{Updated jQuery and Lodash Bundled
Versions}\label{updated-jquery-and-lodash-bundled-versions}

\begin{itemize}
\tightlist
\item
  \textbf{Date:} 2018-Feb-07
\item
  \textbf{JIRA Ticket:}
  \href{https://issues.liferay.com/browse/LPS-77764}{LPS-77764},
  \href{https://issues.liferay.com/browse/LPS-77765}{LPS-77765}
\end{itemize}

\paragraph{What changed?}\label{what-changed-15}

The bundled jQuery version has been updated from 2.1.4 to 3.3.1. The
bundled Lodash version has been updated from 3.10.1 to 4.17.4.

\paragraph{Who is affected?}\label{who-is-affected-15}

This affects anyone using the previous API versions in their code.

\paragraph{How should I update my
code?}\label{how-should-i-update-my-code-15}

Follow the changelogs on the
\href{http://jquery.com/upgrade-guide/3.0/}{jQuery} and
\href{https://github.com/lodash/lodash/wiki/Changelog\#v400}{Lodash}
sites to update any affected code.

\paragraph{Why was this change made?}\label{why-was-this-change-made-15}

This change provides the latest jQuery and Lodash versions available.

\subsubsection{Removed the VALIDATE\_DDM\_FORM\_VALUES Constant from
DDMWebKeys}\label{removed-the-validate_ddm_form_values-constant-from-ddmwebkeys}

\begin{itemize}
\tightlist
\item
  \textbf{Date:} 2018-Feb-22
\item
  \textbf{JIRA Ticket:}
  \href{https://issues.liferay.com/browse/LPS-77168}{LPS-77168}
\end{itemize}

\paragraph{What changed?}\label{what-changed-16}

The \texttt{VALIDATE\_DDM\_FORM\_VALUES} constant has been removed from
\texttt{DDMWebKeys}.

\paragraph{Who is affected?}\label{who-is-affected-16}

This affects anyone using this constant.

\paragraph{How should I update my
code?}\label{how-should-i-update-my-code-16}

Use the String \texttt{validateDDMFormValues}, which was the constant's
value.

\paragraph{Why was this change made?}\label{why-was-this-change-made-16}

A constant is unnecessary for a value that's not part of an API.

\subsubsection{Removed JavaScript Minification Properties
minifier.javascript.impl and yui.compressor.* from
portal.properties}\label{removed-javascript-minification-properties-minifier.javascript.impl-and-yui.compressor.-from-portal.properties}

\begin{itemize}
\tightlist
\item
  \textbf{Date:} 2018-Feb-28
\item
  \textbf{JIRA Ticket:}
  \href{https://issues.liferay.com/browse/LPS-74375}{LPS-74375}
\end{itemize}

\paragraph{What changed?}\label{what-changed-17}

The JavaScript minifiers have been extracted from \texttt{portal-kernel}
and moved to their own OSGi module. Thus, they are not configured in
\texttt{portal.properties} any more, but rather, through OSGi
configuration.

\paragraph{Who is affected?}\label{who-is-affected-17}

This affects anyone who had the Yahoo JavaScript minifier active and
configured to override its default settings.

\paragraph{How should I update my
code?}\label{how-should-i-update-my-code-17}

If you are implementing your own JavaScript minifier, you should extract
it to its own OSGi module. See module
\href{https://github.com/liferay/liferay-portal/tree/master/modules/apps/frontend-js/frontend-js-minifier}{frontend-js-minifier}
for an example of how to do this.

\paragraph{Why was this change made?}\label{why-was-this-change-made-17}

The JavaScript minifiers were not easy to customize. For example, the
Google minifier used an old version of the closure-compiler, which was
difficult to upgrade because it required \texttt{portal-kernel}
dependency changes. This could create conflicts.

Having JavaScript minifiers in their own OSGi modules requires less
dependency management and makes it easier to provide new implementations
of JavaScript minifiers. Also, configuration can now be done using OSGi
standards.

\subsubsection{Changed Behavior of liferay-ui:input-date Taglib's
showDisableCheckbox
Argument}\label{changed-behavior-of-liferay-uiinput-date-taglibs-showdisablecheckbox-argument}

\begin{itemize}
\tightlist
\item
  \textbf{Date:} 2018-Mar-06
\item
  \textbf{JIRA Ticket:}
  \href{https://issues.liferay.com/browse/LPS-78475}{LPS-78475}
\end{itemize}

\paragraph{What changed?}\label{what-changed-18}

Previously, when the \texttt{liferay-ui:input-date} taglib's
\texttt{showDisableCheckbox} argument was set to \texttt{true}, the
disable checkbox was hidden. Now, the value \texttt{true} displays it,
and \texttt{false} hides it.

\paragraph{Who is affected?}\label{who-is-affected-18}

This affects anyone trying to hide the \texttt{liferay-ui:input-date}
taglib's disable checkbox.

\paragraph{How should I update my
code?}\label{how-should-i-update-my-code-18}

If you are setting the \texttt{showDisableCheckbox} argument to
\texttt{true} to hide the \texttt{liferay-ui:input-date} taglib's
disable checkbox, you should now set it to \texttt{false}, and vice
versa.

\paragraph{Why was this change made?}\label{why-was-this-change-made-18}

The behavior did not match with the name of the argument and was
counter-intuitive.

\subsubsection{DDLExporterFactory Became an
Interface}\label{ddlexporterfactory-became-an-interface}

\begin{itemize}
\tightlist
\item
  \textbf{Date:} 2018-Apr-20
\item
  \textbf{JIRA Ticket:}
  \href{https://issues.liferay.com/browse/LPS-79221}{LPS-79221}
\end{itemize}

\paragraph{What changed?}\label{what-changed-19}

\texttt{DDLExporterFactory} was renamed to
\texttt{DDLExporterFactoryImpl} and moved to the module
\texttt{dynamic-data-lists-service}, once this class holds the logic
associated to the DDL exporters management. On the other hand, a new
interface named \texttt{DDLExporterFactory} was created on the module
\texttt{dynamic-data-lists-api} and \texttt{DDLExporterFactoryImpl} is
implementing it.

\paragraph{Who is affected?}\label{who-is-affected-19}

This affects anyone who is using \texttt{DDLExporterFactory} to
manipulate (add, get and remove) the DDL exporters.

\paragraph{How should I update my
code?}\label{how-should-i-update-my-code-19}

At a first moment, isn't expected the developer use
\texttt{DDLExporterFactory} to manipulate the DDL exporters. Actually,
there is just one extension point dedicated to the DDL exporters, the
possibility to export the data in formats not provided by default. In
order to export the data to a new format, the developer will need to
create a new Java class that extends the abstract class
\texttt{BaseDDLExporter}.

\paragraph{Why was this change made?}\label{why-was-this-change-made-19}

To encapsulate the default implementation of
\texttt{DDLExporterFactory}, which doesn't need to be exposed, and also
to keep on the module \texttt{dynamic-data-lists-api} only interfaces.

\subsubsection{Updated Liferay Portal's Portlet API
Implementation}\label{updated-liferay-portals-portlet-api-implementation}

\begin{itemize}
\tightlist
\item
  \textbf{Date:} 2018-May-10
\item
  \textbf{JIRA Ticket:}
  \href{https://issues.liferay.com/browse/LPS-73282}{LPS-73282}
\end{itemize}

\paragraph{What changed?}\label{what-changed-20}

Liferay Portal 7.1 CE GA1 provides the Portlet 3.0 API dependency in the
runtime classpath. Previous versions provided the Portlet 2.0 API.

Full support for Portlet 3.0 will not be available until Liferay Portal
7.1 CE GA4 is released. This will serve as a developer preview for the
technology until it's officially promoted in Liferay Portal 7.2.

\paragraph{Who is affected?}\label{who-is-affected-20}

This affects developers planning to upgrade custom portlets from earlier
versions of Liferay Portal.

\paragraph{How should I update my
code?}\label{how-should-i-update-my-code-20}

There are three development use-cases to plan for:

\subparagraph{JSP Considerations}\label{jsp-considerations}

Portlet 3.0 is a binary-backward-compatible upgrade. This means that
Java source that was built against \texttt{portlet-api-2.0.0.jar} is
compatible at runtime. Since JSP files are typically not compiled until
the first request, however, they do not fall under the category of
pre-compiled source.

Specifically, if a JSP contains a Java scriptlet that calls
\href{https://docs.liferay.com/portlet-api/3.0/javadocs/javax/portlet/MimeResponse.html\#createActionURL()}{\texttt{MimeResponse.createActionURL()}}
and
\href{https://docs.liferay.com/portlet-api/3.0/javadocs/javax/portlet/MimeResponse.html\#createRenderURL()}{\texttt{MimeResponse.createRenderURL()}},
then there is a possibility that the JSP will fail to compile or throw a
\texttt{ClassCastException} at runtime. This is because the return type
of these methods has changed.

For example, a Liferay Portal sample portlet's \texttt{view.jsp} had to
be changed from

\begin{verbatim}
<aui:form action="<%= renderResponse.createActionURL() %>" method="post" name="fm">
\end{verbatim}

to

\begin{verbatim}
<aui:form action="<%= (PortletURL)renderResponse.createActionURL() %>" method="post" name="fm">
\end{verbatim}

\subparagraph{Upgrade Considerations}\label{upgrade-considerations}

To take advantage of new features in Portlet 3.0, you must rebuild
portlet projects against the \texttt{portlet-api-3.0.0.jar} dependency
and \emph{opt-in} by specifying version 3.0 in one of two ways:

\begin{enumerate}
\def\labelenumi{\arabic{enumi}.}
\item
  Add the following tag in your portlet's \texttt{portlet.xml} file:

\begin{verbatim}
 <portlet-app version="3.0">
\end{verbatim}
\item
  Add the following property in your portlet's \texttt{@Component} tag:

\begin{verbatim}
@Component(
    property = {
        "javax.portlet.version=3.0"
    },
    service = Portlet.class
)
\end{verbatim}
\end{enumerate}

In addition, you must opt-in to new JSP features by specifying the
Portlet 3.0 tag library in your JSP views. For example,

\begin{verbatim}
<%@ taglib uri="http://xmlns.jcp.org/portlet_3_0" prefix="portlet" %>
\end{verbatim}

JSPs that opt-in with the new tag library may encounter JSP compilation
problems related to the
\texttt{\textless{}portlet:defineObjects\textgreater{}} tag.
Specifically, if JSPs reference variables with the following names in
Java scriptlets, then a JSP compilation will occur:

\begin{itemize}
\tightlist
\item
  \texttt{actionParams}
\item
  \texttt{clientDataRequest}
\item
  \texttt{cookies}
\item
  \texttt{contextPath}
\item
  \texttt{locale}
\item
  \texttt{locales}
\item
  \texttt{mutableRenderParams}
\item
  \texttt{namespace}
\item
  \texttt{portletContext}
\item
  \texttt{portletMode}
\item
  \texttt{portletRequest}
\item
  \texttt{portletResponse}
\item
  \texttt{resourceParams}
\item
  \texttt{windowId}
\item
  \texttt{windowState}
\item
  \texttt{stateAwareResponse}
\end{itemize}

With the Portlet API 3.0 implementation, these variables are already
added to this context by default, so attempting to initialize them in
the JSP would duplicate them. Therefore, your JSP scriptlets adding them
should be removed.

For example, JSP scriptlets like the following had to be removed from
several of Liferay Portal's out-of-the-box portlets' \texttt{view.jsp}:

\begin{verbatim}
<%=
PortletRequest portletRequest = (PortletRequest)request.getAttribute(JavaConstants.JAVAX_PORTLET_REQUEST);

PortletResponse portletResponse = (PortletResponse)request.getAttribute(JavaConstants.JAVAX_PORTLET_RESPONSE);

String namespace = AUIUtil.getNamespace(portletRequest, portletResponse);

if (Validator.isNull(namespace)) {
    namespace = AUIUtil.getNamespace(request);
}
%>
\end{verbatim}

\subparagraph{JSF Considerations}\label{jsf-considerations}

JSF Portlets must be upgraded to the latest version of Liferay Faces
Bridge, which is planned for release in Q4, 2018. Download and upgrade
instructions will be made available at
\url{https://www.liferayfaces.org} at that time.

\paragraph{Why was this change made?}\label{why-was-this-change-made-20}

This change provides the latest features offered by the Portlet 3.0
Specification, which was released in early 2017.

\subsubsection{Changed the From Last Publish Date Option in
Staging}\label{changed-the-from-last-publish-date-option-in-staging}

\begin{itemize}
\tightlist
\item
  \textbf{Date:} 2018-Jun-06
\item
  \textbf{JIRA Ticket:}
  \href{https://issues.liferay.com/browse/LPS-81695}{LPS-81695}
\end{itemize}

\paragraph{What changed?}\label{what-changed-21}

The \emph{From Last Publish Date} option used in the publication process
has programmatically changed.

\paragraph{Who is affected?}\label{who-is-affected-21}

This affects anyone who implemented Staging support for their custom
entities.

\paragraph{How should I update my
code?}\label{how-should-i-update-my-code-21}

You must create a \texttt{*StagingModelListener} class for your custom
entity, which extends the
\href{https://docs.liferay.com/ce/portal/7.1-latest/javadocs/portal-kernel/com/liferay/portal/kernel/model/BaseModelListener.html}{\texttt{com.liferay.portal.kernel.model.BaseModelListener}}.
You can examine the
\href{https://github.com/liferay/liferay-portal/blob/7.1.0-ga1/modules/apps/blogs/blogs-service/src/main/java/com/liferay/blogs/internal/model/listener/BlogsEntryStagingModelListener.java}{\texttt{BlogsEntryStagingModelListener}}
class as an example.

You must also update the \texttt{doPrepareManifestSummary} method in
your custom \texttt{*PortletDataHandler} to use the
\texttt{populateLastPublishDateCounts} method from the
\href{https://docs.liferay.com/ce/apps/web-experience/latest/javadocs/com/liferay/exportimport/staging/StagingImpl.html}{\texttt{com.liferay.exportimport.internal.staging.StagingImpl}},
in case of a \emph{From Last Publish Date} publication. See the
\href{https://github.com/liferay/liferay-portal/blob/7.1.0-ga1/modules/apps/blogs/blogs-web/src/main/java/com/liferay/blogs/web/internal/exportimport/data/handler/BlogsPortletDataHandler.java}{\texttt{BlogsPortletDataHandler}}
as an example.

\paragraph{Why was this change made?}\label{why-was-this-change-made-21}

It was hard to collect which entities should be published to the live
site. Instead of running queries to find the contents that were modified
since the last publication, now changesets are used to track this
information.

\subsubsection{Changed the Dependency for the liferay-util:html-top JSP
tag}\label{changed-the-dependency-for-the-liferay-utilhtml-top-jsp-tag}

\begin{itemize}
\tightlist
\item
  \textbf{Date:} 2018-Jun-07
\item
  \textbf{JIRA Ticket:}
  \href{https://issues.liferay.com/browse/LPS-81983}{LPS-81983}
\end{itemize}

\paragraph{What changed?}\label{what-changed-22}

The usage of \texttt{portal-kernel}'s \texttt{StringBundler} has been
deprecated in favor of Liferay's Petra \texttt{StringBundler}.

\paragraph{Who is affected?}\label{who-is-affected-22}

This affects anyone using the
\texttt{\textless{}liferay-util:html-top\textgreater{}} JSP tag.

\paragraph{How should I update my
code?}\label{how-should-i-update-my-code-22}

You must add the following dependency in your build file for your JSPs
to compile successfully:

\textbf{build.gradle}:

\begin{verbatim}
dependencies {
    ...
    compileOnly group: "com.liferay", name: "com.liferay.petra.string", version: "1.2.0"
    ...
}
\end{verbatim}

\textbf{pom.xml}:

\begin{verbatim}
<dependency>
    <groupId>com.liferay</groupId>
    <artifactId>com.liferay.petra.string</artifactId>
    <version>1.2.0</version>
    <scope>provided</scope>
</dependency>
\end{verbatim}

\paragraph{Why was this change made?}\label{why-was-this-change-made-22}

This change helps stabilize the foundation of Liferay Portal's
utilities.

\subsubsection{Decoupled Several Classes from
PortletURLImpl}\label{decoupled-several-classes-from-portleturlimpl}

\begin{itemize}
\tightlist
\item
  \textbf{Date:} 2018-Jun-08
\item
  \textbf{JIRA Ticket:}
  \href{https://issues.liferay.com/browse/LPS-82119}{LPS-82119}
\end{itemize}

\paragraph{What changed?}\label{what-changed-23}

All classes implementing \texttt{javax.portlet.BaseURL} have had their
inheritance hierarchy change. These classes include

\begin{itemize}
\tightlist
\item
  \texttt{PortletURLImplWrapper}
\item
  \texttt{LiferayStrutsPortletURLImpl}
\item
  \texttt{StrutsActionPortletURL}
\end{itemize}

\paragraph{Who is affected?}\label{who-is-affected-23}

This affects code that attempts to subclass or create a new instance of
the classes listed previously.

\paragraph{How should I update my
code?}\label{how-should-i-update-my-code-23}

You must refactor the constructors of your affected classes to receive
\texttt{com.liferay.portal.kernel.portlet.LiferayPortletResponse}
instead of \texttt{com.liferay.portlet.PortletResponseImpl}.

In addition, their class hierarchies must be changed. For example, the
\texttt{com.liferay.portal.struts.StrutsActionPortletURL} class
hierarchy was changed from

\begin{itemize}
\tightlist
\item
  \texttt{com.liferay.portlet.PortletURLImpl}
\item
  \texttt{com.liferay.portlet.PortletURLImplWrapper}
\item
  \texttt{com.liferay.portal.struts.StrutsActionPortletURL}
\end{itemize}

to

\begin{itemize}
\tightlist
\item
  \texttt{javax.portlet.filter.RenderStateWrapper}
\item
  \texttt{javax.portlet.filter.BaseURLWrapper}
\item
  \texttt{javax.portlet.filter.PortletURLWrapper}
\item
  \texttt{com.liferay.portal.kernel.portlet.LiferayPortletURLWrapper}
\item
  \texttt{com.liferay.portlet.PortletURLImplWrapper}
\item
  \texttt{com.liferay.portal.struts.StrutsActionPortletURL}
\end{itemize}

\paragraph{Why was this change made?}\label{why-was-this-change-made-23}

This change corrects a best practice violation regarding
implementation-specific details being included within an API.

\subsubsection{Changed the Request Object in Web Content
Templates}\label{changed-the-request-object-in-web-content-templates}

\begin{itemize}
\tightlist
\item
  \textbf{Date:} 2018-Jun-12
\item
  \textbf{JIRA Ticket:}
  \href{https://issues.liferay.com/browse/LPS-77766}{LPS-77766}
\end{itemize}

\paragraph{What changed?}\label{what-changed-24}

The request object is no longer accessible as a map, but rather, as an
object of type \texttt{javax.servlet.http.HttpServletRequest}.

\paragraph{Who is affected?}\label{who-is-affected-24}

This affects users with Web Content templates that access request
parameters as a map like this:

\begin{verbatim}
<#assign containerId = request["theme-display"]["portlet-display"]["instance-id"] >
\end{verbatim}

\paragraph{How should I update my
code?}\label{how-should-i-update-my-code-24}

To keep retrieving the request parameter values as a map,
\texttt{requestMap} must be used instead:

\begin{verbatim}
<#assign containerId = requestMap["theme-display"]["portlet-display"]["instance-id"] >
\end{verbatim}

\paragraph{Why was this change made?}\label{why-was-this-change-made-24}

This was done to allow template context contributors to work in Web
Content templates.

\subsubsection{Disabled Access to Gogo Shell Using
Telnet}\label{disabled-access-to-gogo-shell-using-telnet}

\begin{itemize}
\tightlist
\item
  \textbf{Date:} 2018-Jun-25
\item
  \textbf{JIRA Ticket:}
  \href{https://issues.liferay.com/browse/LPS-82849}{LPS-82849}
\end{itemize}

\paragraph{What changed?}\label{what-changed-25}

The ability to access and interact with Liferay Portal's OSGi framework
using the Gogo shell via your system's telnet client has been disabled.

\paragraph{Who is affected?}\label{who-is-affected-25}

This affects anyone who used their system's telnet client to access the
Gogo shell, or leveraged the Gogo shell in external plugins/tooling
using the telnet client.

\paragraph{How should I update my
code?}\label{how-should-i-update-my-code-25}

Liferay Portal now offers the Gogo Shell portlet, which you can access
in the Control Panel → \emph{Configuration} → \emph{Gogo Shell}.

If you prefer using your telnet client to access the Gogo shell, you
must enable Developer Mode. You can do this by creating a
\texttt{portal-ext.properties} file in your Liferay home folder and
adding the following property:

\begin{verbatim}
include-and-override=portal-developer.properties
\end{verbatim}

Developer Mode is enabled upon starting your app server.

\paragraph{Why was this change made?}\label{why-was-this-change-made-25}

This was done to strengthen Liferay Portal's security due to potential
XXE/SSRF vulnerabilities.

\subsubsection{Removed Description HTML Escaping in
PortletDisplay}\label{removed-description-html-escaping-in-portletdisplay}

\begin{itemize}
\tightlist
\item
  \textbf{Date:} 2018-Jul-17
\item
  \textbf{JIRA Ticket:}
  \href{https://issues.liferay.com/browse/LPS-83185}{LPS-83185}
\end{itemize}

\paragraph{What changed?}\label{what-changed-26}

The portlet description stored in \texttt{PortletDisplay.java} is no
longer escaped automatically.

\paragraph{Who is affected?}\label{who-is-affected-26}

This affects anyone who relied on the portlet description's value
already being escaped and used it to generate HTML. In that case, a
small UI change might be observed as some characters could become
unescaped.

\paragraph{How should I update my
code?}\label{how-should-i-update-my-code-26}

If you were using the \texttt{portletDescription} value to generate
HTML, you should escape it using the proper escape sequence using
\texttt{HtmlUtil.escape}.

\paragraph{Why was this change made?}\label{why-was-this-change-made-26}

This change corrects a best practice violation regarding content
escaping.

\subsubsection{Changed modelName Attribute to be Mandatory in
liferay-ui:input-permissions
Taglib}\label{changed-modelname-attribute-to-be-mandatory-in-liferay-uiinput-permissions-taglib}

\begin{itemize}
\tightlist
\item
  \textbf{Date:} 2018-Oct-04
\item
  \textbf{JIRA Ticket:}
  \href{https://issues.liferay.com/browse/LPS-85998}{LPS-85998}
\end{itemize}

\paragraph{What changed?}\label{what-changed-27}

Previously, the taglib \texttt{liferay-ui:input-permissions} could be
used without providing the attribute \texttt{modelName}. Now the
attribute \texttt{modelName} is mandatory.

\paragraph{Who is affected?}\label{who-is-affected-27}

This affects any developer who used the taglib
\texttt{liferay-ui:input-permissions} in their own portlets and was not
setting the \texttt{modelName} attribute of the taglib.

\paragraph{How should I update my
code?}\label{how-should-i-update-my-code-27}

You should invoke the taglib providing the model name to which you are
assigning the permissions.

\paragraph{Why was this change made?}\label{why-was-this-change-made-27}

This change removes old logic that is no longer used in Liferay Portal.

\subsubsection{\texorpdfstring{Liferay
\texttt{AssetEntries\_AssetCategories} Is No Longer
Used}{Liferay AssetEntries\_AssetCategories Is No Longer Used}}\label{liferay-assetentries_assetcategories-is-no-longer-used}

\begin{itemize}
\tightlist
\item
  \textbf{Date:} 2019-Sep-11
\item
  \textbf{JIRA Tickets:}
  \href{https://issues.liferay.com/browse/LPS-99973}{LPS-99973},
  \href{https://issues.liferay.com/browse/LPS-76488}{LPS-76488}
\end{itemize}

\paragraph{What changed?}\label{what-changed-28}

Previously, Liferay used a mapping table and a corresponding interface
for the relationship between \texttt{AssetEntry} and
\texttt{AssetCategory} in \texttt{AssetEntryLocalService} and
\texttt{AssetCategoryLocalService}. This mapping table and the
corresponding interface have been replaced by the table
\texttt{AssetEntryAssetCategoryRel} and the service
\texttt{AssetEntryAssetCategoryRelLocalService}.

\paragraph{Who is affected?}\label{who-is-affected-28}

This affects any content or code that relies on calling the old
interfaces for the \texttt{AssetEntries\_AssetCategories} relationship,
through the \texttt{AssetEntryLocalService} and
\texttt{AssetCategoryLocalService}.

\paragraph{How should I update my
code?}\label{how-should-i-update-my-code-28}

Use the new methods in \texttt{AssetEntryAssetCategoryRelLocalService}
to retrieve the same data as before. The method signatures haven't
changed; they have just been relocated to a different service.

\textbf{Example}

Old way:

\begin{verbatim}
List<AssetEntry> entries =
AssetEntryLocalServiceUtil.getAssetCategoryAssetEntries(categoryId);

for (AssetEntry entry: entries) {
  ...
}
\end{verbatim}

New way:

\begin{verbatim}
long[] assetEntryPKs =
_assetEntryAssetCategoryRelLocalService.getAssetEntryPrimaryKeys(assetCategoryId);

for (long assetEntryPK: assetEntryPKs) {
  AssetEntry = _assetEntryLocalService.getEntry(assetEntryPK);
  ...
}

...

@Reference
private AssetEntryAssetCategoryRelLocalService _assetEntryAssetCategoryRelLocalService;

@Reference
private AssetEntryLocalService _assetEntryLocalService;
\end{verbatim}

\paragraph{Why was this change made?}\label{why-was-this-change-made-28}

This change was made due to changes resulting from
\href{https://issues.liferay.com/browse/LPS-76488}{LPS-76488}, which let
developers control the order of a list of assets for a given category.

\subsubsection{Removed Cache Bootstrap
Feature}\label{removed-cache-bootstrap-feature}

\begin{itemize}
\tightlist
\item
  \textbf{Date:} 2020-Jan-08
\item
  \textbf{JIRA Ticket:}
  \href{https://issues.liferay.com/browse/LPS-96563}{LPS-96563}
\end{itemize}

\paragraph{What changed?}\label{what-changed-29}

The cache bootstrap feature has been removed. These properties can no
longer be used to enable/configure cache bootstrap:

\begin{itemize}
\tightlist
\item
  \texttt{ehcache.bootstrap.cache.loader.enabled}
\item
  \texttt{ehcache.bootstrap.cache.loader.properties.default}
\item
  \texttt{ehcache.bootstrap.cache.loader.properties.\$\{specific.cache.name\}}
\end{itemize}

\paragraph{Who is affected?}\label{who-is-affected-29}

This affects anyone using the properties listed above.

\paragraph{How should I update my
code?}\label{how-should-i-update-my-code-29}

There's no direct replacement for the removed feature. If you have code
that depends on it, you must implement it yourself.

\paragraph{Why was this change made?}\label{why-was-this-change-made-29}

This change was made to avoid security issues.

\subsubsection{Web Content Description Field Is Controlled by
AlloyEditor}\label{web-content-description-field-is-controlled-by-alloyeditor}

\begin{itemize}
\tightlist
\item
  \textbf{Date:} 2020-Apr-07
\item
  \textbf{JIRA Ticket:}
  \href{https://issues.liferay.com/browse/LPS-71850}{LPS-71850}
\end{itemize}

\paragraph{What changed?}\label{what-changed-30}

Previously, the Web Content description field was a plain text field.
This field is now managed by AlloyEditor, so any HTML characters entered
into the field are escaped and rendered as plain text instead of HTML.

\paragraph{Who is affected?}\label{who-is-affected-30}

This affects explicit HTML tags in Web Content descriptions that a
developer expects to be rendered as regular HTML tags by the browser.

\paragraph{How should I update my
code?}\label{how-should-i-update-my-code-30}

If you want these values rendered as HTML, you must unescape them using
the proper unescape sequence: \texttt{HtmlUtil.unescape}.

For example, the FreeMarker expression
\texttt{\$\{.vars{[}\textquotesingle{}reserved-article-description\textquotesingle{}{]}.data\}}
should be unescaped like
\texttt{\$\{htmlUtil.unescape(.vars{[}\textquotesingle{}reserved-article-description\textquotesingle{}{]}.data)\}}

\paragraph{Why was this change made?}\label{why-was-this-change-made-30}

This change was made to take advantage of the AlloyEditor's styling and
formatting tools in Web Content description fields.
