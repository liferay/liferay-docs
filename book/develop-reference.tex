\chapter{Development Reference}\label{development-reference}

Here you'll find reference documentation for Liferay DXP, Liferay
Screens, Liferay Faces, and technologies related to you as a third-party
developer.

The different types of reference docs you'll find in this section of the
Liferay Developer Network are as follows:

\begin{itemize}
\tightlist
\item
  Descriptions of Java and JavaScript APIs, CSS, tags and tag libraries,
  and XML DTDs
\item
  Write ups on the latest Screenlets for Liferay Screens
\item
  Breaking changes
\item
  Cheat sheets and tips on

  \begin{itemize}
  \tightlist
  \item
    Plugin anatomy
  \item
    Design patterns
  \item
    Tools
  \item
    Adapting to new APIs
  \end{itemize}
\end{itemize}

Liferay's reference docs are at your fingertips.

\section{Java APIs}\label{java-apis}

Here you'll find Javadoc for Liferay DXP and Liferay DXP apps.

\subsection{7.0 Java APIs}\label{java-apis-1}

This table links you to the 7.0 API modules. Their root location is\\
here.{ (Opens New Window) } The reference doc JAR is available\\
here.{ (Opens New Window) }

Core:

com.liferay.portal.kernel (portal-kernel):{ (Opens New Window) } ~for
developing applications on Liferay DXP

com.liferay.util.bridges (util-bridges):{ (Opens New Window) } ~for
using various non-proprietary computing languages, frameworks, and
utilities on Liferay DXP

com.liferay.util.java (util-java):{ (Opens New Window) } ~for using
various Java-related frameworks and utilities on Liferay DXP

com.liferay.util.slf4j (util-slf4j):{ (Opens New Window) } ~for using
the Simple Logging Facade for Java (SLF4J)

com.liferay.portal.impl (portal-impl):{ (Opens New Window) } ~refer to
this only if you are an advanced Liferay developer that needs a deeper
understanding of 7.0's implementation in order to contribute to it

\subsection{Liferay DXP App Java APIs}\label{liferay-dxp-app-java-apis}

This table links you to Liferay DXP application APIs. Their root
location is here.{ (Opens New Window) }

Collaboration{ (Opens New Window) } (JAR){ (Opens New Window) }

com.liferay.blogs.api

com.liferay.blogs.item.selector.api

com.liferay.bookmarks.api

com.liferay.comment.api

com.liferay.document.library.api

com.liferay.document.library.repository.cmis.api

com.liferay.flags.api

com.liferay.invitation.invite.members.api

com.liferay.item.selector.api

com.liferay.item.selector.criteria.api

com.liferay.mentions.api

com.liferay.message.boards.api

com.liferay.microblogs.api

com.liferay.ratings.api

com.liferay.social.activity.api

com.liferay.social.privatemessaging.api

com.liferay.wiki.api

Forms \& Workflow{ (Opens New Window) } (JAR){ (Opens New Window) }

com.liferay.calendar.api

com.liferay.dynamic.data.lists.api

com.liferay.dynamic.data.mapping.api

com.liferay.polls.api

com.liferay.portal.reports.engine.api

com.liferay.portal.rules.engine.api

com.liferay.portal.workflow.kaleo.api

com.liferay.portal.workflow.kaleo.definition.api

com.liferay.portal.workflow.kaleo.runtime.api

Foundation{ (Opens New Window) } (JAR){ (Opens New Window) }

com.liferay.contacts.api

com.liferay.frontend.image.editor.api

com.liferay.map.api

com.liferay.mobile.device.rules.api

com.liferay.password.policies.admin.api

com.liferay.portal.background.task.api\\

com.liferay.portal.lock.api

com.liferay.portal.scripting.api

com.liferay.portal.security.audit.api

com.liferay.portal.security.exportimport.api

com.liferay.portal.security.service.access.policy.api

com.liferay.portal.settings.api

com.liferay.roles.admin.api

com.liferay.user.groups.admin.api

com.liferay.users.admin.api

com.liferay.users.admin.demo.data.creator.api

com.liferay.xstream.configurator.api

Web Experience{ (Opens New Window) } (JAR){ (Opens New Window) }

com.liferay.application.list.api

com.liferay.exportimport.api

com.liferay.journal.api

com.liferay.journal.item.selector.api

com.liferay.layout.item.selector.api

com.liferay.layout.prototype.api

com.liferay.layout.set.prototype.api

com.liferay.portlet.configuration.icon.locator.api

com.liferay.portlet.configuration.toolbar.contributor.locator.api

com.liferay.product.navigation.control.menu.api

com.liferay.site.api

com.liferay.site.item.selector.api

com.liferay.staging.api

For help finding API modules for specific common classes, see
\href{/docs/7-0/reference/-/knowledge_base/r/finding-liferay-api-modules}{7.0
API Modules}.

For help finding module attributes and configuring dependencies, see
\href{/docs/7-0/tutorials/-/knowledge_base/t/configuring-dependencies}{Configuring
Dependencies}.

\section{Taglibs}\label{taglibs}

Here you'll find tag library documentation for the Liferay DXP,
Liferay DXP apps, and Liferay Faces.

\subsection{7.0 Taglibs}\label{taglibs-1}

Util Taglibs{ (Opens New Window) }

aui

liferay-portlet

portlet

liferay-security

liferay-theme

liferay-ui

liferay-util

\subsection{Liferay DXP App Taglibs}\label{liferay-dxp-app-taglibs}

Application List:

liferay-application-list{ (Opens New Window) }

Assets:

liferay-asset{ (Opens New Window) }

liferay-trash{ (Opens New Window) }

Import, Export, \& Staging:

liferay-staging{ (Opens New Window) }

Item Selector:

liferay-item-selector{ (Opens New Window) }

Product Navigation:

liferay-product-navigation{ (Opens New Window) }

Sites:

liferay-layout{ (Opens New Window) }

liferay-site-navigation{ (Opens New Window) }

Social:

liferay-flags{ (Opens New Window) }

For help finding module attributes and configuring dependencies, see
\href{/docs/7-0/tutorials/-/knowledge_base/t/configuring-dependencies}{Configuring
Dependencies}.

\subsection{Faces Taglibs}\label{faces-taglibs}

\href{https://docs.liferay.com/faces/3.2/vdldoc/}{\textbf{Faces 3.2
Taglibs}}: the latest version of Liferay Faces JSF tag docs in View
Declaration Language (VDL) format. VDL docs for all versions of Liferay
Faces are available \href{http://docs.liferay.com/faces/}{here}.

\subsection{JavaScript and CSS}\label{javascript-and-css}

\href{https://liferay.github.io/clay/}{\textbf{Lexicon}}: The web
implementation of Liferay's \href{https://lexicondesign.io/}{Lexicon
Experience Language}. Lexicon is a system for building applications in
and outside of Liferay DXP, designed to be fluid and extensible, as well
as provide a consistent and documented API.

\href{http://getbootstrap.com/}{\textbf{Bootstrap}}: The base CSS
library onto which Lexicon is built. Liferay DXP uses Bootstrap natively
and all of its CSS classes and JavaScript features are available within
portlets, templates, and themes.

\href{http://alloyui.com}{\textbf{AlloyUI}}: Liferay includes AlloyUI
and all of its JavaScript APIs are available within portlets, templates
and themes.

\subsection{Descriptor Definitions}\label{descriptor-definitions}

\href{@platform-ref@/7.0-latest/definitions/}{\textbf{DTDs}}: Describes
the XML files used in configuring Liferay DXP apps, 7.0 plugins, and
@product-ver@.

\chapter{Liferay API Modules}\label{liferay-api-modules}

The following table maps commonly used Liferay DXP components to their
API modules and key classes. You
\href{/docs/7-0/tutorials/-/knowledge_base/t/configuring-dependencies}{configure
dependencies} on the component API modules to use them.

\section{API Modules Table}\label{api-modules-table}

\noindent\hrulefill

\begin{longtable}[]{@{}
  >{\raggedright\arraybackslash}p{(\columnwidth - 4\tabcolsep) * \real{0.1721}}
  >{\raggedright\arraybackslash}p{(\columnwidth - 4\tabcolsep) * \real{0.1393}}
  >{\raggedright\arraybackslash}p{(\columnwidth - 4\tabcolsep) * \real{0.6885}}@{}}
\toprule\noalign{}
\begin{minipage}[b]{\linewidth}\raggedright
Component
\end{minipage} & \begin{minipage}[b]{\linewidth}\raggedright
Classes
\end{minipage} & \begin{minipage}[b]{\linewidth}\raggedright
Module Symbolic Name (Artifact ID)
\end{minipage} \\
\midrule\noalign{}
\endhead
\bottomrule\noalign{}
\endlastfoot
Application List & \texttt{PanelApp} &
\texttt{com.liferay.application.list.api} \\
\texttt{PanelCategory} & & \\
\texttt{PanelEntry} & & \\
Background Tasks & \texttt{BackgroundTask{[}Local{]}ServiceUtil} &
\texttt{com.liferay.portal.background.task.api} \\
Blogs & \texttt{BlogsEntry{[}Local{]}ServiceUtil} &
\texttt{com.liferay.blogs.api} \\
Bookmarks & \texttt{BookmarksEntry{[}Local{]}ServiceUtil} &
\texttt{com.liferay.bookmarks.api} \\
\texttt{BookmarksFolder{[}Local{]}ServiceUtil} & & \\
Calendar & \texttt{Calendar{[}Local{]}ServiceUtil} &
\texttt{com.liferay.calendar.api} \\
\texttt{CalendarBooking{[}Local{]}ServiceUtil} & & \\
\texttt{CalendarImporter} & & \\
\texttt{CalendarNotificationTemplate{[}Local{]}ServiceUtil} & & \\
\texttt{CalendarResource{[}Local{]}ServiceUtil} & & \\
Comment & \texttt{Comment} & \texttt{com.liferay.comment.api} \\
\texttt{DiscussionComment} & & \\
Contacts & \texttt{Entry} & \texttt{com.liferay.contacts.api} \\
Document Library & \texttt{DLFileEntry{[}Local{]}ServiceUtil} &
\texttt{com.liferay.document.library.api} \\
\texttt{DLContent{[}Local{]}ServiceUtil} & & \\
\texttt{DLFileEntryType{[}Local{]}ServiceUtil} & & \\
\texttt{DLFileVersion{[}Local{]}ServiceUtil} & & \\
\texttt{DLFolder{[}Local{]}ServiceUtil} & & \\
\texttt{DLSyncEvent{[}Local{]}ServiceUtil} & & \\
Dynamic Data Lists & \texttt{DDLRecord{[}Local{]}ServiceUtil} &
\texttt{com.liferay.dynamic.data.lists.api} \\
\texttt{DDLRecordSet{[}Local{]}ServiceUtil} & & \\
\texttt{DDLRecordVersion{[}Local{]}ServiceUtil} & & \\
Dynamic Data Mapping & \texttt{DDMContent{[}Local{]}ServiceUtil} &
\texttt{com.liferay.dynamic.data.mapping.api} \\
\texttt{DDMStructure{[}Local{]}ServiceUtil} & & \\
\texttt{DDMStorageLink{[}Local{]}ServiceUtil} & & \\
\texttt{DDMStructureLayout{[}Local{]}ServiceUtil} & & \\
\texttt{DDMStructureLink{[}Local{]}ServiceUtil} & & \\
\texttt{DDMStructureVersion{[}Local{]}ServiceUtil} & & \\
\texttt{DDMTemplate{[}Local{]}ServiceUtil} & & \\
\texttt{DDMTemplateLink{[}Local{]}ServiceUtil} & & \\
\texttt{DDMTemplateVersion{[}Local{]}ServiceUtil} & & \\
Export / Import & \texttt{ExportImportConfiguration} &
\texttt{com.liferay.exportimport.api} \\
\texttt{StagingServiceUtil{[}Local{]}ServiceUtil} & & \\
Flags & \texttt{FlagsEntryServiceUtil} &
\texttt{com.liferay.flags.api} \\
Invitation & \texttt{MemberRequest{[}Local{]}ServiceUtil} &
\texttt{com.liferay.invitation.invite.members.api} \\
Item Selector & \texttt{ItemSelector} &
\texttt{com.liferay.item.selector.api} \\
Item Selector Criteria & \texttt{FileEntryItemSelectorReturnType} &
\texttt{com.liferay.item.selector.criteria.api} \\
\texttt{UploadableFileReturnType} & & \\
\texttt{URLItemSelectorReturnType} & & \\
\texttt{UUIDItemSelectorReturnType} & & \\
Lock & \texttt{Lock} & \texttt{com.liferay.portal.lock.api} \\
Map & \texttt{MapProvider} & \texttt{com.liferay.map.api} \\
Marketplace & \texttt{App} & \texttt{com.liferay.marketplace.api} \\
\texttt{Module} & & \\
Mentions & \texttt{MentionsNotifier} &
\texttt{com.liferay.mentions.api} \\
\texttt{MentionsUserFinder} & & \\
\texttt{MentionsUtil} & & \\
Message Boards & \texttt{MBMessage{[}Local{]}ServiceUtil} &
\texttt{com.liferay.message.boards.api} \\
\texttt{MBCategory{[}Local{]}ServiceUtil} & & \\
\texttt{MBThread{[}Local{]}ServiceUtil} & & \\
\texttt{MBDiscussion{[}Local{]}ServiceUtil} & & \\
Microblogs & \texttt{MicroblogsEntry{[}Local{]}ServiceUtil} &
\texttt{com.liferay.microblogs.api} \\
Mobile Device Rules & \texttt{MDRAction{[}Local{]}ServiceUtil} &
\texttt{com.liferay.mobile.device.rules.api} \\
\texttt{MDRRule{[}Local{]}ServiceUtil} & & \\
\texttt{MDRRuleGroup{[}Local{]}ServiceUtil} & & \\
\texttt{MDRRuleGroupInstance{[}Local{]}ServiceUtil} & & \\
Polls & \texttt{PollsChoice{[}Local{]}ServiceUtil} &
\texttt{com.liferay.polls.api} \\
\texttt{PollsQuestion{[}Local{]}ServiceUtil} & & \\
\texttt{PollsVote{[}Local{]}ServiceUtil} & & \\
Portal Access Policy & \texttt{SAPEntry{[}Local{]}ServiceUtil} &
\texttt{com.liferay.portal.security.service.access.policy.api} \\
Portal Settings & \texttt{PortalSettings} &
\texttt{com.liferay.portal.settings.api} \\
Portlet Configuration & \texttt{PortletConfigurationIcon} &
\texttt{com.liferay.portlet.configuration.icon.locator.api} \\
\texttt{PortletToolbar} &
\texttt{com.liferay.portlet.configuration.toolbar.contributor.locator.api}
& \\
Private Messaging & \texttt{UserThread{[}Local{]}ServiceUtil} &
\texttt{com.liferay.social.privatemessaging.api} \\
Product Navigation & \texttt{ProductNavigationControlMenuCategory} &
\texttt{com.liferay.product.navigation.control.menu.api} \\
\texttt{ProductNavigationControlMenuEntry} & & \\
Ratings & \texttt{RatingsEntry{[}Local{]}ServiceUtil} &
\texttt{com.liferay.ratings.api} \\
Reports Engine & \texttt{RulesEngine} & \texttt{reports.engine.api} \\
\texttt{RulesLanguage} & & \\
\texttt{Fact} & & \\
\texttt{Query} & & \\
Screens & \texttt{ScreensAssetEntryServiceUtil} &
\texttt{com.liferay.screens.api} \\
\texttt{ScreensDDLRecordServiceUtil} & & \\
\texttt{ScreensJournalArticleServiceUtil} & & \\
Security Audit & \texttt{AuditEvent} &
\texttt{com.liferay.portal.security.audit.api} \\
\texttt{AuditEventManager} & & \\
\texttt{AuditConfiguration} & & \\
Security Import / Export & \texttt{UserExporter} &
\texttt{com.liferay.portal.security.exportimport.api} \\
\texttt{UserImporter} & & \\
\texttt{UserOperation} & & \\
Shopping Cart & \texttt{ShoppingCart{[}Local{]}ServiceUtil} &
\texttt{com.liferay.shopping.api} \\
\texttt{ShoppingCategory{[}Local{]}ServiceUtil} & & \\
\texttt{ShoppingCoupon{[}Local{]}ServiceUtil} & & \\
\texttt{ShoppingItem{[}Local{]}ServiceUtil} & & \\
\texttt{ShoppingItemPrice{[}Local{]}ServiceUtil} & & \\
\texttt{ShoppingOrder{[}Local{]}ServiceUtil} & & \\
\texttt{ShoppingOrderItem{[}Local{]}ServiceUtil} & & \\
Site & \texttt{GroupSearchProvider} & \texttt{com.liferay.site.api} \\
Social Networking & \texttt{MeetupsEntry{[}Local{]}ServiceUtil} &
\texttt{com.liferay.social.networking.api} \\
\texttt{MeetupsRegistration{[}Local{]}ServiceUtil} & & \\
\texttt{WallEntry{[}Local{]}ServiceUtil} & & \\
Staging & \texttt{Staging{[}Local{]}ServiceUtil} &
\texttt{com.liferay.staging.api} \\
Web Content & \texttt{JournalArticle{[}Local{]}ServiceUtil} &
\texttt{com.liferay.journal.api} \\
\texttt{JournalFolder{[}Local{]}ServiceUtil} & & \\
\texttt{JournalArticleImage{[}Local{]}ServiceUtil} & & \\
\texttt{JournalFeed{[}Local{]}ServiceUtil} & & \\
Wiki & \texttt{WikiNode{[}Local{]}ServiceUtil} &
\texttt{com.liferay.wiki.api} \\
\texttt{WikiPage{[}Local{]}ServiceUtil} & & \\
XStream Configurator & \texttt{XStreamConfigurator} &
\texttt{com.liferay.xstream.configurator.api} \\
\end{longtable}

\noindent\hrulefill

For reference documentation on these APIs and others, see the the app
reference docs at \href{@app-ref@}{Liferay DXP} and the Liferay DXP core
reference docs at
\href{@platform-ref@/7.0-latest}{@platform-ref@/7.0-latest}.

\subsection{Related Articles}\label{related-articles}

\href{/docs/7-0/tutorials/-/knowledge_base/t/configuring-dependencies}{Configuring
Dependencies}

\href{/docs/7-0/reference/-/knowledge_base/r/development-reference}{Development
Reference}

\href{/docs/7-0/tutorials/-/knowledge_base/t/liferay-upgrade-planner}{Liferay
Upgrade Planner}

\chapter{Portlet Descriptor to OSGi Service Property
Map}\label{portlet-descriptor-to-osgi-service-property-map}

This section describes the mapping of portlet XML descriptor values to
OSGi service properties that can be used when publishing OSGi Portlets.

OSGi services can contain properties in their definitions. Using OSGi
service properties makes dealing with configuration concerns simple and
cohesive. These properties are typically represented as key-value pairs
or, more generally, as a Map-like object.

Portlet spec property keys are prefixed by:

\begin{verbatim}
javax.portlet.
\end{verbatim}

Liferay property keys are prefixed by:

\begin{verbatim}
com.liferay.portlet.
\end{verbatim}

The mappings essentially flatten what is found in the XML descriptor,
sticking relatively closely to the original naming in order to have a
memorable relationship with those definitions.

\subsection{JSR-168 \& JSR-286 Descriptor
Mappings}\label{jsr-168-jsr-286-descriptor-mappings}

\textbf{Note:} XPath notation derived from the \textbf{Portlet XSD}
\hyperref[four]{4} is used in this document for simplicity.

\noindent\hrulefill

\texttt{portlet.xml} XPath \textbar{} OSGi Portlet Service
Property\textbar{}
\texttt{/portlet-app/portlet/description}\textbar{}\texttt{javax.portlet.description=\textless{}String\textgreater{}}\textbar{}
\texttt{/portlet-app/portlet/portlet-name}
\hyperref[six]{6}\textbar{}\texttt{javax.portlet.name=\textless{}String\textgreater{}}
\hyperref[six]{6}\textbar{}
\texttt{/portlet-app/portlet/display-name}\textbar{}\texttt{javax.portlet.display-name=\textless{}String\textgreater{}}\textbar{}
\texttt{/portlet-app/portlet/portlet-class}\textbar{}\hyperref[one]{1}\textbar{}
\texttt{/portlet-app/portlet/init-param/name}\textbar{}\texttt{javax.portlet.init-param.\textless{}name\textgreater{}=\textless{}value\textgreater{}}\textbar{}
\texttt{/portlet-app/portlet/expiration-cache}\textbar{}\texttt{javax.portlet.expiration-cache=\textless{}int\textgreater{}}\textbar{}
\texttt{/portlet-app/portlet/cache-scope}\textbar not
supported\textbar{}
\texttt{/portlet-app/portlet/supports/mime-type}\textbar{}\texttt{javax.portlet.mime-type=\textless{}mime-type\textgreater{}}\textbar{}
\texttt{/portlet-app/portlet/supports/portlet-mode}\textbar{}\texttt{javax.portlet.portlet-mode=\textless{}mime-type\textgreater{};\textless{}portlet-mode\textgreater{}{[},\textless{}portlet-mode\textgreater{}{]}*}\textbar{}
\texttt{/portlet-app/portlet/supports/window-state}\textbar{}\texttt{javax.portlet.window-state=\textless{}mime-type\textgreater{};\textless{}window-state\textgreater{}{[},\textless{}window-state\textgreater{}{]}*}\textbar{}
\texttt{/portlet-app/portlet/supported-locale}\textbar not
supported\textbar{}
\texttt{/portlet-app/portlet/resource-bundle}\textbar{}\texttt{javax.portlet.resource-bundle=\textless{}String\textgreater{}}\textbar{}
\texttt{/portlet-app/portlet/portlet-info/title}\textbar{}\texttt{javax.portlet.info.title=\textless{}String\textgreater{}}\textbar{}
\texttt{/portlet-app/portlet/portlet-info/short-title}\textbar{}\texttt{javax.portlet.info.short-title=\textless{}String\textgreater{}}\textbar{}
\texttt{/portlet-app/portlet/portlet-info/keywords}\textbar{}\texttt{javax.portlet.info.keywords=\textless{}String\textgreater{}}\textbar{}
\texttt{/portlet-app/portlet/portlet-preferences}\textbar{}\texttt{javax.portlet.preferences=\textless{}String\textgreater{}}OR\texttt{javax.portlet.preferences=classpath:\textless{}path\_to\_file\_in\_jar\textgreater{}}\textbar{}
\texttt{/portlet-app/portlet/security-role-ref}\textbar{}\texttt{javax.portlet.security-role-ref=\textless{}String\textgreater{}{[},\textless{}String\textgreater{}{]}}\hyperref[two]{2}\textbar{}
\texttt{/portlet-app/portlet/supported-processing-event/name}\textbar{}\texttt{javax.portlet.supported-processing-event=\textless{}String\textgreater{}}OR\texttt{javax.portlet.supported-processing-event=\textless{}String\textgreater{};\textless{}QName\textgreater{}}\hyperref[two]{2}\textbar{}
\texttt{/portlet-app/portlet/supported-publishing-event}\textbar{}\texttt{javax.portlet.supported-publishing-event=\textless{}String\textgreater{}}OR\texttt{javax.portlet.supported-publishing-event=\textless{}String\textgreater{};\textless{}QName\textgreater{}}\hyperref[two]{2}\textbar{}
\texttt{/portlet-app/portlet/supported-public-render-parameter}\textbar{}\texttt{javax.portlet.supported-public-render-parameter=\textless{}String\textgreater{}}\hyperref[two]{2}\textbar{}
\texttt{/portlet-app/portlet/container-runtime-option}\textbar not
supported\textbar{}
\texttt{/portlet-app/custom-portlet-mode}\textbar not
supported\textbar{}
\texttt{/portlet-app/custom-window-state}\textbar not
supported\textbar{} \texttt{/portlet-app/user-attribute}\textbar not
supported\textbar{}
\texttt{/portlet-app/security-constraint}\textbar not
supported\textbar{} \texttt{/portlet-app/resource-bundle}\textbar not
supported\textbar{}
\texttt{/portlet-app/filter}\texttt{/portlet-app/filter-mapping}\textbar{}\hyperref[three]{3}\textbar{}
\texttt{/portlet-app/default-namespace}\textbar not supported\textbar{}
\texttt{/portlet-app/event-definition}\textbar not supported\textbar{}
\texttt{/portlet-app/filter/init-param/name}\textbar{}\texttt{javax.portlet.init-param.\textless{}name\textgreater{}=\textless{}value\textgreater{}}\textbar{}
\texttt{/portlet-app/public-render-parameter}\textbar not
supported\textbar{} \texttt{/portlet-app/listener}\textbar not
supported?\texttt{javax.portlet.PortletURLGenerationListener}?\textbar{}
\texttt{/portlet-app/container-runtime-option}\textbar not
supported\textbar{}

\noindent\hrulefill

\subsection{Liferay Descriptor
Mappings}\label{liferay-descriptor-mappings}

\subsubsection{Liferay Display}\label{liferay-display}

\noindent\hrulefill

\texttt{liferay-display.xml} XPath \textbar{} OSGi Portlet Service
Property\textbar{}
\texttt{/display/category\textbackslash{}{[}@name\textbackslash{}{]}}\textbar{}\texttt{com.liferay.portlet.display-category=\textless{}value\textgreater{}}\textbar{}

\noindent\hrulefill

\subsubsection{Liferay Portlet}\label{liferay-portlet}

\textbf{Note:} XPath notation derived from \textbf{Liferay Portlet}
\hyperref[five]{5} is used in this document for simplicity.

\noindent\hrulefill

\texttt{liferay-portlet.xml} XPath \textbar{} OSGi Liferay Portlet
Service Property\textbar{}
\texttt{/liferay-portlet-app/portlet/portlet-name}\textbar not
supported\textbar{}
\texttt{/liferay-portlet-app/portlet/icon}\textbar{}\texttt{com.liferay.portlet.icon=\textless{}String\textgreater{}}\textbar{}
\texttt{/liferay-portlet-app/portlet/virtual-path}\textbar{}\texttt{com.liferay.portlet.virtual-path=\textless{}String\textgreater{}}\textbar{}
\texttt{/liferay-portlet-app/portlet/struts-path}\textbar{}\texttt{com.liferay.portlet.struts-path=\textless{}String\textgreater{}}\textbar{}
\texttt{/liferay-portlet-app/portlet/parent-struts-path}\textbar{}\texttt{com.liferay.portlet.parent-struts-path=\textless{}String\textgreater{}}\textbar{}
\texttt{/liferay-portlet-app/portlet/configuration-path}\textbar{}\texttt{com.liferay.portlet.configuration-path=\textless{}String\textgreater{}}\textbar{}
\texttt{/liferay-portlet-app/portlet/configuration-action-class}\textbar{}\hyperref[three]{3}\textbar{}
\texttt{/liferay-portlet-app/portlet/indexer-class}\textbar{}\hyperref[three]{3}\textbar{}
\texttt{/liferay-portlet-app/portlet/open-search-class}\textbar{}\hyperref[three]{3}\textbar{}
\texttt{/liferay-portlet-app/portlet/scheduler-entry}\textbar{}\hyperref[three]{3}\textbar{}
\texttt{/liferay-portlet-app/portlet/portlet-url-class}\textbar{}\hyperref[three]{3}\textbar{}
\texttt{/liferay-portlet-app/portlet/friendly-url-mapper-class}\textbar{}\hyperref[three]{3}\textbar{}
\texttt{/liferay-portlet-app/portlet/friendly-url-mapping}\textbar{}\texttt{com.liferay.portlet.friendly-url-mapping=\textless{}String\textgreater{}}\textbar{}
\texttt{/liferay-portlet-app/portlet/friendly-url-routes}\textbar{}\texttt{com.liferay.portlet.friendly-url-routes=\textless{}String\textgreater{}}\textbar{}
\texttt{/liferay-portlet-app/portlet/url-encoder-class}\textbar{}\hyperref[three]{3}\textbar{}
\texttt{/liferay-portlet-app/portlet/portlet-data-handler-class}\textbar{}\hyperref[three]{3}\textbar{}
\texttt{/liferay-portlet-app/portlet/staged-model-data-handler-class}\textbar{}\hyperref[three]{3}\textbar{}
\texttt{/liferay-portlet-app/portlet/template-handler}\textbar{}\hyperref[three]{3}\textbar{}
\texttt{/liferay-portlet-app/portlet/portlet-layout-listener-class}\textbar{}\hyperref[three]{3}\textbar{}
\texttt{/liferay-portlet-app/portlet/poller-processor-class}\textbar{}\hyperref[three]{3}\textbar{}
\texttt{/liferay-portlet-app/portlet/pop-message-listener-class}\textbar{}\hyperref[three]{3}\textbar{}
\texttt{/liferay-portlet-app/portlet/social-activity-interpreter-class}\textbar{}\hyperref[three]{3}\textbar{}
\texttt{/liferay-portlet-app/portlet/social-request-interpreter-class}\textbar{}\hyperref[three]{3}\textbar{}
\texttt{/liferay-portlet-app/portlet/social-interactions-configuration}\textbar{}\hyperref[three]{3}\textbar{}
\texttt{/liferay-portlet-app/portlet/user-notification-definitions}\textbar not
supported\textbar{}
\texttt{/liferay-portlet-app/portlet/user-notification-handler-class}\textbar{}\hyperref[three]{3}\textbar{}
\texttt{/liferay-portlet-app/portlet/webdav-storage-token}\textbar not
supported\textbar{}
\texttt{/liferay-portlet-app/portlet/webdav-storage-class}\textbar{}\hyperref[three]{3}\textbar{}
\texttt{/liferay-portlet-app/portlet/xml-rpc-method-class}\textbar{}\hyperref[three]{3}\textbar{}
\texttt{/liferay-portlet-app/portlet/control-panel-entry-category}\textbar{}\texttt{com.liferay.portlet.control-panel-entry-category=\textless{}String\textgreater{}}\textbar{}
\texttt{/liferay-portlet-app/portlet/control-panel-entry-weight}\textbar{}\texttt{com.liferay.portlet.control-panel-entry-weight=\textless{}double\textgreater{}}\textbar{}
\texttt{/liferay-portlet-app/portlet/control-panel-entry-class}\textbar{}\hyperref[three]{3}\textbar{}
\texttt{/liferay-portlet-app/portlet/asset-renderer-factory}\textbar{}\hyperref[three]{3}\textbar{}
\texttt{/liferay-portlet-app/portlet/atom-collection-adapter}\textbar{}\hyperref[three]{3}\textbar{}
\texttt{/liferay-portlet-app/portlet/custom-attributes-display}\textbar{}\hyperref[three]{3}\textbar{}
\texttt{/liferay-portlet-app/portlet/ddm-display}\textbar{}\hyperref[three]{3}\textbar{}
\texttt{/liferay-portlet-app/portlet/permission-propagator}\textbar{}\hyperref[three]{3}\textbar{}
\texttt{/liferay-portlet-app/portlet/trash-handler}\textbar{}\hyperref[three]{3}\textbar{}
\texttt{/liferay-portlet-app/portlet/workflow-handler}\textbar{}\hyperref[three]{3}\textbar{}
\texttt{/liferay-portlet-app/portlet/preferences-company-wide}\textbar{}\texttt{com.liferay.portlet.preferences-company-wide=\textless{}boolean\textgreater{}}\textbar{}
\texttt{/liferay-portlet-app/portlet/preferences-unique-per-layout}\textbar{}\texttt{com.liferay.portlet.preferences-unique-per-layout=\textless{}boolean\textgreater{}}\textbar{}
\texttt{/liferay-portlet-app/portlet/preferences-owned-by-group}\textbar{}\texttt{com.liferay.portlet.preferences-owned-by-group=\textless{}boolean\textgreater{}}\textbar{}
\texttt{/liferay-portlet-app/portlet/use-default-template}\textbar{}\texttt{com.liferay.portlet.use-default-template=\textless{}boolean\textgreater{}}\textbar{}
\texttt{/liferay-portlet-app/portlet/show-portlet-access-denied}\textbar{}\texttt{com.liferay.portlet.show-portlet-access-denied=\textless{}boolean\textgreater{}}\textbar{}
\texttt{/liferay-portlet-app/portlet/show-portlet-inactive}\textbar{}\texttt{com.liferay.portlet.show-portlet-inactive=\textless{}boolean\textgreater{}}\textbar{}
\texttt{/liferay-portlet-app/portlet/action-url-redirect}\textbar{}\texttt{com.liferay.portlet.action-url-redirect=\textless{}boolean\textgreater{}}\textbar{}
\texttt{/liferay-portlet-app/portlet/restore-current-view}\textbar{}\texttt{com.liferay.portlet.restore-current-view=\textless{}boolean\textgreater{}}\textbar{}
\texttt{/liferay-portlet-app/portlet/maximize-edit}\textbar{}\texttt{com.liferay.portlet.maximize-edit=\textless{}boolean\textgreater{}}\textbar{}
\texttt{/liferay-portlet-app/portlet/maximize-help}\textbar{}\texttt{com.liferay.portlet.maximize-help=\textless{}boolean\textgreater{}}\textbar{}
\texttt{/liferay-portlet-app/portlet/pop-up-print}\textbar{}\texttt{com.liferay.portlet.pop-up-print=\textless{}boolean\textgreater{}}\textbar{}
\texttt{/liferay-portlet-app/portlet/layout-cacheable}\textbar{}\texttt{com.liferay.portlet.layout-cacheable=\textless{}boolean\textgreater{}}\textbar{}
\texttt{/liferay-portlet-app/portlet/instanceable}\textbar{}\texttt{com.liferay.portlet.instanceable=\textless{}boolean\textgreater{}}\textbar{}
\texttt{/liferay-portlet-app/portlet/remoteable}\textbar{}\texttt{com.liferay.portlet.remoteable=\textless{}boolean\textgreater{}}\textbar{}
\texttt{/liferay-portlet-app/portlet/scopeable}\textbar{}\texttt{com.liferay.portlet.scopeable=\textless{}boolean\textgreater{}}\textbar{}
\texttt{/liferay-portlet-app/portlet/single-page-application}\textbar{}\texttt{com.liferay.portlet.single-page-application=\textless{}boolean\textgreater{}}\textbar{}
\texttt{/liferay-portlet-app/portlet/user-principal-strategy}\textbar{}\texttt{com.liferay.portlet.user-principal-strategy=\textless{}String\textgreater{}}\textbar{}
\texttt{/liferay-portlet-app/portlet/private-request-attributes}\textbar{}\texttt{com.liferay.portlet.private-request-attributes=\textless{}boolean\textgreater{}}\textbar{}
\texttt{/liferay-portlet-app/portlet/private-session-attributes}\textbar{}\texttt{com.liferay.portlet.private-session-attributes=\textless{}boolean\textgreater{}}\textbar{}
\texttt{/liferay-portlet-app/portlet/autopropagated-parameters}\textbar{}\texttt{com.liferay.portlet.autopropagated-parameters=\textless{}String\textgreater{}}\hyperref[two]{2}\textbar{}
\texttt{/liferay-portlet-app/portlet/requires-namespaced-parameters}\textbar{}\texttt{com.liferay.portlet.requires-namespaced-parameters=\textless{}boolean\textgreater{}}\textbar{}
\texttt{/liferay-portlet-app/portlet/action-timeout}\textbar{}\texttt{com.liferay.portlet.action-timeout=\textless{}int\textgreater{}}\textbar{}
\texttt{/liferay-portlet-app/portlet/render-timeout}\textbar{}\texttt{com.liferay.portlet.render-timeout=\textless{}int\textgreater{}}\textbar{}
\texttt{/liferay-portlet-app/portlet/render-weight}\textbar{}\texttt{com.liferay.portlet.render-weight=\textless{}int\textgreater{}}\textbar{}
\texttt{/liferay-portlet-app/portlet/ajaxable}\textbar{}\texttt{com.liferay.portlet.ajaxable=\textless{}boolean\textgreater{}}\textbar{}
\texttt{/liferay-portlet-app/portlet/header-portal-css}\textbar{}\texttt{com.liferay.portlet.header-portal-css=\textless{}String\textgreater{}}\hyperref[two]{2}\textbar{}
\texttt{/liferay-portlet-app/portlet/header-portlet-css}\textbar{}\texttt{com.liferay.portlet.header-portlet-css=\textless{}String\textgreater{}}\hyperref[two]{2}\textbar{}
\texttt{/liferay-portlet-app/portlet/header-portal-javascript}\textbar{}\texttt{com.liferay.portlet.header-portal-javascript=\textless{}String\textgreater{}}\hyperref[two]{2}\textbar{}
\texttt{/liferay-portlet-app/portlet/header-portlet-javascript}\textbar{}\texttt{com.liferay.portlet.header-portlet-javascript=\textless{}String\textgreater{}}\hyperref[two]{2}\textbar{}
\texttt{/liferay-portlet-app/portlet/footer-portal-css}\textbar{}\texttt{com.liferay.portlet.footer-portal-css=\textless{}String\textgreater{}}\hyperref[two]{2}\textbar{}
\texttt{/liferay-portlet-app/portlet/footer-portlet-css}\textbar{}\texttt{com.liferay.portlet.footer-portlet-css=\textless{}String\textgreater{}}\hyperref[two]{2}\textbar{}
\texttt{/liferay-portlet-app/portlet/footer-portal-javascript}\textbar{}\texttt{com.liferay.portlet.footer-portal-javascript=\textless{}String\textgreater{}}\hyperref[two]{2}\textbar{}
\texttt{/liferay-portlet-app/portlet/footer-portlet-javascript}\textbar{}\texttt{com.liferay.portlet.footer-portlet-javascript=\textless{}String\textgreater{}}\hyperref[two]{2}\textbar{}
\texttt{/liferay-portlet-app/portlet/css-class-wrapper}\textbar{}\texttt{com.liferay.portlet.css-class-wrapper=\textless{}String\textgreater{}}\textbar{}
\texttt{/liferay-portlet-app/portlet/facebook-integration}\textbar{}\texttt{com.liferay.portlet.facebook-integration=\textless{}String\textgreater{}}\textbar{}
\texttt{/liferay-portlet-app/portlet/add-default-resource}\textbar{}\texttt{com.liferay.portlet.add-default-resource=\textless{}boolean\textgreater{}}\textbar{}
\texttt{/liferay-portlet-app/portlet/system}\textbar{}\texttt{com.liferay.portlet.system=\textless{}boolean\textgreater{}}\textbar{}
\texttt{/liferay-portlet-app/portlet/active}\textbar{}\texttt{com.liferay.portlet.active=\textless{}boolean\textgreater{}}\textbar{}
\texttt{/liferay-portlet-app/portlet/include}\textbar not
supported\textbar{}

\noindent\hrulefill

\begin{itemize}
\item
  {[}1{]} Portlets are registered as concrete objects.
\item
  {[}2{]} Multiples of these properties may be used. This results in an
  array of values.
\item
  {[}3{]} This type is registered as an OSGi service.
\item
  {[}4{]} http://java.sun.com/xml/ns/portlet/portlet-app\_2\_0.xsd
\item
  {[}5{]} http://www.liferay.com/dtd/liferay-portlet-app\_7\_0\_0.dtd
\item
  {[}6{]} Liferay DXP creates each portlet's ID based on the portlet's
  name (i.e., the \texttt{portlet-name} descriptor in
  \texttt{liferay-portlet.xml} or the \texttt{javax.portlet.name} OSGi
  service property). Dashes, periods, and spaces are allowed in the
  portlet name, but they and all other JavaScript unsafe characters are
  stripped from the name value that's used for the portlet ID.
  Therefore, make your portlet name unique in light of the characters
  that are removed. Otherwise, if you try to deploy a portlet whose ID
  is the same as a portlet that's already deployed, your portlet
  deployment fails and Liferay DXP logs a message like this:

\begin{verbatim}
Portlet id [portletId] is already in use
\end{verbatim}
\end{itemize}

\chapter{Classes Moved from
portal-service.jar}\label{classes-moved-from-portal-service.jar}

To leverage the benefits of modularization in 7.0, many classes from
former Liferay Portal 6 JAR file portal-service.jar have been moved into
application and framework API modules. The table below provides details
about these classes and the modules they've moved to. Package changes
and each module's symbolic name (artifact ID) are listed, to facilitate
configuring dependencies.

Classes Moved from portal-service to modules

This information was generated based on comparing classes in
liferay-portal-src-6.2-ce-ga6 to classes in
liferay-portal-src-7.0-ce-ga1.

Class

Package

Module Symbolic Name (Artifact ID)

ActionHandler

Old: com.liferay.portal.kernel.mobile.device.rulegroup.action New:
com.liferay.mobile.device.rules.action

com.liferay.mobile.device.rules.api

ActionHandlerManager

Old: com.liferay.portal.kernel.mobile.device.rulegroup New:
com.liferay.mobile.device.rules.action

com.liferay.mobile.device.rules.api

ActionHandlerManagerUtil

Old: com.liferay.portal.kernel.mobile.device.rulegroup New:
com.liferay.mobile.device.rules.action

com.liferay.mobile.device.rules.api

ActionTypeException

Old: com.liferay.portlet.mobiledevicerules New:
com.liferay.mobile.device.rules.exception

com.liferay.mobile.device.rules.api

AlternateKeywordQueryHitsProcessor

Old: com.liferay.portal.kernel.search New:
com.liferay.portal.search.internal.hits

com.liferay.portal.search

ArticleContentException

Old: com.liferay.portlet.journal New: com.liferay.journal.exception

com.liferay.journal.api

ArticleContentSizeException

Old: com.liferay.portlet.journal New: com.liferay.journal.exception

com.liferay.journal.api

ArticleCreateDateComparator

Old: com.liferay.portlet.journal.util.comparator New:
com.liferay.journal.util.comparator

com.liferay.journal.api

ArticleDisplayDateComparator

Old: com.liferay.portlet.journal.util.comparator New:
com.liferay.journal.util.comparator

com.liferay.journal.api

ArticleDisplayDateException

Old: com.liferay.portlet.journal New: com.liferay.journal.exception

com.liferay.journal.api

ArticleExpirationDateException

Old: com.liferay.portlet.journal New: com.liferay.journal.exception

com.liferay.journal.api

ArticleIDComparator

Old: com.liferay.portlet.journal.util.comparator New:
com.liferay.journal.util.comparator

com.liferay.journal.api

ArticleIdException

Old: com.liferay.portlet.journal New: com.liferay.journal.exception

com.liferay.journal.api

ArticleModifiedDateComparator

Old: com.liferay.portlet.journal.util.comparator New:
com.liferay.journal.util.comparator

com.liferay.journal.api

ArticleReviewDateComparator

Old: com.liferay.portlet.journal.util.comparator New:
com.liferay.journal.util.comparator

com.liferay.journal.api

ArticleReviewDateException

Old: com.liferay.portlet.journal New: com.liferay.journal.exception

com.liferay.journal.api

ArticleSmallImageNameException

Old: com.liferay.portlet.journal New: com.liferay.journal.exception

com.liferay.journal.api

ArticleSmallImageSizeException

Old: com.liferay.portlet.journal New: com.liferay.journal.exception

com.liferay.journal.api

ArticleTitleComparator

Old: com.liferay.portlet.journal.util.comparator New:
com.liferay.journal.util.comparator

com.liferay.journal.api

ArticleTitleException

Old: com.liferay.portlet.journal New: com.liferay.journal.exception

com.liferay.journal.api

ArticleVersionComparator

Old: com.liferay.portlet.journal.util.comparator New:
com.liferay.journal.util.comparator

com.liferay.journal.api

ArticleVersionException

Old: com.liferay.portlet.journal New: com.liferay.journal.exception

com.liferay.journal.api

AssetPublisherUtil

Old: com.liferay.portlet.assetpublisher.util New:
com.liferay.asset.publisher.web.util

com.liferay.asset.publisher.web

AuditMessageProcessor

Old: com.liferay.portal.kernel.audit New:
com.liferay.portal.security.audit

com.liferay.portal.security.audit.api

AverageStatistics

Old: com.liferay.portal.kernel.monitoring.statistics New:
com.liferay.portal.monitoring.internal.statistics

com.liferay.portal.monitoring

BackgroundTaskLocalService

Old: com.liferay.portal.service New:
com.liferay.portal.background.task.service

com.liferay.portal.background.task.api

BackgroundTaskLocalServiceUtil

Old: com.liferay.portal.service New:
com.liferay.portal.background.task.service

com.liferay.portal.background.task.api

BackgroundTaskLocalServiceWrapper

Old: com.liferay.portal.service New:
com.liferay.portal.background.task.service

com.liferay.portal.background.task.api

BackgroundTaskModel

Old: com.liferay.portal.model New:
com.liferay.portal.background.task.model

com.liferay.portal.background.task.api

BackgroundTaskPersistence

Old: com.liferay.portal.service.persistence New:
com.liferay.portal.background.task.service.persistence

com.liferay.portal.background.task.api

BackgroundTaskService

Old: com.liferay.portal.service New:
com.liferay.portal.background.task.service

com.liferay.portal.background.task.api

BackgroundTaskServiceUtil

Old: com.liferay.portal.service New:
com.liferay.portal.background.task.service

com.liferay.portal.background.task.api

BackgroundTaskServiceWrapper

Old: com.liferay.portal.service New:
com.liferay.portal.background.task.service

com.liferay.portal.background.task.api

BackgroundTaskSoap

Old: com.liferay.portal.model New:
com.liferay.portal.background.task.model

com.liferay.portal.background.task.api

BackgroundTaskUtil

Old: com.liferay.portal.service.persistence New:
com.liferay.portal.background.task.service.persistence

com.liferay.portal.background.task.api

BackgroundTaskWrapper

Old: com.liferay.portal.model New:
com.liferay.portal.background.task.model

com.liferay.portal.background.task.api

BaseCmisRepository

Old: com.liferay.portal.kernel.repository.cmis New:
com.liferay.document.library.repository.cmis

com.liferay.document.library.repository.cmis

BaseCmisSearchQueryBuilder

Old: com.liferay.portal.kernel.repository.cmis.search New:
com.liferay.document.library.repository.cmis.search

com.liferay.document.library.repository.cmis

BaseDDLExporter

Old: com.liferay.portlet.dynamicdatalists.util New:
com.liferay.dynamic.data.lists.exporter.impl

com.liferay.dynamic.data.lists.service

BaseDDMDisplay

Old: com.liferay.portlet.dynamicdatamapping.util New:
com.liferay.dynamic.data.mapping.util

com.liferay.dynamic.data.mapping.api

BaseFieldRenderer

Old: com.liferay.portlet.dynamicdatamapping.storage New:
com.liferay.dynamic.data.mapping.storage

com.liferay.dynamic.data.mapping.api

BaseScriptingExecutor

Old: com.liferay.portal.kernel.scripting New:
com.liferay.portal.scripting

com.liferay.portal.scripting

BaseStatistics

Old: com.liferay.portal.kernel.monitoring.statistics New:
com.liferay.portal.monitoring.internal.statistics

com.liferay.portal.monitoring

BaseStorageAdapter

Old: com.liferay.portlet.dynamicdatamapping.storage New:
com.liferay.dynamic.data.mapping.storage

com.liferay.dynamic.data.mapping.api

BillingCityException

Old: com.liferay.portlet.shopping New: com.liferay.shopping.exception

com.liferay.shopping.api

BillingCountryException

Old: com.liferay.portlet.shopping New: com.liferay.shopping.exception

com.liferay.shopping.api

BillingEmailAddressException

Old: com.liferay.portlet.shopping New: com.liferay.shopping.exception

com.liferay.shopping.api

BillingFirstNameException

Old: com.liferay.portlet.shopping New: com.liferay.shopping.exception

com.liferay.shopping.api

BillingLastNameException

Old: com.liferay.portlet.shopping New: com.liferay.shopping.exception

com.liferay.shopping.api

BillingPhoneException

Old: com.liferay.portlet.shopping New: com.liferay.shopping.exception

com.liferay.shopping.api

BillingStateException

Old: com.liferay.portlet.shopping New: com.liferay.shopping.exception

com.liferay.shopping.api

BillingStreetException

Old: com.liferay.portlet.shopping New: com.liferay.shopping.exception

com.liferay.shopping.api

BillingZipException

Old: com.liferay.portlet.shopping New: com.liferay.shopping.exception

com.liferay.shopping.api

BlockingPortalCache

Old: com.liferay.portal.kernel.cache New: com.liferay.portal.cache

com.liferay.portal.cache

BookmarksEntry

Old: com.liferay.portlet.bookmarks.model New:
com.liferay.bookmarks.model

com.liferay.bookmarks.api

BookmarksEntryFinder

Old: com.liferay.portlet.bookmarks.service.persistence New:
com.liferay.bookmarks.service.persistence

com.liferay.bookmarks.api

BookmarksEntryLocalService

Old: com.liferay.portlet.bookmarks.service New:
com.liferay.bookmarks.service

com.liferay.bookmarks.api

BookmarksEntryLocalServiceUtil

Old: com.liferay.portlet.bookmarks.service New:
com.liferay.bookmarks.service

com.liferay.bookmarks.api

BookmarksEntryLocalServiceWrapper

Old: com.liferay.portlet.bookmarks.service New:
com.liferay.bookmarks.service

com.liferay.bookmarks.api

BookmarksEntryModel

Old: com.liferay.portlet.bookmarks.model New:
com.liferay.bookmarks.model

com.liferay.bookmarks.api

BookmarksEntryPersistence

Old: com.liferay.portlet.bookmarks.service.persistence New:
com.liferay.bookmarks.service.persistence

com.liferay.bookmarks.api

BookmarksEntryService

Old: com.liferay.portlet.bookmarks.service New:
com.liferay.bookmarks.service

com.liferay.bookmarks.api

BookmarksEntryServiceUtil

Old: com.liferay.portlet.bookmarks.service New:
com.liferay.bookmarks.service

com.liferay.bookmarks.api

BookmarksEntryServiceWrapper

Old: com.liferay.portlet.bookmarks.service New:
com.liferay.bookmarks.service

com.liferay.bookmarks.api

BookmarksEntrySoap

Old: com.liferay.portlet.bookmarks.model New:
com.liferay.bookmarks.model

com.liferay.bookmarks.api

BookmarksEntryUtil

Old: com.liferay.portlet.bookmarks.service.persistence New:
com.liferay.bookmarks.service.persistence

com.liferay.bookmarks.api

BookmarksEntryWrapper

Old: com.liferay.portlet.bookmarks.model New:
com.liferay.bookmarks.model

com.liferay.bookmarks.api

BookmarksFolder

Old: com.liferay.portlet.bookmarks.model New:
com.liferay.bookmarks.model

com.liferay.bookmarks.api

BookmarksFolderConstants

Old: com.liferay.portlet.bookmarks.model New:
com.liferay.bookmarks.model

com.liferay.bookmarks.api

BookmarksFolderFinder

Old: com.liferay.portlet.bookmarks.service.persistence New:
com.liferay.bookmarks.service.persistence

com.liferay.bookmarks.api

BookmarksFolderLocalService

Old: com.liferay.portlet.bookmarks.service New:
com.liferay.bookmarks.service

com.liferay.bookmarks.api

BookmarksFolderLocalServiceUtil

Old: com.liferay.portlet.bookmarks.service New:
com.liferay.bookmarks.service

com.liferay.bookmarks.api

BookmarksFolderLocalServiceWrapper

Old: com.liferay.portlet.bookmarks.service New:
com.liferay.bookmarks.service

com.liferay.bookmarks.api

BookmarksFolderModel

Old: com.liferay.portlet.bookmarks.model New:
com.liferay.bookmarks.model

com.liferay.bookmarks.api

BookmarksFolderPersistence

Old: com.liferay.portlet.bookmarks.service.persistence New:
com.liferay.bookmarks.service.persistence

com.liferay.bookmarks.api

BookmarksFolderService

Old: com.liferay.portlet.bookmarks.service New:
com.liferay.bookmarks.service

com.liferay.bookmarks.api

BookmarksFolderServiceUtil

Old: com.liferay.portlet.bookmarks.service New:
com.liferay.bookmarks.service

com.liferay.bookmarks.api

BookmarksFolderServiceWrapper

Old: com.liferay.portlet.bookmarks.service New:
com.liferay.bookmarks.service

com.liferay.bookmarks.api

BookmarksFolderSoap

Old: com.liferay.portlet.bookmarks.model New:
com.liferay.bookmarks.model

com.liferay.bookmarks.api

BookmarksFolderUtil

Old: com.liferay.portlet.bookmarks.service.persistence New:
com.liferay.bookmarks.service.persistence

com.liferay.bookmarks.api

BookmarksFolderWrapper

Old: com.liferay.portlet.bookmarks.model New:
com.liferay.bookmarks.model

com.liferay.bookmarks.api

ByteArrayReportResultContainer

Old: com.liferay.portal.kernel.bi.reporting New:
com.liferay.portal.reports.engine

com.liferay.portal.reports.engine.api

CCExpirationException

Old: com.liferay.portlet.shopping New: com.liferay.shopping.exception

com.liferay.shopping.api

CCNameException

Old: com.liferay.portlet.shopping New: com.liferay.shopping.exception

com.liferay.shopping.api

CCNumberException

Old: com.liferay.portlet.shopping New: com.liferay.shopping.exception

com.liferay.shopping.api

CCTypeException

Old: com.liferay.portlet.shopping New: com.liferay.shopping.exception

com.liferay.shopping.api

CMISBetweenExpression

Old: com.liferay.portal.kernel.repository.cmis.search New:
com.liferay.document.library.repository.cmis.search

com.liferay.document.library.repository.cmis

CMISConjunction

Old: com.liferay.portal.kernel.repository.cmis.search New:
com.liferay.document.library.repository.cmis.search

com.liferay.document.library.repository.cmis

CMISContainsExpression

Old: com.liferay.portal.kernel.repository.cmis.search New:
com.liferay.document.library.repository.cmis.search

com.liferay.document.library.repository.cmis

CMISContainsNotExpression

Old: com.liferay.portal.kernel.repository.cmis.search New:
com.liferay.document.library.repository.cmis.search

com.liferay.document.library.repository.cmis

CMISContainsValueExpression

Old: com.liferay.portal.kernel.repository.cmis.search New:
com.liferay.document.library.repository.cmis.search

com.liferay.document.library.repository.cmis

CMISCriterion

Old: com.liferay.portal.kernel.repository.cmis.search New:
com.liferay.document.library.repository.cmis.search

com.liferay.document.library.repository.cmis

CMISDisjunction

Old: com.liferay.portal.kernel.repository.cmis.search New:
com.liferay.document.library.repository.cmis.search

com.liferay.document.library.repository.cmis

CMISFullTextConjunction

Old: com.liferay.portal.kernel.repository.cmis.search New:
com.liferay.document.library.repository.cmis.search

com.liferay.document.library.repository.cmis

CMISInFolderExpression

Old: com.liferay.portal.kernel.repository.cmis.search New:
com.liferay.document.library.repository.cmis.search

com.liferay.document.library.repository.cmis

CMISInTreeExpression

Old: com.liferay.portal.kernel.repository.cmis.search New:
com.liferay.document.library.repository.cmis.search

com.liferay.document.library.repository.cmis

CMISJunction

Old: com.liferay.portal.kernel.repository.cmis.search New:
com.liferay.document.library.repository.cmis.search

com.liferay.document.library.repository.cmis

CMISNotExpression

Old: com.liferay.portal.kernel.repository.cmis.search New:
com.liferay.document.library.repository.cmis.search

com.liferay.document.library.repository.cmis

CMISParameterValueUtil

Old: com.liferay.portal.kernel.repository.cmis.search New:
com.liferay.document.library.repository.cmis.search

com.liferay.document.library.repository.cmis

CMISRepositoryHandler

Old: com.liferay.portal.kernel.repository.cmis New:
com.liferay.document.library.repository.cmis

com.liferay.document.library.repository.cmis

CMISRepositoryUtil

Old: com.liferay.portal.kernel.repository.cmis New:
com.liferay.document.library.repository.cmis.internal

com.liferay.document.library.repository.cmis.impl

CMISSearchQueryBuilder

Old: com.liferay.portal.kernel.repository.cmis.search New:
com.liferay.document.library.repository.cmis.search

com.liferay.document.library.repository.cmis

CMISSimpleExpression

Old: com.liferay.portal.kernel.repository.cmis.search New:
com.liferay.document.library.repository.cmis.search

com.liferay.document.library.repository.cmis

CMISSimpleExpressionOperator

Old: com.liferay.portal.kernel.repository.cmis.search New:
com.liferay.document.library.repository.cmis.search

com.liferay.document.library.repository.cmis

CartMinOrderException

Old: com.liferay.portlet.shopping New: com.liferay.shopping.exception

com.liferay.shopping.api

CartMinQuantityException

Old: com.liferay.portlet.shopping New: com.liferay.shopping.exception

com.liferay.shopping.api

CollatedSpellCheckHitsProcessor

Old: com.liferay.portal.kernel.search New:
com.liferay.portal.search.internal.hits

com.liferay.portal.search

CompoundSessionIdServletRequest

Old: com.liferay.portal.kernel.servlet.filters.compoundsessionid New:
com.liferay.portal.compound.session.id

com.liferay.portal.compound.session.id

Condition

Old: com.liferay.portlet.dynamicdatamapping.storage.query New:
com.liferay.portal.workflow.kaleo.definition

com.liferay.portal.workflow.kaleo.definition.api

ContactConverterKeys

Old: com.liferay.portal.security.ldap New:
com.liferay.portal.security.ldap

com.liferay.portal.security.ldap

ContentException

Old: com.liferay.portlet.dynamicdatamapping New:
com.liferay.dynamic.data.mapping.exception

com.liferay.dynamic.data.mapping.api

ContentNameException

Old: com.liferay.portlet.dynamicdatamapping New:
com.liferay.dynamic.data.mapping.exception

com.liferay.dynamic.data.mapping.api

ContextClassloaderReportDesignRetriever

Old: com.liferay.portal.kernel.bi.reporting New:
com.liferay.portal.reports.engine

com.liferay.portal.reports.engine.api

CountStatistics

Old: com.liferay.portal.kernel.monitoring.statistics New:
com.liferay.portal.monitoring.internal.statistics

com.liferay.portal.monitoring

CouponActiveException

Old: com.liferay.portlet.shopping New: com.liferay.shopping.exception

com.liferay.shopping.api

CouponCodeException

Old: com.liferay.portlet.shopping New: com.liferay.shopping.exception

com.liferay.shopping.api

CouponDateException

Old: com.liferay.portlet.shopping New: com.liferay.shopping.exception

com.liferay.shopping.api

CouponDescriptionException

Old: com.liferay.portlet.shopping New: com.liferay.shopping.exception

com.liferay.shopping.api

CouponDiscountException

Old: com.liferay.portlet.shopping New: com.liferay.shopping.exception

com.liferay.shopping.api

CouponEndDateException

Old: com.liferay.portlet.shopping New: com.liferay.shopping.exception

com.liferay.shopping.api

CouponLimitCategoriesException

Old: com.liferay.portlet.shopping New: com.liferay.shopping.exception

com.liferay.shopping.api

CouponLimitSKUsException

Old: com.liferay.portlet.shopping New: com.liferay.shopping.exception

com.liferay.shopping.api

CouponMinimumOrderException

Old: com.liferay.portlet.shopping New: com.liferay.shopping.exception

com.liferay.shopping.api

CouponNameException

Old: com.liferay.portlet.shopping New: com.liferay.shopping.exception

com.liferay.shopping.api

CouponStartDateException

Old: com.liferay.portlet.shopping New: com.liferay.shopping.exception

com.liferay.shopping.api

DDL

Old: com.liferay.portlet.dynamicdatalists.util New:
com.liferay.dynamic.data.lists.util

com.liferay.dynamic.data.lists.api

DDLExporter

Old: com.liferay.portlet.dynamicdatalists.util New:
com.liferay.dynamic.data.lists.exporter

com.liferay.dynamic.data.lists.api

DDLExporterFactory

Old: com.liferay.portlet.dynamicdatalists.util New:
com.liferay.dynamic.data.lists.exporter

com.liferay.dynamic.data.lists.api

DDLRecord

Old: com.liferay.portlet.dynamicdatalists.model New:
com.liferay.dynamic.data.lists.model

com.liferay.dynamic.data.lists.api

DDLRecordConstants

Old: com.liferay.portlet.dynamicdatalists.model New:
com.liferay.dynamic.data.lists.model

com.liferay.dynamic.data.lists.api

DDLRecordFinder

Old: com.liferay.portlet.dynamicdatalists.service.persistence New:
com.liferay.dynamic.data.lists.service.persistence

com.liferay.dynamic.data.lists.api

DDLRecordLocalService

Old: com.liferay.portlet.dynamicdatalists.service New:
com.liferay.dynamic.data.lists.service

com.liferay.dynamic.data.lists.api

DDLRecordLocalServiceUtil

Old: com.liferay.portlet.dynamicdatalists.service New:
com.liferay.dynamic.data.lists.service

com.liferay.dynamic.data.lists.api

DDLRecordLocalServiceWrapper

Old: com.liferay.portlet.dynamicdatalists.service New:
com.liferay.dynamic.data.lists.service

com.liferay.dynamic.data.lists.api

DDLRecordModel

Old: com.liferay.portlet.dynamicdatalists.model New:
com.liferay.dynamic.data.lists.model

com.liferay.dynamic.data.lists.api

DDLRecordPersistence

Old: com.liferay.portlet.dynamicdatalists.service.persistence New:
com.liferay.dynamic.data.lists.service.persistence

com.liferay.dynamic.data.lists.api

DDLRecordService

Old: com.liferay.portlet.dynamicdatalists.service New:
com.liferay.dynamic.data.lists.service

com.liferay.dynamic.data.lists.api

DDLRecordServiceUtil

Old: com.liferay.portlet.dynamicdatalists.service New:
com.liferay.dynamic.data.lists.service

com.liferay.dynamic.data.lists.api

DDLRecordServiceWrapper

Old: com.liferay.portlet.dynamicdatalists.service New:
com.liferay.dynamic.data.lists.service

com.liferay.dynamic.data.lists.api

DDLRecordSet

Old: com.liferay.portlet.dynamicdatalists.model New:
com.liferay.dynamic.data.lists.model

com.liferay.dynamic.data.lists.api

DDLRecordSetConstants

Old: com.liferay.portlet.dynamicdatalists.model New:
com.liferay.dynamic.data.lists.model

com.liferay.dynamic.data.lists.api

DDLRecordSetFinder

Old: com.liferay.portlet.dynamicdatalists.service.persistence New:
com.liferay.dynamic.data.lists.service.persistence

com.liferay.dynamic.data.lists.api

DDLRecordSetLocalService

Old: com.liferay.portlet.dynamicdatalists.service New:
com.liferay.dynamic.data.lists.service

com.liferay.dynamic.data.lists.api

DDLRecordSetLocalServiceUtil

Old: com.liferay.portlet.dynamicdatalists.service New:
com.liferay.dynamic.data.lists.service

com.liferay.dynamic.data.lists.api

DDLRecordSetLocalServiceWrapper

Old: com.liferay.portlet.dynamicdatalists.service New:
com.liferay.dynamic.data.lists.service

com.liferay.dynamic.data.lists.api

DDLRecordSetModel

Old: com.liferay.portlet.dynamicdatalists.model New:
com.liferay.dynamic.data.lists.model

com.liferay.dynamic.data.lists.api

DDLRecordSetPersistence

Old: com.liferay.portlet.dynamicdatalists.service.persistence New:
com.liferay.dynamic.data.lists.service.persistence

com.liferay.dynamic.data.lists.api

DDLRecordSetService

Old: com.liferay.portlet.dynamicdatalists.service New:
com.liferay.dynamic.data.lists.service

com.liferay.dynamic.data.lists.api

DDLRecordSetServiceUtil

Old: com.liferay.portlet.dynamicdatalists.service New:
com.liferay.dynamic.data.lists.service

com.liferay.dynamic.data.lists.api

DDLRecordSetServiceWrapper

Old: com.liferay.portlet.dynamicdatalists.service New:
com.liferay.dynamic.data.lists.service

com.liferay.dynamic.data.lists.api

DDLRecordSetSoap

Old: com.liferay.portlet.dynamicdatalists.model New:
com.liferay.dynamic.data.lists.model

com.liferay.dynamic.data.lists.api

DDLRecordSetUtil

Old: com.liferay.portlet.dynamicdatalists.service.persistence New:
com.liferay.dynamic.data.lists.service.persistence

com.liferay.dynamic.data.lists.api

DDLRecordSetWrapper

Old: com.liferay.portlet.dynamicdatalists.model New:
com.liferay.dynamic.data.lists.model

com.liferay.dynamic.data.lists.api

DDLRecordSoap

Old: com.liferay.portlet.dynamicdatalists.model New:
com.liferay.dynamic.data.lists.model

com.liferay.dynamic.data.lists.api

DDLRecordUtil

Old: com.liferay.portlet.dynamicdatalists.service.persistence New:
com.liferay.dynamic.data.lists.service.persistence

com.liferay.dynamic.data.lists.api

DDLRecordVersion

Old: com.liferay.portlet.dynamicdatalists.model New:
com.liferay.dynamic.data.lists.model

com.liferay.dynamic.data.lists.api

DDLRecordVersionModel

Old: com.liferay.portlet.dynamicdatalists.model New:
com.liferay.dynamic.data.lists.model

com.liferay.dynamic.data.lists.api

DDLRecordVersionPersistence

Old: com.liferay.portlet.dynamicdatalists.service.persistence New:
com.liferay.dynamic.data.lists.service.persistence

com.liferay.dynamic.data.lists.api

DDLRecordVersionSoap

Old: com.liferay.portlet.dynamicdatalists.model New:
com.liferay.dynamic.data.lists.model

com.liferay.dynamic.data.lists.api

DDLRecordVersionUtil

Old: com.liferay.portlet.dynamicdatalists.service.persistence New:
com.liferay.dynamic.data.lists.service.persistence

com.liferay.dynamic.data.lists.api

DDLRecordVersionVersionComparator

Old: com.liferay.portlet.dynamicdatalists.util.comparator New:
com.liferay.dynamic.data.lists.util.comparator

com.liferay.dynamic.data.lists.api

DDLRecordVersionWrapper

Old: com.liferay.portlet.dynamicdatalists.model New:
com.liferay.dynamic.data.lists.model

com.liferay.dynamic.data.lists.api

DDLRecordWrapper

Old: com.liferay.portlet.dynamicdatalists.model New:
com.liferay.dynamic.data.lists.model

com.liferay.dynamic.data.lists.api

DDM

Old: com.liferay.portlet.dynamicdatamapping.util New:
com.liferay.dynamic.data.mapping.util

com.liferay.dynamic.data.mapping.api

DDMContent

Old: com.liferay.portlet.dynamicdatamapping.model New:
com.liferay.dynamic.data.mapping.model

com.liferay.dynamic.data.mapping.api

DDMContentLocalService

Old: com.liferay.portlet.dynamicdatamapping.service New:
com.liferay.dynamic.data.mapping.service

com.liferay.dynamic.data.mapping.api

DDMContentLocalServiceUtil

Old: com.liferay.portlet.dynamicdatamapping.service New:
com.liferay.dynamic.data.mapping.service

com.liferay.dynamic.data.mapping.api

DDMContentLocalServiceWrapper

Old: com.liferay.portlet.dynamicdatamapping.service New:
com.liferay.dynamic.data.mapping.service

com.liferay.dynamic.data.mapping.api

DDMContentModel

Old: com.liferay.portlet.dynamicdatamapping.model New:
com.liferay.dynamic.data.mapping.model

com.liferay.dynamic.data.mapping.api

DDMContentPersistence

Old: com.liferay.portlet.dynamicdatamapping.service.persistence New:
com.liferay.dynamic.data.mapping.service.persistence

com.liferay.dynamic.data.mapping.api

DDMContentSoap

Old: com.liferay.portlet.dynamicdatamapping.model New:
com.liferay.dynamic.data.mapping.model

com.liferay.dynamic.data.mapping.api

DDMContentUtil

Old: com.liferay.portlet.dynamicdatamapping.service.persistence New:
com.liferay.dynamic.data.mapping.service.persistence

com.liferay.dynamic.data.mapping.api

DDMContentWrapper

Old: com.liferay.portlet.dynamicdatamapping.model New:
com.liferay.dynamic.data.mapping.model

com.liferay.dynamic.data.mapping.api

DDMDisplay

Old: com.liferay.portlet.dynamicdatamapping.util New:
com.liferay.dynamic.data.mapping.util

com.liferay.dynamic.data.mapping.api

DDMDisplayRegistry

Old: com.liferay.portlet.dynamicdatamapping.util New:
com.liferay.dynamic.data.mapping.util

com.liferay.dynamic.data.mapping.api

DDMIndexer

Old: com.liferay.portlet.dynamicdatamapping.util New:
com.liferay.dynamic.data.mapping.util

com.liferay.dynamic.data.mapping.api

DDMStorageLink

Old: com.liferay.portlet.dynamicdatamapping.model New:
com.liferay.dynamic.data.mapping.model

com.liferay.dynamic.data.mapping.api

DDMStorageLinkLocalService

Old: com.liferay.portlet.dynamicdatamapping.service New:
com.liferay.dynamic.data.mapping.service

com.liferay.dynamic.data.mapping.api

DDMStorageLinkLocalServiceUtil

Old: com.liferay.portlet.dynamicdatamapping.service New:
com.liferay.dynamic.data.mapping.service

com.liferay.dynamic.data.mapping.api

DDMStorageLinkLocalServiceWrapper

Old: com.liferay.portlet.dynamicdatamapping.service New:
com.liferay.dynamic.data.mapping.service

com.liferay.dynamic.data.mapping.api

DDMStorageLinkModel

Old: com.liferay.portlet.dynamicdatamapping.model New:
com.liferay.dynamic.data.mapping.model

com.liferay.dynamic.data.mapping.api

DDMStorageLinkPersistence

Old: com.liferay.portlet.dynamicdatamapping.service.persistence New:
com.liferay.dynamic.data.mapping.service.persistence

com.liferay.dynamic.data.mapping.api

DDMStorageLinkSoap

Old: com.liferay.portlet.dynamicdatamapping.model New:
com.liferay.dynamic.data.mapping.model

com.liferay.dynamic.data.mapping.api

DDMStorageLinkUtil

Old: com.liferay.portlet.dynamicdatamapping.service.persistence New:
com.liferay.dynamic.data.mapping.service.persistence

com.liferay.dynamic.data.mapping.api

DDMStorageLinkWrapper

Old: com.liferay.portlet.dynamicdatamapping.model New:
com.liferay.dynamic.data.mapping.model

com.liferay.dynamic.data.mapping.api

DDMStructureConstants

Old: com.liferay.portlet.dynamicdatamapping.model New:
com.liferay.dynamic.data.mapping.model

com.liferay.dynamic.data.mapping.api

DDMStructureFinder

Old: com.liferay.portlet.dynamicdatamapping.service.persistence New:
com.liferay.dynamic.data.mapping.service.persistence

com.liferay.dynamic.data.mapping.api

DDMStructureLinkLocalService

Old: com.liferay.portlet.dynamicdatamapping.service New:
com.liferay.dynamic.data.mapping.service

com.liferay.dynamic.data.mapping.api

DDMStructureLinkLocalServiceUtil

Old: com.liferay.portlet.dynamicdatamapping.service New:
com.liferay.dynamic.data.mapping.service

com.liferay.dynamic.data.mapping.api

DDMStructureLinkLocalServiceWrapper

Old: com.liferay.portlet.dynamicdatamapping.service New:
com.liferay.dynamic.data.mapping.service

com.liferay.dynamic.data.mapping.api

DDMStructureLinkModel

Old: com.liferay.portlet.dynamicdatamapping.model New:
com.liferay.dynamic.data.mapping.model

com.liferay.dynamic.data.mapping.api

DDMStructureLinkPersistence

Old: com.liferay.portlet.dynamicdatamapping.service.persistence New:
com.liferay.dynamic.data.mapping.service.persistence

com.liferay.dynamic.data.mapping.api

DDMStructureLinkSoap

Old: com.liferay.portlet.dynamicdatamapping.model New:
com.liferay.dynamic.data.mapping.model

com.liferay.dynamic.data.mapping.api

DDMStructureLinkUtil

Old: com.liferay.portlet.dynamicdatamapping.service.persistence New:
com.liferay.dynamic.data.mapping.service.persistence

com.liferay.dynamic.data.mapping.api

DDMStructureLinkWrapper

Old: com.liferay.portlet.dynamicdatamapping.model New:
com.liferay.dynamic.data.mapping.model

com.liferay.dynamic.data.mapping.api

DDMStructureLocalService

Old: com.liferay.portlet.dynamicdatamapping.service New:
com.liferay.dynamic.data.mapping.service

com.liferay.dynamic.data.mapping.api

DDMStructureLocalServiceUtil

Old: com.liferay.portlet.dynamicdatamapping.service New:
com.liferay.dynamic.data.mapping.service

com.liferay.dynamic.data.mapping.api

DDMStructureLocalServiceWrapper

Old: com.liferay.portlet.dynamicdatamapping.service New:
com.liferay.dynamic.data.mapping.service

com.liferay.dynamic.data.mapping.api

DDMStructureModel

Old: com.liferay.portlet.dynamicdatamapping.model New:
com.liferay.dynamic.data.mapping.model

com.liferay.dynamic.data.mapping.api

DDMStructurePersistence

Old: com.liferay.portlet.dynamicdatamapping.service.persistence New:
com.liferay.dynamic.data.mapping.service.persistence

com.liferay.dynamic.data.mapping.api

DDMStructureService

Old: com.liferay.portlet.dynamicdatamapping.service New:
com.liferay.dynamic.data.mapping.service

com.liferay.dynamic.data.mapping.api

DDMStructureServiceUtil

Old: com.liferay.portlet.dynamicdatamapping.service New:
com.liferay.dynamic.data.mapping.service

com.liferay.dynamic.data.mapping.api

DDMStructureServiceWrapper

Old: com.liferay.portlet.dynamicdatamapping.service New:
com.liferay.dynamic.data.mapping.service

com.liferay.dynamic.data.mapping.api

DDMStructureSoap

Old: com.liferay.portlet.dynamicdatamapping.model New:
com.liferay.dynamic.data.mapping.model

com.liferay.dynamic.data.mapping.api

DDMStructureUtil

Old: com.liferay.portlet.dynamicdatamapping.service.persistence New:
com.liferay.dynamic.data.mapping.service.persistence

com.liferay.dynamic.data.mapping.api

DDMStructureWrapper

Old: com.liferay.portlet.dynamicdatamapping.model New:
com.liferay.dynamic.data.mapping.model

com.liferay.dynamic.data.mapping.api

DDMTemplateConstants

Old: com.liferay.portlet.dynamicdatamapping.model New:
com.liferay.dynamic.data.mapping.model

com.liferay.dynamic.data.mapping.api

DDMTemplateFinder

Old: com.liferay.portlet.dynamicdatamapping.service.persistence New:
com.liferay.dynamic.data.mapping.service.persistence

com.liferay.dynamic.data.mapping.api

DDMTemplateHelper

Old: com.liferay.portlet.dynamicdatamapping.util New:
com.liferay.dynamic.data.mapping.util

com.liferay.dynamic.data.mapping.api

DDMTemplateLocalService

Old: com.liferay.portlet.dynamicdatamapping.service New:
com.liferay.dynamic.data.mapping.service

com.liferay.dynamic.data.mapping.api

DDMTemplateLocalServiceUtil

Old: com.liferay.portlet.dynamicdatamapping.service New:
com.liferay.dynamic.data.mapping.service

com.liferay.dynamic.data.mapping.api

DDMTemplateLocalServiceWrapper

Old: com.liferay.portlet.dynamicdatamapping.service New:
com.liferay.dynamic.data.mapping.service

com.liferay.dynamic.data.mapping.api

DDMTemplateModel

Old: com.liferay.portlet.dynamicdatamapping.model New:
com.liferay.dynamic.data.mapping.model

com.liferay.dynamic.data.mapping.api

DDMTemplatePersistence

Old: com.liferay.portlet.dynamicdatamapping.service.persistence New:
com.liferay.dynamic.data.mapping.service.persistence

com.liferay.dynamic.data.mapping.api

DDMTemplateService

Old: com.liferay.portlet.dynamicdatamapping.service New:
com.liferay.dynamic.data.mapping.service

com.liferay.dynamic.data.mapping.api

DDMTemplateServiceUtil

Old: com.liferay.portlet.dynamicdatamapping.service New:
com.liferay.dynamic.data.mapping.service

com.liferay.dynamic.data.mapping.api

DDMTemplateServiceWrapper

Old: com.liferay.portlet.dynamicdatamapping.service New:
com.liferay.dynamic.data.mapping.service

com.liferay.dynamic.data.mapping.api

DDMTemplateSoap

Old: com.liferay.portlet.dynamicdatamapping.model New:
com.liferay.dynamic.data.mapping.model

com.liferay.dynamic.data.mapping.api

DDMTemplateUtil

Old: com.liferay.portlet.dynamicdatamapping.service.persistence New:
com.liferay.dynamic.data.mapping.service.persistence

com.liferay.dynamic.data.mapping.api

DDMTemplateWrapper

Old: com.liferay.portlet.dynamicdatamapping.model New:
com.liferay.dynamic.data.mapping.model

com.liferay.dynamic.data.mapping.api

DDMUtil

Old: com.liferay.portlet.dynamicdatamapping.util New:
com.liferay.dynamic.data.mapping.util

com.liferay.dynamic.data.mapping.api

DDMXML

Old: com.liferay.portlet.dynamicdatamapping.util New:
com.liferay.dynamic.data.mapping.util

com.liferay.dynamic.data.mapping.api

DefaultAttributesTransformer

Old: com.liferay.portal.security.ldap New:
com.liferay.portal.security.ldap.internal

com.liferay.portal.security.ldap

DefaultMessageBus

Old: com.liferay.portal.kernel.messaging New:
com.liferay.portal.messaging.internal

com.liferay.portal.messaging

DefaultSingleDestinationMessageSender

Old: com.liferay.portal.kernel.messaging.sender New:
com.liferay.portal.messaging.internal.sender

com.liferay.portal.messaging

DefaultSingleDestinationSynchronousMessageSender

Old: com.liferay.portal.kernel.messaging.sender New:
com.liferay.portal.messaging.internal.sender

com.liferay.portal.messaging

DefaultSynchronousMessageSender

Old: com.liferay.portal.kernel.messaging.sender New:
com.liferay.portal.messaging.internal.sender

com.liferay.portal.messaging

DestinationStatisticsManager

Old: com.liferay.portal.kernel.messaging.jmx New:
com.liferay.portal.messaging.internal.jmx

com.liferay.portal.messaging

DestinationStatisticsManagerMBean

Old: com.liferay.portal.kernel.messaging.jmx New:
com.liferay.portal.messaging.internal.jmx

com.liferay.portal.messaging

DirectSynchronousMessageSender

Old: com.liferay.portal.kernel.messaging.sender New:
com.liferay.portal.messaging.internal.sender

com.liferay.portal.messaging

DummyContext

Old: com.liferay.portal.kernel.ldap New:
com.liferay.portal.security.ldap.dummy

com.liferay.portal.security.ldap

DummyDirContext

Old: com.liferay.portal.kernel.ldap New:
com.liferay.portal.security.ldap.dummy

com.liferay.portal.security.ldap

DuplicateArticleIdException

Old: com.liferay.portlet.journal New: com.liferay.journal.exception

com.liferay.journal.api

DuplicateArticleImageIdException

Old: com.liferay.portlet.journal New: com.liferay.journal.exception

com.liferay.journal.api

DuplicateCouponCodeException

Old: com.liferay.portlet.shopping New: com.liferay.shopping.exception

com.liferay.shopping.api

DuplicateFeedIdException

Old: com.liferay.portlet.journal New: com.liferay.journal.exception

com.liferay.journal.api

DuplicateItemSKUException

Old: com.liferay.portlet.shopping New: com.liferay.shopping.exception

com.liferay.shopping.api

DuplicateLDAPServerNameException

Old: com.liferay.portal.kernel.ldap New:
com.liferay.portal.security.ldap

com.liferay.portal.security.ldap

DuplicateNodeNameException

Old: com.liferay.portlet.wiki New: com.liferay.wiki.exception

com.liferay.wiki.api

DuplicatePageException

Old: com.liferay.portlet.wiki New: com.liferay.wiki.exception

com.liferay.wiki.api

DuplicateRuleGroupInstanceException

Old: com.liferay.portlet.mobiledevicerules New:
com.liferay.mobile.device.rules.exception

com.liferay.mobile.device.rules.api

DuplicateVoteException

Old: com.liferay.portlet.polls New: com.liferay.polls.exception

com.liferay.polls.api

EntryNameComparator

Old: com.liferay.portlet.bookmarks.util.comparator New:
com.liferay.bookmarks.util.comparator

com.liferay.bookmarks.api

EntryPriorityComparator

Old: com.liferay.portlet.bookmarks.util.comparator New:
com.liferay.bookmarks.util.comparator

com.liferay.bookmarks.api

EntryURLComparator

Old: com.liferay.portlet.bookmarks.util.comparator New:
com.liferay.bookmarks.util.comparator

com.liferay.bookmarks.api

EntryVisitsComparator

Old: com.liferay.portlet.bookmarks.util.comparator New:
com.liferay.bookmarks.util.comparator

com.liferay.bookmarks.api

Fact

Old: com.liferay.portal.kernel.bi.rules New:
com.liferay.portal.rules.engine

com.liferay.portal.rules.engine.api

FeedContentFieldException

Old: com.liferay.portlet.journal New: com.liferay.journal.exception

com.liferay.journal.api

FeedIdException

Old: com.liferay.portlet.journal New: com.liferay.journal.exception

com.liferay.journal.api

FeedNameException

Old: com.liferay.portlet.journal New: com.liferay.journal.exception

com.liferay.journal.api

FeedTargetLayoutFriendlyUrlException

Old: com.liferay.portlet.journal New: com.liferay.journal.exception

com.liferay.journal.api

FeedTargetPortletIdException

Old: com.liferay.portlet.journal New: com.liferay.journal.exception

com.liferay.journal.api

FieldConstants

Old: com.liferay.portlet.dynamicdatamapping.storage New:
com.liferay.dynamic.data.mapping.storage

com.liferay.dynamic.data.mapping.api

FieldRenderer

Old: com.liferay.portlet.dynamicdatamapping.storage New:
com.liferay.dynamic.data.mapping.storage

com.liferay.dynamic.data.mapping.api

FieldRendererFactory

Old: com.liferay.portlet.dynamicdatamapping.storage New:
com.liferay.dynamic.data.mapping.storage

com.liferay.dynamic.data.mapping.api

Fields

Old: com.liferay.portlet.dynamicdatamapping.storage New:
com.liferay.dynamic.data.mapping.storage

com.liferay.dynamic.data.mapping.api

FlagsEntryService

Old: com.liferay.portlet.flags.service New: com.liferay.flags.service

com.liferay.flags.api

FlagsEntryServiceUtil

Old: com.liferay.portlet.flags.service New: com.liferay.flags.service

com.liferay.flags.api

FlagsEntryServiceWrapper

Old: com.liferay.portlet.flags.service New: com.liferay.flags.service

com.liferay.flags.api

FlagsRequest

Old: com.liferay.portlet.flags.messaging New:
com.liferay.flags.messaging

com.liferay.flags.service

GroupConverterKeys

Old: com.liferay.portal.security.ldap New:
com.liferay.portal.security.ldap

com.liferay.portal.security.ldap

ImportFilesException

Old: com.liferay.portlet.wiki New: com.liferay.wiki.exception

com.liferay.wiki.api

ItemLargeImageNameException

Old: com.liferay.portlet.shopping New: com.liferay.shopping.exception

com.liferay.shopping.api

ItemLargeImageSizeException

Old: com.liferay.portlet.shopping New: com.liferay.shopping.exception

com.liferay.shopping.api

ItemMediumImageNameException

Old: com.liferay.portlet.shopping New: com.liferay.shopping.exception

com.liferay.shopping.api

ItemMediumImageSizeException

Old: com.liferay.portlet.shopping New: com.liferay.shopping.exception

com.liferay.shopping.api

ItemMinQuantityComparator

Old: com.liferay.portlet.shopping.util.comparator New:
com.liferay.shopping.util.comparator

com.liferay.shopping.api

ItemNameComparator

Old: com.liferay.portlet.shopping.util.comparator New:
com.liferay.shopping.util.comparator

com.liferay.shopping.api

ItemNameException

Old: com.liferay.portlet.shopping New: com.liferay.shopping.exception

com.liferay.shopping.api

ItemPriceComparator

Old: com.liferay.portlet.shopping.util.comparator New:
com.liferay.shopping.util.comparator

com.liferay.shopping.api

ItemSKUComparator

Old: com.liferay.portlet.shopping.util.comparator New:
com.liferay.shopping.util.comparator

com.liferay.shopping.api

ItemSKUException

Old: com.liferay.portlet.shopping New: com.liferay.shopping.exception

com.liferay.shopping.api

ItemSmallImageNameException

Old: com.liferay.portlet.shopping New: com.liferay.shopping.exception

com.liferay.shopping.api

ItemSmallImageSizeException

Old: com.liferay.portlet.shopping New: com.liferay.shopping.exception

com.liferay.shopping.api

JobStateSerializeUtil

Old: com.liferay.portal.kernel.scheduler New:
com.liferay.portal.scheduler

com.liferay.portal.scheduler

JournalArticle

Old: com.liferay.portlet.journal.model New: com.liferay.journal.model

com.liferay.journal.api

JournalArticleConstants

Old: com.liferay.portlet.journal.model New: com.liferay.journal.model

com.liferay.journal.api

JournalArticleDisplay

Old: com.liferay.portlet.journal.model New: com.liferay.journal.model

com.liferay.journal.api

JournalArticleFinder

Old: com.liferay.portlet.journal.service.persistence New:
com.liferay.journal.service.persistence

com.liferay.journal.api

JournalArticleImage

Old: com.liferay.portlet.journal.model New: com.liferay.journal.model

com.liferay.journal.api

JournalArticleImageLocalService

Old: com.liferay.portlet.journal.service New:
com.liferay.journal.service

com.liferay.journal.api

JournalArticleImageLocalServiceUtil

Old: com.liferay.portlet.journal.service New:
com.liferay.journal.service

com.liferay.journal.api

JournalArticleImageLocalServiceWrapper

Old: com.liferay.portlet.journal.service New:
com.liferay.journal.service

com.liferay.journal.api

JournalArticleImageModel

Old: com.liferay.portlet.journal.model New: com.liferay.journal.model

com.liferay.journal.api

JournalArticleImagePersistence

Old: com.liferay.portlet.journal.service.persistence New:
com.liferay.journal.service.persistence

com.liferay.journal.api

JournalArticleImageSoap

Old: com.liferay.portlet.journal.model New: com.liferay.journal.model

com.liferay.journal.api

JournalArticleImageUtil

Old: com.liferay.portlet.journal.service.persistence New:
com.liferay.journal.service.persistence

com.liferay.journal.api

JournalArticleImageWrapper

Old: com.liferay.portlet.journal.model New: com.liferay.journal.model

com.liferay.journal.api

JournalArticleLocalService

Old: com.liferay.portlet.journal.service New:
com.liferay.journal.service

com.liferay.journal.api

JournalArticleLocalServiceUtil

Old: com.liferay.portlet.journal.service New:
com.liferay.journal.service

com.liferay.journal.api

JournalArticleLocalServiceWrapper

Old: com.liferay.portlet.journal.service New:
com.liferay.journal.service

com.liferay.journal.api

JournalArticleModel

Old: com.liferay.portlet.journal.model New: com.liferay.journal.model

com.liferay.journal.api

JournalArticlePersistence

Old: com.liferay.portlet.journal.service.persistence New:
com.liferay.journal.service.persistence

com.liferay.journal.api

JournalArticleResource

Old: com.liferay.portlet.journal.model New: com.liferay.journal.model

com.liferay.journal.api

JournalArticleResourceLocalService

Old: com.liferay.portlet.journal.service New:
com.liferay.journal.service

com.liferay.journal.api

JournalArticleResourceLocalServiceUtil

Old: com.liferay.portlet.journal.service New:
com.liferay.journal.service

com.liferay.journal.api

JournalArticleResourceLocalServiceWrapper

Old: com.liferay.portlet.journal.service New:
com.liferay.journal.service

com.liferay.journal.api

JournalArticleResourceModel

Old: com.liferay.portlet.journal.model New: com.liferay.journal.model

com.liferay.journal.api

JournalArticleResourcePersistence

Old: com.liferay.portlet.journal.service.persistence New:
com.liferay.journal.service.persistence

com.liferay.journal.api

JournalArticleResourceSoap

Old: com.liferay.portlet.journal.model New: com.liferay.journal.model

com.liferay.journal.api

JournalArticleResourceUtil

Old: com.liferay.portlet.journal.service.persistence New:
com.liferay.journal.service.persistence

com.liferay.journal.api

JournalArticleResourceWrapper

Old: com.liferay.portlet.journal.model New: com.liferay.journal.model

com.liferay.journal.api

JournalArticleService

Old: com.liferay.portlet.journal.service New:
com.liferay.journal.service

com.liferay.journal.api

JournalArticleServiceUtil

Old: com.liferay.portlet.journal.service New:
com.liferay.journal.service

com.liferay.journal.api

JournalArticleServiceWrapper

Old: com.liferay.portlet.journal.service New:
com.liferay.journal.service

com.liferay.journal.api

JournalArticleSoap

Old: com.liferay.portlet.journal.model New: com.liferay.journal.model

com.liferay.journal.api

JournalArticleUtil

Old: com.liferay.portlet.journal.service.persistence New:
com.liferay.journal.service.persistence

com.liferay.journal.api

JournalArticleWrapper

Old: com.liferay.portlet.journal.model New: com.liferay.journal.model

com.liferay.journal.api

JournalContent

Old: com.liferay.portlet.journalcontent.util New:
com.liferay.journal.util

com.liferay.journal.api

JournalContentSearch

Old: com.liferay.portlet.journal.model New: com.liferay.journal.model

com.liferay.journal.api

JournalContentSearchLocalService

Old: com.liferay.portlet.journal.service New:
com.liferay.journal.service

com.liferay.journal.api

JournalContentSearchLocalServiceUtil

Old: com.liferay.portlet.journal.service New:
com.liferay.journal.service

com.liferay.journal.api

JournalContentSearchLocalServiceWrapper

Old: com.liferay.portlet.journal.service New:
com.liferay.journal.service

com.liferay.journal.api

JournalContentSearchModel

Old: com.liferay.portlet.journal.model New: com.liferay.journal.model

com.liferay.journal.api

JournalContentSearchPersistence

Old: com.liferay.portlet.journal.service.persistence New:
com.liferay.journal.service.persistence

com.liferay.journal.api

JournalContentSearchSoap

Old: com.liferay.portlet.journal.model New: com.liferay.journal.model

com.liferay.journal.api

JournalContentSearchUtil

Old: com.liferay.portlet.journal.service.persistence New:
com.liferay.journal.service.persistence

com.liferay.journal.api

JournalContentSearchWrapper

Old: com.liferay.portlet.journal.model New: com.liferay.journal.model

com.liferay.journal.api

JournalConverter

Old: com.liferay.portlet.journal.util New: com.liferay.journal.util

com.liferay.journal.api

JournalFeed

Old: com.liferay.portlet.journal.model New: com.liferay.journal.model

com.liferay.journal.api

JournalFeedConstants

Old: com.liferay.portlet.journal.model New: com.liferay.journal.model

com.liferay.journal.api

JournalFeedFinder

Old: com.liferay.portlet.journal.service.persistence New:
com.liferay.journal.service.persistence

com.liferay.journal.api

JournalFeedLocalService

Old: com.liferay.portlet.journal.service New:
com.liferay.journal.service

com.liferay.journal.api

JournalFeedLocalServiceUtil

Old: com.liferay.portlet.journal.service New:
com.liferay.journal.service

com.liferay.journal.api

JournalFeedLocalServiceWrapper

Old: com.liferay.portlet.journal.service New:
com.liferay.journal.service

com.liferay.journal.api

JournalFeedModel

Old: com.liferay.portlet.journal.model New: com.liferay.journal.model

com.liferay.journal.api

JournalFeedPersistence

Old: com.liferay.portlet.journal.service.persistence New:
com.liferay.journal.service.persistence

com.liferay.journal.api

JournalFeedService

Old: com.liferay.portlet.journal.service New:
com.liferay.journal.service

com.liferay.journal.api

JournalFeedServiceUtil

Old: com.liferay.portlet.journal.service New:
com.liferay.journal.service

com.liferay.journal.api

JournalFeedServiceWrapper

Old: com.liferay.portlet.journal.service New:
com.liferay.journal.service

com.liferay.journal.api

JournalFeedSoap

Old: com.liferay.portlet.journal.model New: com.liferay.journal.model

com.liferay.journal.api

JournalFeedUtil

Old: com.liferay.portlet.journal.service.persistence New:
com.liferay.journal.service.persistence

com.liferay.journal.api

JournalFeedWrapper

Old: com.liferay.portlet.journal.model New: com.liferay.journal.model

com.liferay.journal.api

JournalFolder

Old: com.liferay.portlet.journal.model New: com.liferay.journal.model

com.liferay.journal.api

JournalFolderFinder

Old: com.liferay.portlet.journal.service.persistence New:
com.liferay.journal.service.persistence

com.liferay.journal.api

JournalFolderLocalService

Old: com.liferay.portlet.journal.service New:
com.liferay.journal.service

com.liferay.journal.api

JournalFolderLocalServiceUtil

Old: com.liferay.portlet.journal.service New:
com.liferay.journal.service

com.liferay.journal.api

JournalFolderLocalServiceWrapper

Old: com.liferay.portlet.journal.service New:
com.liferay.journal.service

com.liferay.journal.api

JournalFolderModel

Old: com.liferay.portlet.journal.model New: com.liferay.journal.model

com.liferay.journal.api

JournalFolderPersistence

Old: com.liferay.portlet.journal.service.persistence New:
com.liferay.journal.service.persistence

com.liferay.journal.api

JournalFolderService

Old: com.liferay.portlet.journal.service New:
com.liferay.journal.service

com.liferay.journal.api

JournalFolderServiceUtil

Old: com.liferay.portlet.journal.service New:
com.liferay.journal.service

com.liferay.journal.api

JournalFolderServiceWrapper

Old: com.liferay.portlet.journal.service New:
com.liferay.journal.service

com.liferay.journal.api

JournalFolderSoap

Old: com.liferay.portlet.journal.model New: com.liferay.journal.model

com.liferay.journal.api

JournalFolderUtil

Old: com.liferay.portlet.journal.service.persistence New:
com.liferay.journal.service.persistence

com.liferay.journal.api

JournalFolderWrapper

Old: com.liferay.portlet.journal.model New: com.liferay.journal.model

com.liferay.journal.api

JournalSearchConstants

Old: com.liferay.portlet.journal.model New: com.liferay.journal.model

com.liferay.journal.api

JournalStructureConstants

Old: com.liferay.portlet.journal.model New: com.liferay.journal.model

com.liferay.journal.api

LDAPFilterException

Old: com.liferay.portal.kernel.ldap New:
com.liferay.portal.security.ldap.validator

com.liferay.portal.security.ldap

LDAPGroup

Old: com.liferay.portal.security.ldap New:
com.liferay.portal.security.ldap.exportimport

com.liferay.portal.security.ldap

LDAPServerNameException

Old: com.liferay.portal.kernel.ldap New:
com.liferay.portal.security.ldap

com.liferay.portal.security.ldap

LDAPToPortalConverter

Old: com.liferay.portal.security.ldap New:
com.liferay.portal.security.ldap.exportimport

com.liferay.portal.security.ldap

LDAPUser

Old: com.liferay.portal.security.ldap New:
com.liferay.portal.security.ldap.exportimport

com.liferay.portal.security.ldap

LDAPUtil

Old: com.liferay.portal.kernel.ldap New:
com.liferay.portal.security.ldap.util

com.liferay.portal.security.ldap

LockLocalService

Old: com.liferay.portal.service New: com.liferay.portal.lock.service

com.liferay.portal.lock.api

LockLocalServiceUtil

Old: com.liferay.portal.service New: com.liferay.portal.lock.service

com.liferay.portal.lock.api

LockLocalServiceWrapper

Old: com.liferay.portal.service New: com.liferay.portal.lock.service

com.liferay.portal.lock.api

LockModel

Old: com.liferay.portal.model New: com.liferay.portal.lock.model

com.liferay.portal.lock.api

LockPersistence

Old: com.liferay.portal.service.persistence New:
com.liferay.portal.lock.service.persistence

com.liferay.portal.lock.api

LockSoap

Old: com.liferay.portal.model New: com.liferay.portal.lock.model

com.liferay.portal.lock.api

LockUtil

Old: com.liferay.portal.service.persistence New:
com.liferay.portal.lock.service.persistence

com.liferay.portal.lock.api

LockWrapper

Old: com.liferay.portal.model New: com.liferay.portal.lock.model

com.liferay.portal.lock.api

MBeanRegistry

Old: com.liferay.portal.kernel.jmx New: com.liferay.portal.jmx

com.liferay.portal.jmx

MDRAction

Old: com.liferay.portlet.mobiledevicerules.model New:
com.liferay.mobile.device.rules.model

com.liferay.mobile.device.rules.api

MDRActionLocalService

Old: com.liferay.portlet.mobiledevicerules.service New:
com.liferay.mobile.device.rules.service

com.liferay.mobile.device.rules.api

MDRActionLocalServiceUtil

Old: com.liferay.portlet.mobiledevicerules.service New:
com.liferay.mobile.device.rules.service

com.liferay.mobile.device.rules.api

MDRActionLocalServiceWrapper

Old: com.liferay.portlet.mobiledevicerules.service New:
com.liferay.mobile.device.rules.service

com.liferay.mobile.device.rules.api

MDRActionModel

Old: com.liferay.portlet.mobiledevicerules.model New:
com.liferay.mobile.device.rules.model

com.liferay.mobile.device.rules.api

MDRActionPersistence

Old: com.liferay.portlet.mobiledevicerules.service.persistence New:
com.liferay.mobile.device.rules.service.persistence

com.liferay.mobile.device.rules.api

MDRActionService

Old: com.liferay.portlet.mobiledevicerules.service New:
com.liferay.mobile.device.rules.service

com.liferay.mobile.device.rules.api

MDRActionServiceUtil

Old: com.liferay.portlet.mobiledevicerules.service New:
com.liferay.mobile.device.rules.service

com.liferay.mobile.device.rules.api

MDRActionServiceWrapper

Old: com.liferay.portlet.mobiledevicerules.service New:
com.liferay.mobile.device.rules.service

com.liferay.mobile.device.rules.api

MDRActionSoap

Old: com.liferay.portlet.mobiledevicerules.model New:
com.liferay.mobile.device.rules.model

com.liferay.mobile.device.rules.api

MDRActionUtil

Old: com.liferay.portlet.mobiledevicerules.service.persistence New:
com.liferay.mobile.device.rules.service.persistence

com.liferay.mobile.device.rules.api

MDRActionWrapper

Old: com.liferay.portlet.mobiledevicerules.model New:
com.liferay.mobile.device.rules.model

com.liferay.mobile.device.rules.api

MDRPermission

Old: com.liferay.portlet.mobiledevicerules.service.permission New:
com.liferay.mobile.device.rules.service.permission

com.liferay.mobile.device.rules.service

MDRRule

Old: com.liferay.portlet.mobiledevicerules.model New:
com.liferay.mobile.device.rules.model

com.liferay.mobile.device.rules.api

MDRRuleGroup

Old: com.liferay.portlet.mobiledevicerules.model New:
com.liferay.mobile.device.rules.model

com.liferay.mobile.device.rules.api

MDRRuleGroupFinder

Old: com.liferay.portlet.mobiledevicerules.service.persistence New:
com.liferay.mobile.device.rules.service.persistence

com.liferay.mobile.device.rules.api

MDRRuleGroupInstanceLocalService

Old: com.liferay.portlet.mobiledevicerules.service New:
com.liferay.mobile.device.rules.service

com.liferay.mobile.device.rules.api

MDRRuleGroupInstanceLocalServiceUtil

Old: com.liferay.portlet.mobiledevicerules.service New:
com.liferay.mobile.device.rules.service

com.liferay.mobile.device.rules.api

MDRRuleGroupInstanceLocalServiceWrapper

Old: com.liferay.portlet.mobiledevicerules.service New:
com.liferay.mobile.device.rules.service

com.liferay.mobile.device.rules.api

MDRRuleGroupInstanceModel

Old: com.liferay.portlet.mobiledevicerules.model New:
com.liferay.mobile.device.rules.model

com.liferay.mobile.device.rules.api

MDRRuleGroupInstancePermission

Old: com.liferay.portlet.mobiledevicerules.service.permission New:
com.liferay.mobile.device.rules.service.permission

com.liferay.mobile.device.rules.service

MDRRuleGroupInstancePersistence

Old: com.liferay.portlet.mobiledevicerules.service.persistence New:
com.liferay.mobile.device.rules.service.persistence

com.liferay.mobile.device.rules.api

MDRRuleGroupInstanceService

Old: com.liferay.portlet.mobiledevicerules.service New:
com.liferay.mobile.device.rules.service

com.liferay.mobile.device.rules.api

MDRRuleGroupInstanceServiceUtil

Old: com.liferay.portlet.mobiledevicerules.service New:
com.liferay.mobile.device.rules.service

com.liferay.mobile.device.rules.api

MDRRuleGroupInstanceServiceWrapper

Old: com.liferay.portlet.mobiledevicerules.service New:
com.liferay.mobile.device.rules.service

com.liferay.mobile.device.rules.api

MDRRuleGroupInstanceSoap

Old: com.liferay.portlet.mobiledevicerules.model New:
com.liferay.mobile.device.rules.model

com.liferay.mobile.device.rules.api

MDRRuleGroupInstanceUtil

Old: com.liferay.portlet.mobiledevicerules.service.persistence New:
com.liferay.mobile.device.rules.service.persistence

com.liferay.mobile.device.rules.api

MDRRuleGroupInstanceWrapper

Old: com.liferay.portlet.mobiledevicerules.model New:
com.liferay.mobile.device.rules.model

com.liferay.mobile.device.rules.api

MDRRuleGroupLocalService

Old: com.liferay.portlet.mobiledevicerules.service New:
com.liferay.mobile.device.rules.service

com.liferay.mobile.device.rules.api

MDRRuleGroupLocalServiceUtil

Old: com.liferay.portlet.mobiledevicerules.service New:
com.liferay.mobile.device.rules.service

com.liferay.mobile.device.rules.api

MDRRuleGroupLocalServiceWrapper

Old: com.liferay.portlet.mobiledevicerules.service New:
com.liferay.mobile.device.rules.service

com.liferay.mobile.device.rules.api

MDRRuleGroupModel

Old: com.liferay.portlet.mobiledevicerules.model New:
com.liferay.mobile.device.rules.model

com.liferay.mobile.device.rules.api

MDRRuleGroupPermission

Old: com.liferay.portlet.mobiledevicerules.service.permission New:
com.liferay.mobile.device.rules.service.permission

com.liferay.mobile.device.rules.service

MDRRuleGroupPersistence

Old: com.liferay.portlet.mobiledevicerules.service.persistence New:
com.liferay.mobile.device.rules.service.persistence

com.liferay.mobile.device.rules.api

MDRRuleGroupService

Old: com.liferay.portlet.mobiledevicerules.service New:
com.liferay.mobile.device.rules.service

com.liferay.mobile.device.rules.api

MDRRuleGroupServiceUtil

Old: com.liferay.portlet.mobiledevicerules.service New:
com.liferay.mobile.device.rules.service

com.liferay.mobile.device.rules.api

MDRRuleGroupServiceWrapper

Old: com.liferay.portlet.mobiledevicerules.service New:
com.liferay.mobile.device.rules.service

com.liferay.mobile.device.rules.api

MDRRuleGroupSoap

Old: com.liferay.portlet.mobiledevicerules.model New:
com.liferay.mobile.device.rules.model

com.liferay.mobile.device.rules.api

MDRRuleGroupUtil

Old: com.liferay.portlet.mobiledevicerules.service.persistence New:
com.liferay.mobile.device.rules.service.persistence

com.liferay.mobile.device.rules.api

MDRRuleGroupWrapper

Old: com.liferay.portlet.mobiledevicerules.model New:
com.liferay.mobile.device.rules.model

com.liferay.mobile.device.rules.api

MDRRuleLocalService

Old: com.liferay.portlet.mobiledevicerules.service New:
com.liferay.mobile.device.rules.service

com.liferay.mobile.device.rules.api

MDRRuleLocalServiceUtil

Old: com.liferay.portlet.mobiledevicerules.service New:
com.liferay.mobile.device.rules.service

com.liferay.mobile.device.rules.api

MDRRuleLocalServiceWrapper

Old: com.liferay.portlet.mobiledevicerules.service New:
com.liferay.mobile.device.rules.service

com.liferay.mobile.device.rules.api

MDRRuleModel

Old: com.liferay.portlet.mobiledevicerules.model New:
com.liferay.mobile.device.rules.model

com.liferay.mobile.device.rules.api

MDRRulePersistence

Old: com.liferay.portlet.mobiledevicerules.service.persistence New:
com.liferay.mobile.device.rules.service.persistence

com.liferay.mobile.device.rules.api

MDRRuleService

Old: com.liferay.portlet.mobiledevicerules.service New:
com.liferay.mobile.device.rules.service

com.liferay.mobile.device.rules.api

MDRRuleServiceUtil

Old: com.liferay.portlet.mobiledevicerules.service New:
com.liferay.mobile.device.rules.service

com.liferay.mobile.device.rules.api

MDRRuleServiceWrapper

Old: com.liferay.portlet.mobiledevicerules.service New:
com.liferay.mobile.device.rules.service

com.liferay.mobile.device.rules.api

MDRRuleSoap

Old: com.liferay.portlet.mobiledevicerules.model New:
com.liferay.mobile.device.rules.model

com.liferay.mobile.device.rules.api

MDRRuleUtil

Old: com.liferay.portlet.mobiledevicerules.service.persistence New:
com.liferay.mobile.device.rules.service.persistence

com.liferay.mobile.device.rules.api

MDRRuleWrapper

Old: com.liferay.portlet.mobiledevicerules.model New:
com.liferay.mobile.device.rules.model

com.liferay.mobile.device.rules.api

MemoryReportDesignRetriever

Old: com.liferay.portal.kernel.bi.reporting New:
com.liferay.portal.reports.engine

com.liferay.portal.reports.engine.api

MessageBusManager

Old: com.liferay.portal.kernel.messaging.jmx New:
com.liferay.portal.messaging.internal.jmx

com.liferay.portal.messaging

MessageBusManagerMBean

Old: com.liferay.portal.kernel.messaging.jmx New:
com.liferay.portal.messaging.internal.jmx

com.liferay.portal.messaging

Modifications

Old: com.liferay.portal.security.ldap New:
com.liferay.portal.security.ldap.exportimport

com.liferay.portal.security.ldap

NoSuchArticleException

Old: com.liferay.portlet.journal New: com.liferay.journal.exception

com.liferay.journal.api

NoSuchArticleImageException

Old: com.liferay.portlet.journal New: com.liferay.journal.exception

com.liferay.journal.api

NoSuchArticleResourceException

Old: com.liferay.portlet.journal New: com.liferay.journal.exception

com.liferay.journal.api

NoSuchCartException

Old: com.liferay.portlet.shopping New: com.liferay.shopping.exception

com.liferay.shopping.api

NoSuchChoiceException

Old: com.liferay.portlet.polls New: com.liferay.polls.exception

com.liferay.polls.api

NoSuchContentSearchException

Old: com.liferay.portlet.journal New: com.liferay.journal.exception

com.liferay.journal.api

NoSuchCouponException

Old: com.liferay.portlet.shopping New: com.liferay.shopping.exception

com.liferay.shopping.api

NoSuchFeedException

Old: com.liferay.portlet.journal New: com.liferay.journal.exception

com.liferay.journal.api

NoSuchItemException

Old: com.liferay.portlet.shopping New: com.liferay.shopping.exception

com.liferay.shopping.api

NoSuchItemFieldException

Old: com.liferay.portlet.shopping New: com.liferay.shopping.exception

com.liferay.shopping.api

NoSuchItemPriceException

Old: com.liferay.portlet.shopping New: com.liferay.shopping.exception

com.liferay.shopping.api

NoSuchNodeException

Old: com.liferay.portlet.wiki New: com.liferay.wiki.exception

com.liferay.wiki.api

NoSuchOrderException

Old: com.liferay.portlet.shopping New: com.liferay.shopping.exception

com.liferay.shopping.api

NoSuchOrderItemException

Old: com.liferay.portlet.shopping New: com.liferay.shopping.exception

com.liferay.shopping.api

NoSuchPageException

Old: com.liferay.portlet.wiki New: com.liferay.wiki.exception

com.liferay.wiki.api

NoSuchPageResourceException

Old: com.liferay.portlet.wiki New: com.liferay.wiki.exception

com.liferay.wiki.api

NoSuchQuestionException

Old: com.liferay.portlet.polls New: com.liferay.polls.exception

com.liferay.polls.api

NoSuchRecordException

Old: com.liferay.portlet.dynamicdatalists New:
com.liferay.dynamic.data.lists.exception

com.liferay.dynamic.data.lists.api

NoSuchRecordSetException

Old: com.liferay.portlet.dynamicdatalists New:
com.liferay.dynamic.data.lists.exception

com.liferay.dynamic.data.lists.api

NoSuchRecordVersionException

Old: com.liferay.portlet.dynamicdatalists New:
com.liferay.dynamic.data.lists.exception

com.liferay.dynamic.data.lists.api

NoSuchRuleException

Old: com.liferay.portlet.mobiledevicerules New:
com.liferay.mobile.device.rules.exception

com.liferay.mobile.device.rules.api

NoSuchRuleGroupException

Old: com.liferay.portlet.mobiledevicerules New:
com.liferay.mobile.device.rules.exception

com.liferay.mobile.device.rules.api

NoSuchRuleGroupInstanceException

Old: com.liferay.portlet.mobiledevicerules New:
com.liferay.mobile.device.rules.exception

com.liferay.mobile.device.rules.api

NoSuchStorageLinkException

Old: com.liferay.portlet.dynamicdatamapping New:
com.liferay.dynamic.data.mapping.exception

com.liferay.dynamic.data.mapping.api

NoSuchStructureLinkException

Old: com.liferay.portlet.dynamicdatamapping New:
com.liferay.dynamic.data.mapping.exception

com.liferay.dynamic.data.mapping.api

NoSuchTemplateException

Old: com.liferay.portlet.dynamicdatamapping New:
com.liferay.dynamic.data.mapping.exception

com.liferay.dynamic.data.mapping.api

NoSuchTemplateException

Old: com.liferay.portlet.journal New:
com.liferay.dynamic.data.mapping.exception

com.liferay.dynamic.data.mapping.api

NoSuchVoteException

Old: com.liferay.portlet.polls New: com.liferay.polls.exception

com.liferay.polls.api

NodeNameException

Old: com.liferay.portlet.wiki New: com.liferay.wiki.exception

com.liferay.wiki.api

OrderDateComparator

Old: com.liferay.portlet.shopping.util.comparator New:
com.liferay.shopping.util.comparator

com.liferay.shopping.api

PageContentException

Old: com.liferay.portlet.wiki New: com.liferay.wiki.exception

com.liferay.wiki.api

PageCreateDateComparator

Old: com.liferay.portlet.wiki.util.comparator New:
com.liferay.wiki.util.comparator

com.liferay.wiki.service

PageTitleComparator

Old: com.liferay.portlet.wiki.util.comparator New:
com.liferay.wiki.util.comparator

com.liferay.wiki.service

PageTitleException

Old: com.liferay.portlet.wiki New: com.liferay.wiki.exception

com.liferay.wiki.api

PageVersionComparator

Old: com.liferay.portlet.wiki.util.comparator New:
com.liferay.wiki.util.comparator

com.liferay.wiki.service

PageVersionException

Old: com.liferay.portlet.wiki New: com.liferay.wiki.exception

com.liferay.wiki.api

PollsChoice

Old: com.liferay.portlet.polls.model New: com.liferay.polls.model

com.liferay.polls.api

PollsChoiceLocalService

Old: com.liferay.portlet.polls.service New: com.liferay.polls.service

com.liferay.polls.api

PollsChoiceLocalServiceUtil

Old: com.liferay.portlet.polls.service New: com.liferay.polls.service

com.liferay.polls.api

PollsChoiceLocalServiceWrapper

Old: com.liferay.portlet.polls.service New: com.liferay.polls.service

com.liferay.polls.api

PollsChoiceModel

Old: com.liferay.portlet.polls.model New: com.liferay.polls.model

com.liferay.polls.api

PollsChoicePersistence

Old: com.liferay.portlet.polls.service.persistence New:
com.liferay.polls.service.persistence

com.liferay.polls.api

PollsChoiceService

Old: com.liferay.portlet.polls.service New: com.liferay.polls.service

com.liferay.polls.api

PollsChoiceServiceUtil

Old: com.liferay.portlet.polls.service New: com.liferay.polls.service

com.liferay.polls.api

PollsChoiceServiceWrapper

Old: com.liferay.portlet.polls.service New: com.liferay.polls.service

com.liferay.polls.api

PollsChoiceSoap

Old: com.liferay.portlet.polls.model New: com.liferay.polls.model

com.liferay.polls.api

PollsChoiceUtil

Old: com.liferay.portlet.polls.service.persistence New:
com.liferay.polls.service.persistence

com.liferay.polls.api

PollsChoiceWrapper

Old: com.liferay.portlet.polls.model New: com.liferay.polls.model

com.liferay.polls.api

PollsQuestion

Old: com.liferay.portlet.polls.model New: com.liferay.polls.model

com.liferay.polls.api

PollsQuestionLocalService

Old: com.liferay.portlet.polls.service New: com.liferay.polls.service

com.liferay.polls.api

PollsQuestionLocalServiceUtil

Old: com.liferay.portlet.polls.service New: com.liferay.polls.service

com.liferay.polls.api

PollsQuestionLocalServiceWrapper

Old: com.liferay.portlet.polls.service New: com.liferay.polls.service

com.liferay.polls.api

PollsQuestionModel

Old: com.liferay.portlet.polls.model New: com.liferay.polls.model

com.liferay.polls.api

PollsQuestionPersistence

Old: com.liferay.portlet.polls.service.persistence New:
com.liferay.polls.service.persistence

com.liferay.polls.api

PollsQuestionService

Old: com.liferay.portlet.polls.service New: com.liferay.polls.service

com.liferay.polls.api

PollsQuestionServiceUtil

Old: com.liferay.portlet.polls.service New: com.liferay.polls.service

com.liferay.polls.api

PollsQuestionServiceWrapper

Old: com.liferay.portlet.polls.service New: com.liferay.polls.service

com.liferay.polls.api

PollsQuestionSoap

Old: com.liferay.portlet.polls.model New: com.liferay.polls.model

com.liferay.polls.api

PollsQuestionUtil

Old: com.liferay.portlet.polls.service.persistence New:
com.liferay.polls.service.persistence

com.liferay.polls.api

PollsQuestionWrapper

Old: com.liferay.portlet.polls.model New: com.liferay.polls.model

com.liferay.polls.api

PollsVote

Old: com.liferay.portlet.polls.model New: com.liferay.polls.model

com.liferay.polls.api

PollsVoteLocalService

Old: com.liferay.portlet.polls.service New: com.liferay.polls.service

com.liferay.polls.api

PollsVoteLocalServiceUtil

Old: com.liferay.portlet.polls.service New: com.liferay.polls.service

com.liferay.polls.api

PollsVoteLocalServiceWrapper

Old: com.liferay.portlet.polls.service New: com.liferay.polls.service

com.liferay.polls.api

PollsVoteModel

Old: com.liferay.portlet.polls.model New: com.liferay.polls.model

com.liferay.polls.api

PollsVotePersistence

Old: com.liferay.portlet.polls.service.persistence New:
com.liferay.polls.service.persistence

com.liferay.polls.api

PollsVoteService

Old: com.liferay.portlet.polls.service New: com.liferay.polls.service

com.liferay.polls.api

PollsVoteServiceUtil

Old: com.liferay.portlet.polls.service New: com.liferay.polls.service

com.liferay.polls.api

PollsVoteServiceWrapper

Old: com.liferay.portlet.polls.service New: com.liferay.polls.service

com.liferay.polls.api

PollsVoteSoap

Old: com.liferay.portlet.polls.model New: com.liferay.polls.model

com.liferay.polls.api

PollsVoteUtil

Old: com.liferay.portlet.polls.service.persistence New:
com.liferay.polls.service.persistence

com.liferay.polls.api

PollsVoteWrapper

Old: com.liferay.portlet.polls.model New: com.liferay.polls.model

com.liferay.polls.api

PortalExecutorFactory

Old: com.liferay.portal.kernel.executor New:
com.liferay.portal.executor.internal

com.liferay.portal.executor

PortalToLDAPConverter

Old: com.liferay.portal.security.ldap New:
com.liferay.portal.security.ldap.exportimport

com.liferay.portal.security.ldap

PortletDisplayTemplate

Old: com.liferay.portlet.portletdisplaytemplate.util New:
com.liferay.portlet.display.template

com.liferay.portlet.display.template

PortletDisplayTemplateConstants

Old: com.liferay.portlet.portletdisplaytemplate.util New:
com.liferay.portlet.display.template

com.liferay.portlet.display.template

PortletDisplayTemplateUtil

Old: com.liferay.portlet.portletdisplaytemplate.util New:
com.liferay.portlet.display.template

com.liferay.portlet.display.template

QueryIndexingHitsProcessor

Old: com.liferay.portal.kernel.search New:
com.liferay.portal.search.internal.hits

com.liferay.portal.search

QuerySuggestionHitsProcessor

Old: com.liferay.portal.kernel.search New:
com.liferay.portal.search.internal.hits

com.liferay.portal.search

QueryType

Old: com.liferay.portal.kernel.bi.rules New:
com.liferay.portal.rules.engine

com.liferay.portal.rules.engine.api

QuestionChoiceException

Old: com.liferay.portlet.polls New: com.liferay.polls.exception

com.liferay.polls.api

QuestionDescriptionException

Old: com.liferay.portlet.polls New: com.liferay.polls.exception

com.liferay.polls.api

QuestionExpirationDateException

Old: com.liferay.portlet.polls New: com.liferay.polls.exception

com.liferay.polls.api

QuestionExpiredException

Old: com.liferay.portlet.polls New: com.liferay.polls.exception

com.liferay.polls.api

QuestionTitleException

Old: com.liferay.portlet.polls New: com.liferay.polls.exception

com.liferay.polls.api

RecordSetDDMStructureIdException

Old: com.liferay.portlet.dynamicdatalists New:
com.liferay.dynamic.data.lists.exception

com.liferay.dynamic.data.lists.api

RecordSetDuplicateRecordSetKeyException

Old: com.liferay.portlet.dynamicdatalists New:
com.liferay.dynamic.data.lists.exception

com.liferay.dynamic.data.lists.api

RecordSetNameException

Old: com.liferay.portlet.dynamicdatalists New:
com.liferay.dynamic.data.lists.exception

com.liferay.dynamic.data.lists.api

RegistryAwareMBeanServer

Old: com.liferay.portal.kernel.jmx New: com.liferay.portal.jmx.internal

com.liferay.portal.jmx

ReportCompilerRequestMessageListener

Old: com.liferay.portal.kernel.bi.reporting.messaging New:
com.liferay.portal.reports.engine.messaging

com.liferay.portal.reports.engine.api

ReportDataSourceType

Old: com.liferay.portal.kernel.bi.reporting New:
com.liferay.portal.reports.engine

com.liferay.portal.reports.engine.api

ReportDesignRetriever

Old: com.liferay.portal.kernel.bi.reporting New:
com.liferay.portal.reports.engine

com.liferay.portal.reports.engine.api

ReportEngine

Old: com.liferay.portal.kernel.bi.reporting New:
com.liferay.portal.reports.engine

com.liferay.portal.reports.engine.api

ReportExportException

Old: com.liferay.portal.kernel.bi.reporting New:
com.liferay.portal.reports.engine

com.liferay.portal.reports.engine.api

ReportFormat

Old: com.liferay.portal.kernel.bi.reporting New:
com.liferay.portal.reports.engine

com.liferay.portal.reports.engine.api

ReportFormatExporter

Old: com.liferay.portal.kernel.bi.reporting New:
com.liferay.portal.reports.engine

com.liferay.portal.reports.engine.api

ReportFormatExporterRegistry

Old: com.liferay.portal.kernel.bi.reporting New:
com.liferay.portal.reports.engine

com.liferay.portal.reports.engine.api

ReportGenerationException

Old: com.liferay.portal.kernel.bi.reporting New:
com.liferay.portal.reports.engine

com.liferay.portal.reports.engine.api

ReportRequest

Old: com.liferay.portal.kernel.bi.reporting New:
com.liferay.portal.reports.engine

com.liferay.portal.reports.engine.api

ReportRequestContext

Old: com.liferay.portal.kernel.bi.reporting New:
com.liferay.portal.reports.engine

com.liferay.portal.reports.engine.api

ReportRequestMessageListener

Old: com.liferay.portal.kernel.bi.reporting.messaging New:
com.liferay.portal.reports.engine.messaging

com.liferay.portal.reports.engine.api

ReportResultContainer

Old: com.liferay.portal.kernel.bi.reporting New:
com.liferay.portal.reports.engine

com.liferay.portal.reports.engine.api

RequestStatistics

Old: com.liferay.portal.kernel.monitoring.statistics New:
com.liferay.portal.monitoring.internal.statistics

com.liferay.portal.monitoring

RequiredCouponException

Old: com.liferay.portlet.shopping New: com.liferay.shopping.exception

com.liferay.shopping.api

RequiredNodeException

Old: com.liferay.portlet.wiki New: com.liferay.wiki.exception

com.liferay.wiki.api

RequiredTemplateException

Old: com.liferay.portlet.dynamicdatamapping New:
com.liferay.dynamic.data.mapping.exception

com.liferay.dynamic.data.mapping.api

RequiredTemplateException

Old: com.liferay.portlet.journal New:
com.liferay.dynamic.data.mapping.exception

com.liferay.dynamic.data.mapping.api

RuleGroupInstancePriorityComparator

Old: com.liferay.portlet.mobiledevicerules.util New:
com.liferay.mobile.device.rules.util.comparator

com.liferay.mobile.device.rules.api

RuleGroupProcessor

Old: com.liferay.portal.kernel.mobile.device.rulegroup New:
com.liferay.mobile.device.rules.rule

com.liferay.mobile.device.rules.api

RuleGroupProcessorUtil

Old: com.liferay.portal.kernel.mobile.device.rulegroup New:
com.liferay.mobile.device.rules.rule

com.liferay.mobile.device.rules.api

RuleHandler

Old: com.liferay.portal.kernel.mobile.device.rulegroup.rule New:
com.liferay.mobile.device.rules.rule

com.liferay.mobile.device.rules.api

RulesEngine

Old: com.liferay.portal.kernel.bi.rules New:
com.liferay.portal.rules.engine

com.liferay.portal.rules.engine.api

RulesEngineException

Old: com.liferay.portal.kernel.bi.rules New:
com.liferay.portal.rules.engine

com.liferay.portal.rules.engine.api

RulesEngineUtil

Old: com.liferay.portal.kernel.bi.rules New:
com.liferay.portal.rules.engine

com.liferay.portal.rules.engine.api

RulesLanguage

Old: com.liferay.portal.kernel.bi.rules New:
com.liferay.portal.rules.engine

com.liferay.portal.rules.engine.api

RulesResourceRetriever

Old: com.liferay.portal.kernel.bi.rules New:
com.liferay.portal.rules.engine

com.liferay.portal.rules.engine.api

ServletContextReportDesignRetriever

Old: com.liferay.portal.kernel.bi.reporting.servlet New:
com.liferay.portal.reports.engine.servlet

com.liferay.portal.reports.engine.api

ShippingCityException

Old: com.liferay.portlet.shopping New: com.liferay.shopping.exception

com.liferay.shopping.api

ShippingCountryException

Old: com.liferay.portlet.shopping New: com.liferay.shopping.exception

com.liferay.shopping.api

ShippingEmailAddressException

Old: com.liferay.portlet.shopping New: com.liferay.shopping.exception

com.liferay.shopping.api

ShippingFirstNameException

Old: com.liferay.portlet.shopping New: com.liferay.shopping.exception

com.liferay.shopping.api

ShippingLastNameException

Old: com.liferay.portlet.shopping New: com.liferay.shopping.exception

com.liferay.shopping.api

ShippingPhoneException

Old: com.liferay.portlet.shopping New: com.liferay.shopping.exception

com.liferay.shopping.api

ShippingStateException

Old: com.liferay.portlet.shopping New: com.liferay.shopping.exception

com.liferay.shopping.api

ShippingStreetException

Old: com.liferay.portlet.shopping New: com.liferay.shopping.exception

com.liferay.shopping.api

ShippingZipException

Old: com.liferay.portlet.shopping New: com.liferay.shopping.exception

com.liferay.shopping.api

ShoppingCart

Old: com.liferay.portlet.shopping.model New: com.liferay.shopping.model

com.liferay.shopping.api

ShoppingCartItem

Old: com.liferay.portlet.shopping.model New: com.liferay.shopping.model

com.liferay.shopping.api

ShoppingCartLocalService

Old: com.liferay.portlet.shopping.service New:
com.liferay.shopping.service

com.liferay.shopping.api

ShoppingCartLocalServiceUtil

Old: com.liferay.portlet.shopping.service New:
com.liferay.shopping.service

com.liferay.shopping.api

ShoppingCartLocalServiceWrapper

Old: com.liferay.portlet.shopping.service New:
com.liferay.shopping.service

com.liferay.shopping.api

ShoppingCartModel

Old: com.liferay.portlet.shopping.model New: com.liferay.shopping.model

com.liferay.shopping.api

ShoppingCartPersistence

Old: com.liferay.portlet.shopping.service.persistence New:
com.liferay.shopping.service.persistence

com.liferay.shopping.api

ShoppingCartSoap

Old: com.liferay.portlet.shopping.model New: com.liferay.shopping.model

com.liferay.shopping.api

ShoppingCartUtil

Old: com.liferay.portlet.shopping.service.persistence New:
com.liferay.shopping.service.persistence

com.liferay.shopping.api

ShoppingCartWrapper

Old: com.liferay.portlet.shopping.model New: com.liferay.shopping.model

com.liferay.shopping.api

ShoppingCategory

Old: com.liferay.portlet.shopping.model New: com.liferay.shopping.model

com.liferay.shopping.api

ShoppingCategoryConstants

Old: com.liferay.portlet.shopping.model New: com.liferay.shopping.model

com.liferay.shopping.api

ShoppingCategoryLocalService

Old: com.liferay.portlet.shopping.service New:
com.liferay.shopping.service

com.liferay.shopping.api

ShoppingCategoryLocalServiceUtil

Old: com.liferay.portlet.shopping.service New:
com.liferay.shopping.service

com.liferay.shopping.api

ShoppingCategoryLocalServiceWrapper

Old: com.liferay.portlet.shopping.service New:
com.liferay.shopping.service

com.liferay.shopping.api

ShoppingCategoryModel

Old: com.liferay.portlet.shopping.model New: com.liferay.shopping.model

com.liferay.shopping.api

ShoppingCategoryPersistence

Old: com.liferay.portlet.shopping.service.persistence New:
com.liferay.shopping.service.persistence

com.liferay.shopping.api

ShoppingCategoryService

Old: com.liferay.portlet.shopping.service New:
com.liferay.shopping.service

com.liferay.shopping.api

ShoppingCategoryServiceUtil

Old: com.liferay.portlet.shopping.service New:
com.liferay.shopping.service

com.liferay.shopping.api

ShoppingCategoryServiceWrapper

Old: com.liferay.portlet.shopping.service New:
com.liferay.shopping.service

com.liferay.shopping.api

ShoppingCategorySoap

Old: com.liferay.portlet.shopping.model New: com.liferay.shopping.model

com.liferay.shopping.api

ShoppingCategoryUtil

Old: com.liferay.portlet.shopping.service.persistence New:
com.liferay.shopping.service.persistence

com.liferay.shopping.api

ShoppingCategoryWrapper

Old: com.liferay.portlet.shopping.model New: com.liferay.shopping.model

com.liferay.shopping.api

ShoppingCoupon

Old: com.liferay.portlet.shopping.model New: com.liferay.shopping.model

com.liferay.shopping.api

ShoppingCouponConstants

Old: com.liferay.portlet.shopping.model New: com.liferay.shopping.model

com.liferay.shopping.api

ShoppingCouponFinder

Old: com.liferay.portlet.shopping.service.persistence New:
com.liferay.shopping.service.persistence

com.liferay.shopping.api

ShoppingCouponLocalService

Old: com.liferay.portlet.shopping.service New:
com.liferay.shopping.service

com.liferay.shopping.api

ShoppingCouponLocalServiceUtil

Old: com.liferay.portlet.shopping.service New:
com.liferay.shopping.service

com.liferay.shopping.api

ShoppingCouponLocalServiceWrapper

Old: com.liferay.portlet.shopping.service New:
com.liferay.shopping.service

com.liferay.shopping.api

ShoppingCouponModel

Old: com.liferay.portlet.shopping.model New: com.liferay.shopping.model

com.liferay.shopping.api

ShoppingCouponPersistence

Old: com.liferay.portlet.shopping.service.persistence New:
com.liferay.shopping.service.persistence

com.liferay.shopping.api

ShoppingCouponService

Old: com.liferay.portlet.shopping.service New:
com.liferay.shopping.service

com.liferay.shopping.api

ShoppingCouponServiceUtil

Old: com.liferay.portlet.shopping.service New:
com.liferay.shopping.service

com.liferay.shopping.api

ShoppingCouponServiceWrapper

Old: com.liferay.portlet.shopping.service New:
com.liferay.shopping.service

com.liferay.shopping.api

ShoppingCouponSoap

Old: com.liferay.portlet.shopping.model New: com.liferay.shopping.model

com.liferay.shopping.api

ShoppingCouponUtil

Old: com.liferay.portlet.shopping.service.persistence New:
com.liferay.shopping.service.persistence

com.liferay.shopping.api

ShoppingCouponWrapper

Old: com.liferay.portlet.shopping.model New: com.liferay.shopping.model

com.liferay.shopping.api

ShoppingItem

Old: com.liferay.portlet.shopping.model New: com.liferay.shopping.model

com.liferay.shopping.api

ShoppingItemField

Old: com.liferay.portlet.shopping.model New: com.liferay.shopping.model

com.liferay.shopping.api

ShoppingItemFieldLocalService

Old: com.liferay.portlet.shopping.service New:
com.liferay.shopping.service

com.liferay.shopping.api

ShoppingItemFieldLocalServiceUtil

Old: com.liferay.portlet.shopping.service New:
com.liferay.shopping.service

com.liferay.shopping.api

ShoppingItemFieldLocalServiceWrapper

Old: com.liferay.portlet.shopping.service New:
com.liferay.shopping.service

com.liferay.shopping.api

ShoppingItemFieldModel

Old: com.liferay.portlet.shopping.model New: com.liferay.shopping.model

com.liferay.shopping.api

ShoppingItemFieldPersistence

Old: com.liferay.portlet.shopping.service.persistence New:
com.liferay.shopping.service.persistence

com.liferay.shopping.api

ShoppingItemFieldSoap

Old: com.liferay.portlet.shopping.model New: com.liferay.shopping.model

com.liferay.shopping.api

ShoppingItemFieldUtil

Old: com.liferay.portlet.shopping.service.persistence New:
com.liferay.shopping.service.persistence

com.liferay.shopping.api

ShoppingItemFieldWrapper

Old: com.liferay.portlet.shopping.model New: com.liferay.shopping.model

com.liferay.shopping.api

ShoppingItemFinder

Old: com.liferay.portlet.shopping.service.persistence New:
com.liferay.shopping.service.persistence

com.liferay.shopping.api

ShoppingItemLocalService

Old: com.liferay.portlet.shopping.service New:
com.liferay.shopping.service

com.liferay.shopping.api

ShoppingItemLocalServiceUtil

Old: com.liferay.portlet.shopping.service New:
com.liferay.shopping.service

com.liferay.shopping.api

ShoppingItemLocalServiceWrapper

Old: com.liferay.portlet.shopping.service New:
com.liferay.shopping.service

com.liferay.shopping.api

ShoppingItemModel

Old: com.liferay.portlet.shopping.model New: com.liferay.shopping.model

com.liferay.shopping.api

ShoppingItemPersistence

Old: com.liferay.portlet.shopping.service.persistence New:
com.liferay.shopping.service.persistence

com.liferay.shopping.api

ShoppingItemPrice

Old: com.liferay.portlet.shopping.model New: com.liferay.shopping.model

com.liferay.shopping.api

ShoppingItemPriceConstants

Old: com.liferay.portlet.shopping.model New: com.liferay.shopping.model

com.liferay.shopping.api

ShoppingItemPriceLocalService

Old: com.liferay.portlet.shopping.service New:
com.liferay.shopping.service

com.liferay.shopping.api

ShoppingItemPriceLocalServiceUtil

Old: com.liferay.portlet.shopping.service New:
com.liferay.shopping.service

com.liferay.shopping.api

ShoppingItemPriceLocalServiceWrapper

Old: com.liferay.portlet.shopping.service New:
com.liferay.shopping.service

com.liferay.shopping.api

ShoppingItemPriceModel

Old: com.liferay.portlet.shopping.model New: com.liferay.shopping.model

com.liferay.shopping.api

ShoppingItemPricePersistence

Old: com.liferay.portlet.shopping.service.persistence New:
com.liferay.shopping.service.persistence

com.liferay.shopping.api

ShoppingItemPriceSoap

Old: com.liferay.portlet.shopping.model New: com.liferay.shopping.model

com.liferay.shopping.api

ShoppingItemPriceUtil

Old: com.liferay.portlet.shopping.service.persistence New:
com.liferay.shopping.service.persistence

com.liferay.shopping.api

ShoppingItemPriceWrapper

Old: com.liferay.portlet.shopping.model New: com.liferay.shopping.model

com.liferay.shopping.api

ShoppingItemService

Old: com.liferay.portlet.shopping.service New:
com.liferay.shopping.service

com.liferay.shopping.api

ShoppingItemServiceUtil

Old: com.liferay.portlet.shopping.service New:
com.liferay.shopping.service

com.liferay.shopping.api

ShoppingItemServiceWrapper

Old: com.liferay.portlet.shopping.service New:
com.liferay.shopping.service

com.liferay.shopping.api

ShoppingItemSoap

Old: com.liferay.portlet.shopping.model New: com.liferay.shopping.model

com.liferay.shopping.api

ShoppingItemUtil

Old: com.liferay.portlet.shopping.service.persistence New:
com.liferay.shopping.service.persistence

com.liferay.shopping.api

ShoppingItemWrapper

Old: com.liferay.portlet.shopping.model New: com.liferay.shopping.model

com.liferay.shopping.api

ShoppingOrder

Old: com.liferay.portlet.shopping.model New: com.liferay.shopping.model

com.liferay.shopping.api

ShoppingOrderConstants

Old: com.liferay.portlet.shopping.model New: com.liferay.shopping.model

com.liferay.shopping.api

ShoppingOrderFinder

Old: com.liferay.portlet.shopping.service.persistence New:
com.liferay.shopping.service.persistence

com.liferay.shopping.api

ShoppingOrderItem

Old: com.liferay.portlet.shopping.model New: com.liferay.shopping.model

com.liferay.shopping.api

ShoppingOrderItemLocalService

Old: com.liferay.portlet.shopping.service New:
com.liferay.shopping.service

com.liferay.shopping.api

ShoppingOrderItemLocalServiceUtil

Old: com.liferay.portlet.shopping.service New:
com.liferay.shopping.service

com.liferay.shopping.api

ShoppingOrderItemLocalServiceWrapper

Old: com.liferay.portlet.shopping.service New:
com.liferay.shopping.service

com.liferay.shopping.api

ShoppingOrderItemModel

Old: com.liferay.portlet.shopping.model New: com.liferay.shopping.model

com.liferay.shopping.api

ShoppingOrderItemPersistence

Old: com.liferay.portlet.shopping.service.persistence New:
com.liferay.shopping.service.persistence

com.liferay.shopping.api

ShoppingOrderItemSoap

Old: com.liferay.portlet.shopping.model New: com.liferay.shopping.model

com.liferay.shopping.api

ShoppingOrderItemUtil

Old: com.liferay.portlet.shopping.service.persistence New:
com.liferay.shopping.service.persistence

com.liferay.shopping.api

ShoppingOrderItemWrapper

Old: com.liferay.portlet.shopping.model New: com.liferay.shopping.model

com.liferay.shopping.api

ShoppingOrderLocalService

Old: com.liferay.portlet.shopping.service New:
com.liferay.shopping.service

com.liferay.shopping.api

ShoppingOrderLocalServiceUtil

Old: com.liferay.portlet.shopping.service New:
com.liferay.shopping.service

com.liferay.shopping.api

ShoppingOrderLocalServiceWrapper

Old: com.liferay.portlet.shopping.service New:
com.liferay.shopping.service

com.liferay.shopping.api

ShoppingOrderModel

Old: com.liferay.portlet.shopping.model New: com.liferay.shopping.model

com.liferay.shopping.api

ShoppingOrderPersistence

Old: com.liferay.portlet.shopping.service.persistence New:
com.liferay.shopping.service.persistence

com.liferay.shopping.api

ShoppingOrderService

Old: com.liferay.portlet.shopping.service New:
com.liferay.shopping.service

com.liferay.shopping.api

ShoppingOrderServiceUtil

Old: com.liferay.portlet.shopping.service New:
com.liferay.shopping.service

com.liferay.shopping.api

ShoppingOrderServiceWrapper

Old: com.liferay.portlet.shopping.service New:
com.liferay.shopping.service

com.liferay.shopping.api

ShoppingOrderSoap

Old: com.liferay.portlet.shopping.model New: com.liferay.shopping.model

com.liferay.shopping.api

ShoppingOrderUtil

Old: com.liferay.portlet.shopping.service.persistence New:
com.liferay.shopping.service.persistence

com.liferay.shopping.api

ShoppingOrderWrapper

Old: com.liferay.portlet.shopping.model New: com.liferay.shopping.model

com.liferay.shopping.api

SortFactoryImpl

Old: com.liferay.portal.kernel.search New:
com.liferay.portal.search.internal

com.liferay.portal.search

Statistics

Old: com.liferay.portal.kernel.monitoring.statistics New:
com.liferay.portal.monitoring.statistics

com.liferay.portal.monitoring

StorageAdapter

Old: com.liferay.portlet.dynamicdatamapping.storage New:
com.liferay.dynamic.data.mapping.storage

com.liferay.dynamic.data.mapping.api

StorageEngine

Old: com.liferay.portlet.dynamicdatamapping.storage New:
com.liferay.dynamic.data.mapping.storage

com.liferay.dynamic.data.mapping.api

StorageException

Old: com.liferay.portlet.dynamicdatamapping New:
com.liferay.dynamic.data.mapping.exception

com.liferay.dynamic.data.mapping.api

StorageFieldNameException

Old: com.liferay.portlet.dynamicdatamapping New:
com.liferay.dynamic.data.mapping.exception

com.liferay.dynamic.data.mapping.api

StructureDuplicateStructureKeyException

Old: com.liferay.portlet.dynamicdatamapping New:
com.liferay.dynamic.data.mapping.exception

com.liferay.dynamic.data.mapping.api

StructureFieldException

Old: com.liferay.portlet.dynamicdatamapping New:
com.liferay.dynamic.data.mapping.exception

com.liferay.dynamic.data.mapping.api

StructureIdComparator

Old: com.liferay.portlet.dynamicdatamapping.util.comparator New:
com.liferay.dynamic.data.mapping.util.comparator

com.liferay.dynamic.data.mapping.api

StructureModifiedDateComparator

Old: com.liferay.portlet.dynamicdatamapping.util.comparator New:
com.liferay.dynamic.data.mapping.util.comparator

com.liferay.dynamic.data.mapping.api

StructureStructureKeyComparator

Old: com.liferay.portlet.dynamicdatamapping.util.comparator New:
com.liferay.dynamic.data.mapping.util.comparator

com.liferay.dynamic.data.mapping.api

SummaryStatistics

Old: com.liferay.portal.kernel.monitoring.statistics New:
com.liferay.portal.monitoring.statistics

com.liferay.portal.monitoring

SynchronousMessageListener

Old: com.liferay.portal.kernel.messaging.sender New:
com.liferay.portal.messaging.internal.sender

com.liferay.portal.messaging

TemplateDuplicateTemplateKeyException

Old: com.liferay.portlet.dynamicdatamapping New:
com.liferay.dynamic.data.mapping.exception

com.liferay.dynamic.data.mapping.api

TemplateIdComparator

Old: com.liferay.portlet.dynamicdatamapping.util.comparator New:
com.liferay.dynamic.data.mapping.util.comparator

com.liferay.dynamic.data.mapping.api

TemplateModifiedDateComparator

Old: com.liferay.portlet.dynamicdatamapping.util.comparator New:
com.liferay.dynamic.data.mapping.util.comparator

com.liferay.dynamic.data.mapping.api

TemplateNameException

Old: com.liferay.portlet.dynamicdatamapping New:
com.liferay.dynamic.data.mapping.exception

com.liferay.dynamic.data.mapping.api

TemplateNameException

Old: com.liferay.portlet.journal New:
com.liferay.dynamic.data.mapping.exception

com.liferay.dynamic.data.mapping.api

TemplateScriptException

Old: com.liferay.portlet.dynamicdatamapping New:
com.liferay.dynamic.data.mapping.exception

com.liferay.dynamic.data.mapping.api

TemplateSmallImageNameException

Old: com.liferay.portlet.dynamicdatamapping New:
com.liferay.dynamic.data.mapping.exception

com.liferay.dynamic.data.mapping.api

TemplateSmallImageNameException

Old: com.liferay.portlet.journal New:
com.liferay.dynamic.data.mapping.exception

com.liferay.dynamic.data.mapping.api

TemplateSmallImageSizeException

Old: com.liferay.portlet.dynamicdatamapping New:
com.liferay.dynamic.data.mapping.exception

com.liferay.dynamic.data.mapping.api

TemplateSmallImageSizeException

Old: com.liferay.portlet.journal New:
com.liferay.dynamic.data.mapping.exception

com.liferay.dynamic.data.mapping.api

UnknownRuleHandlerException

Old: com.liferay.portal.kernel.mobile.device.rulegroup.rule New:
com.liferay.mobile.device.rules.rule

com.liferay.mobile.device.rules.api

UserConverterKeys

Old: com.liferay.portal.security.ldap New:
com.liferay.portal.security.ldap

com.liferay.portal.security.ldap

WikiFormatException

Old: com.liferay.portlet.wiki New: com.liferay.wiki.exception

com.liferay.wiki.api

WikiNode

Old: com.liferay.portlet.wiki.model New: com.liferay.wiki.model

com.liferay.wiki.api

WikiNodeLocalService

Old: com.liferay.portlet.wiki.service New: com.liferay.wiki.service

com.liferay.wiki.api

WikiNodeLocalServiceUtil

Old: com.liferay.portlet.wiki.service New: com.liferay.wiki.service

com.liferay.wiki.api

WikiNodeLocalServiceWrapper

Old: com.liferay.portlet.wiki.service New: com.liferay.wiki.service

com.liferay.wiki.api

WikiNodeModel

Old: com.liferay.portlet.wiki.model New: com.liferay.wiki.model

com.liferay.wiki.api

WikiNodePersistence

Old: com.liferay.portlet.wiki.service.persistence New:
com.liferay.wiki.service.persistence

com.liferay.wiki.api

WikiNodeService

Old: com.liferay.portlet.wiki.service New: com.liferay.wiki.service

com.liferay.wiki.api

WikiNodeServiceUtil

Old: com.liferay.portlet.wiki.service New: com.liferay.wiki.service

com.liferay.wiki.api

WikiNodeServiceWrapper

Old: com.liferay.portlet.wiki.service New: com.liferay.wiki.service

com.liferay.wiki.api

WikiNodeSoap

Old: com.liferay.portlet.wiki.model New: com.liferay.wiki.model

com.liferay.wiki.api

WikiNodeUtil

Old: com.liferay.portlet.wiki.service.persistence New:
com.liferay.wiki.service.persistence

com.liferay.wiki.api

WikiNodeWrapper

Old: com.liferay.portlet.wiki.model New: com.liferay.wiki.model

com.liferay.wiki.api

WikiPage

Old: com.liferay.portlet.wiki.model New: com.liferay.wiki.model

com.liferay.wiki.api

WikiPageConstants

Old: com.liferay.portlet.wiki.model New: com.liferay.wiki.model

com.liferay.wiki.api

WikiPageDisplay

Old: com.liferay.portlet.wiki.model New: com.liferay.wiki.model

com.liferay.wiki.api

WikiPageFinder

Old: com.liferay.portlet.wiki.service.persistence New:
com.liferay.wiki.service.persistence

com.liferay.wiki.api

WikiPageLocalService

Old: com.liferay.portlet.wiki.service New: com.liferay.wiki.service

com.liferay.wiki.api

WikiPageLocalServiceUtil

Old: com.liferay.portlet.wiki.service New: com.liferay.wiki.service

com.liferay.wiki.api

WikiPageLocalServiceWrapper

Old: com.liferay.portlet.wiki.service New: com.liferay.wiki.service

com.liferay.wiki.api

WikiPageModel

Old: com.liferay.portlet.wiki.model New: com.liferay.wiki.model

com.liferay.wiki.api

WikiPagePersistence

Old: com.liferay.portlet.wiki.service.persistence New:
com.liferay.wiki.service.persistence

com.liferay.wiki.api

WikiPageResource

Old: com.liferay.portlet.wiki.model New: com.liferay.wiki.model

com.liferay.wiki.api

WikiPageResourceLocalService

Old: com.liferay.portlet.wiki.service New: com.liferay.wiki.service

com.liferay.wiki.api

WikiPageResourceLocalServiceUtil

Old: com.liferay.portlet.wiki.service New: com.liferay.wiki.service

com.liferay.wiki.api

WikiPageResourceLocalServiceWrapper

Old: com.liferay.portlet.wiki.service New: com.liferay.wiki.service

com.liferay.wiki.api

WikiPageResourceModel

Old: com.liferay.portlet.wiki.model New: com.liferay.wiki.model

com.liferay.wiki.api

WikiPageResourcePersistence

Old: com.liferay.portlet.wiki.service.persistence New:
com.liferay.wiki.service.persistence

com.liferay.wiki.api

WikiPageResourceSoap

Old: com.liferay.portlet.wiki.model New: com.liferay.wiki.model

com.liferay.wiki.api

WikiPageResourceUtil

Old: com.liferay.portlet.wiki.service.persistence New:
com.liferay.wiki.service.persistence

com.liferay.wiki.api

WikiPageResourceWrapper

Old: com.liferay.portlet.wiki.model New: com.liferay.wiki.model

com.liferay.wiki.api

WikiPageService

Old: com.liferay.portlet.wiki.service New: com.liferay.wiki.service

com.liferay.wiki.api

WikiPageServiceUtil

Old: com.liferay.portlet.wiki.service New: com.liferay.wiki.service

com.liferay.wiki.api

WikiPageServiceWrapper

Old: com.liferay.portlet.wiki.service New: com.liferay.wiki.service

com.liferay.wiki.api

WikiPageSoap

Old: com.liferay.portlet.wiki.model New: com.liferay.wiki.model

com.liferay.wiki.api

WikiPageUtil

Old: com.liferay.portlet.wiki.service.persistence New:
com.liferay.wiki.service.persistence

com.liferay.wiki.api

WikiPageWrapper

Old: com.liferay.portlet.wiki.model New: com.liferay.wiki.model

com.liferay.wiki.api

\textbf{Related Topics}

\href{/docs/7-0/tutorials/-/knowledge_base/t/liferay-upgrade-planner}{Liferay
Upgrade Planner}

\href{/docs/7-0/reference/-/knowledge_base/r/development-reference}{Development
Reference}

\chapter{Theme Gulp Tasks}\label{theme-gulp-tasks}

Theme projects that use the
\href{https://github.com/liferay/liferay-themes-sdk/tree/master/packages}{liferay
JS Theme Toolkit}, such as those created with the
\href{/docs/7-0/tutorials/-/knowledge_base/t/themes-generator}{Liferay
Theme Generator} have access to several
\href{https://www.npmjs.com/package/gulp}{gulp} tasks that you can
execute to manage and deploy your theme.

\noindent\hrulefill

\textbf{Note:} Gulp is included as a local dependency in generated
themes, so you are not required to install it. It can be accessed by
running \texttt{node\_modules\textbackslash{}.bin\textbackslash{}gulp}
followed by the Gulp task from a generated theme's root folder.

\noindent\hrulefill

Here are the gulp tasks you can execute:

\begin{itemize}
\item
  \texttt{build}: generates the base theme files, compiles Sass into
  CSS, and zips all theme files into a WAR file that you can deploy to a
  Liferay server.
\item
  \texttt{deploy}: runs the \texttt{build} task and deploys the WAR file
  to the configured local app server.
\end{itemize}

\noindent\hrulefill

\begin{verbatim}
 **Note:** If you're running the [Felix Gogo shell](/docs/7-0/reference/-/knowledge_base/r/using-the-felix-gogo-shell),
 you can also deploy your theme using the `gulp deploy:gogo` command. **This
 task will NOT work for 6.2 themes.**
\end{verbatim}

\noindent\hrulefill

\begin{itemize}
\item
  \texttt{extend}: allows you to specify a base theme or themelet to
  extend. By default, themes created with the
  \href{https://github.com/liferay/generator-liferay-theme}{Liferay
  Theme Generator} are based off of the
  \href{https://www.npmjs.com/package/liferay-theme-styled}{styled
  theme}.

  You first are prompted if you want to extend a Base theme or Themelet,
  then you're prompted for where you would like to search for modules.
  Selecting \emph{Globally installed npm modules} searches globally
  accessible npm modules on your computer. Selecting \emph{npm registry}
  searches for published modules on npm. If you have v8.x.x of the
  \href{https://github.com/liferay/liferay-themes-sdk/tree/master/packages}{Liferay
  JS Theme Toolkit} installed, you can also \emph{Specify a package URL}
  to locate a themelet.
\end{itemize}

\noindent\hrulefill

\begin{verbatim}
 **Note:** You can retrieve the URL for a package by running 
 `npm show package-name dist.tarball`. 
\end{verbatim}

\noindent\hrulefill

\begin{verbatim}
After you've selected modules from the options it provides, the modules are
added to your `package.json` file as dependencies. Run `npm install` to
install them.
\end{verbatim}

\noindent\hrulefill

\begin{verbatim}
 **Note:** The Classic theme is an implementation of an existing base theme 
 and is therefore not meant to be extended. Extending Liferay's Classic 
 theme is strongly discouraged.
\end{verbatim}

\noindent\hrulefill

\begin{itemize}
\item
  \texttt{kickstart}: allows you to copy the CSS, images, JS, and
  templates from another theme into the \texttt{src} directory of your
  own. While this is similar to the \texttt{extend} task, kickstarting
  from another theme is a one time inheritance, whereas extending from
  another theme is a dynamic inheritance that applies your src files on
  top of the base theme on every build.
\item
  \texttt{init}: prompts you for local and remote app server information
  to use in theme deployment.
\item
  \texttt{status}: displays the name of the base theme/themelets your
  theme extends.
\item
  \texttt{watch}: allows you to preview the changes you make to your
  theme without requiring a full redeploy. After invoking the
  \texttt{watch} task, every time you save any changes to a file in your
  theme, applicable changes are compiled and they're copied directly to
  a proxy port (e.g.~\texttt{9080}) for you to preview live.
  \textbf{Note:} you must have
  \href{/docs/7-0/tutorials/-/knowledge_base/t/using-developer-mode-with-themes}{Developer
  Mode} enabled to use the \texttt{watch} task. Once you're happy with
  the live preview, deploy your theme to apply the changes.
\end{itemize}

\textbf{Related Topics}

\href{/docs/7-0/tutorials/-/knowledge_base/t/themes-generator}{Liferay
Theme Generator}

\chapter{Theme Reference Guide}\label{theme-reference-guide}

A theme is made up of several files. Although most of the files are
named after their matching components, their function may be unclear.

This document explains each file's usage to make clear which files to
modify and which files to leave untouched.

\subsection{Theme Anatomy}\label{theme-anatomy}

There are two main approaches to theme development for 7.0: themes built
using the Node.js build tools with the
\href{/docs/7-0/tutorials/-/knowledge_base/t/themes-generator}{theme
generator} and
\href{/docs/7-0/tutorials/-/knowledge_base/t/creating-themes-with-liferay-ide}{themes
built using @ide@}.

Themes developed with the theme generator have the anatomy shown below.
Although themes developed with @ide@ have a slightly different anatomy
built with the
\href{/docs/7-0/reference/-/knowledge_base/r/theme-template}{theme
project template}, the core theme files are the same. Note that the
\texttt{build} folder is shown for reference, and is generated when the
theme is compiled.

\begin{itemize}
\tightlist
\item
  \texttt{theme-name/}

  \begin{itemize}
  \tightlist
  \item
    \texttt{build/}(generated when the theme is compiled)

    \begin{itemize}
    \tightlist
    \item
      \texttt{css/}

      \begin{itemize}
      \tightlist
      \item
        \href{/docs/7-0/reference/-/knowledge_base/r/theme-reference-guide\#-applicationscss}{\texttt{\_application.scss}}
      \item
        \href{/docs/7-0/reference/-/knowledge_base/r/theme-reference-guide\#-aui-customscss}{\texttt{\_aui\_custom.scss}}
      \item
        \href{/docs/7-0/reference/-/knowledge_base/r/theme-reference-guide\#-aui-variablesscss}{\texttt{\_aui\_variables.scss}}
      \item
        \href{/docs/7-0/reference/-/knowledge_base/r/theme-reference-guide\#-basescss}{\texttt{\_base.scss}}
      \item
        \href{/docs/7-0/reference/-/knowledge_base/r/theme-reference-guide\#-customscss}{\texttt{\_custom.scss}}
      \item
        \href{/docs/7-0/reference/-/knowledge_base/r/theme-reference-guide\#-extrasscss}{\texttt{\_extras.scss}}
      \item
        \href{/docs/7-0/reference/-/knowledge_base/r/theme-reference-guide\#-importsscss}{\texttt{\_imports.scss}}
      \item
        \href{/docs/7-0/reference/-/knowledge_base/r/theme-reference-guide\#-layoutscss}{\texttt{\_layout.scss}}
      \item
        \href{/docs/7-0/reference/-/knowledge_base/r/theme-reference-guide\#-liferay-customscss}{\texttt{\_liferay\_custom.scss}}
      \item
        \href{/docs/7-0/reference/-/knowledge_base/r/theme-reference-guide\#-liferay-variables-customscss}{\texttt{\_liferay\_variables\_custom.scss}}
      \item
        \href{/docs/7-0/reference/-/knowledge_base/r/theme-reference-guide\#-liferay-variablesscss}{\texttt{\_liferay\_variables.scss}}
      \item
        \href{/docs/7-0/reference/-/knowledge_base/r/theme-reference-guide\#-navigationscss}{\texttt{\_navigation.scss}}
      \item
        \href{/docs/7-0/reference/-/knowledge_base/r/theme-reference-guide\#-portalscss}{\texttt{\_portal.scss}}
      \item
        \href{/docs/7-0/reference/-/knowledge_base/r/theme-reference-guide\#-portletscss}{\texttt{\_portlet.scss}}
      \item
        \href{/docs/7-0/reference/-/knowledge_base/r/theme-reference-guide\#-taglibscss}{\texttt{\_taglib.scss}}
      \item
        \texttt{application/}
      \item
        \texttt{aui/}
      \item
        \href{/docs/7-0/reference/-/knowledge_base/r/theme-reference-guide\#auiscss}{\texttt{aui.scss}}
      \item
        \texttt{base/}
      \item
        \href{/docs/7-0/reference/-/knowledge_base/r/theme-reference-guide\#font-awesomescss}{\texttt{font-awesome.scss}}
      \item
        \texttt{layout/}
      \item
        \href{/docs/7-0/reference/-/knowledge_base/r/theme-reference-guide\#mainscss}{\texttt{main.scss}}
      \item
        \texttt{navigation/}
      \item
        \texttt{portal/}
      \item
        \texttt{portlet/}
      \item
        \texttt{taglib/}
      \end{itemize}
    \item
      \texttt{templates/}

      \begin{itemize}
      \tightlist
      \item
        \href{/docs/7-0/reference/-/knowledge_base/r/theme-reference-guide\#init-customftl}{\texttt{init\_custom.ftl}}
      \item
        \href{/docs/7-0/reference/-/knowledge_base/r/theme-reference-guide\#initftl}{\texttt{init.ftl}}
      \item
        \href{/docs/7-0/reference/-/knowledge_base/r/theme-reference-guide\#navigationftl}{\texttt{navigation.ftl}}
      \item
        \href{/docs/7-0/reference/-/knowledge_base/r/theme-reference-guide\#portal-normalftl}{\texttt{portal\_normal.ftl}}
      \item
        \href{/docs/7-0/reference/-/knowledge_base/r/theme-reference-guide\#portal-pop-upftl}{\texttt{portal\_pop\_up.ftl}}
      \item
        \href{/docs/7-0/reference/-/knowledge_base/r/theme-reference-guide\#portletftl}{\texttt{portlet.ftl}}
      \item
        (other directories that have been copied from src)
      \end{itemize}
    \end{itemize}
  \item
    \texttt{dist/} (generated when the theme is compiled. This is where
    the theme's war file is placed after a build/deploy.)
  \item
    \href{/docs/7-0/reference/-/knowledge_base/r/theme-reference-guide\#gulpfile-js}{\texttt{gulpfile.js}}
  \item
    \href{/docs/7-0/reference/-/knowledge_base/r/theme-reference-guide\#liferay-themejson}{\texttt{liferay-theme.json}}
  \item
    \texttt{node\_modules/} (generated when an \texttt{npm\ install}
    command is run from the root of the theme, and can be deleted at
    anytime and re-generated by running \texttt{npm\ install}.)

    \begin{itemize}
    \tightlist
    \item
      (many directories)
    \end{itemize}
  \item
    \href{/docs/7-0/reference/-/knowledge_base/r/theme-reference-guide\#packagejson}{\texttt{package.json}}
  \item
    \texttt{src/}

    \begin{itemize}
    \tightlist
    \item
      \texttt{css/}

      \begin{itemize}
      \tightlist
      \item
        (modified CSS files)
      \end{itemize}
    \item
      \texttt{images/}

      \begin{itemize}
      \tightlist
      \item
        (many directories)
      \end{itemize}
    \item
      \texttt{js/}

      \begin{itemize}
      \tightlist
      \item
        \href{/docs/7-0/reference/-/knowledge_base/r/theme-reference-guide\#mainjs}{\texttt{main.js}}
      \end{itemize}
    \item
      \texttt{templates/}

      \begin{itemize}
      \tightlist
      \item
        (Modified theme templates)
      \end{itemize}
    \item
      \texttt{WEB-INF/} - \texttt{lib/}

      \begin{itemize}
      \tightlist
      \item
        \href{/docs/7-0/reference/-/knowledge_base/r/theme-reference-guide\#liferay-look-and-feelxml}{\texttt{liferay-look-and-feel.xml}}
      \item
        \href{/docs/7-0/reference/-/knowledge_base/r/theme-reference-guide\#liferay-plugin-packageproperties}{\texttt{liferay-plugin-package.properties}}
      \item
        \texttt{src/}

        \begin{itemize}
        \tightlist
        \item
          \texttt{resources-importer/}

          \begin{itemize}
          \tightlist
          \item
            (Many directories)
          \end{itemize}
        \end{itemize}
      \end{itemize}
    \end{itemize}
  \end{itemize}
\end{itemize}

Regarding CSS files, it is recommended that you only modify
\texttt{\_custom.scss}, \texttt{\_aui\_custom.scss},
\texttt{\_aui\_variables.scss}, and
\texttt{\_liferay\_variables\_custom.scss}.

You can of course overwrite any CSS file that you wish, but if you
modify any other files, you will most likely be removing styling that
7.0 needs to work properly.

\section{Theme Files}\label{theme-files}

\subsection{\_application.scss}\label{application.scss}

Contains imports for application styles. Generally these files style
components that aren't Liferay specific, i.e.~Alloy or Bootstrap
components.

\subsection{\_aui\_custom.scss}\label{aui_custom.scss}

Used for AUI custom styles, i.e.~styles for a third party Bootstrap
theme. Anything written in this file is compiled in the same scope as
Bootstrap/Lexicon, so you can use their variables, mixins, etc. You can
also implement any of the variables you define in
\texttt{\_aui\_variables.scss}.

\subsection{\_aui\_variables.scss}\label{aui_variables.scss}

Used to store custom Sass variables. This file get's injected into the
Bootstrap/Lexicon build, so you can overwrite variables and change how
those libraries are compiled.

\subsection{\_base.scss}\label{base.scss}

Contains imports for the base styles for Liferay.

\subsection{\_custom.scss}\label{custom.scss}

Used for custom CSS styles. It is recommended that you place all of your
custom CSS modfications in this file.

\subsection{\_extras.scss}\label{extras.scss}

Contains styling that is considered non-essential and potentially dated
in the near future i.e.~box-shadows, rounded corners, etc. This allows
for easy maintenance.

\subsubsection{\_imports.scss}\label{imports.scss}

Contains imports for third-party libraries required for the theme
e.g.~Bourbon, Liferay Mixins, Lexicon Base Variables, and Bootstrap
Mixins.

\subsection{\_layout.scss}\label{layout.scss}

Contains imports for layout styles and variables.

\subsection{\_liferay\_custom.scss}\label{liferay_custom.scss}

Contains Liferay DXP styles that are compiled in the same scope as
Bootstrap/Lexicon.

\textbf{It's recommended that you NOT overwrite this file.}

\subsection{\_liferay\_variables\_custom.scss}\label{liferay_variables_custom.scss}

Used for overwriting variables defined in
\texttt{\_liferay\_variables.scss} without wiping out the whole file.

\subsection{\_liferay\_variables.scss}\label{liferay_variables.scss}

Contains variables that are used in \texttt{\_liferay\_custom.scss}.

\textbf{It's recommended that you NOT overwrite this file.}

\subsection{\_navigation.scss}\label{navigation.scss}

Contains imports for navigation styles.

\subsection{\_portal.scss}\label{portal.scss}

Contains imports for Portal components.

\subsection{\_portlet.scss}\label{portlet.scss}

Contains imports for portlet components.

\subsection{\_taglib.scss}\label{taglib.scss}

Contains imports for taglib styles.

\subsection{aui.scss}\label{aui.scss}

Contains the Lexicon base CSS import. If you want to just use Bootstrap,
or use Atlas, you can do so by adding one of the following imports:

\begin{verbatim}
@import "aui/lexicon/bootstrap";
\end{verbatim}

or

\begin{verbatim}
@import "aui/lexicon/atlas"; 
\end{verbatim}

\subsection{font-awesome.scss}\label{font-awesome.scss}

Contains the Font Awesome icon imports for Liferay.

\subsection{main.scss}\label{main.scss}

Contains imports for the core CSS files.

\subsection{init\_custom.ftl}\label{init_custom.ftl}

Used for custom FreeMarker variables i.e.~theme setting variables.

\subsection{init.ftl}\label{init.ftl}

Contains common FreeMarker variables that are available to use in your
theme templates. Useful for reference if you need access to theme
objects.

\textbf{It's recommended that you NOT overwrite this file.}

\subsection{navigation.ftl}\label{navigation.ftl}

The theme template for the theme's navigation.

\subsection{portal\_normal.ftl}\label{portal_normal.ftl}

Similar to the \texttt{index.html} of a website, this file acts as a hub
for all of the theme templates.

\subsection{portal\_pop\_up.ftl}\label{portal_pop_up.ftl}

The theme template for pop up dialogs for the theme's portlets.

\subsection{portlet.ftl}\label{portlet.ftl}

The theme template for the theme's portlets. If your theme uses
Application Decorators, you can modify this file to create application
decorator specific theme settings. See the
\href{/docs/7-0/tutorials/-/knowledge_base/t/portlet-decorators}{Portlet
Decorators tutorial} for more info.

\subsection{gulpfile.js}\label{gulpfile.js}

Defines the required gulp tasks for Node.js tool developed themes.

\textbf{It's recommended that you NOT overwrite this file.}

\subsection{liferay-theme.json}\label{liferay-theme.json}

Contains the configuration settings for your app server, in Node.js tool
based themes. You can change this file manually at any time to update
your server settings. The file can also be updated via the
\texttt{gulp\ init} task.

\subsection{package.json}\label{package.json}

contains theme setting information such as the theme template langauge,
version, and base theme, for Node.js tool developed themes. This file
can be updated manually. The \texttt{gulp\ extend} task can also be used
to change the base theme.

\subsection{main.js}\label{main.js}

Used for custom JavaScript.

\subsection{liferay-look-and-feel.xml}\label{liferay-look-and-feel.xml}

Contains basic information for the theme. If your theme has
\href{/docs/6-2/tutorials/-/knowledge_base/t/making-configurable-theme-settings}{theme
settings} , they are defined in this file. For a full explanation of
this file please see the
\href{@platform-ref@/7.0-latest/definitions/liferay-look-and-feel_7_0_0.dtd.html}{Definitions
docs}.

\subsection{liferay-plugin-package.properties}\label{liferay-plugin-package.properties}

Contains general properties for the theme. {[}Resources
Importer{]}\{/docs/7-0/tutorials/-/knowledge\_base/t/importing-resources-with-a-theme\}
configuration settings are also placed in this file. For a full
explanation of the properties available for this file please see the
\href{@platform-ref@/7.0-latest/propertiesdoc/liferay-plugin-package_7_0_0.properties.html}{7.0
Propertiesdoc}.

\chapter{Screenlets in Liferay Screens for
Android}\label{screenlets-in-liferay-screens-for-android}

Liferay Screens for Android contains several Screenlets that you can use
in your Android apps. This section contains the reference documentation
for each. If you're looking for instructions on using Screens, see the
\href{/docs/7-0/tutorials/-/knowledge_base/t/android-apps-with-liferay-screens}{Screens
tutorials}. The Screens tutorials contain instructions on
\href{/docs/7-0/tutorials/-/knowledge_base/t/using-screenlets-in-android-apps}{using
Screenlets} and
\href{/docs/7-0/tutorials/-/knowledge_base/t/using-views-in-android-screenlets}{using
views in Screenlets}. Each Screenlet reference document here lists the
Screenlet's features, compatibility, its module (if any), available
Views, attributes, listener methods, and more. The available Screenlets
are listed here with links to their reference documents:

\begin{itemize}
\item
  \href{/docs/7-0/reference/-/knowledge_base/r/loginscreenlet-for-android}{\textbf{Login
  Screenlet:}} Signs users in to a Liferay DXP instance.
\item
  \href{/docs/7-0/reference/-/knowledge_base/r/signupscreenlet-for-android}{\textbf{Sign
  Up Screenlet:}} Registers new users in a Liferay DXP instance.
\item
  \href{/docs/7-0/reference/-/knowledge_base/r/forgotpasswordscreenlet-for-android}{\textbf{Forgot
  Password Screenlet:}} Sends emails containing a new password or
  password reset link to users.
\item
  \href{/docs/7-0/reference/-/knowledge_base/r/userportraitscreenlet-for-android}{\textbf{User
  Portrait Screenlet:}} Show the user's portrait picture.
\item
  \href{/docs/7-0/reference/-/knowledge_base/r/ddlformscreenlet-for-android}{\textbf{DDL
  Form Screenlet:}} Presents dynamic forms to be filled out by users and
  submitted back to the server.
\item
  \href{/docs/7-0/reference/-/knowledge_base/r/ddllistscreenlet-for-android}{\textbf{DDL
  List Screenlet:}} Shows a list of records based on a pre-existing DDL
  in a Liferay instance.
\item
  \href{/docs/7-0/reference/-/knowledge_base/r/assetlistscreenlet-for-android}{\textbf{Asset
  List Screenlet:}} Shows a list of assets managed by
  \href{/docs/7-0/tutorials/-/knowledge_base/t/asset-framework}{Liferay's
  Asset Framework}. This includes web content, blog entries, documents,
  users, and more.
\item
  \href{/docs/7-0/reference/-/knowledge_base/r/webcontentdisplayscreenlet-for-android}{\textbf{Web
  Content Display Screenlet:}} Shows the web content's HTML or
  structured content. This Screenlet uses the features available in
  \href{/docs/7-0/user/-/knowledge_base/u/creating-web-content}{Web
  Content Management}.
\item
  \href{/docs/7-0/reference/-/knowledge_base/r/web-content-list-screenlet-for-android}{\textbf{Web
  Content List Screenlet:}} Shows a list of web contents from a folder,
  usually based on a pre-existing \texttt{DDMStructure}.
\item
  \href{/docs/7-0/reference/-/knowledge_base/r/image-gallery-screenlet-for-android}{\textbf{Image
  Gallery Screenlet:}} Shows a list of images from a folder. This
  Screenlet also lets users upload and delete images.
\item
  \href{/docs/7-0/reference/-/knowledge_base/r/rating-screenlet-for-android}{\textbf{Rating
  Screenlet:}} Shows the rating for an asset. This Screenlet also lets
  the user update or delete the rating.
\item
  \href{/docs/7-0/reference/-/knowledge_base/r/comment-list-screenlet-for-android}{\textbf{Comment
  List Screenlet:}} Shows a list of comments for an asset.
\item
  \href{/docs/7-0/reference/-/knowledge_base/r/comment-display-screenlet-for-android}{\textbf{Comment
  Display Screenlet:}} Shows a single comment for an asset.
\item
  \href{/docs/7-0/reference/-/knowledge_base/r/comment-add-screenlet-for-android}{\textbf{Comment
  Add Screenlet:}} Lets the user comment on an asset.
\item
  \href{/docs/7-0/reference/-/knowledge_base/r/asset-display-screenlet-for-android}{\textbf{Asset
  Display Screenlet:}} Displays an asset. Currently, this Screenlet can
  display Documents and Media Library files (\texttt{DLFileEntry}
  entities), blog articles (\texttt{BlogsEntry} entities), and web
  content articles (\texttt{WebContent} entities). You can also use it
  to display custom assets.
\item
  \href{/docs/7-0/reference/-/knowledge_base/r/blogs-entry-display-screenlet-for-android}{\textbf{Blogs
  Entry Display Screenlet:}} Shows a single blog entry.
\item
  \href{/docs/7-0/reference/-/knowledge_base/r/image-display-screenlet-for-android}{\textbf{Image
  Display Screenlet:}} Shows a single image file from a Liferay DXP
  instance's Documents and Media Library.
\item
  \href{/docs/7-0/reference/-/knowledge_base/r/video-display-screenlet-for-android}{\textbf{Video
  Display Screenlet:}} Shows a single video file from a Liferay DXP
  instance's Documents and Media Library.
\item
  \href{/docs/7-0/reference/-/knowledge_base/r/audio-display-screenlet-for-android}{\textbf{Audio
  Display Screenlet:}} Shows a single audio file from a Liferay DXP
  instance's Documents and Media Library.
\item
  \href{/docs/7-0/reference/-/knowledge_base/r/pdf-display-screenlet-for-android}{\textbf{PDF
  Display Screenlet:}} Shows a single PDF file from a Liferay DXP
  instance's Documents and Media Library.
\item
  \href{/docs/7-0/reference/-/knowledge_base/r/web-screenlet-for-android}{\textbf{Web
  Screenlet:}} Displays any web page. You can also customize the web
  page through injection of local and remote JavaScript and CSS files.
\end{itemize}

\section{Login Screenlet for Android}\label{login-screenlet-for-android}

\subsection{Requirements}\label{requirements}

\begin{itemize}
\tightlist
\item
  Android SDK 4.1 (API Level 16) or above
\item
  Liferay Portal 6.2 CE/EE, Liferay CE Portal 7.0/7.1, Liferay DXP 7.0+
\end{itemize}

\subsection{Compatibility}\label{compatibility}

\begin{itemize}
\tightlist
\item
  Android SDK 4.1 (API Level 16) or above
\end{itemize}

\subsection{Xamarin Requirements}\label{xamarin-requirements}

\begin{itemize}
\tightlist
\item
  Visual Studio 7.2
\item
  Mono .NET framework 5.4.1.6
\end{itemize}

\subsection{Features}\label{features}

Login Screenlet lets you authenticate portal users in your Android app.
The following types of authentication are supported:

\begin{itemize}
\item
  \textbf{Basic:} uses user login and password according to
  \href{http://tools.ietf.org/html/rfc2617}{HTTP Basic Access
  Authentication specification}. Depending on the authentication method
  used by your Liferay instance, you need to provide the user's email
  address, screen name, or user ID. You also need to provide the user's
  password.
\item
  \textbf{OAuth:} implements the
  \href{http://oauth.net/core/1.0a/}{OAuth 1.0a specification}.
\item
  \textbf{Cookie:} uses a cookie to log in. This lets you access
  documents and images in the portal's document library without the
  guest view permission in the portal. The other authentication types
  require this permission to access such files.
\end{itemize}

\noindent\hrulefill

\textbf{Note:} Cookie authentication with Login Screenlet is broken in
fix packs 14 through 18 of Liferay Digital Enterprise 7.0. This issue is
fixed in newer fix packs for Liferay Digital Enterprise 7.0.

\noindent\hrulefill

For instructions on configuring the Screenlet to use these
authentication types, see the below
\href{/docs/7-0/reference/-/knowledge_base/r/loginscreenlet-for-android\#portal-configuration}{Portal
Configuration} and
\href{/docs/7-0/reference/-/knowledge_base/r/loginscreenlet-for-android\#attributes}{Screenlet
Attributes} sections.

When a user successfully authenticates, their user attributes are
retrieved for use in the app. You can use the \texttt{SessionContext}
class to get the current user's attributes.

Note that user credentials and attributes can be stored in an app's data
store (see the \texttt{saveCredentials} attribute). Android's
\texttt{SharedPreferences} is currently the only data store implemented.
However, new and more secure data stores will be added in the future.
Stored user credentials can be used to automatically log the user in to
subsequent sessions. To do this, you can use the method
\texttt{SessionContext.loadStoredCredentials()}.

\subsection{JSON Services Used}\label{json-services-used}

Screenlets in Liferay Screens call the portal's JSON web services. This
Screenlet calls the following services and methods.

\noindent\hrulefill

\begin{longtable}[]{@{}lll@{}}
\toprule\noalign{}
Service & Method & Notes \\
\midrule\noalign{}
\endhead
\bottomrule\noalign{}
\endlastfoot
\texttt{UserService} & \texttt{getUserByEmailAddress} & Basic login \\
\texttt{UserService} & \texttt{getUserByScreenName} & Basic login \\
\texttt{UserService} & \texttt{getUserById} & Basic login \\
\texttt{UserService} & \texttt{getCurrentUser} & Cookie and OAuth
login \\
\end{longtable}

\noindent\hrulefill

\subsection{Module}\label{module}

\begin{itemize}
\tightlist
\item
  Auth
\end{itemize}

\subsection{Views}\label{views}

\begin{itemize}
\tightlist
\item
  Default
\item
  Material
\end{itemize}

For instructions on using these Views, see the \texttt{layoutId}
attribute in the
\href{/docs/7-0/reference/-/knowledge_base/r/loginscreenlet-for-android\#attributes}{Attributes
section below}.

\begin{figure}
\centering
\includegraphics{./images/screens-android-login.png}
\caption{The Login Screenlet using the Default (left) and Material
(right) Viewsets.}
\end{figure}

\subsection{Portal Configuration}\label{portal-configuration}

\subsubsection{Basic Authentication}\label{basic-authentication}

Before using Login Screenlet, you should make sure your portal is
configured with the authentication option you want to use. You can
choose email address, screen name, or user ID. You can set this in the
Control Panel by selecting \emph{Configuration} → \emph{Instance
Settings}, and then selecting the \emph{Authentication} section. The
authentication options are in the \emph{How do users authenticate?}
selector menu.

\begin{figure}
\centering
\includegraphics{./images/screens-portal-auth.png}
\caption{Setting the authentication method in your Liferay instance.}
\end{figure}

For more details, see the
\href{/docs/7-0/user/-/knowledge_base/u/setting-up-a-liferay-instance}{Setting
up a Liferay Instance} section of the User Guide.

\subsubsection{OAuth Authentication}\label{oauth-authentication}

\noindent\hrulefill

\textbf{Note:} OAuth authentication is only available in Liferay DXP
instances.

\noindent\hrulefill

To use OAuth authentication, first install the OAuth provider app from
the Liferay Marketplace.
\href{https://web.liferay.com/marketplace/-/mp/application/45261909}{Click
here} to get this app. Once it's installed, go to \emph{Control Panel} →
\emph{Users} → \emph{OAuth Admin}, and add a new application to be used
from Liferay Screens. Once the application exists, copy the
\emph{Consumer Key} and \emph{Consumer Secret} values for later use in
Login Screenlet.

\subsection{Offline}\label{offline}

This Screenlet doesn't support offline mode. It requires network
connectivity. If you need to log in users automatically, even when
there's no network connection, you can use the
\texttt{credentialsStorage} attribute together with the
\texttt{SessionContext.loadStoredCredentials} method.

\subsection{Required Attributes}\label{required-attributes}

\begin{itemize}
\tightlist
\item
  None
\end{itemize}

\subsection{Attributes}\label{attributes}

\noindent\hrulefill

Attribute \textbar{} Data type \textbar{} Explanation \textbar{}
\texttt{layoutId} \textbar{} \texttt{@layout} \textbar{} The ID of the
View's layout. You can set this attribute to
\texttt{@layout/login\_default} (Default View) or
\texttt{@layout/login\_material} (Material View). To use the Material
View, you must first install the Material View Set.
\href{/docs/7-0/tutorials/-/knowledge_base/t/using-views-in-android-screenlets}{Click
here} for instructions on installing and using Views and View Sets,
including the Material View Set. \textbar{} \texttt{companyId}
\textbar{} \texttt{number} \textbar{} The ID of the portal instance to
authenticate to. If you don't set this attribute or set it to
\texttt{0}, the Screenlet uses the \texttt{companyId} setting in
\texttt{LiferayServerContext}. \textbar{} \texttt{loginMode} \textbar{}
\texttt{enum} \textbar{} The Screenlet's authentication type. You can
set this attribute to \texttt{basic}, \texttt{oauth}, or
\texttt{cookie}. If you don't set this attribute, the Screenlet defaults
to basic authentication. \textbar{} \texttt{basicAuthMethod} \textbar{}
\texttt{string} \textbar{} Specifies the authentication option to use
with basic or cookie authentication. You can set this attribute to
\texttt{email}, \texttt{screenName} or \texttt{userId}. This must match
the server's authentication option. If you don't set this attribute, and
don't set the \texttt{loginMode} attribute to \texttt{oauth}, the
Screenlet defaults to basic authentication with the \texttt{email}
option. \textbar{} \texttt{OAuthConsumerKey} \textbar{} \texttt{string}
\textbar{} Specifies the \emph{Consumer Key} to use in OAuth
authentication. \textbar{} \texttt{OAuthConsumerSecret} \textbar{}
\texttt{string} \textbar{} Specifies the \emph{Consumer Secret} to use
in OAuth authentication. \textbar{} \texttt{credentialsStorage}
\textbar{} \texttt{enum} \textbar{} Sets the mode for storing user
credentials. The possible values are \texttt{none}, \texttt{auto}, and
\texttt{shared\_preferences}. If set to \texttt{shared\_preferences},
the user credentials and attributes are stored using Android's
\texttt{SharedPreferences} class. If set to \texttt{none}, user
credentials and attributes aren't saved at all. If set to \texttt{auto},
the best of the available storage modes is used. Currently, this is
equivalent to \texttt{shared\_preferences}. The default value is
\texttt{none}. \textbar{} \texttt{shouldHandleCookieExpiration}
\textbar{} \texttt{bool} \textbar~Whether to refresh the cookie
automatically when using cookie login. When set to \texttt{true} (the
default value), the cookie refreshes as it's about to expire. \textbar{}
\texttt{cookieExpirationTime} \textbar{} \texttt{int} \textbar~How long
the cookie lasts, in seconds. This value depends on your portal
instance's configuration. The default value is \texttt{900}. \textbar{}
\texttt{authenticator} \textbar{} \texttt{Authenticator} \textbar~An
instance of a class that implements the \texttt{Authenticator}
interface. The \emph{Challenge-Response Authentication} section below
discusses this further. \textbar{}

\noindent\hrulefill

\subsection{Listener}\label{listener}

The Login Screenlet delegates some events to an object that implements
the \texttt{LoginListener} interface. This interface let you implement
the following methods:

\begin{itemize}
\item
  \texttt{onLoginSuccess(User\ user)}: Called when login successfully
  completes. The \texttt{user} parameter contains a set of the logged in
  user's attributes. The supported keys are the same as those in the
  \href{https://github.com/liferay/liferay-portal/blob/master/portal-impl/src/com/liferay/portal/service.xml\#L2575-L2737}{portal's
  User entity}.
\item
  \texttt{onLoginFailure(Exception\ e)}: Called when an error occurs in
  the process.
\end{itemize}

\subsection{Challenge-Response
Authentication}\label{challenge-response-authentication}

To support
\href{https://en.wikipedia.org/wiki/Challenge\%E2\%80\%93response_authentication}{challenge-response
authentication} when using a cookie to log in to the portal, Login
Screenlet has an \texttt{authenticator} attribute. As mentioned in the
above \emph{Attributes} table, this attribute's value is a class that
implements the
\href{https://square.github.io/okhttp/3.x/okhttp/okhttp3/Authenticator.html}{\texttt{Authenticator}
interface}.

Here's an example of such a class. It sends a basic authorization in
response to an authentication challenge:

\begin{verbatim}
public class BasicAuthAutenticator extends BasicAuthentication implements Authenticator {

    public BasicAuthAutenticator(String username, String password) {
        super(username, password);
    }

    @Override
    public Request authenticate(Proxy proxy, Response response) throws IOException {
        String credential = Credentials.basic(username, password);
        return response.request().newBuilder().header(Headers.AUTHORIZATION, credential).build();
    }

    @Override
    public Request authenticateProxy(Proxy proxy, Response response) throws IOException {
        return null;
    }
}
\end{verbatim}

\section{Sign Up Screenlet for
Android}\label{sign-up-screenlet-for-android}

\subsection{Requirements}\label{requirements-1}

\begin{itemize}
\tightlist
\item
  Android SDK 4.1 (API Level 16) or above
\item
  Liferay Portal 6.2 CE/EE, Liferay CE Portal 7.0/7.1, Liferay DXP 7.0+
\end{itemize}

\subsection{Compatibility}\label{compatibility-1}

\begin{itemize}
\tightlist
\item
  Android SDK 4.1 (API Level 16) or above
\end{itemize}

\subsection{Xamarin Requirements}\label{xamarin-requirements-1}

\begin{itemize}
\tightlist
\item
  Visual Studio 7.2
\item
  Mono .NET framework 5.4.1.6
\end{itemize}

\subsection{Features}\label{features-1}

The Sign Up Screenlet creates a new user in your Liferay instance: a new
user of your app can become a new user in your portal. You can also use
this Screenlet to save new users' credentials on their devices. This
enables auto login for future sessions. The Screenlet also supports
navigation of form fields from the device's keyboard.

\subsection{JSON Services Used}\label{json-services-used-1}

Screenlets in Liferay Screens call JSON web services in the portal. This
Screenlet calls the following services and methods.

\noindent\hrulefill

\begin{longtable}[]{@{}lll@{}}
\toprule\noalign{}
Service & Method & Notes \\
\midrule\noalign{}
\endhead
\bottomrule\noalign{}
\endlastfoot
\texttt{UserService} & \texttt{addUser} & \\
\end{longtable}

\noindent\hrulefill

\subsection{Module}\label{module-1}

\begin{itemize}
\tightlist
\item
  Auth
\end{itemize}

\subsection{Views}\label{views-1}

\begin{itemize}
\tightlist
\item
  Default
\item
  Material
\end{itemize}

\begin{figure}
\centering
\includegraphics{./images/screens-android-signup.png}
\caption{The Sign Up Screenlet with the Default (left) and Material
(right) Viewsets.}
\end{figure}

\subsection{Portal Configuration}\label{portal-configuration-1}

Sign Up Screenlet's corresponding configuration in the Liferay instance
can be set in the Control Panel by selecting \emph{Configuration} →
\emph{Instance Settings}, and then selecting the \emph{Authentication}
section.

\begin{figure}
\centering
\includegraphics{./images/screens-portal-signup.png}
\caption{The Liferay instance's authentication settings.}
\end{figure}

For more details, refer to the
\href{/docs/7-0/user/-/knowledge_base/u/setting-up-a-liferay-instance}{Setting
up a Liferay Instance} section of the User Guide.

\subsection{Anonymous Requests}\label{anonymous-requests}

Anonymous requests are unauthenticated requests. Authentication is still
required, however, to call the API. To allow this operation, the portal
administrator should create a user with minimal permissions. To use Sign
Up Screenlet, you need to use that user in your layout. You should add
that user's credentials to \texttt{server\_context.xml}.

\subsection{Offline}\label{offline-1}

This Screenlet doesn't support offline mode. It requires network
connectivity.

\subsection{Required Attributes}\label{required-attributes-1}

\begin{itemize}
\tightlist
\item
  \texttt{anonymousApiUserName}
\item
  \texttt{anonymousApiPassword}
\end{itemize}

\subsection{Attributes}\label{attributes-1}

\noindent\hrulefill

Attribute \textbar{} Data type \textbar{} Explanation \textbar{}
\texttt{layoutId} \textbar{} \texttt{@layout} \textbar{} The layout used
to show the View.\textbar{} \texttt{anonymousApiUserName} \textbar{}
\texttt{string} \textbar{} The user's name, email address, or ID to use
for authenticating the request. The portal's authentication method
defines which of these is used. \textbar{} \texttt{anoymousApiPassword}
\textbar{} \texttt{string} \textbar{} The password used to authenticate
the request. \textbar{} \texttt{companyId} \textbar{} \texttt{number}
\textbar{} When set, a user in the specified company is authenticated.
If not set, the company specified in \texttt{LiferayServerContext} is
used. \textbar{} \texttt{autoLogin} \textbar{} \texttt{boolean}
\textbar{} Sets whether the user is logged in automatically after a
successful sign up. \textbar{} \texttt{credentialsStorage} \textbar{}
\texttt{enum} \textbar{} Sets the mode for storing user credentials. The
possible values are \texttt{none}, \texttt{auto}, and
\texttt{shared\_preferences}. If set to \texttt{shared\_preferences},
the user credentials and attributes are stored using Android's
\texttt{SharedPreferences} class. If set to \texttt{none}, user
credentials and attributes aren't saved at all. If set to \texttt{auto},
the best of the available storage modes is used. Currently, this is
equivalent to \texttt{shared\_preferences}. The default value is
\texttt{none}. \textbar{}
\texttt{basicAuthMethod}\textbar{}\texttt{enum}\textbar{} Specifies the
authentication method to use after a successful sign up. This must match
the authentication method configured on the server. You can set this
attribute to \texttt{email}, \texttt{screenName} or \texttt{userId}. The
default value is \texttt{email}. \textbar{}

\noindent\hrulefill

\subsection{Listener}\label{listener-1}

The Sign Up Screenlet delegates some events to an object that implements
the \texttt{SignUpListener} interface. This interface lets you implement
the following methods:

\begin{itemize}
\item
  \texttt{onSignUpSuccess(User\ user)}: Called when sign up successfully
  completes. The \texttt{user} parameter contains a set of the created
  user's attributes. The supported keys are the same as those in the
  \href{https://github.com/liferay/liferay-portal/blob/7.0.x/portal-impl/src/com/liferay/portal/service.xml\#L2686}{portal's
  User entity}.
\item
  \texttt{onSignUpFailure(Exception\ e)}: Called when an error occurs in
  the process.
\end{itemize}

\section{Forgot Password Screenlet for
Android}\label{forgot-password-screenlet-for-android}

\subsection{Requirements}\label{requirements-2}

\begin{itemize}
\tightlist
\item
  Android SDK 4.1 (API Level 16) or above
\item
  Liferay Portal 6.2 CE/EE, Liferay CE Portal 7.0/7.1, Liferay DXP 7.0+
\item
  Liferay Screens Compatibility app
  (\href{http://www.liferay.com/marketplace/-/mp/application/54365664}{CE}
  or
  \href{http://www.liferay.com/marketplace/-/mp/application/54369726}{EE/DXP}).
  This app is preinstalled in Liferay CE Portal 7.0/7.1 and Liferay DXP
  7.0+.
\end{itemize}

\subsection{Compatibility}\label{compatibility-2}

\begin{itemize}
\tightlist
\item
  Android SDK 4.1 (API Level 16) or above
\end{itemize}

\subsection{Xamarin Requirements}\label{xamarin-requirements-2}

\begin{itemize}
\tightlist
\item
  Visual Studio 7.2
\item
  Mono .NET framework 5.4.1.6
\end{itemize}

\subsection{Features}\label{features-2}

The Forgot Password Screenlet sends an email to registered users with
their new passwords or password reset links, depending on the server
configuration. The available authentication methods are

\begin{itemize}
\tightlist
\item
  Email address
\item
  Screen name
\item
  User id
\end{itemize}

\subsection{JSON Services Used}\label{json-services-used-2}

Screenlets in Liferay Screens call JSON web services in the portal. This
Screenlet calls the following services and methods.

\noindent\hrulefill

\begin{longtable}[]{@{}lll@{}}
\toprule\noalign{}
Service & Method & Notes \\
\midrule\noalign{}
\endhead
\bottomrule\noalign{}
\endlastfoot
\texttt{UserService} & \texttt{sendPasswordByEmailAddress} & \\
\texttt{UserService} & \texttt{sendPasswordByUserId} & \\
\texttt{UserService} & \texttt{sendPasswordByScreenName} & \\
\end{longtable}

\noindent\hrulefill

\subsection{Module}\label{module-2}

\begin{itemize}
\tightlist
\item
  Auth
\end{itemize}

\subsection{Views}\label{views-2}

\begin{itemize}
\tightlist
\item
  Default
\item
  Material
\end{itemize}

\begin{figure}
\centering
\includegraphics{./images/screens-android-forgotpwd.png}
\caption{The Forgot Password Screenlet with the Default (left) and
Material (right) Viewsets.}
\end{figure}

\subsection{Portal Configuration}\label{portal-configuration-2}

To use Forgot Password Screenlet, the portal must be configured to allow
users to request new passwords. The below sections show you how to do
this. Also,
\href{https://github.com/liferay/liferay-screens/tree/master/portal}{Liferay
Screens' Compatibility Plugin} must be installed.

\subsubsection{Authentication Method}\label{authentication-method}

The authentication method configured in the portal can be different from
the one used by this Screenlet. For example, it's \emph{perfectly fine}
to use \texttt{screenName} for sign in authentication, but allow users
to recover their password using the \texttt{email} authentication
method.

\subsubsection{Password Reset}\label{password-reset}

You can set the Liferay instance's corresponding password reset options
in the Control Panel by selecting \emph{Configuration} → \emph{Instance
Settings}, and then selecting the \emph{Authentication} section. The
Screenlet's password functionality depends on the authentication
settings in the portal:

\begin{figure}
\centering
\includegraphics{./images/screens-password-reset.png}
\caption{Checkboxes for the password recovery features in your Liferay
instance.}
\end{figure}

If these options are both unchecked, password recovery is disabled. If
both options are checked, an email containing a password reset link is
sent when a user requests it. If only the first option is checked, an
email containing a new password is sent when a user requests it.

For more details on authentication in Liferay Portal, please refer to
the
\href{/docs/7-0/user/-/knowledge_base/u/setting-up-a-liferay-instance}{Setting
up a Liferay Instance} section of the User Guide.

\subsubsection{Anonymous Request}\label{anonymous-request}

An anonymous request can be made without the user being logged in.
However, authentication is needed to call the API. To allow this
operation, the portal administrator should create a specific user with
minimal permissions.

\subsection{Offline}\label{offline-2}

This Screenlet doesn't support offline mode. It requires network
connectivity.

\subsection{Required Attributes}\label{required-attributes-2}

\begin{itemize}
\tightlist
\item
  \texttt{anonymousApiUserName}
\item
  \texttt{anonymousApiPassword}
\end{itemize}

\subsection{Attributes}\label{attributes-2}

\noindent\hrulefill

Attribute \textbar{} Data type \textbar{} Explanation \textbar{}
\texttt{layoutId} \textbar{} \texttt{@layout} \textbar{} The layout used
to show the View. \textbar{} \texttt{anonymousApiUserName} \textbar{}
\texttt{string} \textbar{} The user name, email address, or
\texttt{userId} to use for authenticating the request. This depends on
the portal's authentication settings. \textbar{}
\texttt{anonymousApiPassword} \textbar{} \texttt{string} \textbar{} The
password to use to authenticate the request. \textbar{}
\texttt{companyId} \textbar{} \texttt{number} \textbar{} When set, a
user within the specified company is authenticated. If the value is set
to \texttt{0}, the company specified in \texttt{LiferayServerContext} is
used. \textbar{} \texttt{basicAuthMethod} \textbar{} \texttt{string}
\textbar{} The authentication method presented to the user. This can be
\texttt{email}, \texttt{screenName}, or \texttt{userId}. The default
value is \texttt{email}. \textbar{}

\noindent\hrulefill

\subsection{Listener}\label{listener-2}

The Forgot Password Screenlet delegates some events to an object that
implements the \texttt{ForgotPasswordListener} interface. This interface
lets you implement the following methods:

\begin{itemize}
\item
  \texttt{onForgotPasswordRequestSuccess(boolean\ passwordSent)}: Called
  when a password reset email is successfully sent. The boolean
  parameter determines whether the email contains the new password or a
  password reset link.
\item
  \texttt{onForgotPasswordRequestFailure(Exception\ e)}: Called when an
  error occurs in the process.
\end{itemize}

\section{User Portrait Screenlet for
Android}\label{user-portrait-screenlet-for-android}

\subsection{Requirements}\label{requirements-3}

\begin{itemize}
\tightlist
\item
  Android SDK 4.1 (API Level 16) or above
\item
  Liferay Portal 6.2 CE/EE, Liferay CE Portal 7.0/7.1, Liferay DXP 7.0+
\item
  Picasso library
\end{itemize}

\subsection{Compatibility}\label{compatibility-3}

\begin{itemize}
\tightlist
\item
  Android SDK 4.1 (API Level 16) or above
\end{itemize}

\subsection{Xamarin Requirements}\label{xamarin-requirements-3}

\begin{itemize}
\tightlist
\item
  Visual Studio 7.2
\item
  Mono .NET framework 5.4.1.6
\end{itemize}

\subsection{Features}\label{features-3}

The User Portrait Screenlet shows the users' profile pictures. If a user
doesn't have a profile picture, a placeholder image is shown. The
Screenlet allows the profile picture to be edited via the
\texttt{editable} property.

\subsection{JSON Services Used}\label{json-services-used-3}

Screenlets in Liferay Screens call JSON web services in the portal. This
Screenlet calls the following services and methods.

\noindent\hrulefill

\begin{longtable}[]{@{}lll@{}}
\toprule\noalign{}
Service & Method & Notes \\
\midrule\noalign{}
\endhead
\bottomrule\noalign{}
\endlastfoot
\texttt{UserService} & \texttt{getUserById} & \\
\end{longtable}

\noindent\hrulefill

\subsection{Module}\label{module-3}

\begin{itemize}
\tightlist
\item
  None
\end{itemize}

\subsection{Views}\label{views-3}

\begin{itemize}
\tightlist
\item
  Default
\item
  Material
\end{itemize}

\begin{figure}
\centering
\includegraphics{./images/screens-android-userportrait.png}
\caption{The User Portrait Screenlet using the Default (left) and
Material (right) Views.}
\end{figure}

\subsection{Portal Configuration}\label{portal-configuration-3}

No additional steps required.

\subsection{Activity Configuration}\label{activity-configuration}

The User Portrait Screenlet needs the following user permissions:

\begin{verbatim}
<uses-permission android:name="android.permission.CAMERA"/>
<uses-permission android:name="android.permission.WRITE_EXTERNAL_STORAGE"/>
\end{verbatim}

\subsection{Offline}\label{offline-3}

This Screenlet supports offline mode so it can function without a
network connection. For more information on how offline mode works, see
the
\href{/docs/7-0/tutorials/-/knowledge_base/t/architecture-of-offline-mode-in-liferay-screens}{tutorial
on its architecture}.

When loading the portrait, the Screenlet supports the following offline
mode policies:

\noindent\hrulefill

Policy \textbar{} What happens \textbar{} When to use \textbar{}
\texttt{REMOTE\_ONLY} \textbar{} The Screenlet loads the user portrait
from the portal. If a connection issue occurs, the Screenlet uses the
listener to notify the developer about the error. If the Screenlet loads
the portrait, it stores the received image in the local cache for later
use. \textbar{} Use this policy when you always need to show updated
portraits, and show the default placeholder when there's no connection.
\textbar{} \texttt{CACHE\_ONLY} \textbar{} The Screenlet loads the user
portrait from the local cache. If the portrait isn't there, the
Screenlet uses the listener to notify the developer about the error.
\textbar{} Use this policy to show local portraits, without retrieving
remote information under any circumstance. \textbar{}
\texttt{REMOTE\_FIRST} \textbar{} The Screenlet loads the user portrait
from the portal. The Screenlet displays the portrait to the user and
stores it in the local cache for later use. If a connection issue
occurs, the Screenlet retrieves the portrait from the local cache. If
the portrait doesn't exist there, the Screenlet uses the listener to
notify the developer about the error. \textbar{} Use this policy to show
the most recent portrait when connected, but show a potentially outdated
version when there's no connection. \textbar{} \texttt{CACHE\_FIRST}
\textbar{} If the portrait exists in the local cache, the Screenlet
loads it from there. If it doesn't exist there, the Screenlet requests
the portrait from the portal and uses the listener to notify the
developer about any connection errors. \textbar{} Use this policy to
save bandwidth and loading time in the event a local (but probably
outdated) portrait exists. \textbar{}

\noindent\hrulefill

When editing the portrait, the Screenlet supports the following offline
mode policies:

\noindent\hrulefill

Policy \textbar{} What happens \textbar{} When to use \textbar{}
\texttt{REMOTE\_ONLY} \textbar{} The Screenlet sends the user portrait
to the portal. If a connection issue occurs, the Screenlet uses the
listener to notify the developer about the error, but it also discards
the new portrait. \textbar{} Use this policy when you need to make sure
portal always has the most recent version of the portrait. \textbar{}
\texttt{CACHE\_ONLY} \textbar{} The Screenlet stores the user portrait
in the local cache. \textbar{} Use this policy when you need to save the
portrait locally, but don't want to change the portrait in the portal.
\textbar{} \texttt{REMOTE\_FIRST} \textbar{} The Screenlet sends the
user portrait to the portal. If this succeeds, the Screenlet also stores
the portrait in the local cache for later usage. If a connection issue
occurs, the Screenlet stores the portrait in the local cache with the
\emph{dirty flag} enabled. This causes the portrait to be sent to the
portal when the synchronization process runs. \textbar{} Use this policy
when you need to make sure the Screenlet sends the new portrait to the
portal as soon as the connection is restored. \textbar{}
\texttt{CACHE\_FIRST} \textbar{} The Screenlet stores the user portrait
in the local cache and then sends it to the portal. If a connection
issue occurs, the Screenlet stores the portrait in the local cache with
the \emph{dirty flag} enabled. This causes the portrait to be sent to
the portal when the synchronization process runs. \textbar{} Use this
policy when you need to make sure the Screenlet sends the new portrait
to the portal as soon as the connection is restored. Compared to
\texttt{REMOTE\_FIRST}, this policy always stores the portrait in the
cache. The \texttt{REMOTE\_FIRST} policy only stores the new image in
the cache in the event of a network error or a successful upload.
\textbar{}

\noindent\hrulefill

\subsection{Required Attributes}\label{required-attributes-3}

\begin{itemize}
\tightlist
\item
  None
\end{itemize}

Note that if you don't set any attributes, the Screenlet loads the
logged-in user's portrait.

\subsection{Attributes}\label{attributes-3}

\noindent\hrulefill

Attribute \textbar{} Data type \textbar{} Explanation \textbar{}
\texttt{layoutId} \textbar{} \texttt{@layout} \textbar{} The layout used
to show the View. \textbar{} \texttt{autoLoad} \textbar{}
\texttt{boolean} \textbar{} Whether the portrait should load when the
Screenlet is attached to the window. \textbar{} \texttt{userId}
\textbar{} \texttt{number} \textbar{} The ID of the user whose portrait
is being requested. If this attribute is set, the \texttt{male},
\texttt{portraitId}, and \texttt{uuid} attributes are ignored.
\textbar{} \texttt{male} \textbar{} \texttt{boolean} \textbar{} Whether
the default portrait placeholder shows a male or female outline. This
attribute is used if \texttt{userId} isn't specified. \textbar{}
\texttt{portraitId} \textbar{} \texttt{number} \textbar{} The ID of the
portrait to load. This attribute is used if \texttt{userId} isn't
specified. \textbar{} \texttt{uuid} \textbar{} \texttt{string}
\textbar{} The \texttt{uuid} of the user whose portrait is being
requested. This attribute is used if \texttt{userId} isn't specified.
\textbar{} \texttt{editable} \textbar{} \texttt{boolean} \textbar{} Lets
the user change the portrait image by taking a photo or selecting a
gallery picture. \textbar{} \texttt{offlinePolicy} \textbar{}
\texttt{enum} \textbar{} Configure the loading and saving behavior in
case of connectivity issues. For more details, read the ``Offline''
section below. \textbar{}

\noindent\hrulefill

\subsection{Methods}\label{methods}

\noindent\hrulefill

Method \textbar{} Return \textbar{} Explanation \textbar{}
\texttt{load()} \textbar{} \texttt{void} \textbar{} Starts the request
to load the user specified in the \texttt{userId} property, or the
portrait specified in the \texttt{portraitId}and \texttt{uuid}
properties. \textbar{}
\texttt{upload(int\ requestCode,}\texttt{Intent\ onActivityResultData)}
\textbar{} \texttt{void} \textbar{} Starts the request to upload a
profile picture from the source specified in the \texttt{requestCode}
property (gallery or camera), and with the path stored in the
\texttt{onActivityResultData} variable. \textbar{}

\noindent\hrulefill

\subsection{Listener}\label{listener-3}

The User Portrait Screenlet delegates some events to an object that
implements the \texttt{UserPortraitListener} interface. This interface
lets you implement the following methods:

\begin{itemize}
\item
  \texttt{onUserPortraitLoadReceived(Bitmap\ bitmap)}: Called when an
  image is received from the server. You can then apply image filters
  (grayscale, for example) and return the new image. You can return
  \texttt{null} or the original image supplied as the argument if you
  don't want to modify it.
\item
  \texttt{onUserPortraitUploaded()}: Called when the user portrait
  upload service finishes.
\item
  \texttt{error(Exception\ e,\ String\ userAction)}: Called when an
  error occurs in the process. For example, an error can occur when
  receiving or uploading a user portrait. The \texttt{userAction}
  argument distinguishes the specific action in which the error
  occurred.
\end{itemize}

\section{DDL Form Screenlet for
Android}\label{ddl-form-screenlet-for-android}

\subsection{Requirements}\label{requirements-4}

\begin{itemize}
\tightlist
\item
  Android SDK 4.1 (API Level 16) or above
\item
  Liferay Portal 6.2 CE/EE, Liferay CE Portal 7.0/7.1, Liferay DXP 7.0+
\item
  Liferay Screens Compatibility app
  (\href{http://www.liferay.com/marketplace/-/mp/application/54365664}{CE}
  or
  \href{http://www.liferay.com/marketplace/-/mp/application/54369726}{EE/DXP}).
  This app is preinstalled in Liferay CE Portal 7.0/7.1 and Liferay DXP
  7.0+.
\end{itemize}

\subsection{Compatibility}\label{compatibility-4}

\begin{itemize}
\tightlist
\item
  Android SDK 4.1 (API Level 16) or above
\end{itemize}

\subsection{Xamarin Requirements}\label{xamarin-requirements-4}

\begin{itemize}
\tightlist
\item
  Visual Studio 7.2
\item
  Mono .NET framework 5.4.1.6
\end{itemize}

\subsection{Features}\label{features-4}

DDL Form Screenlet shows a set of fields that can be filled in by the
user. Initial or existing values can be shown in the fields. Fields of
the following data types are supported:

\begin{itemize}
\tightlist
\item
  \emph{Boolean}: A two state value typically represented by a checkbox.
\item
  \emph{Date}: A formatted date value. The format depends on the
  device's current locale.
\item
  \emph{Decimal, Integer, and Number}: A numeric value.
\item
  \emph{Documents \& Media}: A file stored on the device. It can be
  uploaded to a specific portal repository.
\item
  \emph{Radio}: A set of options to choose from. A single option must be
  chosen.
\item
  \emph{Select}: A selection box of options to choose from. A single
  option must be chosen.
\item
  \emph{Text}: A single line of text.
\item
  \emph{Text Area}: Multiple lines of text.
\end{itemize}

The DDL Form Screenlet also supports the following features:

\begin{itemize}
\tightlist
\item
  Stored records can support a specific workflow.
\item
  A Submit button can be shown at the end of the form.
\item
  Required values and validation for fields can be used.
\item
  Users can traverse the form fields from the keyboard.
\item
  Supports i18n in record values and labels.
\end{itemize}

There are also a few limitations that you should be aware of when using
DDL Form Screenlet. They are listed here:

\begin{itemize}
\tightlist
\item
  Nested fields in the data definition aren't supported.
\item
  Selection of multiple items in the Radio and Select data types isn't
  supported.
\end{itemize}

\subsection{JSON Services Used}\label{json-services-used-4}

Screenlets in Liferay Screens call JSON web services in the portal. This
Screenlet calls the following services and methods.

\noindent\hrulefill

\begin{longtable}[]{@{}
  >{\raggedright\arraybackslash}p{(\columnwidth - 4\tabcolsep) * \real{0.3889}}
  >{\raggedright\arraybackslash}p{(\columnwidth - 4\tabcolsep) * \real{0.3333}}
  >{\raggedright\arraybackslash}p{(\columnwidth - 4\tabcolsep) * \real{0.2778}}@{}}
\toprule\noalign{}
\begin{minipage}[b]{\linewidth}\raggedright
Service
\end{minipage} & \begin{minipage}[b]{\linewidth}\raggedright
Method
\end{minipage} & \begin{minipage}[b]{\linewidth}\raggedright
Notes
\end{minipage} \\
\midrule\noalign{}
\endhead
\bottomrule\noalign{}
\endlastfoot
\texttt{ScreensddlrecordService} (Screens compatibility plugin) &
\texttt{getDDMStructureVersion} & Load form \\
\texttt{ScreensddlrecordService} (Screens compatibility plugin) &
\texttt{getDdlRecord} & Load record \\
\texttt{DLAppService} & \texttt{addFileEntry} & Upload document \\
\texttt{DDLRecordService} & \texttt{addRecord} & Submit form \\
\texttt{DDLRecordService} & \texttt{updateRecord} & Update form \\
\end{longtable}

\noindent\hrulefill

\subsection{Module}\label{module-4}

\begin{itemize}
\tightlist
\item
  DDL
\end{itemize}

\subsection{Views}\label{views-4}

\begin{itemize}
\tightlist
\item
  Default
\item
  Material
\end{itemize}

The Default View uses a standard vertical \texttt{ScrollView} to show a
scrollable list of fields. Other Views may use different components,
such as \texttt{ViewPager} or others, to show the fields. You can find a
sample of this implementation in the \texttt{DDLFormScreenletPagerView}
class.

\begin{figure}
\centering
\includegraphics{./images/screens-android-ddlform.png}
\caption{DDL Form Screenlet's Default (left) and Material (right)
Views.}
\end{figure}

\subsubsection{Editor Types}\label{editor-types}

Each field defines an editor type. You must define each editor type's
layout by using the following attributes:

\begin{itemize}
\tightlist
\item
  \texttt{checkboxFieldLayoutId}: The layout to use for Boolean fields.
\item
  \texttt{dateFieldLayoutId}: The layout to use for Date fields.
\item
  \texttt{numberFieldLayoutId}: The layout to use for Number, Decimal,
  or Integer fields.
\item
  \texttt{radioFieldLayoutId}: The layout to use for Radio fields.
\item
  \texttt{selectFieldLayoutId}: The layout to use for Select fields.
\item
  \texttt{textFieldLayoutId}: The layout to use for Text fields.
\item
  \texttt{textAreaFieldLayoutId}: The layout to use for Text Box fields.
\item
  \texttt{textDocumentFieldLayoutId}: The layout to use for Documents \&
  Media fields.
\end{itemize}

If you don't define the editor type's layout in DDL Form Screenlet's
attributes, the default layout \texttt{ddlfield\_xxx\_default} is used,
where \texttt{xxx} is the name of the editor type. It's important to
note that you can change the layout used with any editor type at any
point.

\subsubsection{Custom Editors}\label{custom-editors}

If you want to have a unique appearance for one specific field, you can
customize your field's editor View by calling the Screenlet's
\texttt{setCustomFieldLayoutId(fieldName,\ layoutId)} method, where the
first parameter is the name of the field to customize and the second
parameter is the layout to use. You can also create custom editor Views.
For examples of this, see the files
\texttt{ddlfield\_custom\_rating\_number.xml} and
\texttt{CustomRatingNumberView.java}.

\subsection{Activity Configuration}\label{activity-configuration-1}

DDL Form Screenlet needs the following user permissions:

\begin{verbatim}
<uses-permission android:name="android.permission.CAMERA"/>
<uses-permission android:name="android.permission.WRITE_EXTERNAL_STORAGE"/>
\end{verbatim}

Both are used by the Documents and Media fields to take a picture/video
and store it locally before uploading it to the portal. The Documents
and Media fields also need to override the \texttt{onActivityResult}
method to receive the picture/video information. Here's an example
implementation:

\begin{verbatim}
@Override
protected void onActivityResult(int requestCode, int resultCode, Intent data) {
    super.onActivityResult(requestCode, resultCode, data);

    screenlet.startUploadByPosition(requestCode);
}
\end{verbatim}

\subsection{Portal Configuration}\label{portal-configuration-4}

Before using DDL Form Screenlet, you should make sure that Dynamic Data
Lists and Data Types are configured properly in the portal. Refer to the
\href{/docs/7-0/user/-/knowledge_base/u/creating-data-definitions}{Creating
Data Definitions} and
\href{/docs/7-0/user/-/knowledge_base/u/creating-data-lists}{Creating
Data Lists} sections of the User Guide for more details. If Workflow is
required, it must also be configured. See the
\href{/docs/7-0/user/-/knowledge_base/u/using-workflow}{Using Workflow}
section of the User Guide for details.

\subsubsection{Permissions}\label{permissions}

To use DDL Form Screenlet to add new records, you must grant the Add
Record permission in the Dynamic Data List in the portal. If you want to
use DDL Form Screenlet to view or edit record values, you must also
grant the View and Update permissions, respectively. The Add Record,
View, and Update permissions are highlighted by the red boxes in the
following screenshot:

\begin{figure}
\centering
\includegraphics{./images/screens-portal-permission-ddl.png}
\caption{The permissions for adding, viewing, and editing DDL records.}
\end{figure}

Also, if your form includes at least one Documents and Media field, you
must grant permissions in the target repository and folder. For more
details, see the \texttt{repositoryId} and \texttt{folderId} attributes
below.

\begin{figure}
\centering
\includegraphics{./images/screens-portal-permission-folder-add.png}
\caption{The permission for adding a document to a Documents and Media
folder.}
\end{figure}

For more details, see the User Guide sections
\href{/docs/7-0/user/-/knowledge_base/u/creating-data-definitions}{Creating
Data Definitions},
\href{/docs/7-0/user/-/knowledge_base/u/creating-data-lists}{Creating
Data Lists}, and
\href{/docs/7-0/user/-/knowledge_base/u/using-workflow}{Using Workflow}.

\subsection{Offline}\label{offline-4}

This Screenlet supports offline mode so it can function without a
network connection. For more information on how offline mode works, see
the
\href{/docs/7-0/tutorials/-/knowledge_base/t/architecture-of-offline-mode-in-liferay-screens}{tutorial
on its architecture}.

When loading the form or record, the Screenlet supports the following
offline mode policies:

\noindent\hrulefill

Policy \textbar{} What happens \textbar{} When to use \textbar{}
\texttt{REMOTE\_ONLY} \textbar{} The Screenlet loads the form or record
from the portal. If a connection issue occurs, the Screenlet uses the
listener to notify the developer about the error. If the Screenlet loads
the form or record, it stores the received data (record structure and
data) in the local cache for later use. \textbar{} Use this policy when
you always need to show updated data, and show nothing when there's no
connection. \textbar{} \texttt{CACHE\_ONLY} \textbar{} The Screenlet
loads the form or record from the local cache. If the form or record
isn't there, the Screenlet uses the listener to notify the developer
about the error. \textbar{} Use this policy when you always need to show
local data, without retrieving remote information under any
circumstance. \textbar{} \texttt{REMOTE\_FIRST} \textbar{} The Screenlet
requests the form or record from the portal. The Screenlet shows the
record or form to the user and stores it in the local cache for later
use. If a connection issue occurs, the Screenlet retrieves the form or
record from the local cache. If the form or record doesn't exist there,
the Screenlet uses the listener to notify the developer about the error.
\textbar{} Use this policy to show the most recent version of the data
when connected, but show an outdated version when there's no connection.
\textbar{} \texttt{CACHE\_FIRST} \textbar{} If the form or record exists
in the local cache, the Screenlet loads it from there. If it doesn't
exist there, the Screenlet requests it from the portal and notifies the
developer about any errors that occur (including connectivity errors).
\textbar{} Use this policy to save bandwidth and loading time in case
you have local (but probably outdated) data. \textbar{}

\noindent\hrulefill

When editing the record, the Screenlet supports the following offline
mode policies:

\noindent\hrulefill

Policy \textbar{} What happens \textbar{} When to use \textbar{}
\texttt{REMOTE\_ONLY} \textbar{} The Screenlet sends the record to the
portal. If a connection issue occurs, the Screenlet uses the listener to
notify the developer about the error, but it also discards the record.
\textbar{} Use this policy to make sure the portal always has the most
recent version of the record. \textbar{} \texttt{CACHE\_ONLY} \textbar{}
The Screenlet stores the record in the local cache. \textbar{} Use this
policy when you need to save the data locally, but don't want to update
the data in the portal (update or add record). \textbar{}
\texttt{REMOTE\_FIRST} \textbar{} The Screenlet sends the record to the
portal. If this succeeds, it also stores the record in the local cache
for later usage. If a connection issue occurs, then Screenlet stores the
record in the local cache with the \emph{dirty flag} enabled. This
causes the synchronization process to send the record to the portal when
it runs. \textbar{} Use this policy when you need to make sure the
Screenlet sends the record to the portal as soon as the connection is
restored. \textbar{} \texttt{CACHE\_FIRST} \textbar{} The Screenlet
stores the record in the local cache and then sends it to the remote
portal. If a connection issue occurs, then Screenlet stores the record
in the local cache with the \emph{dirty flag} enabled. This causes the
the synchronization process to send the record to the portal when it
runs. \textbar{} Use this policy when you need to make sure the
Screenlet sends the record to the portal as soon as the connection is
restored. Compared to \texttt{REMOTE\_FIRST}, this policy always stores
the record in the cache. The \texttt{REMOTE\_FIRST} policy only stores
the record in the event of a network error. \textbar{}

\noindent\hrulefill

\subsection{Required Attributes}\label{required-attributes-4}

\begin{itemize}
\tightlist
\item
  \texttt{structureId}
\item
  \texttt{recordSetId}
\end{itemize}

\subsection{Attributes}\label{attributes-4}

\noindent\hrulefill

Attribute \textbar{} Data Type \textbar{} Explanation \textbar{}
\texttt{layoutId} \textbar{} \texttt{@layout} \textbar{} The layout to
use to show the View. \textbar{} \texttt{checkboxFieldLayoutId}
\textbar{} \texttt{@layout} \textbar{} The layout to use to show the
view for Boolean fields. \textbar{} \texttt{dateFieldLayoutId}
\textbar{} \texttt{@layout} \textbar{} The layout to use to show the
view for Date fields. \textbar{} \texttt{numberFieldLayoutId} \textbar{}
\texttt{@layout} \textbar{} The layout to use to show the view for
Number, Decimal, and Integer fields. \textbar{}
\texttt{radioFieldLayoutId} \textbar{} \texttt{@layout} \textbar{} The
layout to use to show the view for Radio fields. \textbar{}
\texttt{selectFieldLayoutId} \textbar{} \texttt{@layout} \textbar{} The
layout to use to show the view for Select fields. \textbar{}
\texttt{textFieldLayoutId} \textbar{} \texttt{@layout} \textbar{} The
layout to use to show the view for Text fields. \textbar{}
\texttt{textAreaFieldLayoutId} \textbar{} \texttt{@layout} \textbar{}
The layout to use to show the view for Text Box fields. \textbar{}
\texttt{documentFieldLayoutId} \textbar{} \texttt{@layout} \textbar{}
The layout to use to show the view for Documents \& Media fields.
\textbar{} \texttt{structureId} \textbar{} \texttt{number} \textbar{}
The ID of a data definition in your Liferay site. To find the IDs for
your data definitions, click \emph{Admin} → \emph{Content} from the
Dockbar. Then click \emph{Dynamic Data Lists} on the left and click the
\emph{Manage Data Definitions} button. The ID of each data definition is
in the ID column of the table. \textbar{} \texttt{groupId} \textbar{}
\texttt{number} \textbar{} The ID of the site (group) where the record
is stored. If this value is \texttt{0}, the \texttt{groupId} specified
in \texttt{LiferayServerContext} is used. \textbar{}
\texttt{recordSetId} \textbar{} \texttt{number} \textbar{} A dynamic
data list's ID. To find your dynamic data lists' IDs, click \emph{Admin}
→ \emph{Content} from the Dockbar. Then click \emph{Dynamic Data Lists}
on the left. Each dynamic data list's ID is in the ID column of the
table. \textbar{} \texttt{recordId} \textbar{} \texttt{number}
\textbar{} The ID of the record you want to show. You can also allow the
record's values to be edited. This ID can be obtained from other methods
or listeners. \textbar{} \texttt{repositoryId} \textbar{}
\texttt{number} \textbar{} The ID of the Documents and Media repository
to upload to. If this value is \texttt{0}, the default repository for
the site specified by \texttt{groupId} is used. \textbar{}
\texttt{folderId} \textbar{} \texttt{number} \textbar{} The ID of the
folder where Documents and Media files are uploaded. If this value is
\texttt{0}, the root is used. \textbar{} \texttt{filePrefix} \textbar{}
\texttt{string} \textbar{} The prefix to attach to the names of files
uploaded to a Documents and Media repository. The upload date followed
by the original file name is appended following the prefix. \textbar{}
\texttt{autoLoad} \textbar{} \texttt{boolean} \textbar{} Sets whether
the form loads when the Screenlet is shown. If \texttt{recordId} is set,
the record value is loaded together with the form definition. The
default value is \texttt{false}. \textbar{}
\texttt{autoScrollOnValidation} \textbar{} \texttt{boolean} \textbar{}
Sets whether the form automatically scrolls to the first failed field
when validation is used. The default value is \texttt{true}. \textbar{}
\texttt{showSubmitButton} \textbar{} \texttt{boolean} \textbar{} Sets
whether the form shows a submit button at the bottom. If this is set to
\texttt{false}, you should call the \texttt{submitForm()} method. The
default value is \texttt{true}. \textbar{} \texttt{cachePolicy}
\textbar{} \texttt{string} \textbar{} The offline mode setting. See the
\href{/docs/7-0/reference/-/knowledge_base/r/ddlformscreenlet-for-android\#offline}{Offline
section} for details. \textbar{}

\noindent\hrulefill

\subsection{Methods}\label{methods-1}

\noindent\hrulefill

Method \textbar{} Return Type \textbar{} Explanation \textbar{}
\texttt{loadForm()} \textbar{} \texttt{void} \textbar{} Starts the
request to load the form definition. The form fields are shown when the
response is received. \textbar{} \texttt{loadRecord()} \textbar{}
\texttt{void} \textbar{} Starts the request to load the record specified
by \texttt{recordId}. If needed, the form definition also loads. When
the response is received, the form fields are shown filled with record
values. \textbar{} \texttt{load()} \textbar{} \texttt{void} \textbar{}
Starts the request to load the record if \texttt{recordId} is specified.
Otherwise, the form definition is loaded. \textbar{}
\texttt{submitForm()} \textbar{} \texttt{void} \textbar{} Starts the
request to submit form values to the dynamic data list specified by
\texttt{recordSetId}. If the record is new, a new record is added. If
\texttt{loadRecord} is used to retrieve the record, or the record
already exists, its values are updated. Fields are validated prior to
the request. If validation fails, the validation errors are shown and
the request is terminated. \textbar{}

\noindent\hrulefill

\subsection{Listener}\label{listener-4}

DDL Form Screenlet delegates some events to an object that implements to
the \texttt{DDLFormListener} interface. This interface lets you
implement the following methods:

\begin{itemize}
\item
  \texttt{onDDLFormLoaded(Record\ record)}: Called when the form
  definition successfully loads.
\item
  \texttt{onDDLFormRecordLoaded(Record\ record,\ Map\textless{}String,\ Object\textgreater{}\ valuesAndAttributes)}:
  Called when the form record data successfully loads.
\item
  \texttt{onDDLFormRecordAdded(Record\ record)}: Called when the form
  record is successfully added.
\item
  \texttt{onDDLFormRecordUpdated(Record\ record)}: Called when the form
  record data successfully updates.
\item
  \texttt{error(Exception\ e,\ String\ userAction)}: Called when an
  error occurs in the process. For example, this method is called when
  an error occurs while loading a form definition or record, or adding
  or updating a record. The \texttt{userAction} variable distinguishes
  these events.
\item
  \texttt{onDDLFormDocumentUploaded(DocumentField\ field)}: Called when
  a specified document field's upload completes.
\item
  \texttt{onDDLFormDocumentUploadFailed(DocumentField\ field,\ Exception\ e)}:
  Called when a specified document field's upload fails.
\end{itemize}

\section{DDL List Screenlet for
Android}\label{ddl-list-screenlet-for-android}

\subsection{Requirements}\label{requirements-5}

\begin{itemize}
\tightlist
\item
  Android SDK 4.1 (API Level 16) or above
\item
  Liferay Portal 6.2 CE/EE, Liferay CE Portal 7.0/7.1, Liferay DXP 7.0+
\item
  Liferay Screens Compatibility app
  (\href{http://www.liferay.com/marketplace/-/mp/application/54365664}{CE}
  or
  \href{http://www.liferay.com/marketplace/-/mp/application/54369726}{EE/DXP}).
  This app is preinstalled in Liferay CE Portal 7.0/7.1 and Liferay DXP
  7.0+.
\end{itemize}

\subsection{Compatibility}\label{compatibility-5}

\begin{itemize}
\tightlist
\item
  Android SDK 4.1 (API Level 16) or above
\end{itemize}

\subsection{Xamarin Requirements}\label{xamarin-requirements-5}

\begin{itemize}
\tightlist
\item
  Visual Studio 7.2
\item
  Mono .NET framework 5.4.1.6
\end{itemize}

\subsection{Features}\label{features-5}

The DDL List Screenlet has the following features:

\begin{itemize}
\tightlist
\item
  Shows a scrollable collection of Dynamic Data List (DDL) records.
\item
  Implements
  \href{http://www.iosnomad.com/blog/2014/4/21/fluent-pagination}{fluent
  pagination} with configurable page size.
\item
  Allows record filtering by creator.
\item
  Supports i18n in record values.
\end{itemize}

\subsection{JSON Services Used}\label{json-services-used-5}

Screenlets in Liferay Screens call JSON web services in the portal. This
Screenlet calls the following services and methods.

\noindent\hrulefill

\begin{longtable}[]{@{}
  >{\raggedright\arraybackslash}p{(\columnwidth - 4\tabcolsep) * \real{0.3889}}
  >{\raggedright\arraybackslash}p{(\columnwidth - 4\tabcolsep) * \real{0.3333}}
  >{\raggedright\arraybackslash}p{(\columnwidth - 4\tabcolsep) * \real{0.2778}}@{}}
\toprule\noalign{}
\begin{minipage}[b]{\linewidth}\raggedright
Service
\end{minipage} & \begin{minipage}[b]{\linewidth}\raggedright
Method
\end{minipage} & \begin{minipage}[b]{\linewidth}\raggedright
Notes
\end{minipage} \\
\midrule\noalign{}
\endhead
\bottomrule\noalign{}
\endlastfoot
\texttt{ScreensddlrecordService} (Screens compatibility plugin) &
\texttt{getDdlRecords} & With \texttt{ddlRecordSetId}, or
\texttt{ddlRecordSetId} and \texttt{userId} \\
\texttt{ScreensddlrecordService} (Screens compatibility plugin) &
\texttt{getDdlRecordsCount} & \\
\end{longtable}

\noindent\hrulefill

\subsection{Module}\label{module-5}

\begin{itemize}
\tightlist
\item
  DDL
\end{itemize}

\subsection{Views}\label{views-5}

\begin{itemize}
\tightlist
\item
  Default
\item
  Material
\end{itemize}

The Default View uses a standard \texttt{RecyclerView} to show the
scrollable list. Other Views may use a different component, such as
\texttt{ViewPager} or others, to show the items.

\begin{figure}
\centering
\includegraphics{./images/screens-android-ddllist.png}
\caption{The DDL List Screenlet using the Default and Material Views.}
\end{figure}

\subsection{Portal Configuration}\label{portal-configuration-5}

DDLs and Data Types should be configured in the portal before using DDL
List Screenlet. For more details, see the Liferay User Guide sections
\href{/docs/7-0/user/-/knowledge_base/u/creating-data-definitions}{Creating
Data Definitions} and
\href{/docs/7-0/user/-/knowledge_base/u/creating-data-lists}{Creating
Data Lists} .

Also, to allow remote calls without the \texttt{userId}, the Liferay
Screens Compatibility app must be installed in your Liferay instance.
You can find this app on
\href{https://web.liferay.com/marketplace}{Liferay Marketplace}.

\subsection{Offline}\label{offline-5}

This Screenlet supports offline mode so it can function without a
network connection. For more information on how offline mode works, see
the
\href{/docs/7-0/tutorials/-/knowledge_base/t/architecture-of-offline-mode-in-liferay-screens}{tutorial
on its architecture}.

\noindent\hrulefill

Policy \textbar{} What happens \textbar{} When to use \textbar{}
\texttt{REMOTE\_ONLY} \textbar{} The Screenlet loads the list from the
portal. If a connection issue occurs, the Screenlet uses the listener to
notify the developer about the error. If the Screenlet successfully
loads the list, it stores the data in the local cache for later use.
\textbar{} Use this policy when you always need to show updated data,
and show nothing when there's no connection. \textbar{}
\texttt{CACHE\_ONLY} \textbar{} The Screenlet loads the list from the
local cache. If the list isn't there, the Screenlet uses the listener to
notify the developer about the error. \textbar{} Use this policy when
you always need to show local data, without retrieving remote
information under any circumstance. \textbar{} \texttt{REMOTE\_FIRST}
\textbar{} The Screenlet loads the list from the portal. If this
succeeds, the Screenlet shows the list to the user and stores it in the
local cache for later use. If a connection issue occurs, the Screenlet
retrieves the list from the local cache. If the list doesn't exist
there, the Screenlet uses the listener to notify the developer about the
error. \textbar{} Use this policy to show the most recent version of the
data when connected, but show an outdated version when there's no
connection. \textbar{} \texttt{CACHE\_FIRST} \textbar{} The Screenlet
loads the list from the local cache. If the list isn't there, the
Screenlet requests it from the portal and notifies the developer about
any errors that occur (including connectivity errors). \textbar{} Use
this policy to save bandwidth and loading time in case you have local
(but probably outdated) data. \textbar{}

\noindent\hrulefill

\subsection{Required Attributes}\label{required-attributes-5}

\begin{itemize}
\tightlist
\item
  \texttt{recordSetId}
\item
  \texttt{labelFields}
\end{itemize}

\subsection{Attributes}\label{attributes-5}

\noindent\hrulefill

Attribute \textbar{} Data type \textbar{} Explanation \textbar{}
\texttt{layoutId} \textbar{} \texttt{@layout} \textbar{} The layout to
use to show the View. \textbar{} \texttt{autoLoad} \textbar{}
\texttt{boolean} \textbar{} Defines whether the list should be loaded
when it's presented on the screen. The default value is \texttt{true}.
\textbar{} \texttt{recordSetId} \textbar{} \texttt{number} \textbar{}
The ID of the DDL being called. To find your DDLs' IDs, click
\emph{Admin} → \emph{Content} from the Dockbar. Then click \emph{Dynamic
Data Lists} on the left. Each DDL's ID is in the ID column of the table.
\textbar{} \texttt{userId} \textbar{} \texttt{number} \textbar{} The ID
of the user to filter records on. Records aren't filtered if the
\texttt{userId} is \texttt{0}. The default value is \texttt{0}.
\textbar{} \texttt{cachePolicy} \textbar{} \texttt{string} \textbar{}
The offline mode setting. See the
\href{/docs/7-0/reference/-/knowledge_base/r/ddllistscreenlet-for-android\#offline}{Offline
section} for details. \textbar{} \texttt{firstPageSize} \textbar{}
\texttt{number} \textbar{} The number of items to retrieve from the
server for display on the first page. The default value is \texttt{50}.
\textbar{} \texttt{pageSize} \textbar{} \texttt{number} \textbar{} The
number of items to retrieve from the server for display on the second
and subsequent pages. The default value is \texttt{25}. \textbar{}
\texttt{labelFields} \textbar{} \texttt{string} \textbar{} The
comma-separated names of the DDL fields to show. Refer to the list's
data definition to find the field names. For more information on this,
see
\href{/docs/7-0/user/-/knowledge_base/u/creating-data-definitions}{Creating
Data Definitions}. Note that the appearance of these values in your app
depends on the \texttt{layoutId} set. \textbar{} \texttt{obcClassName}
\textbar{} \texttt{string} \textbar{} The name of the
\texttt{OrderByComparator} class to use to sort the results. Omit this
property if you don't want to sort the results.
\href{https://github.com/liferay/liferay-portal/tree/master/modules/apps/forms-and-workflow/dynamic-data-lists/dynamic-data-lists-api/src/main/java/com/liferay/dynamic/data/lists/util/comparator}{Click
here} to see some comparator classes. Note, however, that not all of
these classes can be used with \texttt{obcClassName}. You can only use
comparator classes that extend
\texttt{OrderByComparator\textless{}DDLRecord\textgreater{}}. You can
also create your own comparator classes that extend
\texttt{OrderByComparator\textless{}DDLRecord\textgreater{}}. \textbar{}

\noindent\hrulefill

\subsection{Methods}\label{methods-2}

\noindent\hrulefill

Method \textbar{} Return \textbar{} Explanation \textbar{}
\texttt{loadPage(pageNumber)} \textbar{} \texttt{void} \textbar{} Starts
the request to load the specified page of records. The page is shown
when the response is received. \textbar{}

\noindent\hrulefill

\subsection{Listener}\label{listener-5}

DDL List Screenlet delegates some events to an object or a class that
implements
\href{https://github.com/liferay/liferay-screens/blob/master/android/library/src/main/java/com/liferay/mobile/screens/base/list/BaseListListener.java}{the
\texttt{BaseListListener} interface}. This interface lets you implement
the following methods:

\begin{itemize}
\item
  \texttt{onListPageFailed(int\ startRow,\ Exception\ e)}: Called when
  the server call to retrieve a page of items fails. This method's
  arguments include the \texttt{Exception} generated when the server
  call fails.
\item
  \texttt{onListPageReceived(int\ startRow,\ int\ endRow,\ List\textless{}Record\textgreater{}\ records,\ int\ rowCount)}:
  Called when the server call to retrieve a page of items succeeds. Note
  that this method may be called more than once; once for each page
  received. Because \texttt{startRow} and \texttt{endRow} change for
  each page, a \texttt{startRow} of \texttt{0} corresponds to the first
  item on the first page.
\item
  \texttt{onListItemSelected(Record\ records,\ View\ view)}: Called when
  an item is selected in the list. This method's arguments include the
  selected list item (\texttt{Record}).
\item
  \texttt{error(Exception\ e,\ String\ userAction)}: Called when an
  error occurs in the process. The \texttt{userAction} argument
  distinguishes the specific action in which the error occurred.
\end{itemize}

\section{Asset List Screenlet for
Android}\label{asset-list-screenlet-for-android}

\subsection{Requirements}\label{requirements-6}

\begin{itemize}
\tightlist
\item
  Android SDK 4.1 (API Level 16) or above
\item
  Liferay Portal 6.2 CE/EE, Liferay CE Portal 7.0/7.1, Liferay DXP 7.0+
\item
  Liferay Screens Compatibility app
  (\href{http://www.liferay.com/marketplace/-/mp/application/54365664}{CE}
  or
  \href{http://www.liferay.com/marketplace/-/mp/application/54369726}{EE/DXP}).
  This app is preinstalled in Liferay CE Portal 7.0/7.1 and Liferay DXP
  7.0+.
\end{itemize}

\subsection{Compatibility}\label{compatibility-6}

\begin{itemize}
\tightlist
\item
  Android SDK 4.1 (API Level 16) or above
\end{itemize}

\subsection{Xamarin Requirements}\label{xamarin-requirements-6}

\begin{itemize}
\tightlist
\item
  Visual Studio 7.2
\item
  Mono .NET framework 5.4.1.6
\end{itemize}

\subsection{Features}\label{features-6}

The Asset List Screenlet can be used to show
\href{/tutorials/-/knowledge_base/7-0/asset-framework}{asset} lists from
a Liferay instance. For example, you can use the Screenlet to show a
scrollable list of assets. It also implements
\href{http://www.iosnomad.com/blog/2014/4/21/fluent-pagination}{fluent
pagination} with configurable page size. The Asset List Screenlet can
show assets belonging to the following classes:

\begin{itemize}
\tightlist
\item
  \texttt{BlogsEntry}
\item
  \texttt{BookmarksEntry}
\item
  \texttt{BookmarksFolder}
\item
  \texttt{CalendarEvent}
\item
  \texttt{DLFileEntry}
\item
  \texttt{DDLRecord}
\item
  \texttt{DDLRecordSet}
\item
  \texttt{Group}
\item
  \texttt{JournalArticle} (Web Content)
\item
  \texttt{JournalFolder}
\item
  \texttt{Layout}
\item
  \texttt{LayoutRevision}
\item
  \texttt{MBThread}
\item
  \texttt{MBCategory}
\item
  \texttt{MBDiscussion}
\item
  \texttt{MBMailingList}
\item
  \texttt{Organization}
\item
  \texttt{User}
\item
  \texttt{WikiPage}
\item
  \texttt{WikiPageResource}
\item
  \texttt{WikiNode}
\end{itemize}

The Asset List Screenlet also supports i18n in asset values.

\subsection{JSON Services Used}\label{json-services-used-6}

Screenlets in Liferay Screens call JSON web services in the portal. This
Screenlet calls the following services and methods.

\noindent\hrulefill

\begin{longtable}[]{@{}
  >{\raggedright\arraybackslash}p{(\columnwidth - 4\tabcolsep) * \real{0.3889}}
  >{\raggedright\arraybackslash}p{(\columnwidth - 4\tabcolsep) * \real{0.3333}}
  >{\raggedright\arraybackslash}p{(\columnwidth - 4\tabcolsep) * \real{0.2778}}@{}}
\toprule\noalign{}
\begin{minipage}[b]{\linewidth}\raggedright
Service
\end{minipage} & \begin{minipage}[b]{\linewidth}\raggedright
Method
\end{minipage} & \begin{minipage}[b]{\linewidth}\raggedright
Notes
\end{minipage} \\
\midrule\noalign{}
\endhead
\bottomrule\noalign{}
\endlastfoot
\texttt{ScreensddlrecordService} (Screens compatibility plugin) &
\texttt{getAssetEntries} & With \texttt{entryQuery} \\
\texttt{ScreensddlrecordService} (Screens compatibility plugin) &
\texttt{getAssetEntries} & With \texttt{companyId}, \texttt{groupId},
and \texttt{portletItemName} \\
\texttt{AssetEntryService} & \texttt{getEntriesCount} & \\
\end{longtable}

\noindent\hrulefill

\subsection{Module}\label{module-6}

\begin{itemize}
\tightlist
\item
  None
\end{itemize}

\subsection{Views}\label{views-6}

\begin{itemize}
\tightlist
\item
  Default
\item
  Material
\end{itemize}

The Default Views use a standard \texttt{RecyclerView} to show the
scrollable list. Other Views may use a different component, such as
\texttt{ViewPager} or others, to show the items.

\begin{figure}
\centering
\includegraphics{./images/screens-android-assetlist.png}
\caption{Asset List Screenlet using the Default (left) and Material
(right) Views.}
\end{figure}

\subsection{Portal Configuration}\label{portal-configuration-6}

Dynamic Data Lists (DDL) and Data Types should be configured properly in
the portal. Refer to the
\href{/docs/7-0/user/-/knowledge_base/u/creating-data-definitions}{Creating
Data Definitions}\\
and
\href{/docs/7-0/user/-/knowledge_base/u/creating-data-lists}{Creating
Data Lists} sections of the User Guide for more details.

Also, to allow remote calls without the \texttt{userId}, the Liferay
Screens Compatibility app must be installed in your Liferay instance.
You can find this app on
\href{https://web.liferay.com/marketplace}{Liferay Marketplace}.

\subsection{Offline}\label{offline-6}

This Screenlet supports offline mode so it can function without a
network connection. For more information on how offline mode works, see
the
\href{/docs/7-0/tutorials/-/knowledge_base/t/architecture-of-offline-mode-in-liferay-screens}{tutorial
on its architecture}.

\noindent\hrulefill

Policy \textbar{} What happens \textbar{} When to use \textbar{}
\texttt{REMOTE\_ONLY} \textbar{} The Screenlet loads the list from the
portal. If a connection issue occurs, the Screenlet uses the listener to
notify the developer about the error. If the Screenlet successfully
loads the list, it stores the data in the local cache for later use.
\textbar{} Use this policy when you always need to show updated data,
and show nothing when there's no connection. \textbar{}
\texttt{CACHE\_ONLY} \textbar{} The Screenlet loads the list from the
local cache. If the list isn't there, the Screenlet uses the listener to
notify the developer about the error. \textbar{} Use this policy when
you always need to show local data, without retrieving remote
information under any circumstance. \textbar{} \texttt{REMOTE\_FIRST}
\textbar{} The Screenlet loads the list from the portal. If this
succeeds, the Screenlet shows the list to the user and stores it in the
local cache for later use. If a connection issue occurs, the Screenlet
retrieves the list from the local cache. If the list doesn't exist
there, the Screenlet uses the listener to notify the developer about the
error. \textbar{} Use this policy to show the most recent version of the
data when connected, but show an outdated version when there's no
connection. \textbar{} \texttt{CACHE\_FIRST} \textbar{} The Screenlet
loads the list from the local cache. If the list isn't there, the
Screenlet requests it from the portal and notifies the developer about
any errors that occur (including connectivity errors). \textbar{} Use
this policy to save bandwidth and loading time in case you have local
(but probably outdated) data. \textbar{}

\noindent\hrulefill

\subsection{Required Attributes}\label{required-attributes-6}

\begin{itemize}
\tightlist
\item
  \texttt{classNameId}
\end{itemize}

If you don't set \texttt{classNameId}, you must set this attribute
instead:

\begin{itemize}
\tightlist
\item
  \texttt{portletItemName}
\end{itemize}

\subsection{Attributes}\label{attributes-6}

\noindent\hrulefill

Attribute \textbar{} Data type \textbar{} Explanation \textbar{}
\texttt{layoutId} \textbar{} \texttt{@layout} \textbar{} The layout to
use to show the View.\textbar{} \texttt{autoLoad} \textbar{}
\texttt{boolean} \textbar{} Whether the list should be loaded when it's
presented on the screen. The default value is \texttt{true}. \textbar{}
\texttt{groupId} \textbar{} \texttt{number} \textbar{} The asset's group
(site) ID. If this value is \texttt{0}, the \texttt{groupId} specified
in \texttt{LiferayServerContext} is used. The default value is
\texttt{0}. \textbar{} \texttt{cachePolicy} \textbar{} \texttt{string}
\textbar{} The offline mode setting. See the
\href{/docs/7-0/reference/-/knowledge_base/r/assetlistscreenlet-for-android\#offline}{Offline
section} for details. \textbar{} \texttt{portletItemName} \textbar{}
\texttt{string} \textbar{} The name of the
\href{/docs/7-0/user/-/knowledge_base/u/configuration-templates}{configuration
template} you used in the Asset Publisher. To use this feature, add an
Asset Publisher to one of your site's pages (it may be a hidden page),
configure the Asset Publisher's filter (in \emph{Configuration} →
\emph{Setup} → \emph{Asset Selection}), and then use the Asset
Publisher's \emph{Configuration Templates} option to save this
configuration with a name. Use this name in this attribute. \textbar{}
\texttt{classNameId} \textbar{} \texttt{number} \textbar{} The asset
class name's ID. Use values from the portal's \texttt{classname\_}
database table. \textbar{} \texttt{firstPageSize} \textbar{}
\texttt{number} \textbar{} The number of items to retrieve from the
server for display on the list's first page. The default value is
\texttt{50}. \textbar{} \texttt{pageSize} \textbar{} \texttt{number}
\textbar{} The number of items to retrieve from the server for display
on the second and subsequent pages. The default value is \texttt{25}.
\textbar{} \texttt{labelFields} \textbar{} \texttt{string} \textbar{}
The comma-separated names of the DDL fields to show. Refer to the list's
data definition to find the field names. For more information on this,
see
\href{/docs/7-0/user/-/knowledge_base/u/creating-data-definitions}{Creating
Data Definitions}. Note that the appearance of these values in your app
depends on the \texttt{layoutId} set. \textbar{}
\texttt{customEntryQuery} \textbar{} \texttt{HashMap} \textbar{} The set
of keys (string) and values (string or number) to be used in the
\href{@platform-ref@/7.0-latest/javadocs/portal-kernel/com/liferay/asset/kernel/service/persistence/AssetEntryQuery.html}{\texttt{AssetEntryQuery}
object}. These values filter the assets returned by the Liferay
instance. \textbar{}

\noindent\hrulefill

\subsection{Methods}\label{methods-3}

\noindent\hrulefill

Method \textbar{} Return \textbar{} Explanation \textbar{}
\texttt{loadPage(pageNumber)} \textbar{} \texttt{void} \textbar{} Starts
the request to load the specified page of assets. The page is shown when
the response is received. \textbar{}

\noindent\hrulefill

\subsection{Listener}\label{listener-6}

Asset List Screenlet delegates some events to an object or a class that
implements
\href{https://github.com/liferay/liferay-screens/blob/master/android/library/src/main/java/com/liferay/mobile/screens/base/list/BaseListListener.java}{the
\texttt{BaseListListener} interface}. This interface lets you implement
the following methods:

\begin{itemize}
\item
  \texttt{onListPageFailed(int\ startRow,\ Exception\ e)}: Called when
  the server call to retrieve a page of items fails. This method's
  arguments include the \texttt{Exception} generated when the server
  call fails.
\item
  \texttt{onListPageReceived(int\ startRow,\ int\ endRow,\ List\textless{}Model\textgreater{}\ entries,\ int\ rowCount)}:
  Called when the server call to retrieve a page of items succeeds. Note
  that this method may be called more than once; once for each page
  received. Because \texttt{startRow} and \texttt{endRow} change for
  each page, a \texttt{startRow} of \texttt{0} corresponds to the first
  item on the first page.
\item
  \texttt{onListItemSelected(Model\ entries,\ View\ view)}: Called when
  an item is selected in the list. This method's arguments include the
  selected list item (\texttt{Model}).
\item
  \texttt{error(Exception\ e,\ String\ userAction)}: Called when an
  error occurs in the process. The \texttt{userAction} argument
  distinguishes the specific action in which the error occurred.
\end{itemize}

\section{Web Content Display Screenlet for
Android}\label{web-content-display-screenlet-for-android}

\subsection{Requirements}\label{requirements-7}

\begin{itemize}
\tightlist
\item
  Android SDK 4.1 (API Level 16) or above
\item
  Liferay Portal 6.2 CE/EE, Liferay CE Portal 7.0/7.1, Liferay DXP 7.0+
\item
  Liferay Screens Compatibility app
  (\href{http://www.liferay.com/marketplace/-/mp/application/54365664}{CE}
  or
  \href{http://www.liferay.com/marketplace/-/mp/application/54369726}{EE/DXP}).
  This app is preinstalled in Liferay CE Portal 7.0/7.1 and Liferay DXP
  7.0+.
\end{itemize}

\subsection{Compatibility}\label{compatibility-7}

\begin{itemize}
\tightlist
\item
  Android SDK 4.1 (API Level 16) or above
\end{itemize}

\subsection{Xamarin Requirements}\label{xamarin-requirements-7}

\begin{itemize}
\tightlist
\item
  Visual Studio 7.2
\item
  Mono .NET framework 5.4.1.6
\end{itemize}

\subsection{Features}\label{features-7}

The Web Content Display Screenlet shows web content elements in your
app, rendering the web content's inner HTML. The Screenlet also supports
i18n, rendering contents differently depending on the device's locale.

\subsection{JSON Services Used}\label{json-services-used-7}

Screenlets in Liferay Screens call JSON web services in the portal. This
Screenlet calls the following services and methods.

\noindent\hrulefill

\begin{longtable}[]{@{}
  >{\raggedright\arraybackslash}p{(\columnwidth - 4\tabcolsep) * \real{0.3889}}
  >{\raggedright\arraybackslash}p{(\columnwidth - 4\tabcolsep) * \real{0.3333}}
  >{\raggedright\arraybackslash}p{(\columnwidth - 4\tabcolsep) * \real{0.2778}}@{}}
\toprule\noalign{}
\begin{minipage}[b]{\linewidth}\raggedright
Service
\end{minipage} & \begin{minipage}[b]{\linewidth}\raggedright
Method
\end{minipage} & \begin{minipage}[b]{\linewidth}\raggedright
Notes
\end{minipage} \\
\midrule\noalign{}
\endhead
\bottomrule\noalign{}
\endlastfoot
\texttt{DDMStructureService} & \texttt{getStructure} & \\
\texttt{JournalArticleService} & \texttt{getArticle} & \\
\texttt{JournalArticleService} & \texttt{getArticleContent} & \\
\texttt{ScreensddlrecordService} (Screens compatibility plugin) &
\texttt{getJournalArticleContent} & With \texttt{entryQuery} \\
\end{longtable}

\noindent\hrulefill

\subsection{Module}\label{module-7}

\begin{itemize}
\tightlist
\item
  None
\end{itemize}

\subsection{Views}\label{views-7}

\begin{itemize}
\tightlist
\item
  Default
\end{itemize}

The Default View uses a standard \texttt{WebView} to render the HTML.

\begin{figure}
\centering
\includegraphics{./images/screens-android-webcontentdisplay.png}
\caption{Web Content Display Screenlet using the Default View.}
\end{figure}

\subsection{Portal Configuration}\label{portal-configuration-7}

For the Web Content Display Screenlet to function properly, there should
be web content in the Liferay instance your app connects to. For more
details on web content, see the
\href{/docs/7-0/user/-/knowledge_base/u/creating-web-content}{Creating
Web Content} section of the Liferay User Guide.

\subsection{Offline}\label{offline-7}

This Screenlet supports offline mode so it can function without a
network connection. For more information on how offline mode works, see
the
\href{/docs/7-0/tutorials/-/knowledge_base/t/architecture-of-offline-mode-in-liferay-screens}{tutorial
on its architecture}. Here are the offline mode policies that you can
use with this Screenlet:

\noindent\hrulefill

Policy \textbar{} What happens \textbar{} When to use \textbar{}
\texttt{REMOTE\_ONLY} \textbar{} The Screenlet loads the content from
the portal. If a connection issue occurs, the Screenlet uses the
listener to notify the developer about the error. If the Screenlet
successfully loads the content, it stores the data in the local cache
for later use. \textbar{} Use this policy when you always need to show
updated content, and show nothing when there's no connection. \textbar{}
\texttt{CACHE\_ONLY} \textbar{} The Screenlet loads the content from the
local cache. If the content isn't there, the Screenlet uses the listener
to notify the developer about the error. \textbar{} Use this policy when
you always need to show local content, without retrieving remote content
under any circumstance. \textbar{} \texttt{REMOTE\_FIRST} \textbar{} The
Screenlet loads the content from the portal. If this succeeds, the
Screenlet shows the content to the user and stores it in the local cache
for later use. If a connection issue occurs, the Screenlet retrieves the
content from the local cache. If the content doesn't exist there, the
Screenlet uses the listener to notify the developer about the error.
\textbar{} Use this policy to show the most recent version of the
content when connected, but show a possibly outdated version when
there's no connection. \textbar{} \texttt{CACHE\_FIRST} \textbar{} The
Screenlet loads the content from the local cache. If the content isn't
there, the Screenlet requests it from the portal and notifies the
developer about any errors that occur (including connectivity errors).
\textbar{} Use this policy to save bandwidth and loading time in case
you have local (but probably outdated) content. \textbar{}

\noindent\hrulefill

\subsection{Required Attributes}\label{required-attributes-7}

\begin{itemize}
\tightlist
\item
  \texttt{articleId}
\end{itemize}

Note that if your web content uses
\href{/docs/7-0/user/-/knowledge_base/u/designing-uniform-content}{structures
and templates}, you can use \texttt{templateId} or \texttt{structureId}
in conjunction with \texttt{articleId}.

\subsection{Attributes}\label{attributes-7}

\noindent\hrulefill

Attribute \textbar{} Data type \textbar{} Explanation \textbar{}
\texttt{layoutId} \textbar{} \texttt{@layout} \textbar{} The layout used
to show the View. \textbar{} \texttt{groupId} \textbar{} \texttt{number}
\textbar{} The site (group) identifier where the asset is stored. If
this value is \texttt{0}, the \texttt{groupId} specified in
\texttt{LiferayServerContext} is used. \textbar{} \texttt{articleId}
\textbar{} \texttt{string} \textbar{} The identifier of the web content
to display. You can find the identifier by clicking \emph{Edit} on the
web content in the portal. \textbar{} \texttt{classPK} \textbar{}
\texttt{number} \textbar{} The corresponding asset's class primary key.
If the web content is an asset (from Asset List Screenlet, for example),
this is the asset's identifier. This attribute is used only if
\texttt{articleId} is empty. \textbar{} \texttt{templateId} \textbar{}
\texttt{number} \textbar{} The identifier of the template used to render
the web content. This only applies to
\href{/docs/7-0/user/-/knowledge_base/u/designing-uniform-content}{structured
web content}. \textbar{} \texttt{structureId} \textbar{} \texttt{number}
\textbar{} The identifier of the \texttt{DDMStructure} used to model the
web content. This parameter lets the Screenlet retrieve and parse the
structure. \textbar{} \texttt{labelFields} \textbar{} \texttt{string}
\textbar{} A comma-delimited list of \texttt{DDMStructure} fields to
display in the Screenlet. \textbar{} \texttt{autoLoad} \textbar{}
\texttt{boolean} \textbar{} Whether the content should be retrieved from
the portal as soon as the screenlet appears. Default value is
\texttt{true}. \textbar{} \texttt{javascriptEnabled} \textbar{}
\texttt{boolean} \textbar{} Enables support for JavaScript. This is
disabled by default. \textbar{} \texttt{cachePolicy} \textbar{}
\texttt{string} \textbar{} The offline mode setting. See the
\href{/docs/7-0/reference/-/knowledge_base/r/webcontentdisplayscreenlet-for-android\#offline}{Offline
section} for details. \textbar{}

\noindent\hrulefill

\subsection{Methods}\label{methods-4}

\noindent\hrulefill

Method \textbar{} Return \textbar{} Explanation \textbar{}
\texttt{load()} \textbar{} \texttt{void} \textbar{} Starts the request
to load the web content. The HTML is rendered when the response is
received. \textbar{} \texttt{getLocalized(String\ name)} \textbar{}
\texttt{String} \textbar{} Returns the value, according to the device
locale, of a field of the \texttt{DDMStructure} used to render the web
content.

\noindent\hrulefill

\subsection{Listener}\label{listener-7}

The Web Content Display Screenlet delegates some events to an object
that implements the \texttt{WebContentDisplayListener} interface. This
interface lets you implement the following methods:

\begin{itemize}
\item
  \texttt{onWebContentReceived(WebContent\ webContent)}: Called when the
  web content's HTML or \texttt{DDMStructure} is received. The HTML is
  available by calling the \texttt{getHtml} method. To make some
  adaptations, the listener may return a modified version of the HTML.
  The original HTML is rendered if the listener returns \texttt{null}.
\item
  \texttt{onUrlClicked(String\ url)}: Called when a URL is clicked.
  Return \texttt{true} to replace the default behavior, or
  \texttt{false} to load the url.
\item
  \texttt{onWebContentTouched(View\ view,\ MotionEvent\ event)}: Called
  when something is touched in the web content. Return \texttt{true} to
  replace the default behavior, or \texttt{false} to keep processing the
  event.
\item
  \texttt{error(Exception\ e,\ String\ userAction)}: Called when an
  error occurs in the process. The \texttt{userAction} argument
  distinguishes the specific action in which the error occurred.
\end{itemize}

\section{Web Content List Screenlet for
Android}\label{web-content-list-screenlet-for-android}

\subsection{Requirements}\label{requirements-8}

\begin{itemize}
\tightlist
\item
  Android SDK 4.1 (API Level 16) or above
\item
  Liferay Portal 6.2 CE/EE, Liferay CE Portal 7.0/7.1, Liferay DXP 7.0+
\end{itemize}

\subsection{Compatibility}\label{compatibility-8}

\begin{itemize}
\tightlist
\item
  Android SDK 4.1 (API Level 16) or above
\end{itemize}

\subsection{Xamarin Requirements}\label{xamarin-requirements-8}

\begin{itemize}
\tightlist
\item
  Visual Studio 7.2
\item
  Mono .NET framework 5.4.1.6
\end{itemize}

\subsection{Features}\label{features-8}

Web Content List Screenlet has the following features:

\begin{itemize}
\tightlist
\item
  Shows a scrollable collection of
  \href{/docs/7-0/user/-/knowledge_base/u/creating-web-content}{web
  content} articles.
\item
  Implements
  \href{http://www.iosnomad.com/blog/2014/4/21/fluent-pagination}{fluent
  pagination} with configurable page size.
\item
  Supports i18n in web content values.
\end{itemize}

\subsection{JSON Services Used}\label{json-services-used-8}

Screenlets in Liferay Screens call JSON web services in the portal. This
Screenlet calls the following services and methods.

\noindent\hrulefill

\begin{longtable}[]{@{}lll@{}}
\toprule\noalign{}
Service & Method & Notes \\
\midrule\noalign{}
\endhead
\bottomrule\noalign{}
\endlastfoot
\texttt{JournalArticleService} & \texttt{getJournalArticles} & \\
\texttt{JournalArticleService} & \texttt{getJournalArticlesCount} & \\
\end{longtable}

\noindent\hrulefill

\subsection{Module}\label{module-8}

\begin{itemize}
\tightlist
\item
  None
\end{itemize}

\subsection{Views}\label{views-8}

\begin{itemize}
\tightlist
\item
  Default
\end{itemize}

The Default View uses a standard \texttt{RecyclerView} to show the
scrollable list. Other Views may use a different component, such as
\texttt{ViewPager} or others, to show the items.

\begin{figure}
\centering
\includegraphics{./images/screens-android-webcontentlist.png}
\caption{The Web Content List Screenlet using the Default View.}
\end{figure}

\subsection{Offline}\label{offline-8}

This Screenlet supports offline mode so it can function without a
network connection. For more information on how offline mode works, see
the
\href{/docs/7-0/tutorials/-/knowledge_base/t/architecture-of-offline-mode-in-liferay-screens}{tutorial
on its architecture}. Here are the offline mode policies that you can
use with this Screenlet:

\noindent\hrulefill

Policy \textbar{} What happens \textbar{} When to use \textbar{}
\texttt{REMOTE\_ONLY} \textbar{} The Screenlet loads the list from the
Liferay instance. If a connection issue occurs, the Screenlet uses the
listener to notify the developer about the error. If the Screenlet
successfully loads the list, it stores the data in the local cache for
later use. \textbar{} Use this policy when you always need to show
updated data, and show nothing when there's no connection. \textbar{}
\texttt{CACHE\_ONLY} \textbar{} The Screenlet loads the list from the
local cache. If the list isn't there, the Screenlet uses the listener to
notify the developer about the error. \textbar{} Use this policy when
you always need to show local data, without retrieving remote
information under any circumstance. \textbar{} \texttt{REMOTE\_FIRST}
\textbar{} The Screenlet loads the list from the Liferay instance. If
this succeeds, the Screenlet shows the list to the user and stores it in
the local cache for later use. If a connection issue occurs, the
Screenlet retrieves the list from the local cache. If the list doesn't
exist there, the Screenlet uses the listener to notify the developer
about the error. \textbar{} Use this policy to show the most recent
version of the data when connected, but show a possibly outdated version
when there's no connection. \textbar{} \texttt{CACHE\_FIRST} \textbar{}
The Screenlet loads the list from the local cache. If the list isn't
there, the Screenlet requests it from the Liferay instance and notifies
the developer about any errors that occur (including connectivity
errors). \textbar{} Use this policy to save bandwidth and loading time
in case you have local (but possibly outdated) data. \textbar{}

\noindent\hrulefill

\subsection{Required Attributes}\label{required-attributes-8}

\begin{itemize}
\tightlist
\item
  \texttt{folderId}
\item
  \texttt{labelFields}
\end{itemize}

\subsection{Attributes}\label{attributes-8}

\noindent\hrulefill

Attribute \textbar{} Data type \textbar{} Explanation \textbar{}
\texttt{layoutId} \textbar{} \texttt{@layout} \textbar{} The ID of the
layout to use to show the View. \textbar{} \texttt{autoLoad} \textbar{}
\texttt{boolean} \textbar{} Whether the list loads automatically when
the Screenlet appears in the app's UI. The default value is
\texttt{true}. \textbar{} \texttt{folderId} \textbar{} \texttt{number}
\textbar{} The ID of the web content folder to retrieve content from.
\textbar{} \texttt{groupId} \textbar{} \texttt{number} \textbar{} The ID
of the site (group) where the asset is stored. If set to \texttt{0}, the
\texttt{groupId} specified in \texttt{LiferayServerContext} is used. The
default value is \texttt{0}. \textbar{} \texttt{cachePolicy} \textbar{}
\texttt{string} \textbar{} The offline mode setting. See the
\href{/docs/7-0/reference/-/knowledge_base/r/web-content-list-screenlet-for-android\#offline}{Offline
section} for details. \textbar{} \texttt{firstPageSize} \textbar{}
\texttt{number} \textbar{} The number of items to retrieve from the
server for display on the first page. The default value is \texttt{50}.
\textbar{} \texttt{pageSize} \textbar{} \texttt{number} \textbar{} The
number of items to retrieve from the server for display on the second
and subsequent pages. The default value is \texttt{25}. \textbar{}
\texttt{labelFields} \textbar{} \texttt{string} \textbar{} The
comma-separated names of the DDM fields to show. Refer to the list's
data definition to find the field names. For more information on this,
see
\href{/docs/7-0/user/-/knowledge_base/u/designing-uniform-content}{the
article on structured web content}. Note that the appearance of data
from a structure's fields depends on the \texttt{layoutId}. \textbar{}
\texttt{obcClassName} \textbar{} \texttt{string} \textbar{} The name of
the \texttt{OrderByComparator} class to use to sort the results. Omit
this property if you don't want to sort the results.
\href{https://github.com/liferay/liferay-portal/tree/master/modules/apps/web-experience/journal/journal-api/src/main/java/com/liferay/journal/util/comparator}{Click
here} to see some comparator classes. Note, however, that not all of
these classes can be used with \texttt{obcClassName}. You can only use
comparator classes that extend
\texttt{OrderByComparator\textless{}JournalArticle\textgreater{}}. You
can also create your own comparator classes that extend
\texttt{OrderByComparator\textless{}JournalArticle\textgreater{}}.
\textbar{}

\noindent\hrulefill

\subsection{Methods}\label{methods-5}

\noindent\hrulefill

Method \textbar{} Return \textbar{} Explanation \textbar{}
\texttt{loadPage(pageNumber)} \textbar{} \texttt{void} \textbar{} Starts
the request to load the specified page of records. The page is shown
when the response is received. \textbar{}

\noindent\hrulefill

\subsection{Listener}\label{listener-8}

Web Content List Screenlet delegates some events to an object or a class
that implements
\href{https://github.com/liferay/liferay-screens/blob/master/android/library/src/main/java/com/liferay/mobile/screens/base/list/BaseListListener.java}{the
\texttt{BaseListListener} interface}. This interface lets you implement
the following methods:

\begin{itemize}
\item
  \texttt{onListPageFailed(int\ startRow,\ Exception\ e)}: Called when
  the server call to retrieve a page of items fails. This method's
  arguments include the \texttt{Exception} generated when the server
  call fails.
\item
  \texttt{onListPageReceived(int\ startRow,\ int\ endRow,\ List\textless{}Record\textgreater{}\ records,\ int\ rowCount)}:
  Called when the server call to retrieve a page of items succeeds. Note
  that this method may be called more than once; once for each page
  received. Because \texttt{startRow} and \texttt{endRow} change for
  each page, a \texttt{startRow} of \texttt{0} corresponds to the first
  item on the first page.
\item
  \texttt{onListItemSelected(Record\ records,\ View\ view)}: Called when
  an item is selected in the list. This method's arguments include the
  selected list item (\texttt{Record}).
\end{itemize}

\section{Image Gallery Screenlet for
Android}\label{image-gallery-screenlet-for-android}

\subsection{Requirements}\label{requirements-9}

\begin{itemize}
\tightlist
\item
  Android SDK 4.1 (API Level 16) or above
\item
  Liferay Portal 6.2 CE/EE, Liferay CE Portal 7.0/7.1, Liferay DXP 7.0+
\item
  Liferay Screens Compatibility app
  (\href{http://www.liferay.com/marketplace/-/mp/application/54365664}{CE}
  or
  \href{http://www.liferay.com/marketplace/-/mp/application/54369726}{EE/DXP}).
  This app is preinstalled in Liferay CE Portal 7.0/7.1 and Liferay DXP
  7.0+.
\end{itemize}

\subsection{Compatibility}\label{compatibility-9}

\begin{itemize}
\tightlist
\item
  Android SDK 4.1 (API Level 16) or above
\end{itemize}

\subsection{Xamarin Requirements}\label{xamarin-requirements-9}

\begin{itemize}
\tightlist
\item
  Visual Studio 7.2
\item
  Mono .NET framework 5.4.1.6
\end{itemize}

\subsection{Features}\label{features-9}

Image Gallery Screenlet shows a list of images from a Documents and
Media folder in a Liferay instance. You can also use Image Gallery
Screenlet to upload images to and delete images from the same folder.
The Screenlet implements fluent pagination with configurable page size,
and supports i18n in asset values.

\subsection{JSON Services Used}\label{json-services-used-9}

Screenlets in Liferay Screens call JSON web services in the portal. This
Screenlet calls the following services and methods.

\noindent\hrulefill

\begin{longtable}[]{@{}lll@{}}
\toprule\noalign{}
Service & Method & Notes \\
\midrule\noalign{}
\endhead
\bottomrule\noalign{}
\endlastfoot
\texttt{DLAppService} & \texttt{getFileEntries} & Load \\
\texttt{DLAppService} & \texttt{getFileEntriesCount} & \\
\texttt{DLAppService} & \texttt{addFileEntry} & Upload \\
\texttt{DLAppService} & \texttt{deleteFileEntry} & Delete \\
\end{longtable}

\noindent\hrulefill

\subsection{Module}\label{module-9}

\begin{itemize}
\tightlist
\item
  None
\end{itemize}

\subsection{Views}\label{views-9}

The included Views use a standard Android \texttt{RecyclerView} to show
the scrollable list. Other custom Views may use a different component,
such as \texttt{ViewPager} or others, to show the items.

This Screenlet has three different Views:

\begin{enumerate}
\def\labelenumi{\arabic{enumi}.}
\tightlist
\item
  Grid (default)
\item
  Slideshow
\item
  List
\end{enumerate}

\begin{figure}
\centering
\includegraphics{./images/screens-android-imagegallery.png}
\caption{Image Gallery Screenlet using the Grid, Slideshow, and List
Views.}
\end{figure}

\subsection{Offline}\label{offline-9}

This Screenlet supports offline mode so it can function without a
network connection when loading or uploading images (deleting images
while offline is unsupported). For more information on how offline mode
works, see the
\href{/docs/7-0/tutorials/-/knowledge_base/t/architecture-of-offline-mode-in-liferay-screens}{tutorial
on its architecture}. This Screenlet supports the \texttt{REMOTE\_ONLY},
\texttt{CACHE\_ONLY}, \texttt{REMOTE\_FIRST}, and \texttt{CACHE\_FIRST}
offline mode policies.

These policies take the following actions when loading images from a
Liferay instance:

\noindent\hrulefill

Policy \textbar{} What happens \textbar{} When to use \textbar{}
\texttt{REMOTE\_ONLY} \textbar{} The Screenlet loads the list from the
Liferay instance. If a connection issue occurs, the Screenlet uses the
listener to notify the developer about the error. If the Screenlet
successfully loads the list, it stores the data in the local cache for
later use. \textbar{} Use this policy when you always need to show
updated data, and show nothing when there's no connection. \textbar{}
\texttt{CACHE\_ONLY} \textbar{} The Screenlet loads the list from the
local cache. If the list isn't there, the Screenlet uses the listener to
notify the developer about the error. \textbar{} Use this policy when
you always need to show local data, without retrieving remote
information under any circumstance. \textbar{} \texttt{REMOTE\_FIRST}
\textbar{} The Screenlet loads the list from the Liferay instance. If
this succeeds, the Screenlet shows the list to the user and stores it in
the local cache for later use. If a connection issue occurs, the
Screenlet retrieves the list from the local cache. If the list doesn't
exist there, the Screenlet uses the listener to notify the developer
about the error. \textbar{} Use this policy to show the most recent
version of the data when connected, but show an outdated version when
there's no connection. \textbar{} \texttt{CACHE\_FIRST} \textbar{} The
Screenlet loads the list from the local cache. If the list isn't there,
the Screenlet requests it from the Liferay instance and notifies the
developer about any errors that occur (including connectivity errors).
\textbar{} Use this policy to save bandwidth and loading time in case
you have local (but probably outdated) data. \textbar{}

\noindent\hrulefill

These policies take the following actions when uploading an image to a
Liferay instance:

\noindent\hrulefill

Policy \textbar{} What happens \textbar{} When to use \textbar{}
\texttt{REMOTE\_ONLY} \textbar{} The Screenlet sends the image to the
Liferay instance. If a connection issue occurs, the Screenlet uses the
delegate to notify the developer about the error, but it also discards
the image. \textbar{} Use this policy to make sure the Liferay instance
always has the most recent version of the image. \textbar{}
\texttt{CACHE\_ONLY} \textbar{} The Screenlet stores the image in the
local cache. \textbar{} Use this policy when you need to save the image
locally, but don't want to update the image in the Liferay instance
(delete or add image). \textbar{} \texttt{REMOTE\_FIRST} \textbar{} The
Screenlet sends the image to the Liferay instance. If this succeeds, it
also stores the image in the local cache for later use. If a connection
issue occurs, the Screenlet stores the image in the local cache and
sends it to the Liferay instance when the connection is re-established.
\textbar{} Use this policy when you need to make sure the Screenlet
sends the image to the Liferay instance as soon as the connection is
restored. \textbar{} \texttt{CACHE\_FIRST} \textbar{} The Screenlet
stores the image in the local cache and then attempts to send it to the
Liferay instance. If a connection issue occurs, the Screenlet sends the
image to the Liferay instance when the connection is re-established.
\textbar{} Use this policy when you need to make sure the Screenlet
sends the image to the Liferay instance as soon as the connection is
restored. Compared to \texttt{REMOTE\_FIRST}, this policy always stores
the image in the cache. The \texttt{REMOTE\_FIRST} policy only stores
the image in the event of a network error. \textbar{}

\noindent\hrulefill

\subsection{Required Attributes}\label{required-attributes-9}

\begin{itemize}
\tightlist
\item
  \texttt{folderId}
\item
  \texttt{repositoryId}
\end{itemize}

\subsection{Attributes}\label{attributes-9}

\noindent\hrulefill

Attribute \textbar{} Data type \textbar{} Explanation \textbar{}
\texttt{repositoryId} \textbar{} \texttt{number} \textbar{} The ID of
the Liferay instance's Documents and Media repository that contains the
image gallery. If you're using a site's default Documents and Media
repository, then the \texttt{repositoryId} matches the site ID
(\texttt{groupId}). \textbar{} \texttt{folderId} \textbar{}
\texttt{number} \textbar{} The ID of the Documents and Media repository
folder that contains the image gallery. When accessing the folder in
your browser, the \texttt{folderId} is at the end of the URL. \textbar{}
\texttt{cachePolicy} \textbar{} \texttt{string} \textbar{} The offline
mode setting. See the
\href{/docs/7-0/reference/-/knowledge_base/r/image-gallery-screenlet-for-android\#offline}{Offline
section} for details. \textbar{} \texttt{firstPageSize} \textbar{}
\texttt{number} \textbar{} The number of items to display on the first
page. The default value is \texttt{50}. \textbar{} \texttt{pageSize}
\textbar{} \texttt{number} \textbar{} The number of items to display on
second and subsequent pages. The default value is \texttt{25}.
\textbar{} \texttt{mimeTypes} \textbar{} \texttt{string} \textbar{} The
comma-separated list of MIME types for the Screenlet to support.
\textbar{} \texttt{autoLoad} \textbar{} \texttt{boolean} \textbar{}
Whether the list automatically loads when the Screenlet appears in the
app's UI. The default value is \texttt{true}. \textbar{}
\texttt{layoutId} \textbar{} \texttt{@layout} \textbar{} The layout to
use to show the View. \textbar{} \texttt{obcClassName} \textbar{}
\texttt{string} \textbar{} The name of the \texttt{OrderByComparator}
class to use to sort the results. Omit this property if you don't want
to sort the results. Note that you can only use comparator classes that
extend \texttt{OrderByComparator\textless{}DLFileEntry\textgreater{}}.
Liferay contains no such comparator classes. You must therefore create
your own by extending
\texttt{OrderByComparator\textless{}DLFileEntry\textgreater{}}. To see
examples of some comparator classes that extend other Document Library
classes,
\href{https://github.com/liferay/liferay-portal/tree/master/portal-impl/src/com/liferay/portlet/documentlibrary/util/comparator}{click
here}. \textbar{}

\noindent\hrulefill

\subsection{Methods}\label{methods-6}

\noindent\hrulefill

Method \textbar{} Return \textbar{} Explanation \textbar{}
\texttt{loadPage(pageNumber)} \textbar{} \texttt{void} \textbar{} Starts
the request to load the specified page of images. The page is shown when
the response is received. \textbar{}

\noindent\hrulefill

\subsection{Listener}\label{listener-9}

Image Gallery Screenlet delegates some events to an object or class that
implements its
\href{https://github.com/liferay/liferay-screens/blob/master/android/library/src/main/java/com/liferay/mobile/screens/imagegallery/ImageGalleryListener.java}{\texttt{ImageGalleryListener}
interface}. This interface extends
\href{https://github.com/liferay/liferay-screens/blob/master/android/library/src/main/java/com/liferay/mobile/screens/base/list/BaseListListener.java}{the
\texttt{BaseListListener} interface}. Therefore, Image Gallery
Screenlet's listener methods are as follows:

\begin{itemize}
\item
  \texttt{onListPageFailed(int\ startRow,\ Exception\ e)}: Called when
  the server call to retrieve a page of items fails. This method's
  arguments include the \texttt{Exception} generated when the server
  call fails.
\item
  \texttt{onListPageReceived(int\ startRow,\ int\ endRow,\ List\textless{}Record\textgreater{}\ records,\ int\ rowCount)}:
  Called when the server call to retrieve a page of items succeeds. Note
  that this method may be called more than once; once for each page
  received. Because \texttt{startRow} and \texttt{endRow} change for
  each page, a \texttt{startRow} of \texttt{0} corresponds to the first
  item on the first page.
\item
  \texttt{onListItemSelected(Record\ records,\ View\ view)}: Called when
  an item is selected in the list. This method's arguments include the
  selected list item (\texttt{Record}).
\item
  \texttt{onImageEntryDeleted(long\ imageEntryId)}: Called when an item
  in the list is deleted.
\item
  \texttt{onImageUploadStarted(String\ picturePath,\ String\ title,\ String\ description,\ String\ changelog)}:
  Called when an item is prepared for upload.
\item
  \texttt{onImageUploadProgress(int\ totalBytes,\ int\ totalBytesSent)}:
  Called when an item is uploading.
\item
  \texttt{onImageUploadEnd(ImageEntry\ entry)}: Called when an item
  finishes uploading.
\item
  \texttt{showUploadImageView(String\ actionName,\ String\ picturePath,\ int\ screenletId)}:
  Called when the View for uploading an image is instantiated. The
  default behavior is to show the default View in a dialog. To retain
  this behavior, all this method needs to do is return \texttt{false}.
  To change the default behavior, use the \texttt{initializeUploadView}
  method to initialize a custom View that extends
  \texttt{BaseDetailUploadView}. Then return \texttt{true} to prevent
  the Screenlet from executing the default behavior. For example, the
  following sample implementation uses \texttt{initializeUploadView} to
  initialize the custom View instance \texttt{uploadDetailView}. It then
  performs a custom UI action (\texttt{uploadImageCard.goRight()}) and
  returns \texttt{true}:

\begin{verbatim}
  @Override
  public boolean showUploadImageView(String actionName, String picturePath, int screenletId) {
      uploadDetailView.initializeUploadView(actionName, picturePath, screenletId);
      uploadImageCard.goRight();

      return true;
  }
\end{verbatim}
\item
  \texttt{provideImageUploadDetailView()}: Called when the Screenlet
  provides the image upload View. To inflate the default View, return
  \texttt{0} in this method. Alternatively, display this View with a
  custom layout by returning its layout ID. Such a layout must have
  \texttt{DefaultUploadDetailView} as its root class.
\item
  \texttt{error(Exception\ e,\ String\ userAction)}: Called when an
  error occurs in the process. The \texttt{userAction} argument
  distinguishes the specific action in which the error occurred.
\end{itemize}

\section{Rating Screenlet for
Android}\label{rating-screenlet-for-android}

\subsection{Requirements}\label{requirements-10}

\begin{itemize}
\tightlist
\item
  Android SDK 4.1 (API Level 16) or above
\item
  Liferay Portal 6.2 CE/EE, Liferay CE Portal 7.0/7.1, Liferay DXP 7.0+
\item
  Liferay Screens Compatibility app
  (\href{http://www.liferay.com/marketplace/-/mp/application/54365664}{CE}
  or
  \href{http://www.liferay.com/marketplace/-/mp/application/54369726}{EE/DXP}).
  This app is preinstalled in Liferay CE Portal 7.0/7.1 and Liferay DXP
  7.0+.
\end{itemize}

\subsection{Compatibility}\label{compatibility-10}

\begin{itemize}
\tightlist
\item
  Android SDK 4.1 (API Level 16) or above
\end{itemize}

\subsection{Xamarin Requirements}\label{xamarin-requirements-10}

\begin{itemize}
\tightlist
\item
  Visual Studio 7.2
\item
  Mono .NET framework 5.4.1.6
\end{itemize}

\subsection{Features}\label{features-10}

Rating Screenlet shows an asset's rating. It also lets users update or
delete the rating. This Screenlet comes with different Views that
display ratings as thumbs, stars, and emojis.

\subsection{JSON Services Used}\label{json-services-used-10}

Screenlets in Liferay Screens call JSON web services in the portal. This
Screenlet calls the following services and methods.

\noindent\hrulefill

\begin{longtable}[]{@{}
  >{\raggedright\arraybackslash}p{(\columnwidth - 4\tabcolsep) * \real{0.3889}}
  >{\raggedright\arraybackslash}p{(\columnwidth - 4\tabcolsep) * \real{0.3333}}
  >{\raggedright\arraybackslash}p{(\columnwidth - 4\tabcolsep) * \real{0.2778}}@{}}
\toprule\noalign{}
\begin{minipage}[b]{\linewidth}\raggedright
Service
\end{minipage} & \begin{minipage}[b]{\linewidth}\raggedright
Method
\end{minipage} & \begin{minipage}[b]{\linewidth}\raggedright
Notes
\end{minipage} \\
\midrule\noalign{}
\endhead
\bottomrule\noalign{}
\endlastfoot
\texttt{ScreensratingsentryService} (Screens compatibility plugin) &
\texttt{getRatingsEntries} & With \texttt{entryId} \\
\texttt{ScreensratingsentryService} (Screens compatibility plugin) &
\texttt{getRatingsEntries} & With \texttt{classPK} and
\texttt{className} \\
\texttt{ScreensratingsentryService} (Screens compatibility plugin) &
\texttt{updateRatingsEntry} & \\
\texttt{ScreensratingsentryService} (Screens compatibility plugin) &
\texttt{deleteRatingsEntry} & \\
\end{longtable}

\noindent\hrulefill

\subsection{Module}\label{module-10}

\begin{itemize}
\tightlist
\item
  None
\end{itemize}

\subsection{Views}\label{views-10}

The default View uses an
\href{https://developer.android.com/reference/android/widget/RatingBar.html}{Android
\texttt{RatingBar}} to show an asset's rating. Other custom Views may
show the rating with a different Android component such as
\texttt{Button}, \texttt{ImageButton}, or others.

This Screenlet has five different Views:

\begin{enumerate}
\def\labelenumi{\arabic{enumi}.}
\tightlist
\item
  Like
\item
  Thumbs (default)
\item
  Stars
\item
  Reactions
\item
  Emojis
\end{enumerate}

\begin{figure}
\centering
\includegraphics{./images/screens-android-ratings.png}
\caption{Rating Screenlet's different Views.}
\end{figure}

\subsection{Offline}\label{offline-10}

This Screenlet supports offline mode so it can function without a
network connection. For more information on how offline mode works, see
the
\href{/docs/7-0/tutorials/-/knowledge_base/t/architecture-of-offline-mode-in-liferay-screens}{tutorial
on its architecture}. Here are the offline mode policies that you can
use with this Screenlet:

\noindent\hrulefill

Policy \textbar{} What happens \textbar{} When to use \textbar{}
\texttt{REMOTE\_ONLY} \textbar{} The Screenlet loads the data from the
Liferay instance. If a connection issue occurs, the Screenlet uses the
listener to notify the developer about the error. If the Screenlet
successfully loads the data, it stores it in the local cache for later
use. \textbar{} Use this policy when you always need to show updated
data, and show nothing when there's no connection. \textbar{}
\texttt{CACHE\_ONLY} \textbar{} The Screenlet loads the data from the
local cache. If the data isn't there, the Screenlet uses the listener to
notify the developer about the error. \textbar{} Use this policy when
you always need to show local data, without retrieving remote
information under any circumstance. \textbar{} \texttt{REMOTE\_FIRST}
\textbar{} The Screenlet loads the data from the Liferay instance. If
this succeeds, the Screenlet shows the data to the user and stores it in
the local cache for later use. If a connection issue occurs, the
Screenlet retrieves the data from the local cache. If the data doesn't
exist there, the Screenlet uses the listener to notify the developer
about the error. \textbar{} Use this policy to show the most recent
version of the data when connected, but show an outdated version when
there's no connection. \textbar{} \texttt{CACHE\_FIRST} \textbar{} The
Screenlet loads the data from the local cache. If the data isn't there,
the Screenlet requests it from the Liferay instance and notifies the
developer about any errors that occur (including connectivity errors).
\textbar{} Use this policy to save bandwidth and loading time in case
you have local (but probably outdated) data. \textbar{}

\noindent\hrulefill

\subsection{Required Attributes}\label{required-attributes-10}

\begin{itemize}
\tightlist
\item
  \texttt{entryId}
\end{itemize}

If you don't use \texttt{entryId}, you must use both of the following
attributes:

\begin{itemize}
\tightlist
\item
  \texttt{className}
\item
  \texttt{classPK}
\end{itemize}

\subsection{Attributes}\label{attributes-10}

\noindent\hrulefill

Attribute \textbar{} Data type \textbar{} Explanation \textbar{}
\texttt{layoutId} \textbar{} \texttt{@layout} \textbar{} The ID of the
layout to use to show the View. \textbar{} \texttt{autoLoad} \textbar{}
\texttt{boolean} \textbar{} Whether the rating loads automatically when
the Screenlet appears in the app's UI. The default value is
\texttt{true}. \textbar{} \texttt{editable} \textbar{} \texttt{boolean}
\textbar{} Whether the user can change the rating. \textbar{}
\texttt{entryId} \textbar{} \texttt{number} \textbar{} The primary key
of the asset with the rating to display. \textbar{} \texttt{className}
\textbar{} \texttt{string} \textbar{} The asset's fully qualified class
name. For example, a blog entry's \texttt{className} is
\href{@platform-ref@/7.0-latest/javadocs/portal-kernel/com/liferay/blogs/kernel/model/BlogsEntry.html}{\texttt{com.liferay.blogs.kernel.model.BlogsEntry}}.
The \texttt{className} attribute is required when using it with
\texttt{classPK} to instantiate the Screenlet. \textbar{}
\texttt{classPK} \textbar{} \texttt{number} \textbar{} The asset's
unique identifier. Only use this attribute when also using
\texttt{className} to instantiate the Screenlet. \textbar{}
\texttt{groupId} \textbar{} \texttt{number} \textbar{} The ID of the
site (group) containing the asset. \textbar{} \texttt{cachePolicy}
\textbar{} \texttt{string} \textbar{} The offline mode setting. See the
\href{/docs/7-0/reference/-/knowledge_base/r/rating-screenlet-for-android\#offline}{Offline
section} for details. \textbar{}

\noindent\hrulefill

\subsection{Methods}\label{methods-7}

\noindent\hrulefill

Method \textbar{} Return \textbar{} Explanation \textbar{}
\texttt{load()} \textbar{} \texttt{void} \textbar{} Starts the request
to load the asset's ratings. \textbar{}

\noindent\hrulefill

\subsection{Listener}\label{listener-10}

Rating Screenlet delegates some events to an object or class that
implements
\href{https://github.com/liferay/liferay-screens/blob/master/android/library/src/main/java/com/liferay/mobile/screens/rating/RatingListener.java}{its
\texttt{RatingListener} interface}. Therefore, Rating Screenlet's
listener methods are as follows:

\begin{itemize}
\tightlist
\item
  \texttt{onRatingOperationSuccess(AssetRating\ assetRating)}: Called
  when the operation finishes successfully and the rating is loaded.
\end{itemize}

\section{Comment List Screenlet for
Android}\label{comment-list-screenlet-for-android}

\subsection{Requirements}\label{requirements-11}

\begin{itemize}
\tightlist
\item
  Android SDK 4.1 (API Level 16) or above
\item
  Liferay Portal 6.2 CE/EE, Liferay CE Portal 7.0/7.1, Liferay DXP 7.0+
\item
  Liferay Screens Compatibility app
  (\href{http://www.liferay.com/marketplace/-/mp/application/54365664}{CE}
  or
  \href{http://www.liferay.com/marketplace/-/mp/application/54369726}{EE/DXP}).
  This app is preinstalled in Liferay CE Portal 7.0/7.1 and Liferay DXP
  7.0+.
\end{itemize}

\subsection{Compatibility}\label{compatibility-11}

\begin{itemize}
\tightlist
\item
  Android SDK 4.1 (API Level 16) or above
\end{itemize}

\subsection{Xamarin Requirements}\label{xamarin-requirements-11}

\begin{itemize}
\tightlist
\item
  Visual Studio 7.2
\item
  Mono .NET framework 5.4.1.6
\end{itemize}

\subsection{Features}\label{features-11}

Comment List Screenlet can list all the comments of an asset in a
Liferay instance. It also lets the user update or delete comments.

\subsection{JSON Services Used}\label{json-services-used-11}

Screenlets in Liferay Screens call JSON web services in the portal. This
Screenlet calls the following services and methods.

\noindent\hrulefill

\begin{longtable}[]{@{}
  >{\raggedright\arraybackslash}p{(\columnwidth - 4\tabcolsep) * \real{0.3889}}
  >{\raggedright\arraybackslash}p{(\columnwidth - 4\tabcolsep) * \real{0.3333}}
  >{\raggedright\arraybackslash}p{(\columnwidth - 4\tabcolsep) * \real{0.2778}}@{}}
\toprule\noalign{}
\begin{minipage}[b]{\linewidth}\raggedright
Service
\end{minipage} & \begin{minipage}[b]{\linewidth}\raggedright
Method
\end{minipage} & \begin{minipage}[b]{\linewidth}\raggedright
Notes
\end{minipage} \\
\midrule\noalign{}
\endhead
\bottomrule\noalign{}
\endlastfoot
\texttt{ScreenscommentService} (Screens compatibility plugin) &
\texttt{getComments} & \\
\texttt{ScreenscommentService} (Screens compatibility plugin) &
\texttt{getCommentsCount} & \\
\end{longtable}

\noindent\hrulefill

\subsection{Module}\label{module-11}

\begin{itemize}
\tightlist
\item
  None
\end{itemize}

\subsection{Views}\label{views-11}

\begin{itemize}
\tightlist
\item
  Default
\end{itemize}

The Default View uses an
\href{https://developer.android.com/training/material/lists-cards.html}{Android
\texttt{RecyclerView}} to show an asset's comments. Other Views may use
a different component, such as \texttt{TableView} or others, to show the
items.

\begin{figure}
\centering
\includegraphics{./images/screens-android-commentlist.png}
\caption{Comment List Screenlet using the Default View.}
\end{figure}

\subsection{Offline}\label{offline-11}

This Screenlet supports offline mode so it can function without a
network connection. For more information on how offline mode works, see
the
\href{/docs/7-0/tutorials/-/knowledge_base/t/architecture-of-offline-mode-in-liferay-screens}{tutorial
on its architecture}. Here are the offline mode policies that you can
use with this Screenlet:

\noindent\hrulefill

Policy \textbar{} What happens \textbar{} When to use \textbar{}
\texttt{REMOTE\_ONLY} \textbar{} The Screenlet loads the comments from
the Liferay instance. If a connection issue occurs, the Screenlet uses
the listener to notify the developer about the error. If the Screenlet
successfully loads the comments, it stores the data in the local cache
for later use. \textbar{} Use this policy when you always need to show
updated data, and show nothing when there's no connection. \textbar{}
\texttt{CACHE\_ONLY} \textbar{} The Screenlet loads the comments from
the local cache. If the data isn't there, the Screenlet uses the
listener to notify the developer about the error. \textbar{} Use this
policy when you always need to show local data, without retrieving
remote information under any circumstance. \textbar{}
\texttt{REMOTE\_FIRST} \textbar{} The Screenlet loads the comments from
the Liferay instance. If this succeeds, the Screenlet shows the data to
the user and stores it in the local cache for later use. If a connection
issue occurs, the Screenlet retrieves the data from the local cache. If
the data doesn't exist there, the Screenlet uses the listener to notify
the developer about the error. \textbar{} Use this policy to show the
most recent version of the data when connected, but show an outdated
version when there's no connection. \textbar{} \texttt{CACHE\_FIRST}
\textbar{} The Screenlet loads the comments from the local cache. If the
data isn't there, the Screenlet requests it from the Liferay instance
and notifies the developer about any errors that occur (including
connectivity errors). \textbar{} Use this policy to save bandwidth and
loading time in case you have local (but probably outdated) data.
\textbar{}

\noindent\hrulefill

\subsection{Required Attributes}\label{required-attributes-11}

\begin{itemize}
\tightlist
\item
  \texttt{className}
\item
  \texttt{classPK}
\end{itemize}

\subsection{Attributes}\label{attributes-11}

\noindent\hrulefill

Attribute \textbar{} Data type \textbar{} Explanation \textbar{}
\texttt{layoutId} \textbar{} \texttt{@layout} \textbar{} The layout to
use to show the View. \textbar{} \texttt{autoLoad} \textbar{}
\texttt{boolean} \textbar{} Whether the list should automatically load
when the Screenlet appears in the app's UI. The default value is
\texttt{true}. \textbar{} \texttt{cachePolicy} \textbar{}
\texttt{string} \textbar{} The offline mode setting. See
\href{/docs/7-0/reference/-/knowledge_base/r/comment-list-screenlet-for-android\#offline}{the
Offline section} for details. \textbar{} \texttt{className} \textbar{}
\texttt{string} \textbar{} The asset's fully qualified class name. For
example, a blog entry's \texttt{className} is
\href{@platform-ref@/7.0-latest/javadocs/portal-kernel/com/liferay/blogs/kernel/model/BlogsEntry.html}{\texttt{com.liferay.blogs.kernel.model.BlogsEntry}}.
The \texttt{className} and \texttt{classPK} attributes are required to
instantiate the Screenlet. \textbar{} \texttt{classPK} \textbar{}
\texttt{number} \textbar{} The asset's unique identifier. The
\texttt{className} and \texttt{classPK} attributes are required to
instantiate the Screenlet. \textbar{} \texttt{firstPageSize} \textbar{}
\texttt{number} \textbar{} The number of items to retrieve from the
server for display on the first page. The default value is \texttt{50}.
\textbar{} \texttt{pageSize} \textbar{} \texttt{number} \textbar{} The
number of items to retrieve from the server for display on the second
and subsequent pages. The default value is \texttt{25}. \textbar{}
\texttt{labelFields} \textbar{} \texttt{string} \textbar{} The
comma-separated names of the DDL fields to show. Refer to the list's
data definition to find the field names. For more information on this,
see
\href{/docs/7-0/user/-/knowledge_base/u/designing-uniform-content}{the
article on structured web content}. Note that the appearance of data
from a structure's fields depends on the \texttt{layoutId}. \textbar{}
\texttt{editable} \textbar{} \texttt{boolean} \textbar{} Whether the
user can edit the comment. \textbar{}

\noindent\hrulefill

\subsection{Methods}\label{methods-8}

\noindent\hrulefill

Method \textbar{} Return \textbar{} Explanation \textbar{}
\texttt{loadPage(pageNumber)} \textbar{} \texttt{void} \textbar{} Starts
the request to load the specified page of records. The page is shown
when the response is received. \textbar{}

\noindent\hrulefill

\subsection{Listener}\label{listener-11}

Comment List Screenlet delegates some events to a class that implements
\texttt{CommentListListener}. This interface lets you implement the
following methods:

\begin{itemize}
\item
  \texttt{onDeleteCommentSuccess(CommentEntry\ commentEntry)}: Called
  when the Screenlet successfully deletes the comment.
\item
  \texttt{onUpdateCommentSuccess(CommentEntry\ commentEntry)}: Called
  when the Screenlet successfully updates the comment.
\item
  \texttt{onListPageFailed(int\ startRow,\ Exception\ e)}: Called when
  the server call to retrieve a page of items fails. This method's
  arguments include the \texttt{Exception} generated when the server
  call fails.
\item
  \texttt{onListPageReceived(int\ startRow,\ int\ endRow,\ List\textless{}CommentEntry\textgreater{}\ entries,\ int\ rowCount)}:
  Called when the server call to retrieve a page of items succeeds. Note
  that this method may be called more than once; once for each page
  received. Because \texttt{startRow} and \texttt{endRow} change for
  each page, a \texttt{startRow} of \texttt{0} corresponds to the first
  item on the first page.
\item
  \texttt{onListItemSelected(CommentEntry\ element,\ View\ view)}:
  Called when an item is selected in the list. This method's arguments
  include the selected list item (\texttt{CommentEntry}).
\item
  \texttt{error(Exception\ e,\ String\ userAction)}: Called when an
  error occurs in the process. The \texttt{userAction} argument
  distinguishes the specific action in which the error occurred.
\end{itemize}

\section{Comment Display Screenlet for
Android}\label{comment-display-screenlet-for-android}

\subsection{Requirements}\label{requirements-12}

\begin{itemize}
\tightlist
\item
  Android SDK 4.1 (API Level 16) or above
\item
  Liferay Portal 6.2 CE/EE, Liferay CE Portal 7.0/7.1, Liferay DXP 7.0+
\item
  Liferay Screens Compatibility app
  (\href{http://www.liferay.com/marketplace/-/mp/application/54365664}{CE}
  or
  \href{http://www.liferay.com/marketplace/-/mp/application/54369726}{EE/DXP}).
  This app is preinstalled in Liferay CE Portal 7.0/7.1 and Liferay DXP
  7.0+.
\end{itemize}

\subsection{Compatibility}\label{compatibility-12}

\begin{itemize}
\tightlist
\item
  Android SDK 4.1 (API Level 16) or above
\end{itemize}

\subsection{Xamarin Requirements}\label{xamarin-requirements-12}

\begin{itemize}
\tightlist
\item
  Visual Studio 7.2
\item
  Mono .NET framework 5.4.1.6
\end{itemize}

\subsection{Features}\label{features-12}

Comment Display Screenlet can show one comment of an asset in a Liferay
instance. It also lets the user update or delete the comment.

\subsection{JSON Services Used}\label{json-services-used-12}

Screenlets in Liferay Screens call JSON web services in the portal. This
Screenlet calls the following services and methods.

\noindent\hrulefill

\begin{longtable}[]{@{}
  >{\raggedright\arraybackslash}p{(\columnwidth - 4\tabcolsep) * \real{0.3889}}
  >{\raggedright\arraybackslash}p{(\columnwidth - 4\tabcolsep) * \real{0.3333}}
  >{\raggedright\arraybackslash}p{(\columnwidth - 4\tabcolsep) * \real{0.2778}}@{}}
\toprule\noalign{}
\begin{minipage}[b]{\linewidth}\raggedright
Service
\end{minipage} & \begin{minipage}[b]{\linewidth}\raggedright
Method
\end{minipage} & \begin{minipage}[b]{\linewidth}\raggedright
Notes
\end{minipage} \\
\midrule\noalign{}
\endhead
\bottomrule\noalign{}
\endlastfoot
\texttt{ScreenscommentService} (Screens compatibility plugin) &
\texttt{getComment} & \\
\texttt{ScreenscommentService} (Screens compatibility plugin) &
\texttt{updateComment} & \\
\texttt{CommentmanagerjsonwsService} & \texttt{deleteComment} & \\
\end{longtable}

\noindent\hrulefill

\subsection{Module}\label{module-12}

\begin{itemize}
\tightlist
\item
  None
\end{itemize}

\subsection{Views}\label{views-12}

\begin{itemize}
\tightlist
\item
  Default
\end{itemize}

The Default View uses
\href{/docs/7-0/reference/-/knowledge_base/r/userportraitscreenlet-for-android}{User
Portrait Screenlet}, and \texttt{TextView} and \texttt{ImageButton}
elements to show an asset's comment. Other Views may different
components to show the comment.

\begin{figure}
\centering
\includegraphics{./images/screens-android-commentdisplay.png}
\caption{Comment Display Screenlet using the Default View.}
\end{figure}

\subsection{Offline}\label{offline-12}

This Screenlet supports offline mode so it can function without a
network connection. For more information on how offline mode works, see
the
\href{/docs/7-0/tutorials/-/knowledge_base/t/architecture-of-offline-mode-in-liferay-screens}{tutorial
on its architecture}. Here are the offline mode policies that you can
use with this Screenlet:

\noindent\hrulefill

Policy \textbar{} What happens \textbar{} When to use \textbar{}
\texttt{REMOTE\_ONLY} \textbar{} The Screenlet loads the data from the
Liferay instance. If a connection issue occurs, the Screenlet uses the
listener to notify the developer about the error. If the Screenlet
successfully loads the data, it stores it in the local cache for later
use. \textbar{} Use this policy when you always need to show updated
data, and show nothing when there's no connection. \textbar{}
\texttt{CACHE\_ONLY} \textbar{} The Screenlet loads the data from the
local cache. If the data isn't there, the Screenlet uses the listener to
notify the developer about the error. \textbar{} Use this policy when
you always need to show local data, without retrieving remote
information under any circumstance. \textbar{} \texttt{REMOTE\_FIRST}
\textbar{} The Screenlet loads the data from the Liferay instance. If
this succeeds, the Screenlet shows the data to the user and stores it in
the local cache for later use. If a connection issue occurs, the
Screenlet retrieves the data from the local cache. If the data doesn't
exist there, the Screenlet uses the listener to notify the developer
about the error. \textbar{} Use this policy to show the most recent
version of the data when connected, but show an outdated version when
there's no connection. \textbar{} \texttt{CACHE\_FIRST} \textbar{} The
Screenlet loads the data from the local cache. If the data isn't there,
the Screenlet requests it from the Liferay instance and notifies the
developer about any errors that occur (including connectivity errors).
\textbar{} Use this policy to save bandwidth and loading time in case
you have local (but probably outdated) data. \textbar{}

\noindent\hrulefill

\subsection{Required Attributes}\label{required-attributes-12}

\begin{itemize}
\tightlist
\item
  \texttt{commentId}
\end{itemize}

\subsection{Attributes}\label{attributes-12}

\noindent\hrulefill

Attribute \textbar{} Data type \textbar{} Explanation \textbar{}
\texttt{layoutId} \textbar{} \texttt{@layout} \textbar{} The layout to
use to show the View.\textbar{} \texttt{autoLoad} \textbar{}
\texttt{boolean} \textbar{} Whether the list should automatically load
when the Screenlet appears in the app's UI. The default value is
\texttt{true}. \textbar{} \texttt{cachePolicy} \textbar{}
\texttt{string} \textbar{} The offline mode setting. See
\href{/docs/7-0/reference/-/knowledge_base/r/comment-display-screenlet-for-android\#offline}{the
Offline section} for details. \textbar{} \texttt{commentId} \textbar{}
\texttt{number} \textbar{} The primary key of the comment to display.
\textbar{} \texttt{editable} \textbar{} \texttt{boolean} \textbar{}
Whether the user can edit the comment. \textbar{}

\noindent\hrulefill

\subsection{Methods}\label{methods-9}

\noindent\hrulefill

Method \textbar{} Return \textbar{} Explanation \textbar{}
\texttt{load()} \textbar{} \texttt{void} \textbar{} Starts the request
to load the comment. \textbar{}

\noindent\hrulefill

\subsection{Listener}\label{listener-12}

Comment Display Screenlet delegates some events to a class that
implements \texttt{CommentDisplayListener}. This interface lets you
implement the following methods:

\begin{itemize}
\item
  \texttt{onLoadCommentSuccess(CommentEntry\ commentEntry)}: Called when
  the Screenlet successfully loads the comment.
\item
  \texttt{onDeleteCommentSuccess(CommentEntry\ commentEntry)}: Called
  when the Screenlet successfully deletes the comment.
\item
  \texttt{onUpdateCommentSuccess(CommentEntry\ commentEntry)}: Called
  when the Screenlet successfully updates the comment.
\item
  \texttt{error(Exception\ e,\ String\ userAction)}: Called when an
  error occurs in the process. The \texttt{userAction} argument
  distinguishes the specific action in which the error occurred.
\end{itemize}

\section{Comment Add Screenlet for
Android}\label{comment-add-screenlet-for-android}

\subsection{Requirements}\label{requirements-13}

\begin{itemize}
\tightlist
\item
  Android SDK 4.1 (API Level 16) or above
\item
  Liferay Portal 6.2 CE/EE, Liferay CE Portal 7.0/7.1, Liferay DXP 7.0+
\item
  Liferay Screens Compatibility app
  (\href{http://www.liferay.com/marketplace/-/mp/application/54365664}{CE}
  or
  \href{http://www.liferay.com/marketplace/-/mp/application/54369726}{EE/DXP}).
  This app is preinstalled in Liferay CE Portal 7.0/7.1 and Liferay DXP
  7.0+.
\end{itemize}

\subsection{Compatibility}\label{compatibility-13}

\begin{itemize}
\tightlist
\item
  Android SDK 4.1 (API Level 16) or above
\end{itemize}

\subsection{Xamarin Requirements}\label{xamarin-requirements-13}

\begin{itemize}
\tightlist
\item
  Visual Studio 7.2
\item
  Mono .NET framework 5.4.1.6
\end{itemize}

\subsection{Features}\label{features-13}

Comment Add Screenlet can add a comment to an asset in a Liferay
instance.

\subsection{JSON Services Used}\label{json-services-used-13}

Screenlets in Liferay Screens call JSON web services in the portal. This
Screenlet calls the following services and methods.

\noindent\hrulefill

\begin{longtable}[]{@{}
  >{\raggedright\arraybackslash}p{(\columnwidth - 4\tabcolsep) * \real{0.3889}}
  >{\raggedright\arraybackslash}p{(\columnwidth - 4\tabcolsep) * \real{0.3333}}
  >{\raggedright\arraybackslash}p{(\columnwidth - 4\tabcolsep) * \real{0.2778}}@{}}
\toprule\noalign{}
\begin{minipage}[b]{\linewidth}\raggedright
Service
\end{minipage} & \begin{minipage}[b]{\linewidth}\raggedright
Method
\end{minipage} & \begin{minipage}[b]{\linewidth}\raggedright
Notes
\end{minipage} \\
\midrule\noalign{}
\endhead
\bottomrule\noalign{}
\endlastfoot
\texttt{ScreenscommentService} (Screens compatibility plugin) &
\texttt{addComment} & \\
\end{longtable}

\noindent\hrulefill

\subsection{Module}\label{module-13}

\begin{itemize}
\tightlist
\item
  None
\end{itemize}

\subsection{Views}\label{views-13}

\begin{itemize}
\tightlist
\item
  Default
\end{itemize}

The Default View uses Android's \texttt{EditText} and \texttt{Button}
elements to show an add comment dialog. Other Views may use different
components to show this dialog.

\begin{figure}
\centering
\includegraphics{./images/screens-android-commentadd.png}
\caption{Comment Add Screenlet using the Default View.}
\end{figure}

\subsection{Offline}\label{offline-13}

This Screenlet supports offline mode so it can function without a
network connection. For more information on how offline mode works, see
the
\href{/docs/7-0/tutorials/-/knowledge_base/t/architecture-of-offline-mode-in-liferay-screens}{tutorial
on its architecture}. Here are the offline mode policies that you can
use with this Screenlet:

\noindent\hrulefill

Policy \textbar{} What happens \textbar{} When to use \textbar{}
\texttt{REMOTE\_ONLY} \textbar{} The Screenlet sends the data to the
Liferay instance. If a connection issue occurs, the Screenlet uses the
listener to notify the developer about the error. If the Screenlet
successfully sends the data, it also stores it in the local cache.
\textbar{} Use this policy when you always need to send updated data,
and send nothing when there's no connection. \textbar{}
\texttt{CACHE\_ONLY} \textbar{} The Screenlet sends the data to the
local cache. If an error occurs, the Screenlet uses the listener to
notify the developer. \textbar{} Use this policy when you always need to
store local data without sending remote information under any
circumstance. \textbar{} \texttt{REMOTE\_FIRST} \textbar{} The Screenlet
sends the data to the Liferay instance. If this succeeds, the Screenlet
also stores the data in the local cache. If a connection issue occurs,
the Screenlet stores the data to the local cache and sends it to the
Liferay instance when the connection is restored. If an error occurs,
the Screenlet uses the listener to notify the developer. \textbar{} Use
this policy to send the most recent version of the data when connected,
and store the data for later synchronization when there's no connection.
\textbar{} \texttt{CACHE\_FIRST} \textbar{} The Screenlet sends the data
to the local cache, then sends it to the Liferay instance. If sending
the data to the Liferay instance fails, the Screenlet still stores the
data locally and then notifies the developer about any errors that occur
(including connectivity errors). \textbar{} Use this policy to save
bandwidth and store local (but possibly outdated) data. \textbar{}

\noindent\hrulefill

\subsection{Required Attributes}\label{required-attributes-13}

\begin{itemize}
\tightlist
\item
  \texttt{className}
\item
  \texttt{classPK}
\end{itemize}

\subsection{Attributes}\label{attributes-13}

\noindent\hrulefill

Attribute \textbar{} Data type \textbar{} Explanation \textbar{}
\texttt{layoutId} \textbar{} \texttt{@layout} \textbar{} The layout to
use to show the View.\textbar{} \texttt{className} \textbar{}
\texttt{string} \textbar{} The asset's fully qualified class name. For
example, a blog entry's \texttt{className} is
\href{@platform-ref@/7.0-latest/javadocs/portal-kernel/com/liferay/blogs/kernel/model/BlogsEntry.html}{\texttt{com.liferay.blogs.kernel.model.BlogsEntry}}.
The \texttt{className} and \texttt{classPK} attributes are required to
instantiate the Screenlet. \textbar{} \texttt{classPK} \textbar{}
\texttt{number} \textbar{} The asset's unique identifier. The
\texttt{className} and \texttt{classPK} attributes are required to
instantiate the Screenlet. \textbar{} \texttt{cachePolicy} \textbar{}
\texttt{string} \textbar{} The offline mode setting. See
\href{/docs/7-0/reference/-/knowledge_base/r/comment-add-screenlet-for-android\#offline}{the
Offline section} for details. \textbar{}

\noindent\hrulefill

\subsection{Listener}\label{listener-13}

Comment Add Screenlet delegates some events to a class that implements
\texttt{CommentAddListener}. This interface lets you implement the
following methods:

\begin{itemize}
\item
  \texttt{onAddCommentSuccess(CommentEntry\ commentEntry)}: Called when
  the Screenlet successfully adds a comment to the asset.
\item
  \texttt{error(Exception\ e,\ String\ userAction)}: Called when an
  error occurs in the process. The \texttt{userAction} argument
  distinguishes the specific action in which the error occurred.
\end{itemize}

\section{Asset Display Screenlet for
Android}\label{asset-display-screenlet-for-android}

\subsection{Requirements}\label{requirements-14}

\begin{itemize}
\tightlist
\item
  Android SDK 4.1 (API Level 16) or above
\item
  Liferay Portal 6.2 CE/EE, Liferay CE Portal 7.0/7.1, Liferay DXP 7.0+
\item
  Liferay Screens Compatibility app
  (\href{http://www.liferay.com/marketplace/-/mp/application/54365664}{CE}
  or
  \href{http://www.liferay.com/marketplace/-/mp/application/54369726}{EE/DXP}).
  This app is preinstalled in Liferay CE Portal 7.0/7.1 and Liferay DXP
  7.0+.
\end{itemize}

\subsection{Compatibility}\label{compatibility-14}

\begin{itemize}
\tightlist
\item
  Android SDK 4.1 (API Level 16) or above
\end{itemize}

\subsection{Xamarin Requirements}\label{xamarin-requirements-14}

\begin{itemize}
\tightlist
\item
  Visual Studio 7.2
\item
  Mono .NET framework 5.4.1.6
\end{itemize}

\subsection{Features}\label{features-14}

Asset Display Screenlet can display an asset from a Liferay instance.
The Screenlet can currently display Documents and Media files
(\texttt{DLFileEntry} images, videos, audio files, and PDFs), blogs
entries (\texttt{BlogsEntry}) and web content articles
(\texttt{WebContent}).

Asset Display Screenlet can also display your custom asset types. See
\href{/docs/7-0/reference/-/knowledge_base/r/asset-display-screenlet-for-android\#listener}{the
Listener section of this document} for details.

\subsection{JSON Services Used}\label{json-services-used-14}

Screenlets in Liferay Screens call JSON web services in the portal. This
Screenlet calls the following services and methods.

\noindent\hrulefill

\begin{longtable}[]{@{}
  >{\raggedright\arraybackslash}p{(\columnwidth - 4\tabcolsep) * \real{0.3889}}
  >{\raggedright\arraybackslash}p{(\columnwidth - 4\tabcolsep) * \real{0.3333}}
  >{\raggedright\arraybackslash}p{(\columnwidth - 4\tabcolsep) * \real{0.2778}}@{}}
\toprule\noalign{}
\begin{minipage}[b]{\linewidth}\raggedright
Service
\end{minipage} & \begin{minipage}[b]{\linewidth}\raggedright
Method
\end{minipage} & \begin{minipage}[b]{\linewidth}\raggedright
Notes
\end{minipage} \\
\midrule\noalign{}
\endhead
\bottomrule\noalign{}
\endlastfoot
\texttt{ScreensassetentryService} (Screens compatibility plugin) &
\texttt{getAssetEntry} & With \texttt{entryId} \\
\texttt{ScreensassetentryService} (Screens compatibility plugin) &
\texttt{getAssetEntry} & With \texttt{classPK} and \texttt{className} \\
\texttt{ScreensassetentryService} (Screens compatibility plugin) &
\texttt{getAssetEntries} & With \texttt{entryQuery} \\
\texttt{ScreensassetentryService} (Screens compatibility plugin) &
\texttt{getAssetEntries} & With \texttt{companyId}, \texttt{groupId},
and \texttt{portletItemName} \\
\end{longtable}

\noindent\hrulefill

\subsection{Module}\label{module-14}

\begin{itemize}
\tightlist
\item
  None
\end{itemize}

\subsection{Views}\label{views-14}

\begin{itemize}
\tightlist
\item
  Default
\end{itemize}

\begin{figure}
\centering
\includegraphics{./images/screens-android-assetdisplay.png}
\caption{Asset Display Screenlet using the Default View.}
\end{figure}

The Default View uses different UI elements to show each asset type. For
example, it displays images with \texttt{ImageView} and blogs with
\texttt{TextView}. Note that other Views may use different UI elements.

This Screenlet can also render other Screenlets as inner Screenlets:

\begin{itemize}
\tightlist
\item
  Images: Image Display Screenlet
\item
  Videos: Video Display Screenlet
\item
  Audio: Audio Display Screenlet
\item
  PDFs: PDF Display Screenlet
\item
  Blog entries: Blogs Entry Display Screenlet
\item
  Web content: Web Content Display Screenlet
\end{itemize}

These Screenlets can also be used alone without Asset Display Screenlet.

\subsection{Offline}\label{offline-14}

This Screenlet supports offline mode so it can function without a
network connection. For more information on how offline mode works, see
the
\href{/docs/7-0/tutorials/-/knowledge_base/t/architecture-of-offline-mode-in-liferay-screens}{tutorial
on its architecture}. Here are the offline mode policies that you can
use with this Screenlet:

\noindent\hrulefill

Policy \textbar{} What happens \textbar{} When to use \textbar{}
\texttt{REMOTE\_ONLY} \textbar{} The Screenlet loads the data from the
Liferay instance. If a connection issue occurs, the Screenlet uses the
listener to notify the developer about the error. If the Screenlet
successfully loads the data, it stores it in the local cache for later
use. \textbar{} Use this policy when you always need to show updated
data, and show nothing when there's no connection. \textbar{}
\texttt{CACHE\_ONLY} \textbar{} The Screenlet loads the data from the
local cache. If the data isn't there, the Screenlet uses the listener to
notify the developer about the error. \textbar{} Use this policy when
you always need to show local data, without retrieving remote
information under any circumstance. \textbar{} \texttt{REMOTE\_FIRST}
\textbar{} The Screenlet loads the data from the Liferay instance. If
this succeeds, the Screenlet shows the data to the user and stores it in
the local cache for later use. If a connection issue occurs, the
Screenlet retrieves the data from the local cache. If the data doesn't
exist there, the Screenlet uses the listener to notify the developer
about the error. \textbar{} Use this policy to show the most recent
version of the data when connected, but show an outdated version when
there's no connection. \textbar{} \texttt{CACHE\_FIRST} \textbar{} The
Screenlet loads the data from the local cache. If the data isn't there,
the Screenlet requests it from the Liferay instance and notifies the
developer about any errors that occur (including connectivity errors).
\textbar{} Use this policy to save bandwidth and loading time in case
you have local (but probably outdated) data. \textbar{}

\noindent\hrulefill

\subsection{Required Attributes}\label{required-attributes-14}

\begin{itemize}
\tightlist
\item
  \texttt{entryId}
\end{itemize}

Instead of \texttt{entryId}, you can use both of the following
attributes:

\begin{itemize}
\tightlist
\item
  \texttt{className}
\item
  \texttt{classPK}
\end{itemize}

If you don't use \texttt{entryId}, \texttt{className}, or
\texttt{classPK}, you must use this attribute:

\begin{itemize}
\tightlist
\item
  \texttt{portletItemName}
\end{itemize}

\subsection{Attributes}\label{attributes-14}

\noindent\hrulefill

Attribute \textbar{} Data type \textbar{} Explanation \textbar{}
\texttt{layoutId} \textbar{} \texttt{@layout} \textbar{} The layout to
use to show the View. \textbar{} \texttt{autoLoad} \textbar{}
\texttt{boolean} \textbar{} Whether the asset automatically loads when
the Screenlet appears in the app's UI. The default value is
\texttt{true}. \textbar{} \texttt{entryId} \textbar{} \texttt{number}
\textbar{} The primary key of the asset. \textbar{} \texttt{className}
\textbar{} \texttt{string} \textbar{} The asset's fully qualified class
name. For example, a blog entry's \texttt{className} is
\href{@platform-ref@/7.0-latest/javadocs/portal-kernel/com/liferay/blogs/kernel/model/BlogsEntry.html}{\texttt{com.liferay.blogs.kernel.model.BlogsEntry}}.
The \texttt{className} and \texttt{classPK} attributes are required to
instantiate the Screenlet. \textbar{} \texttt{classPK} \textbar{}
\texttt{number} \textbar{} The asset's unique identifier. The
\texttt{className} and \texttt{classPK} attributes are required to
instantiate the Screenlet. \textbar{} \texttt{portletItemName}
\textbar{} \texttt{string} \textbar{} The name of the
\href{/docs/7-0/user/-/knowledge_base/u/configuration-templates}{configuration
template} you used in the Asset Publisher. To use this feature, add an
Asset Publisher to one of your site's pages (it may be a hidden page),
configure the Asset Publisher's filter (in \emph{Configuration} →
\emph{Setup} → \emph{Asset Selection}), and then use the Asset
Publisher's \emph{Configuration Templates} option to save this
configuration with a name. Use this name in this attribute. If there is
more than one asset in the configuration, the Screenlet displays only
the first one. \textbar{} \texttt{cachePolicy} \textbar{}
\texttt{string} \textbar{} The offline mode setting. See
\href{/docs/7-0/reference/-/knowledge_base/r/asset-display-screenlet-for-android\#offline}{the
Offline section} for details. \textbar{} \texttt{imageLayoutId}
\textbar{} \texttt{@layout} \textbar{} The layout to use to show an
image (\texttt{DLFileEntry}). \textbar{} \texttt{videoLayoutId}
\textbar{} \texttt{@layout} \textbar{} The layout to use to show a video
(\texttt{DLFileEntry}). \textbar{} \texttt{audioLayoutId} \textbar{}
\texttt{@layout} \textbar{} The layout to use to show an audio file
(\texttt{DLFileEntry}). \textbar{} \texttt{pdfLayoutId} \textbar{}
\texttt{@layout} \textbar{} The layout to use to show a PDF
(\texttt{DLFileEntry}). \textbar{} \texttt{blogsLayoutId} \textbar{}
\texttt{@layout} \textbar{} The layout to use to show a blog entry
(\texttt{BlogsEntry}). \textbar{} \texttt{webDisplayLayoutId} \textbar{}
\texttt{@layout} \textbar{} The layout to use to show a web content
article (\texttt{WebContent}). \textbar{}

\noindent\hrulefill

\subsection{Methods}\label{methods-10}

\noindent\hrulefill

Method \textbar{} Return \textbar{} Explanation \textbar{}
\texttt{load(AssetEntry\ assetEntry)} \textbar{} \texttt{void}
\textbar{} Loads the given \texttt{AssetEntry} in the Screenlet. If no
inner Screenlet is instantiated, the method tries to load the asset with
a custom asset listener method. If this returns \texttt{null}, a new
\texttt{Intent} is called to display the asset. \textbar{}
\texttt{loadAsset()} \textbar{} \texttt{void} \textbar{} Searches for
the \texttt{AssetEntry} defined by the required attributes and loads it
in the Screenlet. \textbar{}

\noindent\hrulefill

\subsection{Listener}\label{listener-14}

Asset Display Screenlet delegates some events to a class that implements
\texttt{AssetDisplayListener}. This interface contains the following
methods:

\begin{itemize}
\tightlist
\item
  \texttt{onRetrieveAssetSuccess(AssetEntry\ assetEntry)}: Called when
  the Screenlet successfully loads the asset.
\end{itemize}

A second listener, \texttt{AssetDisplayInnerScreenletListener}, also
exists for configuring a child Screenlet (the Screenlet rendered inside
Asset Display Screenlet) or rendering a custom asset.

\begin{itemize}
\item
  \texttt{onConfigureChildScreenlet(AssetDisplayScreenlet\ screenlet,\ BaseScreenlet\ innerScreenlet,\ AssetEntry\ assetEntry)}:
  Called when the child Screenlet has been successfully initialized. Use
  this method to configure or customize the child Screenlet. The example
  implementation here sets the child Screenlet's background color to
  light gray if the asset is a blog entry entity (\texttt{BlogsEntry}):

\begin{verbatim}
  @Override
  public void onConfigureChildScreenlet(AssetDisplayScreenlet screenlet,
      BaseScreenlet innerScreenlet, AssetEntry assetEntry) {
          if ("blogsEntry".equals(assetEntry.getObjectType())) {
              innerScreenlet.setBackgroundColor(ContextCompat.getColor(this,
              R.color.light_gray));
          }
  }
\end{verbatim}
\item
  \texttt{onRenderCustomAsset(AssetEntry\ assetEntry)}: Called to render
  a custom asset. The following example implementation inflates and
  returns the custom View necessary to render a user from a Liferay
  instance (\texttt{User}):

\begin{verbatim}
  @Override
  public View onRenderCustomAsset(AssetEntry assetEntry) {
      if (assetEntry instanceof User) {
          View view = getLayoutInflater().inflate(R.layout.user_display, null);
          User user = (User) assetEntry;

          TextView username = (TextView) view.findViewById(R.id.liferay_username);

          username(user.getUsername());

          return view;
      }
      return null;
  }
\end{verbatim}
\end{itemize}

\section{Blogs Entry Display Screenlet for
Android}\label{blogs-entry-display-screenlet-for-android}

\subsection{Requirements}\label{requirements-15}

\begin{itemize}
\tightlist
\item
  Android SDK 4.1 (API Level 16) or above
\item
  Liferay Portal 6.2 CE/EE, Liferay CE Portal 7.0/7.1, Liferay DXP 7.0+
\item
  Liferay Screens Compatibility app
  (\href{http://www.liferay.com/marketplace/-/mp/application/54365664}{CE}
  or
  \href{http://www.liferay.com/marketplace/-/mp/application/54369726}{EE/DXP}).
  This app is preinstalled in Liferay CE Portal 7.0/7.1 and Liferay DXP
  7.0+.
\end{itemize}

\subsection{Compatibility}\label{compatibility-15}

\begin{itemize}
\tightlist
\item
  Android SDK 4.1 (API Level 16) or above
\end{itemize}

\subsection{Xamarin Requirements}\label{xamarin-requirements-15}

\begin{itemize}
\tightlist
\item
  Visual Studio 7.2
\item
  Mono .NET framework 5.4.1.6
\end{itemize}

\subsection{Features}\label{features-15}

Blogs Entry Display Screenlet displays a single blog entry. Image
Display Screenlet renders any header image the blogs entry may have.

\subsection{JSON Services Used}\label{json-services-used-15}

Screenlets in Liferay Screens call JSON web services in the portal. This
Screenlet calls the following services and methods.

\noindent\hrulefill

\begin{longtable}[]{@{}
  >{\raggedright\arraybackslash}p{(\columnwidth - 4\tabcolsep) * \real{0.3889}}
  >{\raggedright\arraybackslash}p{(\columnwidth - 4\tabcolsep) * \real{0.3333}}
  >{\raggedright\arraybackslash}p{(\columnwidth - 4\tabcolsep) * \real{0.2778}}@{}}
\toprule\noalign{}
\begin{minipage}[b]{\linewidth}\raggedright
Service
\end{minipage} & \begin{minipage}[b]{\linewidth}\raggedright
Method
\end{minipage} & \begin{minipage}[b]{\linewidth}\raggedright
Notes
\end{minipage} \\
\midrule\noalign{}
\endhead
\bottomrule\noalign{}
\endlastfoot
\texttt{ScreensassetentryService} (Screens compatibility plugin) &
\texttt{getAssetEntry} & With \texttt{entryId} \\
\texttt{ScreensassetentryService} (Screens compatibility plugin) &
\texttt{getAssetEntry} & With \texttt{classPK} and \texttt{className} \\
\texttt{ScreensassetentryService} (Screens compatibility plugin) &
\texttt{getAssetEntries} & With \texttt{entryQuery} \\
\texttt{ScreensassetentryService} (Screens compatibility plugin) &
\texttt{getAssetEntries} & With \texttt{companyId}, \texttt{groupId},
and \texttt{portletItemName} \\
\end{longtable}

\noindent\hrulefill

\subsection{Module}\label{module-15}

\begin{itemize}
\tightlist
\item
  None
\end{itemize}

\subsection{Views}\label{views-15}

\begin{itemize}
\tightlist
\item
  Default
\end{itemize}

The Default View uses different components to show a blogs entry
(\texttt{BlogsEntry}). For example, it uses an Android \texttt{TextView}
to show the blog's text, and
\href{/docs/7-0/reference/-/knowledge_base/r/userportraitscreenlet-for-android}{User
Portrait Screenlet} to show the profile picture of the Liferay user who
posted it. Note that other custom Views may use different components.

\begin{figure}
\centering
\includegraphics{./images/screens-android-blogsentrydisplay.png}
\caption{Blogs Entry Display Screenlet using the Default View.}
\end{figure}

\subsection{Offline}\label{offline-15}

This Screenlet supports offline mode so it can function without a
network connection. For more information on how offline mode works, see
the
\href{/docs/7-0/tutorials/-/knowledge_base/t/architecture-of-offline-mode-in-liferay-screens}{tutorial
on its architecture}. Here are the offline mode policies that you can
use with this Screenlet:

\noindent\hrulefill

Policy \textbar{} What happens \textbar{} When to use \textbar{}
\texttt{REMOTE\_ONLY} \textbar{} The Screenlet loads the data from the
Liferay instance. If a connection issue occurs, the Screenlet uses the
listener to notify the developer about the error. If the Screenlet
successfully loads the data, it stores it in the local cache for later
use. \textbar{} Use this policy when you always need to show updated
data, and show nothing when there's no connection. \textbar{}
\texttt{CACHE\_ONLY} \textbar{} The Screenlet loads the data from the
local cache. If the data isn't there, the Screenlet uses the listener to
notify the developer about the error. \textbar{} Use this policy when
you always need to show local data, without retrieving remote data under
any circumstance. \textbar{} \texttt{REMOTE\_FIRST} \textbar{} The
Screenlet loads the data from the Liferay instance. If this succeeds,
the Screenlet shows the data to the user and stores it in the local
cache for later use. If a connection issue occurs, the Screenlet
retrieves the data from the local cache. If the data doesn't exist
there, the Screenlet uses the listener to notify the developer about the
error. \textbar{} Use this policy to show the most recent version of the
data when connected, but show an outdated version when there's no
connection. \textbar{} \texttt{CACHE\_FIRST} \textbar{} The Screenlet
loads the data from the local cache. If the data isn't there, the
Screenlet requests it from the Liferay instance and notifies the
developer about any errors that occur (including connectivity errors).
\textbar{} Use this policy to save bandwidth and loading time in case
you have local (but probably outdated) data. \textbar{}

\noindent\hrulefill

\subsection{Required Attributes}\label{required-attributes-15}

\begin{itemize}
\tightlist
\item
  \texttt{entryId}
\end{itemize}

If you don't use \texttt{entryId}, you must use both of the following
attributes:

\begin{itemize}
\tightlist
\item
  \texttt{className}
\item
  \texttt{classPK}
\end{itemize}

\subsection{Attributes}\label{attributes-15}

\noindent\hrulefill

Attribute \textbar{} Data type \textbar{} Explanation \textbar{}
\texttt{layoutId} \textbar{} \texttt{@layout} \textbar{} The layout to
use to show the View.\textbar{} \texttt{autoLoad} \textbar{}
\texttt{boolean} \textbar{} Whether the blog entry automatically loads
when the Screenlet appears in the app's UI. The default value is
\texttt{true}. \textbar{} \texttt{entryId} \textbar{} \texttt{number}
\textbar{} The primary key of the blog entry (\texttt{BlogsEntry}).
\textbar{} \texttt{className} \textbar{} \texttt{string} \textbar{} The
\texttt{BlogsEntry} object's fully qualified class name. This is
\href{@platform-ref@/7.0-latest/javadocs/portal-kernel/com/liferay/blogs/kernel/model/BlogsEntry.html}{\texttt{com.liferay.blogs.kernel.model.BlogsEntry}}.
If you don't use \texttt{entryId}, the \texttt{className} and
\texttt{classPK} attributes are required to instantiate the Screenlet.
\textbar{} \texttt{classPK} \textbar{} \texttt{number} \textbar{} The
\texttt{BlogsEntry} object's unique identifier. If you don't use
\texttt{entryId}, the \texttt{className} and \texttt{classPK} attributes
are required to instantiate the Screenlet. \textbar{}
\texttt{cachePolicy} \textbar{} \texttt{string} \textbar{} The offline
mode setting. See
\href{/docs/7-0/reference/-/knowledge_base/r/blogs-entry-display-screenlet-for-android\#offline}{the
Offline section} for details. \textbar{}

\noindent\hrulefill

\subsection{Listener}\label{listener-15}

Because a blog entry is an asset, Blogs Entry Display Screenlet
delegates its events to a class that implements
\texttt{AssetDisplayListener}. This interface lets you implement the
following method:

\begin{itemize}
\item
  \texttt{onRetrieveAssetSuccess(AssetEntry\ assetEntry)}: Called when
  the Screenlet successfully loads the blog entry.
\item
  \texttt{error(Exception\ e,\ String\ userAction)}: Called when an
  error occurs in the process. The \texttt{userAction} argument
  distinguishes the specific action in which the error occurred.
\end{itemize}

\section{Image Display Screenlet for
Android}\label{image-display-screenlet-for-android}

\subsection{Requirements}\label{requirements-16}

\begin{itemize}
\tightlist
\item
  Android SDK 4.1 (API Level 16) or above
\item
  Liferay Portal 6.2 CE/EE, Liferay CE Portal 7.0/7.1, Liferay DXP 7.0+
\item
  Liferay Screens Compatibility app
  (\href{http://www.liferay.com/marketplace/-/mp/application/54365664}{CE}
  or
  \href{http://www.liferay.com/marketplace/-/mp/application/54369726}{EE/DXP}).
  This app is preinstalled in Liferay CE Portal 7.0/7.1 and Liferay DXP
  7.0+.
\end{itemize}

\subsection{Compatibility}\label{compatibility-16}

\begin{itemize}
\tightlist
\item
  Android SDK 4.1 (API Level 16) or above
\end{itemize}

\subsection{Xamarin Requirements}\label{xamarin-requirements-16}

\begin{itemize}
\tightlist
\item
  Visual Studio 7.2
\item
  Mono .NET framework 5.4.1.6
\end{itemize}

\subsection{Features}\label{features-16}

Image Display Screenlet displays an image file from a Liferay instance's
Documents and Media Library.

\subsection{JSON Services Used}\label{json-services-used-16}

Screenlets in Liferay Screens call JSON web services in the portal. This
Screenlet calls the following services and methods.

\noindent\hrulefill

\begin{longtable}[]{@{}
  >{\raggedright\arraybackslash}p{(\columnwidth - 4\tabcolsep) * \real{0.3889}}
  >{\raggedright\arraybackslash}p{(\columnwidth - 4\tabcolsep) * \real{0.3333}}
  >{\raggedright\arraybackslash}p{(\columnwidth - 4\tabcolsep) * \real{0.2778}}@{}}
\toprule\noalign{}
\begin{minipage}[b]{\linewidth}\raggedright
Service
\end{minipage} & \begin{minipage}[b]{\linewidth}\raggedright
Method
\end{minipage} & \begin{minipage}[b]{\linewidth}\raggedright
Notes
\end{minipage} \\
\midrule\noalign{}
\endhead
\bottomrule\noalign{}
\endlastfoot
\texttt{ScreensassetentryService} (Screens compatibility plugin) &
\texttt{getAssetEntry} & With \texttt{entryId} \\
\texttt{ScreensassetentryService} (Screens compatibility plugin) &
\texttt{getAssetEntry} & With \texttt{classPK} and \texttt{className} \\
\texttt{ScreensassetentryService} (Screens compatibility plugin) &
\texttt{getAssetEntries} & With \texttt{entryQuery} \\
\texttt{ScreensassetentryService} (Screens compatibility plugin) &
\texttt{getAssetEntries} & With \texttt{companyId}, \texttt{groupId},
and \texttt{portletItemName} \\
\end{longtable}

\noindent\hrulefill

\subsection{Module}\label{module-16}

\begin{itemize}
\tightlist
\item
  None
\end{itemize}

\subsection{Views}\label{views-16}

\begin{itemize}
\tightlist
\item
  Default
\end{itemize}

The Default View uses an Android \texttt{ImageView} to display the
image.

\begin{figure}
\centering
\includegraphics{./images/screens-android-imagedisplay.png}
\caption{Image Display Screenlet using the Default View.}
\end{figure}

\subsection{Offline}\label{offline-16}

This Screenlet supports offline mode so it can function without a
network connection. For more information on how offline mode works, see
the
\href{/docs/7-0/tutorials/-/knowledge_base/t/architecture-of-offline-mode-in-liferay-screens}{tutorial
on its architecture}. Here are the offline mode policies that you can
use with this Screenlet:

\noindent\hrulefill

Policy \textbar{} What happens \textbar{} When to use \textbar{}
\texttt{REMOTE\_ONLY} \textbar{} The Screenlet loads the data from the
Liferay instance. If a connection issue occurs, the Screenlet uses the
listener to notify the developer about the error. If the Screenlet
successfully loads the data, it stores it in the local cache for later
use. \textbar{} Use this policy when you always need to show updated
data, and show nothing when there's no connection. \textbar{}
\texttt{CACHE\_ONLY} \textbar{} The Screenlet loads the data from the
local cache. If the data isn't there, the Screenlet uses the listener to
notify the developer about the error. \textbar{} Use this policy when
you always need to show local data, without retrieving remote
information under any circumstance. \textbar{} \texttt{REMOTE\_FIRST}
\textbar{} The Screenlet loads the data from the Liferay instance. If
this succeeds, the Screenlet shows the data to the user and stores it in
the local cache for later use. If a connection issue occurs, the
Screenlet retrieves the data from the local cache. If the data doesn't
exist there, the Screenlet uses the listener to notify the developer
about the error. \textbar{} Use this policy to show the most recent
version of the data when connected, but show an outdated version when
there's no connection. \textbar{} \texttt{CACHE\_FIRST} \textbar{} The
Screenlet loads the data from the local cache. If the data isn't there,
the Screenlet requests it from the Liferay instance and notifies the
developer about any errors that occur (including connectivity errors).
\textbar{} Use this policy to save bandwidth and loading time in case
you have local (but probably outdated) data. \textbar{}

\noindent\hrulefill

\subsection{Required Attributes}\label{required-attributes-16}

\begin{itemize}
\tightlist
\item
  \texttt{entryId} or \texttt{classPK}
\end{itemize}

\subsection{Attributes}\label{attributes-16}

\noindent\hrulefill

Attribute \textbar{} Data type \textbar{} Explanation \textbar{}
\texttt{layoutId} \textbar{} \texttt{@layout} \textbar{} The layout to
use to show the View. \textbar{} \texttt{autoLoad} \textbar{}
\texttt{boolean} \textbar{} Whether the image automatically loads when
the Screenlet appears in the app's UI. The default value is
\texttt{true}. \textbar{} \texttt{entryId} \textbar{} \texttt{number}
\textbar{} The primary key of the image. \textbar{} \texttt{classPK}
\textbar{} \texttt{number} \textbar{} The image's unique identifier.
\textbar{} \texttt{cachePolicy} \textbar{} \texttt{string} \textbar{}
The offline mode setting. See
\href{/docs/7-0/reference/-/knowledge_base/r/image-display-screenlet-for-android\#offline}{the
Offline section} for details. \textbar{} \texttt{imageScaleType}
\textbar{} \texttt{number} \textbar{} Lets you set a scale image type
like \texttt{CENTER}, \texttt{CENTER\_CROP}, \texttt{CENTER\_INSIDE},
\texttt{FIT\_CENTER}, \texttt{FIT\_END}, \texttt{FIT\_START},
\texttt{FIT\_XY}, \texttt{MATRIX}. \textbar{} \texttt{placeHolder}
\textbar{} \texttt{@resource} \textbar{} Image to load until the final
image loads. \textbar{} \texttt{placeHolderScaleType} \textbar{}
\texttt{number} \textbar{} Lets you set a scale image type for the
placeholder like \texttt{CENTER}, \texttt{CENTER\_CROP},
\texttt{CENTER\_INSIDE}, \texttt{FIT\_CENTER}, \texttt{FIT\_END},
\texttt{FIT\_START}, \texttt{FIT\_XY}, \texttt{MATRIX}. \textbar{}

\noindent\hrulefill

Note that the values for \texttt{imageScaleType} and
\texttt{placeHolderScaleType} match those
\href{https://developer.android.com/reference/android/widget/ImageView.ScaleType.html}{described
in Android's \texttt{ImageView.ScaleType}}.

\subsection{Listener}\label{listener-16}

Because images are assets, Image Display Screenlet delegates its events
to a class that implements \texttt{AssetDisplayListener}. This interface
lets you implement the following methods:

\begin{itemize}
\item
  \texttt{onRetrieveAssetSuccess(AssetEntry\ assetEntry)}: Called when
  the Screenlet successfully loads the image.
\item
  \texttt{error(Exception\ e,\ String\ userAction)}: Called when an
  error occurs in the process. The \texttt{userAction} argument
  distinguishes the specific action in which the error occurred.
\end{itemize}

\section{Video Display Screenlet for
Android}\label{video-display-screenlet-for-android}

\subsection{Requirements}\label{requirements-17}

\begin{itemize}
\tightlist
\item
  Android SDK 4.1 (API Level 16) or above
\item
  Liferay Portal 6.2 CE/EE, Liferay CE Portal 7.0/7.1, Liferay DXP 7.0+
\item
  Liferay Screens Compatibility app
  (\href{http://www.liferay.com/marketplace/-/mp/application/54365664}{CE}
  or
  \href{http://www.liferay.com/marketplace/-/mp/application/54369726}{EE/DXP}).
  This app is preinstalled in Liferay CE Portal 7.0/7.1 and Liferay DXP
  7.0+.
\end{itemize}

\subsection{Compatibility}\label{compatibility-17}

\begin{itemize}
\tightlist
\item
  Android SDK 4.1 (API Level 16) or above
\end{itemize}

\subsection{Xamarin Requirements}\label{xamarin-requirements-17}

\begin{itemize}
\tightlist
\item
  Visual Studio 7.2
\item
  Mono .NET framework 5.4.1.6
\end{itemize}

\subsection{Features}\label{features-17}

Video Display Screenlet displays a video file from a Liferay instance's
Documents and Media Library.

\subsection{JSON Services Used}\label{json-services-used-17}

Screenlets in Liferay Screens call JSON web services in the portal. This
Screenlet calls the following services and methods.

\noindent\hrulefill

\begin{longtable}[]{@{}
  >{\raggedright\arraybackslash}p{(\columnwidth - 4\tabcolsep) * \real{0.3889}}
  >{\raggedright\arraybackslash}p{(\columnwidth - 4\tabcolsep) * \real{0.3333}}
  >{\raggedright\arraybackslash}p{(\columnwidth - 4\tabcolsep) * \real{0.2778}}@{}}
\toprule\noalign{}
\begin{minipage}[b]{\linewidth}\raggedright
Service
\end{minipage} & \begin{minipage}[b]{\linewidth}\raggedright
Method
\end{minipage} & \begin{minipage}[b]{\linewidth}\raggedright
Notes
\end{minipage} \\
\midrule\noalign{}
\endhead
\bottomrule\noalign{}
\endlastfoot
\texttt{ScreensassetentryService} (Screens compatibility plugin) &
\texttt{getAssetEntry} & With \texttt{entryId} \\
\texttt{ScreensassetentryService} (Screens compatibility plugin) &
\texttt{getAssetEntry} & With \texttt{classPK} and \texttt{className} \\
\texttt{ScreensassetentryService} (Screens compatibility plugin) &
\texttt{getAssetEntries} & With \texttt{entryQuery} \\
\texttt{ScreensassetentryService} (Screens compatibility plugin) &
\texttt{getAssetEntries} & With \texttt{companyId}, \texttt{groupId},
and \texttt{portletItemName} \\
\end{longtable}

\noindent\hrulefill

\subsection{Module}\label{module-17}

\begin{itemize}
\tightlist
\item
  None
\end{itemize}

\subsection{Views}\label{views-17}

\begin{itemize}
\tightlist
\item
  Default
\end{itemize}

The Default View uses an Android \texttt{VideoView} to display the
video.

\begin{figure}
\centering
\includegraphics{./images/screens-android-videodisplay.png}
\caption{Video Display Screenlet using the Default View.}
\end{figure}

\subsection{Offline}\label{offline-17}

This Screenlet supports offline mode so it can function without a
network connection. For more information on how offline mode works, see
the
\href{/docs/7-0/tutorials/-/knowledge_base/t/architecture-of-offline-mode-in-liferay-screens}{tutorial
on its architecture}. Here are the offline mode policies that you can
use with this Screenlet:

\noindent\hrulefill

Policy \textbar{} What happens \textbar{} When to use \textbar{}
\texttt{REMOTE\_ONLY} \textbar{} The Screenlet loads the data from the
Liferay instance. If a connection issue occurs, the Screenlet uses the
listener to notify the developer about the error. If the Screenlet
successfully loads the data, it stores it in the local cache for later
use. \textbar{} Use this policy when you always need to show updated
data, and show nothing when there's no connection. \textbar{}
\texttt{CACHE\_ONLY} \textbar{} The Screenlet loads the data from the
local cache. If the data isn't there, the Screenlet uses the listener to
notify the developer about the error. \textbar{} Use this policy when
you always need to show local data, without retrieving remote
information under any circumstance. \textbar{} \texttt{REMOTE\_FIRST}
\textbar{} The Screenlet loads the data from the Liferay instance. If
this succeeds, the Screenlet shows the data to the user and stores it in
the local cache for later use. If a connection issue occurs, the
Screenlet retrieves the data from the local cache. If the data doesn't
exist there, the Screenlet uses the listener to notify the developer
about the error. \textbar{} Use this policy to show the most recent
version of the data when connected, but show an outdated version when
there's no connection. \textbar{} \texttt{CACHE\_FIRST} \textbar{} The
Screenlet loads the data from the local cache. If the data isn't there,
the Screenlet requests it from the Liferay instance and notifies the
developer about any errors that occur (including connectivity errors).
\textbar{} Use this policy to save bandwidth and loading time in case
you have local (but probably outdated) data. \textbar{}

\noindent\hrulefill

\subsection{Required Attributes}\label{required-attributes-17}

\begin{itemize}
\tightlist
\item
  \texttt{entryId} or \texttt{classPK}
\end{itemize}

\subsection{Attributes}\label{attributes-17}

\noindent\hrulefill

Attribute \textbar{} Data type \textbar{} Explanation \textbar{}
\texttt{layoutId} \textbar{} \texttt{@layout} \textbar{} The layout to
use to show the View. \textbar{} \texttt{autoLoad} \textbar{}
\texttt{boolean} \textbar{} Whether the video automatically loads when
the Screenlet appears in the app's UI. The default value is
\texttt{true}. \textbar{} \texttt{entryId} \textbar{} \texttt{number}
\textbar{} The primary key of the video file. \textbar{}
\texttt{classPK} \textbar{} \texttt{number} \textbar{} The video file's
unique identifier. \textbar{} \texttt{cachePolicy} \textbar{}
\texttt{string} \textbar{} The offline mode setting. See
\href{/docs/7-0/reference/-/knowledge_base/r/video-display-screenlet-for-android\#offline}{the
Offline section} for details. \textbar{}

\noindent\hrulefill

\subsection{Listener}\label{listener-17}

Video Display Screenlet delegates its events to a class that implements
\texttt{VideoDisplayListener}. This interface lets you implement these
methods:

\begin{itemize}
\item
  \texttt{onVideoPrepared()}: Called when the video is ready for
  display.
\item
  \texttt{onVideoCompleted()}: Called when the video is completed.
\item
  \texttt{onVideoError(Exception\ e)}: Called when an error occurs
  displaying the video.
\item
  \texttt{onRetrieveAssetSuccess(AssetEntry\ assetEntry)}: Called when
  the Screenlet successfully loads the video.
\item
  \texttt{error(Exception\ e,\ String\ userAction)}: Called when an
  error occurs in the process. The \texttt{userAction} argument
  distinguishes the specific action in which the error occurred.
\end{itemize}

\section{Audio Display Screenlet for
Android}\label{audio-display-screenlet-for-android}

\subsection{Requirements}\label{requirements-18}

\begin{itemize}
\tightlist
\item
  Android SDK 4.1 (API Level 16) or above
\item
  Liferay Portal 6.2 CE/EE, Liferay CE Portal 7.0/7.1, Liferay DXP 7.0+
\item
  Liferay Screens Compatibility app
  (\href{http://www.liferay.com/marketplace/-/mp/application/54365664}{CE}
  or
  \href{http://www.liferay.com/marketplace/-/mp/application/54369726}{EE/DXP}).
  This app is preinstalled in Liferay CE Portal 7.0/7.1 and Liferay DXP
  7.0+.
\end{itemize}

\subsection{Compatibility}\label{compatibility-18}

\begin{itemize}
\tightlist
\item
  Android SDK 4.1 (API Level 16) or above
\end{itemize}

\subsection{Xamarin Requirements}\label{xamarin-requirements-18}

\begin{itemize}
\tightlist
\item
  Visual Studio 7.2
\item
  Mono .NET framework 5.4.1.6
\end{itemize}

\subsection{Features}\label{features-18}

Audio Display Screenlet displays an audio file from a Liferay instance's
Documents and Media Library.

\subsection{JSON Services Used}\label{json-services-used-18}

Screenlets in Liferay Screens call JSON web services in the portal. This
Screenlet calls the following services and methods.

\noindent\hrulefill

\begin{longtable}[]{@{}
  >{\raggedright\arraybackslash}p{(\columnwidth - 4\tabcolsep) * \real{0.3889}}
  >{\raggedright\arraybackslash}p{(\columnwidth - 4\tabcolsep) * \real{0.3333}}
  >{\raggedright\arraybackslash}p{(\columnwidth - 4\tabcolsep) * \real{0.2778}}@{}}
\toprule\noalign{}
\begin{minipage}[b]{\linewidth}\raggedright
Service
\end{minipage} & \begin{minipage}[b]{\linewidth}\raggedright
Method
\end{minipage} & \begin{minipage}[b]{\linewidth}\raggedright
Notes
\end{minipage} \\
\midrule\noalign{}
\endhead
\bottomrule\noalign{}
\endlastfoot
\texttt{ScreensassetentryService} (Screens compatibility plugin) &
\texttt{getAssetEntry} & With \texttt{entryId} \\
\texttt{ScreensassetentryService} (Screens compatibility plugin) &
\texttt{getAssetEntry} & With \texttt{classPK} and \texttt{className} \\
\texttt{ScreensassetentryService} (Screens compatibility plugin) &
\texttt{getAssetEntries} & With \texttt{entryQuery} \\
\texttt{ScreensassetentryService} (Screens compatibility plugin) &
\texttt{getAssetEntries} & With \texttt{companyId}, \texttt{groupId},
and \texttt{portletItemName} \\
\end{longtable}

\noindent\hrulefill

\subsection{Module}\label{module-18}

\begin{itemize}
\tightlist
\item
  None
\end{itemize}

\subsection{Views}\label{views-18}

\begin{itemize}
\tightlist
\item
  Default
\end{itemize}

The Default View uses an Android \texttt{VideoView} to display the audio
file.

\begin{figure}
\centering
\includegraphics{./images/screens-android-audiodisplay.png}
\caption{Audio Display Screenlet using the Default View.}
\end{figure}

\subsection{Offline}\label{offline-18}

This Screenlet supports offline mode so it can function without a
network connection. For more information on how offline mode works, see
the
\href{/docs/7-0/tutorials/-/knowledge_base/t/architecture-of-offline-mode-in-liferay-screens}{tutorial
on its architecture}. Here are the offline mode policies that you can
use with this Screenlet:

\noindent\hrulefill

Policy \textbar{} What happens \textbar{} When to use \textbar{}
\texttt{REMOTE\_ONLY} \textbar{} The Screenlet loads the data from the
Liferay instance. If a connection issue occurs, the Screenlet uses the
listener to notify the developer about the error. If the Screenlet
successfully loads the data, it stores it in the local cache for later
use. \textbar{} Use this policy when you always need to show updated
data, and show nothing when there's no connection. \textbar{}
\texttt{CACHE\_ONLY} \textbar{} The Screenlet loads the data from the
local cache. If the data isn't there, the Screenlet uses the listener to
notify the developer about the error. \textbar{} Use this policy when
you always need to show local data, without retrieving remote
information under any circumstance. \textbar{} \texttt{REMOTE\_FIRST}
\textbar{} The Screenlet loads the data from the Liferay instance. If
this succeeds, the Screenlet shows the data to the user and stores it in
the local cache for later use. If a connection issue occurs, the
Screenlet retrieves the data from the local cache. If the data doesn't
exist there, the Screenlet uses the listener to notify the developer
about the error. \textbar{} Use this policy to show the most recent
version of the data when connected, but show an outdated version when
there's no connection. \textbar{} \texttt{CACHE\_FIRST} \textbar{} The
Screenlet loads the data from the local cache. If the data isn't there,
the Screenlet requests it from the Liferay instance and notifies the
developer about any errors that occur (including connectivity errors).
\textbar{} Use this policy to save bandwidth and loading time in case
you have local (but probably outdated) data. \textbar{}

\noindent\hrulefill

\subsection{Required Attributes}\label{required-attributes-18}

\begin{itemize}
\tightlist
\item
  \texttt{entryId} or \texttt{classPK}
\end{itemize}

\subsection{Attributes}\label{attributes-18}

\noindent\hrulefill

Attribute \textbar{} Data type \textbar{} Explanation \textbar{}
\texttt{layoutId} \textbar{} \texttt{@layout} \textbar{} The layout to
use to show the View. \textbar{} \texttt{autoLoad} \textbar{}
\texttt{boolean} \textbar{} Whether the audio file automatically loads
when the Screenlet appears in the app's UI. The default value is
\texttt{true}. \textbar{} \texttt{entryId} \textbar{} \texttt{number}
\textbar{} The primary key of the audio file. \textbar{}
\texttt{classPK} \textbar{} \texttt{number} \textbar{} The audio file's
unique identifier. \textbar{} \texttt{cachePolicy} \textbar{}
\texttt{string} \textbar{} The offline mode setting. See
\href{/docs/7-0/reference/-/knowledge_base/r/audio-display-screenlet-for-android\#offline}{the
Offline section} for details. \textbar{}

\noindent\hrulefill

\subsection{Listener}\label{listener-18}

Because audio files are assets, Audio Display Screenlet delegates its
events to a class that implements \texttt{AssetDisplayListener}. This
interface lets you implement the following methods:

\begin{itemize}
\item
  \texttt{onRetrieveAssetSuccess(AssetEntry\ assetEntry)}: Called when
  the Screenlet successfully loads the audio file.
\item
  \texttt{error(Exception\ e,\ String\ userAction)}: Called when an
  error occurs in the process. The \texttt{userAction} argument
  distinguishes the specific action in which the error occurred.
\end{itemize}

\section{PDF Display Screenlet for
Android}\label{pdf-display-screenlet-for-android}

\subsection{Requirements}\label{requirements-19}

\begin{itemize}
\tightlist
\item
  Android SDK 4.1 (API Level 16) or above
\item
  Liferay Portal 6.2 CE/EE, Liferay CE Portal 7.0/7.1, Liferay DXP 7.0+
\item
  Liferay Screens Compatibility app
  (\href{http://www.liferay.com/marketplace/-/mp/application/54365664}{CE}
  or
  \href{http://www.liferay.com/marketplace/-/mp/application/54369726}{EE/DXP}).
  This app is preinstalled in Liferay CE Portal 7.0/7.1 and Liferay DXP
  7.0+.
\end{itemize}

\subsection{Compatibility}\label{compatibility-19}

\begin{itemize}
\tightlist
\item
  Android SDK 4.1 (API Level 16) or above
\end{itemize}

\subsection{Xamarin Requirements}\label{xamarin-requirements-19}

\begin{itemize}
\tightlist
\item
  Visual Studio 7.2
\item
  Mono .NET framework 5.4.1.6
\end{itemize}

\subsection{Features}\label{features-19}

PDF Display Screenlet displays a PDF file from a Liferay Instance's
Documents and Media Library.

\subsection{JSON Services Used}\label{json-services-used-19}

Screenlets in Liferay Screens call JSON web services in the portal. This
Screenlet calls the following services and methods.

\noindent\hrulefill

\begin{longtable}[]{@{}
  >{\raggedright\arraybackslash}p{(\columnwidth - 4\tabcolsep) * \real{0.3889}}
  >{\raggedright\arraybackslash}p{(\columnwidth - 4\tabcolsep) * \real{0.3333}}
  >{\raggedright\arraybackslash}p{(\columnwidth - 4\tabcolsep) * \real{0.2778}}@{}}
\toprule\noalign{}
\begin{minipage}[b]{\linewidth}\raggedright
Service
\end{minipage} & \begin{minipage}[b]{\linewidth}\raggedright
Method
\end{minipage} & \begin{minipage}[b]{\linewidth}\raggedright
Notes
\end{minipage} \\
\midrule\noalign{}
\endhead
\bottomrule\noalign{}
\endlastfoot
\texttt{ScreensassetentryService} (Screens compatibility plugin) &
\texttt{getAssetEntry} & With \texttt{entryId} \\
\texttt{ScreensassetentryService} (Screens compatibility plugin) &
\texttt{getAssetEntry} & With \texttt{classPK} and \texttt{className} \\
\texttt{ScreensassetentryService} (Screens compatibility plugin) &
\texttt{getAssetEntries} & With \texttt{entryQuery} \\
\texttt{ScreensassetentryService} (Screens compatibility plugin) &
\texttt{getAssetEntries} & With \texttt{companyId}, \texttt{groupId},
and \texttt{portletItemName} \\
\end{longtable}

\noindent\hrulefill

\subsection{Module}\label{module-19}

\begin{itemize}
\tightlist
\item
  None
\end{itemize}

\subsection{Views}\label{views-19}

\begin{itemize}
\tightlist
\item
  Default
\end{itemize}

The Default View uses Android's \texttt{PdfRenderer} to display the PDF.
Note that \texttt{PdfRenderer} requires an Android API level of 21 or
higher.

\begin{figure}
\centering
\includegraphics{./images/screens-android-pdfdisplay.png}
\caption{PDF Display Screenlet using the Default View.}
\end{figure}

\subsection{Offline}\label{offline-19}

This Screenlet supports offline mode so it can function without a
network connection. For more information on how offline mode works, see
the
\href{/docs/7-0/tutorials/-/knowledge_base/t/architecture-of-offline-mode-in-liferay-screens}{tutorial
on its architecture}. Here are the offline mode policies that you can
use with this Screenlet:

\noindent\hrulefill

Policy \textbar{} What happens \textbar{} When to use \textbar{}
\texttt{REMOTE\_ONLY} \textbar{} The Screenlet loads the data from the
Liferay instance. If a connection issue occurs, the Screenlet uses the
listener to notify the developer about the error. If the Screenlet
successfully loads the data, it stores it in the local cache for later
use. \textbar{} Use this policy when you always need to show updated
data, and show nothing when there's no connection. \textbar{}
\texttt{CACHE\_ONLY} \textbar{} The Screenlet loads the data from the
local cache. If the data isn't there, the Screenlet uses the listener to
notify the developer about the error. \textbar{} Use this policy when
you always need to show local data, without retrieving remote
information under any circumstance. \textbar{} \texttt{REMOTE\_FIRST}
\textbar{} The Screenlet loads the data from the Liferay instance. If
this succeeds, the Screenlet shows the data to the user and stores it in
the local cache for later use. If a connection issue occurs, the
Screenlet retrieves the data from the local cache. If the data doesn't
exist there, the Screenlet uses the listener to notify the developer
about the error. \textbar{} Use this policy to show the most recent
version of the data when connected, but show an outdated version when
there's no connection. \textbar{} \texttt{CACHE\_FIRST} \textbar{} The
Screenlet loads the data from the local cache. If the data isn't there,
the Screenlet requests it from the Liferay instance and notifies the
developer about any errors that occur (including connectivity errors).
\textbar{} Use this policy to save bandwidth and loading time in case
you have local (but probably outdated) data. \textbar{}

\noindent\hrulefill

\subsection{Required Attributes}\label{required-attributes-19}

\begin{itemize}
\tightlist
\item
  \texttt{entryId} or \texttt{classPK}
\end{itemize}

\subsection{Attributes}\label{attributes-19}

\noindent\hrulefill

Attribute \textbar{} Data type \textbar{} Explanation \textbar{}
\texttt{layoutId} \textbar{} \texttt{@layout} \textbar{} The layout to
use to show the View. \textbar{} \texttt{autoLoad} \textbar{}
\texttt{boolean} \textbar{} Whether the PDF automatically loads when the
Screenlet appears in the app's UI. The default value is \texttt{true}.
\textbar{} \texttt{entryId} \textbar{} \texttt{number} \textbar{} The
primary key of the PDF file. \textbar{} \texttt{classPK} \textbar{}
\texttt{number} \textbar{} The PDF file's unique identifier. \textbar{}
\texttt{cachePolicy} \textbar{} \texttt{string} \textbar{} The offline
mode setting. See
\href{/docs/7-0/reference/-/knowledge_base/r/pdf-display-screenlet-for-android\#offline}{the
Offline section} for details. \textbar{}

\noindent\hrulefill

\subsection{Listener}\label{listener-19}

Because PDF files are assets, PDF Display Screenlet delegates its events
to a class that implements \texttt{AssetDisplayListener}. This interface
lets you implement the following methods:

\begin{itemize}
\item
  \texttt{onRetrieveAssetSuccess(AssetEntry\ assetEntry)}: Called when
  the Screenlet successfully loads the PDF file.
\item
  \texttt{error(Exception\ e,\ String\ userAction)}: Called when an
  error occurs in the process. The \texttt{userAction} argument
  distinguishes the specific action in which the error occurred.
\end{itemize}

\section{Web Screenlet for Android}\label{web-screenlet-for-android}

\subsection{Requirements}\label{requirements-20}

\begin{itemize}
\tightlist
\item
  Android SDK 4.1 (API Level 16) or above
\item
  Liferay Portal 6.2 CE/EE, Liferay CE Portal 7.0/7.1, Liferay DXP 7.0+
\item
  Liferay Screens Compatibility app
  (\href{http://www.liferay.com/marketplace/-/mp/application/54365664}{CE}
  or
  \href{http://www.liferay.com/marketplace/-/mp/application/54369726}{EE/DXP}).
  This app is preinstalled in Liferay CE Portal 7.0/7.1 and Liferay DXP
  7.0+.
\end{itemize}

\subsection{Compatibility}\label{compatibility-20}

\begin{itemize}
\tightlist
\item
  Android SDK 4.1 (API Level 16) or above
\end{itemize}

\subsection{Xamarin Requirements}\label{xamarin-requirements-20}

\begin{itemize}
\tightlist
\item
  Visual Studio 7.2
\item
  Mono .NET framework 5.4.1.6
\end{itemize}

\subsection{Features}\label{features-20}

Web Screenlet lets you display any web page. It also lets you customize
the web page through injection of local and remote JavaScript and CSS
files. If you're using Liferay DXP as backend, you can use
\href{/docs/7-0/user/-/knowledge_base/u/styling-apps-with-application-display-templates}{Application
Display Templates} in your page to customize its content from the server
side.

\subsection{Module}\label{module-20}

\begin{itemize}
\tightlist
\item
  None
\end{itemize}

\subsection{Views}\label{views-20}

\begin{itemize}
\tightlist
\item
  Default
\end{itemize}

\begin{figure}
\centering
\includegraphics{./images/screens-android-webscreenlet.png}
\caption{The Web Screenlet with the Default View Set.}
\end{figure}

\subsection{Configuration}\label{configuration}

To learn how to use Web Screenlet, follow the steps in the tutorial
\href{/docs/7-0/tutorials/-/knowledge_base/t/rendering-web-pages-in-your-android-app}{Rendering
Web Pages in Your Android App}. That tutorial gives detailed
instructions for using the configuration items described here.

Web Screenlet has \texttt{WebScreenletConfiguration} and
\texttt{WebScreenletConfiguration.Builder} classes that you can use
together to supply the parameters that the Screenlet needs to work.
\texttt{WebScreenletConfiguration.Builder} has the following methods,
which let you supply the described configuration parameters:

\noindent\hrulefill

Method \textbar{} Return \textbar{} Explanation \textbar{}
\texttt{addLocalJs(fileName)} \textbar{}
\texttt{WebScreenletConfiguration.Builder} \textbar{} Adds a local
JavaScript file with the supplied filename. The JavaScript files must be
in the first level of your app's \texttt{assets} folder. Create this
folder at the same level of the \texttt{res} folder. \textbar{}
\texttt{addLocalCss(fileName)} \textbar{}
\texttt{WebScreenletConfiguration.Builder} \textbar{} Adds a local CSS
file with the supplied filename. The CSS files must be in the first
level of your app's \texttt{assets} folder. Create this folder at the
same level of the \texttt{res} folder. \textbar{}
\texttt{addRawJs(rawJs,\ name)} \textbar{}
\texttt{WebScreenletConfiguration.Builder} \textbar{} Adds a JavaScript
file from your app's \texttt{res/raw} folder. Create this folder if it
doesn't exist. Reference the file using \texttt{R.raw.yourfilename}.
This method also takes a second parameter called \texttt{name}, which is
only for debugging purposes. If there's an error, the console displays
it with this \texttt{name} value. \textbar{}
\texttt{addRawCss(rawCss,\ name)} \textbar{}
\texttt{WebScreenletConfiguration.Builder} \textbar{} Adds a CSS file
from your app's \texttt{res/raw} folder. Create this folder if it
doesn't exist. Reference the file using \texttt{R.raw.yourfilename}.
This method also takes a second parameter called \texttt{name}, which is
only for debugging purposes. If there's an error, the console displays
it with this \texttt{name} value. \textbar{} \texttt{addRemoteJs(url)}
\textbar{} \texttt{WebScreenletConfiguration.Builder} \textbar{} Adds a
JavaScript file from the supplied URL. \textbar{}
\texttt{addRemoteCss(url)} \textbar{}
\texttt{WebScreenletConfiguration.Builder} \textbar{} Adds a CSS file
from the supplied URL. \textbar{} \texttt{setWebType(webType)}
\textbar{} \texttt{WebScreenletConfiguration.Builder} \textbar{} Sets
the
\href{/docs/7-0/reference/-/knowledge_base/r/web-screenlet-for-android\#webtype}{\texttt{WebType}}.
\textbar{} \texttt{enableCordova(observer)} \textbar{}
\texttt{WebScreenletConfiguration.Builder} \textbar{} Enables Cordova
inside the Web Screenlet. \textbar{} \texttt{load()} \textbar{}
\texttt{WebScreenletConfiguration} \textbar{} Returns the
\texttt{WebScreenletConfiguration} object that you can set to the
Screenlet instance. \textbar{}

\noindent\hrulefill

\noindent\hrulefill

\textbf{Note:} If you want to add comments in the scripts, use the
\texttt{/**/} notation.

\noindent\hrulefill

\subsubsection{WebType}\label{webtype}

\begin{itemize}
\item
  \textbf{WebType.LIFERAY\_AUTHENTICATED} (default): Displays a Liferay
  DXP page that requires authentication. The user must therefore be
  logged in with Screens via Login Screenlet or a
  \texttt{SessionContext} method. For this \texttt{WebType}, the URL you
  must pass to the \texttt{WebScreenletConfiguration.Builder}
  constructor is a relative URL. For example, if the full URL is
  \texttt{http://screens.liferay.org.es/web/guest/blog}, then the URL
  you must supply to the constructor is \texttt{/web/guest/blog}.
\item
  \textbf{WebType.OTHER}: Displays any other page. For this
  \texttt{WebType}, the URL you must pass to the
  \texttt{WebScreenletConfiguration.Builder} constructor is a full URL.
  For example, if the full URL is
  \texttt{http://screens.liferay.org.es/web/guest/blog}, then you must
  supply that URL to the constructor.
\end{itemize}

\subsection{Attributes}\label{attributes-20}

\noindent\hrulefill

Attribute \textbar{} Data type \textbar{} Explanation \textbar{}
\texttt{autoLoad} \textbar{} \texttt{boolean} \textbar{} Whether to load
the page automatically when the Screenlet appears in the app's UI. The
default value is \texttt{true}. \textbar{} \texttt{layoutId} \textbar{}
\texttt{@layout} \textbar{} The layout to use to show the View.
\textbar{} \texttt{isLoggingEnabled} \textbar{} \texttt{boolean}
\textbar{} Whether logging is enabled. \textbar{}
\texttt{isScrollEnabled} \textbar{} \texttt{boolean} \textbar{} Whether
to enable scrolling on the page inside the Screenlet. \textbar{}

\noindent\hrulefill

\subsection{Methods}\label{methods-11}

\noindent\hrulefill

Method \textbar{} Return \textbar{} Explanation \textbar{}
\texttt{load()} \textbar{} \texttt{void} \textbar{} Checks if the page's
URL is valid, and then loads it. The operation fails if the URL is
invalid. \textbar{} \texttt{clearCache()} \textbar{} \texttt{void}
\textbar{} Clears the Web Screenlet's cache. \textbar{}
\texttt{injectScript(script)} \textbar{} \texttt{void} \textbar{}
Injects a script when the page is already loaded. \textbar{}

\noindent\hrulefill

\subsection{Listener}\label{listener-20}

Web Screenlet delegates some events to an object or class that
implements its
\href{https://github.com/liferay/liferay-screens/blob/master/android/library/src/main/java/com/liferay/mobile/screens/web/WebListener.java}{\texttt{WebListener}
interface}. This interface extends the
\href{https://github.com/liferay/liferay-screens/blob/master/android/library/src/main/java/com/liferay/mobile/screens/base/interactor/listener/BaseCacheListener.java}{\texttt{BaseCacheListener}
interface}. Therefore, Web Screenlet's listener methods are as follows:

\begin{itemize}
\item
  \texttt{onPageLoaded(String\ url)}: Called when the Screenlet loads
  the page correctly.
\item
  \texttt{onScriptMessageHandler(String\ namespace,\ String\ body)}:
  Called when the \texttt{WebView} in the Screenlet sends a message. The
  \texttt{namespace} parameter is the source namespace key, and
  \texttt{body} is the source namespace body.
\item
  \texttt{error(Exception\ e,\ String\ userAction)}: Called when an
  error occurs in the process. The \texttt{userAction} argument
  distinguishes the specific action in which the error occurred.
\end{itemize}

\section{Android Breaking Changes}\label{android-breaking-changes}

This document presents a list of changes in Liferay Screens for Android
2.0 that break preceding functionality. We try our best to minimize
these disruptions, but sometimes they are unavoidable.

\subsection{Breaking Changes List}\label{breaking-changes-list}

\subsubsection{Interactors Now Run in a Background
Process}\label{interactors-now-run-in-a-background-process}

\paragraph{What changed?}\label{what-changed}

Interactors now run in a background process, so you don't need to create
or set callback classes manually. This means you can write what appear
to be synchronous server calls, and Liferay Screens handles the
background threading for you. The Interactor's \texttt{execute} method
makes the server call. Invoking the \texttt{start} method in your
Screenlet class causes \texttt{execute} to run in a background thread.

Note that you no longer have to handle the exception when making the
server call. The Screenlet framework does this for you and propagates
any error via the listeners. Also note that the \texttt{screenletId} is
no longer required. The Screenlet framework automatically decorates the
event with a \texttt{screenletId} that it generates.

\paragraph{Who is affected?}\label{who-is-affected}

This affects all Screenlet Interactors.

\paragraph{How should I update my
code?}\label{how-should-i-update-my-code}

You must rewrite your Interactors. See the tutorial
\href{/docs/7-0/tutorials/-/knowledge_base/t/creating-android-screenlets}{Creating
Android Screenlets} for the most recent instructions on creating an
Interactor.

\paragraph{Why was this change made?}\label{why-was-this-change-made}

Asynchronous calls can be difficult to develop and work with. By
handling them for you, Liferay Screens removes this potential source of
error and frees you to focus on other parts of your Screenlet.

\subsubsection{Changes to View Set
Inheritance}\label{changes-to-view-set-inheritance}

\paragraph{What changed?}\label{what-changed-1}

To use a View Set, your app or activity's theme must also inherit that
View Set's styles. For example, to use the Default View Set, your app or
activity's theme must inherit \texttt{default\_theme}.

\paragraph{Who is affected?}\label{who-is-affected-1}

This affects any apps or activities that use a View Set without
inheriting that View Set's styles. For example, if you use the Default
View for a Screenlet by setting the Screenlet XML's \texttt{layoutId}
attribute, your app or activity's theme must now inherit
\texttt{default\_theme} as well. Likewise, your app or activity's theme
must inherit \texttt{westeros\_theme} or \texttt{material\_theme} to use
the Westeros or Material View Set, respectively.

\paragraph{How should I update my
code?}\label{how-should-i-update-my-code-1}

Change your app or activity's theme to inherit the styles of the View
Set you want to use.

\textbf{Example}

This code snippet from an app's \texttt{res/values/styles.xml} tells
\texttt{AppTheme.NoActionBar} to inherit the Default View Set's styles:

\begin{verbatim}
<resources>

    <style name="AppTheme.NoActionBar" parent="default_theme">
        <item name="colorPrimary">@color/colorPrimary</item>
        <item name="colorPrimaryDark">@color/colorPrimaryDark</item>
        <item name="colorAccent">@color/colorAccent</item>

        <item name="windowActionBar">false</item>
        <item name="windowNoTitle">true</item>
    </style>
    ...
</resources>
\end{verbatim}

\paragraph{Why was this change made?}\label{why-was-this-change-made-1}

This lets you change an Android theme's colors and styles according to
Android conventions. Before, the Android themes were hardcoded inside
the Screenlets.

\subsubsection{The Screenlet Attribute offlinePolicy is now
cachePolicy}\label{the-screenlet-attribute-offlinepolicy-is-now-cachepolicy}

\paragraph{What changed?}\label{what-changed-2}

The Screenlet attribute \texttt{offlinePolicy} is now
\texttt{cachePolicy}.

\paragraph{Who is affected?}\label{who-is-affected-2}

This affects any Screenlets that used the \texttt{offlinePolicy}
attribute to set that Screenlet's offline mode policy.

\paragraph{How should I update my
code?}\label{how-should-i-update-my-code-2}

In the app layouts that contain the Screenlet, change the
\texttt{offlinePolicy} attribute to \texttt{cachePolicy}.

\textbf{Example}

Old way:

\begin{verbatim}
<com.liferay.mobile.screens.assetlist.AssetListScreenlet
    android:id="@+id/asset_list_screenlet"
    android:layout_width="match_parent"
    android:layout_height="match_parent"
    liferay:autoLoad="false"
    liferay:offlinePolicy="REMOTE_FIRST" />
\end{verbatim}

New way:

\begin{verbatim}
<com.liferay.mobile.screens.asset.list.AssetListScreenlet
    android:id="@+id/asset_list_screenlet"
    android:layout_width="match_parent"
    android:layout_height="match_parent"
    liferay:autoLoad="false"
    liferay:cachePolicy="REMOTE_FIRST"
    />
\end{verbatim}

\paragraph{Why was this change made?}\label{why-was-this-change-made-2}

This change was made for consistency throughout Liferay Screens. The
method and class names in the offline mode APIs contain \emph{cache}, as
do the offline policies \texttt{CACHE\_ONLY} and \texttt{CACHE\_FIRST}.

\subsubsection{Some Listener Methods in DDL Form Screenlet Have
Changed}\label{some-listener-methods-in-ddl-form-screenlet-have-changed}

\paragraph{What changed?}\label{what-changed-3}

The following error listener methods in DDL Form Screenlet's
\texttt{DDLFormListener} have been removed:

\begin{itemize}
\tightlist
\item
  \texttt{void\ onDDLFormLoadFailed(Exception\ e)}: Called when an error
  occurs in the load form definition request.
\item
  \texttt{void\ onDDLFormRecordLoadFailed(Exception\ e)}: Called when an
  error occurs in the load form record request.
\item
  \texttt{void\ onDDLFormRecordAddFailed(Exception\ e)}: Called when an
  error occurs in the request to add a new record.
\item
  \texttt{void\ onDDLFormUpdateRecordFailed(Exception\ e)}: Called when
  an error occurs in the request to update an existing record.
\end{itemize}

Also in \texttt{DDLFormListener}, the method
\texttt{onDDLFormRecordLoaded} now takes an additional parameter for the
attribute map received from the server.

\paragraph{Who is affected?}\label{who-is-affected-3}

This affects any classes that implement \texttt{DDLFormListener}.

\paragraph{How should I update my
code?}\label{how-should-i-update-my-code-3}

In place of the removed error listeners, use
\texttt{BaseCacheListener}'s generic error listener:

\begin{verbatim}
void error(Exception e, String userAction)
\end{verbatim}

You must also change any \texttt{onDDLFormRecordLoaded} implementations
to account for the method's new signature:

\begin{verbatim}
public void onDDLFormRecordLoaded(Record record, Map<String, Object> valuesAndAttributes)
\end{verbatim}

\paragraph{Why was this change made?}\label{why-was-this-change-made-3}

The old error listener methods were usually implemented the same way: by
logging the exception. Multiple error listener methods aren't needed for
this. You can use the new error listener method to log the exception and
take any other action that depends on the user action.

\subsubsection{Cache Listener Methods Moved into Their Own
Listener}\label{cache-listener-methods-moved-into-their-own-listener}

\paragraph{What changed?}\label{what-changed-4}

The cache listener methods \texttt{loadingFromCache},
\texttt{retrievingOnline}, and \texttt{storingToCache} have been moved
to their own listener, \texttt{CacheListener}. Note, this change was
introduced in Liferay Screens 1.4.0.

\paragraph{Who is affected?}\label{who-is-affected-4}

All activity classes that implement a listener.

\paragraph{How should I update my
code?}\label{how-should-i-update-my-code-4}

If you don't have special behavior in your old cache listener method
implementations, you can remove them. Otherwise, you must implement the
new \texttt{CacheListener}. When implementing \texttt{CacheListener} (in
an activity or fragment, for example), you should also register a
Screenlet instance as the cache listener:

\begin{verbatim}
    screenlet.setCacheListener(this);
\end{verbatim}

\textbf{Example}

In the Liferay Screens test app,
\href{https://github.com/liferay/liferay-screens/blob/2.0.1/android/samples/test-app/src/main/java/com/liferay/mobile/screens/testapp/UserPortraitActivity.java}{the
activity \texttt{UserPortraitActivity}} implements
\texttt{CacheListener}:

\begin{verbatim}
public class UserPortraitActivity extends ThemeActivity implements UserPortraitListener, 
    CacheListener {

    private UserPortraitScreenlet screenlet;

    @Override
    protected void onCreate(Bundle savedInstanceState) {
            super.onCreate(savedInstanceState);

            setContentView(R.layout.user_portrait);

            screenlet = (UserPortraitScreenlet) findViewById(R.id.user_portrait_screenlet);
            screenlet.setListener(this);
            screenlet.setCacheListener(this);
    }

    …

    @Override
    public void loadingFromCache(boolean success) {
        View content = findViewById(android.R.id.content);
        Snackbar.make(content, "Trying to load from cache: " + success, 
            Snackbar.LENGTH_SHORT).show();
    }

    @Override
    public void retrievingOnline(boolean triedInCache, Exception e) {

    }

    @Override
    public void storingToCache(Object object) {
        View content = findViewById(android.R.id.content);
        Snackbar.make(content, "Storing to cache...", Snackbar.LENGTH_SHORT).show();
    }

    …

}
\end{verbatim}

\paragraph{Why was this change made?}\label{why-was-this-change-made-4}

Reacting to cache errors via the cache listener methods is a niche use
case. Because the old cache listener methods were part of the normal
listener, developers were forced to implement them whether they needed
them or not. Putting them in their own listener makes their
implementation optional.

\subsubsection{Changed BaseListListener
Methods}\label{changed-baselistlistener-methods}

\paragraph{What changed?}\label{what-changed-5}

The \texttt{BaseListListener} methods \texttt{onListPageFailed} and
\texttt{onListPageReceived} no longer have the
\texttt{BaseListScreenlet} argument \texttt{source}. These methods also
now account for a page's start and end row instead of the page number.

\paragraph{Who is affected?}\label{who-is-affected-5}

This affects any classes or interfaces that extend or implement
\texttt{BaseListListener}.

\paragraph{How should I update my
code?}\label{how-should-i-update-my-code-5}

Remove the \texttt{BaseListScreenlet} argument from your
\texttt{onListPageFailed} and \texttt{onListPageReceived}
implementations. You must also replace the \texttt{int\ page} argument
in \texttt{onListPageFailed} with an \texttt{int} argument representing
the page's start row. Likewise, replace the \texttt{int\ page} argument
in \texttt{onListPageReceived} with two \texttt{int} arguments that
represent the page's start row and end row, respectively.

\textbf{Example}

Old signatures:

\begin{verbatim}
void onListPageFailed(BaseListScreenlet source, int page, Exception e)
void onListPageReceived(BaseListScreenlet source, int page, List<E> entries, int rowCount)
\end{verbatim}

New signatures:

\begin{verbatim}
void onListPageFailed(int startRow, Exception e)
void onListPageReceived(int startRow, int endRow, List<E> entries, int rowCount)
\end{verbatim}

\paragraph{Why was this change made?}\label{why-was-this-change-made-5}

The \texttt{BaseListScreenlet} argument served to disambiguate two
instances of the same Screenlet in a single activity. This is a very
rare use case. Therefore, forcing the argument on all
\texttt{BaseListListener} implementations was unnecessary. If you still
need this use case, create a Screenlet instance and listener for each
Screenlet instead of relying on the \texttt{BaseListScreenlet} argument
in a single listener. The start row and end row change was made for
consistency with other listeners that also use start row and end row
arguments.

\subsubsection{Changed Asset List Screenlet
Package}\label{changed-asset-list-screenlet-package}

\paragraph{What changed?}\label{what-changed-6}

Asset List Screenlet's package is now
\texttt{com.liferay.mobile.screens.asset.list} instead of
\texttt{com.liferay.mobile.screens.assetlist}.

\paragraph{Who is affected?}\label{who-is-affected-6}

This affects any activities or fragments that use Asset List Screenlet.

\paragraph{How should I update my
code?}\label{how-should-i-update-my-code-6}

Change your \texttt{com.liferay.mobile.screens.assetlist} imports to
\texttt{com.liferay.mobile.screens.asset.list}.

\paragraph{Why was this change made?}\label{why-was-this-change-made-6}

This allows for other Screenlets that work with assets, like Asset
Display Screenlet. For example, the package
\texttt{com.liferay.mobile.screens.asset} now contains Asset List
Screenlet, Asset Display Screenlet, and classes common to both.

\subsubsection{Changed Return Type for a DDL Record
Method}\label{changed-return-type-for-a-ddl-record-method}

\paragraph{What changed?}\label{what-changed-7}

The \texttt{getModelValues()} method for
\href{https://github.com/liferay/liferay-screens/blob/2.0.1/android/library/src/main/java/com/liferay/mobile/screens/ddl/model/Record.java}{DDL
records} now returns a \texttt{Map} instead of a \texttt{HashMap}.

\paragraph{Who is affected?}\label{who-is-affected-7}

This affects any code that expects \texttt{getModelValues()} to return a
\texttt{HashMap}.

\paragraph{How should I update my
code?}\label{how-should-i-update-my-code-7}

Change any code that uses \texttt{getModelValues()} to expect a
\texttt{Map} instead of a \texttt{HashMap}.

\paragraph{Why was this change made?}\label{why-was-this-change-made-7}

This follows general Java conventions.

\subsubsection{Changed Code Conventions for Private and Protected
Fields}\label{changed-code-conventions-for-private-and-protected-fields}

\paragraph{What changed?}\label{what-changed-8}

Private and protected fields in Screenlets are no longer prefixed by
\texttt{\_}.

\paragraph{Who is affected?}\label{who-is-affected-8}

This affects any code that directly accesses protected fields.

\paragraph{How should I update my
code?}\label{how-should-i-update-my-code-8}

Change your code to use the new variable name. For example, if your code
directly accesses a protected Screenlet variable named
\texttt{\_fields}, change it to use \texttt{fields} instead.

\paragraph{Why was this change made?}\label{why-was-this-change-made-8}

This follows general Java naming conventions.

\subsubsection{Changes to Using a Screenlet without a
View}\label{changes-to-using-a-screenlet-without-a-view}

\paragraph{What changed?}\label{what-changed-9}

If you're using a Screenlet without View (like you might be if you need
to log a user in programmatically), you no longer have to call
\texttt{LiferayScreensContext.init(this)} to initialise the library.
This is now called automatically.

\paragraph{Who is affected?}\label{who-is-affected-9}

This affects any apps that use a Screenlet without a View.

\paragraph{How should I update my
code?}\label{how-should-i-update-my-code-9}

Remove your manual call to \texttt{LiferayScreensContext.init(this)}.

\paragraph{Why was this change made?}\label{why-was-this-change-made-9}

This removes the possibility of an error if you forget to call
\texttt{LiferayScreensContext.init(this)} when using a Screenlet without
a View.

\chapter{Screenlets in Liferay Screens for
iOS}\label{screenlets-in-liferay-screens-for-ios}

Liferay Screens for iOS contains several Screenlets that you can use in
your iOS apps. This section contains the reference documentation for
each. If you're looking for instructions on using Screens, see the
\href{/docs/7-0/tutorials/-/knowledge_base/t/ios-apps-with-liferay-screens}{Screens
tutorials}. The Screens tutorials contain instructions on
\href{/docs/7-0/tutorials/-/knowledge_base/t/using-screenlets-in-ios-apps}{using
Screenlets} and
\href{/docs/7-0/tutorials/-/knowledge_base/t/using-themes-in-ios-screenlets}{using
Themes in Screenlets}. Each Screenlet reference document here lists the
Screenlet's features, compatibility, its module (if any), available
Themes, attributes, delegate methods, and more. The available Screenlets
are listed here with links to their reference documents:

\begin{itemize}
\item
  \href{/docs/7-0/reference/-/knowledge_base/r/loginscreenlet-for-ios}{\textbf{Login
  Screenlet:}} Signs users in to a Liferay DXP instance.
\item
  \href{/docs/7-0/reference/-/knowledge_base/r/signupscreenlet-for-ios}{\textbf{Sign
  Up Screenlet:}} Registers new users in a Liferay DXP instance.
\item
  \href{/docs/7-0/reference/-/knowledge_base/r/forgotpasswordscreenlet-for-ios}{\textbf{Forgot
  Password Screenlet:}} Sends emails containing a new password or
  password reset link to users.
\item
  \href{/docs/7-0/reference/-/knowledge_base/r/userportraitscreenlet-for-ios}{\textbf{User
  Portrait Screenlet:}} Shows the user's portrait picture.
\item
  \href{/docs/7-0/reference/-/knowledge_base/r/ddlformscreenlet-for-ios}{\textbf{DDL
  Form Screenlet:}} Presents dynamic forms to be filled out by users and
  submitted back to the server.
\item
  \href{/docs/7-0/reference/-/knowledge_base/r/ddllistscreenlet-for-ios}{\textbf{DDL
  List Screenlet:}} Shows a list of records based on a pre-existing DDL
  in a Liferay DXP instance.
\item
  \href{/docs/7-0/reference/-/knowledge_base/r/assetlistscreenlet-for-ios}{\textbf{Asset
  List Screenlet:}} Shows a list of assets managed by
  \href{/docs/7-0/tutorials/-/knowledge_base/t/asset-framework}{Liferay
  DXP's Asset Framework}. This includes web content, blog entries,
  documents, and more.
\item
  \href{/docs/7-0/reference/-/knowledge_base/r/webcontentdisplayscreenlet-for-ios}{\textbf{Web
  Content Display Screenlet:}} Shows the web content's HTML or
  structured content. This Screenlet uses the features available in
  \href{/docs/7-0/user/-/knowledge_base/u/creating-web-content}{Web
  Content Management}.
\item
  \href{/docs/7-0/reference/-/knowledge_base/r/web-content-list-screenlet-for-ios}{\textbf{Web
  Content List Screenlet:}} Shows a list of web contents from a folder,
  usually based on a pre-existing \texttt{DDMStructure}.
\item
  \href{/docs/7-0/reference/-/knowledge_base/r/image-gallery-screenlet-for-ios}{\textbf{Image
  Gallery Screenlet:}} Shows a list of images from a folder. This
  Screenlet also lets users upload and delete images.
\item
  \href{/docs/7-0/reference/-/knowledge_base/r/rating-screenlet-for-ios}{\textbf{Rating
  Screenlet:}} Shows the rating for an asset. This Screenlet also lets
  the user update or delete the rating.
\item
  \href{/docs/7-0/reference/-/knowledge_base/r/comment-list-screenlet-for-ios}{\textbf{Comment
  List Screenlet:}} Shows a list of comments for an asset.
\item
  \href{/docs/7-0/reference/-/knowledge_base/r/comment-display-screenlet-for-ios}{\textbf{Comment
  Display Screenlet:}} Shows a single comment for an asset.
\item
  \href{/docs/7-0/reference/-/knowledge_base/r/comment-add-screenlet-for-ios}{\textbf{Comment
  Add Screenlet:}} Lets the user comment on an asset.
\item
  \href{/docs/7-0/reference/-/knowledge_base/r/asset-display-screenlet-for-ios}{\textbf{Asset
  Display Screenlet:}} Displays an asset. Currently, this Screenlet can
  display Documents and Media Library files (\texttt{DLFileEntry}
  entities), blog articles (\texttt{BlogsEntry} entities), and web
  content articles (\texttt{WebContent} entities). You can also use it
  to display custom assets.
\item
  \href{/docs/7-0/reference/-/knowledge_base/r/blogs-entry-display-screenlet-for-ios}{\textbf{Blogs
  Entry Display Screenlet:}} Shows a single blogs entry.
\item
  \href{/docs/7-0/reference/-/knowledge_base/r/image-display-screenlet-for-ios}{\textbf{Image
  Display Screenlet:}} Shows a single image file from a Liferay DXP
  instance's Documents and Media Library.
\item
  \href{/docs/7-0/reference/-/knowledge_base/r/video-display-screenlet-for-ios}{\textbf{Video
  Display Screenlet:}} Shows a single video file from a Liferay DXP
  instance's Documents and Media Library.
\item
  \href{/docs/7-0/reference/-/knowledge_base/r/audio-display-screenlet-for-ios}{\textbf{Audio
  Display Screenlet:}} Shows a single audio file from a Liferay DXP
  instance's Documents and Media Library.
\item
  \href{/docs/7-0/reference/-/knowledge_base/r/pdf-display-screenlet-for-ios}{\textbf{PDF
  Display Screenlet:}} Shows a single PDF file from a Liferay DXP
  instance's Documents and Media Library.
\item
  \href{/docs/7-0/reference/-/knowledge_base/r/file-display-screenlet-for-ios}{\textbf{File
  Display Screenlet:}} Shows a single file from a Liferay DXP instance's
  Documents and Media Library. Use this Screenlet to display file types
  not covered by the other display Screenlets (e.g., DOC, PPT, XLS).
\item
  \href{/docs/7-0/reference/-/knowledge_base/r/web-screenlet-for-ios}{\textbf{Web
  Screenlet:}} Displays any web page. You can also customize the web
  page through injection of local and remote JavaScript and CSS files.
\end{itemize}

\section{Login Screenlet for iOS}\label{login-screenlet-for-ios}

\subsection{Requirements}\label{requirements-21}

\begin{itemize}
\tightlist
\item
  Xcode 9.3 or above
\item
  iOS 11 SDK
\item
  Liferay Portal 6.2 CE/EE, Liferay CE Portal 7.0/7.1, Liferay DXP
\end{itemize}

\subsection{Compatibility}\label{compatibility-21}

\begin{itemize}
\tightlist
\item
  iOS 9 and above
\end{itemize}

\subsection{Xamarin Requirements}\label{xamarin-requirements-21}

\begin{itemize}
\tightlist
\item
  Visual Studio 7.2
\item
  Mono .NET framework 5.4.1.6
\end{itemize}

\subsection{Features}\label{features-21}

The Login Screenlet authenticates portal users in your iOS app. The
following authentication methods are supported:

\begin{itemize}
\item
  \textbf{Basic:} uses user login and password according to
  \href{http://tools.ietf.org/html/rfc2617}{HTTP Basic Access
  Authentication specification}. Depending on the authentication method
  used by your Liferay instance, you need to provide the user's email
  address, screen name, or user ID. You also need to provide the user's
  password.
\item
  \textbf{OAuth:} implements the
  \href{http://oauth.net/core/1.0a/}{OAuth 1.0a specification}.
\item
  \textbf{Cookie:} uses a cookie to log in. This lets you access
  documents and images in the portal's document library without the
  guest view permission in the portal. The other authentication types
  require this permission to access such files.
\end{itemize}

\noindent\hrulefill

\textbf{Note:} Cookie authentication with Login Screenlet is broken in
fix packs 14 through 18 of Liferay Digital Enterprise 7.0. This issue is
fixed in newer fix packs for Liferay Digital Enterprise 7.0.

\noindent\hrulefill

For instructions on configuring the Screenlet to use these
authentication types, see the below
\href{/docs/7-0/reference/-/knowledge_base/r/loginscreenlet-for-ios\#portal-configuration}{Portal
Configuration} and
\href{/docs/7-0/reference/-/knowledge_base/r/loginscreenlet-for-ios\#attributes}{Screenlet
Attributes} sections.

When a user successfully authenticates, their attributes are retrieved
for use in the app. You can use the \texttt{SessionContext} class to get
the current user's attributes.

Note that user credentials and attributes can be stored securely in the
keychain (see the \texttt{saveCredentials} attribute). Stored user
credentials can be used to automatically log the user in to subsequent
sessions. To do this, you can use the method
\texttt{SessionContext.loadStoredCredentials()} method.

\subsection{JSON Services Used}\label{json-services-used-20}

Screenlets in Liferay Screens call the portal's JSON web services. This
Screenlet calls the following services and methods.

\noindent\hrulefill

\begin{longtable}[]{@{}lll@{}}
\toprule\noalign{}
Service & Method & Notes \\
\midrule\noalign{}
\endhead
\bottomrule\noalign{}
\endlastfoot
\texttt{UserService} & \texttt{getUserByEmailAddress} & Basic login \\
\texttt{UserService} & \texttt{getUserByScreenName} & Basic login \\
\texttt{UserService} & \texttt{getUserById} & Basic login \\
\texttt{UserService} & \texttt{getCurrentUser} & Cookie and OAuth
login \\
\end{longtable}

\noindent\hrulefill

\subsection{Module}\label{module-21}

\begin{itemize}
\tightlist
\item
  Auth
\end{itemize}

\subsection{Themes}\label{themes}

\begin{itemize}
\tightlist
\item
  Default (\texttt{default})
\item
  Flat7 (\texttt{flat7})
\end{itemize}

For instructions on using Themes,
\href{/docs/7-0/tutorials/-/knowledge_base/t/using-themes-in-ios-screenlets}{click
here}.

\begin{figure}
\centering
\includegraphics{./images/screens-ios-login.png}
\caption{The Login Screenlet using the Default and Flat7 Themes.}
\end{figure}

\subsection{Portal Configuration}\label{portal-configuration-8}

\subsubsection{Basic Authentication}\label{basic-authentication-1}

Before using Login Screenlet, you should make sure your portal is
configured with the authentication option you want to use. You can
choose email address, screen name, or user ID. You can set this in the
Control Panel by selecting \emph{Configuration} → \emph{Instance
Settings}, and then selecting the \emph{Authentication} section. The
authentication options are in the \emph{How do users authenticate?}
selector menu.

\begin{figure}
\centering
\includegraphics{./images/screens-portal-auth.png}
\caption{Setting the authentication method in your Liferay instance.}
\end{figure}

For more details, please refer to the
\href{/docs/7-0/user/-/knowledge_base/u/setting-up-a-liferay-instance}{Setting
up a Liferay Instance} section of the User Guide.

\subsubsection{OAuth}\label{oauth}

\noindent\hrulefill

\textbf{Note:} OAuth authentication is only available in Liferay DXP
instances.

\noindent\hrulefill

To use OAuth authentication, first install the OAuth provider app from
the Liferay Marketplace.
\href{https://web.liferay.com/marketplace/-/mp/application/45261909}{Click
here} to get this app. Once it's installed, go to \emph{Control Panel} →
\emph{Users} → \emph{OAuth Admin}, and add a new application to be used
from Liferay Screens. Once the application exists, copy the
\emph{Consumer Key} and \emph{Consumer Secret} values for later use in
Login Screenlet.

\subsection{Offline}\label{offline-20}

This Screenlet doesn't support offline mode. It requires network
connectivity. If you need to log in users automatically, even when
there's no network connection, you can use the \texttt{saveCredentials}
attribute together with the
\texttt{SessionContext.loadStoredCredentials()} method.

\subsection{Attributes}\label{attributes-21}

\noindent\hrulefill

Attribute \textbar{} Data type \textbar{} Explanation \textbar{}
\texttt{companyId} \textbar{} \texttt{number} \textbar{} The ID of the
portal instance to authenticate to. If you don't set this attribute or
set it to \texttt{0}, the Screenlet uses the \texttt{companyId} setting
in \texttt{LiferayServerContext}. \textbar{} \texttt{loginMode}
\textbar{} \texttt{string} \textbar{} The Screenlet's authentication
type. You can set this attribute to \texttt{basic}, \texttt{oauth}, or
\texttt{cookie}. If you don't set this attribute, the Screenlet defaults
to basic authentication. \textbar{} \texttt{basicAuthMethod} \textbar{}
\texttt{string} \textbar{} Specifies the basic authentication option to
use. You can set this attribute to \texttt{email}, \texttt{screenName}
or \texttt{userId}. This must match the server's authentication option.
If you don't set this attribute, and don't set the \texttt{loginMode}
attribute to \texttt{oauth} or \texttt{cookie}, the Screenlet defaults
to basic authentication with the \texttt{email} option. \textbar{}
\texttt{OAuthConsumerKey} \textbar{} \texttt{string} \textbar{}
Specifies the \emph{Consumer Key} to use in OAuth authentication.
\textbar{} \texttt{OAuthConsumerSecret} \textbar{} \texttt{string}
\textbar{} Specifies the \emph{Consumer Secret} to use in OAuth
authentication. \textbar{} \texttt{saveCredentials} \textbar{}
\texttt{boolean} \textbar{} When set, the user credentials and
attributes are stored securely in the keychain. This information can
then be loaded in subsequent sessions by calling the
\texttt{SessionContext.loadStoredCredentials()} method. \textbar{}
\texttt{shouldHandleCookieExpiration} \textbar{} \texttt{bool}
\textbar~Whether to refresh the cookie automatically when using cookie
login. When set to \texttt{true} (the default value), the cookie
refreshes as it's about to expire. \textbar{}
\texttt{cookieExpirationTime} \textbar{} \texttt{int} \textbar~How long
the cookie lasts, in seconds. This value depends on your portal
instance's configuration. The default value is \texttt{900}. \textbar{}

\noindent\hrulefill

\subsection{Delegate}\label{delegate}

The Login Screenlet delegates some events to an object that conforms to
the \texttt{LoginScreenletDelegate} protocol. This protocol lets you
implement the following methods:

\begin{itemize}
\item
  \texttt{-\ screenlet:onLoginResponseUserAttributes:}: Called when
  login successfully completes. The user attributes are passed as a
  dictionary of keys (\texttt{String} or \texttt{NSStrings}) and values
  (\texttt{AnyObject} or \texttt{NSObject}). The supported keys are the
  same as the
  \href{https://github.com/liferay/liferay-portal/blob/master/portal-impl/src/com/liferay/portal/service.xml\#L2575-L2737}{portal's
  User entity}.
\item
  \texttt{-\ screenlet:onLoginError:}: Called when an error occurs
  during login. The \texttt{NSError} object describes the error.
\item
  \texttt{-\ screenlet:onCredentialsSavedUserAttributes:}: Called when
  the user credentials are stored after a successful login.
\item
  \texttt{-\ screenlet:onCredentialsLoadedUserAttributes:}: Called when
  the user credentials are retrieved. Note that this only occurs when
  the Screenlet is used and stored credentials are available.
\end{itemize}

\subsection{Challenge-Response
Authentication}\label{challenge-response-authentication-1}

To support
\href{https://en.wikipedia.org/wiki/Challenge\%E2\%80\%93response_authentication}{challenge-response
authentication} when using a cookie to log in to the portal, the
\texttt{SessionContext} class has a \texttt{challengeResolver}
attribute. For more information about how iOS handles challenge-response
authentication, see the article
\href{https://developer.apple.com/library/content/documentation/Cocoa/Conceptual/URLLoadingSystem/Articles/AuthenticationChallenges.html}{Authentication
Challenges and TLS Chain Validation}.

The challenge resolver type is a closure or block that receives two
parameters:

\begin{enumerate}
\def\labelenumi{\arabic{enumi}.}
\tightlist
\item
  \texttt{URLAuthenticationChallenge}
\item
  Another closure or block. You must call this to resolve the challenge
  (e.g., by passing credentials, canceling the challenge, etc.). You can
  do this by passing a \texttt{URLSession.AuthChallengeDisposition}.
\end{enumerate}

Here's an example that sends a basic authorization in response to an
authentication challenge:

\begin{verbatim}
SessionContext.challengeResolver = resolver

func resolver(challenge: URLAuthenticationChallenge,
    decisionCallback: (URLSession.AuthChallengeDisposition, URLCredential) -> Void) {

    // Use the challenge variable to get information about the challenge itself
    if challenge.previousFailureCount == 0 {
        // To solve the challenge, call the decision callback with your decision
        // Pass the credentials to the server
        decisionCallback(.useCredential, URLCredential(user: "user", password: "password", 
            persistence: .forSession))
    }
    else {
        // Something went wrong, so let the system handle the challenge
        decisionCallback(.performDefaultHandling, URLCredential(user: "these credentials", 
            password: "are ignored", persistence: .none))
    }

}
\end{verbatim}

\section{Sign Up Screenlet for iOS}\label{sign-up-screenlet-for-ios}

\subsection{Requirements}\label{requirements-22}

\begin{itemize}
\tightlist
\item
  Xcode 9.3 or above
\item
  iOS 11 SDK
\item
  Liferay Portal 6.2 CE/EE, Liferay CE Portal 7.0/7.1, Liferay DXP 7.0+
\end{itemize}

\subsection{Compatibility}\label{compatibility-22}

\begin{itemize}
\tightlist
\item
  iOS 9 and above
\end{itemize}

\subsection{Xamarin Requirements}\label{xamarin-requirements-22}

\begin{itemize}
\tightlist
\item
  Visual Studio 7.2
\item
  Mono .NET framework 5.4.1.6
\end{itemize}

\subsection{Features}\label{features-22}

The Sign Up Screenlet creates a new user in your Liferay instance: a new
user of your app can become a new user in your portal. You can also use
this Screenlet to save the credentials of the new user in their
keychain. This enables auto login for future sessions. The Screenlet
also supports navigation of form fields from the keyboard of the user's
device.

\subsection{JSON Services Used}\label{json-services-used-21}

Screenlets in Liferay Screens call JSON web services in the portal. This
Screenlet calls the following services and methods.

\noindent\hrulefill

\begin{longtable}[]{@{}lll@{}}
\toprule\noalign{}
Service & Method & Notes \\
\midrule\noalign{}
\endhead
\bottomrule\noalign{}
\endlastfoot
\texttt{UserService} & \texttt{addUser} & \\
\end{longtable}

\noindent\hrulefill

\subsection{Module}\label{module-22}

\begin{itemize}
\tightlist
\item
  Auth
\end{itemize}

\subsection{Themes}\label{themes-1}

\begin{itemize}
\tightlist
\item
  Default (\texttt{default})
\item
  Flat7 (\texttt{flat7})
\end{itemize}

\begin{figure}
\centering
\includegraphics{./images/screens-ios-signup.png}
\caption{The Sign Up Screenlet with the Default and Flat7 Themes.}
\end{figure}

\subsection{Portal Configuration}\label{portal-configuration-9}

Sign Up Screenlet's corresponding configuration in the Liferay instance
can be set in the Control Panel by selecting \emph{Configuration} →
\emph{Instance Settings}, and then selecting the \emph{Authentication}
section.

\begin{figure}
\centering
\includegraphics{./images/screens-portal-signup.png}
\caption{The Liferay instance's authentication settings.}
\end{figure}

For more details, please refer to the
\href{/docs/7-0/user/-/knowledge_base/u/setting-up-a-liferay-instance}{Setting
up a Liferay Instance} section of the User Guide.

\subsection{Anonymous Request}\label{anonymous-request-1}

Anonymous requests are unauthenticated requests. Authentication is
needed, however, to call the API. To allow this operation, the portal
administrator should create a specific user with minimal permissions.

\subsection{Offline}\label{offline-21}

This Screenlet doesn't support offline mode. It requires network
connectivity.

\subsection{Attributes}\label{attributes-22}

\noindent\hrulefill

Attribute \textbar{} Data type \textbar{} Explanation \textbar{}
\texttt{anonymousApiUserName} \textbar{} \texttt{string} \textbar{} The
user name, email address, or user ID (depending on the portal's
authentication method) to use for authenticating the request. \textbar{}
\texttt{anoymousApiPassword} \textbar{} \texttt{string} \textbar{} The
password for use in authenticating the request. \textbar{}
\texttt{companyId} \textbar{} \texttt{number} \textbar{} When set,
authentication is done for a user in the specified company. If the value
is \texttt{0}, the company specified in \texttt{LiferayServerContext} is
used. \textbar{} \texttt{autoLogin} \textbar{} \texttt{boolean}
\textbar{} Whether the user is logged in automatically after a
successful sign up. \textbar{} \texttt{saveCredentials} \textbar{}
\texttt{boolean} \textbar{} Sets whether or not the user's credentials
and attributes are stored in the keychain after a successful log in.
This attribute is ignored if \texttt{autologin} is disabled. \textbar{}

\noindent\hrulefill

\subsection{Delegate}\label{delegate-1}

The Sign Up Screenlet delegates some events to an object that conforms
to the \texttt{SignUpScreenletDelegate} protocol. If the
\texttt{autologin} attribute is enabled, login events are delegated to
an object conforming to the \texttt{LoginScreenletDelegate} protocol.
Refer to the \href{LoginScreenlet.md}{\texttt{LoginScreenlet}
documentation} for more details.

The \texttt{SignUpScreenletDelegate} protocol lets you implement the
following methods:

\begin{itemize}
\item
  \texttt{-\ screenlet:onSignUpResponseUserAttributes:}: Called when
  sign up successfully completes. The user attributes are passed as a
  dictionary of keys (\texttt{String} or \texttt{NSStrings}) and values
  (\texttt{AnyObject} or \texttt{NSObject}). The supported keys are the
  same as
  \href{https://github.com/liferay/liferay-portal/blob/7.0.x/portal-impl/src/com/liferay/portal/service.xml\#L2686}{Liferay
  Portal's User entity}.
\item
  \texttt{-\ screenlet:onSignUpError:}: Called when an error occurs in
  the process. The \texttt{NSError} object describes the error.
\end{itemize}

\section{Forgot Password Screenlet for
iOS}\label{forgot-password-screenlet-for-ios}

\subsection{Requirements}\label{requirements-23}

\begin{itemize}
\tightlist
\item
  Xcode 9.3 or above
\item
  iOS 11 SDK
\item
  Liferay Portal 6.2 CE/EE, Liferay CE Portal 7.0/7.1, Liferay DXP 7.0+
\item
  Liferay Screens Compatibility app
  (\href{http://www.liferay.com/marketplace/-/mp/application/54365664}{CE}
  or
  \href{http://www.liferay.com/marketplace/-/mp/application/54369726}{EE/DXP}).
  This app is preinstalled in Liferay CE Portal 7.0/7.1 and Liferay DXP
  7.0+.
\end{itemize}

\subsection{Compatibility}\label{compatibility-23}

\begin{itemize}
\tightlist
\item
  iOS 9 and above
\end{itemize}

\subsection{Xamarin Requirements}\label{xamarin-requirements-23}

\begin{itemize}
\tightlist
\item
  Visual Studio 7.2
\item
  Mono .NET framework 5.4.1.6
\end{itemize}

\subsection{Features}\label{features-23}

The Forgot Password Screenlet sends emails to registered users with
their new passwords or password reset links, depending on the server
configuration. The available authentication methods are:

\begin{itemize}
\tightlist
\item
  Email address
\item
  Screen name
\item
  User id
\end{itemize}

\subsection{JSON Services Used}\label{json-services-used-22}

Screenlets in Liferay Screens call JSON web services in the portal. This
Screenlet calls the following services and methods.

\noindent\hrulefill

\begin{longtable}[]{@{}lll@{}}
\toprule\noalign{}
Service & Method & Notes \\
\midrule\noalign{}
\endhead
\bottomrule\noalign{}
\endlastfoot
\texttt{UserService} & \texttt{sendPasswordByEmailAddress} & \\
\texttt{UserService} & \texttt{sendPasswordByUserId} & \\
\texttt{UserService} & \texttt{sendPasswordByScreenName} & \\
\end{longtable}

\noindent\hrulefill

\subsection{Module}\label{module-23}

\begin{itemize}
\tightlist
\item
  Auth
\end{itemize}

\subsection{Themes}\label{themes-2}

\begin{itemize}
\tightlist
\item
  Default (\texttt{default})
\item
  Flat7 (\texttt{flat7})
\end{itemize}

\begin{figure}
\centering
\includegraphics{./images/screens-ios-forgotpwd.png}
\caption{The Forgot Password Screenlet with the Default and Flat7
Themes.}
\end{figure}

\subsection{Portal Configuration}\label{portal-configuration-10}

To use the Forgot Password Screenlet, you must allow users to request
new passwords in the portal. The next sections show you how to do this.

\subsubsection{Authentication Method}\label{authentication-method-1}

Note that the authentication method configured in the portal can be
different from the one used by this Screenlet. For example, it's
\emph{perfectly fine} to use \texttt{screenName} for sign in
authentication, but allow users to recover their password using the
\texttt{email} authentication method.

\subsubsection{Password Reset}\label{password-reset-1}

You can set the Liferay instance's corresponding password reset options
in the Control Panel by selecting \emph{Configuration} → \emph{Instance
Settings}, and then selecting the \emph{Authentication} section. The
Screenlet's password functionality depends on the authentication
settings in the portal:

\begin{figure}
\centering
\includegraphics{./images/screens-password-reset.png}
\caption{Checkboxes for the password recovery features in Liferay
Portal.}
\end{figure}

If both of these options are unchecked, password recovery is disabled.
If both options are checked, an email containing a password reset link
is sent when a user requests it. If only the first option is checked, an
email containing a new password is sent when a user requests it.

For more details on authentication in Liferay Portal, please refer to
the
\href{/docs/7-0/user/-/knowledge_base/u/setting-up-a-liferay-instance}{Setting
up a Liferay Instance} section of the User Guide.

\subsubsection{Anonymous Request}\label{anonymous-request-2}

An anonymous request can be made without the user being logged in.
However, authentication is needed to call the API. To allow this
operation, the portal administrator should create a specific user with
minimal permissions.

\subsection{Offline}\label{offline-22}

This Screenlet doesn't support offline mode. It requires network
connectivity.

\subsection{Attributes}\label{attributes-23}

\noindent\hrulefill

Attribute \textbar{} Data type \textbar{} Explanation \textbar{}
\texttt{anonymousApiUserName} \textbar{} \texttt{string} \textbar{} The
user name, email address, or userId (depending on the portal's
authentication method) to use for authenticating the request. \textbar{}
\texttt{anonymousApiPassword} \textbar{} \texttt{string} \textbar{} The
password to use to authenticate the request. \textbar{}
\texttt{companyId} \textbar{} \texttt{number} \textbar{} When set, the
authentication is done for a user within the specified company. If the
value is \texttt{0}, the company specified in
\texttt{LiferayServerContext} is used. \textbar{}
\texttt{basicAuthMethod} \textbar{} \texttt{string} \textbar{} The
authentication method that is presented to the user. This can be
\texttt{email}, \texttt{screenName}, or \texttt{userId}. \textbar{}

\noindent\hrulefill

\subsection{Delegate}\label{delegate-2}

The Forgot Password Screenlet delegates some events to an object that
conforms to the \texttt{ForgotPasswordScreenletDelegate} protocol. This
protocol lets you implement the following methods:

\begin{itemize}
\item
  \texttt{-\ screenlet:onForgotPasswordSent:}: Called when a password
  reset email is successfully sent. The Boolean parameter indicates
  whether the email contains the new password or a password reset link.
\item
  \texttt{-\ screenlet:onForgotPasswordError:}: Called when an error
  occurs in the process. The \texttt{NSError} object describes the
  error.
\end{itemize}

\section{User Portrait Screenlet for
iOS}\label{user-portrait-screenlet-for-ios}

\subsection{Requirements}\label{requirements-24}

\begin{itemize}
\tightlist
\item
  Xcode 9.3 or above
\item
  iOS 11 SDK
\item
  Liferay Portal 6.2 CE/EE, Liferay CE Portal 7.0/7.1, Liferay DXP 7.0+
\end{itemize}

\subsection{Compatibility}\label{compatibility-24}

\begin{itemize}
\tightlist
\item
  iOS 9 and above
\end{itemize}

\subsection{Xamarin Requirements}\label{xamarin-requirements-24}

\begin{itemize}
\tightlist
\item
  Visual Studio 7.2
\item
  Mono .NET framework 5.4.1.6
\end{itemize}

\subsection{Features}\label{features-24}

The User Portrait Screenlet shows the user's portrait from Liferay
Portal. If the user doesn't have a portrait configured, a placeholder
image is shown.

\subsection{JSON Services Used}\label{json-services-used-23}

Screenlets in Liferay Screens call JSON web services in the portal. This
Screenlet calls the following services and methods.

\noindent\hrulefill

\begin{longtable}[]{@{}lll@{}}
\toprule\noalign{}
Service & Method & Notes \\
\midrule\noalign{}
\endhead
\bottomrule\noalign{}
\endlastfoot
\texttt{UserService} & \texttt{getUserById} & \\
\texttt{UserService} & \texttt{getUserByEmailAddress} & \\
\texttt{UserService} & \texttt{getUserByScreenName} & \\
\end{longtable}

\noindent\hrulefill

\subsection{Module}\label{module-24}

\begin{itemize}
\tightlist
\item
  None
\end{itemize}

\subsection{Themes}\label{themes-3}

\begin{itemize}
\tightlist
\item
  Default (\texttt{default})
\item
  Flat7 (\texttt{flat7})
\end{itemize}

\begin{figure}
\centering
\includegraphics{./images/screens-ios-portrait.png}
\caption{The User Portrait Screenlet using the Default and Flat7
Themes.}
\end{figure}

\subsection{Portal Configuration}\label{portal-configuration-11}

None

\subsection{Offline}\label{offline-23}

This Screenlet supports offline mode so it can function without a
network connection. For more information on how offline mode works, see
the
\href{/docs/7-0/tutorials/-/knowledge_base/t/architecture-of-offline-mode-in-liferay-screens}{tutorial
on its architecture}.

When loading the portrait, the Screenlet supports the following offline
mode policies:

\noindent\hrulefill

Policy \textbar{} What happens \textbar{} When to use \textbar{}
\texttt{remote-only} \textbar{} The Screenlet loads the user portrait
from the portal. If a connection issue occurs, the Screenlet uses the
delegate to notify the developer about the error. If the Screenlet loads
the portrait, it stores the received image in the local cache for later
use. \textbar{} Use this policy when you always need to show updated
portraits, and show the default placeholder when there's no connection.
\textbar{} \texttt{cache-only} \textbar{} The Screenlet loads the user
portrait from the local cache. If the portrait isn't there, the
Screenlet uses the delegate to notify the developer about the error.
\textbar{} Use this policy to show local portraits, without retrieving
remote information under any circumstance. \textbar{}
\texttt{remote-first} \textbar{} The Screenlet loads the user portrait
from the portal. The Screenlet displays the portrait to the user and
stores it in the local cache for later use. If a connection issue
occurs, the Screenlet retrieves the portrait from the local cache. If
the portrait doesn't exist there, the Screenlet uses the delegate to
notify the developer about the error. \textbar{} Use this policy to show
the most recent portrait when connected, but show a potentially outdated
version when there's no connection. \textbar{} \texttt{cache-first}
\textbar{} If the portrait exists in the local cache, the Screenlet
loads it from there. If it doesn't exist there, the Screenlet requests
the portrait from the portal and uses the delegate to notify the
developer about any connection errors. \textbar{} Use this policy to
save bandwidth and loading time in the event a local (but probably
outdated) portrait exists. \textbar{}

\noindent\hrulefill

When editing the portrait, the Screenlet supports the following offline
mode policies:

\noindent\hrulefill

Policy \textbar{} What happens \textbar{} When to use \textbar{}
\texttt{remote-only} \textbar{} The Screenlet sends the user portrait to
the portal. If a connection issue occurs, the Screenlet uses the
delegate to notify the developer about the error, but it also discards
the new portrait. \textbar{} Use this policy when you need to make sure
portal always has the most recent version of the portrait. \textbar{}
\texttt{cache-only} \textbar{} The Screenlet stores the user portrait in
the local cache. \textbar{} Use this policy when you need to save the
portrait locally, but don't want to change the portrait in the portal.
\textbar{} \texttt{remote-first} \textbar{} The Screenlet sends the user
portrait to the portal. If this succeeds, the Screenlet also stores the
portrait in the local cache for later usage. If a connection issue
occurs, the Screenlet stores the portrait in the local cache with the
\emph{dirty flag} enabled. This causes the portrait to be sent to the
portal when the synchronization process runs. \textbar{} Use this policy
when you need to make sure the Screenlet sends the new portrait to the
portal as soon as the connection is restored. \textbar{}
\texttt{cache-first} \textbar{} The Screenlet stores the user portrait
in the local cache and then sends it to the portal. If a connection
issue occurs, the Screenlet stores the portrait in the local cache with
the \emph{dirty flag} enabled. This causes the portrait to be sent to
the portal when the synchronization process runs. \textbar{} Use this
policy when you need to make sure the Screenlet sends the new portrait
to the portal as soon as the connection is restored. Compared to
\texttt{remote-first}, this policy always stores the portrait in the
cache. The \texttt{remote-first} policy only stores the new image in the
event of a network error. \textbar{}

\noindent\hrulefill

\subsection{Attributes}\label{attributes-24}

\noindent\hrulefill

Attribute \textbar{} Data type \textbar{} Explanation \textbar{}
\texttt{borderWidth} \textbar{} \texttt{number} \textbar{} The size in
pixels for the portrait's border. The default value is 1. Set this to
\texttt{0} if you want to hide the border.\textbar{}
\texttt{borderColor} \textbar{} \texttt{UIColor} \textbar{} The border's
color. Use the system's transparent color to hide the border. \textbar{}
\texttt{editable} \textbar{} \texttt{boolean} \textbar{} Lets the user
change the portrait image by taking a photo or selecting a gallery
picture. The default value is \texttt{false}. Portraits loaded with the
\texttt{load(portraitId,\ uuid,\ male)} method aren't editable.
\textbar{} \texttt{offlinePolicy} \textbar{} \texttt{string} \textbar{}
Configure the loading and saving behavior in case of connectivity
issues. For more details, read the ``Offline'' section below. \textbar{}

\noindent\hrulefill

\subsection{Methods}\label{methods-12}

\noindent\hrulefill

Method \textbar{} Return \textbar{} Explanation \textbar{}
\texttt{loadLoggedUserPortrait()} \textbar{} \texttt{boolean} \textbar{}
Starts the request to load the currently logged in user's portrait image
(see the \texttt{SessionContext} class). \textbar{}
\texttt{load(userId)} \textbar{} \texttt{boolean} \textbar{} Starts the
request to load the specified user's portrait image. \textbar{}
\texttt{load(portraitId,\ uuid,\ male)} \textbar{} \texttt{boolean}
\textbar{} Starts the request to load the portrait image using the
specified user's data. The parameters \texttt{portraitId} and
\texttt{uuid} can be retrieved by using the
\texttt{SessionContext.userAttributes()} method. \textbar{}
\texttt{load(companyId,\ emailAddress)} \textbar{} \texttt{boolean}
\textbar{} Starts the request to load the portrait image using the
user's email address. \textbar{} \texttt{load(companyId,\ screenName)}
\textbar{} \texttt{boolean} \textbar{} Starts the request to load the
portrait image using the user's screen name. \textbar{}

\noindent\hrulefill

\subsection{Delegate}\label{delegate-3}

The User Portrait Screenlet delegates some events to an object that
conforms to the \texttt{UserPortraitScreenletDelegate} protocol. This
protocol lets you implement the following methods:

\begin{itemize}
\item
  \texttt{-\ screenlet:onUserPortraitResponseImage:}: Called when an
  image is received from the server. You can then apply image filters
  (grayscale, for example) and return the new image. You can return the
  original image supplied as the argument if you don't want to modify
  it.
\item
  \texttt{-\ screenlet:onUserPortraitError:}: Called when an error
  occurs in the process. The \texttt{NSError} object describes the
  error.
\item
  \texttt{-\ screenlet:onUserPortraitUploaded:}: Called when a new
  portrait is uploaded to the server. You receive the user attributes as
  a parameter.
\item
  \texttt{-\ screenlet:onUserPortraitUploadError:}: Called when an error
  occurs in the upload process. The \texttt{NSError} object describes
  the error.
\end{itemize}

\section{DDL Form Screenlet for iOS}\label{ddl-form-screenlet-for-ios}

\subsection{Requirements}\label{requirements-25}

\begin{itemize}
\tightlist
\item
  Xcode 9.3 or above
\item
  iOS 11 SDK
\item
  Liferay Portal 6.2 CE/EE, Liferay CE Portal 7.0/7.1, Liferay DXP 7.0+
\item
  Liferay Screens Compatibility app
  (\href{http://www.liferay.com/marketplace/-/mp/application/54365664}{CE}
  or
  \href{http://www.liferay.com/marketplace/-/mp/application/54369726}{EE/DXP}).
  This app is preinstalled in Liferay CE Portal 7.0/7.1 and Liferay DXP
  7.0+.
\end{itemize}

\subsection{Compatibility}\label{compatibility-25}

\begin{itemize}
\tightlist
\item
  iOS 9 and above
\end{itemize}

\subsection{Xamarin Requirements}\label{xamarin-requirements-25}

\begin{itemize}
\tightlist
\item
  Visual Studio 7.2
\item
  Mono .NET framework 5.4.1.6
\end{itemize}

\subsection{Features}\label{features-25}

DDL Form Screenlet can be used to show a collection of fields so that a
user can fill in their values. Initial or existing values may be shown
in the fields. Fields of the following data types are supported:

\begin{itemize}
\tightlist
\item
  \emph{Boolean}: A two state value typically shown using a checkbox.
\item
  \emph{Date}: A formatted date value. The format depends on the
  device's locale.
\item
  \emph{Decimal, Integer, and Number}: A numeric value.
\item
  \emph{Document and Media}: A file stored on the current device. It can
  be uploaded to a specific portal repository.
\item
  \emph{Radio}: A set of options to choose from. A single option must be
  chosen.
\item
  \emph{Select}: A selection box of options to choose from. A single
  option must be chosen.
\item
  \emph{Text}: A single line of text.
\item
  \emph{Text Box}: Supports multiple lines of text.
\end{itemize}

DDL Form Screenlet also supports the following features:

\begin{itemize}
\tightlist
\item
  Stored records can support a specific workflow.
\item
  A Submit button can be shown at the end of the form.
\item
  Required values and validation for fields can be used.
\item
  Users can traverse the form fields from the keyboard.
\item
  Supports i18n in record values and labels.
\end{itemize}

There are also a few limitations you should be aware of when using DDL
Form Screenlet. They are listed here:

\begin{itemize}
\tightlist
\item
  Nested fields in the data definition aren't supported.
\item
  Selection of multiple items in the Radio and Select data types isn't
  supported yet.
\end{itemize}

\subsection{JSON Services Used}\label{json-services-used-24}

Screenlets in Liferay Screens call JSON web services in the portal. This
Screenlet calls the following services and methods.

\noindent\hrulefill

\begin{longtable}[]{@{}
  >{\raggedright\arraybackslash}p{(\columnwidth - 4\tabcolsep) * \real{0.3889}}
  >{\raggedright\arraybackslash}p{(\columnwidth - 4\tabcolsep) * \real{0.3333}}
  >{\raggedright\arraybackslash}p{(\columnwidth - 4\tabcolsep) * \real{0.2778}}@{}}
\toprule\noalign{}
\begin{minipage}[b]{\linewidth}\raggedright
Service
\end{minipage} & \begin{minipage}[b]{\linewidth}\raggedright
Method
\end{minipage} & \begin{minipage}[b]{\linewidth}\raggedright
Notes
\end{minipage} \\
\midrule\noalign{}
\endhead
\bottomrule\noalign{}
\endlastfoot
\texttt{DDMStructureService} & \texttt{getStructureWithStructureId} &
Load form \\
\texttt{ScreensddlrecordService} (Screens compatibility plugin) &
\texttt{getDdlRecord} & Load record \\
\texttt{DLAppService} & \texttt{addFileEntry} & Upload document \\
\texttt{DDLRecordService} & \texttt{addRecord} & Submit form \\
\texttt{DDLRecordService} & \texttt{updateRecord} & Update form \\
\end{longtable}

\noindent\hrulefill

\subsection{Module}\label{module-25}

\begin{itemize}
\tightlist
\item
  DDL
\end{itemize}

\subsection{Themes}\label{themes-4}

\begin{itemize}
\tightlist
\item
  Default
\end{itemize}

The Default Theme uses a standard \texttt{UITableView} to show a
scrollable list of fields. Other Themes may use a different component,
such as \texttt{UICollectionView} or others, to show the fields.

\begin{figure}
\centering
\includegraphics{./images/screens-ios-ddlform.png}
\caption{DDL Form Screenlet using the Default (\texttt{default}) Theme.}
\end{figure}

\subsubsection{Custom Cells}\label{custom-cells}

A Theme needs to define a cell view for each field type. For instance,
the \texttt{xib} file \texttt{DDLFieldDateTableCell\_default} is used to
render \texttt{Date} fields in the Default Theme.

If you want a specific field to have a unique appearance, you can
customize your field's display by using the following filename pattern,
where \texttt{XXX} is your field's name:
\texttt{DDLCustomFieldXXXTableCell\_default}. For example, the ``Are you
a subscriber?'' field in screenshot above shows how text fields appear
in the Default Theme. If you want to customize this, you don't need to
create an entire Theme. You just need to create an \texttt{xib} file for
the field \texttt{subscriberName}. The filename is therefore
\texttt{DDLCustomFieldSubscriberNameTableCell\_default}. Be careful to
keep the same components and \texttt{IBOutlet} defined in the custom
file.

\subsection{Portal Configuration}\label{portal-configuration-12}

Before using DDL Form Screenlet, you should make sure that Dynamic Data
Lists and Data Types are configured properly in the portal. Refer to the
\href{/docs/7-0/user/-/knowledge_base/u/creating-data-definitions}{Creating
Data Definitions} and
\href{/docs/7-0/user/-/knowledge_base/u/creating-data-lists}{Creating
Data Lists} sections of the User Guide for more details. If Workflow is
required, it must also be configured. See the
\href{/docs/7-0/user/-/knowledge_base/u/using-workflow}{Using Workflow}
section of the User Guide for details.

\subsubsection{Permissions}\label{permissions-1}

To use DDL Form Screenlet to add new records, you must grant the Add
Record permission in the Dynamic Data List in the portal. If you want to
use DDL Form Screenlet to view or edit record values, you must also
grant the View and Update permissions, respectively. The Add Record,
View, and Update permissions are highlighted by the red boxes in the
following screenshot:

\begin{figure}
\centering
\includegraphics{./images/screens-portal-permission-ddl.png}
\caption{The permissions for adding, viewing, and editing DDL records.}
\end{figure}

Also, if your form includes at least one Documents and Media field, you
must grant permissions in the target repository and folder. For more
details, see the \texttt{repositoryId} and \texttt{folderId} attributes
below.

\begin{figure}
\centering
\includegraphics{./images/screens-portal-permission-folder-add.png}
\caption{The permission for adding a document to a Documents and Media
folder.}
\end{figure}

For more details, please see the User Guide sections
\href{/docs/7-0/user/-/knowledge_base/u/creating-data-definitions}{Creating
Data Definitions},
\href{/docs/7-0/user/-/knowledge_base/u/creating-data-lists}{Creating
Data Lists}, and
\href{/docs/7-0/user/-/knowledge_base/u/using-workflow}{Using Workflow}.

\subsection{Offline}\label{offline-24}

This Screenlet supports offline mode so it can function without a
network connection. For more information on how offline mode works, see
the
\href{/docs/7-0/tutorials/-/knowledge_base/t/architecture-of-offline-mode-in-liferay-screens}{tutorial
on its architecture}.

When loading the form or record, the Screenlet supports the following
offline mode policies:

\noindent\hrulefill

Policy \textbar{} What happens \textbar{} When to use \textbar{}
\texttt{remote-only} \textbar{} The Screenlet loads the form or record
from the portal. If a connection issue occurs, the Screenlet uses the
delegate to notify the developer about the error. If the Screenlet loads
the form or record, it stores the received data (record structure and
data) in the local cache for later use. \textbar{} Use this policy when
you always need to show updated data, and show nothing when there's no
connection.\textbar{} \texttt{cache-only} \textbar{} The Screenlet loads
the form or record from the local cache. If the form or record isn't
there, the Screenlet uses the delegate to notify the developer about the
error. \textbar{} Use this policy when you always need to show local
data, without retrieving remote information under any
circumstance.\textbar{} \texttt{remote-first} \textbar{} The Screenlet
requests the form or record from the portal. The Screenlet shows the
record or form to the user and stores it in the local cache for later
use. If a connection issue occurs, the Screenlet retrieves the form or
record from the local cache. If the form or record doesn't exist there,
the Screenlet uses the delegate to notify the developer about the error.
\textbar{} Use this policy to show the most recent version of the data
when connected, but show an outdated version when there's no connection.
\textbar{} \texttt{cache-first} \textbar{} If the form or record exists
in the local cache, the Screenlet loads it from there. If it doesn't
exist there, the Screenlet requests it from the portal and notifies the
developer about any errors that occur (including connectivity errors).
\textbar{} Use this policy to save bandwidth and loading time in case
you have local (but probably outdated) data. \textbar{}

\noindent\hrulefill

When editing the record, the Screenlet supports the following offline
mode policies:

\noindent\hrulefill

Policy \textbar{} What happens \textbar{} When to use \textbar{}
\texttt{remote-only} \textbar{} The Screenlet sends the record to the
portal. If a connection issue occurs, the Screenlet uses the delegate to
notify the developer about the error, but it also discards the record.
\textbar{} Use this policy to make sure the portal always has the most
recent version of the record. \textbar{} \texttt{cache-only} \textbar{}
The Screenlet stores the record in the local cache. \textbar{} Use this
policy when you need to save the data locally, but don't want to update
the data in the portal (update or add record). \textbar{}
\texttt{remote-first} \textbar{} The Screenlet sends the record to the
portal. If this succeeds, it also stores the record in the local cache
for later usage. If a connection issue occurs, then Screenlet stores the
record in the local cache with the \emph{dirty flag} enabled. This
causes the synchronization process to send the record to the portal when
it runs. \textbar{} Use this policy when you need to make sure the
Screenlet sends the record to the portal as soon as the connection is
restored. \textbar{} \texttt{cache-first} \textbar{} The Screenlet
stores the record in the local cache and then sends it to the remote
portal. If a connection issue occurs, then Screenlet stores the record
in the local cache with the \emph{dirty flag} enabled. This causes the
the synchronization process to send the record to the portal when it
runs. \textbar{} Use this policy when you need to make sure the
Screenlet sends the record to the portal as soon as the connection is
restored. Compared to \texttt{remote-first}, this policy always stores
the record in the cache. The \texttt{remote-first} policy only stores
the record in the event of a network error. \textbar{}

\noindent\hrulefill

\subsection{Required Attributes}\label{required-attributes-20}

\begin{itemize}
\tightlist
\item
  \texttt{structureId}
\item
  \texttt{recordSetId}
\end{itemize}

\subsection{Attributes}\label{attributes-25}

\noindent\hrulefill

Attribute \textbar{} Data Type \textbar{} Explanation \textbar{}
\texttt{structureId} \textbar{} \texttt{number} \textbar{} This is the
identifier of a data definition for your site in Liferay. To find the
identifiers for your data definitions, click \emph{Admin} from the
Dockbar and select \emph{Content}. Then click \emph{Dynamic Data Lists}
and click the \emph{Manage Data Definitions} button. The identifier of
each data definition is in the ID column of the table that appears.
\textbar{} \texttt{groupId} \textbar{} \texttt{number} \textbar{} The
site (group) identifier where the record is stored. If this value is
\texttt{0}, the \texttt{groupId} specified in
\texttt{LiferayServerContext} is used. \textbar{} \texttt{recordSetId}
\textbar{} \texttt{number} \textbar{} The identifier of a dynamic data
list. To find the identifiers for your dynamic data lists, click
\emph{Admin} from the Dockbar and select \emph{Content}. Then click
\emph{Dynamic Data Lists}. The identifier of each dynamic data list is
in the ID column of the table that appears. \textbar{} \texttt{recordId}
\textbar{} \texttt{number} \textbar{} The identifier of the record you
want to show. Setting the \texttt{editable} attribute to \texttt{true}
allows editing of the record's values. The \texttt{recordId} can be
obtained from other methods or delegates. \textbar{}
\texttt{repositoryId} \textbar{} \texttt{number} \textbar{} The
identifier of the Documents and Media repository to upload to. If this
value is \texttt{0}, the default repository for the site specified in
\texttt{groupId} is used. \textbar{} \texttt{folderId} \textbar{}
\texttt{number} \textbar{} The identifier of the folder where Documents
and Media files are uploaded. If this value is \texttt{0}, the root
folder is used. \textbar{} \texttt{filePrefix} \textbar{}
\texttt{string} \textbar{} The prefix to attach to the names of files
uploaded to a Documents and Media repository. A random GUID string is
appended following the prefix. \textbar{} \texttt{autoLoad} \textbar{}
\texttt{boolean} \textbar{} Sets whether or not the form is loaded when
the Screenlet is shown. If \texttt{recordId} is set, the record value is
loaded together with the form definition. \textbar{}
\texttt{autoscrollOnValidation} \textbar{} \texttt{boolean} \textbar{}
Sets whether or not the form automatically scrolls to the first failed
field when validation is used. \textbar{} \texttt{showSubmitButton}
\textbar{} \texttt{boolean} \textbar{} Sets whether or not the form
shows a submit button at the bottom. If this is set to \texttt{false},
you should call the \texttt{submitForm()} method. \textbar{}
\texttt{editable} \textbar{} \texttt{boolean} \textbar{} Sets whether
the values can be changed by the user. The default is \texttt{true}.
\textbar{}

\noindent\hrulefill

\subsection{Methods}\label{methods-13}

\noindent\hrulefill

Method \textbar{} Return Type \textbar{} Explanation \textbar{}
\texttt{loadForm()} \textbar{} \texttt{boolean} \textbar{} Starts the
request to load the form definition. The form fields are shown when the
response is received. This method returns \texttt{true} if the request
is sent. \textbar{} \texttt{loadRecord()} \textbar{} \texttt{boolean}
\textbar{} Starts the request to load the record specified in
\texttt{recordId}. If needed, the form definition is also loaded. The
form fields are shown filled with record values when the response is
received. This method returns \texttt{true} if the request is sent.
\textbar{} \texttt{submitForm()} \textbar{} \texttt{boolean} \textbar{}
Starts the request to submit form values to the dynamic data list
specified in \texttt{recordSetId}. All fields are validated prior to
submission. Validation errors stop the submit process. \textbar{}

\noindent\hrulefill

\subsection{Delegate}\label{delegate-4}

DDL Form Screenlet delegates some events to an object that conforms with
the \texttt{DDLFormScreenletDelegate} protocol. This protocol lets you
implement the following methods:

\begin{itemize}
\item
  \texttt{-\ screenlet:onFormLoaded:}: Called when the form is loaded.
  The second parameter (\texttt{record}) contains only field
  definitions.
\item
  \texttt{-\ screenlet:onFormLoadError:}: Called when an error occurs
  while loading the form. The \texttt{NSError} object describes the
  error.
\item
  \texttt{-\ screenlet:onRecordLoaded:}: Called when a form with values
  loads. The second parameter (\texttt{record}) contains field
  definitions and values. The method \texttt{onFormLoadResult} is called
  before \texttt{onRecordLoaded}.
\item
  \texttt{-\ screenlet:onRecordLoadError:}: Called when an error occurs
  while loading a record. The \texttt{NSError} object describes the
  error.
\item
  \texttt{-\ screenlet:onFormSubmitted:}: Called when the form values
  are successfully submitted to the server.
\item
  \texttt{-\ screenlet:onFormSubmitError:}: Called when an error occurs
  while submitting the form. The \texttt{NSError} object describes the
  error.
\item
  \texttt{-\ screenlet:onDocumentFieldUploadStarted:}: Called when the
  upload of a Documents and Media field begins.
\item
  \texttt{-\ screenlet:onDocumentField:uploadedBytes:totalBytes:}:
  Called when a block of bytes in a Documents and Media field is
  uploaded. This method is intended to track progress of the uploads.
\item
  \texttt{-\ screenlet:onDocumentField:uploadResult:}: Called when a
  Documents and Media field upload is completed.
\item
  \texttt{-\ screenlet:onDocumentField:uploadError:}: Called when an
  error occurs in the Documents and Media upload process. The
  \texttt{NSError} object describes the error.
\end{itemize}

\section{DDL List Screenlet for iOS}\label{ddl-list-screenlet-for-ios}

\subsection{Requirements}\label{requirements-26}

\begin{itemize}
\tightlist
\item
  Xcode 9.3 or above
\item
  iOS 11 SDK
\item
  Liferay Portal 6.2 CE/EE, Liferay CE Portal 7.0/7.1, Liferay DXP 7.0+
\item
  Liferay Screens Compatibility app
  (\href{http://www.liferay.com/marketplace/-/mp/application/54365664}{CE}
  or
  \href{http://www.liferay.com/marketplace/-/mp/application/54369726}{EE/DXP}).
  This app is preinstalled in Liferay CE Portal 7.0/7.1 and Liferay DXP
  7.0+.
\end{itemize}

\subsection{Compatibility}\label{compatibility-26}

\begin{itemize}
\tightlist
\item
  iOS 9 and above
\end{itemize}

\subsection{Xamarin Requirements}\label{xamarin-requirements-26}

\begin{itemize}
\tightlist
\item
  Visual Studio 7.2
\item
  Mono .NET framework 5.4.1.6
\end{itemize}

\subsection{Features}\label{features-26}

The DDL List Screenlet enables the following features:

\begin{itemize}
\tightlist
\item
  Shows a scrollable collection of DDL records.
\item
  Implements
  \href{http://www.iosnomad.com/blog/2014/4/21/fluent-pagination}{fluent
  pagination} with configurable page size.
\item
  Allows filtering of records by creator.
\item
  Supports i18n in record values.
\end{itemize}

\subsection{JSON Services Used}\label{json-services-used-25}

Screenlets in Liferay Screens call JSON web services in the portal. This
Screenlet calls the following services and methods.

\noindent\hrulefill

\begin{longtable}[]{@{}
  >{\raggedright\arraybackslash}p{(\columnwidth - 4\tabcolsep) * \real{0.3889}}
  >{\raggedright\arraybackslash}p{(\columnwidth - 4\tabcolsep) * \real{0.3333}}
  >{\raggedright\arraybackslash}p{(\columnwidth - 4\tabcolsep) * \real{0.2778}}@{}}
\toprule\noalign{}
\begin{minipage}[b]{\linewidth}\raggedright
Service
\end{minipage} & \begin{minipage}[b]{\linewidth}\raggedright
Method
\end{minipage} & \begin{minipage}[b]{\linewidth}\raggedright
Notes
\end{minipage} \\
\midrule\noalign{}
\endhead
\bottomrule\noalign{}
\endlastfoot
\texttt{ScreensddlrecordService} (Screens compatibility plugin) &
\texttt{getDdlRecords} & With \texttt{ddlRecordSetId}, or
\texttt{ddlRecordSetId} and \texttt{userId} \\
\texttt{ScreensddlrecordService} (Screens compatibility plugin) &
\texttt{getDdlRecordsCount} & \\
\end{longtable}

\noindent\hrulefill

\subsection{Module}\label{module-26}

\begin{itemize}
\tightlist
\item
  DDL
\end{itemize}

\subsection{Themes}\label{themes-5}

\begin{itemize}
\tightlist
\item
  The Default Theme uses a standard \texttt{UITableView} to show the
  scrollable list. Other Themes may use a different component, such as
  \texttt{UICollectionView} or others, to show the items.
\end{itemize}

\begin{figure}
\centering
\includegraphics{./images/screens-ios-ddllist.png}
\caption{The DDL List Screenlet using the Default (\texttt{default})
Theme.}
\end{figure}

\subsection{Portal Configuration}\label{portal-configuration-13}

Dynamic Data Lists (DDL) and Data Types should be configured in the
portal. For more details, please refer to the Liferay User Guide
sections
\href{/docs/7-0/user/-/knowledge_base/u/creating-data-definitions}{Creating
Data Definitions} and
\href{/docs/7-0/user/-/knowledge_base/u/creating-data-lists}{Creating
Data Lists}.

\subsection{Offline}\label{offline-25}

This Screenlet supports offline mode so it can function without a
network connection. For more information on how offline mode works, see
the
\href{/docs/7-0/tutorials/-/knowledge_base/t/architecture-of-offline-mode-in-liferay-screens}{tutorial
on its architecture}. Here are the offline mode policies that you can
use with this Screenlet:

\noindent\hrulefill

Policy \textbar{} What happens \textbar{} When to use \textbar{}
\texttt{remote-only} \textbar{} The Screenlet loads the list from the
portal. If a connection issue occurs, the Screenlet uses the delegate to
notify the developer about the error. If the Screenlet successfully
loads the list, it stores the data in the local cache for later use.
\textbar{} Use this policy when you always need to show updated data,
and show nothing when there's no connection. \textbar{}
\texttt{cache-only} \textbar{} The Screenlet loads the list from the
local cache. If the list isn't there, the Screenlet uses the delegate to
notify the developer about the error. \textbar{} Use this policy when
you always need to show local data, without retrieving remote
information under any circumstance. \textbar{} \texttt{remote-first}
\textbar{} The Screenlet loads the list from the portal. If this
succeeds, the Screenlet shows the list to the user and stores it in the
local cache for later use. If a connection issue occurs, the Screenlet
retrieves the list from the local cache. If the list doesn't exist
there, the Screenlet uses the delegate to notify the developer about the
error. \textbar{} Use this policy to show the most recent version of the
data when connected, but show an outdated version when there's no
connection. \textbar{} \texttt{cache-first} \textbar{} The Screenlet
loads the list from the local cache. If the list isn't there, the
Screenlet requests it from the portal and notifies the developer about
any errors that occur (including connectivity errors). \textbar{} Use
this policy to save bandwidth and loading time in case you have local
(but probably outdated) data. \textbar{}

\noindent\hrulefill

\subsection{Required Attributes}\label{required-attributes-21}

\begin{itemize}
\tightlist
\item
  \texttt{recordSetId}
\item
  \texttt{labelFields}
\end{itemize}

\subsection{Attributes}\label{attributes-26}

\noindent\hrulefill

Attribute \textbar{} Data type \textbar{} Explanation \textbar{}
\texttt{recordSetId} \textbar{} \texttt{number} \textbar{} The ID of the
DDL being called. To find the IDs for your DDLs, first open the Product
Menu and select the site that contains your DDLs. Then click
\emph{Content} → \emph{Dynamic Data Lists}. Each DDL's ID is in the
table's ID column. \textbar{} \texttt{userId} \textbar{} \texttt{number}
\textbar{} The ID of the user to filter records on. Records aren't
filtered if the \texttt{userId} is \texttt{0}. The default value is
\texttt{0}. \textbar{} \texttt{labelFields} \textbar{} \texttt{string}
\textbar{} The comma-separated names of the DDL fields to show. Refer to
the list's data definition to find the field names. To do so, first open
the Product Menu and select the site that contains your DDLs. Then click
\emph{Content} → \emph{Dynamic Data Lists}, and find the find the icon
(\includegraphics{./images/icon-options.png}) for the Dynamic Data
List configuration menu at the upper right. Click this icon and select
\emph{Manage Data Definitions}. You can view the fields by clicking on
any of the data definitions in the table that appears. Note that the
appearance of these values in your app depends on the Theme selected by
the user. \textbar{} \texttt{offlinePolicy} \textbar{} \texttt{string}
\textbar{} The offline mode setting. The default value is
\texttt{remote-first}. See the
\href{/docs/7-0/reference/-/knowledge_base/r/ddllistscreenlet-for-ios\#offline}{Offline
section} for details. \textbar{} \texttt{autoLoad} \textbar{}
\texttt{boolean} \textbar{} Whether the list loads automatically when
the Screenlet appears in the app's UI. The default value is
\texttt{true}. \textbar{} \texttt{refreshControl} \textbar{}
\texttt{boolean} \textbar{} Whether a standard
\href{https://developer.apple.com/library/ios/documentation/UIKit/Reference/UIRefreshControl_class/}{iOS
\texttt{UIRefreshControl}} appears when the user performs the pull to
refresh gesture. The default value is \texttt{true}. \textbar{}
\texttt{firstPageSize} \textbar{} \texttt{number} \textbar{} The number
of items retrieved from the server for display on the first page. The
default value is \texttt{50}. \textbar{} \texttt{pageSize} \textbar{}
\texttt{number} \textbar{} The number of items retrieved from the server
for display on the second and subsequent pages. The default value is
\texttt{25}. \textbar{} \texttt{obcClassName} \textbar{} \texttt{string}
\textbar{} The name of the \texttt{OrderByComparator} class to use to
sort the results. Omit this property if you don't want to sort the
results.
\href{https://github.com/liferay/liferay-portal/tree/master/modules/apps/forms-and-workflow/dynamic-data-lists/dynamic-data-lists-api/src/main/java/com/liferay/dynamic/data/lists/util/comparator}{Click
here} to see some comparator classes. Note, however, that not all of
these classes can be used with \texttt{obcClassName}. You can only use
comparator classes that extend
\texttt{OrderByComparator\textless{}DDLRecord\textgreater{}}. You can
also create your own comparator classes that extend
\texttt{OrderByComparator\textless{}DDLRecord\textgreater{}}. \textbar{}

\noindent\hrulefill

\subsection{Methods}\label{methods-14}

\noindent\hrulefill

Method \textbar{} Return \textbar{} Explanation \textbar{}
\texttt{loadList()} \textbar{} \texttt{boolean} \textbar{} Starts the
request to load the list of records. The list is shown when the response
is received. This method returns \texttt{true} if the request is sent.
\textbar{}

\noindent\hrulefill

\subsection{Delegate}\label{delegate-5}

The DDL List Screenlet delegates some events in an object that conforms
to the \texttt{DDLListScreenletDelegate} protocol. This protocol lets
you implement the following methods:

\begin{itemize}
\item
  \texttt{-\ screenlet:onDDLListResponseRecords:}: Called when a page of
  contents is received. Note that this method may be called more than
  once; once for each retrieved page.
\item
  \texttt{-\ screenlet:onDDLListError:}: Called when an error occurs in
  the process. The \texttt{NSError} object describes the error.
\item
  \texttt{-\ screenlet:onDDLSelectedRecord:}: Called when an item in the
  list is selected.
\end{itemize}

\section{Asset List Screenlet for
iOS}\label{asset-list-screenlet-for-ios}

\subsection{Requirements}\label{requirements-27}

\begin{itemize}
\tightlist
\item
  Xcode 9.3 or above
\item
  iOS 11 SDK
\item
  Liferay Portal 6.2 CE/EE, Liferay CE Portal 7.0/7.1, Liferay DXP 7.0+
\item
  Liferay Screens Compatibility app
  (\href{http://www.liferay.com/marketplace/-/mp/application/54365664}{CE}
  or
  \href{http://www.liferay.com/marketplace/-/mp/application/54369726}{EE/DXP}).
  This app is preinstalled in Liferay CE Portal 7.0/7.1 and Liferay DXP
  7.0+.
\end{itemize}

\subsection{Compatibility}\label{compatibility-27}

\begin{itemize}
\tightlist
\item
  iOS 9 and above
\end{itemize}

\subsection{Xamarin Requirements}\label{xamarin-requirements-27}

\begin{itemize}
\tightlist
\item
  Visual Studio 7.2
\item
  Mono .NET framework 5.4.1.6
\end{itemize}

\subsection{Features}\label{features-27}

The Asset List Screenlet can be used to show lists of
\href{/docs/7-0/tutorials/-/knowledge_base/t/asset-framework}{assets}
from a Liferay instance. For example, you can use the Screenlet to show
a scrollable collection of assets. It also implements
\href{http://www.iosnomad.com/blog/2014/4/21/fluent-pagination}{fluent
pagination} with configurable page size. The Asset List Screenlet can
show assets of the following classes:

\begin{itemize}
\tightlist
\item
  \texttt{BlogsEntry}
\item
  \texttt{BookmarksEntry}
\item
  \texttt{BookmarksFolder}
\item
  \texttt{CalendarEvent}
\item
  \texttt{DLFileEntry}
\item
  \texttt{DDLRecord}
\item
  \texttt{DDLRecordSet}
\item
  \texttt{Group}
\item
  \texttt{JournalArticle} (Web Content)
\item
  \texttt{JournalFolder}
\item
  \texttt{Layout}
\item
  \texttt{LayoutRevision}
\item
  \texttt{MBThread}
\item
  \texttt{MBCategory}
\item
  \texttt{MBDiscussion}
\item
  \texttt{MBMailingList}
\item
  \texttt{Organization}
\item
  \texttt{User}
\item
  \texttt{WikiPage}
\item
  \texttt{WikiPageResource}
\item
  \texttt{WikiNode}
\end{itemize}

The Asset List Screenlet also supports i18n in asset values.

\subsection{JSON Services Used}\label{json-services-used-26}

Screenlets in Liferay Screens call JSON web services in the portal. This
Screenlet calls the following services and methods.

\noindent\hrulefill

\begin{longtable}[]{@{}
  >{\raggedright\arraybackslash}p{(\columnwidth - 4\tabcolsep) * \real{0.3889}}
  >{\raggedright\arraybackslash}p{(\columnwidth - 4\tabcolsep) * \real{0.3333}}
  >{\raggedright\arraybackslash}p{(\columnwidth - 4\tabcolsep) * \real{0.2778}}@{}}
\toprule\noalign{}
\begin{minipage}[b]{\linewidth}\raggedright
Service
\end{minipage} & \begin{minipage}[b]{\linewidth}\raggedright
Method
\end{minipage} & \begin{minipage}[b]{\linewidth}\raggedright
Notes
\end{minipage} \\
\midrule\noalign{}
\endhead
\bottomrule\noalign{}
\endlastfoot
\texttt{ScreensddlrecordService} (Screens compatibility plugin) &
\texttt{getAssetEntries} & With \texttt{entryQuery} \\
\texttt{ScreensddlrecordService} (Screens compatibility plugin) &
\texttt{getAssetEntries} & With \texttt{companyId}, \texttt{groupId},
and \texttt{portletItemName} \\
\texttt{AssetEntryService} & \texttt{getEntriesCount} & \\
\end{longtable}

\noindent\hrulefill

\subsection{Module}\label{module-27}

\begin{itemize}
\tightlist
\item
  None
\end{itemize}

\subsection{Themes}\label{themes-6}

\begin{itemize}
\tightlist
\item
  Default
\end{itemize}

The Default Theme uses a standard \texttt{UITableView} to show the
scrollable list. Other Themes may use a different component, such as
\texttt{UICollectionView} or others, to show the items.

\begin{figure}
\centering
\includegraphics{./images/screens-ios-assetlist.png}
\caption{Asset List Screenlet using the Default (\texttt{default})
Theme.}
\end{figure}

\subsection{Offline}\label{offline-26}

This Screenlet supports offline mode so it can function without a
network connection. For more information on how offline mode works, see
the
\href{/docs/7-0/tutorials/-/knowledge_base/t/architecture-of-offline-mode-in-liferay-screens}{tutorial
on its architecture}. Here are the offline mode policies that you can
use with this Screenlet:

\noindent\hrulefill

Policy \textbar{} What happens \textbar{} When to use \textbar{}
\texttt{remote-only} \textbar{} The Screenlet loads the list from the
portal. If a connection issue occurs, the Screenlet uses the delegate to
notify the developer about the error. If the Screenlet successfully
loads the list, it stores the data in the local cache for later use.
\textbar{} Use this policy when you always need to show updated data,
and show nothing when there's no connection. \textbar{}
\texttt{cache-only} \textbar{} The Screenlet loads the list from the
local cache. If the list isn't there, the Screenlet uses the delegate to
notify the developer about the error. \textbar{} Use this policy when
you always need to show local data, without retrieving remote
information under any circumstance. \textbar{} \texttt{remote-first}
\textbar{} The Screenlet loads the list from the portal. If this
succeeds, the Screenlet shows the list to the user and stores it in the
local cache for later use. If a connection issue occurs, the Screenlet
retrieves the list from the local cache. If the list doesn't exist
there, the Screenlet uses the delegate to notify the developer about the
error. \textbar{} Use this policy to show the most recent version of the
data when connected, but show an outdated version when there's no
connection. \textbar{} \texttt{cache-first} \textbar{} The Screenlet
loads the list from the local cache. If the list isn't there, the
Screenlet requests it from the portal and notifies the developer about
any errors that occur (including connectivity errors). \textbar{} Use
this policy to save bandwidth and loading time in case you have local
(but probably outdated) data. \textbar{}

\noindent\hrulefill

\subsection{Required Attributes}\label{required-attributes-22}

\begin{itemize}
\tightlist
\item
  \texttt{classNameId}
\end{itemize}

If you don't use \texttt{classNameId}, you must use this attribute:

\begin{itemize}
\tightlist
\item
  \texttt{portletItemName}
\end{itemize}

\subsection{Attributes}\label{attributes-27}

\noindent\hrulefill

Attribute \textbar{} Data type \textbar{} Explanation \textbar{}
\texttt{groupId} \textbar{} \texttt{number} \textbar{} The ID of the
site (group) where the asset is stored. If set to \texttt{0}, the
\texttt{groupId} specified in \texttt{LiferayServerContext} is used. The
default value is \texttt{0}. \textbar{} \texttt{classNameId} \textbar{}
\texttt{number} \textbar{} The ID of the asset's class name. Use values
from the \texttt{AssetClassNameId} class or the Liferay Instance's
\texttt{classname\_} database table. \textbar{} \texttt{portletItemName}
\textbar{} \texttt{string} \textbar{} The name of the
\href{/docs/7-0/user/-/knowledge_base/u/configuration-templates}{configuration
template} you used in the Asset Publisher. To use this feature, add an
Asset Publisher to one of your site's pages (it may be a hidden page),
configure the Asset Publisher's filter (in \emph{Configuration} →
\emph{Setup} → \emph{Asset Selection}), and then use the Asset
Publisher's \emph{Configuration Templates} option to save this
configuration with a name. Use this name as this attribute's value.
\textbar{} \texttt{offlinePolicy} \textbar{} \texttt{string} \textbar{}
The offline mode setting. The default value is \texttt{remote-first}.
See the
\href{/docs/7-0/reference/-/knowledge_base/r/assetlistscreenlet-for-ios\#offline}{Offline
section} for details. \textbar{} \texttt{autoLoad} \textbar{}
\texttt{boolean} \textbar{} Whether the list loads automatically when
the Screenlet appears in the app's UI. The default value is
\texttt{true}. \textbar{} \texttt{refreshControl} \textbar{}
\texttt{boolean} \textbar{} Defines whether a standard
\href{https://developer.apple.com/library/ios/documentation/UIKit/Reference/UIRefreshControl_class/}{ios
\texttt{UIRefreshControl}} appears when the user does the pull to
refresh gesture. The default value is \texttt{true}. \textbar{}
\texttt{firstPageSize} \textbar{} \texttt{number} \textbar{} The number
of items retrieved from the server for display on the first page. The
default value is \texttt{50}. \textbar{} \texttt{pageSize} \textbar{}
\texttt{number} \textbar{} The number of items retrieved from the server
for display on the second and subsequent pages. The default value is
\texttt{25}. \textbar{} \texttt{customEntryQuery} \textbar{}
\texttt{Dictionary} \textbar{} The set of keys (string) and values
(string or number) to be used in the
\href{@platform-ref@/7.0-latest/javadocs/portal-kernel/com/liferay/asset/kernel/service/persistence/AssetEntryQuery.html}{\texttt{AssetEntryQuery}
object}. These values filter the assets returned by the Liferay
instance. \textbar{}

\noindent\hrulefill

\subsection{Methods}\label{methods-15}

\noindent\hrulefill

Method \textbar{} Return \textbar{} Explanation \textbar{}
\texttt{loadList()} \textbar{} \texttt{boolean} \textbar{} Starts the
request to load the list of assets. This list is shown when the response
is received. Returns \texttt{true} if the request is sent. \textbar{}

\noindent\hrulefill

\subsection{Delegate}\label{delegate-6}

The Asset List Screenlet delegates some events to an object that
conforms to the \texttt{AssetListScreenletDelegate} protocol. This
protocol lets you implement the following methods:

\begin{itemize}
\item
  \texttt{-\ screenlet:onAssetListResponse:}: Called when a page of
  assets is received. Note that this method may be called more than
  once; one call for each page received.
\item
  \texttt{-\ screenlet:onAssetListError:}: Called when an error occurs
  in the process. The \texttt{NSError} object describes the error.
\item
  \texttt{-\ screenlet:onAssetSelected:}: Called when an item in the
  list is selected.
\end{itemize}

\section{Web Content Display Screenlet for
iOS}\label{web-content-display-screenlet-for-ios}

\subsection{Requirements}\label{requirements-28}

\begin{itemize}
\tightlist
\item
  Xcode 9.3 or above
\item
  iOS 11 SDK
\item
  Liferay Portal 6.2 CE/EE, Liferay CE Portal 7.0/7.1, Liferay DXP 7.0+
\item
  Liferay Screens Compatibility app
  (\href{http://www.liferay.com/marketplace/-/mp/application/54365664}{CE}
  or
  \href{http://www.liferay.com/marketplace/-/mp/application/54369726}{EE/DXP}).
  This app is preinstalled in Liferay CE Portal 7.0/7.1 and Liferay DXP
  7.0+.
\end{itemize}

\subsection{Compatibility}\label{compatibility-28}

\begin{itemize}
\tightlist
\item
  iOS 9 and above
\end{itemize}

\subsection{Xamarin Requirements}\label{xamarin-requirements-28}

\begin{itemize}
\tightlist
\item
  Visual Studio 7.2
\item
  Mono .NET framework 5.4.1.6
\end{itemize}

\subsection{Features}\label{features-28}

The Web Content Display Screenlet shows web content elements in your
app, rendering the inner HTML of the web content. The Screenlet also
supports i18n, rendering contents differently depending on the device's
current locale.

\subsection{JSON Services Used}\label{json-services-used-27}

Screenlets in Liferay Screens call JSON web services in the portal. This
Screenlet calls the following services and methods.

\noindent\hrulefill

\begin{longtable}[]{@{}
  >{\raggedright\arraybackslash}p{(\columnwidth - 4\tabcolsep) * \real{0.3889}}
  >{\raggedright\arraybackslash}p{(\columnwidth - 4\tabcolsep) * \real{0.3333}}
  >{\raggedright\arraybackslash}p{(\columnwidth - 4\tabcolsep) * \real{0.2778}}@{}}
\toprule\noalign{}
\begin{minipage}[b]{\linewidth}\raggedright
Service
\end{minipage} & \begin{minipage}[b]{\linewidth}\raggedright
Method
\end{minipage} & \begin{minipage}[b]{\linewidth}\raggedright
Notes
\end{minipage} \\
\midrule\noalign{}
\endhead
\bottomrule\noalign{}
\endlastfoot
\texttt{DDMStructureService} & \texttt{getStructureWithStructureId} & \\
\texttt{JournalArticleService} & \texttt{getArticleWithGroupId} & \\
\texttt{JournalArticleService} & \texttt{getArticleContent} & \\
\texttt{ScreensddlrecordService} (Screens compatibility plugin) &
\texttt{getJournalArticleContent} & With \texttt{entryQuery} \\
\end{longtable}

\noindent\hrulefill

\subsection{Module}\label{module-28}

\begin{itemize}
\tightlist
\item
  WebContent
\end{itemize}

\subsection{Themes}\label{themes-7}

\begin{itemize}
\tightlist
\item
  Default
\end{itemize}

The Default Theme uses a standard \texttt{UIWebView} to render the HTML.
Other Themes may use a different component, such as iOS 8's.

\begin{figure}
\centering
\includegraphics{./images/screens-ios-webcontent.png}
\caption{The Web Content Display Screenlet using the Default
(\texttt{default}) Theme}
\end{figure}

\subsection{Portal Configuration}\label{portal-configuration-14}

For the Web Content Display Screenlet to function properly, there should
be web content in the Liferay instance your app connects to. For more
details on web content, please refer to the
\href{/docs/7-0/user/-/knowledge_base/u/creating-web-content}{Creating
Web Content} section of the Liferay User Guide.

\subsection{Offline}\label{offline-27}

This Screenlet supports offline mode so it can function without a
network connection. For more information on how offline mode works, see
the
\href{/docs/7-0/tutorials/-/knowledge_base/t/architecture-of-offline-mode-in-liferay-screens}{tutorial
on its architecture}. Here are the offline mode policies that you can
use with this Screenlet:

\noindent\hrulefill

Policy \textbar{} What happens \textbar{} When to use \textbar{}
\texttt{remote-only} \textbar{} The Screenlet loads the content from the
portal. If a connection issue occurs, the Screenlet uses the delegate to
notify the developer about the error. If the Screenlet successfully
loads the content, it stores the data in the local cache for later use.
\textbar{} Use this policy when you always need to show updated content,
and show nothing when there's no connection. \textbar{}
\texttt{cache-only} \textbar{} The Screenlet loads the content from the
local cache. If the content isn't there, the Screenlet uses the delegate
to notify the developer about the error. \textbar{} Use this policy when
you always need to show local content, without retrieving remote content
under any circumstance. \textbar{} \texttt{remote-first} \textbar{} The
Screenlet loads the content from the portal. If this succeeds, the
Screenlet shows the content to the user and stores it in the local cache
for later use. If a connection issue occurs, the Screenlet retrieves the
content from the local cache. If the content doesn't exist there, the
Screenlet uses the delegate to notify the developer about the error.
\textbar{} Use this policy to show the most recent version of the
content when connected, but show a possibly outdated version when
there's no connection. \textbar{} \texttt{cache-first} \textbar{} The
Screenlet loads the content from the local cache. If the content isn't
there, the Screenlet requests it from the portal and notifies the
developer about any errors that occur (including connectivity errors).
\textbar{} Use this policy to save bandwidth and loading time in case
you have local (but probably outdated) content. \textbar{}

\noindent\hrulefill

\subsection{Required Attributes}\label{required-attributes-23}

\begin{itemize}
\tightlist
\item
  \texttt{articleId}
\end{itemize}

If you have
\href{/docs/7-0/user/-/knowledge_base/u/designing-uniform-content}{structured
web content}, you can alternatively use \texttt{templateId} or
\texttt{structureId} with \texttt{articleId}.

\subsection{Attributes}\label{attributes-28}

\noindent\hrulefill

Attribute \textbar{} Data type \textbar{} Explanation \textbar{}
\texttt{groupId} \textbar{} \texttt{number} \textbar{} The site (group)
identifier where the asset is stored. If this value is \texttt{0}, the
\texttt{groupId} specified in \texttt{LiferayServerContext} is used.
\textbar{} \texttt{articleId} \textbar{} \texttt{string} \textbar{} The
identifier of the web content to display. You can find the identifier by
clicking \emph{Edit} on the web content in the portal. \textbar{}
\texttt{templateId} \textbar{} \texttt{number} \textbar{} The identifier
of the template used to render the web content. This is applicable only
with
\href{/docs/7-0/user/-/knowledge_base/u/designing-uniform-content}{structured
web content}. \textbar{} \texttt{structureId} \textbar{} \texttt{number}
\textbar{} The identifier of the \texttt{DDMStructure} used to model the
web content. This parameter lets the Screenlet retrieve and parse the
structure. \textbar{} \texttt{autoLoad} \textbar{} \texttt{boolean}
\textbar{} Whether the content should be retrieved from the portal as
soon as the Screenlet appears. The default value is \texttt{true}.
\textbar{}

\noindent\hrulefill

\subsection{Methods}\label{methods-16}

\noindent\hrulefill

Method \textbar{} Return \textbar{} Explanation \textbar{}
\texttt{loadWebContent()} \textbar{} \texttt{boolean} \textbar{} Starts
the request to load the web content. The HTML is rendered when the
response is received. Returns \texttt{true} if the request is sent.
\textbar{}

\noindent\hrulefill

\subsection{Delegate}\label{delegate-7}

The Web Content Display Screenlet delegates some events to an object
that conforms to the \texttt{WebContentDisplayScreenletDelegate}
protocol. This protocol lets you implement the following methods:

\begin{itemize}
\item
  \texttt{-\ screenlet:onWebContentResponse:}: Called when the web
  content's HTML is received.
\item
  \texttt{-\ screenlet:onWebContentError:}: Called when an error occurs
  in the process. The \texttt{NSError} object describes the error.
\item
  \texttt{-\ screenlet:onRecordContentResponse:}: Called when a web
  content record is received.
\item
  \texttt{-\ screenlet:onUrlClicked:}: Called when a URL is clicked in
  the web content. Return \texttt{true} to handle the navigation, or
  \texttt{false} to cancel it.
\end{itemize}

\section{Web Content List Screenlet for
iOS}\label{web-content-list-screenlet-for-ios}

\subsection{Requirements}\label{requirements-29}

\begin{itemize}
\tightlist
\item
  Xcode 9.3 or above
\item
  iOS 11 SDK
\item
  Liferay Portal 6.2 CE/EE, Liferay CE Portal 7.0/7.1, Liferay DXP 7.0+
\item
  Liferay Screens Compatibility app
  (\href{http://www.liferay.com/marketplace/-/mp/application/54365664}{CE}
  or
  \href{http://www.liferay.com/marketplace/-/mp/application/54369726}{EE/DXP}).
  This app is preinstalled in Liferay CE Portal 7.0/7.1 and Liferay DXP
  7.0+.
\end{itemize}

\subsection{Compatibility}\label{compatibility-29}

\begin{itemize}
\tightlist
\item
  iOS 9 and above
\end{itemize}

\subsection{Xamarin Requirements}\label{xamarin-requirements-29}

\begin{itemize}
\tightlist
\item
  Visual Studio 7.2
\item
  Mono .NET framework 5.4.1.6
\end{itemize}

\subsection{Features}\label{features-29}

Web Content List Screenlet can show lists of
\href{/docs/7-0/user/-/knowledge_base/u/creating-web-content}{web
content} from a Liferay instance. It can show both basic and
\href{/docs/7-0/user/-/knowledge_base/u/designing-uniform-content}{structured
web content}. The Screenlet also implements
\href{http://www.iosnomad.com/blog/2014/4/21/fluent-pagination}{fluent
pagination} with configurable page size, and supports i18n in asset
values.

\subsection{JSON Services Used}\label{json-services-used-28}

Screenlets in Liferay Screens call JSON web services in the portal. This
Screenlet calls the following services and methods.

\noindent\hrulefill

\begin{longtable}[]{@{}lll@{}}
\toprule\noalign{}
Service & Method & Notes \\
\midrule\noalign{}
\endhead
\bottomrule\noalign{}
\endlastfoot
\texttt{JournalArticleService} & \texttt{getArticlesWithGroupId} & \\
\texttt{JournalArticleService} & \texttt{getArticlesCount} & \\
\end{longtable}

\noindent\hrulefill

\subsection{Module}\label{module-29}

\begin{itemize}
\tightlist
\item
  WebContent
\end{itemize}

\subsection{Themes}\label{themes-8}

\begin{itemize}
\tightlist
\item
  Default
\end{itemize}

The Default Theme uses a standard \texttt{UITableView} to show the
scrollable list. Other Themes may use a different component, such as
\texttt{UICollectionView} or others, to show the contents.

\begin{figure}
\centering
\includegraphics{./images/screens-ios-webcontent-list.png}
\caption{Web Content List Screenlet using the Default (\texttt{default})
Theme.}
\end{figure}

\subsection{Offline}\label{offline-28}

This Screenlet supports offline mode so it can function without a
network connection. For more information on how offline mode works, see
the
\href{/docs/7-0/tutorials/-/knowledge_base/t/architecture-of-offline-mode-in-liferay-screens}{tutorial
on its architecture}. Here are the offline mode policies that you can
use with this Screenlet:

\noindent\hrulefill

Policy \textbar{} What happens \textbar{} When to use \textbar{}
\texttt{remote-only} \textbar{} The Screenlet loads the list from the
Liferay instance. If a connection issue occurs, the Screenlet uses the
delegate to notify the developer about the error. If the Screenlet
successfully loads the list, it stores the data in the local cache for
later use. \textbar{} Use this policy when you always need to show
updated data, and show nothing when there's no connection. \textbar{}
\texttt{cache-only} \textbar{} The Screenlet loads the list from the
local cache. If the list isn't there, the Screenlet uses the delegate to
notify the developer about the error. \textbar{} Use this policy when
you always need to show local data, without retrieving remote
information under any circumstance. \textbar{} \texttt{remote-first}
\textbar{} The Screenlet loads the list from the Liferay instance. If
this succeeds, the Screenlet shows the list to the user and stores it in
the local cache for later use. If a connection issue occurs, the
Screenlet retrieves the list from the local cache. If the list doesn't
exist there, the Screenlet uses the delegate to notify the developer
about the error. \textbar{} Use this policy to show the most recent
version of the data when connected, but show a possibly outdated version
when there's no connection. \textbar{} \texttt{cache-first} \textbar{}
The Screenlet loads the list from the local cache. If the list isn't
there, the Screenlet requests it from the Liferay instance and notifies
the developer about any errors that occur (including connectivity
errors). \textbar{} Use this policy to save bandwidth and loading time
in case you have local (but possibly outdated) data. \textbar{}

\noindent\hrulefill

\subsection{Required Attributes}\label{required-attributes-24}

\begin{itemize}
\tightlist
\item
  \texttt{folderId}
\end{itemize}

\subsection{Attributes}\label{attributes-29}

\noindent\hrulefill

Attribute \textbar{} Data type \textbar{} Explanation \textbar{}
\texttt{groupId} \textbar{} \texttt{number} \textbar{} The ID of the
site (group) where the web content exists. If set to \texttt{0}, the
\texttt{groupId} specified in \texttt{LiferayServerContext} is used. The
default value is \texttt{0}. \textbar{} \texttt{folderId} \textbar{}
\texttt{number} \textbar{} The ID of the web content folder. If set to
\texttt{0}, the root folder is used. The default value is \texttt{0}.
\textbar{} \texttt{offlinePolicy} \textbar{} \texttt{string} \textbar{}
The offline mode setting. The default value is \texttt{remote-first}.
See the
\href{/docs/7-0/reference/-/knowledge_base/r/web-content-list-screenlet-for-ios\#offline}{Offline
section} for details. \textbar{} \texttt{autoLoad} \textbar{}
\texttt{boolean} \textbar{} Whether the list loads automatically when
the Screenlet appears in the app's UI. The default value is
\texttt{true}. \textbar{} \texttt{refreshControl} \textbar{}
\texttt{boolean} \textbar{} Whether a standard
\href{https://developer.apple.com/library/ios/documentation/UIKit/Reference/UIRefreshControl_class/}{iOS
\texttt{UIRefreshControl}} appears when the user does the pull to
refresh gesture. The default value is \texttt{true}. \textbar{}
\texttt{firstPageSize} \textbar{} \texttt{number} \textbar{} The number
of items to display on the first page. The default value is \texttt{50}.
\textbar{} \texttt{pageSize} \textbar{} \texttt{number} \textbar{} The
number of items to display on the second and subsequent pages. The
default value is \texttt{25}. \textbar{} \texttt{obcClassName}
\textbar{} \texttt{string} \textbar{} The name of the
\texttt{OrderByComparator} class to use to sort the results. Omit this
property if you don't want to sort the results.
\href{https://github.com/liferay/liferay-portal/tree/master/modules/apps/web-experience/journal/journal-api/src/main/java/com/liferay/journal/util/comparator}{Click
here} to see some comparator classes. Note, however, that not all of
these classes can be used with \texttt{obcClassName}. You can only use
comparator classes that extend
\texttt{OrderByComparator\textless{}JournalArticle\textgreater{}}. You
can also create your own comparator classes that extend
\texttt{OrderByComparator\textless{}JournalArticle\textgreater{}}.
\textbar{}

\noindent\hrulefill

\subsection{Methods}\label{methods-17}

\noindent\hrulefill

Method \textbar{} Return \textbar{} Explanation \textbar{}
\texttt{loadList()} \textbar{} \texttt{boolean} \textbar{} Starts the
request to load the web content list. This list is shown when the
response is received. Returns \texttt{true} if the request is sent
successfully. \textbar{}

\noindent\hrulefill

\subsection{Delegate}\label{delegate-8}

Web Content List Screenlet delegates some events to an object that
conforms to the \texttt{WebContentListScreenletDelegate} protocol. This
protocol lets you implement the following methods:

\begin{itemize}
\item
  \texttt{-\ screenlet:onWebContentListResponse:}: Called when a page of
  contents is received. Note that this method may be called more than
  once: one call for each page received.
\item
  \texttt{-\ screenlet:onWebContentListError:}: Called when an error
  occurs in the process. The \texttt{NSError} object describes the
  error.
\item
  \texttt{-\ screenlet:onWebContentSelected:}: Called when an item in
  the list is selected.
\end{itemize}

\section{Image Gallery Screenlet for
iOS}\label{image-gallery-screenlet-for-ios}

\subsection{Requirements}\label{requirements-30}

\begin{itemize}
\tightlist
\item
  Xcode 9.3 or above
\item
  iOS 11 SDK
\item
  Liferay Portal 6.2 CE/EE, Liferay CE Portal 7.0/7.1, Liferay DXP 7.0+
\item
  Liferay Screens Compatibility app
  (\href{http://www.liferay.com/marketplace/-/mp/application/54365664}{CE}
  or
  \href{http://www.liferay.com/marketplace/-/mp/application/54369726}{EE/DXP}).
  This app is preinstalled in Liferay CE Portal 7.0/7.1 and Liferay DXP
  7.0+.
\end{itemize}

\subsection{Compatibility}\label{compatibility-30}

\begin{itemize}
\tightlist
\item
  iOS 9 and above
\end{itemize}

\subsection{Xamarin Requirements}\label{xamarin-requirements-30}

\begin{itemize}
\tightlist
\item
  Visual Studio 7.2
\item
  Mono .NET framework 5.4.1.6
\end{itemize}

\subsection{Features}\label{features-30}

Image Gallery Screenlet shows a list of images from a Documents and
Media folder in a Liferay instance. You can also use Image Gallery
Screenlet to upload images to and delete images from the same folder.
The Screenlet implements
\href{http://www.iosnomad.com/blog/2014/4/21/fluent-pagination}{fluent
pagination} with configurable page size, and supports i18n in asset
values.

\subsection{JSON Services Used}\label{json-services-used-29}

Screenlets in Liferay Screens call JSON web services in the portal. This
Screenlet calls the following services and methods.

\noindent\hrulefill

\begin{longtable}[]{@{}lll@{}}
\toprule\noalign{}
Service & Method & Notes \\
\midrule\noalign{}
\endhead
\bottomrule\noalign{}
\endlastfoot
\texttt{DLAppService} & \texttt{getFileEntries} & Load \\
\texttt{DLAppService} & \texttt{getFileEntriesCount} & \\
\texttt{DLAppService} & \texttt{addFileEntry} & Upload \\
\texttt{DLAppService} & \texttt{deleteFileEntry} & Delete \\
\end{longtable}

\noindent\hrulefill

\subsection{Module}\label{module-30}

\begin{itemize}
\tightlist
\item
  None
\end{itemize}

\subsection{Themes}\label{themes-9}

The default Theme uses a standard iOS \texttt{UICollectionView} to show
the scrollable list as a grid. Other Themes may use a different
component, such as \texttt{UITableView} or others, to show the contents.

This screenlet has three different Themes:

\begin{enumerate}
\def\labelenumi{\arabic{enumi}.}
\tightlist
\item
  Grid (default)
\item
  Slideshow
\item
  List
\end{enumerate}

\begin{figure}
\centering
\includegraphics{./images/screens-ios-imagegallery.png}
\caption{Image Gallery Screenlet using the Grid, Slideshow, and List
Themes.}
\end{figure}

\subsection{Offline}\label{offline-29}

This Screenlet supports offline mode so it can function without a
network connection when loading or uploading images (deleting images
while offline is unsupported). For more information on how offline mode
works, see the
\href{/docs/7-0/tutorials/-/knowledge_base/t/architecture-of-offline-mode-in-liferay-screens}{tutorial
on its architecture}. This Screenlet supports the \texttt{remote-only},
\texttt{cache-only}, \texttt{remote-first}, and \texttt{cache-first}
offline mode policies.

These policies take the following actions when loading images from a
Liferay instance:

\noindent\hrulefill

Policy \textbar{} What happens \textbar{} When to use \textbar{}
\texttt{remote-only} \textbar{} The Screenlet loads the data from the
Liferay instance. If a connection issue occurs, the Screenlet uses the
delegate to notify the developer about the error. If the Screenlet
successfully loads the data, it stores it in the local cache for later
use. \textbar{} Use this policy when you always need to show updated
data, and show nothing when there's no connection. \textbar{}
\texttt{cache-only} \textbar{} The Screenlet loads the data from the
local cache. If the data isn't there, the Screenlet uses the delegate to
notify the developer about the error. \textbar{} Use this policy when
you always need to show local data, without retrieving remote
information under any circumstance. \textbar{} \texttt{remote-first}
\textbar{} The Screenlet loads the data from the Liferay instance. If
this succeeds, the Screenlet shows the data to the user and stores it in
the local cache for later use. If a connection issue occurs, the
Screenlet retrieves the data from the local cache. If the data doesn't
exist there, the Screenlet uses the delegate to notify the developer
about the error. \textbar{} Use this policy to show the most recent
version of the data when connected, but show a possibly outdated version
when there's no connection. \textbar{} \texttt{cache-first} \textbar{}
The Screenlet loads the data from the local cache. If the data isn't
there, the Screenlet requests it from the Liferay instance and notifies
the developer about any errors that occur (including connectivity
errors). \textbar{} Use this policy to save bandwidth and loading time
in case you have local (but possibly outdated) data. \textbar{}

\noindent\hrulefill

These policies take the following actions when uploading an image to a
Liferay instance:

\noindent\hrulefill

Policy \textbar{} What happens \textbar{} When to use \textbar{}
\texttt{remote-only} \textbar{} The Screenlet sends the image to the
Liferay instance. If a connection issue occurs, the Screenlet uses the
delegate to notify the developer about the error, but it also discards
the image. \textbar{} Use this policy to make sure the Liferay instance
always has the most recent version of the image. \textbar{}
\texttt{cache-only} \textbar{} The Screenlet stores the image in the
local cache. \textbar{} Use this policy when you need to save the image
locally, but don't want to update it in the Liferay instance. \textbar{}
\texttt{remote-first} \textbar{} The Screenlet sends the image to the
Liferay instance. If this succeeds, it also stores the image in the
local cache for later use. If a connection issue occurs, the Screenlet
stores the image in the local cache and sends it to the Liferay instance
when the connection is re-established. \textbar{} Use this policy when
you need to make sure the Screenlet sends the image to the Liferay
instance as soon as the connection is restored. \textbar{}
\texttt{cache-first} \textbar{} The Screenlet stores the image in the
local cache and then attempts to send it to the Liferay instance. If a
connection issue occurs, the Screenlet sends the image to the Liferay
instance when the connection is re-established. \textbar{} Use this
policy when you need to make sure the Screenlet sends the image to the
Liferay instance as soon as the connection is restored. Compared to
\texttt{remote-first}, this policy always stores the image in the cache.
The \texttt{remote-first} policy only stores the image in the event of a
network error. \textbar{}

\noindent\hrulefill

\subsection{Required Attributes}\label{required-attributes-25}

\begin{itemize}
\tightlist
\item
  \texttt{repositoryId}
\item
  \texttt{folderId}
\end{itemize}

\subsection{Attributes}\label{attributes-30}

\noindent\hrulefill

Attribute \textbar{} Data type \textbar{} Explanation \textbar{}
\texttt{repositoryId} \textbar{} \texttt{number} \textbar{} The ID of
the Liferay instance's Documents and Media repository that contains the
image gallery. If you're using a site's default Documents and Media
repository, then the \texttt{repositoryId} matches the site ID
(\texttt{groupId}). \textbar{} \texttt{folderId} \textbar{}
\texttt{number} \textbar{} The ID of the Documents and Media repository
folder that contains the image gallery. When accessing the folder in
your browser, the \texttt{folderId} is at the end of the URL. \textbar{}
\texttt{mimeTypes} \textbar{} \texttt{string} \textbar{} The
comma-separated list of MIME types for the Screenlet to support.
\textbar{} \texttt{filePrefix} \textbar{} \texttt{string} \textbar{} The
prefix to use on uploaded image file names. \textbar{}
\texttt{offlinePolicy} \textbar{} \texttt{string} \textbar{} The offline
mode setting. The default value is \texttt{remote-first}. See the
\href{/docs/7-0/reference/-/knowledge_base/r/image-gallery-screenlet-for-ios\#offline}{Offline
section} for details. \textbar{} \texttt{autoLoad} \textbar{}
\texttt{boolean} \textbar{} Whether the list automatically loads when
the Screenlet appears in the app's UI. The default value is
\texttt{true}. \textbar{} \texttt{refreshControl} \textbar{}
\texttt{boolean} \textbar{} Whether a standard
\href{https://developer.apple.com/library/ios/documentation/UIKit/Reference/UIRefreshControl_class/}{iOS
\texttt{UIRefreshControl}} appears when the user does the pull to
refresh gesture. The default value is \texttt{true}. \textbar{}
\texttt{firstPageSize} \textbar{} \texttt{number} \textbar{} The number
of items to display on the first page. The default value is \texttt{50}.
\textbar{} \texttt{pageSize} \textbar{} \texttt{number} \textbar{} The
number of items to display on the second and subsequent pages. The
default value is \texttt{25}. \textbar{} \texttt{obcClassName}
\textbar{} \texttt{string} \textbar{} The name of the
\texttt{OrderByComparator} class to use to sort the results. Omit this
property if you don't want to sort the results. Note that you can only
use comparator classes that extend
\texttt{OrderByComparator\textless{}DLFileEntry\textgreater{}}. Liferay
contains no such comparator classes. You must therefore create your own
by extending
\texttt{OrderByComparator\textless{}DLFileEntry\textgreater{}}. To see
examples of some comparator classes that extend other Document Library
classes,
\href{https://github.com/liferay/liferay-portal/tree/master/portal-impl/src/com/liferay/portlet/documentlibrary/util/comparator}{click
here}. \textbar{}

\noindent\hrulefill

\subsection{Methods}\label{methods-18}

\noindent\hrulefill

Method \textbar{} Return \textbar{} Explanation \textbar{}
\texttt{loadList()} \textbar{} \texttt{boolean} \textbar{} Starts the
request to load the list of images. This list is shown when the response
is received. Returns \texttt{true} if the request is sent successfully.
\textbar{}

\noindent\hrulefill

\subsection{Delegate}\label{delegate-9}

Image Gallery Screenlet delegates some events to an object that conforms
to the \texttt{ImageGalleryScreenletDelegate} protocol. This protocol
lets you implement the following methods:

\begin{itemize}
\item
  \texttt{-\ screenlet:onImageEntriesResponse:}: Called when a page of
  contents is received. Note that this method may be called more than
  once: one call for each page received.
\item
  \texttt{-\ screenlet:onImageEntriesError:}: Called when an error
  occurs in the process. The \texttt{NSError} object describes the
  error.
\item
  \texttt{-\ screenlet:onImageEntrySelected:}: Called when an item in
  the list is selected.
\item
  \texttt{-\ screenlet:onImageEntryDeleted:}: Called when an image in
  the list is deleted.
\item
  \texttt{-\ screenlet:onImageEntryDeleteError:}: Called when an error
  occurs during image file deletion. The \texttt{NSError} object
  describes the error.
\item
  \texttt{-\ screenlet:onImageUploadStart:}: Called when an image is
  prepared for upload.
\item
  \texttt{-\ screenlet:onImageUploadProgress:}: Called when the image
  upload progress changes.
\item
  \texttt{-\ screenlet:onImageUploadError:}: Called when an error occurs
  in the image upload process. The \texttt{NSError} object describes the
  error.
\item
  \texttt{-\ screenlet:onImageUploaded:}: Called when the image upload
  finishes.
\item
  \texttt{-\ screenlet:onImageUploadDetailViewCreated:}: Called when the
  image upload View is instantiated. By default, the Screenlet uses a
  modal view controller to present this View. You only need to implement
  this method if you want to override this behavior. This method should
  present the View, passed as parameter, and then return \texttt{true}.
  For example, the following example implementation presents
  \texttt{ImageUploadDetailViewBase} as a parameter, and then uses it to
  customize the View's appearance:

\begin{verbatim}
  func screenlet(screenlet: ImageGalleryScreenlet, 
      onImageUploadDetailViewCreated uploadView: ImageUploadDetailViewBase) -> Bool {
          self.cardDeck?.cards[safe: 0]?.addPage(uploadView)
          self.cardDeck?.cards[safe: 0]?.moveRight()
          return true
  }
\end{verbatim}
\end{itemize}

\section{Rating Screenlet for iOS}\label{rating-screenlet-for-ios}

\subsection{Requirements}\label{requirements-31}

\begin{itemize}
\tightlist
\item
  Xcode 9.3 or above
\item
  iOS 11 SDK
\item
  Liferay Portal 6.2 CE/EE, Liferay CE Portal 7.0/7.1, Liferay DXP 7.0+
\item
  Liferay Screens Compatibility app
  (\href{http://www.liferay.com/marketplace/-/mp/application/54365664}{CE}
  or
  \href{http://www.liferay.com/marketplace/-/mp/application/54369726}{EE/DXP}).
  This app is preinstalled in Liferay CE Portal 7.0/7.1 and Liferay DXP
  7.0+.
\end{itemize}

\subsection{Compatibility}\label{compatibility-31}

\begin{itemize}
\tightlist
\item
  iOS 9 and above
\end{itemize}

\subsection{Xamarin Requirements}\label{xamarin-requirements-31}

\begin{itemize}
\tightlist
\item
  Visual Studio 7.2
\item
  Mono .NET framework 5.4.1.6
\end{itemize}

\subsection{Features}\label{features-31}

Rating Screenlet shows an asset's rating. It also lets users update or
delete the rating. This Screenlet comes with different Themes that
display ratings as thumbs, stars, and emojis.

\subsection{JSON Services Used}\label{json-services-used-30}

Screenlets in Liferay Screens call JSON web services in the portal. This
Screenlet calls the following services and methods.

\noindent\hrulefill

\begin{longtable}[]{@{}
  >{\raggedright\arraybackslash}p{(\columnwidth - 4\tabcolsep) * \real{0.3889}}
  >{\raggedright\arraybackslash}p{(\columnwidth - 4\tabcolsep) * \real{0.3333}}
  >{\raggedright\arraybackslash}p{(\columnwidth - 4\tabcolsep) * \real{0.2778}}@{}}
\toprule\noalign{}
\begin{minipage}[b]{\linewidth}\raggedright
Service
\end{minipage} & \begin{minipage}[b]{\linewidth}\raggedright
Method
\end{minipage} & \begin{minipage}[b]{\linewidth}\raggedright
Notes
\end{minipage} \\
\midrule\noalign{}
\endhead
\bottomrule\noalign{}
\endlastfoot
\texttt{ScreensratingsentryService} (Screens compatibility plugin) &
\texttt{getRatingsEntries} & With \texttt{entryId} \\
\texttt{ScreensratingsentryService} (Screens compatibility plugin) &
\texttt{getRatingsEntries} & With \texttt{classPK} and
\texttt{className} \\
\texttt{ScreensratingsentryService} (Screens compatibility plugin) &
\texttt{updateRatingsEntry} & \\
\texttt{ScreensratingsentryService} (Screens compatibility plugin) &
\texttt{deleteRatingsEntry} & \\
\end{longtable}

\noindent\hrulefill

\subsection{Module}\label{module-31}

\begin{itemize}
\tightlist
\item
  None
\end{itemize}

\subsection{Themes}\label{themes-10}

The default Theme uses \href{https://github.com/marketplacer/Cosmos}{the
\texttt{CosmosView} library} to show an asset's rating. Other custom
Themes may use a different component, such as \texttt{UIButton} or
others, to show the items.

This screenlet has four different Themes:

\begin{enumerate}
\def\labelenumi{\arabic{enumi}.}
\tightlist
\item
  Like
\item
  Thumbs (default)
\item
  Stars
\item
  Emojis
\end{enumerate}

\begin{figure}
\centering
\includegraphics{./images/screens-ios-ratings.png}
\caption{Rating Screenlet's different Themes.}
\end{figure}

\subsection{Offline}\label{offline-30}

This Screenlet supports offline mode so it can function without a
network connection. For more information on how offline mode works, see
the
\href{/docs/7-0/tutorials/-/knowledge_base/t/architecture-of-offline-mode-in-liferay-screens}{tutorial
on its architecture}. Here are the offline mode policies that you can
use with this Screenlet:

\noindent\hrulefill

Policy \textbar{} What happens \textbar{} When to use \textbar{}
\texttt{remote-only} \textbar{} The Screenlet loads the data from the
Liferay instance. If a connection issue occurs, the Screenlet uses the
delegate to notify the developer about the error. If the Screenlet
successfully loads the data, it stores it in the local cache for later
use. \textbar{} Use this policy when you always need to show updated
data, and show nothing when there's no connection. \textbar{}
\texttt{cache-only} \textbar{} The Screenlet loads the data from the
local cache. If the data isn't there, the Screenlet uses the delegate to
notify the developer about the error. \textbar{} Use this policy when
you always need to show local data, without retrieving remote
information under any circumstance. \textbar{} \texttt{remote-first}
\textbar{} The Screenlet loads the data from the Liferay instance. If
this succeeds, the Screenlet shows the data to the user and stores it in
the local cache for later use. If a connection issue occurs, the
Screenlet retrieves the data from the local cache. If the data doesn't
exist there, the Screenlet uses the delegate to notify the developer
about the error. \textbar{} Use this policy to show the most recent
version of the data when connected, but show a possibly outdated version
when there's no connection. \textbar{} \texttt{cache-first} \textbar{}
The Screenlet loads the data from the local cache. If the data isn't
there, the Screenlet requests it from the Liferay instance and notifies
the developer about any errors that occur (including connectivity
errors). \textbar{} Use this policy to save bandwidth and loading time
in case you have local (but possibly outdated) data. \textbar{}

\noindent\hrulefill

\subsection{Required Attributes}\label{required-attributes-26}

\begin{itemize}
\tightlist
\item
  \texttt{entryId}
\end{itemize}

If you don't use \texttt{entryId}, you must use these attributes:

\begin{itemize}
\tightlist
\item
  \texttt{className}
\item
  \texttt{classPK}
\end{itemize}

\subsection{Attributes}\label{attributes-31}

\noindent\hrulefill

Attribute \textbar{} Data type \textbar{} Explanation \textbar{}
\texttt{layoutId} \textbar{} \texttt{@layout} \textbar{} The ID of the
layout to use to show the Theme. \textbar{} \texttt{autoLoad} \textbar{}
\texttt{boolean} \textbar{} Whether the rating loads automatically when
the Screenlet appears in the app's UI. The default value is
\texttt{true}. \textbar{} \texttt{editable} \textbar{} \texttt{boolean}
\textbar{} Whether the user can change the rating. \textbar{}
\texttt{entryId} \textbar{} \texttt{number} \textbar{} The primary key
of the asset with the rating to display. \textbar{} \texttt{className}
\textbar{} \texttt{string} \textbar{} The asset's fully qualified class
name. For example, a blog entry's \texttt{className} is
\href{@platform-ref@/7.0-latest/javadocs/portal-kernel/com/liferay/blogs/kernel/model/BlogsEntry.html}{\texttt{com.liferay.blogs.kernel.model.BlogsEntry}}.
The \texttt{className} attribute is required when using it with
\texttt{classPK} to instantiate the Screenlet.. \textbar{}
\texttt{classPK} \textbar{} \texttt{number} \textbar{} The asset's
unique identifier. Only use this attribute when also using
\texttt{className} to instantiate the Screenlet. \textbar{}
\texttt{groupId} \textbar{} \texttt{number} \textbar{} The ID of the
site (group) containing the asset. \textbar{} \texttt{offlinePolicy}
\textbar{} \texttt{string} \textbar{} The offline mode setting. See the
\href{/docs/7-0/reference/-/knowledge_base/r/rating-screenlet-for-ios\#offline}{Offline
section} for details. \textbar{}

\noindent\hrulefill

\subsection{Methods}\label{methods-19}

\noindent\hrulefill

Method \textbar{} Return \textbar{} Explanation \textbar{}
\texttt{loadRatings()} \textbar{} \texttt{boolean} \textbar{} Starts the
request to load the asset's ratings. \textbar{}

\noindent\hrulefill

\subsection{Delegate}\label{delegate-10}

Rating Screenlet delegates some events to an object that conforms to the
\texttt{RatingScreenletDelegate} protocol. This protocol lets you
implement the following methods:

\begin{itemize}
\item
  \texttt{-\ screenlet:onRatingRetrieve:}: Called when the ratings are
  received.
\item
  \texttt{-\ screenlet:onRatingDeleted:}: Called when a rating is
  deleted.
\item
  \texttt{-\ screenlet:onRatingUpdated:}: Called when a rating is
  updated.
\item
  \texttt{-\ screenlet:onRatingError:}: Called when an error occurs in
  the process. The \texttt{NSError} object describes the error.
\end{itemize}

\section{Comment List Screenlet for
iOS}\label{comment-list-screenlet-for-ios}

\subsection{Requirements}\label{requirements-32}

\begin{itemize}
\tightlist
\item
  Xcode 9.3 or above
\item
  iOS 11 SDK
\item
  Liferay Portal 6.2 CE/EE, Liferay CE Portal 7.0/7.1, Liferay DXP 7.0+
\item
  Liferay Screens Compatibility app
  (\href{http://www.liferay.com/marketplace/-/mp/application/54365664}{CE}
  or
  \href{http://www.liferay.com/marketplace/-/mp/application/54369726}{EE/DXP}).
  This app is preinstalled in Liferay CE Portal 7.0/7.1 and Liferay DXP
  7.0+.
\end{itemize}

\subsection{Compatibility}\label{compatibility-32}

\begin{itemize}
\tightlist
\item
  iOS 9 and above
\end{itemize}

\subsection{Xamarin Requirements}\label{xamarin-requirements-32}

\begin{itemize}
\tightlist
\item
  Visual Studio 7.2
\item
  Mono .NET framework 5.4.1.6
\end{itemize}

\subsection{Features}\label{features-32}

Comment List Screenlet can list all the comments of an asset in a
Liferay instance. It also lets the user update or delete comments.

\subsection{JSON Services Used}\label{json-services-used-31}

Screenlets in Liferay Screens call JSON web services in the portal. This
Screenlet calls the following services and methods.

\noindent\hrulefill

\begin{longtable}[]{@{}
  >{\raggedright\arraybackslash}p{(\columnwidth - 4\tabcolsep) * \real{0.3889}}
  >{\raggedright\arraybackslash}p{(\columnwidth - 4\tabcolsep) * \real{0.3333}}
  >{\raggedright\arraybackslash}p{(\columnwidth - 4\tabcolsep) * \real{0.2778}}@{}}
\toprule\noalign{}
\begin{minipage}[b]{\linewidth}\raggedright
Service
\end{minipage} & \begin{minipage}[b]{\linewidth}\raggedright
Method
\end{minipage} & \begin{minipage}[b]{\linewidth}\raggedright
Notes
\end{minipage} \\
\midrule\noalign{}
\endhead
\bottomrule\noalign{}
\endlastfoot
\texttt{ScreenscommentService} (Screens compatibility plugin) &
\texttt{getCommentsWithClassName} & \\
\texttt{ScreenscommentService} (Screens compatibility plugin) &
\texttt{getCommentsCount} & \\
\end{longtable}

\noindent\hrulefill

\subsection{Module}\label{module-32}

\begin{itemize}
\tightlist
\item
  None
\end{itemize}

\subsection{Themes}\label{themes-11}

\begin{itemize}
\tightlist
\item
  Default
\end{itemize}

The Default Theme uses an
\href{https://developer.apple.com/reference/uikit/uitableview}{iOS
\texttt{UITableView}} to show an asset's comments. Other Themes may use
a different component, such as
\href{https://developer.apple.com/reference/uikit/uicollectionview}{iOS's
\texttt{UICollectionView}} or others, to show the items.

\begin{figure}
\centering
\includegraphics{./images/screens-ios-commentlist.png}
\caption{Comment List Screenlet using the Default Theme.}
\end{figure}

\subsection{Offline}\label{offline-31}

This Screenlet supports offline mode so it can function without a
network connection. For more information on how offline mode works, see
the
\href{/docs/7-0/tutorials/-/knowledge_base/t/architecture-of-offline-mode-in-liferay-screens}{tutorial
on its architecture}. Here are the offline mode policies that you can
use with this Screenlet:

\noindent\hrulefill

Policy \textbar{} What happens \textbar{} When to use \textbar{}
\texttt{remote-only} \textbar{} The Screenlet loads the list from the
Liferay instance. If a connection issue occurs, the Screenlet uses the
delegate to notify the developer about the error. If the Screenlet
successfully loads the list, it stores the data in the local cache for
later use. \textbar{} Use this policy when you always need to show
updated data, and show nothing when there's no connection. \textbar{}
\texttt{cache-only} \textbar{} The Screenlet loads the list from the
local cache. If the list isn't there, the Screenlet uses the delegate to
notify the developer about the error. \textbar{} Use this policy when
you always need to show local data, without retrieving remote
information under any circumstance. \textbar{} \texttt{remote-first}
\textbar{} The Screenlet loads the list from the Liferay instance. If
this succeeds, the Screenlet shows the list to the user and stores it in
the local cache for later use. If a connection issue occurs, the
Screenlet retrieves the list from the local cache. If the list doesn't
exist there, the Screenlet uses the delegate to notify the developer
about the error. \textbar{} Use this policy to show the most recent
version of the data when connected, but show a possibly outdated version
when there's no connection. \textbar{} \texttt{cache-first} \textbar{}
The Screenlet loads the list from the local cache. If the list isn't
there, the Screenlet requests it from the Liferay instance and notifies
the developer about any errors that occur (including connectivity
errors). \textbar{} Use this policy to save bandwidth and loading time
in case you have local (but possibly outdated) data. \textbar{}

\noindent\hrulefill

\subsection{Required Attributes}\label{required-attributes-27}

\begin{itemize}
\tightlist
\item
  \texttt{className}
\item
  \texttt{classPK}
\end{itemize}

\subsection{Attributes}\label{attributes-32}

\noindent\hrulefill

Attribute \textbar{} Data type \textbar{} Explanation \textbar{}
\texttt{className} \textbar{} \texttt{string} \textbar{} The asset's
fully qualified class name. For example, a blog entry's
\texttt{className} is
\href{@platform-ref@/7.0-latest/javadocs/portal-kernel/com/liferay/blogs/kernel/model/BlogsEntry.html}{\texttt{com.liferay.blogs.kernel.model.BlogsEntry}}.
The \texttt{className} and \texttt{classPK} attributes are required to
instantiate the Screenlet. \textbar{} \texttt{classPK} \textbar{}
\texttt{number} \textbar{} The asset's unique identifier. The
\texttt{className} and \texttt{classPK} attributes are required to
instantiate the Screenlet. \textbar{} \texttt{offlinePolicy} \textbar{}
\texttt{string} \textbar{} The offline mode setting. The default is
\texttt{remote-first}. See
\href{/docs/7-0/reference/-/knowledge_base/r/comment-list-screenlet-for-ios\#offline}{the
Offline section} for details. \textbar{} \texttt{editable} \textbar{}
\texttt{boolean} \textbar{} Whether the user can edit the comment.
\textbar{} \texttt{autoLoad} \textbar{} \texttt{boolean} \textbar{}
Whether the list should automatically load when the Screenlet appears in
the app's UI. The default value is \texttt{true}. \textbar{}
\texttt{refreshControl} \textbar{} \texttt{boolean} \textbar{} Defines
whether a standard
\href{https://developer.apple.com/library/ios/documentation/UIKit/Reference/UIRefreshControl_class/}{iOS
\texttt{UIRefreshControl}} is shown when the user does the pull to
refresh gesture. The default value is \texttt{true}. \textbar{}
\texttt{firstPageSize} \textbar{} \texttt{number} \textbar{} The number
of items retrieved from the server for display on the first page. The
default value is \texttt{50}. \textbar{} \texttt{pageSize} \textbar{}
\texttt{number} \textbar{} The number of items retrieved from the server
for display on the second and subsequent pages. The default value is
\texttt{25}. \textbar{} \texttt{obcClassName} \textbar{} \texttt{string}
\textbar{} The name of the
\href{@platform-ref@/7.0-latest/javadocs/portal-kernel/com/liferay/portal/kernel/util/OrderByComparator.html}{\texttt{OrderByComparator}
class} to use to sort the results. You can only use classes that extend
\texttt{OrderByComparator\textless{}MBMessage\textgreater{}}. If you
don't want to sort the results, you can omit this property. \textbar{}

\noindent\hrulefill

\subsection{Methods}\label{methods-20}

\noindent\hrulefill

Method \textbar{} Return \textbar{} Explanation \textbar{}
\texttt{loadList()} \textbar{} \texttt{boolean} \textbar{} Starts the
request to load the list. This list is shown when the response is
received. Returns \texttt{true} if the request is sent. \textbar{}

\noindent\hrulefill

\subsection{Delegate}\label{delegate-11}

Comment List Screenlet delegates some events to an object that conforms
to the \texttt{ComentListScreenletDelegate} protocol. This protocol lets
you implement the following methods:

\begin{itemize}
\item
  \texttt{-\ screenlet:onListResponseComments:}: Called when the
  Screenlet receives the comments.
\item
  \texttt{-\ screenlet:onCommentListError:}: Called when an error occurs
  in the process. The \texttt{NSError} object describes the error.
\item
  \texttt{-\ screenlet:onSelectedComment:}: Called when a comment is
  selected.
\item
  \texttt{-\ screenlet:onDeletedComment:}: Called when a comment is
  deleted.
\item
  \texttt{-\ screenlet:onCommentDelete:}: Called when the Screenlet
  prepares a comment for deletion.
\item
  \texttt{-\ screenlet:onUpdatedComment:}: Called when a comment is
  updated.
\item
  \texttt{-\ screenlet:onCommentUpdate:}: Called when the Screenlet
  prepares a comment for update.
\end{itemize}

\section{Comment Display Screenlet for
iOS}\label{comment-display-screenlet-for-ios}

\subsection{Requirements}\label{requirements-33}

\begin{itemize}
\tightlist
\item
  Xcode 9.3 or above
\item
  iOS 11 SDK
\item
  Liferay Portal 6.2 CE/EE, Liferay CE Portal 7.0/7.1, Liferay DXP 7.0+
\item
  Liferay Screens Compatibility app
  (\href{http://www.liferay.com/marketplace/-/mp/application/54365664}{CE}
  or
  \href{http://www.liferay.com/marketplace/-/mp/application/54369726}{EE/DXP}).
  This app is preinstalled in Liferay CE Portal 7.0/7.1 and Liferay DXP
  7.0+.
\end{itemize}

\subsection{Compatibility}\label{compatibility-33}

\begin{itemize}
\tightlist
\item
  iOS 9 and above
\end{itemize}

\subsection{Xamarin Requirements}\label{xamarin-requirements-33}

\begin{itemize}
\tightlist
\item
  Visual Studio 7.2
\item
  Mono .NET framework 5.4.1.6
\end{itemize}

\subsection{Features}\label{features-33}

Comment Display Screenlet can show one comment of an asset in a Liferay
instance. It also lets the user update or delete the comment.

\subsection{JSON Services Used}\label{json-services-used-32}

Screenlets in Liferay Screens call JSON web services in the portal. This
Screenlet calls the following services and methods.

\noindent\hrulefill

\begin{longtable}[]{@{}
  >{\raggedright\arraybackslash}p{(\columnwidth - 4\tabcolsep) * \real{0.3889}}
  >{\raggedright\arraybackslash}p{(\columnwidth - 4\tabcolsep) * \real{0.3333}}
  >{\raggedright\arraybackslash}p{(\columnwidth - 4\tabcolsep) * \real{0.2778}}@{}}
\toprule\noalign{}
\begin{minipage}[b]{\linewidth}\raggedright
Service
\end{minipage} & \begin{minipage}[b]{\linewidth}\raggedright
Method
\end{minipage} & \begin{minipage}[b]{\linewidth}\raggedright
Notes
\end{minipage} \\
\midrule\noalign{}
\endhead
\bottomrule\noalign{}
\endlastfoot
\texttt{ScreenscommentService} (Screens compatibility plugin) &
\texttt{getCommentWithCommentId} & \\
\texttt{ScreenscommentService} (Screens compatibility plugin) &
\texttt{updateComment} & \\
\texttt{CommentmanagerjsonwsService} & \texttt{deleteComment} & \\
\end{longtable}

\noindent\hrulefill

\subsection{Module}\label{module-33}

\begin{itemize}
\tightlist
\item
  None
\end{itemize}

\subsection{Themes}\label{themes-12}

\begin{itemize}
\tightlist
\item
  Default
\end{itemize}

The Default Theme uses
\href{/docs/7-0/reference/-/knowledge_base/r/userportraitscreenlet-for-ios}{User
Portrait Screenlet} and iOS \texttt{UILabel} elements to show an asset's
comment. Other Themes may use different components to show the comment.

\begin{figure}
\centering
\includegraphics{./images/screens-ios-commentdisplay.png}
\caption{Comment Display Screenlet using the Default Theme.}
\end{figure}

\subsection{Offline}\label{offline-32}

This Screenlet supports offline mode so it can function without a
network connection. For more information on how offline mode works, see
the
\href{/docs/7-0/tutorials/-/knowledge_base/t/architecture-of-offline-mode-in-liferay-screens}{tutorial
on its architecture}. This Screenlet supports the \texttt{remote-only},
\texttt{cache-only}, \texttt{remote-first}, and \texttt{cache-first}
offline mode policies.

These policies take the following actions when loading a comment from a
Liferay instance:

\noindent\hrulefill

Policy \textbar{} What happens \textbar{} When to use \textbar{}
\texttt{remote-only} \textbar{} The Screenlet loads the data from the
Liferay instance. If a connection issue occurs, the Screenlet uses the
listener to notify the developer about the error. If the Screenlet
successfully loads the data, it stores it in the local cache for later
use. \textbar{} Use this policy when you always need to show updated
data, and show nothing when there's no connection. \textbar{}
\texttt{cache-only} \textbar{} The Screenlet loads the data from the
local cache. If the data isn't there, the Screenlet uses the listener to
notify the developer about the error. \textbar{} Use this policy when
you always need to show local data, without retrieving remote
information under any circumstance. \textbar{} \texttt{remote-first}
\textbar{} The Screenlet loads the data from the Liferay instance. If
this succeeds, the Screenlet shows the data to the user and stores it in
the local cache for later use. If a connection issue occurs, the
Screenlet retrieves the data from the local cache. If the data doesn't
exist there, the Screenlet uses the listener to notify the developer
about the error. \textbar{} Use this policy to show the most recent
version of the data when connected, but show an outdated version when
there's no connection. \textbar{} \texttt{cache-first} \textbar{} The
Screenlet loads the data from the local cache. If the data isn't there,
the Screenlet requests it from the Liferay instance and notifies the
developer about any errors that occur (including connectivity errors).
\textbar{} Use this policy to save bandwidth and loading time in case
you have local (but probably outdated) data. \textbar{}

\noindent\hrulefill

These policies take the following actions when updating or deleting a
comment in a Liferay instance:

\noindent\hrulefill

Policy \textbar{} What happens \textbar{} When to use \textbar{}
\texttt{remote-only} \textbar{} The Screenlet sends the data to the
Liferay instance. If a connection issue occurs, the Screenlet uses the
delegate to notify the developer about the error, but it also discards
the data. \textbar{} Use this policy to make sure the Liferay instance
always has the most recent version of the data. \textbar{}
\texttt{cache-only} \textbar{} The Screenlet stores the data in the
local cache. \textbar{} Use this policy when you need to save the data
locally, but don't want to update it in the Liferay instance. \textbar{}
\texttt{remote-first} \textbar{} The Screenlet sends the data to the
Liferay instance. If this succeeds, it also stores the data in the local
cache for later use. If a connection issue occurs, the Screenlet stores
the data in the local cache and sends it to the Liferay instance when
the connection is re-established. \textbar{} Use this policy when you
need to make sure the Screenlet sends the data to the Liferay instance
as soon as the connection is restored. \textbar{} \texttt{cache-first}
\textbar{} The Screenlet stores the data in the local cache and then
attempts to send it to the Liferay instance. If a connection issue
occurs, the Screenlet sends the data to the Liferay instance when the
connection is re-established. \textbar{} Use this policy when you need
to make sure the Screenlet sends the data to the Liferay instance as
soon as the connection is restored. Compared to \texttt{remote-first},
this policy always stores the data in the cache. The
\texttt{remote-first} policy only stores the data in the event of a
network error. \textbar{}

\noindent\hrulefill

\subsection{Required Attributes}\label{required-attributes-28}

\begin{itemize}
\tightlist
\item
  \texttt{commentId}
\end{itemize}

\subsection{Attributes}\label{attributes-33}

\noindent\hrulefill

Attribute \textbar{} Data type \textbar{} Explanation \textbar{}
\texttt{commentId} \textbar{} \texttt{number} \textbar{} The primary key
of the comment to display. \textbar{} \texttt{autoLoad} \textbar{}
\texttt{boolean} \textbar{} Whether the list should automatically load
when the Screenlet appears in the app's UI. The default value is
\texttt{true}. \textbar{} \texttt{editable} \textbar{} \texttt{boolean}
\textbar{} Whether the user can edit the comment. \textbar{}
\texttt{offlinePolicy} \textbar{} \texttt{string} \textbar{} The offline
mode setting. The default is \texttt{remote-first}. See
\href{/docs/7-0/reference/-/knowledge_base/r/comment-display-screenlet-for-ios\#offline}{the
Offline section} for details. \textbar{}

\noindent\hrulefill

\subsection{Methods}\label{methods-21}

\noindent\hrulefill

Method \textbar{} Return \textbar{} Explanation \textbar{}
\texttt{load()} \textbar{} none \textbar{} Starts the request to load
the comment. \textbar{}

\noindent\hrulefill

\subsection{Delegate}\label{delegate-12}

Comment Display Screenlet delegates some events to an object that
conforms to the \texttt{CommentDisplayScreenletDelegate} protocol. This
protocol lets you implement the following methods:

\begin{itemize}
\item
  \texttt{-\ screenlet:onCommentLoaded:}: Called when the Screenlet
  loads the comment.
\item
  \texttt{-\ screenlet:onLoadCommentError:}: Called when an error occurs
  in the process. The \texttt{NSError} object describes the error.
\item
  \texttt{-\ screenlet:onSelectedComment:}: Called when a comment is
  selected.
\item
  \texttt{-\ screenlet:onDeletedComment:}: Called when a comment is
  deleted.
\item
  \texttt{-\ screenlet:onCommentDelete:}: Called when the Screenlet
  prepares the comment for deletion.
\item
  \texttt{-\ screenlet:onUpdatedComment:}: Called when a comment is
  updated.
\item
  \texttt{-\ screenlet:onCommentUpdate:}: Called when the Screenlet
  prepares the comment for update.
\end{itemize}

\section{Comment Add Screenlet for
iOS}\label{comment-add-screenlet-for-ios}

\subsection{Requirements}\label{requirements-34}

\begin{itemize}
\tightlist
\item
  Xcode 9.3 or above
\item
  iOS 11 SDK
\item
  Liferay Portal 6.2 CE/EE, Liferay CE Portal 7.0/7.1, Liferay DXP 7.0+
\item
  Liferay Screens Compatibility app
  (\href{http://www.liferay.com/marketplace/-/mp/application/54365664}{CE}
  or
  \href{http://www.liferay.com/marketplace/-/mp/application/54369726}{EE/DXP}).
  This app is preinstalled in Liferay CE Portal 7.0/7.1 and Liferay DXP
  7.0+.
\end{itemize}

\subsection{Compatibility}\label{compatibility-34}

\begin{itemize}
\tightlist
\item
  iOS 9 and above
\end{itemize}

\subsection{Xamarin Requirements}\label{xamarin-requirements-34}

\begin{itemize}
\tightlist
\item
  Visual Studio 7.2
\item
  Mono .NET framework 5.4.1.6
\end{itemize}

\subsection{Features}\label{features-34}

Comment Add Screenlet can add a comment to an asset in a Liferay
instance.

\subsection{JSON Services Used}\label{json-services-used-33}

Screenlets in Liferay Screens call JSON web services in the portal. This
Screenlet calls the following services and methods.

\noindent\hrulefill

\begin{longtable}[]{@{}
  >{\raggedright\arraybackslash}p{(\columnwidth - 4\tabcolsep) * \real{0.3889}}
  >{\raggedright\arraybackslash}p{(\columnwidth - 4\tabcolsep) * \real{0.3333}}
  >{\raggedright\arraybackslash}p{(\columnwidth - 4\tabcolsep) * \real{0.2778}}@{}}
\toprule\noalign{}
\begin{minipage}[b]{\linewidth}\raggedright
Service
\end{minipage} & \begin{minipage}[b]{\linewidth}\raggedright
Method
\end{minipage} & \begin{minipage}[b]{\linewidth}\raggedright
Notes
\end{minipage} \\
\midrule\noalign{}
\endhead
\bottomrule\noalign{}
\endlastfoot
\texttt{ScreenscommentService} (Screens compatibility plugin) &
\texttt{addComment} & \\
\end{longtable}

\noindent\hrulefill

\subsection{Module}\label{module-34}

\begin{itemize}
\tightlist
\item
  None
\end{itemize}

\subsection{Themes}\label{themes-13}

\begin{itemize}
\tightlist
\item
  Default
\end{itemize}

The Default Theme uses the iOS elements \texttt{UITextField} and
\texttt{UIButton} to add a comment to an asset. Other Themes may use
other components to show the comment.

\begin{figure}
\centering
\includegraphics{./images/screens-ios-commentadd.png}
\caption{Comment Add Screenlet using the Default Theme.}
\end{figure}

\subsection{Offline}\label{offline-33}

This Screenlet supports offline mode so it can function without a
network connection. For more information on how offline mode works, see
the
\href{/docs/7-0/tutorials/-/knowledge_base/t/architecture-of-offline-mode-in-liferay-screens}{tutorial
on its architecture}. Here are the offline mode policies that you can
use with this Screenlet:

\noindent\hrulefill

Policy \textbar{} What happens \textbar{} When to use \textbar{}
\texttt{remote-only} \textbar{} The Screenlet sends the data to the
Liferay instance. If a connection issue occurs, the Screenlet uses the
listener to notify the developer about the error. If the Screenlet
successfully sends the data, it also stores it in the local cache.
\textbar{} Use this policy when you always need to send updated data,
and send nothing when there's no connection. \textbar{}
\texttt{cache-only} \textbar{} The Screenlet sends the data to the local
cache. If an error occurs, the Screenlet uses the listener to notify the
developer. \textbar{} Use this policy when you always need to store
local data without sending remote information under any circumstance.
\textbar{} \texttt{remote-first} \textbar{} The Screenlet sends the data
to the Liferay instance. If this succeeds, the Screenlet also stores the
data in the local cache. If a connection issue occurs, the Screenlet
stores the data to the local cache and sends it to the Liferay instance
when the connection is restored. If an error occurs, the Screenlet uses
the listener to notify the developer. \textbar{} Use this policy to send
the most recent version of the data when connected, and store the data
for later synchronization when there's no connection. \textbar{}
\texttt{cache-first} \textbar{} The Screenlet sends the data to the
local cache, then sends it to the Liferay instance. If sending the data
to the Liferay instance fails, the Screenlet still stores the data
locally and then notifies the developer about any errors that occur
(including connectivity errors). \textbar{} Use this policy to save
bandwidth and store local (but possibly outdated) data. \textbar{}

\noindent\hrulefill

\subsection{Required Attributes}\label{required-attributes-29}

\begin{itemize}
\tightlist
\item
  \texttt{className}
\item
  \texttt{classPK}
\end{itemize}

\subsection{Attributes}\label{attributes-34}

\noindent\hrulefill

Attribute \textbar{} Data type \textbar{} Explanation \textbar{}
\texttt{className} \textbar{} \texttt{string} \textbar{} The asset's
fully qualified class name. For example, a blog entry's
\texttt{className} is
\href{@platform-ref@/7.0-latest/javadocs/portal-kernel/com/liferay/blogs/kernel/model/BlogsEntry.html}{\texttt{com.liferay.blogs.kernel.model.BlogsEntry}}.
The \texttt{className} and \texttt{classPK} attributes are required to
instantiate the Screenlet. \textbar{} \texttt{classPK} \textbar{}
\texttt{number} \textbar{} The asset's unique identifier. The
\texttt{className} and \texttt{classPK} attributes are required to
instantiate the Screenlet. \textbar{} \texttt{offlinePolicy} \textbar{}
\texttt{string} \textbar{} The offline mode setting. The default value
is \texttt{remote-first}. See
\href{/docs/7-0/reference/-/knowledge_base/r/comment-add-screenlet-for-ios\#offline}{the
Offline section} for details. \textbar{}

\noindent\hrulefill

\subsection{Delegate}\label{delegate-13}

Comment Add Screenlet delegates some events to an object that conforms
to the \texttt{CommentAddScreenletDelegate} protocol. This protocol lets
you implement the following methods:

\begin{itemize}
\item
  \texttt{-\ screenlet:onCommentAdded:}: Called when the Screenlet adds
  a comment.
\item
  \texttt{-\ screenlet:onAddCommentError:}: Called when an error occurs
  while adding a comment. The \texttt{NSError} object describes the
  error.
\item
  \texttt{-\ screenlet:onCommentUpdated:}: Called when the Screenlet
  prepares a comment for update.
\item
  \texttt{-\ screenlet:onUpdateCommentError:}: Called when an error
  occurs while updating a comment. The \texttt{NSError} object describes
  the error.
\end{itemize}

\section{Asset Display Screenlet for
iOS}\label{asset-display-screenlet-for-ios}

\subsection{Requirements}\label{requirements-35}

\begin{itemize}
\tightlist
\item
  Xcode 9.3 or above
\item
  iOS 11 SDK
\item
  Liferay Portal 6.2 CE/EE, Liferay CE Portal 7.0/7.1, Liferay DXP 7.0+
\item
  Liferay Screens Compatibility app
  (\href{http://www.liferay.com/marketplace/-/mp/application/54365664}{CE}
  or
  \href{http://www.liferay.com/marketplace/-/mp/application/54369726}{EE/DXP}).
  This app is preinstalled in Liferay CE Portal 7.0/7.1 and Liferay DXP
  7.0+.
\end{itemize}

\subsection{Compatibility}\label{compatibility-35}

\begin{itemize}
\tightlist
\item
  iOS 9 and above
\end{itemize}

\subsection{Xamarin Requirements}\label{xamarin-requirements-35}

\begin{itemize}
\tightlist
\item
  Visual Studio 7.2
\item
  Mono .NET framework 5.4.1.6
\end{itemize}

\subsection{Features}\label{features-35}

Asset Display Screenlet can display an asset from a Liferay instance.
The Screenlet can currently display Documents and Media files
(\texttt{DLFileEntry} images, videos, audio files, and PDFs), blogs
entries (\texttt{BlogsEntry}) and web content articles
(\texttt{WebContent}).

Asset Display Screenlet can also display your custom asset types. See
\href{/docs/7-0/reference/-/knowledge_base/r/asset-display-screenlet-for-ios\#delegate}{the
delegate section of this document} for details.

\subsection{JSON Services Used}\label{json-services-used-34}

Screenlets in Liferay Screens call JSON web services in the portal. This
Screenlet calls the following services and methods.

\noindent\hrulefill

\begin{longtable}[]{@{}
  >{\raggedright\arraybackslash}p{(\columnwidth - 4\tabcolsep) * \real{0.3889}}
  >{\raggedright\arraybackslash}p{(\columnwidth - 4\tabcolsep) * \real{0.3333}}
  >{\raggedright\arraybackslash}p{(\columnwidth - 4\tabcolsep) * \real{0.2778}}@{}}
\toprule\noalign{}
\begin{minipage}[b]{\linewidth}\raggedright
Service
\end{minipage} & \begin{minipage}[b]{\linewidth}\raggedright
Method
\end{minipage} & \begin{minipage}[b]{\linewidth}\raggedright
Notes
\end{minipage} \\
\midrule\noalign{}
\endhead
\bottomrule\noalign{}
\endlastfoot
\texttt{ScreensassetentryService} (Screens compatibility plugin) &
\texttt{getAssetEntry} & With \texttt{entryId} \\
\texttt{ScreensassetentryService} (Screens compatibility plugin) &
\texttt{getAssetEntry} & With \texttt{classPK} and \texttt{className} \\
\texttt{ScreensassetentryService} (Screens compatibility plugin) &
\texttt{getAssetEntries} & With \texttt{entryQuery} \\
\texttt{ScreensassetentryService} (Screens compatibility plugin) &
\texttt{getAssetEntries} & With \texttt{companyId}, \texttt{groupId},
and \texttt{portletItemName} \\
\end{longtable}

\noindent\hrulefill

\subsection{Module}\label{module-35}

\begin{itemize}
\tightlist
\item
  None
\end{itemize}

\subsection{Themes}\label{themes-14}

\begin{itemize}
\tightlist
\item
  Default
\end{itemize}

The Default Theme uses different UI elements to show each asset type.
For example, it displays images with \texttt{UIImageView}, and blogs
with \texttt{UILabel}.

This Screenlet can also render other Screenlets:

\begin{itemize}
\tightlist
\item
  Images: Image Display Screenlet
\item
  Videos: Video Display Screenlet
\item
  Audio: Audio Display Screenlet
\item
  PDFs: PDF Display Screenlet
\item
  Blog entries: Blogs Entry Display Screenlet
\item
  Web content: Web Content Display Screenlet
\end{itemize}

These Screenlets can also be used alone without Asset Display Screenlet.

\begin{figure}
\centering
\includegraphics{./images/screens-ios-assetdisplay.png}
\caption{Asset Display Screenlet using the Default Theme.}
\end{figure}

\subsection{Offline}\label{offline-34}

This Screenlet supports offline mode so it can function without a
network connection. For more information on how offline mode works, see
the
\href{/docs/7-0/tutorials/-/knowledge_base/t/architecture-of-offline-mode-in-liferay-screens}{tutorial
on its architecture}. Here are the offline mode policies that you can
use with this Screenlet:

\noindent\hrulefill

Policy \textbar{} What happens \textbar{} When to use \textbar{}
\texttt{remote-only} \textbar{} The Screenlet loads the data from the
Liferay instance. If a connection issue occurs, the Screenlet uses the
listener to notify the developer about the error. If the Screenlet
successfully loads the data, it stores it in the local cache for later
use. \textbar{} Use this policy when you always need to show updated
data, and show nothing when there's no connection. \textbar{}
\texttt{cache-only} \textbar{} The Screenlet loads the data from the
local cache. If the data isn't there, the Screenlet uses the listener to
notify the developer about the error. \textbar{} Use this policy when
you always need to show local data, without retrieving remote
information under any circumstance. \textbar{} \texttt{remote-first}
\textbar{} The Screenlet loads the data from the Liferay instance. If
this succeeds, the Screenlet shows the data to the user and stores it in
the local cache for later use. If a connection issue occurs, the
Screenlet retrieves the data from the local cache. If the data doesn't
exist there, the Screenlet uses the listener to notify the developer
about the error. \textbar{} Use this policy to show the most recent
version of the data when connected, but show an outdated version when
there's no connection. \textbar{} \texttt{cache-first} \textbar{} The
Screenlet loads the data from the local cache. If the data isn't there,
the Screenlet requests it from the Liferay instance and notifies the
developer about any errors that occur (including connectivity errors).
\textbar{} Use this policy to save bandwidth and loading time in case
you have local (but probably outdated) data. \textbar{}

\noindent\hrulefill

\subsection{Required Attributes}\label{required-attributes-30}

\begin{itemize}
\tightlist
\item
  \texttt{assetEntryId}
\end{itemize}

Instead of \texttt{assetEntryId}, you can use both of these attributes:

\begin{itemize}
\tightlist
\item
  \texttt{className}
\item
  \texttt{classPK}
\end{itemize}

If you don't use the above attributes, you must use this attribute:

\begin{itemize}
\tightlist
\item
  \texttt{portletItemName}
\end{itemize}

\subsection{Attributes}\label{attributes-35}

\noindent\hrulefill

Attribute \textbar{} Data type \textbar{} Explanation \textbar{}
\texttt{assetEntryId} \textbar{} \texttt{number} \textbar{} The primary
key of the asset. \textbar{} \texttt{className} \textbar{}
\texttt{string} \textbar{} The asset's fully qualified class name. For
example, a blog entry's \texttt{className} is
\href{@platform-ref@/7.0-latest/javadocs/portal-kernel/com/liferay/blogs/kernel/model/BlogsEntry.html}{\texttt{com.liferay.blogs.kernel.model.BlogsEntry}}.
The \texttt{className} and \texttt{classPK} attributes are required to
instantiate the Screenlet. \textbar{} \texttt{classPK} \textbar{}
\texttt{number} \textbar{} The asset's unique identifier. The
\texttt{className} and \texttt{classPK} attributes are required to
instantiate the Screenlet. \textbar{} \texttt{portletItemName}
\textbar{} \texttt{string} \textbar{} The name of the
\href{/docs/7-0/user/-/knowledge_base/u/configuration-templates}{configuration
template} you used in the Asset Publisher. To use this feature, add an
Asset Publisher to one of your site's pages (it may be a hidden page),
configure the Asset Publisher's filter (in \emph{Configuration} →
\emph{Setup} → \emph{Asset Selection}), and then use the Asset
Publisher's \emph{Configuration Templates} option to save this
configuration with a name. Use this name as this attribute's value. If
there is more than one asset in the configuration, the Screenlet
displays only the first one. \textbar{} \texttt{assetEntry} \textbar{}
\texttt{Asset} \textbar{} The \texttt{Asset} object to display, selected
from a list of assets. Note that if you use this attribute, the
Screenlet doesn't need to call the server. \textbar{} \texttt{autoLoad}
\textbar{} \texttt{boolean} \textbar{} Whether the asset automatically
loads when the Screenlet appears in the app's UI. The default value is
\texttt{true}. \textbar{} \texttt{offlinePolicy} \textbar{}
\texttt{string} \textbar{} The offline mode setting. The default value
is \texttt{remote-first}. See
\href{/docs/7-0/reference/-/knowledge_base/r/asset-display-screenlet-for-ios\#offline}{the
Offline section} for details. \textbar{}

\noindent\hrulefill

\subsection{Delegate}\label{delegate-14}

Asset Display Screenlet delegates some events to an object that conforms
to the \texttt{AssetDisplayScreenletDelegate} protocol. This protocol
lets you implement the following methods:

\begin{itemize}
\item
  \texttt{-\ screenlet:onAssetResponse:}: Called when the Screenlet
  receives the asset.
\item
  \texttt{-\ screenlet:onAssetError:}: Called when an error occurs in
  the process. The \texttt{NSError} object describes the error.
\item
  \texttt{-\ screenlet:onConfigureScreenlet:}: Called when the child
  Screenlet (the Screenlet rendered inside Asset Display Screenlet) has
  been successfully initialized. Use this method to configure or
  customize it. The example implementation here sets the child Blogs
  Entry Display Screenlet's background color to gray:

\begin{verbatim}
  func screenlet(screenlet: AssetDisplayScreenlet, onConfigureScreenlet, 
      childScreenlet: BaseScreenlet?, onAsset asset: Asset) {
          if childScreenlet is BlogsEntryDisplayScreenlet {
              childScreenlet?.screenletView?.backgroundColor = UIColor.grayColor()
          }
  }
\end{verbatim}
\item
  \texttt{-\ screenlet:onAsset:}: Called to render a custom asset. The
  following example implementation renders a portal user
  (\texttt{User}). If the asset is a user, this method instantiates a
  custom \texttt{UserProfileView} to render that user:

\begin{verbatim}
  public func screenlet(screenlet: AssetDisplayScreenlet, onAsset asset: Asset) -> UIView? {
      if let type = asset.attributes["object"]?.allKeys.first as? String {
          if type == "user" {
              let view = NSBundle.mainBundle().loadNibNamed("UserProfileView", owner: self, 
                  options: nil)![0] as? UserProfileView

              view?.user = User(attributes: asset.attributes)

              return view
          }
      }
      return nil
  }
\end{verbatim}
\end{itemize}

\section{Blogs Entry Display Screenlet for
iOS}\label{blogs-entry-display-screenlet-for-ios}

\subsection{Requirements}\label{requirements-36}

\begin{itemize}
\tightlist
\item
  Xcode 9.3 or above
\item
  iOS 11 SDK
\item
  Liferay Portal 6.2 CE/EE, Liferay CE Portal 7.0/7.1, Liferay DXP 7.0+
\item
  Liferay Screens Compatibility app
  (\href{http://www.liferay.com/marketplace/-/mp/application/54365664}{CE}
  or
  \href{http://www.liferay.com/marketplace/-/mp/application/54369726}{EE/DXP}).
  This app is preinstalled in Liferay CE Portal 7.0/7.1 and Liferay DXP
  7.0+.
\end{itemize}

\subsection{Compatibility}\label{compatibility-36}

\begin{itemize}
\tightlist
\item
  iOS 9 and above
\end{itemize}

\subsection{Xamarin Requirements}\label{xamarin-requirements-36}

\begin{itemize}
\tightlist
\item
  Visual Studio 7.2
\item
  Mono .NET framework 5.4.1.6
\end{itemize}

\subsection{Features}\label{features-36}

Blogs Entry Display Screenlet displays a single blog entry. Image
Display Screenlet renders any header image the blogs entry may have.

\subsection{JSON Services Used}\label{json-services-used-35}

Screenlets in Liferay Screens call JSON web services in the portal. This
Screenlet calls the following services and methods.

\noindent\hrulefill

\begin{longtable}[]{@{}
  >{\raggedright\arraybackslash}p{(\columnwidth - 4\tabcolsep) * \real{0.3889}}
  >{\raggedright\arraybackslash}p{(\columnwidth - 4\tabcolsep) * \real{0.3333}}
  >{\raggedright\arraybackslash}p{(\columnwidth - 4\tabcolsep) * \real{0.2778}}@{}}
\toprule\noalign{}
\begin{minipage}[b]{\linewidth}\raggedright
Service
\end{minipage} & \begin{minipage}[b]{\linewidth}\raggedright
Method
\end{minipage} & \begin{minipage}[b]{\linewidth}\raggedright
Notes
\end{minipage} \\
\midrule\noalign{}
\endhead
\bottomrule\noalign{}
\endlastfoot
\texttt{ScreensassetentryService} (Screens compatibility plugin) &
\texttt{getAssetEntry} & With \texttt{entryId} \\
\texttt{ScreensassetentryService} (Screens compatibility plugin) &
\texttt{getAssetEntry} & With \texttt{classPK} and \texttt{className} \\
\texttt{ScreensassetentryService} (Screens compatibility plugin) &
\texttt{getAssetEntries} & With \texttt{entryQuery} \\
\texttt{ScreensassetentryService} (Screens compatibility plugin) &
\texttt{getAssetEntries} & With \texttt{companyId}, \texttt{groupId},
and \texttt{portletItemName} \\
\end{longtable}

\noindent\hrulefill

\subsection{Module}\label{module-36}

\begin{itemize}
\tightlist
\item
  None
\end{itemize}

\subsection{Themes}\label{themes-15}

\begin{itemize}
\tightlist
\item
  Default
\end{itemize}

The Default Theme can use different components to show a blogs entry
(\texttt{BlogsEntry}). For example, it uses \texttt{UILabel} to show a
blog's text, and
\href{/docs/7-0/reference/-/knowledge_base/r/userportraitscreenlet-for-ios}{User
Portrait Screenlet} to show the profile picture of the Liferay user who
posted it. Note that other Themes may use different components.

\begin{figure}
\centering
\includegraphics{./images/screens-ios-blogsentrydisplay.png}
\caption{Blogs Entry Display Screenlet using the Default Theme.}
\end{figure}

\subsection{Offline}\label{offline-35}

This Screenlet supports offline mode so it can function without a
network connection. For more information on how offline mode works, see
the
\href{/docs/7-0/tutorials/-/knowledge_base/t/architecture-of-offline-mode-in-liferay-screens}{tutorial
on its architecture}. Here are the offline mode policies that you can
use with this Screenlet:

\noindent\hrulefill

Policy \textbar{} What happens \textbar{} When to use \textbar{}
\texttt{remote-only} \textbar{} The Screenlet loads the data from the
Liferay instance. If a connection issue occurs, the Screenlet uses the
listener to notify the developer about the error. If the Screenlet
successfully loads the data, it stores it in the local cache for later
use. \textbar{} Use this policy when you always need to show updated
data, and show nothing when there's no connection. \textbar{}
\texttt{cache-only} \textbar{} The Screenlet loads the data from the
local cache. If the data isn't there, the Screenlet uses the listener to
notify the developer about the error. \textbar{} Use this policy when
you always need to show local data, without retrieving remote data under
any circumstance. \textbar{} \texttt{remote-first} \textbar{} The
Screenlet loads the data from the Liferay instance. If this succeeds,
the Screenlet shows the data to the user and stores it in the local
cache for later use. If a connection issue occurs, the Screenlet
retrieves the data from the local cache. If the data doesn't exist
there, the Screenlet uses the listener to notify the developer about the
error. \textbar{} Use this policy to show the most recent version of the
data when connected, but show an outdated version when there's no
connection. \textbar{} \texttt{cache-first} \textbar{} The Screenlet
loads the data from the local cache. If the data isn't there, the
Screenlet requests it from the Liferay instance and notifies the
developer about any errors that occur (including connectivity errors).
\textbar{} Use this policy to save bandwidth and loading time in case
you have local (but probably outdated) data. \textbar{}

\noindent\hrulefill

\subsection{Required Attributes}\label{required-attributes-31}

\begin{itemize}
\tightlist
\item
  \texttt{assetEntryId} or \texttt{classPK}
\end{itemize}

\subsection{Attributes}\label{attributes-36}

\noindent\hrulefill

Attribute \textbar{} Data type \textbar{} Explanation \textbar{}
\texttt{assetEntryId} \textbar{} \texttt{number} \textbar{} The primary
key of the blog entry (\texttt{BlogsEntry}). \textbar{} \texttt{classPK}
\textbar{} \texttt{number} \textbar{} The \texttt{BlogsEntry} object's
unique identifier. \textbar{} \texttt{autoLoad} \textbar{}
\texttt{boolean} \textbar{} Whether the blog entry automatically loads
when the Screenlet appears in the app's UI. The default value is
\texttt{true}. \textbar{} \texttt{offlinePolicy} \textbar{}
\texttt{string} \textbar{} The offline mode setting. The default value
is \texttt{remote-first}. See
\href{/docs/7-0/reference/-/knowledge_base/r/blogs-entry-display-screenlet-for-ios\#offline}{the
Offline section} for details. \textbar{}

\noindent\hrulefill

\subsection{Delegate}\label{delegate-15}

Blogs Entry Display Screenlet delegates some events to an object that
conforms to the \texttt{BlogsEntryDisplayScreenletDelegate} protocol.
This protocol lets you implement the following methods:

\begin{itemize}
\item
  \texttt{-\ screenlet:onBlogEntryResponse:}: Called when the Screenlet
  receives the \texttt{BlogsEntry} object.
\item
  \texttt{-\ screenlet:onBlogEntryError:}: Called when an error occurs
  in the process. The \texttt{NSError} object describes the error.
\end{itemize}

\section{Image Display Screenlet for
iOS}\label{image-display-screenlet-for-ios}

\subsection{Requirements}\label{requirements-37}

\begin{itemize}
\tightlist
\item
  Xcode 9.3 or above
\item
  iOS 11 SDK
\item
  Liferay Portal 6.2 CE/EE, Liferay CE Portal 7.0/7.1, Liferay DXP 7.0+
\item
  Liferay Screens Compatibility app
  (\href{http://www.liferay.com/marketplace/-/mp/application/54365664}{CE}
  or
  \href{http://www.liferay.com/marketplace/-/mp/application/54369726}{EE/DXP}).
  This app is preinstalled in Liferay CE Portal 7.0/7.1 and Liferay DXP
  7.0+.
\end{itemize}

\subsection{Compatibility}\label{compatibility-37}

\begin{itemize}
\tightlist
\item
  iOS 9 and above
\end{itemize}

\subsection{Xamarin Requirements}\label{xamarin-requirements-37}

\begin{itemize}
\tightlist
\item
  Visual Studio 7.2
\item
  Mono .NET framework 5.4.1.6
\end{itemize}

\subsection{Features}\label{features-37}

Image Display Screenlet displays an image file from a Liferay instance's
Documents and Media Library.

\subsection{JSON Services Used}\label{json-services-used-36}

Screenlets in Liferay Screens call JSON web services in the portal. This
Screenlet calls the following services and methods.

\noindent\hrulefill

\begin{longtable}[]{@{}
  >{\raggedright\arraybackslash}p{(\columnwidth - 4\tabcolsep) * \real{0.3889}}
  >{\raggedright\arraybackslash}p{(\columnwidth - 4\tabcolsep) * \real{0.3333}}
  >{\raggedright\arraybackslash}p{(\columnwidth - 4\tabcolsep) * \real{0.2778}}@{}}
\toprule\noalign{}
\begin{minipage}[b]{\linewidth}\raggedright
Service
\end{minipage} & \begin{minipage}[b]{\linewidth}\raggedright
Method
\end{minipage} & \begin{minipage}[b]{\linewidth}\raggedright
Notes
\end{minipage} \\
\midrule\noalign{}
\endhead
\bottomrule\noalign{}
\endlastfoot
\texttt{ScreensassetentryService} (Screens compatibility plugin) &
\texttt{getAssetEntry} & With \texttt{entryId} \\
\texttt{ScreensassetentryService} (Screens compatibility plugin) &
\texttt{getAssetEntry} & With \texttt{classPK} and \texttt{className} \\
\texttt{ScreensassetentryService} (Screens compatibility plugin) &
\texttt{getAssetEntries} & With \texttt{entryQuery} \\
\texttt{ScreensassetentryService} (Screens compatibility plugin) &
\texttt{getAssetEntries} & With \texttt{companyId}, \texttt{groupId},
and \texttt{portletItemName} \\
\end{longtable}

\noindent\hrulefill

\subsection{Module}\label{module-37}

\begin{itemize}
\tightlist
\item
  None
\end{itemize}

\subsection{Themes}\label{themes-16}

\begin{itemize}
\tightlist
\item
  Default
\end{itemize}

The Default Theme uses an iOS \texttt{UIImageView} for displaying the
image.

\begin{figure}
\centering
\includegraphics{./images/screens-ios-imagedisplay.png}
\caption{Image Display Screenlet using the Default Theme.}
\end{figure}

\subsection{Offline}\label{offline-36}

This Screenlet supports offline mode so it can function without a
network connection. For more information on how offline mode works, see
the
\href{/docs/7-0/tutorials/-/knowledge_base/t/architecture-of-offline-mode-in-liferay-screens}{tutorial
on its architecture}. Here are the offline mode policies that you can
use with this Screenlet:

\noindent\hrulefill

Policy \textbar{} What happens \textbar{} When to use \textbar{}
\texttt{remote-only} \textbar{} The Screenlet loads the data from the
Liferay instance. If a connection issue occurs, the Screenlet uses the
listener to notify the developer about the error. If the Screenlet
successfully loads the data, it stores it in the local cache for later
use. \textbar{} Use this policy when you always need to show updated
data, and show nothing when there's no connection. \textbar{}
\texttt{cache-only} \textbar{} The Screenlet loads the data from the
local cache. If the data isn't there, the Screenlet uses the listener to
notify the developer about the error. \textbar{} Use this policy when
you always need to show local data, without retrieving remote
information under any circumstance. \textbar{} \texttt{remote-first}
\textbar{} The Screenlet loads the data from the Liferay instance. If
this succeeds, the Screenlet shows the data to the user and stores it in
the local cache for later use. If a connection issue occurs, the
Screenlet retrieves the data from the local cache. If the data doesn't
exist there, the Screenlet uses the listener to notify the developer
about the error. \textbar{} Use this policy to show the most recent
version of the data when connected, but show an outdated version when
there's no connection. \textbar{} \texttt{cache-first} \textbar{} The
Screenlet loads the data from the local cache. If the data isn't there,
the Screenlet requests it from the Liferay instance and notifies the
developer about any errors that occur (including connectivity errors).
\textbar{} Use this policy to save bandwidth and loading time in case
you have local (but probably outdated) data. \textbar{}

\noindent\hrulefill

\subsection{Required Attributes}\label{required-attributes-32}

\begin{itemize}
\tightlist
\item
  \texttt{assetEntryId}
\end{itemize}

If you don't use \texttt{assetEntryId}, you must use these attributes:

\begin{itemize}
\tightlist
\item
  \texttt{className}
\item
  \texttt{classPK}
\end{itemize}

\subsection{Attributes}\label{attributes-37}

\noindent\hrulefill

Attribute \textbar{} Data type \textbar{} Explanation \textbar{}
\texttt{assetEntryId} \textbar{} \texttt{number} \textbar{} The primary
key of the image. \textbar{} \texttt{className} \textbar{}
\texttt{string} \textbar{} The image's fully qualified class name. Since
files in a Documents and Media Library are \texttt{DLFileEntry} objects,
their \texttt{className} is
\href{@platform-ref@/7.0-latest/javadocs/portal-kernel/com/liferay/document/library/kernel/model/DLFileEntry.html}{\texttt{com.liferay.document.library.kernel.model.DLFileEntry}}.
The \texttt{className} and \texttt{classPK} attributes are required to
instantiate the Screenlet. \textbar{} \texttt{classPK} \textbar{}
\texttt{number} \textbar{} The image's unique identifier. The
\texttt{className} and \texttt{classPK} attributes are required to
instantiate the Screenlet. \textbar{} \texttt{autoLoad} \textbar{}
\texttt{boolean} \textbar{} Whether the image automatically loads when
the Screenlet appears in the app's UI. The default value is
\texttt{true}. \textbar{} \texttt{offlinePolicy} \textbar{}
\texttt{string} \textbar{} The offline mode setting. The default value
is \texttt{remote-first}. See the
\href{/docs/7-0/reference/-/knowledge_base/r/image-display-screenlet-for-ios\#offline}{Offline
section} for details. \textbar{}

\noindent\hrulefill

\subsection{Delegate}\label{delegate-16}

Because images are files, Image Display Screenlet delegates its events
to an object that conforms to the \texttt{FileDisplayScreenletDelegate}
protocol. This protocol lets you implement the following methods:

\begin{itemize}
\item
  \texttt{-\ screenlet:onFileAssetResponse:}: Called when the Screenlet
  receives the image file.
\item
  \texttt{-\ screenlet:onFileAssetError:}: Called when an error occurs
  in the process. The \texttt{NSError} object describes the error.
\end{itemize}

\section{Video Display Screenlet for
iOS}\label{video-display-screenlet-for-ios}

\subsection{Requirements}\label{requirements-38}

\begin{itemize}
\tightlist
\item
  Xcode 9.3 or above
\item
  iOS 11 SDK
\item
  Liferay Portal 6.2 CE/EE, Liferay CE Portal 7.0/7.1, Liferay DXP 7.0+
\item
  Liferay Screens Compatibility app
  (\href{http://www.liferay.com/marketplace/-/mp/application/54365664}{CE}
  or
  \href{http://www.liferay.com/marketplace/-/mp/application/54369726}{EE/DXP}).
  This app is preinstalled in Liferay CE Portal 7.0/7.1 and Liferay DXP
  7.0+.
\end{itemize}

\subsection{Compatibility}\label{compatibility-38}

\begin{itemize}
\tightlist
\item
  iOS 9 and above
\end{itemize}

\subsection{Xamarin Requirements}\label{xamarin-requirements-38}

\begin{itemize}
\tightlist
\item
  Visual Studio 7.2
\item
  Mono .NET framework 5.4.1.6
\end{itemize}

\subsection{Features}\label{features-38}

Video Display Screenlet displays a video file from a Liferay instance's
Documents and Media Library.

\subsection{JSON Services Used}\label{json-services-used-37}

Screenlets in Liferay Screens call JSON web services in the portal. This
Screenlet calls the following services and methods.

\noindent\hrulefill

\begin{longtable}[]{@{}
  >{\raggedright\arraybackslash}p{(\columnwidth - 4\tabcolsep) * \real{0.3889}}
  >{\raggedright\arraybackslash}p{(\columnwidth - 4\tabcolsep) * \real{0.3333}}
  >{\raggedright\arraybackslash}p{(\columnwidth - 4\tabcolsep) * \real{0.2778}}@{}}
\toprule\noalign{}
\begin{minipage}[b]{\linewidth}\raggedright
Service
\end{minipage} & \begin{minipage}[b]{\linewidth}\raggedright
Method
\end{minipage} & \begin{minipage}[b]{\linewidth}\raggedright
Notes
\end{minipage} \\
\midrule\noalign{}
\endhead
\bottomrule\noalign{}
\endlastfoot
\texttt{ScreensassetentryService} (Screens compatibility plugin) &
\texttt{getAssetEntry} & With \texttt{entryId} \\
\texttt{ScreensassetentryService} (Screens compatibility plugin) &
\texttt{getAssetEntry} & With \texttt{classPK} and \texttt{className} \\
\texttt{ScreensassetentryService} (Screens compatibility plugin) &
\texttt{getAssetEntries} & With \texttt{entryQuery} \\
\texttt{ScreensassetentryService} (Screens compatibility plugin) &
\texttt{getAssetEntries} & With \texttt{companyId}, \texttt{groupId},
and \texttt{portletItemName} \\
\end{longtable}

\noindent\hrulefill

\subsection{Module}\label{module-38}

\begin{itemize}
\tightlist
\item
  None
\end{itemize}

\subsection{Themes}\label{themes-17}

\begin{itemize}
\tightlist
\item
  Default
\end{itemize}

The Default Theme uses an iOS \texttt{AVPlayerViewController} to display
the video.

\begin{figure}
\centering
\includegraphics{./images/screens-ios-videodisplay.png}
\caption{Video Display Screenlet using the Default Theme.}
\end{figure}

\subsection{Offline}\label{offline-37}

This Screenlet supports offline mode so it can function without a
network connection. For more information on how offline mode works, see
the
\href{/docs/7-0/tutorials/-/knowledge_base/t/architecture-of-offline-mode-in-liferay-screens}{tutorial
on its architecture}. Here are the offline mode policies that you can
use with this Screenlet:

\noindent\hrulefill

Policy \textbar{} What happens \textbar{} When to use \textbar{}
\texttt{remote-only} \textbar{} The Screenlet loads the data from the
Liferay instance. If a connection issue occurs, the Screenlet uses the
listener to notify the developer about the error. If the Screenlet
successfully loads the data, it stores it in the local cache for later
use. \textbar{} Use this policy when you always need to show updated
data, and show nothing when there's no connection. \textbar{}
\texttt{cache-only} \textbar{} The Screenlet loads the data from the
local cache. If the data isn't there, the Screenlet uses the listener to
notify the developer about the error. \textbar{} Use this policy when
you always need to show local data, without retrieving remote
information under any circumstance. \textbar{} \texttt{remote-first}
\textbar{} The Screenlet loads the data from the Liferay instance. If
this succeeds, the Screenlet shows the data to the user and stores it in
the local cache for later use. If a connection issue occurs, the
Screenlet retrieves the data from the local cache. If the data doesn't
exist there, the Screenlet uses the listener to notify the developer
about the error. \textbar{} Use this policy to show the most recent
version of the data when connected, but show an outdated version when
there's no connection. \textbar{} \texttt{cache-first} \textbar{} The
Screenlet loads the data from the local cache. If the data isn't there,
the Screenlet requests it from the Liferay instance and notifies the
developer about any errors that occur (including connectivity errors).
\textbar{} Use this policy to save bandwidth and loading time in case
you have local (but probably outdated) data. \textbar{}

\noindent\hrulefill

\subsection{Required Attributes}\label{required-attributes-33}

\begin{itemize}
\tightlist
\item
  \texttt{assetEntryId}
\end{itemize}

If you don't use \texttt{assetEntryId}, you must use these attributes:

\begin{itemize}
\tightlist
\item
  \texttt{className}
\item
  \texttt{classPK}
\end{itemize}

\subsection{Attributes}\label{attributes-38}

\noindent\hrulefill

Attribute \textbar{} Data type \textbar{} Explanation \textbar{}
\texttt{assetEntryId} \textbar{} \texttt{number} \textbar{} The primary
key of the video file. \textbar{} \texttt{className} \textbar{}
\texttt{string} \textbar{} The video file's fully qualified class name.
Since files in a Documents and Media Library are \texttt{DLFileEntry}
objects, the \texttt{className} is
\href{@platform-ref@/7.0-latest/javadocs/portal-kernel/com/liferay/document/library/kernel/model/DLFileEntry.html}{\texttt{com.liferay.document.library.kernel.model.DLFileEntry}}.
The \texttt{className} and \texttt{classPK} attributes are required to
instantiate the Screenlet. \textbar{} \texttt{classPK} \textbar{}
\texttt{number} \textbar{} The video file's unique identifier. The
\texttt{className} and \texttt{classPK} attributes are required to
instantiate the Screenlet. \textbar{} \texttt{autoLoad} \textbar{}
\texttt{boolean} \textbar{} Whether the video automatically loads when
the Screenlet appears in the app's UI. The default value is
\texttt{true}. \textbar{} \texttt{offlinePolicy} \textbar{}
\texttt{string} \textbar{} The offline mode setting. See
\href{/docs/7-0/reference/-/knowledge_base/r/video-display-screenlet-for-ios\#offline}{the
Offline section} for details. \textbar{}

\noindent\hrulefill

\subsection{Delegate}\label{delegate-17}

Because images are files, Video Display Screenlet delegates its events
to an object that conforms to the \texttt{FileDisplayScreenletDelegate}
protocol. This protocol lets you implement the following methods:

\begin{itemize}
\item
  \texttt{-\ screenlet:onFileAssetResponse:}: Called when the Screenlet
  receives the image file.
\item
  \texttt{-\ screenlet:onFileAssetError:}: Called when an error occurs
  in the process. The \texttt{NSError} object describes the error.
\end{itemize}

\section{Audio Display Screenlet for
iOS}\label{audio-display-screenlet-for-ios}

\subsection{Requirements}\label{requirements-39}

\begin{itemize}
\tightlist
\item
  Xcode 9.3 or above
\item
  iOS 11 SDK
\item
  Liferay Portal 6.2 CE/EE, Liferay CE Portal 7.0/7.1, Liferay DXP 7.0+
\item
  Liferay Screens Compatibility app
  (\href{http://www.liferay.com/marketplace/-/mp/application/54365664}{CE}
  or
  \href{http://www.liferay.com/marketplace/-/mp/application/54369726}{EE/DXP}).
  This app is preinstalled in Liferay CE Portal 7.0/7.1 and Liferay DXP
  7.0+.
\end{itemize}

\subsection{Compatibility}\label{compatibility-39}

\begin{itemize}
\tightlist
\item
  iOS 9 and above
\end{itemize}

\subsection{Xamarin Requirements}\label{xamarin-requirements-39}

\begin{itemize}
\tightlist
\item
  Visual Studio 7.2
\item
  Mono .NET framework 5.4.1.6
\end{itemize}

\subsection{Features}\label{features-39}

Audio Display Screenlet displays an audio file from a Liferay instance's
Documents and Media Library.

\subsection{JSON Services Used}\label{json-services-used-38}

Screenlets in Liferay Screens call JSON web services in the portal. This
Screenlet calls the following services and methods.

\noindent\hrulefill

\begin{longtable}[]{@{}
  >{\raggedright\arraybackslash}p{(\columnwidth - 4\tabcolsep) * \real{0.3889}}
  >{\raggedright\arraybackslash}p{(\columnwidth - 4\tabcolsep) * \real{0.3333}}
  >{\raggedright\arraybackslash}p{(\columnwidth - 4\tabcolsep) * \real{0.2778}}@{}}
\toprule\noalign{}
\begin{minipage}[b]{\linewidth}\raggedright
Service
\end{minipage} & \begin{minipage}[b]{\linewidth}\raggedright
Method
\end{minipage} & \begin{minipage}[b]{\linewidth}\raggedright
Notes
\end{minipage} \\
\midrule\noalign{}
\endhead
\bottomrule\noalign{}
\endlastfoot
\texttt{ScreensassetentryService} (Screens compatibility plugin) &
\texttt{getAssetEntry} & With \texttt{entryId} \\
\texttt{ScreensassetentryService} (Screens compatibility plugin) &
\texttt{getAssetEntry} & With \texttt{classPK} and \texttt{className} \\
\texttt{ScreensassetentryService} (Screens compatibility plugin) &
\texttt{getAssetEntries} & With \texttt{entryQuery} \\
\texttt{ScreensassetentryService} (Screens compatibility plugin) &
\texttt{getAssetEntries} & With \texttt{companyId}, \texttt{groupId},
and \texttt{portletItemName} \\
\end{longtable}

\noindent\hrulefill

\subsection{Module}\label{module-39}

\begin{itemize}
\tightlist
\item
  None
\end{itemize}

\subsection{Themes}\label{themes-18}

\begin{itemize}
\tightlist
\item
  Default
\end{itemize}

The Default Theme uses an iOS \texttt{AVAudioPlayer} to display the
audio player. For the player components, this Theme uses
\texttt{UIButton}, \texttt{UISlider}, and several \texttt{UILabel}
instances.

\begin{figure}
\centering
\includegraphics{./images/screens-ios-audiodisplay.png}
\caption{Audio Display Screenlet using the Default Theme.}
\end{figure}

\subsection{Offline}\label{offline-38}

This Screenlet supports offline mode so it can function without a
network connection. For more information on how offline mode works, see
the
\href{/docs/7-0/tutorials/-/knowledge_base/t/architecture-of-offline-mode-in-liferay-screens}{tutorial
on its architecture}. Here are the offline mode policies that you can
use with this Screenlet:

\noindent\hrulefill

Policy \textbar{} What happens \textbar{} When to use \textbar{}
\texttt{remote-only} \textbar{} The Screenlet loads the data from the
Liferay instance. If a connection issue occurs, the Screenlet uses the
listener to notify the developer about the error. If the Screenlet
successfully loads the data, it stores it in the local cache for later
use. \textbar{} Use this policy when you always need to show updated
data, and show nothing when there's no connection. \textbar{}
\texttt{cache-only} \textbar{} The Screenlet loads the data from the
local cache. If the data isn't there, the Screenlet uses the listener to
notify the developer about the error. \textbar{} Use this policy when
you always need to show local data, without retrieving remote
information under any circumstance. \textbar{} \texttt{remote-first}
\textbar{} The Screenlet loads the data from the Liferay instance. If
this succeeds, the Screenlet shows the data to the user and stores it in
the local cache for later use. If a connection issue occurs, the
Screenlet retrieves the data from the local cache. If the data doesn't
exist there, the Screenlet uses the listener to notify the developer
about the error. \textbar{} Use this policy to show the most recent
version of the data when connected, but show an outdated version when
there's no connection. \textbar{} \texttt{cache-first} \textbar{} The
Screenlet loads the data from the local cache. If the data isn't there,
the Screenlet requests it from the Liferay instance and notifies the
developer about any errors that occur (including connectivity errors).
\textbar{} Use this policy to save bandwidth and loading time in case
you have local (but probably outdated) data. \textbar{}

\noindent\hrulefill

\subsection{Required Attributes}\label{required-attributes-34}

\begin{itemize}
\tightlist
\item
  \texttt{assetEntryId}
\end{itemize}

If you don't use \texttt{assetEntryId}, you must use these attributes:

\begin{itemize}
\tightlist
\item
  \texttt{className}
\item
  \texttt{classPK}
\end{itemize}

\subsection{Attributes}\label{attributes-39}

\noindent\hrulefill

Attribute \textbar{} Data type \textbar{} Explanation \textbar{}
\texttt{assetEntryId} \textbar{} \texttt{number} \textbar{} The primary
key of the audio file. \textbar{} \texttt{className} \textbar{}
\texttt{string} \textbar{} The audio file's fully qualified class name.
Since files in a Documents and Media Library are \texttt{DLFileEntry}
objects, their \texttt{className} is
\href{@platform-ref@/7.0-latest/javadocs/portal-kernel/com/liferay/document/library/kernel/model/DLFileEntry.html}{\texttt{com.liferay.document.library.kernel.model.DLFileEntry}}.
The \texttt{className} and \texttt{classPK} attributes are required to
instantiate the Screenlet. \textbar{} \texttt{classPK} \textbar{}
\texttt{number} \textbar{} The audio file's unique identifier. The
\texttt{className} and \texttt{classPK} attributes are required to
instantiate the Screenlet. \textbar{} \texttt{autoLoad} \textbar{}
\texttt{boolean} \textbar{} Whether the audio file automatically loads
when the Screenlet appears in the app's UI. The default value is
\texttt{true}. \textbar{} \texttt{offlinePolicy} \textbar{}
\texttt{string} \textbar{} The offline mode setting. See
\href{/docs/7-0/reference/-/knowledge_base/r/audio-display-screenlet-for-ios\#offline}{the
Offline section} for details. \textbar{}

\noindent\hrulefill

\subsection{Delegate}\label{delegate-18}

Audio Display Screenlet delegates its events to an object that conforms
to the \texttt{FileDisplayScreenletDelegate} protocol. This protocol
lets you implement the following methods:

\begin{itemize}
\item
  \texttt{-\ screenlet:onFileAssetResponse:}: Called when the Screenlet
  receives the audio file.
\item
  \texttt{-\ screenlet:onFileAssetError:}: Called when an error occurs
  in the process. An \texttt{NSError} object describes the error.
\end{itemize}

\section{PDF Display Screenlet for
iOS}\label{pdf-display-screenlet-for-ios}

\subsection{Requirements}\label{requirements-40}

\begin{itemize}
\tightlist
\item
  Xcode 9.3 or above
\item
  iOS 11 SDK
\item
  Liferay Portal 6.2 CE/EE, Liferay CE Portal 7.0/7.1, Liferay DXP 7.0+
\item
  Liferay Screens Compatibility app
  (\href{http://www.liferay.com/marketplace/-/mp/application/54365664}{CE}
  or
  \href{http://www.liferay.com/marketplace/-/mp/application/54369726}{EE/DXP}).
  This app is preinstalled in Liferay CE Portal 7.0/7.1 and Liferay DXP
  7.0+.
\end{itemize}

\subsection{Compatibility}\label{compatibility-40}

\begin{itemize}
\tightlist
\item
  iOS 9 and above
\end{itemize}

\subsection{Xamarin Requirements}\label{xamarin-requirements-40}

\begin{itemize}
\tightlist
\item
  Visual Studio 7.2
\item
  Mono .NET framework 5.4.1.6
\end{itemize}

\subsection{Features}\label{features-40}

PDF Display Screenlet displays a PDF file from a Liferay Instance's
Documents and Media Library.

\subsection{JSON Services Used}\label{json-services-used-39}

Screenlets in Liferay Screens call JSON web services in the portal. This
Screenlet calls the following services and methods.

\noindent\hrulefill

\begin{longtable}[]{@{}
  >{\raggedright\arraybackslash}p{(\columnwidth - 4\tabcolsep) * \real{0.3889}}
  >{\raggedright\arraybackslash}p{(\columnwidth - 4\tabcolsep) * \real{0.3333}}
  >{\raggedright\arraybackslash}p{(\columnwidth - 4\tabcolsep) * \real{0.2778}}@{}}
\toprule\noalign{}
\begin{minipage}[b]{\linewidth}\raggedright
Service
\end{minipage} & \begin{minipage}[b]{\linewidth}\raggedright
Method
\end{minipage} & \begin{minipage}[b]{\linewidth}\raggedright
Notes
\end{minipage} \\
\midrule\noalign{}
\endhead
\bottomrule\noalign{}
\endlastfoot
\texttt{ScreensassetentryService} (Screens compatibility plugin) &
\texttt{getAssetEntry} & With \texttt{entryId} \\
\texttt{ScreensassetentryService} (Screens compatibility plugin) &
\texttt{getAssetEntry} & With \texttt{classPK} and \texttt{className} \\
\texttt{ScreensassetentryService} (Screens compatibility plugin) &
\texttt{getAssetEntries} & With \texttt{entryQuery} \\
\texttt{ScreensassetentryService} (Screens compatibility plugin) &
\texttt{getAssetEntries} & With \texttt{companyId}, \texttt{groupId},
and \texttt{portletItemName} \\
\end{longtable}

\noindent\hrulefill

\subsection{Module}\label{module-40}

\begin{itemize}
\tightlist
\item
  None
\end{itemize}

\subsection{Themes}\label{themes-19}

\begin{itemize}
\tightlist
\item
  Default
\end{itemize}

The Default Theme uses an iOS \texttt{UIWebView} for displaying the PDF
file.

\begin{figure}
\centering
\includegraphics{./images/screens-ios-pdfdisplay.png}
\caption{PDF Display Screenlet using the Default Theme.}
\end{figure}

\subsection{Offline}\label{offline-39}

This Screenlet supports offline mode so it can function without a
network connection. For more information on how offline mode works, see
the
\href{/docs/7-0/tutorials/-/knowledge_base/t/architecture-of-offline-mode-in-liferay-screens}{tutorial
on its architecture}. Here are the offline mode policies that you can
use with this Screenlet:

\noindent\hrulefill

Policy \textbar{} What happens \textbar{} When to use \textbar{}
\texttt{remote-only} \textbar{} The Screenlet loads the data from the
Liferay instance. If a connection issue occurs, the Screenlet uses the
listener to notify the developer about the error. If the Screenlet
successfully loads the data, it stores it in the local cache for later
use. \textbar{} Use this policy when you always need to show updated
data, and show nothing when there's no connection. \textbar{}
\texttt{cache-only} \textbar{} The Screenlet loads the data from the
local cache. If the data isn't there, the Screenlet uses the listener to
notify the developer about the error. \textbar{} Use this policy when
you always need to show local data, without retrieving remote
information under any circumstance. \textbar{} \texttt{remote-first}
\textbar{} The Screenlet loads the data from the Liferay instance. If
this succeeds, the Screenlet shows the data to the user and stores it in
the local cache for later use. If a connection issue occurs, the
Screenlet retrieves the data from the local cache. If the data doesn't
exist there, the Screenlet uses the listener to notify the developer
about the error. \textbar{} Use this policy to show the most recent
version of the data when connected, but show an outdated version when
there's no connection. \textbar{} \texttt{cache-first} \textbar{} The
Screenlet loads the data from the local cache. If the data isn't there,
the Screenlet requests it from the Liferay instance and notifies the
developer about any errors that occur (including connectivity errors).
\textbar{} Use this policy to save bandwidth and loading time in case
you have local (but probably outdated) data. \textbar{}

\noindent\hrulefill

\subsection{Required Attributes}\label{required-attributes-35}

\begin{itemize}
\tightlist
\item
  \texttt{assetEntryId}
\end{itemize}

If you don't use \texttt{assetEntryId}, you must use these attributes:

\begin{itemize}
\tightlist
\item
  \texttt{className}
\item
  \texttt{classPK}
\end{itemize}

\subsection{Attributes}\label{attributes-40}

\noindent\hrulefill

Attribute \textbar{} Data type \textbar{} Explanation \textbar{}
\texttt{assetEntryId} \textbar{} \texttt{number} \textbar{} The primary
key of the PDF file. \textbar{} \texttt{className} \textbar{}
\texttt{string} \textbar{} The PDF file's fully qualified class name.
Since files in a Documents and Media Library are \texttt{DLFileEntry}
objects, their \texttt{className} is
\href{@platform-ref@/7.0-latest/javadocs/portal-kernel/com/liferay/document/library/kernel/model/DLFileEntry.html}{\texttt{com.liferay.document.library.kernel.model.DLFileEntry}}.
The \texttt{className} and \texttt{classPK} attributes are required to
instantiate the Screenlet. \textbar{} \texttt{classPK} \textbar{}
\texttt{number} \textbar{} The PDF file's unique identifier. The
\texttt{className} and \texttt{classPK} attributes are required to
instantiate the Screenlet. \textbar{} \texttt{autoLoad} \textbar{}
\texttt{boolean} \textbar{} Whether the PDF automatically loads when the
Screenlet appears in the app's UI. The default value is \texttt{true}.
\textbar{} \texttt{offlinePolicy} \textbar{} \texttt{string} \textbar{}
The offline mode setting. See
\href{/docs/7-0/reference/-/knowledge_base/r/pdf-display-screenlet-for-ios\#offline}{the
Offline section} for details. \textbar{}

\noindent\hrulefill

\subsection{Delegate}\label{delegate-19}

Because PDFs are files, PDF Display Screenlet delegates some events to
an object that conforms to the \texttt{FileDisplayScreenletDelegate}
protocol. This protocol lets you implement the following methods:

\begin{itemize}
\item
  \texttt{-\ screenlet:onFileAssetResponse:}: Called when the Screenlet
  receives the PDF.
\item
  \texttt{-\ screenlet:onFileAssetError:}: Called when an error occurs
  in the process. An \texttt{NSError} object describes the error.
\end{itemize}

\section{File Display Screenlet for
iOS}\label{file-display-screenlet-for-ios}

\subsection{Requirements}\label{requirements-41}

\begin{itemize}
\tightlist
\item
  Xcode 9.3 or above
\item
  iOS 11 SDK
\item
  Liferay Portal 6.2 CE/EE, Liferay CE Portal 7.0/7.1, Liferay DXP 7.0+
\item
  Liferay Screens Compatibility app
  (\href{http://www.liferay.com/marketplace/-/mp/application/54365664}{CE}
  or
  \href{http://www.liferay.com/marketplace/-/mp/application/54369726}{EE/DXP}).
  This app is preinstalled in Liferay CE Portal 7.0/7.1 and Liferay DXP
  7.0+.
\end{itemize}

\subsection{Compatibility}\label{compatibility-41}

\begin{itemize}
\tightlist
\item
  iOS 9 and above
\end{itemize}

\subsection{Xamarin Requirements}\label{xamarin-requirements-41}

\begin{itemize}
\tightlist
\item
  Visual Studio 7.2
\item
  Mono .NET framework 5.4.1.6
\end{itemize}

\subsection{Features}\label{features-41}

File Display Screenlet shows a single file from a Liferay DXP instance's
Documents and Media Library. Use this Screenlet to display file types
not covered by the other display Screenlets (e.g., DOC, PPT, XLS).

\subsection{JSON Services Used}\label{json-services-used-40}

Screenlets in Liferay Screens call JSON web services in the portal. This
Screenlet calls the following services and methods.

\noindent\hrulefill

\begin{longtable}[]{@{}
  >{\raggedright\arraybackslash}p{(\columnwidth - 4\tabcolsep) * \real{0.3889}}
  >{\raggedright\arraybackslash}p{(\columnwidth - 4\tabcolsep) * \real{0.3333}}
  >{\raggedright\arraybackslash}p{(\columnwidth - 4\tabcolsep) * \real{0.2778}}@{}}
\toprule\noalign{}
\begin{minipage}[b]{\linewidth}\raggedright
Service
\end{minipage} & \begin{minipage}[b]{\linewidth}\raggedright
Method
\end{minipage} & \begin{minipage}[b]{\linewidth}\raggedright
Notes
\end{minipage} \\
\midrule\noalign{}
\endhead
\bottomrule\noalign{}
\endlastfoot
\texttt{ScreensassetentryService} (Screens compatibility plugin) &
\texttt{getAssetEntry} & With \texttt{entryId} \\
\texttt{ScreensassetentryService} (Screens compatibility plugin) &
\texttt{getAssetEntry} & With \texttt{classPK} and \texttt{className} \\
\texttt{ScreensassetentryService} (Screens compatibility plugin) &
\texttt{getAssetEntries} & With \texttt{entryQuery} \\
\texttt{ScreensassetentryService} (Screens compatibility plugin) &
\texttt{getAssetEntries} & With \texttt{companyId}, \texttt{groupId},
and \texttt{portletItemName} \\
\end{longtable}

\noindent\hrulefill

\subsection{Module}\label{module-41}

\begin{itemize}
\tightlist
\item
  None
\end{itemize}

\subsection{Themes}\label{themes-20}

\begin{itemize}
\tightlist
\item
  Default
\end{itemize}

The Default View uses an iOS \texttt{UIWebView} for displaying the file.

\begin{figure}
\centering
\includegraphics{./images/screens-ios-filedisplay.png}
\caption{File Display Screenlet using the Default View.}
\end{figure}

\subsection{Offline}\label{offline-40}

This Screenlet supports offline mode so it can function without a
network connection. For more information on how offline mode works, see
the
\href{/docs/7-0/tutorials/-/knowledge_base/t/architecture-of-offline-mode-in-liferay-screens}{tutorial
on its architecture}. Here are the offline mode policies that you can
use with this Screenlet:

\noindent\hrulefill

Policy \textbar{} What happens \textbar{} When to use \textbar{}
\texttt{remote-only} \textbar{} The Screenlet loads the data from the
Liferay instance. If a connection issue occurs, the Screenlet uses the
listener to notify the developer about the error. If the Screenlet
successfully loads the data, it stores it in the local cache for later
use. \textbar{} Use this policy when you always need to show updated
data, and show nothing when there's no connection. \textbar{}
\texttt{cache-only} \textbar{} The Screenlet loads the data from the
local cache. If the data isn't there, the Screenlet uses the listener to
notify the developer about the error. \textbar{} Use this policy when
you always need to show local data, without retrieving remote
information under any circumstance. \textbar{} \texttt{remote-first}
\textbar{} The Screenlet loads the data from the Liferay instance. If
this succeeds, the Screenlet shows the data to the user and stores it in
the local cache for later use. If a connection issue occurs, the
Screenlet retrieves the data from the local cache. If the data doesn't
exist there, the Screenlet uses the listener to notify the developer
about the error. \textbar{} Use this policy to show the most recent
version of the data when connected, but show an outdated version when
there's no connection. \textbar{} \texttt{cache-first} \textbar{} The
Screenlet loads the data from the local cache. If the data isn't there,
the Screenlet requests it from the Liferay instance and notifies the
developer about any errors that occur (including connectivity errors).
\textbar{} Use this policy to save bandwidth and loading time in case
you have local (but probably outdated) data. \textbar{}

\noindent\hrulefill

\subsection{Attributes}\label{attributes-41}

\noindent\hrulefill

Attribute \textbar{} Data type \textbar{} Explanation \textbar{}
\texttt{assetEntryId} \textbar{} \texttt{number} \textbar{} The primary
key of the file. \textbar{} \texttt{className} \textbar{}
\texttt{string} \textbar{} The file's fully qualified class name. Since
files in a Documents and Media Library are \texttt{DLFileEntry} objects,
their \texttt{className} is
\href{@platform-ref@/7.0-latest/javadocs/portal-kernel/com/liferay/document/library/kernel/model/DLFileEntry.html}{\texttt{com.liferay.document.library.kernel.model.DLFileEntry}}.
The \texttt{className} and \texttt{classPK} attributes are required to
instantiate the Screenlet. \textbar{} \texttt{classPK} \textbar{}
\texttt{number} \textbar{} The file's unique identifier. The
\texttt{className} and \texttt{classPK} attributes are required to
instantiate the Screenlet. \textbar{} \texttt{autoLoad} \textbar{}
\texttt{boolean} \textbar{} Whether the file automatically loads when
the Screenlet appears in the app's UI. The default value is
\texttt{true}. \textbar{} \texttt{offlinePolicy} \textbar{}
\texttt{string} \textbar{} The offline mode setting. See
\href{/docs/7-0/reference/-/knowledge_base/r/file-display-screenlet-for-ios\#offline}{the
Offline section} for details. \textbar{}

\noindent\hrulefill

\subsection{Delegate}\label{delegate-20}

File Display Screenlet delegates some events to an object that conforms
to the \texttt{FileDisplayScreenletDelegate} protocol. This protocol
lets you implement the following methods:

\begin{itemize}
\item
  \texttt{-\ screenlet:onFileAssetResponse:}: Called when the Screenlet
  receives the file.
\item
  \texttt{-\ screenlet:onFileAssetError:}: Called when an error occurs
  in the process. An \texttt{NSError} object describes the error.
\end{itemize}

\section{Web Screenlet for iOS}\label{web-screenlet-for-ios}

\subsection{Requirements}\label{requirements-42}

\begin{itemize}
\tightlist
\item
  Xcode 9.3 or above
\item
  iOS 11 SDK
\item
  Liferay Portal 6.2 CE/EE, Liferay CE Portal 7.0/7.1, Liferay DXP 7.0+
\item
  Liferay Screens Compatibility app
  (\href{http://www.liferay.com/marketplace/-/mp/application/54365664}{CE}
  or
  \href{http://www.liferay.com/marketplace/-/mp/application/54369726}{EE/DXP}).
  This app is preinstalled in Liferay CE Portal 7.0/7.1 and Liferay DXP
  7.0+.
\end{itemize}

\subsection{Compatibility}\label{compatibility-42}

\begin{itemize}
\tightlist
\item
  iOS 9 and above
\end{itemize}

\subsection{Xamarin Requirements}\label{xamarin-requirements-42}

\begin{itemize}
\tightlist
\item
  Visual Studio 7.2
\item
  Mono .NET framework 5.4.1.6
\end{itemize}

\subsection{Features}\label{features-42}

Web Screenlet lets you display any web page. It also lets you customize
the web page through injection of local and remote JavaScript and CSS
files. If you're using Liferay DXP as backend, you can use
\href{/docs/7-0/user/-/knowledge_base/u/styling-apps-with-application-display-templates}{Application
Display Templates} in your page to customize its content from the server
side.

\subsection{Module}\label{module-42}

\begin{itemize}
\tightlist
\item
  None
\end{itemize}

\subsection{Themes}\label{themes-21}

\begin{itemize}
\tightlist
\item
  Default
\end{itemize}

The Default Theme uses an iOS \texttt{WKWebView} for displaying the web
page.

\begin{figure}
\centering
\includegraphics{./images/screens-ios-webscreenlet.png}
\caption{Web Screenlet using the Default Theme.}
\end{figure}

\subsection{Configuration}\label{configuration-1}

To learn how to use Web Screenlet, follow the steps in the tutorial
\href{/docs/7-0/tutorials/-/knowledge_base/t/rendering-web-pages-in-your-ios-app}{Rendering
Web Pages in Your iOS App}. That tutorial gives detailed instructions
for using the configuration items described here.

Web Screenlet has \texttt{WebScreenletConfiguration} and
\texttt{WebScreenletConfigurationBuilder} objects that you can use
together to supply the parameters that the Screenlet needs to work.
\texttt{WebScreenletConfigurationBuilder} has the following methods,
which let you supply the described configuration parameters:

\noindent\hrulefill

Method \textbar{} Returns \textbar{} Explanation \textbar{}
\texttt{addJs(localFile:\ String)} \textbar{}
\texttt{WebScreenletConfigurationBuilder} \textbar{} Adds a local
JavaScript file with the supplied filename. \textbar{}
\texttt{addCss(localFile:\ String)} \textbar{}
\texttt{WebScreenletConfigurationBuilder} \textbar{} Adds a local CSS
file with the supplied filename. \textbar{} \texttt{addJs(url:\ String)}
\textbar{} \texttt{WebScreenletConfigurationBuilder} \textbar{} Adds a
JavaScript file from the supplied URL. \textbar{}
\texttt{addCss(url:\ String)} \textbar{}
\texttt{WebScreenletConfigurationBuilder} \textbar{} Adds a CSS file
from the supplied URL. \textbar{} \texttt{set(webType:\ WebType)}
\textbar{} \texttt{WebScreenletConfigurationBuilder} \textbar{} Sets the
\href{/docs/7-0/reference/-/knowledge_base/r/web-screenlet-for-ios\#webtype}{\texttt{WebType}}.
\textbar{} \texttt{enableCordova()} \textbar{}
\texttt{WebScreenletConfigurationBuilder} \textbar{} Enables Cordova
inside the Web Screenlet. \textbar{} \texttt{load()} \textbar{}
\texttt{WebScreenletConfiguration} \textbar{} Returns the
\texttt{WebScreenletConfiguration} object that you can set to the
Screenlet instance. \textbar{}

\noindent\hrulefill

\noindent\hrulefill

\textbf{Note:} If you want to add comments in the scripts, use the
\texttt{/**/} notation.

\noindent\hrulefill

\subsubsection{WebType}\label{webtype-1}

\begin{itemize}
\item
  \textbf{WebType.liferayAuthenticated} (default): Displays a Liferay
  DXP page that requires authentication. The user must therefore be
  logged in with Screens via Login Screenlet or a
  \texttt{SessionContext} method. For this \texttt{WebType}, the URL you
  must pass to the \texttt{WebScreenletConfigurationBuilder} constructor
  is a relative URL. For example, if the full URL is
  \texttt{http://screens.liferay.org.es/web/guest/blog}, then the URL
  you must supply to the constructor is \texttt{/web/guest/blog}.
\item
  \textbf{WebType.other}: Displays any other page. For this
  \texttt{WebType}, the URL you must pass to the
  \texttt{WebScreenletConfigurationBuilder} constructor is a full URL.
  For example, if the full URL is
  \texttt{http://screens.liferay.org.es/web/guest/blog}, then you must
  supply that URL to the constructor.
\end{itemize}

\subsection{Attributes}\label{attributes-42}

\noindent\hrulefill

Attribute \textbar{} Data type \textbar{} Explanation \textbar{}
\texttt{autoLoad} \textbar{} \texttt{boolean} \textbar{} Whether to load
the page automatically when the Screenlet appears in the app's UI. The
default value is \texttt{true}. \textbar{} \texttt{loggingEnabled}
\textbar{} \texttt{boolean} \textbar{} Whether logging is enabled.
\textbar{} \texttt{isScrollEnabled} \textbar{} \texttt{boolean}
\textbar{} Whether to enable scrolling on the page inside the Screenlet.
\textbar{}

\noindent\hrulefill

\subsection{Delegate}\label{delegate-21}

Web Screenlet delegates some events to an object that conforms to the
\texttt{WebScreenletDelegate} protocol. This protocol lets you implement
the following methods:

\begin{itemize}
\item
  \texttt{onWebLoad(\_:url:)}: Called when the Screenlet loads the page.

\begin{verbatim}
func onWebLoad(_ screenlet: WebScreenlet, url: String) {
    ...
}
\end{verbatim}
\item
  \texttt{screenlet(\_:onScriptMessageNamespace:onScriptMessage:)}:
  Called when the \texttt{WKWebView} sends a message.

\begin{verbatim}
func screenlet(_ screenlet: WebScreenlet,
           onScriptMessageNamespace namespace: String,
           onScriptMessage message: String) {
    ...
}
\end{verbatim}
\item
  \texttt{screenlet(\_:onError:)}: Called when an error occurs in the
  process. The \texttt{NSError} object describes the error.

\begin{verbatim}
func screenlet(_ screenlet: WebScreenlet, onError error: NSError) {
    ...
}
\end{verbatim}
\end{itemize}

\section{SyncManagerDelegate}\label{syncmanagerdelegate}

The \texttt{SyncManagerDelegate} class is required to
\href{/docs/7-0/tutorials/-/knowledge_base/t/using-offline-mode-in-ios}{use
Screenlets with offline mode}. This class receives the events produced
in the synchronization process. This document describes the class's
methods.

\subsection{Methods}\label{methods-22}

The following method is invoked when the synchronization process is
started. The number of items to be synced are passed.

\begin{verbatim}
syncManager(manager: SyncManager, itemsCount: UInt)
\end{verbatim}

The following method is invoked when an item synchronization is about to
start.

\begin{verbatim}
syncManager(manager: SyncManager, onItemSyncScreenlet screenlet: String, 
    startKey: String, attributes: [String:AnyObject])
\end{verbatim}

\begin{itemize}
\tightlist
\item
  \texttt{screenlet}: the screenlet name that stored this cache element
\item
  \texttt{startKey}: the cache key where the item is stored
\item
  \texttt{attributes}: some attributes stored together with the element.
  The specific attributes depend on the type of the entry. For more
  details, read the screenlet reference documentation.
\end{itemize}

The following method is invoked when an item synchronization is
successfully completed.

\begin{verbatim}
syncManager(manager: SyncManager, onItemSyncScreenlet screenlet: String, 
    completedKey: String, attributes: [String:AnyObject])
\end{verbatim}

\begin{itemize}
\tightlist
\item
  \texttt{screenlet}: the screenlet name that stored this cache element
\item
  \texttt{completedKey}: the cache key where the item is stored
\item
  \texttt{attributes}: some attributes stored together with the element.
  The specific attributes depend on the type of the entry. For more
  details, read the screenlet reference documentation.
\end{itemize}

The following method is invoked when an item synchronization fails.

\begin{verbatim}
syncManager(manager: SyncManager, onItemSyncScreenlet screenlet: String, 
    failedKey: String, attributes: [String:AnyObject], error: NSError)
\end{verbatim}

\begin{itemize}
\tightlist
\item
  \texttt{screenlet}: the screenlet name that stored this cache element
\item
  \texttt{failedKey}: the cache key where the item is stored
\item
  \texttt{attributes}: some attributes stored together with the element.
  The specific attributes will depend on the type of the entry. For more
  details, read the screenlet reference documentation.
\item
  \texttt{error}: the error occurred in the synchronization
\end{itemize}

The following method is invoked when an item synchronization detects a
conflict. The method must invoke asynchronously a
\href{https://en.wikipedia.org/wiki/Continuation-passing_style}{continuation}
argument with the conflict action result.

\begin{verbatim}
syncManager(manager: SyncManager, onItemSyncScreenlet screenlet: String, 
    conflictedKey: String, remoteValue: AnyObject, localValue: AnyObject, 
    resolve: SyncConflictResolution -> ())
\end{verbatim}

\begin{itemize}
\tightlist
\item
  \texttt{screenlet}: the screenlet name that stored this cache element
\item
  \texttt{conflictedKey}: the cache key where the item is stored
\item
  \texttt{remoteValue}: the value stored in the server for the item
  being synchronized
\item
  \texttt{localValue}: the value stored in the cache for the item being
  synchronized
\item
  \texttt{resolve}: this is the continuation function to be called with
  the action result.
\end{itemize}

Supported values for \texttt{resolve} are:

\begin{itemize}
\tightlist
\item
  \texttt{UseRemote}: the remote version is overwritten with the local
  one. Both the local cache and the portal have the same version.
\item
  \texttt{UseLocal}: the local version is overwritten with the remote
  one. Both the local cache and the portal have the same version
\item
  \texttt{Discard}: the local version is removed and the remote one
  isn't overwritten.
\item
  \texttt{Ignore}: data is not changed, so the next synchronization will
  detect the conflict again.
\end{itemize}

\chapter{Liferay Faces}\label{liferay-faces}

Liferay Faces is an umbrella project that provides support for the
JavaServer™ Faces (JSF) standard within Liferay Portal. It encompasses
the following projects:

\begin{itemize}
\tightlist
\item
  \textbf{Liferay Faces Bridge} enables you to deploy JSF web apps as
  portlets without writing portlet-specific Java code. It also contains
  innovative features that make it possible to leverage the power of JSF
  2.x inside a portlet application. Liferay Faces Bridge implements the
  JSR 329 Portlet Bridge Standard.
\item
  \textbf{Liferay Faces Alloy} enables you to use AlloyUI components in
  a way that is consistent with JSF development.
\item
  \textbf{Liferay Faces Portal} enables you to leverage Liferay-specific
  utilities and UI components in JSF portlets.
\end{itemize}

In this section of reference documentation, you'll learn more about each
of these projects. You'll also learn about the Liferay Faces version
scheme.

\section{Liferay Faces Version
Scheme}\label{liferay-faces-version-scheme}

In this article, you'll learn which Liferay Faces artifacts should be
used with your portlet and explore the Liferay Faces versioning scheme
by discovering what each component of a version means. Once you have the
versioning scheme mastered, you can view several example configurations.

\subsection{Using The Liferay Faces Archetype
Portlet}\label{using-the-liferay-faces-archetype-portlet}

The \href{http://liferayfaces.org}{Liferay Faces Archetype portlet} can
be used to determine the Liferay Faces artifacts and versions that you
must include in your portlet. Select your preferred Liferay Portal
version, JSF version, component suite (optional), and build tool, and
the portlet will provide you with both a command to generate your
portlet from a Maven archetype and a list of dependencies that can be
copied into your build files. In the next section, you'll be provided
with compatibility information about each version of the Liferay Faces
artifacts.

\subsection{Liferay Faces Alloy}\label{liferay-faces-alloy}

Provides a suite of JSF components that utilize
\href{http://alloyui.com/}{AlloyUI}.

\noindent\hrulefill

Branch\textbar Example Artifact\textbar AlloyUI\textbar JSF
API\textbar Additional Info\textbar{}
\href{https://github.com/liferay/liferay-faces-alloy/tree/master}{master
(3.x)}\textbar com.liferay.faces.alloy-3.0.1.jar\textbar3.0.x\textbar2.2+\textbar{}\emph{AlloyUI
3.0.x is the version that comes bundled with Liferay Portal
7.0+.}\textbar{}
\href{https://github.com/liferay/liferay-faces-alloy/tree/2.x}{2.x}\textbar com.liferay.faces.alloy-2.0.1.jar\textbar2.0.x\textbar2.1+\textbar{}\emph{AlloyUI
2.0.x is the version that comes bundled with Liferay Portal
6.2.}\textbar{}
\href{https://github.com/liferay/liferay-faces-alloy/tree/1.x}{1.x}\textbar com.liferay.faces.alloy-1.0.1.jar\textbar2.0.x\textbar1.2\textbar{}\emph{AlloyUI
2.0.x is the version that comes bundled with Liferay Portal
6.2.}\textbar{}

\noindent\hrulefill

\subsection{Liferay Faces Bridge}\label{liferay-faces-bridge}

Provides the ability to deploy JSF web applications as portlets within
\href{https://portals.apache.org/pluto/}{Apache Pluto}, the reference
implementation for JSR 286 (Portlet 2.0) and JSR 362 (Portlet 3.0).

\noindent\hrulefill

Branch\textbar Example Artifacts\textbar Portlet API\textbar JSF
API\textbar JCP Specification\textbar Additional Info\textbar{} API:
\href{https://github.com/liferay/liferay-faces-bridge-api/tree/5.x}{5.x}IMPL:
\href{https://github.com/liferay/liferay-faces-bridge-impl/tree/5.x}{5.x}\textbar com.liferay.faces.bridge.api-5.0.0.jarcom.liferay.faces.bridge.impl-5.0.0.jar\textbar3.0\textbar2.2\textbar{}\href{https://www.jcp.org/en/jsr/detail?id=378}{JSR
378}\textbar{}\emph{The Expert Group began work in September 2015 and
the Specification is currently under development.}\textbar{} API:
\href{https://github.com/liferay/liferay-faces-bridge-api/tree/4.x}{4.x}IMPL:
\href{https://github.com/liferay/liferay-faces-bridge-impl/tree/4.x}{4.x}\textbar com.liferay.faces.bridge.api-4.1.0.jarcom.liferay.faces.bridge.impl-4.0.0.jar\textbar2.0\textbar2.2\textbar{}\href{https://www.jcp.org/en/jsr/detail?id=329}{JSR
329}\textbar{}\emph{Includes non-standard bridge extensions for JSF
2.2.}\textbar{} API:
\href{https://github.com/liferay/liferay-faces-bridge-api/tree/3.x}{3.x}IMPL:
\href{https://github.com/liferay/liferay-faces-bridge-impl/tree/3.x}{3.x}\textbar com.liferay.faces.bridge.api-3.1.0.jarcom.liferay.faces.bridge.impl-3.0.0.jar\textbar2.0\textbar2.1\textbar{}\href{https://www.jcp.org/en/jsr/detail?id=329}{JSR
329}\textbar{}\emph{Includes non-standard bridge extensions for JSF
2.1.}\textbar{} API:
\href{https://github.com/liferay/liferay-faces-bridge-api/tree/2.x}{2.x}IMPL:
\href{https://github.com/liferay/liferay-faces-bridge-impl/tree/2.x}{2.x}\textbar com.liferay.faces.bridge.api-2.1.0.jarcom.liferay.faces.bridge.impl-2.0.0.jar\textbar2.0\textbar1.2\textbar{}\href{https://www.jcp.org/en/jsr/detail?id=329}{JSR
329} (MR1)\textbar{}\emph{Includes support for Maintenance Release 1
(MR1).}\textbar{}
1.x\textbar N/A\textbar1.0\textbar1.2\textbar{}\href{https://www.jcp.org/en/jsr/detail?id=301}{JSR
301}\textbar{}\emph{N/A (Not Applicable) since Liferay Faces Bridge has
never implemented JSR 301.}\textbar{}

\noindent\hrulefill

\subsection{Liferay Faces Bridge Ext}\label{liferay-faces-bridge-ext}

Extension to Liferay Faces Bridge that provides compatibility with
\href{https://liferay.dev/-/portal}{Liferay Portal} and also takes
advantage of Liferay-specific features such as friendly URLs.

\noindent\hrulefill

Branch \textbar Example Artifact \textbar~~Liferay Portal
API~~\textbar~~Bridge API~~\textbar~~Portlet API~~\textbar JSF
API\textbar{}
\href{https://github.com/liferay/liferay-faces-bridge-ext/tree/master}{8.x}\textbar com.liferay.faces.bridge.ext-8.0.0.jar\textbar7.3.0+\textbar5.x\textbar3.0\textbar2.3\textbar{}
\href{https://github.com/liferay/liferay-faces-bridge-ext/tree/7.x}{7.x}\textbar com.liferay.faces.bridge.ext-7.0.0.jar\textbar7.3.0+\textbar5.x\textbar3.0\textbar2.2\textbar{}
\href{https://github.com/liferay/liferay-faces-bridge-ext/tree/6.x}{6.x}\textbar com.liferay.faces.bridge.ext-6.0.0.jar\textbar7.3.0+\textbar4.x\textbar2.0\textbar2.2\textbar{}
\href{https://github.com/liferay/liferay-faces-bridge-ext/tree/5.x}{5.x}\textbar com.liferay.faces.bridge.ext-5.0.4.jar\textbar7.0.x/7.1.x/7.2.x\textbar4.x\textbar2.0\textbar2.2\textbar{}
\href{https://github.com/liferay/liferay-faces-bridge-ext/tree/4.x}{4.x}\textbar UNUSED\textbar N/A\textbar N/A\textbar N/A\textbar N/A\textbar{}
\href{https://github.com/liferay/liferay-faces-bridge-ext/tree/3.x}{3.x}\textbar com.liferay.faces.bridge.ext-3.0.1.jar\textbar6.2.x\textbar4.x\textbar2.0\textbar2.2\textbar{}
\href{https://github.com/liferay/liferay-faces-bridge-ext/tree/2.x}{2.x}\textbar com.liferay.faces.bridge.ext-2.0.1.jar\textbar6.2.x\textbar3.x\textbar2.0\textbar2.1\textbar{}
\href{https://github.com/liferay/liferay-faces-bridge-ext/tree/1.x}{1.x}\textbar com.liferay.faces.bridge.ext-1.0.1.jar\textbar6.2.x\textbar2.x\textbar2.0\textbar1.2\textbar{}

\noindent\hrulefill

\subsection{Liferay Faces Portal}\label{liferay-faces-portal}

Provides a suite of JSF components that are based on the JSP tags
provided by \href{https://liferay.dev/-/portal}{Liferay Portal}.

\noindent\hrulefill

Branch\textbar Example Artifact\textbar Liferay Portal
API~~\textbar~~JSF API\textbar{}
\href{https://github.com/liferay/liferay-faces-portal/tree/3.x}{3.x}\textbar com.liferay.faces.portal-3.0.1.jar\textbar7.0.x+\textbar2.2+\textbar{}
\href{https://github.com/liferay/liferay-faces-portal/tree/2.x}{2.x}\textbar com.liferay.faces.portal-2.0.1.jar\textbar6.2.x\textbar2.1+\textbar{}
\href{https://github.com/liferay/liferay-faces-portal/tree/1.x}{1.x}\textbar com.liferay.faces.portal-1.0.1.jar\textbar6.2.x\textbar1.2\textbar{}

\noindent\hrulefill

\subsection{Liferay Faces Util}\label{liferay-faces-util}

Library that contains general purpose JSF utilities to support many of
the sub-projects that comprise Liferay Faces.

\noindent\hrulefill

Branch\textbar Example Artifact\textbar~~JSF API\textbar{}
\href{https://github.com/liferay/liferay-faces-util/tree/4.x}{4.x}\textbar com.liferay.faces.util-3.1.0.jar\textbar2.3\textbar{}
\href{https://github.com/liferay/liferay-faces-util/tree/3.x}{3.x}\textbar com.liferay.faces.util-3.1.0.jar\textbar2.2\textbar{}
\href{https://github.com/liferay/liferay-faces-util/tree/2.x}{2.x}\textbar com.liferay.faces.util-2.1.0.jar\textbar2.1\textbar{}
\href{https://github.com/liferay/liferay-faces-util/tree/1.x}{1.x}\textbar com.liferay.faces.util-1.1.0.jar\textbar1.2\textbar{}

\noindent\hrulefill

Now that you know all about the Liferay Faces versioning scheme, you may
be curious as to how these components interact with each other. Refer to
the following figure to view the Liferay Faces dependency diagram.

\begin{figure}
\centering
\includegraphics{./images/liferay-faces-dependency-diagram.png}
\caption{The Liferay Faces dependency diagram helps visualize how
components interact and depend on each other.}
\end{figure}

Next, you can view some example configurations to see the new versioning
scheme in action.

\section{Understanding Liferay Faces
Bridge}\label{understanding-liferay-faces-bridge}

The Liferay Faces Bridge enables you to deploy JSF web apps as portlets
without writing portlet-specific code. It also contains innovative
features that make it possible to leverage the power of JSF 2.x inside a
portlet application.

Liferay Faces Bridge is distributed in a \texttt{.jar} file. You can add
Liferay Faces Bridge as a dependency to your portlet projects, in order
to deploy your JSF web applications as portlets within JSR 286 (Portlet
2.0) compliant portlet containers, like Liferay Portal 5.2, 6.0, 6.1,
6.2, and 7.0.

The Liferay Faces Bridge project home page can be found
\href{https://web.liferay.com/community/liferay-projects/liferay-faces/bridge}{here}.

To fully understand Liferay Faces Bridge, you must first understand the
portlet bridge standard. Because the Portlet 1.0 and JSF 1.0 specs were
being created at essentially the same time, the Expert Group (EG) for
the JSF specification constructed the JSF framework to be compliant with
portlets. For example, the
\href{http://docs.oracle.com/javaee/7/api/javax/faces/context/ExternalContext.html\#getRequest--}{ExternalContext.getRequest()}
method returns an \texttt{Object} instead of an
\href{http://download.oracle.com/javaee/7/api/javax/servlet/http/HttpServletRequest.html}{javax.servlet.http.HttpServletRequest}.
When this method is used in a portal, the \texttt{Object} can be cast to
a
\href{http://portals.apache.org/pluto/portlet-2.0-apidocs/javax/portlet/PortletRequest.html}{javax.portlet.PortletRequest}.
Despite the EG's consciousness of portlet compatibility within the
design of JSF, the gap between the portlet and JSF lifecycles had to be
bridged.

Portlet bridge standards and implementations evolved over time.

Starting in 2004, several different JSF portlet bridge implementations
were developed in order to provide JSF developers with the ability to
deploy their JSF web apps as portlets. In 2006, the JCP formed the
Portlet Bridge 1.0 (\href{http://www.jcp.org/en/jsr/detail?id=301}{JSR
301}) EG in order to define a standard bridge API, as well as detailed
requirements for bridge implementations. JSR 301 was released in 2010,
targeting Portlet 1.0 and JSF 1.2.

When the Portlet 2.0 (\href{http://www.jcp.org/en/jsr/detail?id=286}{JSR
286}) standard was released in 2008, it became necessary for the JCP to
form the Portlet Bridge 2.0
(\href{http://www.jcp.org/en/jsr/detail?id=329}{JSR 329}) EG. JSR 329
was also released in 2010, targeting Portlet 2.0 and JSF 1.2.

After the \href{http://www.jcp.org/en/jsr/detail?id=314}{JSR 314} EG
released JSF 2.0 in 2009 and JSF 2.1 in 2010, it became evident that a
Portlet Bridge 3.0 standard would be beneficial. In 2015 the JCP formed
\href{http://www.jcp.org/en/jsr/detail?id=378}{JSR 378}) which is
defining a bridge for Portlet 3.0 and JSF 2.2. There are also variants
of \emph{Liferay Faces Bridge} that support Portlet 2.0 and JSF
1.2/2.1/2.2.

Liferay Faces Bridge is the Reference Implementation (RI) of the Portlet
Bridge Standard. It also contains innovative features that make it
possible to leverage the power of JSF 2.x inside a portlet application.

Now that you're familiar with some of the history of the Portlet Bridge
standards, you'll learn about the responsibilities required of the
portlet bridge.

A JSF portlet bridge aligns the correct phases of the JSF lifecycle with
each phase of the portlet lifecycle. For instance, if a browser sends an
HTTP GET request to a portal page with a JSF portlet in it, the
\texttt{RENDER\_PHASE} is perfomed in the portlet's lifecycle. The JSF
portlet bridge then initiates the \texttt{RESTORE\_VIEW} and
\texttt{RENDER\_RESPONSE} phases in the JSF lifecycle. Likewise, when an
HTTP POST is executed on a portlet and the portlet enters the
\texttt{ACTION\_PHASE}, then the full JSF lifecycle is initiated by the
bridge.

\begin{figure}
\centering
\includegraphics{./images/lifecycle-bridge.png}
\caption{The different phases of the JSF Lifecycle are executed
depending on which phase of the Portlet lifecycle is being executed.}
\end{figure}

Besides ensuring that the two lifecycles connect correctly, the JSF
portlet bridge also acts as a mediator between the portal URL generator
and JSF navigation rules. JSF portlet bridges ensure that URLs created
by the portal comply with JSF navigation rules, so that a JSF portlet is
able to switch to different views.

The JSR 329/378 standards defines several configuration options prefixed
with the \texttt{javax.portlet.faces} namespace. Liferay Faces Bridge
defines additional implementation-specific options prefixed with the
\texttt{com.liferay.faces.bridge} namespace.

Liferay Faces Bridge is an essential part of the JSF development process
for Liferay DXP.

\subsection{Related Topics}\label{related-topics}

\href{/docs/7-0/reference/-/knowledge_base/r/understanding-liferay-faces-alloy}{Understanding
Liferay Faces Alloy}

\href{/docs/7-0/reference/-/knowledge_base/r/understanding-liferay-faces-portal}{Understanding
Liferay Faces Portal}

\href{/docs/7-0/tutorials/-/knowledge_base/t/what-is-service-builder}{What
is Service Builder?}

\section{Understanding Liferay Faces
Alloy}\label{understanding-liferay-faces-alloy}

Liferay Faces Alloy is distributed in a \texttt{.jar} file. You can add
Liferay Faces Alloy as a dependency to your portlet projects, in order
to use AlloyUI in a way that is consistent with JSF development.

During the creation of a JSF portlet in Liferay IDE/Developer Studio,
you have the option of choosing the portlet's JSF Component Suite. The
options include \emph{JSF standard},
\href{http://www.icesoft.org/java/projects/ICEfaces/overview.jsf}{\emph{ICEfaces}},
\href{http://primefaces.org/}{\emph{PrimeFaces}},
\href{http://richfaces.jboss.org/}{\emph{RichFaces}}, and \emph{Liferay
Faces Alloy}.

If you selected the Liferay Faces Alloy JSF Component Suite during your
portlet's setup, the \texttt{.jar} file is included in your portlet
project.

The Liferay Faces Alloy project provides a set of UI components that
utilize AlloyUI. For example, a brief list of some of the supported
\texttt{aui:} tags are listed below:

\begin{itemize}
\tightlist
\item
  Input: \texttt{alloy:inputText}, \texttt{alloy:inputDate},
  \texttt{alloy:inputFile}
\item
  Panel: \texttt{alloy:accordion}, \texttt{alloy:column},
  \texttt{alloy:fieldset}, \texttt{alloy:row}
\item
  Select: \texttt{alloy:selectOneMenu}, \texttt{alloy:selectOneRadio},
  \texttt{alloy:selectStarRating}
\end{itemize}

If you want to utilize Liferay's AlloyUI technology based on YUI3, you
must include the Liferay Faces Alloy \texttt{.jar} file in your JSF
portlet project. If you selected \emph{Liferay Faces Alloy} during your
JSF portlet's setup, you have Liferay Faces Alloy preconfigured in your
project, so you're automatically able to use the \texttt{alloy:} tags.

As you can see, it's extremely easy to configure your JSF application to
use Liferay's AlloyUI tags.

\subsection{Related Topics}\label{related-topics-1}

\href{/docs/7-0/tutorials/-/knowledge_base/t/creating-a-jsf-project-manually}{Creating
a JSF Project Manually}

\href{/docs/7-0/reference/-/knowledge_base/r/understanding-liferay-faces-bridge}{Understanding
Liferay Faces Bridge}

\href{/docs/7-0/reference/-/knowledge_base/r/understanding-liferay-faces-portal}{Understanding
Liferay Faces Portal}

\section{Understanding Liferay Faces
Portal}\label{understanding-liferay-faces-portal}

\emph{Liferay Faces Portal} is distributed in a \texttt{.jar} file. You
can add Liferay Faces Portal as a dependency for your portlet projects
to use its Liferay-specific utilities and UI components. When Liferay
Faces Portal is included in a JSF portlet project, the
\texttt{com.liferay.faces.portal.{[}version{]}.jar} file resides in the
portlet's library.

\begin{figure}
\centering
\includegraphics{./images/jsf-jars-package-explorer.png}
\caption{The required \texttt{.jar} files are downloaded for your JSF
portlet based on the JSF UI Component Suite you configured.}
\end{figure}

Some of the features included in Liferay Faces Portal are:

\begin{itemize}
\item
  Utilities: Provides the \texttt{LiferayPortletHelperUtil} which
  contains a variety Portlet-API and Liferay-specific convenience
  methods.
\item
  JSF Components: Provides a set of JSF equivalents for popular Liferay
  DXP JSP tags (not exhaustive):

  \begin{itemize}
  \tightlist
  \item
    \texttt{liferay-ui:captcha} → \texttt{portal:captcha}
  \item
    \texttt{liferay-ui:input-editor} → \texttt{portal:inputRichText}
  \item
    \texttt{liferay-ui:search} → \texttt{portal:inputSearch}
  \item
    \texttt{liferay-ui:header} → \texttt{portal:header}
  \item
    \texttt{aui:nav} → \texttt{portal:nav}
  \item
    \texttt{aui:nav-item} → \texttt{portal:navItem}
  \item
    \texttt{aui:nav-bar} → \texttt{portal:navBar}
  \item
    \texttt{liferay-security:permissionsURL} →
    \texttt{portal:permissionsURL}
  \item
    \texttt{liferay-portlet:runtime} → \texttt{portal:runtime}
  \end{itemize}

  For more information, visit
  \url{https://liferayfaces.org/web/guest/portal-showcase}.
\item
  Expression Language: Adds a set of EL keywords such as
  \texttt{liferay} for getting Liferay-specific info, and \texttt{i18n}
  for integration with out-of-the-box Liferay internationalized
  messages.
\end{itemize}

Great! You now have an understanding of what Liferay Faces Portal is,
and what it accomplishes in your JSF application.

\subsection{Related Topics}\label{related-topics-2}

\href{/docs/7-0/tutorials/-/knowledge_base/t/creating-a-jsf-project-manually}{Creating
a JSF Project Manually}

\href{/docs/7-0/tutorials/-/knowledge_base/t/customizing-liferay-search}{Customizing
Liferay Search}

\href{/docs/7-0/reference/-/knowledge_base/r/understanding-liferay-faces-alloy}{Understanding
Liferay Faces Alloy}

\chapter{Gradle}\label{gradle}

Liferay provides plugins that you can apply to your Gradle project. This
reference documentation describes how to apply and use Liferay's Gradle
plugins.

\textbf{Important:} If you're using
\href{/docs/7-0/tutorials/-/knowledge_base/t/liferay-workspace}{Liferay
Workspace} to create Liferay apps, many Liferay Gradle plugins are
already applied by default. The \texttt{com.liferay.workspace} plugin
provides the following plugins to all your apps in a Liferay Workspace:

\begin{itemize}
\tightlist
\item
  \texttt{com.liferay.css.builder}
\item
  \texttt{com.liferay.js.module.config.generator}
\item
  \texttt{com.liferay.js.transpiler}
\item
  \texttt{com.liferay.javadoc.formatter}
\item
  \texttt{com.liferay.jspc}
\item
  \texttt{com.liferay.lang.builder}
\item
  \texttt{com.liferay.source.formatter}
\item
  \texttt{com.liferay.soy}
\item
  \texttt{com.liferay.soy.translation}
\item
  \texttt{com.liferay.tlddoc.builder}
\item
  \texttt{com.liferay.tld.formatter}
\item
  \texttt{com.liferay.test.integration}
\item
  \texttt{com.liferay.xml.formatter}
\end{itemize}

Do not apply a Liferay Gradle plugin to an app that already has access
to it.

Each article in this section describes how to apply the plugin, what
Gradle tasks the plugin provides, the plugin's configuration properties,
and the plugin's dependencies.

\section{App Javadoc Builder Gradle
Plugin}\label{app-javadoc-builder-gradle-plugin}

The App Javadoc Builder Gradle plugin lets you generate API
documentation as a single, combined HTML document for an application
that spans different subprojects, each one representing a different
component of the same application.

The plugin has been successfully tested with Gradle 4.10.2.

\subsection{Usage}\label{usage}

To use the plugin, include it in the build script of the root project:

\begin{verbatim}
buildscript {
    dependencies {
        classpath group: "com.liferay", name: "com.liferay.gradle.plugins.app.javadoc.builder", version: "1.2.2"
    }

    repositories {
        maven {
            url "https://repository-cdn.liferay.com/nexus/content/groups/public"
        }
    }
}

apply plugin: "com.liferay.app.javadoc.builder"
\end{verbatim}

The App Javadoc Builder plugin automatically applies the
\href{https://docs.gradle.org/current/userguide/standard_plugins.html\#N135C1}{\texttt{base}}
and \texttt{reporting-base} plugins.

\subsection{Project Extension}\label{project-extension}

The App Javadoc Builder plugin exposes the following properties through
the extension named \texttt{appJavadocBuilder}:

Property Name \textbar{} Type \textbar{} Default Value \textbar{}
Description \texttt{copyTags} \textbar{} \texttt{boolean} \textbar{}
\texttt{true} \textbar{} Whether to copy the custom block tags
configuration from the subprojects. It sets the Javadoc
\href{http://docs.oracle.com/javase/8/docs/technotes/tools/windows/javadoc.html\#tag}{\texttt{-tag}}
argument for the \hyperref[appjavadoc]{\texttt{appJavadoc}} task.
\texttt{doclintDisabled} \textbar{} \texttt{boolean} \textbar{}
\texttt{true} on JDK8+, \texttt{false} otherwise. \textbar{} Whether to
ignore Javadoc errors. It sets the Javadoc
\href{docs.oracle.com/javase/8/docs/technotes/tools/windows/javadoc.html\#BEJEFABE}{\texttt{-Xdoclint}}
and
\href{http://docs.oracle.com/javase/8/docs/technotes/tools/windows/javadoc.html\#CHDGFHAA}{\texttt{-quiet}}
arguments for the \hyperref[appjavadoc]{\texttt{appJavadoc}} task.
\texttt{groupNameClosure} \textbar{}
\texttt{Closure\textless{}String\textgreater{}} \textbar{} The
subproject's description, or the subproject's name if the description is
empty. \textbar{} The closure invoked in order to get the group heading
for a subproject. The given closure is passed a
\href{https://docs.gradle.org/current/javadoc/org/gradle/api/Project.html}{\texttt{Project}}
as its parameter. If \texttt{groupPackages} is \texttt{false}, this
property is not used. \texttt{groupPackages} \textbar{} \texttt{boolean}
\textbar{} \texttt{true} \textbar{} Whether to separate packages on the
overview page based on the subprojects they belong to. It sets the
\href{docs.oracle.com/javase/8/docs/technotes/tools/unix/javadoc.html\#CHDIGGII}{\texttt{-group}}
argument for the \hyperref[appjavadoc]{\texttt{appJavadoc}} task.
\texttt{subprojects} \textbar{}
\texttt{Set\textless{}Project\textgreater{}} \textbar{}
\texttt{project.subprojects} \textbar{} The subprojects to include in
the API documentation of the app.

The same extension exposes the following methods:

Method \textbar{} Description
\texttt{AppJavadocBuilderExtension\ onlyIf(Closure\textless{}Boolean\textgreater{}\ onlyIfClosure)}
\textbar{} Includes a subproject in the API documentation if the given
closure returns \texttt{true}. The closure is evaluated at the end of
the subproject configuration phase and is passed a single parameter: the
subproject. If the closure returns \texttt{false}, the subproject is not
included in the API documentation.
\texttt{AppJavadocBuilderExtension\ onlyIf(Spec\textless{}Project\textgreater{}\ onlyIfSpec)}
\textbar{} Includes a subproject in the API documentation if the given
spec is satisfied. The spec is evaluated at the end of the subproject
configuration phase. If the spec is not satisfied, the subproject is not
included in the API documentation.
\texttt{AppJavadocBuilderExtension\ subprojects(Iterable\textless{}Project\textgreater{}\ subprojects)}
\textbar{} Include additional projects in the API documentation of the
app.
\texttt{AppJavadocBuilderExtension\ subprojects(Project...\ subprojects)}
\textbar{} Include additional projects in the API documentation of the
app.

\subsection{Tasks}\label{tasks}

The plugin adds two tasks to your project:

Name \textbar{} Depends On \textbar{} Type \textbar{} Description
\texttt{appJavadoc} \textbar{} The \texttt{javadoc} tasks of the
subprojects. \textbar{}
\href{https://docs.gradle.org/current/dsl/org.gradle.api.tasks.javadoc.Javadoc.html}{\texttt{Javadoc}}
\textbar{} Generates Javadoc API documentation for the app.
\texttt{jarAppJavadoc} \textbar{} \texttt{appJavadoc} \textbar{}
\href{https://docs.gradle.org/current/dsl/org.gradle.api.tasks.bundling.Jar.html}{\texttt{Jar}}
\textbar{} Assembles a JAR archive containing the Javadoc files for this
app.

The \texttt{appJavadoc} task is automatically configured with sensible
defaults:

Property Name \textbar{} Default Value
\href{https://docs.gradle.org/current/dsl/org.gradle.api.tasks.javadoc.Javadoc.html\#org.gradle.api.tasks.javadoc.Javadoc:classpath}{\texttt{classpath}}
\textbar{} The \texttt{javadoc.classpath} of all the subprojects.
\href{https://docs.gradle.org/current/dsl/org.gradle.api.tasks.javadoc.Javadoc.html\#org.gradle.api.tasks.javadoc.Javadoc:destinationDir}{\texttt{destinationDir}}
\textbar{} \texttt{\$\{project.buildDir\}/docs/javadoc}
\href{https://docs.gradle.org/current/javadoc/org/gradle/external/javadoc/MinimalJavadocOptions.html\#getEncoding()}{\texttt{options.encoding}}
\textbar{} \texttt{"UTF-8"}
\href{https://docs.gradle.org/current/dsl/org.gradle.api.tasks.javadoc.Javadoc.html\#org.gradle.api.tasks.javadoc.Javadoc:source}{\texttt{source}}
\textbar{} The \texttt{javadoc.source} of all the subprojects.
\href{https://docs.gradle.org/current/dsl/org.gradle.api.tasks.javadoc.Javadoc.html\#org.gradle.api.tasks.javadoc.Javadoc:title}{\texttt{title}}
\textbar{} \texttt{project.reporting.apiDocTitle}

\section{Baseline Gradle Plugin}\label{baseline-gradle-plugin}

The Baseline Gradle plugin lets you verify that the OSGi
\href{http://semver.org/}{semantic versioning} rules are obeyed by your
OSGi bundle.

When you run the \hyperref[baseline]{\texttt{baseline}} task, the plugin
\emph{baselines} the new bundle against the latest released non-snapshot
bundle (i.e., the \emph{baseline}). That is, it compares the public
exported API of the new bundle with the baseline. If there are any
changes, it uses the OSGi semantic versioning rules to calculate the
minimum new version. If the new bundle has a lower version, errors are
thrown.

The plugin has been successfully tested with Gradle 4.10.2.

\subsection{Usage}\label{usage-1}

To use the plugin, include it in your build script:

\begin{verbatim}
buildscript {
    dependencies {
        classpath group: "com.liferay", name: "com.liferay.gradle.plugins.baseline", version: "2.1.0"
    }

    repositories {
        maven {
            url "https://repository-cdn.liferay.com/nexus/content/groups/public"
        }
    }
}

apply plugin: "com.liferay.baseline"
\end{verbatim}

The Baseline plugin automatically applies the
\href{https://docs.gradle.org/current/userguide/java_plugin.html}{\texttt{java}}
and
\href{https://docs.gradle.org/current/userguide/standard_plugins.html\#sec:base_plugins}{\texttt{reporting-base}}
plugins.

Since the plugin needs to download the baseline, you have to configure a
\href{https://docs.gradle.org/current/userguide/artifact_dependencies_tutorial.html\#sec:repositories_tutorial}{repository}
that hosts it; for example, the central Maven 2 repository:

\begin{verbatim}
repositories {
    mavenCentral()
}
\end{verbatim}

\subsection{Project Extension}\label{project-extension-1}

The Baseline plugin exposes the following properties through the
\texttt{baselineConfiguration} extension:

Property Name \textbar{} Type \textbar{} Default Value \textbar{}
Description \texttt{allowMavenLocal} \textbar{} \texttt{boolean}
\textbar{} \texttt{false} \textbar{} Whether to let the baseline come
from the local Maven cache (by default: \texttt{\$\{user.home\}/.m2}).
If the local Maven cache is not
\href{https://docs.gradle.org/current/userguide/dependency_management.html\#sub:maven_local}{configured}
as a project repository, this property has no effect.
\texttt{lowestBaselineVersion} \textbar{} \texttt{String} \textbar{}
\texttt{"1.0.0"} \textbar{} The greatest project version to ignore for
the baseline check. If the
\href{https://docs.gradle.org/current/dsl/org.gradle.api.tasks.bundling.Jar.html\#org.gradle.api.tasks.bundling.Jar:version}{project
version} is less than or equal to the value of this property, the
\hyperref[baseline]{\texttt{baseline}} task is skipped.
\texttt{lowestMajorVersion} \textbar{} \texttt{Integer} \textbar{}
Content of the file
\texttt{\$\{project.projectDir\}/.lfrbuild-lowest-major-version}, where
the default file name can be changed by setting the project property
\texttt{baseline.lowest.major.version.file}. \textbar{} The lowest major
version of the released artifact to use in the baseline check.
\texttt{lowestMajorVersionRequired} \textbar{} \texttt{boolean}
\textbar{} \texttt{false} \textbar{} Whether to fail the build if the
\hyperref[lowestmajorversion]{\texttt{lowestMajorVersion}} is not
specified.

If the \texttt{lowestMajorVersion} is not specified, the plugin runs the
check using the most recent released non-snapshot bundle as baseline,
which matches the
\href{http://ant.apache.org/ivy/history/latest-milestone/settings/version-matchers.html}{version
range} \texttt{(,\$\{project.version\})}. Otherwise, if the
\texttt{lowestMajorVersion} is equal to a value \texttt{L} and the
project has version \texttt{M.x.y} (with \texttt{L} less or equal than
\texttt{M}), multiple checks are performed in order, using the following
version ranges as baseline:

\begin{enumerate}
\def\labelenumi{\arabic{enumi}.}
\tightlist
\item
  \texttt{{[}L.0.0,\ (L\ +\ 1).0.0)}
\item
  \texttt{{[}(L\ +\ 1).0.0,\ (L\ +\ 2).0.0)}
\item
  \ldots{}
\item
  \texttt{{[}(M\ -\ 2).0.0,\ (M\ -\ 1).0.0)}
\item
  \texttt{{[}(M\ -\ 1).0.0,\ M.0.0)}
\item
  \texttt{{[}M.0.0,\ M.x.y)}
\end{enumerate}

The first failing check fails the whole build.

\subsection{Tasks}\label{tasks-1}

The plugin adds one task to your project:

Name \textbar{} Depends On \textbar{} Type \textbar{} Description
\texttt{baseline} \textbar{}
\href{(https://docs.gradle.org/current/userguide/java_plugin.html\#sec:jar)}{\texttt{jar}}
\textbar{} \hyperref[baselinetask]{\texttt{BaselineTask}} \textbar{}
Compares the public API of this project with the public API of the
previous released version, if found.

The \texttt{baseline} task is automatically configured with sensible
defaults:

Property Name \textbar{} Default Value
\hyperref[baselineconfiguration]{\texttt{baselineConfiguration}}
\textbar{}
\hyperref[baseline-dependency]{\texttt{configurations.baseline}}
\hyperref[bndfile]{\texttt{bndFile}} \textbar{}
\texttt{\$\{project.projectDir\}/bnd.bnd}
\hyperref[newjarfile]{\texttt{newJarFile}} \textbar{}
\href{https://docs.gradle.org/current/dsl/org.gradle.api.tasks.bundling.Jar.html\#org.gradle.api.tasks.bundling.Jar:archivePath}{\texttt{project.tasks.jar.archivePath}}
\hyperref[sourcedir]{\texttt{sourceDir}} \textbar{} The first
\texttt{resources} directory of the \texttt{main} source set (by
default: \texttt{src/main/resources}).

\subsubsection{BaselineTask}\label{baselinetask}

\paragraph{Task Properties}\label{task-properties}

Property Name \textbar{} Type \textbar{} Default Value \textbar{}
Description \texttt{baselineConfiguration} \textbar{}
\texttt{Configuration} \textbar{} \texttt{null} \textbar{} The
configuration that contains exactly one dependency to the baseline
bundle. \texttt{bndFile} \textbar{} \texttt{File} \textbar{}
\texttt{null} \textbar{} The BND file of the project. If provided, the
task will automatically update the
\href{http://bnd.bndtools.org/heads/bundle_version.html}{\texttt{Bundle-Version}}
header. \texttt{forceCalculatedVersion} \textbar{} \texttt{boolean}
\textbar{} \texttt{false} \textbar{} Whether to fail the baseline check
if the \texttt{Bundle-Version} has been excessively increased.
\texttt{ignoreExcessiveVersionIncreases} \textbar{} \texttt{boolean}
\textbar{} \texttt{false} \textbar{} Whether to ignore excessive package
version increase warnings. \texttt{ignoreFailures} \textbar{}
\texttt{boolean} \textbar{} \texttt{false} \textbar{} Whether the build
should not break when semantic versioning errors are found.
\texttt{logFile} \textbar{} \texttt{File} \textbar{} \texttt{null}
\textbar{} The file to which the results of the baseline check are
written. \emph{(Read-only)} \texttt{logFileName} \textbar{}
\texttt{String} \textbar{} \texttt{"baseline/\$\{task.name\}.log"}
\textbar{} The name of the file to which the results of the baseline
check are written. If the \texttt{reporting-base} plugin is applied, the
file name is relative to
\href{https://docs.gradle.org/current/dsl/org.gradle.api.reporting.ReportingExtension.html\#org.gradle.api.reporting.ReportingExtension:baseDir}{\texttt{reporting.baseDir}};
otherwise, it's relative to the project directory. \texttt{newJarFile}
\textbar{} \texttt{File} \textbar{} \texttt{null} \textbar{} The file of
the new OSGi bundle. \texttt{reportDiff} \textbar{} \texttt{boolean}
\textbar{} \texttt{true} if the project property
\texttt{baseline.jar.report.level} has either value \texttt{"diff"} or
\texttt{"persist"}; \texttt{false} otherwise \textbar{} Whether to show
a granular, differential report of all changes that occurred in the
exported packages of the OSGi bundle. \texttt{reportOnlyDirtyPackages}
\textbar{} \texttt{boolean} \textbar{} Value of the project property
\texttt{baseline.jar.report.only.dirty.packages} if specified;
\texttt{true} otherwise. \textbar{} Whether to show only packages with
API changes in the report. \texttt{sourceDir} \textbar{} \texttt{File}
\textbar{} \texttt{null} \textbar{} The directory to which the
\href{http://bnd.bndtools.org/chapters/170-versioning.html\#versioning-packages}{\texttt{packageinfo}}
files are generated or updated.

The properties of type \texttt{File} support any type that can be
resolved by
\href{https://docs.gradle.org/current/dsl/org.gradle.api.Project.html\#org.gradle.api.Project:file(java.css.Object)}{\texttt{project.file}}.
Moreover, it is possible to use Closures and Callables as values for the
\texttt{String} properties to defer evaluation until task execution.

\subsubsection{Helper Tasks}\label{helper-tasks}

If the \hyperref[lowestmajorversion]{\texttt{lowestMajorVersion}}
property is specified with a value \texttt{L}, the plugin creates a
series of helper tasks of type
\hyperref[baselinetask]{\texttt{BaselineTask}} at the end of the
\href{https://docs.gradle.org/current/userguide/build_lifecycle.html\#N11BAE}{project
evaluation}, one for each major version between \texttt{L} and the major
version \texttt{M} of the project:

\begin{enumerate}
\def\labelenumi{\arabic{enumi}.}
\tightlist
\item
  Task \texttt{baseline\$\{L\ +\ 1\}}, which depends on
  \texttt{baseline\$\{L\ +\ 2\}} and uses the version range
  \texttt{{[}(L\ +\ 1).0.0,\ (L\ +\ 2).0.0)} as baseline.
\item
  Task \texttt{baseline\$\{L\ +\ 2\}}, which depends on
  \texttt{baseline\$\{L\ +\ 3\}} and uses the version range
  \texttt{{[}(L\ +\ 2).0.0,\ (L\ +\ 3).0.0)} as baseline.
\item
  \ldots{}
\item
  Task \texttt{baseline\$\{M\ -\ 2\}}, which depends on
  \texttt{baseline\$\{M\ -\ 1\}} and uses the version range
  \texttt{{[}(M\ -\ 2).0.0,\ (M\ -\ 1).0.0)} as baseline.
\item
  Task \texttt{baseline\$\{M\ -\ 1\}}, which depends on
  \texttt{baseline\$\{M\}} and uses the version range
  \texttt{{[}(M\ -\ 1).0.0,\ M.0.0)} as baseline.
\item
  Task \texttt{baseline\$\{M\}}, which uses the version range
  \texttt{{[}M.0.0,\ M.x.y)} as baseline.
\end{enumerate}

The \texttt{baseline} task is also configured to use the version range
\texttt{{[}L.0.0,\ (L\ +\ 1).0.0)} as baseline, and to depend on the
task \texttt{baseline\$\{L\ +\ 1\}}. This means that running the
\texttt{baseline} task runs the baseline check against multiple
versions, starting from the most recent \texttt{M} and going back to
\texttt{L}.

Moreover, all tasks except \texttt{baseline\$\{M\}} have the property
\hyperref[ignoreexcessiveversionincreases]{\texttt{ignoreExcessiveVersionIncreases}}
set to \texttt{true}.

\subsection{Additional Configuration}\label{additional-configuration}

There are additional configurations that can help you baseline your OSGi
bundle.

\subsubsection{Baseline Dependency}\label{baseline-dependency}

The plugin creates a configuration called \texttt{baseline} with a
default dependency to a released non-snapshot version of the bundle:

\begin{itemize}
\tightlist
\item
  version range \texttt{{[}L.0.0,\ (L\ +\ 1).0.0)} if the
  \hyperref[lowestmajorversion]{\texttt{lowestMajorVersion}} property is
  specified with a value \texttt{L}.
\item
  version range \texttt{(,\$\{project.version\})} otherwise.
\end{itemize}

It is possible to override this setting and use a different version of
the bundle as baseline.

\subsubsection{System Properties}\label{system-properties}

It is possible to set the default values of the
\hyperref[ignorefailures]{\texttt{ignoreFailures}} property for a
\texttt{BaselineTask} task via system properties:

\begin{verbatim}
-D${task.name}.ignoreFailures=true
\end{verbatim}

For example, run the following Bash command to execute the baseline
check without breaking the build, in case of errors:

\begin{verbatim}
./gradlew baseline -Dbaseline.ignoreFailures=true
\end{verbatim}

\section{Change Log Builder Gradle
Plugin}\label{change-log-builder-gradle-plugin}

The Change Log Builder Gradle plugin lets you generate and maintain a
change log file based on the Git commits in your project. A change log
file generated by this plugin looks like this

\begin{verbatim}
    #
    # Bundle Version 1.0.1
    #
    9c77ff4c95cb1a325db3bdd089be105206e8b63c^..b421f00ac84b065685b131833fecc594fc01c760=LPS-123 LPS-1321

    #
    # Bundle Version 1.0.2
    #
    b421f00ac84b065685b131833fecc594fc01c760^..bc15d8d84e12b9544f78e4e3743c510dbaec2d89=LPS-456
\end{verbatim}

Every time the \hyperref[buildchangelog]{\texttt{buildChangeLog}} task
is executed, a new line is added to the change log, which lists all Git
\hyperref[ticketidprefixes]{commit prefixes} (usually issue ticket IDs)
that occurred in a certain range. The end of the range is always the tip
of the current branch. The start range can vary, depending on the case:

\begin{itemize}
\tightlist
\item
  If \texttt{buildChangeLog} has never been executed for the project,
  the change log does not exist. Therefore, the most recent commit from
  two years ago is used for the range start.
\item
  If a change log already exists for your project, the start range
  begins at the range end of the last line in the change log.
\end{itemize}

The plugin has been successfully tested with Gradle 4.10.2.

\subsection{Usage}\label{usage-2}

To use the plugin, include it in your build script:

\begin{verbatim}
buildscript {
    dependencies {
        classpath group: "com.liferay", name: "com.liferay.gradle.plugins.change.log.builder", version: "1.1.3"
    }

    repositories {
        maven {
            url "https://repository-cdn.liferay.com/nexus/content/groups/public"
        }
    }
}

apply plugin: "com.liferay.change.log.builder"
\end{verbatim}

\subsection{Tasks}\label{tasks-2}

The plugin adds one task to your project:

Name \textbar{} Depends On \textbar{} Type \textbar{} Description
\texttt{buildChangeLog} \textbar{} - \textbar{}
\hyperref[buildchangelogtask]{\texttt{BuildChangeLogTask}} \textbar{}
Builds the change log file for this project.

The \texttt{buildChangeLog} task is automatically configured with
sensible defaults, depending on whether the
\href{https://docs.gradle.org/current/userguide/java_plugin.html}{\texttt{java}}
plugin is applied:

Property Name \textbar{} Default Value
\hyperref[changelogheader]{\texttt{changeLogHeader}} \textbar{}
\texttt{"Bundle\ Version\ \$\{project.version\}"}
\hyperref[changelogfile]{\texttt{changeLogFile}} \textbar{}

\textbf{If the \texttt{java} plugin is applied:} The
\texttt{META-INF/liferay-releng.changelog} file in the first
\texttt{resources} directory of the \texttt{main} source set (by
default, \texttt{src/main/resources/META-INF/liferay-releng.changelog}).

\textbf{Otherwise:}
\texttt{"\$\{project.projectDir\}/liferay-releng.changelog"}

\hyperref[dirs]{\texttt{dirs}} \textbar{}
\texttt{{[}project.projectDir{]}}

\subsubsection{BuildChangeLogTask}\label{buildchangelogtask}

\paragraph{Task Properties}\label{task-properties-1}

Property Name \textbar{} Type \textbar{} Default Value \textbar{}
Description \texttt{changeLogFile} \textbar{} \texttt{File} \textbar{}
\texttt{null} \textbar{} The change log file to build.
\texttt{changeLogHeader} \textbar{} \texttt{String} \textbar{}
\texttt{null} \textbar{} The header for the new line in the change log.
\texttt{dirs} \textbar{} \texttt{FileCollection} \textbar{}
\texttt{{[}{]}} \textbar{} The directories to consider when listing the
commits in the range specified. \texttt{gitDir} \textbar{} \texttt{File}
\textbar{} \texttt{project.rootDir} \textbar{} The base directory to
start searching for the \texttt{.git} directory. The search proceeds in
all the ancestors of the directory specified. \texttt{rangeEnd}
\textbar{} \texttt{String} \textbar{} \texttt{null} \textbar{} The hash
of the last commit to consider. If not set, it corresponds to the range
end of the last line in the change log, or the most recent commit from
at least two years ago if the change log file does not exist yet.
\texttt{rangeStart} \textbar{} \texttt{String} \textbar{} \texttt{null}
\textbar{} The hash of the first commit to consider. If not set, it
corresponds to the hash of the tip of the current branch.
\texttt{ticketIdPrefixes} \textbar{}
\texttt{Set\textless{}String\textgreater{}} \textbar{}
\texttt{{[}"CLDSVCS",\ "LPS",\ "SOS",\ "SYNC"{]}} \textbar{} The valid
prefix of the Git commit messages to add to the change log. For example,
if a commit message is \texttt{"LPS-123\ Bugfix"}, \texttt{"LPS-123"}
will be added to the change log.

The properties of type \texttt{File} support any type that can be
resolved by
\href{https://docs.gradle.org/current/dsl/org.gradle.api.Project.html\#org.gradle.api.Project:file(java.css.Object)}{\texttt{project.file}}.
Moreover, it is possible to use Closures and Callables as values for the
\texttt{String} properties to defer evaluation until task execution.

\paragraph{Task Methods}\label{task-methods}

Method \textbar{} Description
\texttt{BuildChangeLogTask\ dirs(Iterable\textless{}?\textgreater{}\ dirs)}
\textbar{} Adds directories to consider when listing the commits in the
range specified. \texttt{BuildChangeLogTask\ dirs(Object...\ dirs)}
\textbar{} Adds directories to consider when listing the commits in the
range specified.
\texttt{BuildChangeLogTask\ ticketIdPrefixes(Iterable\textless{}String\textgreater{}\ ticketIdPrefixes)}
\textbar{} Adds valid prefixes of the Git commit messages to add to the
change log.
\texttt{BuildChangeLogTask\ ticketIdPrefixes(String...\ ticketIdPrefixes)}
\textbar{} Adds valid prefixes of the Git commit messages to add to the
change log.

\section{CSS Builder Gradle Plugin}\label{css-builder-gradle-plugin}

The CSS Builder Gradle plugin lets you run the
\href{https://github.com/liferay/liferay-portal/tree/master/modules/util/css-builder}{Liferay
CSS Builder} tool to compile \href{http://sass-lang.com/}{Sass} files in
your project.

The plugin has been successfully tested with Gradle 4.10.2.

\subsection{Usage}\label{usage-3}

To use the plugin, include it in your build script:

\begin{verbatim}
buildscript {
    dependencies {
        classpath group: "com.liferay", name: "com.liferay.gradle.plugins.css.builder", version: "3.0.0"
    }

    repositories {
        maven {
            url "https://repository-cdn.liferay.com/nexus/content/groups/public"
        }
    }
}

apply plugin: "com.liferay.css.builder"
\end{verbatim}

Since the plugin automatically resolves the Liferay CSS Builder library
as a dependency, you have to configure a repository that hosts the
library and its transitive dependencies. The Liferay CDN repository
hosts them all:

\begin{verbatim}
repositories {
    maven {
        url "https://repository-cdn.liferay.com/nexus/content/groups/public"
    }
}
\end{verbatim}

\subsection{Tasks}\label{tasks-3}

The plugin adds one task to your project:

Name \textbar{} Depends On \textbar{} Type \textbar{} Description
\texttt{buildCSS} \textbar{} - \textbar{}
\hyperref[buildcsstask]{\texttt{BuildCSSTask}} \textbar{} Compiles the
Sass files in this project.

The plugin also adds the following dependencies to tasks defined by the
\href{https://docs.gradle.org/current/userguide/java_plugin.html}{\texttt{java}}
plugin:

Name \textbar{} Depends On \texttt{processResources} \textbar{}
\texttt{buildCSS}

The \texttt{buildCSS} task is automatically configured with sensible
defaults, depending on whether the
\href{https://docs.gradle.org/current/userguide/java_plugin.html}{\texttt{java}}
or the
\href{https://docs.gradle.org/current/userguide/war_plugin.html}{\texttt{war}}
plugins are applied:

Property Name \textbar{} Default Value
\hyperref[basedir]{\texttt{baseDir}} \textbar{}

\textbf{If the \texttt{java} plugin is applied:} The first
\texttt{resources} directory of the \texttt{main} source set (by
default: \texttt{src/main/resources}).

\textbf{If the \texttt{war} plugin is applied:}
\texttt{project.webAppDir}.

\textbf{Otherwise:} \texttt{null}

\subsubsection{BuildCSSTask}\label{buildcsstask}

Tasks of type \texttt{BuildCSSTask} extend
\href{https://docs.gradle.org/current/dsl/org.gradle.api.tasks.JavaExec.html}{\texttt{JavaExec}},
so all its properties and methods, such as
\href{https://docs.gradle.org/current/dsl/org.gradle.api.tasks.JavaExec.html\#org.gradle.api.tasks.JavaExec:args(java.css.Iterable)}{\texttt{args}}
and
\href{https://docs.gradle.org/current/dsl/org.gradle.api.tasks.JavaExec.html\#org.gradle.api.tasks.JavaExec:maxHeapSize}{\texttt{maxHeapSize}},
are available. They also have the following properties set by default:

Property Name \textbar{} Default Value
\href{https://docs.gradle.org/current/dsl/org.gradle.api.tasks.JavaExec.html\#org.gradle.api.tasks.JavaExec:args}{\texttt{args}}
\textbar{} CSS Builder command line arguments
\href{https://docs.gradle.org/current/dsl/org.gradle.api.tasks.JavaExec.html\#org.gradle.api.tasks.JavaExec:classpath}{\texttt{classpath}}
\textbar{}
\hyperref[liferay-css-builder-dependency]{\texttt{project.configurations.cssBuilder}}
\href{https://docs.gradle.org/current/javadoc/org/gradle/api/tasks/JavaExec.html\#setDefaultCharacterEncoding(java.lang.String)}{\texttt{defaultCharacterEncoding}}
\textbar{} \texttt{"UTF-8"}
\href{https://docs.gradle.org/current/dsl/org.gradle.api.tasks.JavaExec.html\#org.gradle.api.tasks.JavaExec:main}{\texttt{main}}
\textbar{} \texttt{"com.liferay.css.builder.CSSBuilder"}
\href{https://docs.gradle.org/current/dsl/org.gradle.api.tasks.JavaExec.html\#org.gradle.api.tasks.JavaExec:systemProperties}{\texttt{systemProperties}}
\textbar{} \texttt{{[}"sass.compiler.jni.clean.temp.dir",\ true{]}}

\paragraph{Task Properties}\label{task-properties-2}

Property Name \textbar{} Type \textbar{} Default Value \textbar{}
Description \texttt{appendCssImportTimestamps} \textbar{}
\texttt{boolean} \textbar{} \texttt{true} \textbar{} Whether to append
the current timestamp to the URLs in the \texttt{@import} CSS at-rules.
It sets the \texttt{sass.append.css.import.timestamps} argument.
\texttt{baseDir} \textbar{} \texttt{File} \textbar{} \texttt{null}
\textbar{} The base directory that contains the SCSS files to compile.
It sets the \texttt{sass.docroot.dir} argument. \texttt{cssFiles}
\textbar{} \texttt{FileCollection} \textbar{} - \textbar{} The SCSS
files to compile. \emph{(Read-only)} \texttt{dirNames} \textbar{}
\texttt{List\textless{}String\textgreater{}} \textbar{}
\texttt{{[}"/"{]}} \textbar{} The name of the directories, relative to
\hyperref[basedir]{\texttt{baseDir}}, which contain the SCSS files to
compile. All sub-directories are searched for SCSS files as well. It
sets the \texttt{sass.dir} argument. \texttt{generateSourceMap}
\textbar{} \texttt{boolean} \textbar{} \texttt{false} \textbar{} Whether
to generate
\href{https://developers.google.com/web/tools/chrome-devtools/debug/readability/source-maps}{source
maps} for easier debugging. It sets the
\texttt{sass.generate.source.map} argument. \texttt{importDir}
\textbar{} \texttt{File} \textbar{} \texttt{null} \textbar{} The
\texttt{META-INF/resources} directory of the
\href{https://github.com/liferay/liferay-portal/tree/master/modules/apps/foundation/frontend-css/frontend-css-common}{Liferay
Frontend Common CSS} artifact. This is required in order to make
\href{http://bourbon.io}{Bourbon} and other CSS libraries available to
the compilation. \texttt{importFile} \textbar{} \texttt{File} \textbar{}
\hyperref[liferay-frontend-common-css-dependency]{\texttt{configurations.portalCommonCSS.singleFile}}
\textbar{} The Liferay Frontend Common CSS JAR file. If
\hyperref[importdir]{\texttt{importDir}} is set, this property has no
effect. \texttt{importPath} \textbar{} \texttt{File} \textbar{} -
\textbar{} The value of the \texttt{importDir} property if set;
otherwise \texttt{importFile}. It sets the
\texttt{sass.portal.common.path} argument. \emph{(Read-only)}
\texttt{outputDirName} \textbar{} \texttt{String} \textbar{}
\texttt{".sass-cache/"} \textbar{} The name of the sub-directories where
the SCSS files are compiled to. For each directory that contains SCSS
files, a sub-directory with this name is created. It sets the
\texttt{sass.output.dir} argument. \texttt{outputDirs} \textbar{}
\texttt{FileCollection} \textbar{} - \textbar{} The directories where
the SCSS files are compiled to. Usually, these directories are ignored
by the Version Control System. \emph{(Read-only)} \texttt{precision}
\textbar{} \texttt{int} \textbar{} \texttt{5} \textbar{} The numeric
precision of numbers in Sass. It sets the \texttt{sass.precision}
argument. \texttt{rtlExcludedPathRegexps} \textbar{}
\texttt{List\textless{}String\textgreater{}} \textbar{} \texttt{{[}{]}}
\textbar{} The SCSS file patterns to exclude when converting for
right-to-left (RTL) support. It sets the
\texttt{sass.rtl.excluded.path.regexps} argument.
\texttt{sassCompilerClassName} \textbar{} \texttt{String} \textbar{}
\texttt{null} \textbar{} The type of Sass compiler to use. Supported
values are \texttt{"jni"} and \texttt{"ruby"}. If not set, defaults to
\texttt{"jni"}. It sets the \texttt{sass.compiler.class.name} argument.

\noindent\hrulefill

\textbf{Note:} Liferay's CSS Builder is supported for Oracle's JDK and
uses a native compiler for increased speed. If you're using an IBM JDK,
you may experience issues when building your Sass files (e.g., when
building a theme). It's recommended to switch to using the Oracle JDK,
but if you prefer using the IBM JDK, you must use the fallback Ruby
compiler. You can do this two ways:

\begin{itemize}
\tightlist
\item
  If you're working in a
  \href{/docs/7-0/tutorials/-/knowledge_base/t/liferay-workspace}{Liferay
  Workspace} or using the
  \href{https://github.com/liferay/liferay-portal/tree/master/modules/sdk/gradle-plugins}{Liferay
  Gradle Plugins} plugin, set \texttt{sass.compiler.class.name=ruby} in
  your \texttt{gradle.properties} file.
\item
  Otherwise, set
  \texttt{buildCSS.sassCompilerClassName=\textquotesingle{}ruby\textquotesingle{}}
  in the project's \texttt{build.gradle} file.
\end{itemize}

The \texttt{sass.compiler.class.name=ruby} Gradle property only works
for modules, so if you're using the Ruby compiler in a WAR project
(e.g., theme), you must use the second option.

Be aware that the Ruby-based compiler doesn't perform as well as the
native compiler, so expect longer compile times.

\noindent\hrulefill

The properties of type \texttt{File} support any type that can be
resolved by
\href{https://docs.gradle.org/current/dsl/org.gradle.api.Project.html\#org.gradle.api.Project:file(java.css.Object)}{\texttt{project.file}}.
Moreover, it is possible to use Closures and Callables as values for the
\texttt{int} and \texttt{String} properties, to defer evaluation until
task execution.

\paragraph{Task Methods}\label{task-methods-1}

Method \textbar{} Description
\texttt{BuildCSSTask\ dirNames(Iterable\textless{}Object\textgreater{}\ dirNames)}
\textbar{} Adds sub-directory names, relative to
\hyperref[basedir]{\texttt{baseDir}}, which contain the SCSS files to
compile. \texttt{BuildCSSTask\ dirNames(Object...\ dirNames)} \textbar{}
Adds sub-directory names, relative to
\hyperref[basedir]{\texttt{baseDir}}, which contain the SCSS files to
compile.
\texttt{BuildCSSTask\ rtlExcludedPathRegexps(Iterable\textless{}Object\textgreater{}\ rtlExcludedPathRegexps)}
\textbar{} Adds SCSS file patterns to exclude when converting for
right-to-left (RTL) support.
\texttt{BuildCSSTask\ rtlExcludedPathRegexps(Object...\ rtlExcludedPathRegexps)}
\textbar{} Adds SCSS file patterns to exclude when converting for
right-to-left (RTL) support.

\subsection{Additional Configuration}\label{additional-configuration-1}

There are additional configurations that can help you use the CSS
Builder.

\subsubsection{Liferay CSS Builder
Dependency}\label{liferay-css-builder-dependency}

By default, the plugin creates a configuration called
\texttt{cssBuilder} and adds a dependency to the latest released version
of the Liferay CSS Builder. It is possible to override this setting and
use a specific version of the tool by manually adding a dependency to
the \texttt{cssBuilder} configuration:

\begin{verbatim}
dependencies {
    cssBuilder group: "com.liferay", name: "com.liferay.css.builder", version: "3.0.0"
}
\end{verbatim}

\subsubsection{Liferay Frontend Common CSS
Dependency}\label{liferay-frontend-common-css-dependency}

By default, the plugin creates a configuration called
\texttt{portalCommonCSS} and adds a dependency to the latest released
version of the Liferay Frontend Common CSS artifact. It is possible to
override this setting and use a specific version of the artifact by
manually adding a dependency to the \texttt{portalCommonCSS}
configuration:

\begin{verbatim}
dependencies {
    portalCommonCSS group: "com.liferay", name: "com.liferay.frontend.css.common", version: "2.0.1"
}
\end{verbatim}

\section{DB Support Gradle Plugin}\label{db-support-gradle-plugin}

The DB Support Gradle plugin lets you run the
\href{https://github.com/liferay/liferay-portal/tree/master/modules/util/portal-tools-db-support}{Liferay
DB Support} tool to execute certain actions on a local Liferay database.
So far, the following actions are available:

\begin{itemize}
\tightlist
\item
  Cleans the Liferay database from the Service Builder tables and rows
  of a module.
\end{itemize}

The plugin has been successfully tested with Gradle 4.10.2.

\subsection{Usage}\label{usage-4}

To use the plugin, include it in your build script:

\begin{verbatim}
buildscript {
    dependencies {
        classpath group: "com.liferay", name: "com.liferay.gradle.plugins.db.support", version: "1.0.5"
    }

    repositories {
        maven {
            url "https://repository-cdn.liferay.com/nexus/content/groups/public"
        }
    }
}

apply plugin: "com.liferay.portal.tools.db.support"
\end{verbatim}

Since the plugin automatically resolves the Liferay DB Support library
as a dependency, you have to configure a repository that hosts the
library and its transitive dependencies. The Liferay CDN repository
hosts them all:

\begin{verbatim}
repositories {
    maven {
        url "https://repository-cdn.liferay.com/nexus/content/groups/public"
    }
}
\end{verbatim}

\subsection{Tasks}\label{tasks-4}

The plugin adds one task to your project:

Name \textbar{} Depends On \textbar{} Type \textbar{} Description
\texttt{cleanServiceBuilder} \textbar{} - \textbar{}
\hyperref[cleanservicebuildertask]{\texttt{CleanServiceBuilderTask}}
\textbar{} Cleans the Liferay database from the Service Builder tables
and rows of a module.

The \texttt{cleanServiceBuilder} task is automatically configured with
sensible defaults, depending on whether the
\href{https://docs.gradle.org/current/userguide/standard_plugins.html\#N135C1}{\texttt{base}}
plugin is applied:

Property Name \textbar{} Default Value
\hyperref[servletcontextname]{\texttt{servletContextName}} \textbar{}

\textbf{If the \texttt{base} plugin is applied:} The bundle symbolic
name of the project inferred via the
\href{https://github.com/gradle/gradle/blob/master/subprojects/osgi/src/main/java/org/gradle/api/internal/plugins/osgi/OsgiHelper.java}{\texttt{OsgiHelper}}
class.

\textbf{Otherwise:} \texttt{null}

\hyperref[servicexmlfile]{\texttt{serviceXmlFile}} \textbar{}
\texttt{"\$\{project.projectDir\}/service.xml"}

\subsubsection{CleanServiceBuilderTask}\label{cleanservicebuildertask}

Tasks of type \texttt{BuildDeploymentHelperTask} extend
\href{https://docs.gradle.org/current/dsl/org.gradle.api.tasks.JavaExec.html}{\texttt{JavaExec}},
so all its properties and methods, such as
\href{https://docs.gradle.org/current/dsl/org.gradle.api.tasks.JavaExec.html\#org.gradle.api.tasks.JavaExec:args(java.lang.Iterable)}{\texttt{args}}
and
\href{https://docs.gradle.org/current/dsl/org.gradle.api.tasks.JavaExec.html\#org.gradle.api.tasks.JavaExec:maxHeapSize}{\texttt{maxHeapSize}},
are available. They also have the following properties set by default:

Property Name \textbar{} Default Value
\href{https://docs.gradle.org/current/dsl/org.gradle.api.tasks.JavaExec.html\#org.gradle.api.tasks.JavaExec:args}{\texttt{args}}
\textbar{} The DB Support command line arguments.
\href{https://docs.gradle.org/current/dsl/org.gradle.api.tasks.JavaExec.html\#org.gradle.api.tasks.JavaExec:classpath}{\texttt{classpath}}
\textbar{}
\hyperref[jdbc-drivers-dependency]{\texttt{project.configurations.dbSupport}}
+
\hyperref[liferay-db-support-dependency]{\texttt{project.configurations.dbSupportTool}}
\href{https://docs.gradle.org/current/dsl/org.gradle.api.tasks.JavaExec.html\#org.gradle.api.tasks.JavaExec:main}{\texttt{main}}
\textbar{} \texttt{"com.liferay.portal.tools.db.support.DBSupport"}

\paragraph{Task Properties}\label{task-properties-3}

Property Name \textbar{} Type \textbar{} Default Value \textbar{}
Description \texttt{password} \textbar{} \texttt{String} \textbar{}
\texttt{null} \textbar{} The user password for connecting to the Liferay
database. It sets the \texttt{-\/-password} argument. If
\hyperref[propertiesfile]{\texttt{propertiesFile}} is set, this property
has no effect. \texttt{propertiesFile} \textbar{} \texttt{File}
\textbar{} \texttt{null} \textbar{} The \texttt{portal-ext.properties}
file that contains the JDBC settings for connecting to the Liferay
database. It sets the \texttt{-\/-properties-file} argument.
\texttt{servletContextName} \textbar{} \texttt{String} \textbar{}
\texttt{null} \textbar{} The servlet context name (usually the value of
the \texttt{Bundle-Symbolic-Name} manifest header) of the module. It
sets the \texttt{-\/-servlet-context-name} argument.
\texttt{serviceXmlFile} \textbar{} \texttt{File} \textbar{}
\texttt{null} \textbar{} The \texttt{service.xml} file of the module. It
sets the \texttt{-\/-service-xml-file} argument. \texttt{url} \textbar{}
\texttt{String} \textbar{} \texttt{null} \textbar{} The JDBC URL for
connecting to the Liferay database. It sets the \texttt{-\/-url}
argument. If \hyperref[propertiesfile]{\texttt{propertiesFile}} is set,
this property has no effect. \texttt{userName} \textbar{}
\texttt{String} \textbar{} \texttt{null} \textbar{} The user name for
connecting to the Liferay database. It sets the \texttt{-\/-user-name}
argument. If \hyperref[propertiesfile]{\texttt{propertiesFile}} is set,
this property has no effect.

The properties of type \texttt{File} support any type that can be
resolved by
\href{https://docs.gradle.org/current/dsl/org.gradle.api.Project.html\#org.gradle.api.Project:file(java.css.Object)}{\texttt{project.file}}.
Moreover, it is possible to use Closures and Callables as values for the
\texttt{int} and \texttt{String} properties to defer evaluation until
task execution.

\subsection{Additional Configuration}\label{additional-configuration-2}

There are additional configurations that can help you use the Deployment
Helper.

\subsubsection{JDBC Drivers Dependency}\label{jdbc-drivers-dependency}

The plugin creates a configuration called \texttt{dbSupport}, which can
be used to provide the suitable JDBC driver for your Liferay database:

\begin{verbatim}
dependencies {
    dbSupport group: "mysql", name: "mysql-connector-java", version: "5.1.23"
    dbSupport group: "org.mariadb.jdbc", name: "mariadb-java-client", version: "1.1.9"
    dbSupport group: "org.postgresql", name: "postgresql", version: "9.4-1201-jdbc41"
}
\end{verbatim}

\subsubsection{Liferay DB Support
Dependency}\label{liferay-db-support-dependency}

By default, the plugin creates a configuration called
\texttt{dbSupportTool} and adds a dependency to the latest released
version of the Liferay DB Support. It is possible to override this
setting and use a specific version of the tool by manually adding a
dependency to the \texttt{dbSupportTool} configuration:

\begin{verbatim}
dependencies {
    dbSupportTool group: "com.liferay", name: "com.liferay.portal.tools.db.support", version: "1.0.8"
}
\end{verbatim}

\section{Dependency Checker Gradle
Plugin}\label{dependency-checker-gradle-plugin}

The Dependency Checker Gradle plugin lets you warn users if a specific
configuration dependency is not the latest one available from the Maven
central repository. The plugin eventually fails the build if the
dependency age (the difference between the timestamp of the current
version and the latest version) is above a predetermined threshold.

The plugin has been successfully tested with Gradle 4.10.2.

\subsection{Usage}\label{usage-5}

To use the plugin, include it in your build script:

\begin{verbatim}
buildscript {
    dependencies {
        classpath group: "com.liferay", name: "com.liferay.gradle.plugins.dependency.checker", version: "1.0.3"
    }

    repositories {
        maven {
            url "https://repository-cdn.liferay.com/nexus/content/groups/public"
        }
    }
}

apply plugin: "com.liferay.dependency.checker"
\end{verbatim}

\subsection{Project Extension}\label{project-extension-2}

The Dependency Checker Gradle plugin exposes the following properties
through the extension named \texttt{dependencyChecker}:

Property Name \textbar{} Type \textbar{} Default Value \textbar{}
Description \texttt{ignoreFailures} \textbar{} \texttt{boolean}
\textbar{} \texttt{true} \textbar{} Whether to print an error message
instead of failing the build when the dependency check fails, either for
a network error or because the dependency is out-of-date.

The same extension exposes the following methods:

Method \textbar{} Description
\texttt{void\ maxAge(Map\textless{}?,\ ?\textgreater{}\ args)}
\textbar{} Declares the max age allowed for a dependency. The
\texttt{args} map must contain the following entries:

\texttt{configuration}: the configuration name

\texttt{group}: the dependency group

\texttt{name}: the dependency name

\texttt{maxAge}: an instance of
\href{http://docs.groovy-lang.org/latest/html/api/groovy/time/Duration.html}{\texttt{groovy.time.Duration}}
that represents the maximum age allowed for the dependency

\texttt{throwError}: a \texttt{boolean} value representing whether to
throw an error if the dependency is out-of-date

\subsection{Additional Configuration}\label{additional-configuration-3}

There are additional configurations that can help you use the Deployment
Helper.

\subsubsection{Project Properties}\label{project-properties}

It is possible to set the default values of the
\hyperref[ignorefailures]{\texttt{ignoreFailures}} property via the
project property \texttt{dependencyCheckerIgnoreFailures}:

\begin{verbatim}
-PdependencyCheckerIgnoreFailures=false
\end{verbatim}

\section{Deployment Helper Gradle
Plugin}\label{deployment-helper-gradle-plugin}

The Deployment Helper Gradle plugin lets you run the
\href{https://github.com/liferay/liferay-portal/tree/master/modules/util/deployment-helper}{Liferay
Deployment Helper} tool to create a cluster deployable WAR from your
OSGi artifacts.

The plugin has been successfully tested with Gradle 4.10.2.

\subsection{Usage}\label{usage-6}

To use the plugin, include it in your build script:

\begin{verbatim}
buildscript {
    dependencies {
        classpath group: "com.liferay", name: "com.liferay.gradle.plugins.deployment.helper", version: "1.0.5"
    }

    repositories {
        maven {
            url "https://repository-cdn.liferay.com/nexus/content/groups/public"
        }
    }
}

apply plugin: "com.liferay.deployment.helper"
\end{verbatim}

Since the plugin automatically resolves the Liferay Deployment Helper
library as a dependency, you have to configure a repository that hosts
the library and its transitive dependencies. The Liferay CDN repository
hosts them all:

\begin{verbatim}
repositories {
    maven {
        url "https://repository-cdn.liferay.com/nexus/content/groups/public"
    }
}
\end{verbatim}

\subsection{Tasks}\label{tasks-5}

The plugin adds one task to your project:

Name \textbar{} Depends On \textbar{} Type \textbar{} Description
\texttt{buildDeploymentHelper} \textbar{} - \textbar{}
\hyperref[builddeploymenthelpertask]{\texttt{BuildDeploymentHelperTask}}
\textbar{} Builds a WAR which contains one or more files that are copied
once the WAR is deployed.

\subsubsection{BuildDeploymentHelperTask}\label{builddeploymenthelpertask}

Tasks of type \texttt{BuildDeploymentHelperTask} extend
\href{https://docs.gradle.org/current/dsl/org.gradle.api.tasks.JavaExec.html}{\texttt{JavaExec}},
so all its properties and methods, such as
\href{https://docs.gradle.org/current/dsl/org.gradle.api.tasks.JavaExec.html\#org.gradle.api.tasks.JavaExec:args(java.lang.Iterable)}{\texttt{args}}
and
\href{https://docs.gradle.org/current/dsl/org.gradle.api.tasks.JavaExec.html\#org.gradle.api.tasks.JavaExec:maxHeapSize}{\texttt{maxHeapSize}},
are available. They also have the following properties set by default:

Property Name \textbar{} Default Value
\href{https://docs.gradle.org/current/dsl/org.gradle.api.tasks.JavaExec.html\#org.gradle.api.tasks.JavaExec:args}{\texttt{args}}
\textbar{} The Deployment Helper command line arguments.
\href{https://docs.gradle.org/current/dsl/org.gradle.api.tasks.JavaExec.html\#org.gradle.api.tasks.JavaExec:classpath}{\texttt{classpath}}
\textbar{}
\hyperref[liferay-deployment-helper-dependency]{\texttt{project.configurations.deploymentHelper}}
\hyperref[deploymentfiles]{\texttt{deploymentFiles}} \textbar{} The
output files of the
\href{https://docs.gradle.org/current/userguide/java_plugin.html\#sec:jar}{\texttt{jar}}
tasks of this project and all its sub-projects.
\href{https://docs.gradle.org/current/dsl/org.gradle.api.tasks.JavaExec.html\#org.gradle.api.tasks.JavaExec:main}{\texttt{main}}
\textbar{} \texttt{"com.liferay.deployment.helper.DeploymentHelper"}
\hyperref[outputfile]{\texttt{outputFile}} \textbar{}
\texttt{"\$\{project.buildDir\}/\$\{project.name\}.war"}

\paragraph{Task Properties}\label{task-properties-4}

Property Name \textbar{} Type \textbar{} Default Value \textbar{}
Description \texttt{deploymentFiles} \textbar{} \texttt{FileCollection}
\textbar{} \texttt{{[}{]}} \textbar{} The files or directories to
include in the WAR and copy once the WAR is deployed. If a directory is
added to this collection, all the JAR files contained in the directory
are included in the WAR. \texttt{deploymentPath} \textbar{}
\texttt{File} \textbar{} \texttt{null} \textbar{} The directory to which
the included files are copied. \texttt{outputFile} \textbar{}
\texttt{File} \textbar{} \texttt{null} \textbar{} The WAR file to build.

The properties of type \texttt{File} support any type that can be
resolved by
\href{https://docs.gradle.org/current/dsl/org.gradle.api.Project.html\#org.gradle.api.Project:file(java.css.Object)}{\texttt{project.file}}.

\paragraph{Task Methods}\label{task-methods-2}

Method \textbar{} Description
\texttt{BuildDeploymentHelperTask\ deploymentFiles(Iterable\textless{}?\textgreater{}\ deploymentFiles)}
\textbar{} Adds files or directories to include in the WAR and copy once
the WAR is deployed. The values are evaluated as per
\href{https://docs.gradle.org/current/dsl/org.gradle.api.Project.html\#org.gradle.api.Project:files(java.lang.Object\%5B\%5D)}{\texttt{project.files}}.
\texttt{BuildDeploymentHelperTask\ deploymentFiles(Object...\ deploymentFiles)}
\textbar{} Adds files or directories to include in the WAR and copy once
the WAR is deployed. The values are evaluated as per
\href{https://docs.gradle.org/current/dsl/org.gradle.api.Project.html\#org.gradle.api.Project:files(java.lang.Object\%5B\%5D)}{\texttt{project.files}}.

\subsection{Additional Configuration}\label{additional-configuration-4}

There are additional configurations that can help you use the Deployment
Helper.

\subsubsection{Liferay Deployment Helper
Dependency}\label{liferay-deployment-helper-dependency}

By default, the plugin creates a configuration called
\texttt{deploymentHelper} and adds a dependency to the latest released
version of the Liferay Deployment Helper. It is possible to override
this setting and use a specific version of the tool by manually adding a
dependency to the \texttt{deploymentHelper} configuration:

\begin{verbatim}
dependencies {
    deploymentHelper group: "com.liferay", name: "com.liferay.deployment.helper", version: "1.0.4"
}
\end{verbatim}

\section{Go Gradle Plugin}\label{go-gradle-plugin}

The Go Gradle plugin lets you run \href{https://golang.org/}{Go} as part
of your build.

The plugin has been successfully tested with Gradle 3.5.1 up to 4.10.2.

\subsection{Usage}\label{usage-7}

To use the plugin, include it in your build script:

\begin{verbatim}
buildscript {
    dependencies {
        classpath group: "com.liferay", name: "com.liferay.gradle.plugins.go", version: "1.0.0"
    }

    repositories {
        maven {
            url "https://repository-cdn.liferay.com/nexus/content/groups/public"
        }
    }
}

apply plugin: "com.liferay.go"
\end{verbatim}

\subsection{Project Extension}\label{project-extension-3}

The Go Gradle plugin exposes the following properties through the
extension named \texttt{go}:

Property Name \textbar{} Type \textbar{} Default Value \textbar{}
Description \texttt{goDir} \textbar{} \texttt{File} \textbar{}
\texttt{"\$\{project.buildDir\}/go"} \textbar{} The directory where the
Go distribution is unpacked. \texttt{goUrl} \textbar{} \texttt{String}
\textbar{}
\texttt{"https://dl.google.com/go/go\$\{go.goVersion\}.\$\{platform\}-\$\{bitMode\}.\$\{extension\}}
\textbar{} The URL of the Go distribution to download.
\texttt{goVersion} \textbar{} \texttt{String} \textbar{}
\texttt{"1.11.4"} \textbar{} The Go distribution's version to use.
\texttt{workingDir} \textbar{} \texttt{File} \textbar{}
\texttt{"\$\{project.projectDir\}"} \textbar{} The directory that
contains the project's Go source code.

\subsection{Tasks}\label{tasks-6}

The plugin adds a series of tasks to your project:

Name \textbar{} Depends On \textbar{} Type \textbar{} Description
\texttt{downloadGo} \textbar{} - \textbar{}
\hyperref[downloadgotask]{\texttt{DownloadGoTask}} \textbar{} Downloads
and unpacks the local Go distribution for the project.
\hyperref[gocommandprogramname-task]{\texttt{goBuild\$\{programName\}}}
\textbar{} \texttt{downloadGo} \textbar{}
\hyperref[executegotask]{\texttt{ExecuteGoTask}} \textbar{} Compiles
packages and dependencies for the Go program.
\hyperref[gocommandprogramname-task]{\texttt{goClean\$\{programName\}}}
\textbar{} \texttt{downloadGo} \textbar{}
\hyperref[executegotask]{\texttt{ExecuteGoTask}} \textbar{} Removes
object files for the Go program.
\hyperref[gocommandprogramname-task]{\texttt{goRun\$\{programName\}}}
\textbar{} \texttt{downloadGo} \textbar{}
\hyperref[executegotask]{\texttt{ExecuteGoTask}} \textbar{} Compiles and
runs the Go program.
\hyperref[gocommandprogramname-task]{\texttt{goTest\$\{programName\}}}
\textbar{} \texttt{downloadGo} \textbar{}
\hyperref[executegotask]{\texttt{ExecuteGoTask}} \textbar{} Tests
packages for the Go program.

\subsubsection{DownloadGoTask}\label{downloadgotask}

The purpose of this task is to download and unpack a Go distribution.

\paragraph{Task Properties}\label{task-properties-5}

Property Name \textbar{} Type \textbar{} Default Value \textbar{}
Description \texttt{goDir} \textbar{} \texttt{File} \textbar{}
\texttt{null} \textbar{} The directory where the Go distribution is
unpacked. \texttt{goUrl} \textbar{} \texttt{String} \textbar{}
\texttt{null} \textbar{} The URL of the Go distribution to download.

The \texttt{File} type properties support any type that can be resolved
by
\href{https://docs.gradle.org/current/dsl/org.gradle.api.Project.html\#org.gradle.api.Project:file(java.css.Object)}{\texttt{project.file}}.
Moreover, it is possible to use Closures and Callables as values for the
\texttt{String} properties, to defer evaluation until task execution.

\subsubsection{ExecuteGoTask}\label{executegotask}

This is the base task to run Go in a Gradle build. All tasks of type
\texttt{ExecuteGoTask} automatically depend on
\hyperref[downloadgo]{\texttt{downloadGo}}.

\paragraph{Task Properties}\label{task-properties-6}

Property Name \textbar{} Type \textbar{} Default Value \textbar{}
Description \texttt{args} \textbar{}
\texttt{List\textless{}Object\textgreater{}} \textbar{} \texttt{{[}{]}}
\textbar{} The arguments for the Go invocation. \texttt{command}
\textbar{} \texttt{String} \textbar{} \texttt{"go"} \textbar{} The file
name of the executable to invoke. \texttt{environment} \textbar{}
\texttt{Map\textless{}Object,\ Object\textgreater{}} \textbar{}
\texttt{{[}{]}} \textbar{} The environment variables for the Go
invocation. \texttt{inheritProxy} \textbar{} \texttt{boolean} \textbar{}
\texttt{true} \textbar{} Whether to set the \texttt{http\_proxy},
\texttt{https\_proxy}, and \texttt{no\_proxy} environment variables in
the Go invocation based on the values of the system properties
\texttt{https.proxyHost}, \texttt{https.proxyPort},
\texttt{https.proxyUser}, \texttt{https.proxyPassword},
\texttt{https.nonProxyHosts}, \texttt{https.proxyHost},
\texttt{https.proxyPort}, \texttt{https.proxyUser},
\texttt{https.proxyPassword}, and \texttt{https.nonProxyHosts}. If these
environment variables are already set, their values will not be
overwritten. \texttt{goDir} \textbar{} \texttt{File} \textbar{}
\texttt{go.goDir}{]}(\#godir) \textbar{} The directory that contains the
executable to invoke. \texttt{useGradleExec} \textbar{} \texttt{boolean}
\textbar{}

\textbf{If running in a
\href{https://docs.gradle.org/current/userguide/gradle_daemon.html}{Gradle
Daemon}:} \texttt{true}

\textbf{Otherwise:} \texttt{false}

\textbar{} Whether to invoke Go using
\href{https://docs.gradle.org/current/dsl/org.gradle.api.Project.html\#org.gradle.api.Project:exec(org.gradle.api.Action)}{\texttt{project.exec}},
which can solve hanging problems with the Gradle Daemon.
\texttt{workingDir} \textbar{} \texttt{File} \textbar{}
\texttt{go.workingDir}{]}(\#workingdir) \textbar{} The working directory
to use in the Go invocation.

The \texttt{File} type properties support any type that can be resolved
by
\href{https://docs.gradle.org/current/dsl/org.gradle.api.Project.html\#org.gradle.api.Project:file(java.css.Object)}{\texttt{project.file}}.
Moreover, it is possible to use Closures and Callables as values for the
\texttt{String} properties to defer evaluation until task execution.

\paragraph{Task Methods}\label{task-methods-3}

Method \textbar{} Description
\texttt{ExecuteGoTask\ args(Iterable\textless{}?\textgreater{}\ args)}
\textbar{} Adds arguments for the Go invocation.
\texttt{ExecuteGoTask\ args(Object...\ args)} \textbar{} Adds arguments
for the Go invocation.
\texttt{ExecuteGoTask\ environment(Map\textless{}?,\ ?\textgreater{}\ environment)}
\textbar{} Adds environment variables for the Go invocation.
\texttt{ExecuteGoTask\ environment(Object\ key,\ Object\ value)}
\textbar{} Adds an environment variable for the Go invocation.

\subsubsection{\texorpdfstring{go\({command}\)\{programName\}
Task}{go\{command\}\{programName\} Task}}\label{gocommandprogramname-task}

For each Go program in \hyperref[workingdir]{\texttt{workingDir}}, four
tasks of type \hyperref[executegotask]{\texttt{ExecuteGoTask}} are
added. Each of these tasks are automatically configured with sensible
defaults:

Property Name \textbar{} Default Value \texttt{args} \textbar{}
\texttt{{[}"\$\{command\}",\ "\$\{programFile.absolutePath\}"{]}}

\section{Gulp Gradle Plugin}\label{gulp-gradle-plugin}

The Gulp Gradle plugin lets you run \href{http://gulpjs.com/}{Gulp}
tasks as part of your build.

The plugin has been successfully tested with Gradle 4.10.2.

\subsection{Usage}\label{usage-8}

To use the plugin, include it in your build script:

\begin{verbatim}
buildscript {
    dependencies {
        classpath group: "com.liferay", name: "com.liferay.gradle.plugins.gulp", version: "2.0.59"
    }

    repositories {
        maven {
            url "https://repository-cdn.liferay.com/nexus/content/groups/public"
        }
    }
}

apply plugin: "com.liferay.gulp"
\end{verbatim}

The Gulp plugin automatically applies the
\href{https://github.com/liferay/liferay-portal/tree/master/modules/sdk/gradle-plugins-node}{\texttt{com.liferay.node}}
plugin.

\subsection{Tasks}\label{tasks-7}

The plugin adds one
\href{https://docs.gradle.org/current/userguide/more_about_tasks.html\#sec:task_rules}{task
rule} to your project:

Name \textbar{} Depends On \textbar{} Type \textbar{} Description
\texttt{gulp\textless{}Task\textgreater{}} \textbar{}
\texttt{downloadNode}, \texttt{npmInstall} \textbar{}
\hyperref[executegulptask]{\texttt{ExecuteGulpTask}} \textbar{} Executes
a named Gulp task.

\subsubsection{ExecuteGulpTask}\label{executegulptask}

Tasks of type \texttt{ExecuteGulpTask} extend
\href{/docs/7-0/reference/-/knowledge_base/r/node-gradle-plugin\#executenodescripttask}{\texttt{ExecuteNodeScriptTask}},
so all its properties and methods, such as \texttt{args} and
\texttt{inheritProxy}, are available. They also have the following
properties set by default:

Property Name \textbar{} Default Value \texttt{scriptFile} \textbar{}
\texttt{"node\_modules/gulp/bin/gulp.js"}

Gulp must be already installed in the \texttt{node\_modules} directory
of the project; otherwise, it will not be downloaded by the task. In
order to ensure Gulp is installed, you can add the Gulp dependency to
the project's \texttt{package.json} file.

\paragraph{Task Properties}\label{task-properties-7}

Property Name \textbar{} Type \textbar{} Default Value \textbar{}
Description \texttt{gulpCommand} \textbar{} \texttt{String} \textbar{}
\texttt{null} \textbar{} The Gulp task to execute.

It is possible to use Closures and Callables as values for the
\texttt{String} properties to defer evaluation until task execution.

\section{Jasper JSPC Gradle Plugin}\label{jasper-jspc-gradle-plugin}

The Jasper JSPC Gradle plugin lets you run the
\href{https://github.com/liferay/liferay-portal/tree/master/modules/util/jasper-jspc}{Liferay
Jasper JSPC} tool to compile the JavaServer Pages (JSP) files in your
project. This can be useful to

\begin{itemize}
\tightlist
\item
  check for errors in the JSP files.
\item
  pre-compile the JSP files for better performance.
\end{itemize}

The plugin has been successfully tested with Gradle 4.10.2.

\subsection{Usage}\label{usage-9}

To use the plugin, include it in your build script:

\begin{verbatim}
buildscript {
    dependencies {
        classpath group: "com.liferay", name: "com.liferay.gradle.plugins.jasper.jspc", version: "2.0.5"
    }

    repositories {
        maven {
            url "https://repository-cdn.liferay.com/nexus/content/groups/public"
        }
    }
}

apply plugin: "com.liferay.jasper.jspc"
\end{verbatim}

The Jasper JSPC plugin automatically applies the
\href{https://docs.gradle.org/current/userguide/java_plugin.html}{\texttt{java}}
plugin.

Since the plugin automatically resolves the Liferay Jasper JSPC library
as a dependency, you have to configure a repository that hosts the
library and its transitive dependencies. The Liferay CDN repository
hosts them all:

\begin{verbatim}
repositories {
    maven {
        url "https://repository-cdn.liferay.com/nexus/content/groups/public"
    }
}
\end{verbatim}

\subsection{Tasks}\label{tasks-8}

The plugin adds two tasks to your project:

Name \textbar{} Depends On \textbar{} Type \textbar{} Description
\texttt{compileJSP} \textbar{} \texttt{generateJSPJava} \textbar{}
\href{https://docs.gradle.org/current/dsl/org.gradle.api.tasks.compile.JavaCompile.html}{\texttt{JavaCompile}}
\textbar{} Compiles JSP files to check for errors.
\texttt{generateJSPJava} \textbar{}
\href{https://docs.gradle.org/current/userguide/java_plugin.html\#sec:jar}{\texttt{jar}}
\textbar{} \hyperref[compilejsptask]{\texttt{CompileJSPTask}} \textbar{}
Compiles JSP files to Java source files to check for errors.

The \texttt{generateJSPJava} task is automatically configured with
sensible defaults, depending on whether the
\href{https://docs.gradle.org/current/userguide/war_plugin.html}{\texttt{war}}
plugin is applied:

Property Name \textbar{} Default Value
\href{https://docs.gradle.org/current/dsl/org.gradle.api.tasks.JavaExec.html\#org.gradle.api.tasks.JavaExec:classpath}{\texttt{classpath}}
\textbar{}
\hyperref[liferay-jasper-jspc-dependency]{\texttt{project.configurations.jspCTool}}
\hyperref[destinationdir]{\texttt{destinationDir}} \textbar{}
\texttt{"\$\{project.buildDir\}/jspc"}
\hyperref[jspcclasspath]{\texttt{jspCClasspath}} \textbar{}
\hyperref[jsp-compilation-classpath]{\texttt{project.configurations.jspC}}
\hyperref[webappdir]{\texttt{webAppDir}} \textbar{}

\textbf{If the \texttt{war} plugin is applied:}
\texttt{project.webAppDir}.

\textbf{Otherwise:} The first \texttt{resources} directory of the
\texttt{main} source set (by default, \texttt{src/main/resources}).

The \texttt{compileJSP} task is also configured with the following
defaults:

Property Name \textbar{} Default Value
\href{https://docs.gradle.org/current/dsl/org.gradle.api.tasks.compile.JavaCompile.html\#org.gradle.api.tasks.compile.JavaCompile:classpath}{\texttt{classpath}}
\textbar{}
\texttt{project.configurations.jspCTool\ +\ project.configurations.jspC}
\href{https://docs.gradle.org/current/dsl/org.gradle.api.tasks.compile.JavaCompile.html\#org.gradle.api.tasks.compile.JavaCompile:destinationDir}{\texttt{destinationDir}}
\textbar{} \texttt{compileJSP.temporaryDir}
\href{https://docs.gradle.org/current/dsl/org.gradle.api.tasks.compile.JavaCompile.html\#org.gradle.api.tasks.compile.JavaCompile:source}{\texttt{source}}
\textbar{} \texttt{generateJSPJava.outputs}

\subsubsection{CompileJSPTask}\label{compilejsptask}

Tasks of type \texttt{CompileJSPTask} extend
\href{https://docs.gradle.org/current/dsl/org.gradle.api.tasks.JavaExec.html}{\texttt{JavaExec}},
so all its properties and methods, such as
\href{https://docs.gradle.org/current/dsl/org.gradle.api.tasks.JavaExec.html\#org.gradle.api.tasks.JavaExec:args(java.css.Iterable)}{\texttt{args}}
and
\href{https://docs.gradle.org/current/dsl/org.gradle.api.tasks.JavaExec.html\#org.gradle.api.tasks.JavaExec:maxHeapSize}{\texttt{maxHeapSize}},
are available. They also have the following properties set by default:

Property Name \textbar{} Default Value
\href{https://docs.gradle.org/current/dsl/org.gradle.api.tasks.JavaExec.html\#org.gradle.api.tasks.JavaExec:main}{\texttt{main}}
\textbar{} \texttt{"com.liferay.jasper.jspc.JspC"}

\paragraph{Task Properties}\label{task-properties-8}

Property Name \textbar{} Type \textbar{} Default Value \textbar{}
Description \texttt{destinationDir} \textbar{} \texttt{File} \textbar{}
\texttt{null} \textbar{} The directory where the the JSP files are
compiled to. Package directories are automatically generated based on
the directories containing the uncompiled JSP files. It sets the
\texttt{-d} argument. \texttt{jspCClasspath} \textbar{}
\texttt{FileCollection} \textbar{} \texttt{null} \textbar{} The
classpath to use for the JSP files compilation. \texttt{webAppDir}
\textbar{} \texttt{File} \textbar{} \texttt{null} \textbar{} The
directory containing the web application. All JSP files in the directory
and its subdirectories are compiled. It sets the \texttt{-webapp}
argument.

The properties of type \texttt{File} support any type that can be
resolved by
\href{https://docs.gradle.org/current/dsl/org.gradle.api.Project.html\#org.gradle.api.Project:file(java.css.Object)}{\texttt{project.file}}.

\subsection{Additional Configuration}\label{additional-configuration-5}

There are additional configurations that can help you use Jasper JSPC.

\subsubsection{JSP Compilation
Classpath}\label{jsp-compilation-classpath}

The plugin creates a configuration called \texttt{jspC} and adds several
dependencies at the end of the configuration phase of the project:

\begin{itemize}
\tightlist
\item
  the JAR file of the project generated by the
  \href{https://docs.gradle.org/current/userguide/java_plugin.html\#sec:jar}{\texttt{jar}}
  task.
\item
  the output files of the \texttt{main} source set.
\item
  the \texttt{compileClasspath} file collection of the \texttt{main}
  source set.
\end{itemize}

If necessary, it is possible to add more dependencies to the
\texttt{jspC} configuration.

\subsubsection{Liferay Jasper JSPC
Dependency}\label{liferay-jasper-jspc-dependency}

By default, the plugin creates a configuration called \texttt{jspCTool}
and adds a dependency to the latest released version of the Liferay
Jasper JSPC. It is possible to override this setting and use a specific
version of the tool by manually adding a dependency to the
\texttt{jspCTool} configuration:

\begin{verbatim}
dependencies {
    jspCTool group: "com.liferay", name: "com.liferay.jasper.jspc", version: "1.0.11"
    jspCTool group: "org.apache.ant", name: "ant", version: "1.9.4"
}
\end{verbatim}

\section{Javadoc Formatter Gradle
Plugin}\label{javadoc-formatter-gradle-plugin}

The Javadoc Formatter Gradle plugin lets you format project Javadoc
comments using the
\href{https://github.com/liferay/liferay-portal/tree/master/modules/util/javadoc-formatter}{Liferay
Javadoc Formatter tool}. The tool lets you generate:

\begin{itemize}
\tightlist
\item
  Default
  \href{http://www.oracle.com/technetwork/java/javase/documentation/index-137868.html\#@author}{\texttt{@author}}
  tags to all classes.
\item
  Comment stubs to classes, fields, and methods.
\item
  Missing
  \href{https://docs.oracle.com/javase/8/docs/api/java/lang/Override.html}{\texttt{@Override}}
  annotations.
\item
  An XML representation of the Javadoc comments, which can be used by
  tools in order to index the Javadocs of the project.
\end{itemize}

The plugin has been successfully tested with Gradle 4.10.2.

\subsection{Usage}\label{usage-10}

To use the plugin, include it in your build script:

\begin{verbatim}
buildscript {
    dependencies {
        classpath group: "com.liferay", name: "com.liferay.gradle.plugins.javadoc.formatter", version: "1.0.27"
    }

    repositories {
        maven {
            url "https://repository-cdn.liferay.com/nexus/content/groups/public"
        }
    }
}

apply plugin: "com.liferay.javadoc.formatter"
\end{verbatim}

Since the plugin automatically resolves the Liferay Javadoc Formatter
library as a dependency, you have to configure a repository that hosts
the library and its transitive dependencies. The Liferay CDN repository
hosts them all:

\begin{verbatim}
repositories {
    maven {
        url "https://repository-cdn.liferay.com/nexus/content/groups/public"
    }
}
\end{verbatim}

\subsection{Tasks}\label{tasks-9}

The plugin adds one task to your project:

Name \textbar{} Depends On \textbar{} Type \textbar{} Description
\texttt{formatJavadoc} \textbar{} - \textbar{}
\hyperref[formatjavadoctask]{\texttt{FormatJavadocTask}} \textbar{} Runs
the Liferay Javadoc Formatter to format files.

\subsubsection{FormatJavadocTask}\label{formatjavadoctask}

Tasks of type \texttt{FormatJavadocTask} extend
\href{https://docs.gradle.org/current/dsl/org.gradle.api.tasks.JavaExec.html}{\texttt{JavaExec}},
so all its properties and methods, like
\href{https://docs.gradle.org/current/dsl/org.gradle.api.tasks.JavaExec.html\#org.gradle.api.tasks.JavaExec:args(java.lang.Iterable)}{\texttt{args}}
and
\href{https://docs.gradle.org/current/dsl/org.gradle.api.tasks.JavaExec.html\#org.gradle.api.tasks.JavaExec:maxHeapSize}{\texttt{maxHeapSize}},
are available. They also have the following properties set by default:

Property Name \textbar{} Default Value
\href{https://docs.gradle.org/current/dsl/org.gradle.api.tasks.JavaExec.html\#org.gradle.api.tasks.JavaExec:args}{\texttt{args}}
\textbar{} Javadoc Formatter command line arguments
\href{https://docs.gradle.org/current/dsl/org.gradle.api.tasks.JavaExec.html\#org.gradle.api.tasks.JavaExec:classpath}{\texttt{classpath}}
\textbar{}
\hyperref[liferay-javadoc-formatter-dependency]{\texttt{project.configurations.javadocFormatter}}
\href{https://docs.gradle.org/current/dsl/org.gradle.api.tasks.JavaExec.html\#org.gradle.api.tasks.JavaExec:main}{\texttt{main}}
\textbar{} \texttt{"com.liferay.javadoc.formatter.JavadocFormatter"}

\paragraph{Task Properties}\label{task-properties-9}

Property Name \textbar{} Type \textbar{} Default Value \textbar{}
Description \texttt{author} \textbar{} \texttt{String} \textbar{}
\texttt{"Brian\ Wing\ Shun\ Chan"} \textbar{} The value of the
\texttt{@author} tag to add at class level if missing. It sets the
\texttt{javadoc.author} argument. \texttt{generateXML} \textbar{}
\texttt{boolean} \textbar{} \texttt{false} \textbar{} Whether to
generate a XML representation of the Javadoc comments. The XML files are
generated in the \texttt{src/main/resources} directory only if the Java
files are contained in \texttt{src/main/java}. It sets the
\texttt{javadoc.generate.xml} argument.
\texttt{initializeMissingJavadocs} \textbar{} \texttt{boolean}
\textbar{} \texttt{false} \textbar{} Whether to add comment stubs at the
class, field, and method levels. If \texttt{false}, only the class-level
\texttt{@author} is added. It sets the \texttt{javadoc.init} argument.
\texttt{limits} \textbar{} \texttt{List\textless{}String\textgreater{}}
\textbar{} \texttt{{[}{]}} \textbar{} The Java file name patterns,
relative to
\href{https://docs.gradle.org/current/dsl/org.gradle.api.tasks.JavaExec.html\#org.gradle.api.tasks.JavaExec:workingDir}{\texttt{workingDir}},
to include when formatting Javadoc comments. The patterns must be
specified without the \texttt{.java} file type suffix. If empty, all
Java files are formatted. It sets the \texttt{javadoc.limit} argument.
\texttt{lowestSupportedJavaVersion} \textbar{} \texttt{double}
\textbar{} \texttt{1.7} \textbar{} If a method is annotated with the
\href{https://github.com/liferay/liferay-portal/blob/master/modules/util/javadoc-formatter/src/main/java/com/liferay/javadoc/formatter/SinceJava.java}{\texttt{@SinceJava}}
annotation and its \texttt{value} argument is greater than the value
specified for the \texttt{lowestSupportedJavaVersion} property, then the
\texttt{@Override} annotation is not automatically added, even if it is
missing. It sets the \texttt{javadoc.lowest.supported.java.version}
argument. See
\href{https://issues.liferay.com/browse/LPS-37353}{LPS-37353}.
\texttt{outputFilePrefix} \textbar{} \texttt{String} \textbar{}
\texttt{"javadocs"} \textbar{} The file name prefix of the XML
representation of the Javadoc comments. If \texttt{generateXML} is
\texttt{false}, this property is not used. It sets the
\texttt{javadoc.output.file.prefix} argument. \texttt{updateJavadocs}
\textbar{} \texttt{boolean} \textbar{} \texttt{false} \textbar{} Whether
to fix existing comment blocks by adding missing tags. It sets the
\texttt{javadoc.update} argument.

It is possible to use Closures and Callables as values for the
\texttt{String} properties, to defer evaluation until task execution.

\paragraph{Task Methods}\label{task-methods-4}

Method \textbar{} Description
\texttt{FormatJavadocTask\ dirNames(Iterable\textless{}Object\textgreater{}\ limits)}
\textbar{} Adds Java file name patterns, relative to
\texttt{workingDir}, to include when formatting Javadoc comments.
\texttt{FormatJavadocTask\ dirNames(Object...\ limits)} \textbar{} Adds
Java file name patterns, relative to \texttt{workingDir}, to include
when formatting Javadoc comments.

\subsection{Additional Configuration}\label{additional-configuration-6}

There are additional configurations that can help you use the Javadoc
Formatter.

\subsubsection{Liferay Javadoc Formatter
Dependency}\label{liferay-javadoc-formatter-dependency}

By default, the plugin creates a configuration called
\texttt{javadocFormatter} and adds a dependency to the latest released
version of the Liferay Javadoc Formatter. It is possible to override
this setting and use a specific version of the tool by manually adding a
dependency to the \texttt{javadocFormatter} configuration:

\begin{verbatim}
dependencies {
    javadocFormatter group: "com.liferay", name: "com.liferay.javadoc.formatter", version: "1.0.32"
}
\end{verbatim}

If the
\href{https://docs.gradle.org/current/userguide/java_plugin.html}{\texttt{java}}
plugin is applied, the \texttt{javadocFormatter} configuration
automatically extends from the
\href{https://docs.gradle.org/current/userguide/java_plugin.html\#sec:java_plugin_and_dependency_management}{\texttt{compile}}
configuration.

\subsubsection{System Properties}\label{system-properties-1}

It is possible to set the default values of the \texttt{generateXML},
\texttt{initializeMissingJavadocs}, \texttt{limits}, and
\texttt{updateJavadocs} properties for a \texttt{FormatJavadocTask} task
via system properties:

\begin{itemize}
\tightlist
\item
  \texttt{-D\$\{task.name\}.generate.xml=true}
\item
  \texttt{-D\$\{task.name\}.init=SomeClassName1,SomeClassName2,com.liferay.portal.**}
\item
  \texttt{-D\$\{task.name\}.limit=**/com/example/}
\item
  \texttt{-D\$\{task.name\}.update=true}
\end{itemize}

\section{JS Module Config Generator Gradle
Plugin}\label{js-module-config-generator-gradle-plugin}

The JS Module Config Generator Gradle plugin lets you run the
\href{https://github.com/liferay/liferay-module-config-generator}{Liferay
AMD Module Config Generator} to generate the configuration file needed
to load AMD files via combo loader in Liferay.

The plugin has been successfully tested with Gradle 4.10.2.

\subsection{Usage}\label{usage-11}

To use the plugin, include it in your build script:

\begin{verbatim}
buildscript {
    dependencies {
        classpath group: "com.liferay", name: "com.liferay.gradle.plugins.js.module.config.generator", version: "2.1.57"
    }

    repositories {
        maven {
            url "https://repository-cdn.liferay.com/nexus/content/groups/public"
        }
    }
}

apply plugin: "com.liferay.js.module.config.generator"
\end{verbatim}

The JS Module Config Generator plugin automatically applies the
\href{https://github.com/liferay/liferay-portal/tree/master/modules/sdk/gradle-plugins-node}{\texttt{com.liferay.node}}
plugin.

\subsection{Project Extension}\label{project-extension-4}

The JS Module Config Generator plugin exposes the following properties
through the extension named \texttt{jsModuleConfigGenerator}:

Property Name \textbar{} Type \textbar{} Default Value \textbar{}
Description \texttt{version} \textbar{} \texttt{String} \textbar{}
\texttt{"1.2.1"} \textbar{} The version of the Liferay AMD Module Config
Generator to use.

\subsection{Tasks}\label{tasks-10}

The plugin adds two tasks to your project:

Name \textbar{} Depends On \textbar{} Type \textbar{} Description
\texttt{configJSModules} \textbar{}
\texttt{downloadLiferayModuleConfigGenerator}, \texttt{processResources}
\textbar{} \hyperref[configjsmodulestask]{\texttt{ConfigJSModulesTask}}
\textbar{} Generates the configuration file needed to load AMD files via
combo loader in Liferay. \texttt{downloadLiferayModuleConfigGenerator}
\textbar{} \texttt{downloadNode} \textbar{}
\texttt{DownloadNodeModuleTask} \textbar{} Downloads the Liferay AMD
Module Config Generator in the project's \texttt{node\_modules}
directory.

By default, the \texttt{downloadLiferayModuleConfigGenerator} task
downloads the version of \texttt{liferay-module-config-generator}
declared in the
\hyperref[version]{\texttt{jsModuleConfigGenerator.version}} property.
If the project's \texttt{package.json} file, however, already lists the
\texttt{liferay-module-config-generator} package in its
\texttt{dependencies} or \texttt{devDependencies}, the
\texttt{downloadLiferayModuleConfigGenerator} task is disabled.

The \texttt{configJSModules} task is automatically configured with
sensible defaults, depending on whether the
\href{https://docs.gradle.org/current/userguide/java_plugin.html}{\texttt{java}}
plugin is applied:

Property Name \textbar{} Default Value
\hyperref[moduleconfigfile]{\texttt{moduleConfigFile}} \textbar{}
\texttt{"\$\{project.projectDir\}/package.json"}
\hyperref[outputfile]{\texttt{outputFile}} \textbar{}
\texttt{"\$\{sourceSets.main.output.resourcesDir\}/META-INF/config.json"}
\hyperref[sourcedir]{\texttt{sourceDir}} \textbar{}
\texttt{"\$\{sourceSets.main.output.resourcesDir\}/META-INF/resources"}

The plugin also adds the following dependencies to tasks defined by the
\texttt{java} plugin:

Name \textbar{} Depends On \texttt{classes} \textbar{}
\texttt{configJSModules}

If the
\href{https://github.com/liferay/liferay-portal/tree/master/modules/sdk/gradle-plugins-js-transpiler}{\texttt{com.liferay.js.transpiler}}
plugin is applied, the \texttt{configJSModules} task is configured to
always run after the \texttt{transpileJS} task.

\subsubsection{ConfigJSModulesTask}\label{configjsmodulestask}

Tasks of type \texttt{ConfigJSModulesTask} extend
\texttt{ExecuteNodeScriptTask}, so all its properties and methods, such
as \texttt{args}, \texttt{inheritProxy}, and \texttt{workingDir}, are
available. The \texttt{ConfigJSModulesTask} instances also implement the
\href{https://docs.gradle.org/current/javadoc/org/gradle/api/tasks/util/PatternFilterable.html}{\texttt{PatternFilterable}}
interface, which lets you specify include and exclude patterns for the
files in \hyperref[sourcedir]{\texttt{sourceDir}} to process.

They also have the following properties set by default:

Property Name \textbar{} Default Value
\href{https://docs.gradle.org/current/javadoc/org/gradle/api/tasks/util/PatternFilterable.html\#getIncludes()}{\texttt{includes}}
\textbar{} \texttt{{[}"**/*.es.js*",\ "**/*.soy.js*"{]}}
\texttt{scriptFile} \textbar{}
\texttt{"\$\{downloadLiferayModuleConfigGenerator.moduleDir\}/bin/index.js"}

The purpose of this task is to run the Liferay AMD Module Config
Generator from the included files in
\hyperref[sourcedir]{\texttt{sourceDir}}. The generator processes these
files and creates a configuration file in the location specified by the
\hyperref[outputfile]{\texttt{outputFile}} property.

\paragraph{Task Properties}\label{task-properties-10}

Property Name \textbar{} Type \textbar{} Default Value \textbar{}
Description \texttt{configVariable} \textbar{} \texttt{String}
\textbar{} \texttt{null} \textbar{}
The~configuration~variable~to~which~the~modules~should~be~added. It sets
the \texttt{-\/-config} argument. \texttt{customDefine} \textbar{}
\texttt{String} \textbar{} \texttt{"Liferay.Loader"} \textbar{} The
namespace of the \texttt{define(...)} call to use in the JS files. It
sets the \texttt{-\/-namespace} argument. \texttt{ignorePath} \textbar{}
\texttt{boolean} \textbar{} \texttt{false} \textbar{} Whether not to
create module \texttt{path} and \texttt{fullPath} properties. It sets
the \texttt{-\/-ignorePath} argument. \texttt{keepFileExtension}
\textbar{} \texttt{boolean} \textbar{} \texttt{false} \textbar{} Whether
to keep the file extension when generating the module name. It sets the
\texttt{-\/-keepExtension} argument. \texttt{lowerCase} \textbar{}
\texttt{boolean} \textbar{} \texttt{false} \textbar{} Whether to convert
file name to lower case before using it as the module name. It sets the
\texttt{-\/-lowerCase} argument. \texttt{moduleConfigFile} \textbar{}
\texttt{File} \textbar{} \texttt{null} \textbar{} The JSON file which
contains configuration data for the modules. It sets the
\texttt{-\/-moduleConfig} argument. \texttt{moduleExtension} \textbar{}
\texttt{String} \textbar{} \texttt{null} \textbar{} The extension for
the module file (e.g., \texttt{.js}). If specified, use the provided
string~as~an~extension~instead~to~get~it~automatically~from~the~file~name.
It sets the \texttt{-\/-extension} argument. \texttt{moduleFormat}
\textbar{} \texttt{String} \textbar{} \texttt{null} \textbar{} The
regular expression and value to apply to the file name when generating
the module name. It sets the \texttt{-\/-format} argument.
\texttt{outputFile} \textbar{} \texttt{File} \textbar{} \texttt{null}
\textbar{} The file where the generated configuration is stored. It sets
the \texttt{-\/-output} argument. \texttt{sourceDir} \textbar{}
\texttt{File} \textbar{} \texttt{null} \textbar{} The directory that
contains the files to process.

The properties of type \texttt{File} support any type that can be
resolved by
\href{https://docs.gradle.org/current/dsl/org.gradle.api.Project.html\#org.gradle.api.Project:file(java.css.Object)}{\texttt{project.file}}.
Moreover, it is possible to use Closures and Callables as values for the
\texttt{int} and \texttt{String} properties to defer evaluation until
task execution.

\section{JS Transpiler Gradle Plugin}\label{js-transpiler-gradle-plugin}

The JS Transpiler Gradle plugin lets you run
\href{https://github.com/metal/metal-cli}{\texttt{metal-cli}} to build
\href{http://metaljs.com/}{Metal.js} code, compile Soy files, and
transpile ES6 to ES5.

\noindent\hrulefill

\textbf{Important:} If you're using
\href{/docs/7-0/tutorials/-/knowledge_base/t/liferay-workspace}{Liferay
Workspace} to create your app, the JS Transpiler Gradle plugin is
applied by default. Do not apply the JS Transpiler Gradle plugin if
you're using Liferay Workspace.

\noindent\hrulefill

The plugin has been successfully tested with Gradle 4.10.2.

\subsection{Usage}\label{usage-12}

To use the plugin, include it in your build script:

\begin{verbatim}
buildscript {
    dependencies {
        classpath group: "com.liferay", name: "com.liferay.gradle.plugins.js.transpiler", version: "2.4.36"
    }

    repositories {
        maven {
            url "https://repository-cdn.liferay.com/nexus/content/groups/public"
        }
    }
}

apply plugin: "com.liferay.js.transpiler"
\end{verbatim}

There are two JS Transpiler Gradle plugins you can apply to your
project:

\begin{itemize}
\item
  \hyperref[js-transpiler-plugin]{\emph{JS Transpiler Plugin}}: builds
  Metal.js code, compiles Soy files, and transpiles ES6 to ES5:

\begin{verbatim}
apply plugin: "com.liferay.js.transpiler"
\end{verbatim}
\item
  \hyperref[js-transpiler-base-plugin]{\emph{JS Transpiler Base
  Plugin}}: provides a way to use Gradle dependencies (such as an
  \href{https://docs.gradle.org/current/userguide/dependency_management.html\#sub:module_dependencies}{external
  module} or
  \href{https://docs.gradle.org/current/userguide/dependency_management.html\#sub:project_dependencies}{project
  dependencies}) in Node.js scripts:

\begin{verbatim}
apply plugin: "com.liferay.js.transpiler.base"
\end{verbatim}
\end{itemize}

\subsection{JS Transpiler Plugin}\label{js-transpiler-plugin}

The JS Transpiler plugin automatically applies the
\hyperref[js-transpiler-base-plugin]{\emph{JS Transpiler Base Plugin}}.

The plugin adds two tasks to your project:

Name \textbar{} Depends On \textbar{} Type \textbar{} Description
\texttt{downloadMetalCli} \textbar{} \texttt{downloadNode} \textbar{}
\texttt{DownloadNodeModuleTask} \textbar{} Downloads \texttt{metal-cli}
in the project's \texttt{node\_modules} directory. \texttt{transpileJS}
\textbar{} \texttt{downloadMetalCli},
\texttt{expandJSCompileDependencies}, \texttt{npmInstall},
\texttt{processResources} \textbar{}
\hyperref[transpilejstask]{\texttt{TranspileJSTask}} \textbar{} Builds
Metal.js code.

By default, the \texttt{downloadMetalCli} task downloads the version
1.3.1 of \texttt{metal-cli}. If the project's \texttt{package.json}
file, however, already lists the \texttt{metal-cli} package in its
\texttt{dependencies} or \texttt{devDependencies}, the
\texttt{downloadMetalCli} task is disabled.

The \texttt{transpileJS} task is automatically configured with sensible
defaults, depending on whether the
\href{https://docs.gradle.org/current/userguide/java_plugin.html}{\texttt{java}}
plugin is applied:

Property Name \textbar{} Default Value
\hyperref[sourcedir]{\texttt{sourceDir}} \textbar{} The directory
\texttt{META-INF/resources} in the first \texttt{resources} directory of
the \texttt{main} source set (by default,
\texttt{src/main/resources/META-INF/resources}). \texttt{workingDir}
\textbar{}
\texttt{"\$\{sourceSets.main.output.resourcesDir\}/META-INF/resources"}

The plugin also adds the following dependencies to tasks defined by the
\texttt{java} plugin:

Name \textbar{} Depends On \texttt{classes} \textbar{}
\texttt{transpileJS}

The plugin adds a new configuration to the project called
\texttt{soyCompile}. If one or more dependencies are added to this
configuration, they will be expanded into temporary directories and
passed to the \texttt{transpileJS} task as additional
\hyperref[soydependencies]{\texttt{soyDependencies}} values.

\subsection{JS Transpiler Base Plugin}\label{js-transpiler-base-plugin}

The JS Transpiler Base plugin automatically applies the
\href{https://github.com/liferay/liferay-portal/tree/master/modules/sdk/gradle-plugins-node}{\texttt{com.liferay.node}}
plugin.

The plugin adds a new configuration to the project called
\texttt{jsCompile}. If one or more dependencies are added to this
configuration, they will be expanded into sub-directories of the
\texttt{node\_modules} directory, with names equal to the names of the
dependencies.

The plugin also adds one task to your project:

Name \textbar{} Depends On \textbar{} Type \textbar{} Description
\texttt{expandJSCompileDependencies} \textbar{} - \textbar{}
\href{https://docs.gradle.org/current/javadoc/org/gradle/api/DefaultTask.html}{\texttt{DefaultTask}}
\textbar{} Expands the additional configured JavaScript dependencies.
The task itself does not do any work, but depends on a series of
\href{https://docs.gradle.org/current/dsl/org.gradle.api.tasks.Copy.html}{Copy}
tasks called \texttt{expandJSCompileDependency\$\{file\}}, which expand
each dependency declared in the \texttt{jsCompile} configuration into
the \texttt{node\_modules} directory.

All the tasks of type \texttt{ExecuteNpmTask} whose name starts with
\texttt{"npmRun"} are configured to depend on
\texttt{expandJSCompileDependencies}. This means that, before running
any \href{https://docs.npmjs.com/misc/scripts}{script} declared in the
\texttt{package.json} file of the project, all the \texttt{jsCompile}
dependencies will be expanded into the \texttt{node\_modules} directory.

\subsection{Tasks}\label{tasks-11}

\subsubsection{TranspileJSTask}\label{transpilejstask}

Tasks of type \texttt{TranspileJSTask} extend
\texttt{ExecuteNodeScriptTask}, so all its properties and methods, such
as \texttt{args}, \texttt{inheritProxy}, and \texttt{workingDir}, are
available. They also have the following properties set by default:

Property Name \textbar{} Default Value \texttt{scriptFile} \textbar{}
\texttt{"\$\{downloadMetalCli.moduleDir\}/index.js"}
\texttt{soySrcIncludes} \textbar{} \texttt{{[}"**/*.soy"{]}}
\texttt{srcIncludes} \textbar{}
\texttt{{[}"**/*.es.js*",\ "**/*.soy.js*"{]}}

The purpose of this task is to run the \texttt{build} command of
\texttt{metal-cli} to build Metal.js code from
\hyperref[sourcedir]{\texttt{sourceDir}} into the \texttt{workingDir}
directory.

\paragraph{Task Properties}\label{task-properties-11}

Property Name \textbar{} Type \textbar{} Default Value \textbar{}
Description \texttt{bundleFileName} \textbar{} \texttt{String}
\textbar{} \texttt{null} \textbar{} The name of the final bundle file
for formats (e.g., \emph{globals}) that create one. It sets the
\texttt{-\/-bundleFileName} argument. \texttt{globalName} \textbar{}
\texttt{String} \textbar{} \texttt{null} \textbar{} The name of the
global variable that holds exported modules. It sets the
\texttt{-\/-globalName} argument. This is only used by the
\emph{globals} format build. \texttt{moduleName} \textbar{}
\texttt{String} \textbar{} \texttt{null} \textbar{} The name of the
project that is being compiled. All built modules are stored in a folder
with this name. It sets the \texttt{-\/-moduleName} argument. This is
only used by the \emph{amd} format build. \texttt{modules} \textbar{}
\texttt{String} \textbar{} \texttt{"amd"} \textbar{} The format(s) that
the source files are built to. It sets the \texttt{-\/-format} argument.
\texttt{skipWhenEmpty} \textbar{} \texttt{boolean} \textbar{}
\texttt{true} \textbar{} Whether to disable the task and remove its
dependencies if the \hyperref[sourcefiles]{\texttt{sourceFiles}}
property does not return any file at the end of the project evaluation.
\texttt{sourceDir} \textbar{} \texttt{File} \textbar{} \texttt{null}
\textbar{} The directory that contains the files to build.
\texttt{sourceFiles} \textbar{} \texttt{FileCollection} \textbar{}
\texttt{{[}{]}} \textbar{} The Soy and JS files to compile.
\emph{(Read-only)} \texttt{sourceMaps} \textbar{} \texttt{SourceMaps}
\textbar{} \texttt{enabled} \textbar{} Whether to generate source map
files. Available values include \texttt{disabled}, \texttt{enabled}, and
\texttt{enabled\_inline}. \texttt{soyDependencies} \textbar{}
\texttt{Set\textless{}String\textgreater{}} \textbar{}
\texttt{{[}"\$\{npmInstall.workingDir\}/node\_modules/clay*/src/**/*.soy",\ "\$\{npmInstall.workingDir\}/node\_modules/metal*/src/**/*.soy"{]}}
\textbar{} The path GLOBs of Soy files that the main source files depend
on, but that should not be compiled. It sets the \texttt{-\/-soyDeps}
argument. \texttt{soySkipMetalGeneration} \textbar{} \texttt{boolean}
\textbar{} \texttt{false} \textbar{} Whether to just compile Soy files,
without adding Metal.js generated code, like the \texttt{component}
class. It sets the \texttt{-\/-soySkipMetalGeneration} argument.
\texttt{soySrcIncludes} \textbar{}
\texttt{Set\textless{}String\textgreater{}} \textbar{} \texttt{{[}{]}}
\textbar{} The path GLOBs of the Soy files to compile. It sets the
\texttt{-\/-soySrc} argument. \texttt{srcIncludes} \textbar{}
\texttt{Set\textless{}String\textgreater{}} \textbar{} \texttt{{[}{]}}
\textbar{} The path GLOBs of the JS files to compile. It sets the
\texttt{-\/-src} argument.

The properties of type \texttt{File} support any type that can be
resolved by
\href{https://docs.gradle.org/current/dsl/org.gradle.api.Project.html\#org.gradle.api.Project:file(java.css.Object)}{\texttt{project.file}}.
Moreover, it is possible to use Closures and Callables as values for the
\texttt{int} and \texttt{String} properties to defer evaluation until
task execution.

\paragraph{Task Methods}\label{task-methods-5}

Method \textbar{} Description
\texttt{TranspileJSTask\ soyDependency(Iterable\textless{}?\textgreater{}\ soyDependencies)}
\textbar{} Adds path GLOBs of Soy files that the main source files
depend on, but that should not be compiled.
\texttt{TranspileJSTask\ soyDependency(Object...\ soyDependencies)}
\textbar{} Adds path GLOBs of Soy files that the main source files
depend on, but that should not be compiled.
\texttt{TranspileJSTask\ soySrcInclude(Iterable\textless{}?\textgreater{}\ soySrcIncludes)}
\textbar{} Adds path GLOBs of Soy files to compile.
\texttt{TranspileJSTask\ soySrcInclude(Object...\ soySrcIncludes)}
\textbar{} Adds path GLOBs of Soy files to compile.
\texttt{TranspileJSTask\ srcInclude(Iterable\textless{}?\textgreater{}\ srcIncludes)}
\textbar{} Adds path GLOBs of JS files to compile.
\texttt{TranspileJSTask\ srcInclude(Object...\ srcIncludes)} \textbar{}
Adds path GLOBs of JS files to compile.

\section{JSDoc Gradle Plugin}\label{jsdoc-gradle-plugin}

The JSDoc Gradle plugin lets you run the
\href{http://usejsdoc.org/}{JSDoc} tool in order to generate
documentation for your project's JavaScript files.

The plugin has been successfully tested with Gradle 4.10.2.

\subsection{Usage}\label{usage-13}

To use the plugin, include it in your build script:

\begin{verbatim}
buildscript {
    dependencies {
        classpath group: "com.liferay", name: "com.liferay.gradle.plugins.jsdoc", version: "2.0.33"
    }

    repositories {
        maven {
            url "https://repository-cdn.liferay.com/nexus/content/groups/public"
        }
    }
}
\end{verbatim}

There are two JSDoc Gradle plugins you can apply to your project:

\begin{itemize}
\item
  Apply the \hyperref[jsdoc-plugin]{JSDoc Plugin} to generate JavaScript
  documentation for your project:

\begin{verbatim}
apply plugin: "com.liferay.jsdoc"
\end{verbatim}
\item
  Apply the \hyperref[appjsdoc-plugin]{App JSDoc Plugin} in a parent
  project to generate the JavaScript documentation as a single, combined
  HTML document for an application that spans different subprojects,
  each one representing a different component of the same application:

\begin{verbatim}
apply plugin: "com.liferay.app.jsdoc"
\end{verbatim}
\end{itemize}

Both plugins automatically apply the
\href{https://github.com/liferay/liferay-portal/tree/master/modules/sdk/gradle-plugins-node}{\texttt{com.liferay.node}}
plugin.

\subsection{JSDoc Plugin}\label{jsdoc-plugin}

The plugin adds two tasks to your project:

Name \textbar{} Depends On \textbar{} Type \textbar{} Description
\texttt{downloadJSDoc} \textbar{} \texttt{downloadNode} \textbar{}
\texttt{DownloadNodeModuleTask} \textbar{} Downloads JSDoc in the
project's \texttt{node\_modules} directory. \texttt{jsdoc} \textbar{}
\texttt{downloadJSDoc} \textbar{}
\hyperref[jsdoctask]{\texttt{JSDocTask}} \textbar{} Generates API
documentation for the project's JavaScript code.

By default, the \texttt{downloadJSDoc} task downloads version
\texttt{3.5.5} of the \texttt{jsdoc} package. If the project's
\texttt{package.json} file, however, already lists the \texttt{jsdoc}
package in its \texttt{dependencies} or \texttt{devDependencies}, the
\texttt{downloadJSDoc} task is disabled.

The \texttt{jsdoc} task is automatically configured with sensible
defaults, depending on whether the
\href{https://docs.gradle.org/current/userguide/java_plugin.html}{\texttt{java}}
plugin is applied:

Property Name \textbar{} Default Value
\hyperref[destinationdir]{\texttt{destinationDir}} \textbar{}

\textbf{If the \texttt{java} plugin is applied:}
\texttt{"\$\{project.docsDir\}/jsdoc"}

\textbf{Otherwise:} \texttt{"\$\{project.buildDir\}/jsdoc"}

\hyperref[sourcedirs]{\texttt{sourceDirs}} \textbar{} The directory
\texttt{META-INF/resources} in the first \texttt{resources} directory of
the \texttt{main} source set (by default,
\texttt{src/main/resources/META-INF/resources}).

\subsection{AppJSDoc Plugin}\label{appjsdoc-plugin}

To use the App JSDoc plugin, it is required to apply the
\texttt{com.liferay.app.jsdoc} plugin in a parent project (that is, a
project that is a common ancestor of all the subprojects representing
the various components of the app). It is also required to apply the
\hyperref[jsdoc-plugin]{\texttt{com.liferay.jsdoc}} plugin to all the
subprojects that contain JavaScript files.

The App JSDoc plugin adds three tasks to your project:

Name \textbar{} Depends On \textbar{} Type \textbar{} Description
\texttt{appJSDoc} \textbar{} \texttt{downloadJSDoc} \textbar{}
\hyperref[jsdoctask]{\texttt{JSDocTask}} \textbar{} Generates API
documentation for the app's JavaScript code. \texttt{downloadJSDoc}
\textbar{} \texttt{downloadNode} \textbar{}
\texttt{DownloadNodeModuleTask} \textbar{} Downloads JSDoc in the app's
\texttt{node\_modules} directory. \texttt{jarAppJSDoc} \textbar{}
\texttt{appJSDoc} \textbar{}
\href{https://docs.gradle.org/current/dsl/org.gradle.api.tasks.bundling.Jar.html}{\texttt{Jar}}
\textbar{} Assembles a JAR archive containing the JavaScript
documentation files for this app.

By default, the \texttt{downloadJSDoc} task downloads version
\texttt{3.5.5} of the \texttt{jsdoc} package. If the project's
\texttt{package.json} file, however, already lists the \texttt{jsdoc}
package in its \texttt{dependencies} or \texttt{devDependencies}, the
\texttt{downloadJSDoc} task is disabled.

The \texttt{appJSDoc} task is automatically configured with sensible
defaults:

Property Name \textbar{} Default Value
\hyperref[destinationdir]{\texttt{destinationDir}} \textbar{}
\texttt{\$\{project.buildDir\}/docs/jsdoc}
\hyperref[sourcedirs]{\texttt{sourceDirs}} \textbar{} The sum of all the
\texttt{jsdoc.sourceDirs} values of the subprojects.

\subsection{Project Extension}\label{project-extension-5}

The App JSDoc plugin exposes the following properties through the
extension named \texttt{appJSDocConfiguration}:

Property Name \textbar{} Type \textbar{} Default Value \textbar{}
Description \texttt{subprojects} \textbar{}
\texttt{Set\textless{}Project\textgreater{}} \textbar{}
\texttt{project.subprojects} \textbar{} The subprojects to include in
the JavaScript documentation of the app.

The same extension exposes the following methods:

Method \textbar{} Description
\texttt{AppJSDocConfigurationExtension\ subprojects(Iterable\textless{}Project\textgreater{}\ subprojects)}
\textbar{} Include additional projects in the JavaScript documentation
of the app.
\texttt{AppJSDocConfigurationExtension\ subprojects(Project...\ subprojects)}
\textbar{} Include additional projects in the JavaScript documentation
of the app.

\subsection{Tasks}\label{tasks-12}

\subsubsection{JSDocTask}\label{jsdoctask}

Tasks of type \texttt{JSDocTask} extend \texttt{ExecuteNodeScriptTask},
so all its properties and methods, such as \texttt{args},
\texttt{inheritProxy}, and \texttt{workingDir}, are available.

They also have the following properties set by default:

Property Name \textbar{} Default Value \texttt{scriptFile} \textbar{}
\texttt{"\$\{downloadJSDoc.moduleDir\}/jsdoc.js"}

\paragraph{Task Properties}\label{task-properties-12}

Property Name \textbar{} Type \textbar{} Default Value \textbar{}
Description \texttt{configuration} \textbar{}
\href{https://docs.gradle.org/current/dsl/org.gradle.api.resources.TextResource.html}{\texttt{TextResource}}
\textbar{} \texttt{null} \textbar{} The JSDoc configuration file. It
sets the \texttt{-\/-configure} argument. \texttt{destinationDir}
\textbar{} \texttt{File} \textbar{} \texttt{null} \textbar{} The
directory where the JavaScript API documentation files are saved. It
sets the \texttt{-\/-destination} argument. \texttt{packageJsonFile}
\textbar{} \texttt{File} \textbar{}
\texttt{"\$\{project.projectDir\}/package.json"} \textbar{} The path to
the project's package file. It sets the \texttt{-\/-package} argument.
\texttt{sourceDirs} \textbar{} \texttt{FileCollection} \textbar{}
\texttt{{[}{]}} \textbar{} The directories that contains the files to
process. \texttt{readmeFile} \textbar{} \texttt{File} \textbar{}
\texttt{null} \textbar{} The path to the project's README file. It sets
the \texttt{-\/-readme} argument. \texttt{tutorialsDir} \textbar{}
\texttt{File} \textbar{} \texttt{null} \textbar{} The directory in which
JSDoc should search for tutorials. It sets the \texttt{-\/-tutorials}
argument.

The properties of type \texttt{File} support any type that can be
resolved by
\href{https://docs.gradle.org/current/dsl/org.gradle.api.Project.html\#org.gradle.api.Project:file(java.css.Object)}{\texttt{project.file}}.

\section{Lang Builder Gradle Plugin}\label{lang-builder-gradle-plugin}

The Lang Builder Gradle plugin lets you run the
\href{https://github.com/liferay/liferay-portal/tree/master/modules/util/lang-builder}{Liferay
Lang Builder} tool to sort and translate the language keys in your
project.

The plugin has been successfully tested with Gradle 4.10.2.

\subsection{Usage}\label{usage-14}

To use the plugin, include it in your build script:

\begin{verbatim}
buildscript {
    dependencies {
        classpath group: "com.liferay", name: "com.liferay.gradle.plugins.lang.builder", version: "3.0.12"
    }

    repositories {
        maven {
            url "https://repository-cdn.liferay.com/nexus/content/groups/public"
        }
    }
}

apply plugin: "com.liferay.lang.builder"
\end{verbatim}

Since the plugin automatically resolves the Liferay Lang Builder library
as a dependency, you have to configure a repository that hosts the
library and its transitive dependencies. The Liferay CDN repository
hosts them all:

\begin{verbatim}
repositories {
    maven {
        url "https://repository-cdn.liferay.com/nexus/content/groups/public"
    }
}
\end{verbatim}

See
\href{/docs/7-0/tutorials/-/knowledge_base/t/automatically-generating-language-files}{this
page} on the \emph{Liferay Developer Network} for more information about
usage of the Lang Builder Gradle plugin.

\subsection{Tasks}\label{tasks-13}

The plugin adds one task to your project:

Name \textbar{} Depends On \textbar{} Type \textbar{} Description
\texttt{buildLang} \textbar{} - \textbar{}
\hyperref[buildlangtask]{\texttt{BuildLangTask}} \textbar{} Runs Liferay
Lang Builder to translate language property files.

The \texttt{buildLang} task is automatically configured with sensible
defaults, depending on whether the
\href{https://docs.gradle.org/current/userguide/java_plugin.html}{\texttt{java}}
plugin is applied:

Property Name \textbar{} Default Value
\hyperref[langdir]{\texttt{langDir}} \textbar{}

\textbf{If the \texttt{java} plugin is applied:} The directory
\texttt{content} in the first \texttt{resources} directory of the
\texttt{main} source set (by default:
\texttt{src/main/resources/content}).

\textbf{Otherwise:} \texttt{null}

\subsubsection{BuildLangTask}\label{buildlangtask}

Tasks of type \texttt{BuildLangTask} extend
\href{https://docs.gradle.org/current/dsl/org.gradle.api.tasks.JavaExec.html}{\texttt{JavaExec}},
so all its properties and methods, such as
\href{https://docs.gradle.org/current/dsl/org.gradle.api.tasks.JavaExec.html\#org.gradle.api.tasks.JavaExec:args(java.lang.Iterable)}{\texttt{args}}
and
\href{https://docs.gradle.org/current/dsl/org.gradle.api.tasks.JavaExec.html\#org.gradle.api.tasks.JavaExec:maxHeapSize}{\texttt{maxHeapSize}},
are available. They also have the following properties set by default:

Property Name \textbar{} Default Value
\href{https://docs.gradle.org/current/dsl/org.gradle.api.tasks.JavaExec.html\#org.gradle.api.tasks.JavaExec:args}{\texttt{args}}
\textbar{} Lang Builder command line arguments
\href{https://docs.gradle.org/current/dsl/org.gradle.api.tasks.JavaExec.html\#org.gradle.api.tasks.JavaExec:classpath}{\texttt{classpath}}
\textbar{}
\hyperref[liferay-lang-builder-dependency]{\texttt{project.configurations.langBuilder}}
\href{https://docs.gradle.org/current/dsl/org.gradle.api.tasks.JavaExec.html\#org.gradle.api.tasks.JavaExec:main}{\texttt{main}}
\textbar{} \texttt{"com.liferay.lang.builder.LangBuilder"}

\paragraph{Task Properties}\label{task-properties-13}

Property Name \textbar{} Type \textbar{} Default Value \textbar{}
Description \texttt{excludedLanguageIds} \textbar{}
\texttt{Set\textless{}String\textgreater{}} \textbar{}
\texttt{{[}"da",\ "de",\ "fi",\ "ja",\ "nl",\ "pt\_PT",\ "sv"{]}}
\textbar{} The language IDs to exclude in the automatic translation. It
sets the \texttt{lang.excluded.language.ids} argument. \texttt{langDir}
\textbar{} \texttt{File} \textbar{} \texttt{null} \textbar{} The
directory where the language properties files are saved. It sets the
\texttt{lang.dir} argument. \texttt{langFileName} \textbar{}
\texttt{String} \textbar{} \texttt{"Language"} \textbar{} The file name
prefix of the language properties files (e.g.,
\texttt{Language\_it.properties}). It sets the \texttt{lang.file}
argument. \texttt{plugin} \textbar{} \texttt{boolean} \textbar{}
\texttt{true} \textbar{} Whether to check for duplicate language keys
between the project and the portal. If
\texttt{portalLanguagePropertiesFile} is not set, this property has no
effect. It sets the \texttt{lang.plugin} argument.
\texttt{portalLanguagePropertiesFile} \textbar{} \texttt{File}
\textbar{} \texttt{null} \textbar{} The \texttt{Language.properties}
file of the portal. It sets the
\texttt{lang.portal.language.properties.file} argument.
\texttt{translate} \textbar{} \texttt{boolean} \textbar{} \texttt{true}
\textbar{} Whether to translate the language keys and generate a
language properties file for each locale that's supported by Liferay. It
sets the \texttt{lang.translate} argument.
\texttt{translateSubscriptionKey} \textbar{} \texttt{String} \textbar{}
\texttt{null} \textbar{} The subscription key for Microsoft Translation
integration. Subscription to the Translator Text Translation API on
Microsoft Cognitive Services is required. Basic subscriptions, up to 2
million characters a month, are free. See
\href{http://docs.microsofttranslator.com/text-translate.html}{here} for
more information. It sets the \texttt{lang.translate.subscription.key}
argument.

The properties of type \texttt{File} support any type that can be
resolved by
\href{https://docs.gradle.org/current/dsl/org.gradle.api.Project.html\#org.gradle.api.Project:file(java.lang.Object)}{\texttt{project.file}}.
Moreover, it is possible to use Closures and Callables as values for the
\texttt{String} properties, to defer evaluation until task execution.

\paragraph{Task Methods}\label{task-methods-6}

Method \textbar{} Description
\texttt{BuildLangTask\ excludedLanguageIds(Iterable\textless{}Object\textgreater{}\ excludedLanguageIds)}
\textbar{} Adds language IDs to exclude in the automatic translation.
\texttt{BuildLangTask\ excludedLanguageIds(Object...\ excludedLanguageIds)}
\textbar{} Adds language IDs to exclude in the automatic translation.

\subsection{Additional Configuration}\label{additional-configuration-7}

There are additional configurations that can help you use the Lang
Builder.

\subsubsection{Liferay Lang Builder
Dependency}\label{liferay-lang-builder-dependency}

By default, the plugin creates a configuration called
\texttt{langBuilder} and adds a dependency to the latest released
version of the Liferay Lang Builder. It is possible to override this
setting and use a specific version of the tool by manually adding a
dependency to the \texttt{langBuilder} configuration:

\begin{verbatim}
dependencies {
    langBuilder group: "com.liferay", name: "com.liferay.lang.builder", version: "1.0.31"
}
\end{verbatim}

\section{Maven Plugin Builder Gradle
Plugin}\label{maven-plugin-builder-gradle-plugin}

The Maven Plugin Builder Gradle Plugin lets you generate the
\href{https://maven.apache.org/ref/current/maven-plugin-api/plugin.html}{Maven
plugin descriptor} for any
\href{https://maven.apache.org/general.html\#What_is_a_Mojo}{Mojos}
found in your project.

The plugin has been successfully tested with Gradle 4.10.2.

\subsection{Usage}\label{usage-15}

To use the plugin, include it in your build script:

\begin{verbatim}
buildscript {
    dependencies {
        classpath group: "com.liferay", name: "com.liferay.gradle.plugins.maven.plugin.builder", version: "1.2.4"
    }

    repositories {
        maven {
            url "https://repository-cdn.liferay.com/nexus/content/groups/public"
        }
    }
}

apply plugin: "com.liferay.maven.plugin.builder"
\end{verbatim}

\subsection{Tasks}\label{tasks-14}

The plugin adds two tasks to your project:

Name \textbar{} Depends On \textbar{} Type \textbar{} Description
\texttt{buildPluginDescriptor}
\textbar{}\href{https://docs.gradle.org/current/userguide/java_plugin.html\#sec:compile}{\texttt{compileJava}},
\hyperref[writemavensettings]{\texttt{WriteMavenSettings}} \textbar{}
\hyperref[buildplugindescriptortask]{\texttt{BuildPluginDescriptorTask}}
\textbar{} Generates the Maven plugin descriptor for the project.
\texttt{WriteMavenSettings} \textbar{} - \textbar{}
\hyperref[writemavensettingstask]{\texttt{WriteMavenSettingsTask}}
\textbar{} Writes a temporary Maven settings file to be used during
subsequent Maven invocations.

The Maven Plugin Builder Plugin automatically applies the
\href{https://docs.gradle.org/current/userguide/java_plugin.html}{\texttt{java}}
plugin.

The plugin also adds the following dependencies to tasks defined by the
\href{https://docs.gradle.org/current/userguide/maven_plugin.html}{\texttt{maven}}
plugin:

Name \textbar{} Depends On \texttt{install}, \texttt{uploadArchives},
and all the other tasks of type
\href{https://docs.gradle.org/current/dsl/org.gradle.api.tasks.Upload.html}{\texttt{Upload}}
\textbar{} \texttt{buildPluginDescriptor}

The \texttt{buildPluginDescriptor} task is automatically configured with
sensible defaults:

Property Name \textbar{} Default Value
\hyperref[classesdir]{\texttt{classesDir}} \textbar{}
\texttt{sourceSets.main.output.classesDir}
\hyperref[mavenembedderclasspath]{\texttt{mavenEmbedderClasspath}}
\textbar{}
\hyperref[maven-embedder-dependency]{\texttt{configurations.mavenEmbedder}}
\hyperref[mavensettingsfile]{\texttt{mavenSettingsFile}} \textbar{}
\hyperref[outputfile]{\texttt{writeMavenSettings.outputFile}}
\hyperref[outputdir]{\texttt{outputDir}} \textbar{} The directory
\texttt{META-INF/maven} in the first \texttt{resources} directory of the
\texttt{main} source set (by default:
\texttt{src/main/resources/META-INF/maven}).
\hyperref[pomartifactid]{\texttt{pomArtifactId}} \textbar{} The bundle
symbolic name of the project inferred via the
\href{https://github.com/gradle/gradle/blob/master/subprojects/osgi/src/main/java/org/gradle/api/internal/plugins/osgi/OsgiHelper.java}{\texttt{OsgiHelper}}
class. \hyperref[pomgroupid]{\texttt{pomGroupId}} \textbar{}
\href{https://docs.gradle.org/current/dsl/org.gradle.api.Project.html\#org.gradle.api.Project:group}{\texttt{project.group}}
\hyperref[pomversion]{\texttt{pomVersion}} \textbar{}
\href{https://docs.gradle.org/current/dsl/org.gradle.api.Project.html\#org.gradle.api.Project:version}{\texttt{project.version}}
(if it ends with \texttt{"-SNAPSHOT"}, the suffix will be removed)
\hyperref[sourcedir]{\texttt{sourceDir}} \textbar{} The first
\texttt{java} directory of the \texttt{main} source set (by default:
\texttt{src/main/java}).

The plugin ensures that the \texttt{processResources} task always runs
before \texttt{buildPluginDescriptor} to let \texttt{processResources}
copy the newly generated files in the
\texttt{buildPluginDescriptor.outputDir} directory.

The \texttt{writeMavenSettings} task is also automatically configured
with sensible defaults:

Property Name \textbar{} Default Value
\hyperref[localrepositorydir]{\texttt{localRepositoryDir}} \textbar{}
\texttt{maven.repo.local} system property
\hyperref[nonproxyhosts]{\texttt{nonProxyHosts}} \textbar{}
\texttt{http.nonProxyHosts} system property
\hyperref[outputfile]{\texttt{outputFile}} \textbar{}
\texttt{"\$\{project.buildDir\}/settings.xml"}
\hyperref[proxyhost]{\texttt{proxyHost}} \textbar{}
\texttt{http.ProxyHost} or \texttt{https.proxyHost} system property
(depending on the protocol of
\hyperref[repositoryurl]{\texttt{repositoryUrl}})
\hyperref[proxypassword]{\texttt{proxyPassword}} \textbar{}
\texttt{http.ProxyPassword} or \texttt{https.proxyPassword} system
property (depending on the protocol of
\hyperref[repositoryurl]{\texttt{repositoryUrl}})
\hyperref[proxyport]{\texttt{proxyPort}} \textbar{}
\texttt{http.ProxyPort} or \texttt{https.proxyPort} system property
(depending on the protocol of
\hyperref[repositoryurl]{\texttt{repositoryUrl}})
\hyperref[proxyuser]{\texttt{proxyUser}} \textbar{}
\texttt{http.ProxyUser} or \texttt{https.proxyUser} system property
(depending on the protocol of
\hyperref[repositoryurl]{\texttt{repositoryUrl}})
\hyperref[repositoryurl]{\texttt{repositoryUrl}} \textbar{}
\texttt{repository.url} system property

If running on JDK8+, the plugin also disables the
\href{http://docs.oracle.com/javase/8/docs/technotes/tools/unix/javadoc.html\#BEJEFABE}{\emph{doclint}}
feature in all tasks of type
\href{https://docs.gradle.org/current/dsl/org.gradle.api.tasks.javadoc.Javadoc.html}{\texttt{Javadoc}}.

\subsubsection{BuildPluginDescriptorTask}\label{buildplugindescriptortask}

Tasks of type \texttt{BuildPluginDescriptorTask} work by generating a
temporary \texttt{pom.xml} file based on the project, and then invoking
the \href{http://maven.apache.org/ref/3.3.9/maven-embedder/}{Maven
Embedder} to build the Maven plugin descriptor.

It is possible to declare information for the plugin descriptor
generation using either
\href{https://maven.apache.org/plugin-tools/maven-plugin-tools-annotations/}{Java
5 Annotations} or
\href{https://maven.apache.org/plugin-tools/maven-plugin-tools-java/}{Javadoc
Tags}.

\paragraph{Task Properties}\label{task-properties-14}

Property Name \textbar{} Type \textbar{} Default Value \textbar{}
Description \texttt{classesDir} \textbar{} \texttt{File} \textbar{}
\texttt{null} \textbar{} The directory that contains the compiled
classes. It sets the value of the
\href{http://maven.apache.org/ref/3.3.9/maven-model/maven.html\#class_build}{\texttt{build.outputDirectory}}
element in the generated \texttt{pom.xml} file.
\texttt{configurationScopeMappings} \textbar{}
\texttt{Map\textless{}String,\ String\textgreater{}} \textbar{}
\texttt{{[}"compile":\ "compile",\ "provided",\ "provided"{]}}
\textbar{} The mapping between the configuration names in the Gradle
project and the
\href{https://maven.apache.org/guides/introduction/introduction-to-dependency-mechanism.html\#Dependency_Scope}{dependency
scopes} in the \texttt{pom.xml} file. It is used to add
\href{http://maven.apache.org/ref/3.3.3/maven-model/maven.html\#class_dependency}{\texttt{dependencies.dependency}}
elements in the generated \texttt{pom.xml} file.
\texttt{forcedExclusions} \textbar{}
\texttt{Set\textless{}String\textgreater{}} \textbar{} \texttt{{[}{]}}
\textbar{} The \emph{group:name:version} notation of the dependencies to
always exclude from the ones added in the \texttt{pom.xml} file. It adds
\href{http://maven.apache.org/ref/3.3.3/maven-model/maven.html\#class_exclusion}{\texttt{dependencies.dependency.exclusions.exclusion}}
elements to the generated \texttt{pom.xml} file. \texttt{goalPrefix}
\textbar{} \texttt{String} \textbar{} \texttt{null} \textbar{} The goal
prefix for the Maven plugin specified in the descriptor. It sets the
value of the
\href{https://maven.apache.org/plugin-tools/maven-plugin-plugin/examples/generate-descriptor.html}{\texttt{build.plugins.plugin.configuration.goalPrefix}}
element in the generated \texttt{pom.xml} file. \texttt{mavenDebug}
\textbar{} \texttt{boolean} \textbar{} \texttt{false} \textbar{} Whether
to invoke the Maven Embedder in debug mode.
\texttt{mavenEmbedderClasspath} \textbar{} \texttt{FileCollection}
\textbar{} \texttt{null} \textbar{} The classpath used to invoke the
Maven Embedder. \texttt{mavenEmbedderMainClassName} \textbar{}
\texttt{String} \textbar{} \texttt{"org.apache.maven.cli.MavenCli"}
\textbar{} The Maven Embedder's main class name.
\texttt{mavenPluginPluginVersion} \textbar{} \texttt{String} \textbar{}
\texttt{"3.4"} \textbar{} The version of the
\href{https://maven.apache.org/plugin-tools/maven-plugin-plugin/}{Maven
Plugin Plugin} to use to generate the plugin descriptor for the project.
\texttt{mavenSettingsFile} \textbar{} \texttt{File} \textbar{}
\texttt{null} \textbar{} The custom \texttt{settings.xml} file to use.
It sets the \texttt{-\/-settings} argument on the Maven Embedder
invocation. \texttt{outputDir} \textbar{} \texttt{File} \textbar{}
\texttt{null} \textbar{} The directory where the Maven plugin descriptor
files are saved. \texttt{pomArtifactId} \textbar{} \texttt{String}
\textbar{} \texttt{null} \textbar{} The identifier for the artifact that
is unique within the group. It sets the value of the
\href{http://maven.apache.org/ref/3.3.3/maven-model/maven.html\#class_project}{\texttt{project.artifactId}}
element in the generated \texttt{pom.xml} file. \texttt{pomGroupId}
\textbar{} \texttt{String} \textbar{} \texttt{null} \textbar{} The
universally unique identifier for the project. It sets the value of the
\href{http://maven.apache.org/ref/3.3.3/maven-model/maven.html\#class_project}{\texttt{project.groupId}}
element in the generated \texttt{pom.xml} file. \texttt{pomRepositories}
\textbar{} \texttt{Map\textless{}String,\ Object\textgreater{}}
\textbar{}
\texttt{{[}"liferay-public":\ "http://repository.liferay.com/nexus/content/groups/public"{]}}
\textbar{} The name and URL of the remote repositories. It adds
\href{http://maven.apache.org/ref/3.3.3/maven-model/maven.html\#class_repository}{\texttt{repositories.repository}}
elements in the generated \texttt{pom.xml} file. \texttt{pomVersion}
\textbar{} \texttt{String} \textbar{} \texttt{null} \textbar{} The
version of the artifact produced by this project. It sets the value of
the
\href{http://maven.apache.org/ref/3.3.3/maven-model/maven.html\#class_project}{\texttt{project.version}}
element in the generated \texttt{pom.xml} file. \texttt{sourceDir}
\textbar{} \texttt{String} \textbar{} \texttt{null} \textbar{} The
directory that contains the source files. It sets the value of the
\href{http://maven.apache.org/ref/3.3.9/maven-model/maven.html\#class_build}{\texttt{build.sourceDirectory}}
element in the generated \texttt{pom.xml} file.
\texttt{useSetterComments} \textbar{} \texttt{boolean} \textbar{}
\texttt{true} \textbar{} Whether to allow
\href{https://maven.apache.org/plugin-tools/maven-plugin-tools-java/}{Mojo
Javadoc Tags} in the setter methods of the Mojo.

The properties of type \texttt{File} support any type that can be
resolved by
\href{https://docs.gradle.org/current/dsl/org.gradle.api.Project.html\#org.gradle.api.Project:file(java.lang.Object)}{\texttt{project.file}}.
Moreover, it is possible to use Closures and Callables as values for the
\texttt{String} properties, to defer evaluation until task execution.

\subsubsection{Task Methods}\label{task-methods-7}

Method \textbar{} Description
\texttt{void\ configurationScopeMapping(String\ configurationName,\ String\ scope)}
\textbar{} Adds a mapping between a configuration name in the Gradle
project and the dependency scope in the \texttt{pom.xml} file.
\texttt{BuildPluginDescriptorTask\ forcedExclusions(Iterable\textless{}String\textgreater{}\ forcedExclusions)}
\textbar{} Adds \emph{group:name:version} notations of dependencies to
always exclude from the ones added in the \texttt{pom.xml} file.
\texttt{BuildPluginDescriptorTask\ forcedExclusions(String...\ forcedExclusions)}
\textbar{} Adds \emph{group:name:version} notations of dependencies to
always exclude from the ones added in the \texttt{pom.xml} file.
\texttt{BuildPluginDescriptorTask\ pomRepositories(Map\textless{}String,\ ?\textgreater{}\ pomRepositories}
\textbar{} Adds names and URLs of remote repositories in the
\texttt{pom.xml} file.
\texttt{BuildPluginDescriptorTask\ pomRepository(String\ id,\ Object\ url)}
\textbar{} Adds the name and URL of a remote repository in the
\texttt{pom.xml} file.

\subsubsection{WriteMavenSettingsTask}\label{writemavensettingstask}

\paragraph{Task Properties}\label{task-properties-15}

Property Name \textbar{} Type \textbar{} Default Value \textbar{}
Description \texttt{localRepositoryDir} \textbar{} \texttt{String}
\textbar{} \texttt{null} \textbar{} The directory of the system's local
repository. It sets the value of the
\href{https://maven.apache.org/settings.html\#Simple_Values}{\texttt{localRepository}}
element in the generated \texttt{settings.xml} file.
\texttt{nonProxyHosts} \textbar{} \texttt{String} \textbar{}
\texttt{null} \textbar{} The patterns of the host that should be
accessed without going through the proxy. It sets the value of the
\href{https://maven.apache.org/settings.html\#Proxies}{\texttt{proxies.proxy.nonProxyHosts}}
element in the generated \texttt{settings.xml} file. \texttt{outputFile}
\textbar{} \texttt{File} \textbar{} \texttt{null} \textbar{} The
generated \texttt{settings.xml} file. \texttt{proxyHost} \textbar{}
\texttt{String} \textbar{} \texttt{null} \textbar{} The host name or
address of the proxy server. It sets the value of the
\href{https://maven.apache.org/settings.html\#Proxies}{\texttt{proxies.proxy.host}}
element in the generated \texttt{settings.xml} file.
\texttt{proxyPassword} \textbar{} \texttt{String} \textbar{}
\texttt{null} \textbar{} The password to use to access a protected proxy
server. It sets the value of the
\href{https://maven.apache.org/settings.html\#Proxies}{\texttt{proxies.proxy.password}}
element in the generated \texttt{settings.xml} file. \texttt{proxyPort}
\textbar{} \texttt{String} \textbar{} \texttt{null} \textbar{} The port
number of the proxy server. It sets the value of the
\href{https://maven.apache.org/settings.html\#Proxies}{\texttt{proxies.proxy.port}}
element in the generated \texttt{settings.xml} file. \texttt{proxyUser}
\textbar{} \texttt{String} \textbar{} \texttt{null} \textbar{} The user
name to use to access a protected proxy server. It sets the value of the
\href{https://maven.apache.org/settings.html\#Proxies}{\texttt{proxies.proxy.username}}
element in the generated \texttt{settings.xml} file.
\texttt{repositoryUrl} \textbar{} \texttt{String} \textbar{}
\texttt{null} \textbar{} The URL of the repository
\href{https://maven.apache.org/guides/mini/guide-mirror-settings.html\#Using_A_Single_Repository}{mirror}.
It sets the value of the
\href{https://maven.apache.org/settings.html\#Mirrors}{\texttt{mirrors.mirror.url}}
element in the generated \texttt{settings.xml} file.

The properties of type \texttt{File} support any type that can be
resolved by
\href{https://docs.gradle.org/current/dsl/org.gradle.api.Project.html\#org.gradle.api.Project:file(java.lang.Object)}{\texttt{project.file}}.
Moreover, it is possible to use Closures and Callables as values for the
\texttt{String} properties, to defer evaluation until task execution.

\subsection{Additional Configuration}\label{additional-configuration-8}

There are additional configurations that can help you use the Maven
Plugin Builder.

\subsubsection{Maven Embedder
Dependency}\label{maven-embedder-dependency}

By default, the plugin creates a configuration called
\texttt{mavenEmbedder} and adds a dependency to the 3.3.9 version of the
Maven Embedder. It is possible to override this setting and use a
specific version of the tool by manually adding a dependency to the
\texttt{mavenEmbedder} configuration:

\begin{verbatim}
dependencies {
    mavenEmbedder group: "org.apache.maven", name: "maven-embedder", version: "3.3.9"
    mavenEmbedder group: "org.apache.maven.wagon", name: "wagon-http", version: "2.10"
    mavenEmbedder group: "org.eclipse.aether", name: "aether-connector-basic", version: "1.0.2.v20150114"
    mavenEmbedder group: "org.eclipse.aether", name: "aether-transport-wagon", version: "1.0.2.v20150114"
    mavenEmbedder group: "org.slf4j", name: "slf4j-simple", version: "1.7.5"
}
\end{verbatim}

\subsubsection{System Properties}\label{system-properties-2}

It is possible to set the default value of the \texttt{mavenDebug}
property for a \texttt{BuildPluginDescriptorTask} task via system
property:

\begin{itemize}
\tightlist
\item
  \texttt{-D\$\{task.name\}.maven.debug=true}
\end{itemize}

For example, run the following Bash command to invoke the Maven Embedder
in debug mode to attach a remote debugger:

\begin{verbatim}
./gradlew buildPluginDescriptor -DbuildPluginDescriptor.maven.debug=true
\end{verbatim}

\section{Node Gradle Plugin}\label{node-gradle-plugin}

The Node Gradle plugin lets you run \href{https://nodejs.org/}{Node.js}
and \href{https://www.npmjs.com/}{NPM} as part of your build.

The plugin has been successfully tested with Gradle 4.10.2.

\subsection{Usage}\label{usage-16}

To use the plugin, include it in your build script:

\begin{verbatim}
buildscript {
    dependencies {
        classpath group: "com.liferay", name: "com.liferay.gradle.plugins.node", version: "4.6.18"
    }

    repositories {
        maven {
            url "https://repository-cdn.liferay.com/nexus/content/groups/public"
        }
    }
}

apply plugin: "com.liferay.node"
\end{verbatim}

\subsection{Project Extension}\label{project-extension-6}

The Node Gradle plugin exposes the following properties through the
extension named \texttt{node}:

Property Name \textbar{} Type \textbar{} Default Value \textbar{}
Description \texttt{download} \textbar{} \texttt{boolean} \textbar{}
\texttt{true} \textbar{} Whether to download and use a local and
isolated Node.js distribution instead of the one installed in the
system. \texttt{global} \textbar{} \texttt{boolean} \textbar{}
\texttt{false} \textbar{} Whether to use a single Node.js installation
for the whole multi-project build. This reduces the time required to
unpack the Node.js distribution and the time required to download NPM
packages thanks to a shared packages cache. If \texttt{download} is
\texttt{false}, this property has no effect. \texttt{nodeDir} \textbar{}
\texttt{File} \textbar{}

\textbf{If \texttt{global} is \texttt{true}:}
\texttt{"\$\{rootProject.buildDir\}/node"}

\textbf{Otherwise:} \texttt{"\$\{project.buildDir\}/node"}

\textbar{} The directory where the Node.js distribution is unpacked. If
\texttt{download} is \texttt{false}, this property has no effect.
\texttt{nodeUrl} \textbar{} \texttt{String} \textbar{}
\texttt{"http://nodejs.org/dist/v\$\{node.nodeVersion\}/node-v\$\{node.nodeVersion\}-\$\{platform\}-x\$\{bitMode\}.\$\{extension\}"}
\textbar{} The URL of the Node.js distribution to download. If
\texttt{download} is \texttt{false}, this property has no effect.
\texttt{nodeVersion} \textbar{} \texttt{String} \textbar{}
\texttt{"5.5.0"} \textbar{} The version of the Node.js distribution to
use. If \texttt{download} is \texttt{false}, this property has no
effect. \texttt{npmArgs} \textbar{}
\texttt{List\textless{}String\textgreater{}} \textbar{} \texttt{{[}{]}}
\textbar{} The arguments added automatically to every task of type
\hyperref[executenpmtask]{\texttt{ExecuteNpmTask}}. \texttt{npmUrl}
\textbar{} \texttt{String} \textbar{}
\texttt{"https://registry.npmjs.org/npm/-/npm-\$\{node.npmVersion\}.tgz"}
\textbar{} The URL of the NPM version to download. If \texttt{download}
is \texttt{false}, this property has no effect. \texttt{npmVersion}
\textbar{} \texttt{String} \textbar{} \texttt{null} \textbar{} The
version of NPM to use. If \texttt{null}, the version of NPM embedded
inside the Node.js distribution is used. If \texttt{download} is
\texttt{false}, this property has no effect.

It is possible to override the default value of the \texttt{download}
property by setting the \texttt{nodeDownload} project property. For
example, this can be done via command line argument:

\begin{verbatim}
./gradlew -PnodeDownload=false npmInstall
\end{verbatim}

The same extension exposes the following methods:

Method \textbar{} Description
\texttt{NodeExtension\ npmArgs(Iterable\textless{}?\textgreater{}\ npmArgs)}
\textbar{} Adds arguments to automatically add to every task of type
\hyperref[executenpmtask]{\texttt{ExecuteNpmTask}}.
\texttt{NodeExtension\ npmArgs(Object...\ npmArgs)} \textbar{} Adds
arguments to automatically add to every task of type
\hyperref[executenpmtask]{\texttt{ExecuteNpmTask}}.

The properties of type \texttt{File} support any type that can be
resolved by
\href{https://docs.gradle.org/current/dsl/org.gradle.api.Project.html\#org.gradle.api.Project:file(java.css.Object)}{\texttt{project.file}}.
Moreover, it is possible to use Closures and Callables as values for
\texttt{String}, to defer evaluation until execution.

Please note that setting the \texttt{global} property of the
\texttt{node} extension via the command line is not supported. It can
only be set via Gradle script, which can be done by adding the following
code to the \texttt{build.gradle} file in the root of a project (e.g.,
Liferay Workspace):

\begin{verbatim}
allprojects {
    plugins.withId("com.liferay.node") {
        node.global = true
    }
}
\end{verbatim}

\subsection{Tasks}\label{tasks-15}

The plugin adds a series of tasks to your project:

Name \textbar{} Depends On \textbar{} Type \textbar{} Description
\texttt{cleanNPM} \textbar{} - \textbar{}
\href{https://docs.gradle.org/current/dsl/org.gradle.api.tasks.Delete.html}{\texttt{Delete}}
\textbar{} Deletes the \texttt{node\_modules} directory, the
\texttt{npm-shrinkwrap.json} file and the \texttt{package-lock.json}
files from the project, if present. \texttt{downloadNode} \textbar{} -
\textbar{} \hyperref[downloadnodetask]{\texttt{DownloadNodeTask}}
\textbar{} Downloads and unpacks the local Node.js distribution for the
project. If \texttt{node.download} is \texttt{false}, this task is
disabled. \texttt{npmInstall} \textbar{} \texttt{downloadNode}
\textbar{} \hyperref[npminstalltask]{\texttt{NpmInstallTask}} \textbar{}
Runs \texttt{npm\ install} to install the dependencies declared in the
project's \texttt{package.json} file, if present. By default, the task
is \hyperref[npminstallretries]{configured} to run \texttt{npm\ install}
two more times if it fails.
\hyperref[npmrunscript-task]{\texttt{npmRun\$\{script\}}} \textbar{}
\texttt{npmInstall} \textbar{}
\hyperref[executenpmtask]{\texttt{ExecuteNpmTask}} \textbar{} Runs the
\texttt{\$\{script\}} NPM script. \texttt{npmPackageLock} \textbar{}
\texttt{cleanNPM}, \texttt{npmInstall} \textbar{}
\href{https://docs.gradle.org/current/javadoc/org/gradle/api/DefaultTask.html}{\texttt{DefaultTask}}
\textbar{} Deletes the NPM files and runs \texttt{npm\ install} to
install the dependencies declared in the project's \texttt{package.json}
file, if present. \texttt{npmShrinkwrap} \textbar{} \texttt{cleanNPM},
\texttt{npmInstall} \textbar{}
\hyperref[npmshrinkwraptask]{\texttt{NpmShrinkwrapTask}} \textbar{}
Locks down the versions of a package's dependencies in order to control
which dependency versions are used.

\subsubsection{DownloadNodeTask}\label{downloadnodetask}

The purpose of this task is to download and unpack a Node.js
distribution.

\paragraph{Task Properties}\label{task-properties-16}

Property Name \textbar{} Type \textbar{} Default Value \textbar{}
Description \texttt{nodeDir} \textbar{} \texttt{File} \textbar{}
\texttt{null} \textbar{} The directory where the Node.js distribution is
unpacked. \texttt{nodeExeUrl} \textbar{} \texttt{String} \textbar{}
\texttt{null} \textbar{} The URL of \texttt{node.exe} to download when
on Windows. \texttt{nodeUrl} \textbar{} \texttt{String} \textbar{}
\texttt{null} \textbar{} The URL of the Node.js distribution to
download. \texttt{npmUrl} \textbar{} \texttt{String} \textbar{}
\texttt{null} \textbar{} The URL of the NPM version to download.

The properties of type \texttt{File} support any type that can be
resolved by
\href{https://docs.gradle.org/current/dsl/org.gradle.api.Project.html\#org.gradle.api.Project:file(java.css.Object)}{\texttt{project.file}}.
Moreover, it is possible to use Closures and Callables as values for the
\texttt{String} properties, to defer evaluation until task execution.

\subsubsection{ExecuteNodeTask}\label{executenodetask}

This is the base task to run Node.js in a Gradle build. All tasks of
type \texttt{ExecuteNodeTask} automatically depend on
\hyperref[downloadnode]{\texttt{downloadNode}}.

\paragraph{Task Properties}\label{task-properties-17}

Property Name \textbar{} Type \textbar{} Default Value \textbar{}
Description \texttt{args} \textbar{}
\texttt{List\textless{}Object\textgreater{}} \textbar{} \texttt{{[}{]}}
\textbar{} The arguments for the Node.js invocation. \texttt{command}
\textbar{} \texttt{String} \textbar{} \texttt{"node"} \textbar{} The
file name of the executable to invoke. \texttt{environment} \textbar{}
\texttt{Map\textless{}Object,\ Object\textgreater{}} \textbar{}
\texttt{{[}{]}} \textbar{} The environment variables for the Node.js
invocation. \texttt{inheritProxy} \textbar{} \texttt{boolean} \textbar{}
\texttt{true} \textbar{} Whether to set the \texttt{http\_proxy},
\texttt{https\_proxy}, and \texttt{no\_proxy} environment variables in
the Node.js invocation based on the values of the system properties
\texttt{https.proxyHost}, \texttt{https.proxyPort},
\texttt{https.proxyUser}, \texttt{https.proxyPassword},
\texttt{https.nonProxyHosts}, \texttt{https.proxyHost},
\texttt{https.proxyPort}, \texttt{https.proxyUser},
\texttt{https.proxyPassword}, and \texttt{https.nonProxyHosts}. If these
environment variables are already set, their values will not be
overwritten. \texttt{nodeDir} \textbar{} \texttt{File} \textbar{}

\textbf{If \hyperref[download]{\texttt{node.download}} is
\texttt{true}:} \hyperref[nodedir]{\texttt{node.nodeDir}}

\textbf{Otherwise:} \texttt{null}

\textbar{} The directory that contains the executable to invoke. If
\texttt{null}, the executable must be available in the system
\texttt{PATH}. \texttt{npmInstallRetries} \textbar{} \texttt{int}
\textbar{} \texttt{0} \textbar{} The number of times the
\texttt{node\_modules} is deleted, the NPM cached data is verified
(\texttt{npm\ cache\ verify}), and \texttt{npm\ install} is retried in
case the Node.js invocation defined by this task fails. This can help
solving corrupted \texttt{node\_modules} directories by re-downloading
the project's dependencies. \texttt{workingDir} \textbar{} \texttt{File}
\textbar{} \texttt{project.projectDir} \textbar{} The working directory
to use in the Node.js invocation.

The properties of type \texttt{File} support any type that can be
resolved by
\href{https://docs.gradle.org/current/dsl/org.gradle.api.Project.html\#org.gradle.api.Project:file(java.css.Object)}{\texttt{project.file}}.
Moreover, it is possible to use Closures and Callables as values for the
\texttt{String} properties to defer evaluation until task execution.

\paragraph{Task Methods}\label{task-methods-8}

Method \textbar{} Description
\texttt{ExecuteNodeTask\ args(Iterable\textless{}?\textgreater{}\ args)}
\textbar{} Adds arguments for the Node.js invocation.
\texttt{ExecuteNodeTask\ args(Object...\ args)} \textbar{} Adds
arguments for the Node.js invocation.
\texttt{ExecuteNodeTask\ environment(Map\textless{}?,\ ?\textgreater{}\ environment)}
\textbar{} Adds environment variables for the Node.js invocation.
\texttt{ExecuteNodeTask\ environment(Object\ key,\ Object\ value)}
\textbar{} Adds an environment variable for the Node.js invocation.

\subsubsection{ExecuteNodeScriptTask}\label{executenodescripttask}

The purpose of this task is to execute a Node.js script. Tasks of type
\texttt{ExecuteNodeScriptTask} extend
\hyperref[executenodetask]{\texttt{ExecuteNodeTask}}.

\paragraph{Task Properties}\label{task-properties-18}

Property Name \textbar{} Type \textbar{} Default Value \textbar{}
Description \texttt{scriptFile} \textbar{} \texttt{File} \textbar{}
\texttt{null} \textbar{} The Node.js script to execute.

The properties of type \texttt{File} support any type that can be
resolved by
\href{https://docs.gradle.org/current/dsl/org.gradle.api.Project.html\#org.gradle.api.Project:file(java.css.Object)}{\texttt{project.file}}.

\subsubsection{ExecuteNpmTask}\label{executenpmtask}

The purpose of this task is to execute an NPM command. Tasks of type
\texttt{ExecuteNpmTask} extend
\hyperref[executenodescripttask]{\texttt{ExecuteNodeScriptTask}} with
the following properties set by default:

Property Name \textbar{} Default Value \texttt{command} \textbar{}

\textbf{If \texttt{nodeDir} is \texttt{null}:} \texttt{"npm"}

\textbf{Otherwise:} \texttt{"node"}

\texttt{scriptFile} \textbar{}

\textbf{If \texttt{nodeDir} is \texttt{null}:} \texttt{null}

\textbf{Otherwise:}
\texttt{"\$\{nodeDir\}/lib/node\_modules/npm/bin/npm-cli.js"} or
\texttt{"\$\{nodeDir\}/node\_modules/npm/bin/npm-cli.js"} on Windows.

\paragraph{Task Properties}\label{task-properties-19}

Property Name \textbar{} Type \textbar{} Default Value \textbar{}
Description \texttt{cacheConcurrent} \textbar{} \texttt{boolean}
\textbar{}

\textbf{If \texttt{node.npmVersion} is greater than or equal to
\texttt{5.0.0}, or \texttt{node.nodeVersion} is greater than or equal to
\texttt{8.0.0}:} \texttt{true}

\textbf{Otherwise:} \texttt{false}

\textbar{} Whether to run this task concurrently, in case the version of
NPM in use supports multiple concurrent accesses to the same cache
directory. \texttt{cacheDir} \textbar{} \texttt{File} \textbar{}

\textbf{If \texttt{nodeDir} is \texttt{null}, or
\texttt{node.npmVersion} is greater than or equal to \texttt{5.0.0}, or
\texttt{node.nodeVersion} is greater than or equal to \texttt{8.0.0}:}
\texttt{null}

\textbf{Otherwise:} \texttt{"\$\{nodeDir\}/.cache"}

\textbar{} The location of NPM's cache directory. It sets the
\href{https://docs.npmjs.com/misc/config\#cache}{\texttt{-\/-cache}}
argument. Leave the property \texttt{null} to keep the default value.
\texttt{logLevel} \textbar{} \texttt{String} \textbar{} Value to mirror
the log level set in the task's
\href{https://docs.gradle.org/current/dsl/org.gradle.api.Task.html\#org.gradle.api.Task:logger}{\texttt{logger}}
object. \textbar{} The NPM log level. It sets the
\href{https://docs.npmjs.com/misc/config\#loglevel}{--loglevel}
argument. \texttt{production} \textbar{} \texttt{boolean} \textbar{}
\texttt{false} \textbar{} Whether to run in production mode during the
NPM invocation. It sets the
\href{https://docs.npmjs.com/misc/config\#production}{\texttt{-\/-production}}
argument. \texttt{progress} \textbar{} \texttt{boolean} \textbar{}
\texttt{true} \textbar{} Whether to show a progress bar during the NPM
invocation. It sets the
\href{https://docs.npmjs.com/misc/config\#progress}{\texttt{-\/-progress}}
argument. \texttt{registry} \textbar{} \texttt{String} \textbar{}
\texttt{null} \textbar{} The base URL of the NPM package registry. It
sets the
\href{https://docs.npmjs.com/misc/config\#registry}{\texttt{-\/-registry}}
argument. Leave the property \texttt{null} or empty to keep the default
value.

The properties of type \texttt{File} support any type that can be
resolved by
\href{https://docs.gradle.org/current/dsl/org.gradle.api.Project.html\#org.gradle.api.Project:file(java.css.Object)}{\texttt{project.file}}.
Moreover, it is possible to use Closures and Callables as values for the
\texttt{String} properties, to defer evaluation until task execution.

\subsubsection{DownloadNodeModuleTask}\label{downloadnodemoduletask}

The purpose of this task is to download a Node.js package. The packages
are downloaded in the \texttt{\$\{workingDir\}/node\_modules} directory,
which is equal, by default, to the \texttt{node\_modules} directory of
the project. Tasks of type \texttt{DownloadNodeModuleTask} extend
\hyperref[executenpmtask]{\texttt{ExecuteNpmTask}} in order to execute
the command
\href{https://docs.npmjs.com/cli/install}{\texttt{npm\ install\ \$\{moduleName\}@\$\{moduleVersion\}}}.

\texttt{DownloadNodeModuleTask} instances are automatically disabled if
the project's \texttt{package.json} file already lists a module with the
same name in its \texttt{dependencies} or \texttt{devDependencies}
object.

\paragraph{Task Properties}\label{task-properties-20}

Property Name \textbar{} Type \textbar{} Default Value \textbar{}
Description \texttt{moduleName} \textbar{} \texttt{String} \textbar{}
\texttt{null} \textbar{} The name of the Node.js package to download.
\texttt{moduleVersion} \textbar{} \texttt{String} \textbar{}
\texttt{null} \textbar{} The version of the Node.js package to download.

It is possible to use Closures and Callables as values for the
\texttt{String} properties, to defer evaluation until task execution.

\subsubsection{NpmInstallTask}\label{npminstalltask}

Purpose of these tasks is to install the dependencies declared in a
\texttt{package.json} file. Tasks of type \texttt{NpmInstallTask} extend
\hyperref[executenpmtask]{\texttt{ExecuteNpmTask}} in order to run the
command
\href{https://docs.npmjs.com/cli/install}{\texttt{npm\ install}}.

\texttt{NpmInstallTask} instances are automatically disabled if the
\texttt{package.json} file does not declare any dependency in the
\texttt{dependency} or \texttt{devDependencies} object.

\paragraph{Task Properties}\label{task-properties-21}

Property Name \textbar{} Type \textbar{} Default Value \textbar{}
Description \texttt{nodeModulesCacheDir} \textbar{} \texttt{File}
\textbar{} \texttt{null} \textbar{}

The directory where \texttt{node\_modules} directories are cached. By
setting this property, it is possible to cache the
\texttt{node\_modules} directory of a project and avoid unnecessary
invocations of \texttt{npm\ install}, useful especially in Continuous
Integration environments.

The \texttt{node\_modules} directory is cached based on the content of
the project's \texttt{package-lock.json} (or
\texttt{npm-shrinkwrap.json}, or \texttt{package.json} if absent).
Therefore, if \texttt{NpmInstallTask} tasks in multiple projects are
configured with the same \texttt{nodeModulesCacheDir}, and their
\texttt{package-lock.json}, \texttt{npm-shrinkwrap.json} or
\texttt{package.json} declare the same dependencies, their
\texttt{node\_modules} caches will be shared.

This feature is not available if the
\href{https://github.com/liferay/liferay-portal/tree/master/modules/sdk/gradle-plugins-cache}{\texttt{com.liferay.cache}}
plugin is applied.

\texttt{nodeModulesCacheNativeSync} \textbar{} \texttt{boolean}
\textbar{} \texttt{true} \textbar{} Whether to use \texttt{rsync} (on
Linux/macOS) or \texttt{robocopy} (on Windows) to cache and restore the
\texttt{node\_modules} directory. If \texttt{nodeModulesCacheDir} is not
set, this property has no effect. \texttt{nodeModulesDigestFile}
\textbar{} \texttt{File} \textbar{} \texttt{null} \textbar{}

If this property is set, the content of the project's
\texttt{package-lock.json} (or \texttt{npm-shrinkwrap.json}, or
\texttt{package.json} if absent) is checked with the digest from the
\texttt{node\_modules} directory. If the digests match, do nothing. If
the digests don't match, the \texttt{node\_modules} directory is deleted
before running \texttt{npm\ install}.

This feature is not available if the \texttt{com.liferay.cache} plugin
is applied or if the property \texttt{nodeModulesCacheDir} is set.

\texttt{removeShrinkwrappedUrls} \textbar{} \texttt{boolean} \textbar{}
\texttt{true} if the \hyperref[registry]{registry} property has a value,
\texttt{false} otherwise. \textbar{} Whether to temporarily remove all
the hard-coded URLs in the \texttt{from} and \texttt{resolved} fields of
the \texttt{npm-shinkwrap.json} file before invoking
\texttt{npm\ install}. This way, it is possible to force NPM to download
all dependencies from a custom registry declared in the
\hyperref[registry]{\texttt{registry}} property. \texttt{useNpmCI}
\textbar{} \texttt{boolean} \textbar{} \texttt{false} \textbar{} Whether
to run \texttt{npm\ ci} instead of \texttt{npm\ install}. If the
\texttt{package-lock.json} file does not exist, this property has no
effect.

The properties of type \texttt{File} support any type that can be
resolved by
\href{https://docs.gradle.org/current/dsl/org.gradle.api.Project.html\#org.gradle.api.Project:file(java.css.Object)}{\texttt{project.file}}.

\subsubsection{NpmShrinkwrapTask}\label{npmshrinkwraptask}

The purpose of this task is to lock down the versions of a package's
dependencies so that you can control exactly which dependency versions
are used when your package is installed. Tasks of type
\texttt{NpmShrinkwrapTask} extend
\hyperref[executenpmtask]{\texttt{ExecuteNpmTask}} to execute the
command
\href{https://docs.npmjs.com/cli/shrinkwrap}{\texttt{npm\ shrinkwrap}}.

The generated \texttt{npm-shrinkwrap.json} file is automatically sorted
and formatted, so it's easier to see the changes with the previous
version.

\texttt{NpmShrinkwrapTask} instances are automatically disabled if the
\texttt{package.json} file does not exist.

\paragraph{Task Properties}\label{task-properties-22}

Property Name \textbar{} Type \textbar{} Default Value \textbar{}
Description \texttt{excludedDependencies} \textbar{}
\texttt{List\textless{}String\textgreater{}} \textbar{} \texttt{{[}{]}}
\textbar{} The package names to exclude from the generated
\texttt{npm-shrinkwrap.json} file. \texttt{includeDevDependencies}
\textbar{} \texttt{boolean} \textbar{} \texttt{true} \textbar{} Whether
to include the package's \texttt{devDependencies}. It sets the
\href{https://docs.npmjs.com/cli/shrinkwrap\#other-notes}{\texttt{-\/-dev}}
argument.

It is possible to use Closures and Callables as values for the
\texttt{String} properties to defer evaluation until task execution.

\paragraph{Task Methods}\label{task-methods-9}

Method \textbar{} Description
\texttt{NpmShrinkwrapTask\ excludeDependencies(Iterable\textless{}?\textgreater{}\ excludedDependencies)}
\textbar{} Adds package names to exclude from the generated
\texttt{npm-shrinkwrap.json} file.
\texttt{NpmShrinkwrapTask\ excludeDependencies(Object...\ excludedDependencies)}
\textbar{} Adds package names to exclude from the generated
\texttt{npm-shrinkwrap.json} file.

\subsubsection{PublishNodeModuleTask}\label{publishnodemoduletask}

The purpose of this task is to publish a package to the
\href{https://www.npmjs.com/}{NPM registry}. Tasks of type
\texttt{PublishNodeModuleTask} extend
\hyperref[executenpmtask]{\texttt{ExecuteNpmTask}} in order to execute
the command
\href{https://docs.npmjs.com/cli/publish}{\texttt{npm\ publish}}.

These tasks generate a new temporary \texttt{package.json} file in the
directory assigned to the \hyperref[workingdir]{\texttt{workingDir}}
property; then the \texttt{npm\ publish} command is executed. If the
\texttt{package.json} file in that location does not exist, the one in
the root of the project directory (if found) is copied; otherwise, a new
file is created.

The \texttt{package.json} is then processed by adding the values
provided by the task properties, if not already present in the file
itself. It is still possible to override one or more fields of the
\texttt{package.json} file with the values provided by the task
properties by adding one or more keys (e.g., \texttt{"version"}) to the
\texttt{overriddenPackageJsonKeys} property.

\paragraph{Task Properties}\label{task-properties-23}

Property Name \textbar{} Type \textbar{} Default Value \textbar{}
Description \texttt{moduleAuthor} \textbar{} \texttt{String} \textbar{}
\texttt{null} \textbar{} The value of the
\href{https://docs.npmjs.com/files/package.json\#people-fields-author-contributors}{\texttt{author}}
field in the generated \texttt{package.json} file.
\texttt{moduleBugsUrl} \textbar{} \texttt{String} \textbar{}
\texttt{null} \textbar{} The value of the
\href{https://docs.npmjs.com/files/package.json\#bugs}{\texttt{bugs.url}}
field in the generated \texttt{package.json} file.
\texttt{moduleDescription} \textbar{} \texttt{String} \textbar{}
\texttt{project.description} \textbar{} The value of the
\href{https://docs.npmjs.com/files/package.json\#description-1}{\texttt{description}}
field in the generated \texttt{package.json} file.
\texttt{moduleKeywords} \textbar{}
\texttt{List\textless{}String\textgreater{}} \textbar{} \texttt{{[}{]}}
\textbar{} The value of the
\href{https://docs.npmjs.com/files/package.json\#keywords}{\texttt{keywords}}
field in the generated \texttt{package.json} file.
\texttt{moduleLicense} \textbar{} \texttt{String} \textbar{}
\texttt{null} \textbar{} The value of the
\href{https://docs.npmjs.com/files/package.json\#license}{\texttt{license}}
field in the generated \texttt{package.json} file. \texttt{moduleMain}
\textbar{} \texttt{String} \textbar{} \texttt{null} \textbar{} The value
of the
\href{https://docs.npmjs.com/files/package.json\#main}{\texttt{main}}
field in the generated \texttt{package.json} file. \texttt{moduleName}
\textbar{} \texttt{String} \textbar{} Name based on
\href{https://github.com/gradle/gradle/blob/master/subprojects/osgi/src/main/java/org/gradle/api/internal/plugins/osgi/OsgiHelper.java}{\texttt{osgiHelper.bundleSymbolicName}}:
for example, if \texttt{osgiHelper.bundleSymbolicName} is
\texttt{"com.liferay.gradle.plugins.node"}, the default value for the
\texttt{moduleName} property is \texttt{"liferay-gradle-plugins-node"}.
\textbar{} The value of the
\href{https://docs.npmjs.com/files/package.json\#name}{\texttt{name}}
field in the generated \texttt{package.json} file.
\texttt{moduleRepository} \textbar{} \texttt{String} \textbar{}
\texttt{null} \textbar{} The value of the
\href{https://docs.npmjs.com/files/package.json\#repository}{\texttt{repository}}
field in the generated \texttt{package.json} file.
\texttt{moduleVersion} \textbar{} \texttt{String} \textbar{}
\texttt{project.version} \textbar{} The value of the
\href{https://docs.npmjs.com/files/package.json\#version}{\texttt{version}}
field in the generated \texttt{package.json} file.
\texttt{npmEmailAddress} \textbar{} \texttt{String} \textbar{}
\texttt{null} \textbar{} The email address of the npmjs.com user that
publishes the package. \texttt{npmPassword} \textbar{} \texttt{String}
\textbar{} \texttt{null} \textbar{} The password of the npmjs.com user
that publishes the package. \texttt{npmUserName} \textbar{}
\texttt{String} \textbar{} \texttt{null} \textbar{} The name of the
npmjs.com user that publishes the package.
\texttt{overriddenPackageJsonKeys} \textbar{}
\texttt{Set\textless{}String\textgreater{}} \textbar{} \texttt{{[}{]}}
\textbar{} The field values to override in the generated
\texttt{package.json} file.

\paragraph{Task Methods}\label{task-methods-10}

\begin{longtable}[]{@{}
  >{\raggedright\arraybackslash}p{(\columnwidth - 2\tabcolsep) * \real{0.3529}}
  >{\raggedright\arraybackslash}p{(\columnwidth - 2\tabcolsep) * \real{0.6471}}@{}}
\toprule\noalign{}
\begin{minipage}[b]{\linewidth}\raggedright
Method
\end{minipage} & \begin{minipage}[b]{\linewidth}\raggedright
Description
\end{minipage} \\
\midrule\noalign{}
\endhead
\bottomrule\noalign{}
\endlastfoot
\texttt{PublishNodeModuleTask\ overriddenPackageJsonKeys(Iterable\textless{}String\textgreater{}\ overriddenPackageJsonKeys)}
& Adds field values to override in the generated \texttt{package.json}
file. \\
\texttt{PublishNodeModuleTask\ overriddenPackageJsonKeys(String...\ overriddenPackageJsonKeys)}
& Adds field values to override in the generated \texttt{package.json}
file. \\
\end{longtable}

\subsubsection{npmRun\$\{script\} Task}\label{npmrunscript-task}

For each \href{https://docs.npmjs.com/misc/scripts}{script} declared in
the \texttt{package.json} file of the project, one task
\texttt{npmRun\$\{script\}} of type
\hyperref[executenpmtask]{\texttt{ExecuteNpmTask}} is added. Each of
these tasks is automatically configured with sensible defaults:

Property Name \textbar{} Default Value \texttt{args} \textbar{}
\texttt{{[}"run-script",\ "\$\{script\}"{]}}

If the
\href{https://docs.gradle.org/current/userguide/java_plugin.html}{\texttt{java}}
plugin is applied and the \texttt{package.json} file declares a script
named \texttt{"build"}, the script is executed before the
\texttt{classes} task but after the
\href{https://docs.gradle.org/4.0/userguide/java_plugin.html\#sec:java_resources}{\texttt{processResources}}
task.

If the
\href{https://docs.gradle.org/current/javadoc/org/gradle/language/base/plugins/LifecycleBasePlugin.html}{\texttt{lifecycle-base}}
plugin is applied and the \texttt{package.json} file declares a script
named \texttt{test}, the script is executed when running the
\texttt{check} task.

\section{Service Builder Gradle
Plugin}\label{service-builder-gradle-plugin}

The Service Builder Gradle plugin lets you generate a service layer
defined in a
\href{/docs/7-0/tutorials/-/knowledge_base/t/what-is-service-builder}{Service
Builder} \texttt{service.xml} file.

The plugin has been successfully tested with Gradle 4.10.2.

\subsection{Usage}\label{usage-17}

To use the plugin, include it in your build script:

\begin{verbatim}
buildscript {
    dependencies {
        classpath group: "com.liferay", name: "com.liferay.gradle.plugins.service.builder", version: "2.2.46"
    }

    repositories {
        maven {
            url "https://repository-cdn.liferay.com/nexus/content/groups/public"
        }
    }
}

apply plugin: "com.liferay.portal.tools.service.builder"
\end{verbatim}

The Service Builder plugin automatically applies the
\href{https://docs.gradle.org/current/userguide/java_plugin.html}{\texttt{java}}
plugin.

Since the plugin automatically resolves the
\href{https://github.com/liferay/liferay-portal/tree/master/modules/util/portal-tools-service-builder}{Liferay
Service Builder} library as a dependency, you have to configure a
repository that hosts the library and its transitive dependencies. The
Liferay CDN repository hosts them all:

\begin{verbatim}
repositories {
    maven {
        url "https://repository-cdn.liferay.com/nexus/content/groups/public"
    }
}
\end{verbatim}

\subsection{Tasks}\label{tasks-16}

The plugin adds one task to your project:

Name \textbar{} Depends On \textbar{} Type \textbar{} Description
\texttt{buildService} \textbar{} - \textbar{}
\hyperref[buildservicetask]{\texttt{BuildServiceTask}} \textbar{} Runs
the Liferay Service Builder.

The \texttt{buildService} task is automatically configured with sensible
defaults, depending on whether the
\href{https://docs.gradle.org/current/userguide/war_plugin.html}{\texttt{war}}
plugin is applied, or whether the
\hyperref[osgimodule]{\texttt{osgiModule}} property is \texttt{true}:

Property Name \textbar{} Default Value
\hyperref[apidir]{\texttt{apiDir}} \textbar{}

\textbf{If the \texttt{war} plugin is applied:}
\texttt{\$\{project.webAppDir\}/WEB-INF/service}

\textbf{Otherwise:} \texttt{null}

\hyperref[hbmfile]{\texttt{hbmFile}} \textbar{}

\textbf{If \texttt{osgiModule} is \texttt{true}:}
\texttt{\$\{buildService.resourcesDir\}/META-INF/module-hbm.xml}

\textbf{Otherwise:}
\texttt{\$\{buildService.resourcesDir\}/META-INF/portlet-hbm.xml}

\hyperref[impldir]{\texttt{implDir}} \textbar{} The first \texttt{java}
directory of the \texttt{main} source set (by default:
\texttt{src/main/java}). \hyperref[inputfile]{\texttt{inputFile}}
\textbar{}

\textbf{If the \texttt{war} plugin is applied:}
\texttt{\$\{project.webAppDir\}/WEB-INF/service.xml}

\textbf{Otherwise:} \texttt{\$\{project.projectDir\}/service.xml}

\hyperref[modelhintsfile]{\texttt{modelHintsFile}} \textbar{} The file
\texttt{META-INF/portlet-model-hints.xml} in the first
\texttt{resources} directory of the \texttt{main} source set (by
default: \texttt{src/main/resources/META-INF/portlet-model-hints.xml}).
\hyperref[pluginname]{\texttt{pluginName}} \textbar{}

\textbf{If \texttt{osgiModule} is \texttt{true}:} \texttt{""}

\textbf{Otherwise:} \texttt{project.name}

\hyperref[pluginname]{\texttt{propsUtil}} \textbar{}

\textbf{If \texttt{osgiModule} is \texttt{true}:}
\texttt{"\$\{bundleSymbolicName\}.util.ServiceProps"}The
\texttt{bundleSymbolicName} of the project is inferred via the
\href{https://github.com/gradle/gradle/blob/master/subprojects/osgi/src/main/java/org/gradle/api/internal/plugins/osgi/OsgiHelper.java}{\texttt{OsgiHelper}}
class.

\textbf{Otherwise:} \texttt{"com.liferay.util.service.ServiceProps"}

\hyperref[resourcesdir]{\texttt{resourcesDir}} \textbar{} The first
\texttt{resources} directory of the \texttt{main} source set (by
default: \texttt{src/main/resources}).
\hyperref[springfile]{\texttt{springFile}} \textbar{}

\textbf{If \texttt{osgiModule} is \texttt{true}:} the file
\texttt{META-INF/spring/module-spring.xml} in the first
\texttt{resources} directory of the \texttt{main} source set (by
default: \texttt{src/main/resources/META-INF/spring/module-spring.xml})

\textbf{Otherwise:} the file \texttt{META-INF/portlet-spring.xml} in the
first \texttt{resources} directory of the \texttt{main} source set (by
default: \texttt{src/main/resources/META-INF/portlet-spring.xml})

\hyperref[sqldir]{\texttt{sqlDir}} \textbar{}

\textbf{If the \texttt{war} plugin is applied:}
\texttt{\$\{project.webAppDir\}/WEB-INF/sql}

\textbf{Otherwise:} The directory \texttt{META-INF/sql} in the first
\texttt{resources} directory of the \texttt{main} source set (by
default: \texttt{src/main/resources/META-INF/sql}).

In the
\href{/docs/7-0/tutorials/-/knowledge_base/t/defining-an-object-relational-map-with-service-builder}{typical
scenario} of a data-driven Liferay OSGi application split in
\texttt{myapp-app}, \texttt{myapp-service} and \texttt{myapp-web}
modules, the \texttt{service.xml} file is usually contained in the root
directory of \texttt{myapp-service}. In the \texttt{build.gradle} of the
same module, it is enough to apply the
\texttt{com.liferay.service.builder} plugin \hyperref[usage]{as
described}, and then add the following snippet to enable the use of
Liferay Service Builder:

\begin{verbatim}
buildService {
    apiDir = "../myapp-api/src/main/java"
    testDir = "../myapp-test/src/testIntegration/java"
}
\end{verbatim}

While \texttt{apiDir} is required, the \texttt{testDir} property
assignment can be left out, in which case Arquillian-based integration
test classes are generated.

\subsubsection{BuildServiceTask}\label{buildservicetask}

Tasks of type \texttt{BuildWSDDTask} extend
\href{https://docs.gradle.org/current/dsl/org.gradle.api.tasks.JavaExec.html}{\texttt{JavaExec}},
so all its properties and methods, such as
\href{https://docs.gradle.org/current/dsl/org.gradle.api.tasks.JavaExec.html\#org.gradle.api.tasks.JavaExec:args(java.lang.Iterable)}{\texttt{args}}
and
\href{https://docs.gradle.org/current/dsl/org.gradle.api.tasks.JavaExec.html\#org.gradle.api.tasks.JavaExec:maxHeapSize}{\texttt{maxHeapSize}}
are available. They also have the following properties set by default:

Property Name \textbar{} Default Value
\href{https://docs.gradle.org/current/dsl/org.gradle.api.tasks.JavaExec.html\#org.gradle.api.tasks.JavaExec:args}{\texttt{args}}
\textbar{} Service Builder command line arguments
\href{https://docs.gradle.org/current/dsl/org.gradle.api.tasks.JavaExec.html\#org.gradle.api.tasks.JavaExec:classpath}{\texttt{classpath}}
\textbar{}
\hyperref[liferay-service-builder-dependency]{\texttt{project.configurations.serviceBuilder}}
\href{https://docs.gradle.org/current/dsl/org.gradle.api.tasks.JavaExec.html\#org.gradle.api.tasks.JavaExec:main}{\texttt{main}}
\textbar{}
\texttt{"com.liferay.portal.tools.service.builder.ServiceBuilder"}
\href{https://docs.gradle.org/current/dsl/org.gradle.api.tasks.JavaExec.html\#org.gradle.api.tasks.JavaExec:systemProperties}{\texttt{systemProperties}}
\textbar{} \texttt{{[}"file.encoding":\ "UTF-8"{]}}

\paragraph{Task Properties}\label{task-properties-24}

Property Name \textbar{} Type \textbar{} Default Value \textbar{}
Description \texttt{apiDir} \textbar{} \texttt{File} \textbar{}
\texttt{null} \textbar{} A directory where the service API Java source
files are generated. It sets the \texttt{service.api.dir} argument.
\texttt{autoImportDefaultReferences} \textbar{} \texttt{boolean}
\textbar{} \texttt{true} \textbar{} Whether to automatically add default
references, like \texttt{com.liferay.portal.ClassName},
\texttt{com.liferay.portal.Resource} and
\texttt{com.liferay.portal.User}, to the services. It sets the
\texttt{service.auto.import.default.references} argument.
\texttt{autoNamespaceTables} \textbar{} \texttt{boolean} \textbar{}
\texttt{true} \textbar{} Whether to prefix table names by the namespace
specified in the \texttt{service.xml} file. It sets the
\texttt{service.auto.namespace.tables} argument.
\texttt{beanLocatorUtil} \textbar{} \texttt{String} \textbar{}
\texttt{"com.liferay.util.bean.PortletBeanLocatorUtil"} \textbar{} The
fully qualified class name of a bean locator class to use in the
generated service classes. It sets the
\texttt{service.bean.locator.util} argument. \texttt{buildNumber}
\textbar{} \texttt{long} \textbar{} \texttt{1} \textbar{} A specific
value to assign the \texttt{build.number} property in the
\texttt{service.properties} file. It sets the
\texttt{service.build.number} argument. \texttt{buildNumberIncrement}
\textbar{} \texttt{boolean} \textbar{} \texttt{true} \textbar{} Whether
to automatically increment the \texttt{build.number} property in the
\texttt{service.properties} file by one at every service generation. It
sets the \texttt{service.build.number.increment} argument.
\texttt{databaseNameMaxLength} \textbar{} \texttt{int} \textbar{}
\texttt{30} \textbar{} The upper bound for database table and column
name lengths to ensure it works on all databases. It sets the
\texttt{service.database.name.max.length} argument. \texttt{hbmFile}
\textbar{} \texttt{File} \textbar{} \texttt{null} \textbar{} A Hibernate
Mapping file to generate. It sets the \texttt{service.hbm.file}
argument. \texttt{implDir} \textbar{} \texttt{File} \textbar{}
\texttt{null} \textbar{} A directory where the service Java source files
are generated. It sets the \texttt{service.impl.dir} argument.
\texttt{inputFile} \textbar{} \texttt{File} \textbar{} \texttt{null}
\textbar{} The project's \texttt{service.xml} file. It sets the
\texttt{service.input.file} argument. \texttt{modelHintsConfigs}
\textbar{} \texttt{Set} \textbar{}
\texttt{{[}"classpath*:META-INF/portal-model-hints.xml",\ "META-INF/portal-model-hints.xml",\ "classpath*:META-INF/ext-model-hints.xml",\ "classpath*:META-INF/portlet-model-hints.xml"{]}}
\textbar{} Paths to the
\href{/docs/7-0/tutorials/-/knowledge_base/t/customizing-model-entities-with-model-hints}{model
hints} files for Liferay Service Builder to use in generating the
service layer. It sets the \texttt{service.model.hints.configs}
argument. \texttt{modelHintsFile} \textbar{} \texttt{File} \textbar{}
\texttt{null} \textbar{} A model hints file for the project. It sets the
\texttt{service.model.hints.file} argument. \texttt{osgiModule}
\textbar{} \texttt{boolean} \textbar{} \texttt{false} \textbar{} Whether
to generate the service layer for OSGi modules. It sets the
\texttt{service.osgi.module} argument. \texttt{pluginName} \textbar{}
\texttt{String} \textbar{} \texttt{null} \textbar{} If specified, a
plugin can enable additional generation features, such as \texttt{Clp}
class generation, for non-OSGi modules. It sets the
\texttt{service.plugin.name} argument. \texttt{propsUtil} \textbar{}
\texttt{String} \textbar{} \texttt{null} \textbar{} The fully qualified
class name of the service properties util class to generate. It sets the
\texttt{service.props.util} argument. \texttt{readOnlyPrefixes}
\textbar{} \texttt{Set} \textbar{}
\texttt{{[}"fetch",\ "get",\ "has",\ "is",\ "load",\ "reindex",\ "search"{]}}
\textbar{} Prefixes of methods to consider read-only. It sets the
\texttt{service.read.only.prefixes} argument.
\texttt{resourceActionsConfigs} \textbar{} \texttt{Set} \textbar{}
\texttt{{[}"META-INF/resource-actions/default.xml",\ "resource-actions/default.xml"{]}}
\textbar{} Paths to the
\href{/docs/7-0/tutorials/-/knowledge_base/t/adding-permissions-to-resources}{resource
actions} files for Liferay Service Builder to use in generating the
service layer. It sets the \texttt{service.resource.actions.configs}
argument. \texttt{resourcesDir} \textbar{} \texttt{File} \textbar{}
\texttt{null} \textbar{} A directory where the service non-Java files
are generated. It sets the \texttt{service.resources.dir} argument.
\texttt{springFile} \textbar{} \texttt{File} \textbar{} \texttt{null}
\textbar{} A service Spring file to generate. It sets the
\texttt{service.spring.file} argument. \texttt{springNamespaces}
\textbar{} \texttt{Set} \textbar{} \texttt{{[}"beans"{]}} \textbar{}
Namespaces of Spring XML Schemas to add to the service Spring file. It
sets the \texttt{service.spring.namespaces} argument. \texttt{sqlDir}
\textbar{} \texttt{File} \textbar{} \texttt{null} \textbar{} A directory
where the SQL files are generated. It sets the \texttt{service.sql.dir}
argument. \texttt{sqlFileName} \textbar{} \texttt{String} \textbar{}
\texttt{"tables.sql"} \textbar{} A name (relative to \texttt{sqlDir})
for the file in which the SQL table creation instructions are generated.
It sets the \texttt{service.sql.file} argument.
\texttt{sqlIndexesFileName} \textbar{} \texttt{String} \textbar{}
\texttt{"indexes.sql"} \textbar{} A name (relative to \texttt{sqlDir})
for the file in which the SQL index creation instructions are generated.
It sets the \texttt{service.sql.indexes.file} argument.
\texttt{sqlSequencesFileName} \textbar{} \texttt{String} \textbar{}
\texttt{"sequences.sql"} \textbar{} A name (relative to \texttt{sqlDir})
for the file in which the SQL sequence creation instructions are
generated. It sets the \texttt{service.sql.sequences.file} argument.
\texttt{targetEntityName} \textbar{} \texttt{String} \textbar{}
\texttt{null} \textbar{} If specified, it's the name of the entity for
which Liferay Service Builder should generate the service. It sets the
\texttt{service.target.entity.name} argument. \texttt{testDir}
\textbar{} \texttt{File} \textbar{} \texttt{null} \textbar{} If
specified, it's a directory where integration test Java source files are
generated. It sets the \texttt{service.test.dir} argument.
\texttt{uadDir} \textbar{} \texttt{File} \textbar{} \texttt{null}
\textbar{} A directory where the UAD (user-associated data) Java source
files are generated. It sets the \texttt{service.uad.dir} argument.
\texttt{uadTestIntegrationDir} \textbar{} \texttt{File} \textbar{}
\texttt{null} \textbar{} A directory where integration test UAD
(user-associated data) Java source files are generated. It sets the
\texttt{service.uad.test.integration.dir} argument.

The properties of type \texttt{File} supports any type that can be
resolved by
\href{https://docs.gradle.org/current/dsl/org.gradle.api.Project.html\#org.gradle.api.Project:file(java.lang.Object)}{\texttt{project.file}}.
Moreover, it is possible to use Closures and Callables as values for the
\texttt{String} properties, to defer evaluation until task execution.

\subsection{Additional Configuration}\label{additional-configuration-9}

There are additional configurations that can help you use Service
Builder.

\subsubsection{Liferay Service Builder
Dependency}\label{liferay-service-builder-dependency}

By default, the plugin creates a configuration called
\texttt{serviceBuilder} and adds a dependency to the latest released
version of Liferay Service Builder. It is possible to override this
setting and use a specific version of the tool by manually adding a
dependency to the \texttt{serviceBuilder} configuration:

\begin{verbatim}
dependencies {
    serviceBuilder group: "com.liferay", name: "com.liferay.portal.tools.service.builder", version: "1.0.292"
}
\end{verbatim}

If you're applying the
\href{https://github.com/liferay/liferay-portal/tree/master/modules/sdk/gradle-plugins}{\texttt{com.liferay.gradle.plugins}}
or
\href{https://github.com/liferay/liferay-portal/blob/master/modules/sdk/gradle-plugins-workspace}{\texttt{com.liferay.gradle.plugins.workspace}}
plugins to your project, the Service Builder dependency is already added
to the \texttt{serviceBuilder} configuration. Therefore, if you try to
apply a customized version of Service Builder, it's not recognized; you
must override the configuration already applied.

To do this, you must customize the classpath of the
\texttt{buildService} task. If you're supplying the customized Service
Builder plugin through a module named \texttt{custom-sb-api}, you could
modify the \texttt{buildService} task like this:

\begin{verbatim}
buildService {
    apiDir = "../custom-sb-api/src/main/java"
    classpath = configurations.serviceBuilder.filter { file -> !file.name.contains("com.liferay.portal.tools.service.builder") }
}
\end{verbatim}

If you do this in conjunction with the \texttt{serviceBuilder}
dependency configuration, the custom Service Builder version is used.

\section{Source Formatter Gradle
Plugin}\label{source-formatter-gradle-plugin}

The Source Formatter Gradle plugin lets you format project files using
the
\href{https://github.com/liferay/liferay-portal/tree/master/modules/util/source-formatter}{Liferay
Source Formatter} tool.

The plugin has been successfully tested with Gradle 4.10.2.

\subsection{Usage}\label{usage-18}

To use the plugin, include it in your build script:

\begin{verbatim}
buildscript {
    dependencies {
        classpath group: "com.liferay", name: "com.liferay.gradle.plugins.source.formatter", version: "2.3.413"
    }

    repositories {
        maven {
            url "https://repository-cdn.liferay.com/nexus/content/groups/public"
        }
    }
}

apply plugin: "com.liferay.source.formatter"
\end{verbatim}

Since the plugin automatically resolves the Liferay Source Formatter
library as a dependency, you have to configure a repository that hosts
the library and its transitive dependencies. The Liferay CDN repository
hosts them all:

\begin{verbatim}
repositories {
    maven {
        url "https://repository-cdn.liferay.com/nexus/content/groups/public"
    }
}
\end{verbatim}

\subsection{Tasks}\label{tasks-17}

The plugin adds two tasks to your project:

Name \textbar{} Depends On \textbar{} Type \textbar{} Description
\texttt{checkSourceFormatting} \textbar{} - \textbar{}
\hyperref[formatsourcetask]{\texttt{FormatSourceTask}} \textbar{} Runs
the Liferay Source Formatter to check for source formatting errors.
\texttt{formatSource} \textbar{} - \textbar{}
\hyperref[formatsourcetask]{\texttt{FormatSourceTask}} \textbar{} Runs
the Liferay Source Formatter to format the project files.

If desired, it is possible to check for source formatting errors while
executing the
\href{https://docs.gradle.org/current/userguide/java_plugin.html\#N15056}{\texttt{check}}
task by adding the following dependency:

\begin{verbatim}
check {
    dependsOn checkSourceFormatting
}
\end{verbatim}

The same can be achieved by adding the following snippet to the
\texttt{build.gradle} file in the root directory of a
\href{/docs/7-0/tutorials/-/knowledge_base/t/liferay-workspace}{\emph{Liferay
Workspace}}:

\begin{verbatim}
subprojects {
    afterEvaluate {
        if (plugins.hasPlugin("base") && plugins.hasPlugin("com.liferay.source.formatter")) {
            check.dependsOn checkSourceFormatting
        }
    }
}
\end{verbatim}

The tasks \texttt{checkSourceFormatting} and \texttt{formatSource} are
automatically skipped if another task with the same name is being
executed in a parent project.

\subsubsection{FormatSourceTask}\label{formatsourcetask}

Tasks of type \texttt{FormatSourceTask} extend
\href{https://docs.gradle.org/current/dsl/org.gradle.api.tasks.JavaExec.html}{\texttt{JavaExec}},
so all its properties and methods, like
\href{https://docs.gradle.org/current/dsl/org.gradle.api.tasks.JavaExec.html\#org.gradle.api.tasks.JavaExec:args(java.lang.Iterable)}{\texttt{args}}
and
\href{https://docs.gradle.org/current/dsl/org.gradle.api.tasks.JavaExec.html\#org.gradle.api.tasks.JavaExec:maxHeapSize}{\texttt{maxHeapSize}}
are available. They also have the following properties set by default:

Property Name \textbar{} Default Value
\href{https://docs.gradle.org/current/dsl/org.gradle.api.tasks.JavaExec.html\#org.gradle.api.tasks.JavaExec:args}{\texttt{args}}
\textbar{} Source Formatter command line arguments
\href{https://docs.gradle.org/current/dsl/org.gradle.api.tasks.JavaExec.html\#org.gradle.api.tasks.JavaExec:classpath}{\texttt{classpath}}
\textbar{}
\hyperref[liferay-source-formatter-dependency]{\texttt{project.configurations.sourceFormatter}}
\href{https://docs.gradle.org/current/dsl/org.gradle.api.tasks.JavaExec.html\#org.gradle.api.tasks.JavaExec:main}{\texttt{main}}
\textbar{} \texttt{"com.liferay.source.formatter.SourceFormatter"}

\paragraph{Task Properties}\label{task-properties-25}

Property Name \textbar{} Type \textbar{} Default Value \textbar{}
Description \texttt{autoFix} \textbar{} \texttt{boolean} \textbar{}
\texttt{false} \textbar{} Whether to automatically fix source formatting
errors. It sets the \texttt{source.auto.fix} argument. \texttt{baseDir}
\textbar{} \texttt{File} \textbar{} \textbar{} The Source Formatter base
directory. It sets the \texttt{source.base.dir} argument.
\emph{(Read-only)} \texttt{baseDirName} \textbar{} \texttt{String}
\textbar{} \texttt{"./"} \textbar{} The name of the Source Formatter
base directory, relative to the project directory.
\texttt{fileExtensions} \textbar{}
\texttt{List\textless{}String\textgreater{}} \textbar{} \texttt{{[}{]}}
\textbar{} The file extensions to format. If empty, all file extensions
will be formatted. It sets the \texttt{source.file.extensions} argument.
\texttt{files} \textbar{} \texttt{List\textless{}File\textgreater{}}
\textbar{} \textbar{} The list of files to format. It sets the
\texttt{source.files} argument. \emph{(Read-only)} \texttt{fileNames}
\textbar{} \texttt{List\textless{}String\textgreater{}} \textbar{}
\texttt{null} \textbar{} The file names to format, relative to the
project directory. If \texttt{null}, all files contained in
\texttt{baseDir} will be formatted. \texttt{formatCurrentBranch}
\textbar{} \texttt{boolean} \textbar{} \texttt{false} \textbar{} Whether
to format only the files contained in \texttt{baseDir} that are added or
modified in the current Git branch. It sets the
\texttt{format.current.branch} argument. \texttt{formatLatestAuthor}
\textbar{} \texttt{boolean} \textbar{} \texttt{false} \textbar{} Whether
to format only the files contained in \texttt{baseDir} that are added or
modified in the latest Git commits of the same author. It sets the
\texttt{format.latest.author} argument. \texttt{formatLocalChanges}
\textbar{} \texttt{boolean} \textbar{} \texttt{false} \textbar{} Whether
to format only the unstaged files contained in \texttt{baseDir}. It sets
the \texttt{format.local.changes} argument.
\texttt{gitWorkingBranchName} \textbar{} \texttt{String} \textbar{}
\texttt{"master"} \textbar{} The Git working branch name. It sets the
\texttt{git.working.branch.name} argument.
\texttt{includeSubrepositories} \textbar{} \texttt{boolean} \textbar{}
\texttt{false} \textbar{} Whether to format files that are in read-only
subrepositories. It sets the \texttt{include.subrepositories} argument.
\texttt{maxLineLength} \textbar{} \texttt{int} \textbar{} \texttt{80}
\textbar{} The maximum number of characters allowed in Java files. It
sets the \texttt{max.line.length} argument. \texttt{printErrors}
\textbar{} \texttt{boolean} \textbar{} \texttt{true} \textbar{} Whether
to print formatting errors on the Standard Output stream. It sets the
\texttt{source.print.errors} argument. \texttt{processorThreadCount}
\textbar{} \texttt{int} \textbar{} \texttt{5} \textbar{} The number of
threads used by Source Formatter. It sets the
\texttt{processor.thread.count} argument. \texttt{showDebugInformation}
\textbar{} \texttt{boolean} \textbar{} \texttt{false} \textbar{} Whether
to show debug information, if present. It sets the
\texttt{show.debug.information} argument. \texttt{showDocumentation}
\textbar{} \texttt{boolean} \textbar{} \texttt{false} \textbar{} Whether
to show the documentation for the source formatting issues, if present.
It sets the \texttt{show.documentation} argument.
\texttt{showStatusUpdates} \textbar{} \texttt{boolean} \textbar{}
\texttt{false} \textbar{} Whether to show status updates during source
formatting, if present. It sets the \texttt{show.status.updates}
argument. \texttt{throwException} \textbar{} \texttt{boolean} \textbar{}
\texttt{false} \textbar{} Whether to fail the build if formatting errors
are found. It sets the \texttt{source.throw.exception} argument.

\subsection{Additional Configuration}\label{additional-configuration-10}

There are additional configurations that can help you use the Source
Formatter.

\subsubsection{Liferay Source Formatter
Dependency}\label{liferay-source-formatter-dependency}

By default, the plugin creates a configuration called
\texttt{sourceFormatter} and adds a dependency to the latest released
version of Liferay Source Formatter. It is possible to override this
setting and use a specific version of the tool by manually adding a
dependency to the \texttt{sourceFormatter} configuration:

\begin{verbatim}
dependencies {
    sourceFormatter group: "com.liferay", name: "com.liferay.source.formatter", version: "1.0.885"
}
\end{verbatim}

\subsubsection{System Properties}\label{system-properties-3}

It is possible to set the default values of the \texttt{fileExtensions},
\texttt{fileNames}, \texttt{formatCurrentBranch},
\texttt{formatLatestAuthor}, and \texttt{formatLocalChanges} properties
for a \texttt{FormatSourceTask} task via system properties:

\begin{itemize}
\tightlist
\item
  \texttt{-D\$\{task.name\}.file.extensions=java,xml}
\item
  \texttt{-D\$\{task.name\}.file.names=README.markdown,src/main/resources/hello.txt}
\item
  \texttt{-D\$\{task.name\}.format.current.branch=true}
\item
  \texttt{-D\$\{task.name\}.format.latest.author=true}
\item
  \texttt{-D\$\{task.name\}.format.local.changes=true}
\end{itemize}

For example, run the following Bash command to format only the unstaged
files in the project:

\begin{verbatim}
./gradlew formatSource -DformatSource.format.local.changes=true
\end{verbatim}

\section{Soy Gradle Plugin}\label{soy-gradle-plugin}

The Soy Gradle plugin lets you compile
\href{https://developers.google.com/closure/templates/}{Closure
Templates} into JavaScript functions. It also lets you use a custom
localization mechanism in the generated \texttt{.soy.js} files by
replacing
\href{https://developers.google.com/closure/templates/docs/translation\#closurecompiler}{\texttt{goog.getMsg}}
definitions with a different function call (e.g.,
\texttt{Liferay.Language.get}).

The plugin has been successfully tested with Gradle 4.10.2.

\subsection{Usage}\label{usage-19}

To use the plugin, include it in your build script:

\begin{verbatim}
buildscript {
    dependencies {
        classpath group: "com.liferay", name: "com.liferay.gradle.plugins.soy", version: "3.1.8"
    }

    repositories {
        maven {
            url "https://repository-cdn.liferay.com/nexus/content/groups/public"
        }
    }
}
\end{verbatim}

There are two Soy Gradle plugins you can apply to your project:

\begin{itemize}
\item
  Apply the \hyperref[soy-plugin]{\emph{Soy Plugin}} to compile Closure
  Templates into JavaScript functions:

\begin{verbatim}
apply plugin: "com.liferay.soy"
\end{verbatim}
\item
  Apply the \hyperref[soy-translation-plugin]{\emph{Soy Translation
  Plugin}} to use a custom localization mechanism in the generated
  \texttt{.soy.js} files:

\begin{verbatim}
apply plugin: "com.liferay.soy.translation"
\end{verbatim}
\end{itemize}

Since the Soy Gradle plugin automatically resolves the Soy library as a
dependency, you have to configure a repository that hosts the library
and its transitive dependencies. The Liferay CDN repository hosts them
all:

\begin{verbatim}
repositories {
    maven {
        url "https://repository-cdn.liferay.com/nexus/content/groups/public"
    }
}
\end{verbatim}

\subsection{Soy Plugin}\label{soy-plugin}

The Soy plugin adds two tasks to your project:

Name \textbar{} Depends On \textbar{} Type \textbar{} Description
\texttt{buildSoy} \textbar{} - \textbar{}
\hyperref[buildsoytask]{\texttt{BuildSoyTask}} \textbar{} Compiles
Closure Templates into JavaScript functions.
\texttt{wrapSoyAlloyTemplate} \textbar{} - \texttt{configJSModules} if
\href{https://github.com/liferay/liferay-portal/tree/master/modules/sdk/gradle-plugins-js-module-config-generator}{\texttt{com.liferay.js.module.config.generator}}
is applied - \texttt{processResources} if \texttt{java} is applied -
\texttt{transpileJS} if
\href{https://github.com/liferay/liferay-portal/tree/master/modules/sdk/gradle-plugins-js-transpiler}{\texttt{com.liferay.js.transpiler}}
is applied \textbar{}
\hyperref[wrapsoyalloytemplatetask]{\texttt{WrapSoyAlloyTemplateTask}}
\textbar{} Wraps the JavaScript functions compiled from Closure
Templates into AlloyUI modules.

The plugin also adds the following dependencies to tasks defined by the
\texttt{java} plugin:

Name \textbar{} Depends On \texttt{classes} \textbar{}
\texttt{wrapSoyAlloyTemplate}

The \texttt{buildSoy} task is automatically configured with sensible
defaults, depending on whether the
\href{https://docs.gradle.org/current/userguide/java_plugin.html}{\texttt{java}}
plugin is applied:

Property Name \textbar{} Default Value
\href{https://docs.gradle.org/current/dsl/org.gradle.api.tasks.SourceTask.html\#org.gradle.api.tasks.SourceTask:includes}{\texttt{includes}}
\textbar{} \texttt{{[}"**/*.soy"{]}}
\href{https://docs.gradle.org/current/dsl/org.gradle.api.tasks.SourceTask.html\#org.gradle.api.tasks.SourceTask:source}{\texttt{source}}
\textbar{}

\textbf{If the \texttt{java} plugin is applied:} The first
\texttt{resources} directory of the \texttt{main} source set (by
default, \texttt{src/main/resources}).

\textbf{Otherwise:} \texttt{{[}{]}}

The \texttt{wrapSoyAlloyTemplate} task is \textbf{disabled by default},
and it is automatically configured with sensible defaults, depending on
whether the \texttt{java} plugin is applied:

Property Name \textbar{} Default Value
\href{https://docs.gradle.org/current/dsl/org.gradle.api.Task.html\#org.gradle.api.Task:enabled}{\texttt{enabled}}
\textbar{} \texttt{false}
\href{https://docs.gradle.org/current/dsl/org.gradle.api.tasks.SourceTask.html\#org.gradle.api.tasks.SourceTask:includes}{\texttt{includes}}
\textbar{} \texttt{{[}"**/*.soy.js"{]}}
\href{https://docs.gradle.org/current/dsl/org.gradle.api.tasks.SourceTask.html\#org.gradle.api.tasks.SourceTask:source}{\texttt{source}}
\textbar{}

\textbf{If the \texttt{java} plugin is applied:}
\texttt{project.sourceSets.main.output.resourcesDir}

\textbf{Otherwise:} \texttt{{[}{]}}

\subsubsection{Additional
Configuration}\label{additional-configuration-11}

There are additional configurations that can help you use the Soy
library.

\paragraph{Soy Dependency}\label{soy-dependency}

By default, the plugin creates a configuration called \texttt{soy} and
adds a dependency to the \texttt{2015-04-10} version of the Soy library.
It is possible to override this setting and use a specific version of
the tool by manually adding a dependency to the \texttt{soy}
configuration:

\begin{verbatim}
dependencies {
    soy group: "com.google.template", name: "soy", version: "2015-04-10"
}
\end{verbatim}

\subsection{Soy Translation Plugin}\label{soy-translation-plugin}

The Soy Translation plugin adds one task to your project:

Name \textbar{} Depends On \textbar{} Type \textbar{} Description
\texttt{replaceSoyTranslation} \textbar{} - \texttt{configJSModules} if
\href{https://github.com/liferay/liferay-portal/tree/master/modules/sdk/gradle-plugins-js-module-config-generator}{\texttt{com.liferay.js.module.config.generator}}
is applied - \texttt{processResources} if \texttt{java} is applied -
\texttt{transpileJS} if
\href{https://github.com/liferay/liferay-portal/tree/master/modules/sdk/gradle-plugins-js-transpiler}{\texttt{com.liferay.js.transpiler}}
is applied \textbar{}
\hyperref[replacesoytranslationtask]{\texttt{ReplaceSoyTranslationTask}}
\textbar{} Replaces \texttt{goog.getMsg} definitions with
\texttt{Liferay.Language.get} calls.

The plugin also adds the following dependencies to tasks defined by the
\texttt{java} plugin:

Name \textbar{} Depends On \texttt{classes} \textbar{}
\texttt{replaceSoyTranslation}

The \texttt{replaceSoyTranslation} task is automatically configured with
sensible defaults, depending on whether the \texttt{java} plugin is
applied:

Property Name \textbar{} Default Value
\href{https://docs.gradle.org/current/dsl/org.gradle.api.tasks.SourceTask.html\#org.gradle.api.tasks.SourceTask:includes}{\texttt{includes}}
\textbar{} \texttt{{[}"**/*.soy.js"{]}}
\hyperref[replacementclosure]{\texttt{replacementClosure}} \textbar{}
Replaces \texttt{goog.getMsg} definitions with
\texttt{Liferay.Language.get} calls.
\href{https://docs.gradle.org/current/dsl/org.gradle.api.tasks.SourceTask.html\#org.gradle.api.tasks.SourceTask:source}{\texttt{source}}
\textbar{}

\textbf{If the \texttt{java} plugin is applied:}
\texttt{project.sourceSets.main.output.resourcesDir}

\textbf{Otherwise:} \texttt{{[}{]}}

\subsection{Tasks}\label{tasks-18}

\subsubsection{BuildSoyTask}\label{buildsoytask}

Tasks of type \texttt{BuildSoyTask} extend
\href{https://docs.gradle.org/current/dsl/org.gradle.api.tasks.SourceTask.html}{\texttt{SourceTask}},
so all its properties and methods, such as
\href{https://docs.gradle.org/current/dsl/org.gradle.api.tasks.SourceTask.html\#org.gradle.api.tasks.SourceTask:include(java.lang.Iterable)}{\texttt{include}}
and
\href{https://docs.gradle.org/current/dsl/org.gradle.api.tasks.SourceTask.html\#org.gradle.api.tasks.SourceTask:exclude(java.lang.Iterable)}{\texttt{exclude}},
are available.

\paragraph{Task Properties}\label{task-properties-26}

Property Name \textbar{} Type \textbar{} Default Value \textbar{}
Description \texttt{classpath} \textbar{}
\href{https://docs.gradle.org/current/javadoc/org/gradle/api/file/FileCollection.html}{\texttt{FileCollection}}
\textbar{}
\hyperref[soy-dependency]{\texttt{project.configurations.soy}}
\textbar{} The classpath for executing the
\href{https://github.com/liferay/liferay-portal/tree/master/modules/util/portal-tools-soy-builder}{Liferay
Portal Tools Soy Builder}.

\subsubsection{WrapSoyAlloyTemplateTask}\label{wrapsoyalloytemplatetask}

Tasks of type \texttt{WrapSoyAlloyTemplateTask} extend
\href{https://docs.gradle.org/current/dsl/org.gradle.api.tasks.SourceTask.html}{\texttt{SourceTask}},
so all its properties and methods, such as
\href{https://docs.gradle.org/current/dsl/org.gradle.api.tasks.SourceTask.html\#org.gradle.api.tasks.SourceTask:include(java.lang.Iterable)}{\texttt{include}}
and
\href{https://docs.gradle.org/current/dsl/org.gradle.api.tasks.SourceTask.html\#org.gradle.api.tasks.SourceTask:exclude(java.lang.Iterable)}{\texttt{exclude}},
are available.

\paragraph{Task Properties}\label{task-properties-27}

Property Name \textbar{} Type \textbar{} Default Value \textbar{}
Description \texttt{moduleName} \textbar{} \texttt{String} \textbar{}
\texttt{null} \textbar{} The name of the AlloyUI module.
\texttt{namespace} \textbar{} \texttt{String} \textbar{} \texttt{null}
\textbar{} The namespace of the Closure Templates of the project.

It is possible to use Closures and Callables as values for the
\texttt{String} properties to defer evaluation until task execution.

\subsubsection{ReplaceSoyTranslationTask}\label{replacesoytranslationtask}

The \texttt{ReplaceSoyTranslationTask} task type finds all the
\texttt{goog.getMsg} definitions in the project's files and replaces
them with a custom function call.

\begin{verbatim}
var MSG_EXTERNAL_123 = goog.getMsg('welcome-to-{$releaseInfo}', { 'releaseInfo': opt_data.releaseInfo });
\end{verbatim}

A \texttt{goog.getMsg} definition looks like the example above, and it
has the following components:

\begin{itemize}
\tightlist
\item
  \emph{variable name}: \texttt{MSG\_EXTERNAL\_123}
\item
  \emph{language key}: \texttt{welcome-to-\{\$releaseInfo\}}
\item
  \emph{arguments object}:
  \texttt{\{\ \textquotesingle{}releaseInfo\textquotesingle{}:\ opt\_data.releaseInfo\ \}}
\end{itemize}

Tasks of type \texttt{ReplaceSoyTranslationTask} extend
\href{https://docs.gradle.org/current/dsl/org.gradle.api.tasks.SourceTask.html}{\texttt{SourceTask}},
so all its properties and methods, such as
\href{https://docs.gradle.org/current/dsl/org.gradle.api.tasks.SourceTask.html\#org.gradle.api.tasks.SourceTask:include(java.lang.Iterable)}{\texttt{include}}
and
\href{https://docs.gradle.org/current/dsl/org.gradle.api.tasks.SourceTask.html\#org.gradle.api.tasks.SourceTask:exclude(java.lang.Iterable)}{\texttt{exclude}},
are available.

\paragraph{Task Properties}\label{task-properties-28}

Property Name \textbar{} Type \textbar{} Default Value \textbar{}
Description \texttt{replacementClosure} \textbar{}
\texttt{Closure\textless{}String\textgreater{}} \textbar{} \texttt{null}
\textbar{} The Closure invoked in order to get the replacement for
\texttt{goog.getMsg} definitions. The given Closure is passed the
\emph{variable name}, \emph{language key}, and \emph{arguments object}
as its parameters.

\section{Target Platform Gradle
Plugin}\label{target-platform-gradle-plugin}

The Target Platform Gradle plugin helps with building multiple projects
against a declared API target platform. Java dependencies can be managed
with Maven BOMs and OSGi modules can be resolved against an OSGi
distribution.

The plugin has been successfully tested with Gradle 4.10.2.

\subsection{Usage}\label{usage-20}

To use the plugin, include it in your build script:

\begin{verbatim}
buildscript {
    dependencies {
        classpath group: "com.liferay", name: "com.liferay.gradle.plugins.target.platform", version: "1.1.13"
    }

    repositories {
        maven {
            url "https://repository-cdn.liferay.com/nexus/content/groups/public"
        }
    }
}
\end{verbatim}

There are two Target Platform Gradle plugins you can apply to your
project:

\begin{itemize}
\item
  The \hyperref[target-platform-plugin]{\emph{Target Platform Plugin}}
  helps to configure your projects to build against an established set
  of platform artifacts, including Java and OSGi dependencies.

\begin{verbatim}
apply plugin: "com.liferay.target.platform"
\end{verbatim}
\item
  The \hyperref[target-platform-ide-plugin]{\emph{Target Platform IDE
  Plugin}} is a superset of the Target Platform Plugin (it applies the
  above plugin) and also adds IDE integration for searching and
  debugging source code in the target platform artifacts.

\begin{verbatim}
apply plugin: "com.liferay.target.platform.ide"
\end{verbatim}
\end{itemize}

Since the plugin automatically resolves target platform configurations
as dependencies, you must configure a repository that hosts these
artifacts. The Liferay CDN repository hosts them all:

\begin{verbatim}
repositories {
    maven {
        url "https://repository-cdn.liferay.com/nexus/content/groups/public"
    }
}
\end{verbatim}

\subsection{Target Platform Plugin}\label{target-platform-plugin}

The plugin applies the
\href{https://github.com/spring-gradle-plugins/dependency-management-plugin}{Spring
Dependency Management Plugin} and then adds several specific
configurations to configure the BOMs that are imported to manage Java
dependencies and the various artifacts used in resolving OSGi
dependencies. Also, a new \texttt{resolve} task is added to resolve all
OSGi requirements against a declared distribution artifact.

The plugin adds a series of configurations to your project:

Name \textbar{} Description \texttt{targetPlatformBOMs} \textbar{}
Configures all the BOMs to import as managed dependencies.
\texttt{targetPlatformBundles} \textbar{} Configures all the bundles in
addition to the distro to resolve against. \texttt{targetPlatformDistro}
\textbar{} Configures the distro JAR file to use as base for resolving
against. \texttt{targetPlatformRequirements} \textbar{} Configures the
list of JAR files to use as run requirements for resolving.

The plugin adds a task \texttt{resolve} of type
\hyperref[resolvetask]{\texttt{ResolveTask}} to your project that
performs an OSGi resolve operation using the
\texttt{targetPlatformRequirements} configuration as the basis of the
requirements. The \texttt{targetPlatformBundles} configuration is used
as a repository for the resolver to resolve requirements. Lastly, the
\texttt{targetPlatformDistro} configuration is used to provide the
\emph{distro} artifact for the resolve process. The \emph{distro} is the
artifact that provides all the OSGi capabilities of the target platform.
All of these parameters are used to create a \texttt{bndrun} file that
can be used as input into the Bndrun resolve operation.

\subsection{Target Platform IDE
Plugin}\label{target-platform-ide-plugin}

The plugin applies the \hyperref[target-platform-plugin]{Target
Platform} and the
\href{https://docs.gradle.org/current/userguide/eclipse_plugin.html}{\texttt{eclipse}}
plugins to your project, and also adds a special
\texttt{targetPlatformIDE} configuration, which is used to configure
both the \texttt{eclipse} and \texttt{idea} plugin model in Gradle to
add all target platform artifacts to the classpath so they are visible
to both Eclipse and IntelliJ's Java Model Search (for looking up sources
to classes).

\subsection{Project Extension}\label{project-extension-7}

The Target Platform plugin exposes the following properties through the
extension named \texttt{targetPlatform}:

Property Name \textbar{} Type \textbar{} Default Value \textbar{}
Description \texttt{ignoreResolveFailures} \textbar{} \texttt{boolean}
\textbar{} \texttt{true} \textbar{} Whether to ignore resolve failures
found when executing tasks of type
\hyperref[resolvetask]{\texttt{ResolveTask}}. \texttt{subprojects}
\textbar{} \texttt{Set\textless{}Project\textgreater{}} \textbar{}
\texttt{project.subprojects} \textbar{} The subprojects to configure
with target platform support, including dependency management and the
\texttt{resolve} task.

The same extension exposes the following methods:

Method \textbar{} Description
\texttt{TargetPlatformExtension\ applyToConfiguration(Iterable\textless{}?\textgreater{}\ configurationNames)}
\textbar{} Adds additional configurations to configure the BOMs that are
imported to manage Java dependencies and the various artifacts used in
resolving OSGi dependencies.
\texttt{TargetPlatformExtension\ applyToConfiguration(Object...\ configurationNames)}
\textbar{} Adds additional configurations to configure the BOMs that are
imported to manage Java dependencies and the various artifacts used in
resolving OSGi dependencies.
\texttt{TargetPlatformExtension\ onlyIf(Closure\textless{}Boolean\textgreater{}\ onlyIfClosure)}
\textbar{} Includes a subproject in the target platform configuration if
the given closure returns \texttt{true}. The closure is evaluated at the
end of the subproject configuration phase and is passed a single
parameter: the subproject. If the closure returns \texttt{false}, the
subproject is not included in the target platform configuration
\texttt{TargetPlatformExtension\ onlyIf(Spec\textless{}Project\textgreater{}\ onlyIfSpec)}
\textbar{} Includes a subproject in the target platform configuration if
the given spec is satisfied. The spec is evaluated at the end of the
subproject configuration phase. If the spec is not satisfied, the
subproject is not included in the target platform configuration.
\texttt{TargetPlatformExtension\ resolveOnlyIf(Closure\textless{}Boolean\textgreater{}\ resolveOnlyIfClosure)}
\textbar{} Includes a subproject in the resolving process (including
both the requirements and bundles configuration) if the given closure
returns \texttt{true}. The closure is evaluated at the end of the
subproject configuration phase and is passed a single parameter: the
subproject. If the closure returns \texttt{false}, the subproject is the
resolution process.
\texttt{TargetPlatformExtension\ resolveOnlyIf(Spec\textless{}Project\textgreater{}\ resolveOnlyIfSpec)}
\textbar{} Includes a subproject in the resolving platform configuration
if the given spec is satisfied. The spec is evaluated at the end of the
subproject configuration phase. If the spec is not satisfied, the
subproject is not included in the target platform configuration.
\texttt{TargetPlatformExtension\ subprojects(Iterable\textless{}Project\textgreater{}\ subprojects)}
\textbar{} Includes additional projects to be configured with Target
Platform support.
\texttt{TargetPlatformExtension\ subprojects(Project...\ subprojects)}
\textbar{} Includes additional projects to be configured with Target
Platform support.

\subsection{Tasks}\label{tasks-19}

\subsubsection{ResolveTask}\label{resolvetask}

The purpose of this task is to resolve an OSGi module (or all OSGi
modules of subprojects) against the available
\texttt{targetPlatformBundles} and \texttt{targetPlatformDistro}
configurations. By default, the \texttt{targetPlatformBundles} are all
the artifacts created by all the subprojects. The
\texttt{targetPlatformDistro} must be set explicitly to a valid
distribution artifact. When the task is performed, a \texttt{bndrun}
file is generated using the specified \texttt{targetPlatformDistro} as
the \texttt{-distro} instruction; the \texttt{-runrequirements} are a
set of \texttt{osgi.identity} requirements for the
\texttt{targetPlatformRequirements} configuration. If the resolve
operation is able to find a valid set of \texttt{-runbundles} that match
the \texttt{-runrequirements}, then the task passes successfully (the
resolution is valid). If a set of run bundles can't be found, the
resolution has failed and the failed requirements are listed as output
of the task.

\paragraph{Task Properties}\label{task-properties-29}

Property Name \textbar{} Type \textbar{} Default Value \textbar{}
Description \texttt{bndrunFile} \textbar{} \texttt{File} \textbar{}
\texttt{null} \textbar{} If this property is specified, it is used as
the \texttt{bndrun} file to input into the resolver.
\texttt{bundlesFileCollection} \textbar{} \texttt{FileCollection}
\textbar{} All JAR files of subprojects with \texttt{jar} task
\textbar{} The input to \texttt{bndrun} resolve operation.
\texttt{distroFileCollection} \textbar{} \texttt{FileCollection}
\textbar{} \texttt{null} \textbar{} The \emph{distro} parameter for the
generated \texttt{bndrun} file. \texttt{ignoreFailures} \textbar{}
\texttt{boolean} \textbar{} \texttt{false} \textbar{} Whether the
\texttt{resolve} task should ignore failing the build for resolution
errors. \texttt{offline} \textbar{} \texttt{boolean} \textbar{}
\texttt{null} \textbar{} Whether to run the bndrun resolve operation in
offline mode. \texttt{requirementsFileCollection} \textbar{}
\texttt{FileCollection} \textbar{}

\textbf{For the root project:} All the output JAR files of the
subprojects.

\textbf{For subprojects:} The output JAR file of the subproject.

\textbar{} For each resolve operation, the requirements must be
specified in the \texttt{bndrun} file; each of the JARs in this
collection generate an \texttt{osgi.identify} requirement in the
\texttt{bndrun} file.

\subsection{Additional Configuration}\label{additional-configuration-12}

There are additional configurations that you can use to configure the
target platform.

\subsubsection{Target Platform BOMs
Dependency}\label{target-platform-boms-dependency}

The plugin creates a configuration called \texttt{targetPlatformBOMs}
with no defaults. You can use this dependency to set which BOMs to
import to configure your target platform.

\begin{verbatim}
dependencies {
    targetPlatformBOMs group: "com.liferay", name: "com.liferay.ce.portal.bom", version: "7.1.0"
    targetPlatformBOMs group: "com.liferay", name: "com.liferay.ce.portal.compile.only", version: "7.1.0"
}
\end{verbatim}

\subsubsection{Target Platform Bundles
Dependency}\label{target-platform-bundles-dependency}

The plugin creates a configuration called
\texttt{targetPlatformBundles}. It is configured with default
dependencies to all resolvable bundles in a multi-project build (e.g.,
all projects in multi-project build that have a \texttt{jar} task). This
can be used to specify additional bundles that should be added to the
set of bundles given to \texttt{resolve} task to resolve against when
checking for OSGi requirements.

\begin{verbatim}
dependencies {
    targetPlatformBundles group: "com.google.guava", name: "guava", version: "23.0"
}
\end{verbatim}

\subsubsection{Target Platform Distro
Dependency}\label{target-platform-distro-dependency}

The plugin creates a configuration called \texttt{targetPlatformDistro}.
It is has no default so you must specify which artifact you want to use
as the distribution to resolve against.

\begin{verbatim}
dependencies {
    targetPlatformDistro group: "com.liferay", name: "com.liferay.ce.portal.distro", version: "7.1.0"
}
\end{verbatim}

If you have created your own custom distro JAR that is available
locally, you can use the \texttt{files} method to add it to the
configuration.

\begin{verbatim}
dependencies {
    targetPlatformDistro files("custom-distro.jar")
}
\end{verbatim}

\subsubsection{Target Platform Requirements
Dependency}\label{target-platform-requirements-dependency}

The plugin creates a configuration called
\texttt{targetPlatformRequirements}. It is configured with default
dependencies to all resolvable bundles in a multi-project build (e.g.,
all projects in multi-project build that have a \texttt{jar} task). This
is can be used to specify additional bundles that should be added to the
set of bundles given to the \texttt{resolve} task to set as
\texttt{osgi.identity} requirements.

\begin{verbatim}
dependencies {
    targetPlatformRequirements group: "com.liferay", name: "com.liferay.other.bundle", version: "1.0"
}
\end{verbatim}

\section{Theme Builder Gradle Plugin}\label{theme-builder-gradle-plugin}

The Theme Builder Gradle plugin lets you run the
\href{https://github.com/liferay/liferay-portal/tree/master/modules/util/portal-tools-theme-builder}{Liferay
Theme Builder} tool to build the Liferay theme files in your project.

The plugin has been successfully tested with Gradle 4.10.2.

\subsection{Usage}\label{usage-21}

To use the plugin, include it in your build script:

\begin{verbatim}
buildscript {
    dependencies {
        classpath group: "com.liferay", name: "com.liferay.gradle.plugins.theme.builder", version: "2.0.7"
    }

    repositories {
        maven {
            url "https://repository-cdn.liferay.com/nexus/content/groups/public"
        }
    }
}

apply plugin: "com.liferay.portal.tools.theme.builder"
\end{verbatim}

The Theme Builder plugin automatically applies the
\href{https://docs.gradle.org/current/userguide/war_plugin.html}{\texttt{war}}
plugin. It also applies the
\href{https://github.com/liferay/liferay-portal/tree/master/modules/sdk/gradle-plugins-css-builder}{\texttt{com.liferay.css.builder}}
plugin to compile the \href{http://sass-lang.com/}{Sass} files in the
theme.

Since the plugin automatically resolves the Liferay Theme Builder
library as a dependency, you have to configure a repository that hosts
the library and its transitive dependencies. The Liferay CDN repository
hosts them all:

\begin{verbatim}
repositories {
    maven {
        url "https://repository-cdn.liferay.com/nexus/content/groups/public"
    }
}
\end{verbatim}

\subsection{Tasks}\label{tasks-20}

The plugin adds one task to your project:

Name \textbar{} Depends On \textbar{} Type \textbar{} Description
\texttt{buildTheme} \textbar{} - \textbar{}
\hyperref[buildthemetask]{\texttt{BuildThemeTask}} \textbar{} Builds the
theme files.

The plugin also adds the following dependencies to tasks defined by the
\texttt{com.liferay.css.builder} and \texttt{war} plugins:

Name \textbar{} Depends On
\href{https://github.com/liferay/liferay-portal/tree/master/modules/sdk/gradle-plugins-css-builder\#tasks}{\texttt{buildCSS}}
\textbar{} \texttt{buildTheme}
\href{https://docs.gradle.org/current/userguide/war_plugin.html\#sec:war_default_settings}{\texttt{war}}
\textbar{} \texttt{buildTheme}

The \texttt{buildCSS} dependency compiles the Sass files contained in
the directory specified by the
\hyperref[outputdir]{\texttt{buildTheme.outputDir}} property. Moreover,
the \texttt{war} task is configured as follows

\begin{itemize}
\tightlist
\item
  exclude the directory specified in the
  \hyperref[diffsdir]{\texttt{buildTheme.diffsDir}} property from the
  WAR file.
\item
  include the files contained in the
  \hyperref[outputdir]{\texttt{buildTheme.outputDir}} directory into the
  WAR file.
\item
  include only the compiled CSS files, not SCSS files, into the WAR
  file.
\end{itemize}

The \texttt{buildTheme} task is automatically configured with sensible
defaults:

Property Name \textbar{} Default Value
\hyperref[diffsdir]{\texttt{diffsDir}} \textbar{}
\texttt{project.webAppDir} \hyperref[outputdir]{\texttt{outputDir}}
\textbar{} \texttt{"\$\{project.buildDir\}/buildTheme"}
\hyperref[parentfile]{\texttt{parentFile}} \textbar{} The first JAR file
in the \hyperref[parent-theme-dependencies]{\texttt{parentThemes}}
configuration that contains a
\texttt{META-INF/resources/\$\{buildTheme.parentName\}} directory, or
the first WAR file in the \texttt{parentThemes} configuration whose name
starts with \texttt{\$\{parentName\}-theme-}.
\hyperref[parentname]{\texttt{parentName}} \textbar{}
\texttt{"\_styled"}
\hyperref[templateextension]{\texttt{templateExtension}} \textbar{}
\texttt{"ftl"} \hyperref[themename]{\texttt{themeName}} \textbar{}
\texttt{project.name} \hyperref[unstyledfile]{\texttt{unstyledFile}}
\textbar{} The first JAR file in the
\hyperref[parent-theme-dependencies]{\texttt{parentThemes}}
configuration that contains a \texttt{META-INF/resources/\_unstyled}
directory.

\subsubsection{BuildThemeTask}\label{buildthemetask}

Tasks of type \texttt{BuildThemeTask} extend
\href{https://docs.gradle.org/current/dsl/org.gradle.api.tasks.JavaExec.html}{\texttt{JavaExec}},
so all its properties and methods, such as
\href{https://docs.gradle.org/current/dsl/org.gradle.api.tasks.JavaExec.html\#org.gradle.api.tasks.JavaExec:args(java.css.Iterable)}{\texttt{args}}
and
\href{https://docs.gradle.org/current/dsl/org.gradle.api.tasks.JavaExec.html\#org.gradle.api.tasks.JavaExec:maxHeapSize}{\texttt{maxHeapSize}},
are available. They also have the following properties set by default:

Property Name \textbar{} Default Value
\href{https://docs.gradle.org/current/dsl/org.gradle.api.tasks.JavaExec.html\#org.gradle.api.tasks.JavaExec:args}{\texttt{args}}
\textbar{} Theme Builder command line arguments
\href{https://docs.gradle.org/current/dsl/org.gradle.api.tasks.JavaExec.html\#org.gradle.api.tasks.JavaExec:classpath}{\texttt{classpath}}
\textbar{}
\hyperref[liferay-theme-builder-dependency]{\texttt{project.configurations.themeBuilder}}
\href{https://docs.gradle.org/current/dsl/org.gradle.api.tasks.JavaExec.html\#org.gradle.api.tasks.JavaExec:main}{\texttt{main}}
\textbar{}
\texttt{"com.liferay.portal.tools.theme.builder.ThemeBuilder"}

\paragraph{Task Properties}\label{task-properties-30}

Property Name \textbar{} Type \textbar{} Default Value \textbar{}
Description \texttt{diffsDir} \textbar{} \texttt{File} \textbar{}
\texttt{null} \textbar{} The directory that contains the files to copy
over the parent theme. It sets the \texttt{-\/-diffs-dir} argument.
\texttt{outputDir} \textbar{} \texttt{File} \textbar{} \texttt{null}
\textbar{} The directory where to build the theme. It sets the
\texttt{-\/-output-dir} argument. \texttt{parentDir} \textbar{}
\texttt{File} \textbar{} \texttt{null} \textbar{} The directory of the
parent theme. It sets the \texttt{-\/-parent-path} argument.
\texttt{parentFile} \textbar{} \texttt{File} \textbar{} \texttt{null}
\textbar{} The JAR file of the parent theme. If \texttt{parentDir} is
specified, this property has no effect. It sets the
\texttt{-\/-parent-path} argument. \texttt{parentName} \textbar{}
\texttt{String} \textbar{} \texttt{null} \textbar{} The name of the
parent theme. It sets the \texttt{-\/-parent-name} argument.
\texttt{templateExtension} \textbar{} \texttt{String} \textbar{}
\texttt{null} \textbar{} The extension of the template files, usually
\texttt{"ftl"} or \texttt{"vm"}. It sets the
\texttt{-\/-template-extension} argument. \texttt{themeName} \textbar{}
\texttt{String} \textbar{} \texttt{null} \textbar{} The name of the new
theme. It sets the \texttt{-\/-name} argument. \texttt{unstyledDir}
\textbar{} \texttt{File} \textbar{} \texttt{null} \textbar{} The
directory of
\href{https://github.com/liferay/liferay-portal/tree/master/modules/apps/foundation/frontend-theme/frontend-theme-unstyled}{Liferay
Frontend Theme Unstyled}. It sets the \texttt{-\/-unstyled-dir}
argument. \texttt{unstyledFile} \textbar{} \texttt{File} \textbar{}
\texttt{null} \textbar{} The JAR file of
\href{https://github.com/liferay/liferay-portal/tree/master/modules/apps/foundation/frontend-theme/frontend-theme-unstyled}{Liferay
Frontend Theme Unstyled}. If \texttt{unstyledDir} is specified, this
property has no effect. It sets the \texttt{-\/-unstyled-dir} argument.

The properties of type \texttt{File} support any type that can be
resolved by
\href{https://docs.gradle.org/current/dsl/org.gradle.api.Project.html\#org.gradle.api.Project:file(java.css.Object)}{\texttt{project.file}}.
Moreover, it is possible to use Closures and Callables as values for the
\texttt{String} properties to defer evaluation until task execution.

\subsection{Additional Configuration}\label{additional-configuration-13}

There are additional configurations that can help you use the CSS
Builder.

\subsubsection{Liferay Theme Builder
Dependency}\label{liferay-theme-builder-dependency}

By default, the plugin creates a configuration called
\texttt{themeBuilder} and adds a dependency to the latest released
version of the Liferay Theme Builder. It is possible to override this
setting and use a specific version of the tool by manually adding a
dependency to the \texttt{themeBuilder} configuration:

\begin{verbatim}
dependencies {
    themeBuilder group: "com.liferay", name: "com.liferay.portal.tools.theme.builder", version: "1.1.7"
}
\end{verbatim}

\subsubsection{Parent Theme
Dependencies}\label{parent-theme-dependencies}

By default, the plugin creates a configuration called
\texttt{parentThemes} and adds dependencies to the latest released
versions of the
\href{https://github.com/liferay/liferay-portal/tree/master/modules/apps/foundation/frontend-theme/frontend-theme-styled}{Liferay
Frontend Theme Styled},
\href{https://github.com/liferay/liferay-portal/tree/master/modules/apps/foundation/frontend-theme/frontend-theme-unstyled}{Liferay
Frontend Theme Unstyled}, and
\href{https://github.com/liferay/liferay-portal/tree/master/modules/apps/foundation/frontend-theme/frontend-theme-classic}{Liferay
Frontend Theme Classic} artifacts. It is possible to override this
setting and use a specific version of the artifacts by manually adding
dependencies to the \texttt{parentThemes} configuration. For example,

\begin{verbatim}
dependencies {
    parentThemes group: "com.liferay", name: "com.liferay.frontend.theme.styled", version: "VERSION"
    parentThemes group: "com.liferay", name: "com.liferay.frontend.theme.unstyled", version: "VERSION"
    parentThemes group: "com.liferay.plugins", name: "classic-theme", version: "VERSION"
}
\end{verbatim}

Specifying dependency versions is not required when leveraging
workspace's
\href{/docs/7-0/tutorials/-/knowledge_base/t/managing-the-target-platform-for-liferay-workspace}{Target
Platform} functionality. All dependencies with the group ID
\texttt{com.liferay} or \texttt{com.liferay.portal} are automatically
set when targeting a platform. For external theme dependencies (e.g.,
\texttt{classic-theme} with the group ID \texttt{com.liferay.plugins}),
you can find the version used by your specific Liferay DXP instance by
leveraging the
\href{/docs/7-0/reference/-/knowledge_base/r/using-the-felix-gogo-shell}{Gogo
shell}. In a Gogo shell prompt, execute the following command:

\begin{verbatim}
lb -s theme
\end{verbatim}

This lists the deployed theme bundles and their versions. Extract the
versions for the theme dependencies you want to leverage and add them to
your configuration.

\section{TLDDoc Builder Gradle
Plugin}\label{tlddoc-builder-gradle-plugin}

The TLDDoc Builder Gradle plugin lets you run the
\href{http://web.archive.org/web/20070624180825/https://taglibrarydoc.dev.java.net/}{Tag
Library Documentation Generator} tool in order to generate documentation
for the JSP Tag Library Descriptor (TLD) files in your project.

The plugin has been successfully tested with Gradle 4.10.2.

\subsection{Usage}\label{usage-22}

To use the plugin, include it in your build script:

\begin{verbatim}
buildscript {
    dependencies {
        classpath group: "com.liferay", name: "com.liferay.gradle.plugins.tlddoc.builder", version: "1.3.3"
    }

    repositories {
        maven {
            url "https://repository-cdn.liferay.com/nexus/content/groups/public"
        }
    }
}
\end{verbatim}

There are two TLDDoc Builder Gradle plugins you can apply to your
project:

\begin{itemize}
\item
  Apply the \hyperref[tlddoc-builder-plugin]{\emph{TLDDoc Builder
  Plugin}} to generate tag library documentation for your project:

\begin{verbatim}
apply plugin: "com.liferay.tlddoc.builder"
\end{verbatim}
\item
  Apply the \hyperref[app-tlddoc-builder-plugin]{\emph{App TLDDoc
  Builder Plugin}} in a parent project to generate the tag library
  documentation as a single, combined HTML document for an application
  that spans different subprojects, each one representing a different
  component of the same application:

\begin{verbatim}
apply plugin: "com.liferay.app.tlddoc.builder"
\end{verbatim}
\end{itemize}

Since the plugin automatically resolves the Tag Library Documentation
Generator library as a dependency, you must configure a repository that
hosts the library and its transitive dependencies. The Liferay CDN
repository hosts them all:

\begin{verbatim}
repositories {
    maven {
        url "https://repository-cdn.liferay.com/nexus/content/groups/public"
    }
}
\end{verbatim}

\subsection{TLDDoc Builder Plugin}\label{tlddoc-builder-plugin}

The plugin adds three tasks to your project:

Name \textbar{} Depends On \textbar{} Type \textbar{} Description
\texttt{copyTLDDocResources} \textbar{} - \textbar{}
\href{https://docs.gradle.org/current/dsl/org.gradle.api.tasks.Copy.html}{\texttt{Copy}}
\textbar{} Copies the tag library documentation resources from
\texttt{src/main/tlddoc} to the \hyperref[destinationdir]{destination
directory} of the \texttt{tlddoc} task. \texttt{tlddoc} \textbar{}
\texttt{copyTLDDocResources}, \texttt{validateTLD} \textbar{}
\hyperref[tlddoctask]{\texttt{TLDDocTask}} \textbar{} Generates the tag
library documentation. \texttt{validateTLD} \textbar{} - \textbar{}
\hyperref[validateschematask]{\texttt{ValidateSchemaTask}} \textbar{}
Validates the TLD files in the project.

The \texttt{tlddoc} task is automatically configured with sensible
defaults, depending on whether the
\href{https://docs.gradle.org/current/userguide/java_plugin.html}{\texttt{java}}
plugin is applied:

Property Name \textbar{} Default Value with the \texttt{java} plugin
\hyperref[destinationdir]{\texttt{destinationDir}} \textbar{}
\texttt{\$\{project.docsDir\}/tlddoc}
\hyperref[includes]{\texttt{includes}} \textbar{}
\texttt{{[}"**/*.tld"{]}} \hyperref[source]{\texttt{source}} \textbar{}
\texttt{project.sourceSets.main.resources.srcDirs}

The \texttt{validateTLD} task is also automatically configured with
sensible defaults, depending on whether the \texttt{java} plugin is
applied:

Property Name \textbar{} Default Value \texttt{includes} \textbar{}

\textbf{If the \texttt{java} plugin is applied:}
\texttt{{[}"**/*.tld"{]}}

\textbf{Otherwise:} \texttt{{[}{]}}

\texttt{source} \textbar{}

\textbf{If the \texttt{java} plugin is applied:}
\texttt{project.sourceSets.main.resources.srcDirs}

\textbf{Otherwise:} \texttt{null}

By default, the \texttt{tlddoc} task generates the documentation for all
the TLD files that are found in the resources directories of the
\texttt{main} source set. The documentation files are saved in
\texttt{build/docs/tlddoc} and include the files copied from
\texttt{src/main/tlddoc}.

The \texttt{copyTLDDocResources} task lets you add references to images
and other resources directly in the TLD files. For example, if the
project includes an image called \texttt{breadcrumb.png} in the
\texttt{src/main/tlddoc/images} directory, you can reference it in a TLD
file contained in the \texttt{src/main/resources} directory:

\begin{verbatim}
<description>Hello World <![CDATA[<img src="../images/breadcrumb.png"]]></description>
\end{verbatim}

\subsection{App TLDDoc Builder Plugin}\label{app-tlddoc-builder-plugin}

In order to use the App TLDDoc Builder plugin, it is required to apply
the \texttt{com.liferay.app.tlddoc.builder} plugin in a parent project
(that is, a project that is a common ancestor of all the subprojects
representing the various components of the app). It is also required to
apply the
\hyperref[tlddoc-builder-plugin]{\texttt{com.liferay.tlddoc.builder}}
plugin to all the subprojects that contain TLD files.

The App TLDDoc Builder plugin automatically applies the
\href{https://docs.gradle.org/current/userguide/standard_plugins.html\#N135C1}{\texttt{base}}
plugin. It also adds three tasks to your project:

Name \textbar{} Depends On \textbar{} Type \textbar{} Description
\texttt{appTLDDoc} \textbar{} \texttt{copyAppTLDDocResources}, the
\hyperref[validatetld]{\texttt{validateTLD}} tasks of the subprojects
\textbar{} \hyperref[tlddoctask]{\texttt{TLDDocTask}} \textbar{}
Generates tag library documentation for the app.
\texttt{copyAppTLDDocResources} \textbar{} - \textbar{}
\href{https://docs.gradle.org/current/dsl/org.gradle.api.tasks.Copy.html}{\texttt{Copy}}
\textbar{} Copies the tag library documentation resources defined as
\href{https://docs.gradle.org/current/javadoc/org/gradle/api/tasks/TaskInputs.html\#getFiles()}{inputs}
for the \hyperref[copytlddocresources]{\texttt{copyTDLDocResources}}
tasks of the subprojects, aggregating them into the
\hyperref[destinationdir]{destination directory} of the
\texttt{appTLDDoc} task. \texttt{jarAppTLDDoc} \textbar{}
\texttt{appTLDDoc} \textbar{}
\href{https://docs.gradle.org/current/dsl/org.gradle.api.tasks.bundling.Jar.html}{\texttt{Jar}}
\textbar{} Assembles a JAR archive containing the tag library
documentation files for this app.

The \texttt{appTLDDoc} task is automatically configured with sensible
defaults:

Property Name \textbar{} Default Value
\hyperref[destinationdir]{\texttt{destinationDir}} \textbar{}
\texttt{\$\{project.buildDir\}/docs/tlddoc}
\hyperref[source]{\texttt{source}} \textbar{} The sum of all the
\texttt{tlddoc.source} values of the subprojects

\subsection{Project Extension}\label{project-extension-8}

The App TLDDoc Builder plugin exposes the following properties through
the extension named \texttt{appTLDDocBuilder}:

Property Name \textbar{} Type \textbar{} Default Value \textbar{}
Description \texttt{subprojects} \textbar{}
\texttt{Set\textless{}Project\textgreater{}} \textbar{}
\texttt{project.subprojects} \textbar{} The subprojects to include in
the tag library documentation of the app.

The same extension exposes the following methods:

Method \textbar{} Description
\texttt{AppTLDDocBuilderExtension\ subprojects(Iterable\textless{}Project\textgreater{}\ subprojects)}
\textbar{} Include additional projects in the tag library documentation
of the app.
\texttt{AppTLDDocBuilderExtension\ subprojects(Project...\ subprojects)}
\textbar{} Include additional projects in the tag library documentation
of the app.

\subsection{Tasks}\label{tasks-21}

\subsubsection{TLDDocTask}\label{tlddoctask}

Tasks of type \texttt{TLDDocTask} extend
\href{https://docs.gradle.org/current/dsl/org.gradle.api.tasks.JavaExec.html}{\texttt{JavaExec}},
so all its properties and methods, such as
\href{https://docs.gradle.org/current/dsl/org.gradle.api.tasks.JavaExec.html\#org.gradle.api.tasks.JavaExec:args(java.tlddoc.Iterable)}{\texttt{args}}
and
\href{https://docs.gradle.org/current/dsl/org.gradle.api.tasks.JavaExec.html\#org.gradle.api.tasks.JavaExec:maxHeapSize}{\texttt{maxHeapSize}},
are available. They also have the following properties set by default:

Property Name \textbar{} Default Value
\href{https://docs.gradle.org/current/dsl/org.gradle.api.tasks.JavaExec.html\#org.gradle.api.tasks.JavaExec:args}{\texttt{args}}
\textbar{} Tag Library Documentation Generator command line arguments
\href{https://docs.gradle.org/current/dsl/org.gradle.api.tasks.JavaExec.html\#org.gradle.api.tasks.JavaExec:classpath}{\texttt{classpath}}
\textbar{}
\hyperref[tag-library-documentation-generator-dependency]{\texttt{project.configurations.tlddoc}}
\href{https://docs.gradle.org/current/dsl/org.gradle.api.tasks.JavaExec.html\#org.gradle.api.tasks.JavaExec:main}{\texttt{main}}
\textbar{} \texttt{"com.sun.tlddoc.TLDDoc"}
\href{https://docs.gradle.org/current/dsl/org.gradle.api.tasks.JavaExec.html\#org.gradle.api.tasks.JavaExec:maxHeapSize}{\texttt{maxHeapSize}}
\textbar{} \texttt{"256m"}

The \texttt{TLDDocTask} class is also very similar to
\href{https://docs.gradle.org/current/dsl/org.gradle.api.tasks.SourceTask.html}{\texttt{SourceTask}},
which means it provides a \texttt{source} property and lets you specify
include and exclude patterns.

\paragraph{Task Properties}\label{task-properties-31}

Property Name \textbar{} Type \textbar{} Default Value \textbar{}
Description \texttt{destinationDir} \textbar{} \texttt{File} \textbar{}
\texttt{null} \textbar{} The directory where the tag library
documentation files are saved. \texttt{excludes} \textbar{}
\texttt{Set\textless{}String\textgreater{}} \textbar{} \texttt{{[}{]}}
\textbar{} The TLD file patterns to exclude. \texttt{includes}
\textbar{} \texttt{Set\textless{}String\textgreater{}} \textbar{}
\texttt{{[}{]}} \textbar{} The TLD file patterns to include.
\texttt{source} \textbar{}
\href{https://docs.gradle.org/current/javadoc/org/gradle/api/file/FileTree.html}{\texttt{FileTree}}
\textbar{} \texttt{{[}{]}} \textbar{} The TLD files to generate
documentation for, after the include and exclude patterns have been
applied. \texttt{xsltDir} \textbar{} \texttt{File} \textbar{}
\texttt{null} \textbar{} The directory that contains the custom XSLT
stylesheets used by the Tag Library Documentation Generator to produce
the final documentation files. It sets the \texttt{-xslt} argument.

The properties of type \texttt{File} support any type that can be
resolved by
\href{https://docs.gradle.org/current/dsl/org.gradle.api.Project.html\#org.gradle.api.Project:file(java.tlddoc.Object)}{\texttt{project.file}}.

\paragraph{Task Methods}\label{task-methods-11}

The methods available for \texttt{TLDDocTask} are exactly the same as
the one defined in the
\href{https://docs.gradle.org/current/dsl/org.gradle.api.tasks.SourceTask.html}{\texttt{SourceTask}}
class.

\subsubsection{ValidateSchemaTask}\label{validateschematask}

Tasks of type \texttt{ValidateSchemaTask} extend
\href{https://docs.gradle.org/current/dsl/org.gradle.api.tasks.SourceTask.html}{\texttt{SourceTask}},
so all its properties and methods, such as
\href{https://docs.gradle.org/current/dsl/org.gradle.api.tasks.SourceTask.html\#org.gradle.api.tasks.SourceTask:include(java.lang.Iterable)}{\texttt{include}}
and
\href{https://docs.gradle.org/current/dsl/org.gradle.api.tasks.SourceTask.html\#org.gradle.api.tasks.SourceTask:exclude(java.lang.Iterable)}{\texttt{exclude}},
are available.

Tasks of this type invoke the
\href{http://ant.apache.org/manual/Tasks/schemavalidate.html}{\texttt{schemavalidate}}
Ant task in order to validate XML files described by an XML schema.

\paragraph{Task Properties}\label{task-properties-32}

Property Name \textbar{} Type \textbar{} Default Value \textbar{}
Description \texttt{dtdDisabled} \textbar{} \texttt{boolean} \textbar{}
\texttt{false} \textbar{} Whether to disable DTD support.
\texttt{fullChecking} \textbar{} \texttt{boolean} \textbar{}
\texttt{true} \textbar{} Whether to enable full schema checking.
\texttt{lenient} \textbar{} \texttt{boolean} \textbar{} \texttt{false}
\textbar{} Whether to only check if the XML document is well-formed.
\texttt{xmlParserClassName} \textbar{} \texttt{String} \textbar{}
\texttt{null} \textbar{} The class name of the XML parser to use.
\texttt{xmlParserClasspath} \textbar{} \texttt{FileCollection}
\textbar{} \texttt{null} \textbar{} The classpath with the XML parser.

It is possible to use Closures and Callables as values for the
\texttt{String} properties to defer evaluation until task execution.

\subsection{Additional Configuration}\label{additional-configuration-14}

There are additional configurations that can help you use the TLDDoc
Builder.

\subsubsection{Tag Library Documentation Generator
Dependency}\label{tag-library-documentation-generator-dependency}

By default, the plugin creates a configuration called \texttt{tlddoc}
and adds a dependency to the 1.3 version of the Tag Library
Documentation Generator. It is possible to override this setting and use
a specific version of the tool by manually adding a dependency to the
\texttt{tlddoc} configuration:

\begin{verbatim}
dependencies {
    tlddoc group: "taglibrarydoc", name: "tlddoc", version: "1.3"
}
\end{verbatim}

\section{TLD Formatter Gradle Plugin}\label{tld-formatter-gradle-plugin}

The TLD Formatter Gradle plugin lets you format a project's TLD files
using the
\href{https://github.com/liferay/liferay-portal/tree/master/modules/util/tld-formatter}{Liferay
TLD Formatter} tool.

The plugin has been successfully tested with Gradle 4.10.2.

\subsection{Usage}\label{usage-23}

To use the plugin, include it in your build script:

\begin{verbatim}
buildscript {
    dependencies {
        classpath group: "com.liferay", name: "com.liferay.gradle.plugins.tld.formatter", version: "1.0.9"
    }

    repositories {
        maven {
            url "https://repository-cdn.liferay.com/nexus/content/groups/public"
        }
    }
}

apply plugin: "com.liferay.tld.formatter"
\end{verbatim}

Since the plugin automatically resolves the Liferay TLD Formatter
library as a dependency, you have to configure a repository that hosts
the library and its transitive dependencies. The Liferay CDN repository
hosts them all:

\begin{verbatim}
repositories {
    maven {
        url "https://repository-cdn.liferay.com/nexus/content/groups/public"
    }
}
\end{verbatim}

\subsection{Tasks}\label{tasks-22}

The plugin adds one task to your project:

Name \textbar{} Depends On \textbar{} Type \textbar{} Description
\texttt{formatTLD} \textbar{} - \textbar{}
\hyperref[formattldtask]{\texttt{FormatTLDTask}} \textbar{} Runs the
Liferay TLD Formatter to format files.

\subsubsection{FormatTLDTask}\label{formattldtask}

Tasks of type \texttt{FormatTLDTask} extend
\href{https://docs.gradle.org/current/dsl/org.gradle.api.tasks.JavaExec.html}{\texttt{JavaExec}},
so all its properties and methods, such as
\href{https://docs.gradle.org/current/dsl/org.gradle.api.tasks.JavaExec.html\#org.gradle.api.tasks.JavaExec:args(java.lang.Iterable)}{\texttt{args}}
and
\href{https://docs.gradle.org/current/dsl/org.gradle.api.tasks.JavaExec.html\#org.gradle.api.tasks.JavaExec:maxHeapSize}{\texttt{maxHeapSize}},
are available. They also have the following properties set by default:

Property Name \textbar{} Default Value
\href{https://docs.gradle.org/current/dsl/org.gradle.api.tasks.JavaExec.html\#org.gradle.api.tasks.JavaExec:args}{\texttt{args}}
\textbar{} TLD Formatter command line arguments
\href{https://docs.gradle.org/current/dsl/org.gradle.api.tasks.JavaExec.html\#org.gradle.api.tasks.JavaExec:classpath}{\texttt{classpath}}
\textbar{}
\hyperref[liferay-tld-formatter-dependency]{\texttt{project.configurations.tldFormatter}}
\href{https://docs.gradle.org/current/dsl/org.gradle.api.tasks.JavaExec.html\#org.gradle.api.tasks.JavaExec:main}{\texttt{main}}
\textbar{} \texttt{"com.liferay.tld.formatter.TLDFormatter"}

\paragraph{Task Properties}\label{task-properties-33}

Property Name \textbar{} Type \textbar{} Default Value \textbar{}
Description \texttt{plugin} \textbar{} \texttt{boolean} \textbar{}
\texttt{true} \textbar{} Whether to format all the TLD files contained
in the
\href{https://docs.gradle.org/current/dsl/org.gradle.api.tasks.JavaExec.html\#org.gradle.api.tasks.JavaExec:workingDir}{\texttt{workingDir}}
directory. If \texttt{false}, all \texttt{liferay-portlet-ext.tld} files
are ignored. It sets the \texttt{tld.plugin} argument.

\subsection{Additional Configuration}\label{additional-configuration-15}

There are additional configurations that can help you use the TLD
Formatter.

\subsubsection{Liferay TLD Formatter
Dependency}\label{liferay-tld-formatter-dependency}

By default, the plugin creates a configuration called
\texttt{tldFormatter} and adds a dependency to the latest released
version of Liferay TLD Formatter. It is possible to override this
setting and use a specific version of the tool by manually adding a
dependency to the \texttt{tldFormatter} configuration:

\begin{verbatim}
dependencies {
    tldFormatter group: "com.liferay", name: "com.liferay.tld.formatter", version: "1.0.5"
}
\end{verbatim}

\section{Whip Gradle Plugin}\label{whip-gradle-plugin}

The Whip Gradle plugin lets you use the
\href{https://github.com/liferay/liferay-portal/tree/master/modules/test/whip}{Liferay
Whip} library to ensure that unit tests fully cover your project's code.
See
\href{https://github.com/liferay/liferay-portal/tree/master/modules/sdk/gradle-plugins-whip/src/gradleTest/smoke}{here}
for a usage sample.

The plugin has been successfully tested with Gradle 4.10.2.

\subsection{Usage}\label{usage-24}

To use the plugin, include it in your build script:

\begin{verbatim}
buildscript {
    dependencies {
        classpath group: "com.liferay", name: "com.liferay.gradle.plugins.whip", version: "1.0.7"
    }

    repositories {
        maven {
            url "https://repository-cdn.liferay.com/nexus/content/groups/public"
        }
    }
}

apply plugin: "com.liferay.whip"
\end{verbatim}

Since the plugin automatically resolves the Liferay Whip library as a
dependency, you have to configure a repository that hosts the library
and its transitive dependencies. The Liferay CDN repository hosts them
all:

\begin{verbatim}
repositories {
    maven {
        url "https://repository-cdn.liferay.com/nexus/content/groups/public"
    }
}
\end{verbatim}

By default, Whip is automatically applied to all tasks of type
\href{https://docs.gradle.org/current/javadoc/org/gradle/api/tasks/testing/Test.html}{\texttt{Test}}.
If a task has Whip applied and Whip is \hyperref[enabled]{enabled}, then
Whip is configured as a Java Agent.

\subsection{Project Extension}\label{project-extension-9}

The Whip Gradle plugin exposes the following properties through the
extension named \texttt{whip}:

Property Name \textbar{} Type \textbar{} Default Value \textbar{}
Description \texttt{version} \textbar{} \texttt{String} \textbar{}
\texttt{latest.release} \textbar{} The version of the Liferay Whip
library to use.

The same extension exposes the following methods:

Method \textbar{} Description \texttt{void\ applyTo(Task\ task)}
\textbar{} Applies Whip to a task. The task instance must implement the
\href{https://docs.gradle.org/current/javadoc/org/gradle/process/JavaForkOptions.html}{\texttt{JavaForkOptions}}
interface.

\subsection{Task Extension}\label{task-extension}

If Whip is applied, the following task properties are available through
the extension named \texttt{whip}:

Property Name \textbar{} Type \textbar{} Default Value \textbar{}
Description \texttt{dataFile} \textbar{} \texttt{File} \textbar{}
\texttt{test-coverage/whip.dat} \textbar{} \texttt{enabled} \textbar{}
\texttt{boolean} \textbar{} \texttt{true} \textbar{} Whether to
configure Whip as a Java Agent. \texttt{excludes} \textbar{}
\texttt{List\textless{}String\textgreater{}} \textbar{} \texttt{{[}{]}}
\textbar{} The class name patterns to exclude when checking for unit
test code coverage. For example, a value could be
\texttt{{[}\textquotesingle{}.*Test\textquotesingle{},\ \textquotesingle{}.*Test\textbackslash{}\textbackslash{}\$.*\textquotesingle{},\ \textquotesingle{}.*\textbackslash{}\textbackslash{}\$Proxy.*\textquotesingle{},\ \textquotesingle{}com/liferay/whip/.*\textquotesingle{}{]}}.
\texttt{includes} \textbar{}
\texttt{List\textless{}String\textgreater{}} \textbar{} \texttt{{[}{]}}
\textbar{} The class name patterns to include when checking for unit
test code coverage. \texttt{instrumentDump} \textbar{} \texttt{boolean}
\textbar{} \texttt{false} \textbar{} \texttt{whipJarFile} \textbar{}
\texttt{File} \textbar{} The first file in the \texttt{whip}
configuration whose name starts with \texttt{com.liferay.whip-}.
\textbar{} The Whip JAR file.

The same extension exposes the following methods:

Method \textbar{} Description
\texttt{WhipTaskExtension\ excludes(Iterable\textless{}Object\textgreater{}\ excludes)}
\textbar{} Adds class name patterns to exclude when checking for unit
test coverage. \texttt{WhipTaskExtension\ excludes(Object...\ excludes)}
\textbar{} Adds class name patterns to exclude when checking for unit
test coverage.
\texttt{WhipTaskExtension\ includes(Iterable\textless{}Object\textgreater{}\ includes)}
\textbar{} Adds class name patterns to include when checking for unit
test coverage. \texttt{WhipTaskExtension\ includes(Object...\ includes)}
\textbar{} Adds class name patterns to include when checking for unit
test coverage.

\subsection{Additional Configuration}\label{additional-configuration-16}

There are additional configurations that can help you use Whip.

\subsubsection{Liferay Whip Dependency}\label{liferay-whip-dependency}

By default, the Whip Gradle plugin creates a configuration called
\texttt{whip} and adds a dependency to the version of Liferay Whip
configured in the \hyperref[version]{\texttt{whip.version}} extension
property. It is possible to override this setting and use a specific
version of the library by manually adding a dependency to the
\texttt{whip} configuration:

\begin{verbatim}
dependencies {
    whip group: "com.liferay", name: "com.liferay.whip", version: "1.0.1"
}
\end{verbatim}

In order to leverage the sensible default of the
\hyperref[whipjarfile]{\texttt{whip.whipJarFile}} task property, the
name of the dependency must be \texttt{com.liferay.whip}. Otherwise, it
will be necessary to set the value of the \texttt{whip.whipJarFile}
property manually.

\section{WSDD Builder Gradle Plugin}\label{wsdd-builder-gradle-plugin}

The WSDD Builder Gradle plugin lets you run the
\href{https://github.com/liferay/liferay-portal/tree/master/modules/util/portal-tools-wsdd-builder}{Liferay
WSDD Builder} tool to generate the
\href{http://axis.apache.org/axis/}{Apache Axis} Web Service Deployment
Descriptor (WSDD) files from a
\href{/docs/7-0/tutorials/-/knowledge_base/t/what-is-service-builder}{Service
Builder} \texttt{service.xml} file.

The plugin has been successfully tested with Gradle 4.10.2.

\subsection{Usage}\label{usage-25}

To use the plugin, include it in your build script:

\begin{verbatim}
buildscript {
    dependencies {
        classpath group: "com.liferay", name: "com.liferay.gradle.plugins.wsdd.builder", version: "1.0.13"
    }

    repositories {
        maven {
            url "https://repository-cdn.liferay.com/nexus/content/groups/public"
        }
    }
}

apply plugin: "com.liferay.portal.tools.wsdd.builder"
\end{verbatim}

The WSDD Builder plugin automatically applies the
\href{https://docs.gradle.org/current/userguide/java_plugin.html}{\texttt{java}}
plugin.

Since the plugin automatically resolves the Liferay WSDD Builder library
as a dependency, you have to configure a repository that hosts the
library and its transitive dependencies. The Liferay CDN repository
hosts them all:

\begin{verbatim}
repositories {
    maven {
        url "https://repository-cdn.liferay.com/nexus/content/groups/public"
    }
}
\end{verbatim}

\subsection{Tasks}\label{tasks-23}

The plugin adds one task to your project:

Name \textbar{} Depends On \textbar{} Type \textbar{} Description
\texttt{buildWSDD} \textbar{}
\href{https://docs.gradle.org/current/userguide/java_plugin.html\#sec:compile}{\texttt{compileJava}}
\textbar{} \hyperref[buildwsddtask]{\texttt{BuildWSDDTask}} \textbar{}
Runs the Liferay WSDD Builder.

By default, the \texttt{buildWSDD} task uses the
\texttt{\$\{project.projectDir\}/service.xml} file as input. Then, it
generates \texttt{\$\{project.projectDir\}/server-config.wsdd} and the
\texttt{*\_deploy.wsdd} and \texttt{*\_undeploy.wsdd} files in the first
\href{https://docs.gradle.org/current/dsl/org.gradle.api.tasks.SourceSet.html\#org.gradle.api.tasks.SourceSet:resources}{\texttt{resources}}
directory of the \texttt{main}
\href{https://docs.gradle.org/current/userguide/java_plugin.html\#N1503E}{source
set} (by default: \texttt{src/main/resources}).

If the
\href{https://docs.gradle.org/current/userguide/war_plugin.html}{\texttt{war}}
plugin is applied, the task uses
\texttt{\$\{project.webAppDir\}/WEB-INF/service.xml} as input to
generate \texttt{\$\{project.webAppDir\}/WEB-INF/server-config.wsdd}.
The \texttt{*\_deploy.wsdd} and \texttt{*\_undeploy.wsdd} files are
still generated in the first \texttt{resources} directory of the
\texttt{main} source set.

Liferay WSDD Build Service requires an additional classpath (configured
with the \texttt{buildWSDD.builderClasspath} property), to correctly
generate the WSDD files. The \texttt{buildWSDD} task uses the following
default value, which creates an implicit dependency to the
\texttt{compileJava} task:

\begin{verbatim}
tasks.compileJava.outputs.files + sourceSets.main.compileClasspath + sourceSets.main.runtimeClasspath
\end{verbatim}

\subsubsection{BuildWSDDTask}\label{buildwsddtask}

Tasks of type \texttt{BuildWSDDTask} extend
\href{https://docs.gradle.org/current/dsl/org.gradle.api.tasks.JavaExec.html}{\texttt{JavaExec}},
so all its properties and methods, such as
\href{https://docs.gradle.org/current/dsl/org.gradle.api.tasks.JavaExec.html\#org.gradle.api.tasks.JavaExec:args(java.lang.Iterable)}{\texttt{args}}
and
\href{https://docs.gradle.org/current/dsl/org.gradle.api.tasks.JavaExec.html\#org.gradle.api.tasks.JavaExec:maxHeapSize}{\texttt{maxHeapSize}},
are available. They also have the following properties set by default:

Property Name \textbar{} Default Value
\href{https://docs.gradle.org/current/dsl/org.gradle.api.tasks.JavaExec.html\#org.gradle.api.tasks.JavaExec:args}{\texttt{args}}
\textbar{} WSDD Builder command line arguments
\href{https://docs.gradle.org/current/dsl/org.gradle.api.tasks.JavaExec.html\#org.gradle.api.tasks.JavaExec:classpath}{\texttt{classpath}}
\textbar{}
\hyperref[liferay-wsdd-builder-dependency]{\texttt{project.configurations.wsddBuilder}}
\href{https://docs.gradle.org/current/dsl/org.gradle.api.tasks.JavaExec.html\#org.gradle.api.tasks.JavaExec:main}{\texttt{main}}
\textbar{} \texttt{"com.liferay.portal.tools.wsdd.builder.WSDDBuilder"}

\paragraph{Task Properties}\label{task-properties-34}

Property Name \textbar{} Type \textbar{} Default Value \textbar{}
Description \texttt{builderClasspath} \textbar{} \texttt{String}
\textbar{} \texttt{null} \textbar{} A classpath that the Liferay WSDD
Builder uses to generate WSDD files. It sets the
\texttt{wsdd.class.path} argument. \texttt{inputFile} \textbar{}
\texttt{File} \textbar{} \texttt{null} \textbar{} A \texttt{service.xml}
from which to generate the WSDD files. It sets the
\texttt{wsdd.input.file} argument. \texttt{outputDir} \textbar{}
\texttt{File} \textbar{} \texttt{null} \textbar{} A directory where the
\texttt{*\_deploy.wsdd} and \texttt{*\_undeploy.wsdd} files are
generated. It sets the \texttt{wsdd.output.path} argument.
\texttt{serverConfigFile} \textbar{} \texttt{File} \textbar{}
\texttt{\$\{project.projectDir\}/server-config.wsdd} \textbar{} A
\texttt{server-config.wsdd} file to generate. It sets the
\texttt{wsdd.server.config.file} argument. \texttt{serviceNamespace}
\textbar{} \texttt{String} \textbar{} \texttt{"Plugin"} \textbar{} A
namespace for the WSDD Service. It sets the
\texttt{wsdd.service.namespace} argument.

The properties of type \texttt{File} support any type that can be
resolved by
\href{https://docs.gradle.org/current/dsl/org.gradle.api.Project.html\#org.gradle.api.Project:file(java.lang.Object)}{\texttt{project.file}}.
Moreover, it is possible to use Closures and Callables as values for the
\texttt{String} properties, to defer evaluation until task execution.

\subsection{Additional Configuration}\label{additional-configuration-17}

There are additional configurations that can help you use the WSDD
Builder.

\subsubsection{Liferay WSDD Builder
Dependency}\label{liferay-wsdd-builder-dependency}

By default, the plugin creates a configuration called
\texttt{wsddBuilder} and adds a dependency to the latest released
version of the Liferay WSDD Builder. It is possible to override this
setting and use a specific version of the tool by manually adding a
dependency to the \texttt{wsddBuilder} configuration:

\begin{verbatim}
dependencies {
    wsddBuilder group: "com.liferay", name: "com.liferay.portal.tools.wsdd.builder", version: "1.0.10"
}
\end{verbatim}

\section{WSDL Builder Gradle Plugin}\label{wsdl-builder-gradle-plugin}

The WSDL Builder Gradle plugin lets you generate
\href{http://axis.apache.org/axis/}{Apache Axis} client stubs from Web
Service Description (WSDL) files.

The plugin has been successfully tested with Gradle 4.10.2.

\subsection{Usage}\label{usage-26}

To use the plugin, include it in your build script:

\begin{verbatim}
buildscript {
    dependencies {
        classpath group: "com.liferay", name: "com.liferay.gradle.plugins.wsdl.builder", version: "2.0.3"
    }

    repositories {
        maven {
            url "https://repository-cdn.liferay.com/nexus/content/groups/public"
        }
    }
}

apply plugin: "com.liferay.wsdl.builder"
\end{verbatim}

The WSDL Builder plugin automatically applies the
\href{https://docs.gradle.org/current/userguide/java_plugin.html}{\texttt{java}}
plugin.

Since the plugin automatically resolves the Apache Axis library as a
dependency, you have to configure a repository that hosts the library
and its transitive dependencies. The Liferay CDN repository hosts them
all:

\begin{verbatim}
repositories {
    maven {
        url "https://repository-cdn.liferay.com/nexus/content/groups/public"
    }
}
\end{verbatim}

\subsection{Tasks}\label{tasks-24}

The plugin adds one main task to your project:

Name \textbar{} Depends On \textbar{} Type \textbar{} Description
\texttt{buildWSDL} \textbar{} - \textbar{}
\hyperref[buildwsdltask]{\texttt{BuildWSDLTask}} \textbar{} Generates
WSDL client stubs.

By default, the \texttt{buildWSDL} task looks for WSDL files in the
\texttt{\$\{project.projectDir\}/wsdl} directory. If the
\href{https://docs.gradle.org/current/userguide/war_plugin.html}{\texttt{war}}
plugin is applied, it looks in the
\texttt{\$\{project.webAppDir\}/WEB-INF/wsdl} directory.

For each WSDL file that can be found, the task generates client stubs
via direct invocation of the
\href{http://axis.apache.org/axis/java/user-guide.html\#Client-side_bindings}{\emph{WSDL2Java}}
tool, saving them in the first
\href{https://docs.gradle.org/current/dsl/org.gradle.api.tasks.SourceSet.html\#org.gradle.api.tasks.SourceSet:java}{\texttt{java}}
directory of the \texttt{main}
\href{https://docs.gradle.org/current/userguide/java_plugin.html\#N1503E}{source
set} (by default: \texttt{src/main/java}).

If configured to do so, \texttt{buildWSDL} can instead save the client
stub Java files in a temporary directory, compile them, and package them
in JAR files. The JAR files are named after the WSDL file and saved in
\texttt{\$\{project.projectDir\}/lib}, by default, or in
\texttt{\$\{project.webAppDir\}/WEB-INF/lib}, if the \texttt{war} plugin
is applied.

\subsubsection{BuildWSDLTask}\label{buildwsdltask}

Tasks of type \texttt{FormatWSDLTask} extend
\href{https://docs.gradle.org/current/dsl/org.gradle.api.tasks.SourceTask.html}{\texttt{SourceTask}},
so all its properties and methods, such as
\href{https://docs.gradle.org/current/dsl/org.gradle.api.tasks.SourceTask.html\#org.gradle.api.tasks.SourceTask:include(java.lang.Iterable)}{\texttt{include}}
and
\href{https://docs.gradle.org/current/dsl/org.gradle.api.tasks.SourceTask.html\#org.gradle.api.tasks.SourceTask:exclude(java.lang.Iterable)}{\texttt{exclude}},
are available.

\paragraph{Task Properties}\label{task-properties-35}

Property Name \textbar{} Type \textbar{} Default Value \textbar{}
Description \texttt{buildLibs} \textbar{} \texttt{boolean} \textbar{}
\texttt{true} \textbar{} Whether to package the client stub classes of
each WSDL file in JAR files, saved to the directory the
\texttt{destinationDir} property references. If \texttt{false}, the task
generates the client stub Java files to the \texttt{destinationDir}
directory. \texttt{destinationDir} \textbar{} \texttt{File} \textbar{}
\texttt{null} \textbar{} A directory where the client stub Java files
(if \texttt{buildLibs} is \texttt{false}) or the client stub JAR files
(if \texttt{buildLibs} is \texttt{true}) are saved.
\texttt{generateOptions.mapping} \textbar{} \texttt{Map} \textbar{}
\texttt{{[}:{]}} \textbar{} Namespace-to-package mappings (sets the
\texttt{-\/-NStoPkg} argument in the \emph{WSDL2Java} invocation). It is
possible to use a \texttt{Closure} or a \texttt{Callable}, to defer
evaluation until task execution.. \texttt{generateOptions.noWrapped}
\textbar{} \texttt{boolean} \textbar{} \texttt{false} \textbar{} Whether
to turn off support for ``wrapped'' document/literal (sets the
\texttt{-\/-noWrapped} argument in the \emph{WSDL2Java} invocation).
\texttt{generateOptions.serverSide} \textbar{} \texttt{boolean}
\textbar{} \texttt{false} \textbar{} Whether to emit server-side
bindings for the web service (sets the \texttt{-\/-server-side} argument
in the \emph{WSDL2Java} invocation). \texttt{generateOptions.verbose}
\textbar{} \texttt{boolean} \textbar{} \texttt{false} \textbar{} Whether
to print informational messages (sets the \texttt{-\/-verbose} argument
in the \emph{WSDL2Java} invocation). \texttt{includeSource} \textbar{}
\texttt{boolean} \textbar{} \texttt{true} \textbar{} Whether to package
the client stub Java files in the JAR file's \texttt{OSGI-OPT/src}
directory. If \texttt{buildLibs} is \texttt{false}, this property has no
effect. \texttt{includeWSDLs} \textbar{} \texttt{boolean} \textbar{}
\texttt{true} \textbar{} Whether to configure the
\href{https://docs.gradle.org/current/userguide/java_plugin.html\#sec:resources}{\texttt{processResources}}
task to include the WSDL files in the project JAR's \texttt{wsdl}
directory.

The properties of type \texttt{File} support any type that can be
resolved by
\href{https://docs.gradle.org/current/dsl/org.gradle.api.Project.html\#org.gradle.api.Project:file(java.lang.Object)}{\texttt{project.file}}.

\paragraph{Task Methods}\label{task-methods-12}

Method Signature \textbar{} Description
\texttt{generateOptions.mapping(Object\ namespace,\ Object\ packageName)}
\textbar{} Adds a namespace-to-package mapping.
\texttt{generateOptions.mappings(Map\ mappings)} \textbar{} Adds
multiple namespace-to-package mappings.

\paragraph{Helper Tasks}\label{helper-tasks-1}

At the end of the
\href{https://docs.gradle.org/current/userguide/build_lifecycle.html\#N11BAE}{project
evaluation}, a series of helper tasks are created for each WSDL file
returned by the
\href{https://docs.gradle.org/current/dsl/org.gradle.api.tasks.SourceTask.html\#org.gradle.api.tasks.SourceTask:source}{\texttt{source}}
property of the \texttt{BuildWSDLTask} tasks. The names of the helper
tasks start with the WSDL file name, without any extension.

\begin{itemize}
\tightlist
\item
  \texttt{\$\{WSDL\ file\ title\}Generate} of type
  \href{https://docs.gradle.org/current/dsl/org.gradle.api.tasks.JavaExec.html}{\texttt{JavaExec}}:
  invokes
  \href{https://axis.apache.org/axis/java/reference.html\#WSDL2Java_Reference}{\emph{WSDL2Java}}
  to generate the client stubs for the WSDL file.
\end{itemize}

If \texttt{buildWSDLTask.buildLibs} is \texttt{true}, the following
helper tasks are also created:

\begin{itemize}
\tightlist
\item
  \texttt{\$\{WSDL\ file\ title\}Compile} of type
  \href{https://docs.gradle.org/current/dsl/org.gradle.api.tasks.compile.JavaCompile.html}{\texttt{JavaCompile}}:
  compiles the client stub Java files for the WSDL file.
\item
  \texttt{\$\{WSDL\ file\ title\}Jar} of type
  \href{https://docs.gradle.org/current/dsl/org.gradle.api.tasks.bundling.Jar.html}{\texttt{Jar}}:
  packages in a JAR file called \texttt{\$\{WSDL\ file\ title\}-ws.jar},
  the client stub for the WSDL file.
\end{itemize}

\subsection{Additional Configuration}\label{additional-configuration-18}

There are additional configurations that can help you use WSDL Builder.

\subsubsection{Apache Axis Dependency}\label{apache-axis-dependency}

By default, the plugin creates a configuration called
\texttt{wsdlBuilder} and adds the following dependencies:

\begin{itemize}
\tightlist
\item
  \texttt{axis:axis-wsdl4j:1.5.1}
\item
  \texttt{com.liferay:org.apache.axis:1.4.LIFERAY-PATCHED-1}
\item
  \texttt{commons-discovery:commons-discovery:0.2}
\item
  \texttt{commons-logging:commons-logging:1.0.4}
\item
  \texttt{javax.activation:activation:1.1}
\item
  \texttt{javax.mail:mail:1.4}
\item
  \texttt{org.apache.axis:axis-jaxrpc:1.4}
\item
  \texttt{org.apache.axis:axis-saaj:1.4}
\end{itemize}

It is possible to override this setting and use a specific version of
Apache Axis, by manually populating the \texttt{wsdlBuilder}
configuration with the desired dependencies.

\section{XML Formatter Gradle Plugin}\label{xml-formatter-gradle-plugin}

The XML Formatter Gradle plugin lets you format a project's XML files
using the
\href{https://github.com/liferay/liferay-portal/tree/master/modules/util/xml-formatter}{Liferay
XML Formatter} tool.

The plugin has been successfully tested with Gradle 4.10.2.

\subsection{Usage}\label{usage-27}

To use the plugin, include it in your build script:

\begin{verbatim}
buildscript {
    dependencies {
        classpath group: "com.liferay", name: "com.liferay.gradle.plugins.xml.formatter", version: "1.0.11"
    }

    repositories {
        maven {
            url "https://repository-cdn.liferay.com/nexus/content/groups/public"
        }
    }
}

apply plugin: "com.liferay.xml.formatter"
\end{verbatim}

Since the plugin automatically resolves the Liferay XML Formatter
library as a dependency, you have to configure a repository that hosts
the library and its transitive dependencies. The Liferay CDN repository
hosts them all:

\begin{verbatim}
repositories {
    maven {
        url "https://repository-cdn.liferay.com/nexus/content/groups/public"
    }
}
\end{verbatim}

\subsection{Tasks}\label{tasks-25}

The plugin adds one task to your project:

Name \textbar{} Depends On \textbar{} Type \textbar{} Description
\texttt{formatXML} \textbar{} - \textbar{}
\hyperref[formatxmltask]{\texttt{FormatXMLTask}} \textbar{} Runs the
Liferay XML Formatter to format the project files.

If the
\href{https://docs.gradle.org/current/userguide/java_plugin.html}{\texttt{java}}
plugin is applied, the task formats XML files contained in the
\href{https://docs.gradle.org/current/dsl/org.gradle.api.tasks.SourceSet.html\#org.gradle.api.tasks.SourceSet:resources}{\texttt{resources}}
directories of the \texttt{main}
\href{https://docs.gradle.org/current/userguide/java_plugin.html\#N1503E}{source
set} (by default: \texttt{src/main/resources/**/*.xml}).

\subsubsection{FormatXMLTask}\label{formatxmltask}

Tasks of type \texttt{FormatXMLTask} extend
\href{https://docs.gradle.org/current/dsl/org.gradle.api.tasks.SourceTask.html}{\texttt{SourceTask}},
so all its properties and methods, such as
\href{https://docs.gradle.org/current/dsl/org.gradle.api.tasks.SourceTask.html\#org.gradle.api.tasks.SourceTask:include(java.lang.Iterable)}{\texttt{include}}
and
\href{https://docs.gradle.org/current/dsl/org.gradle.api.tasks.SourceTask.html\#org.gradle.api.tasks.SourceTask:exclude(java.lang.Iterable)}{\texttt{exclude}},
are available.

\paragraph{Task Properties}\label{task-properties-36}

Property Name \textbar{} Type \textbar{} Default Value \textbar{}
Description \texttt{classpath} \textbar{}
\href{https://docs.gradle.org/current/javadoc/org/gradle/api/file/FileCollection.html}{\texttt{FileCollection}}
\textbar{}
\hyperref[liferay-xml-formatter-dependency]{\texttt{project.configurations.xmlFormatter}}
\textbar{} The classpath for executing the main class.
\texttt{mainClassName} \textbar{} \texttt{String} \textbar{}
\texttt{"com.liferay.xml.formatter.XMLFormatter"} \textbar{} The fully
qualified name of the XML Formatter Main class. \texttt{stripComments}
\textbar{} \texttt{boolean} \textbar{} \texttt{false} \textbar{} Whether
to remove all the comments from the XML files. It sets the
\texttt{xml.formatter.strip.comments} argument.

\subsection{Additional Configuration}\label{additional-configuration-19}

There are additional configurations that can help you use the XML
Formatter.

\subsubsection{Liferay XML Formatter
Dependency}\label{liferay-xml-formatter-dependency}

By default, the plugin creates a configuration called
\texttt{xmlFormatter} and adds a dependency to the latest released
version of the Liferay XML Formatter. It is possible to override this
setting and use a specific version of the tool by manually adding a
dependency to the \texttt{xmlFormatter} configuration:

\begin{verbatim}
dependencies {
    xmlFormatter group: "com.liferay", name: "com.liferay.xml.formatter", version: "1.0.5"
}
\end{verbatim}

\section{XSD Builder Gradle Plugin}\label{xsd-builder-gradle-plugin}

The XSD Builder Gradle plugin lets you generate
\href{https://xmlbeans.apache.org/}{Apache XMLBeans} bindings from XML
Schema (XSD) files.

The plugin has been successfully tested with Gradle 4.10.2.

\subsection{Usage}\label{usage-28}

To use the plugin, include it in your build script:

\begin{verbatim}
buildscript {
    dependencies {
        classpath group: "com.liferay", name: "com.liferay.gradle.plugins.xsd.builder", version: "1.0.7"
    }

    repositories {
        maven {
            url "https://repository-cdn.liferay.com/nexus/content/groups/public"
        }
    }
}

apply plugin: "com.liferay.xsd.builder"
\end{verbatim}

The XSD Builder plugin automatically applies the
\href{https://docs.gradle.org/current/userguide/java_plugin.html}{\texttt{java}}
plugin.

Since the plugin automatically resolves the Liferay Service Builder
library as a dependency, you have to configure a repository that hosts
the library and its transitive dependencies. The Liferay CDN repository
hosts them all:

\begin{verbatim}
repositories {
    maven {
        url "https://repository-cdn.liferay.com/nexus/content/groups/public"
    }
}
\end{verbatim}

\subsection{Tasks}\label{tasks-26}

The plugin adds three tasks to your project:

Name \textbar{} Depends On \textbar{} Type \textbar{} Description
\texttt{buildXSD} \textbar{} \texttt{buildXSDCompile} \textbar{}
\hyperref[buildxsdtask]{\texttt{BuildXSDTask}} \textbar{} Generates
XMLBeans bindings and compiles them in a JAR file.
\texttt{buildXSDGenerate} \textbar{} \texttt{cleanBuildXSDGenerate}
\textbar{}
\href{https://docs.gradle.org/current/dsl/org.gradle.api.tasks.JavaExec.html}{\texttt{JavaExec}}
\textbar{} Invokes the
\href{https://xmlbeans.apache.org/docs/2.6.0/guide/tools.html\#scomp}{XMLBeans
Schema Compiler} to generate Java types from XML Schema.
\texttt{buildXSDCompile} \textbar{} \texttt{buildXSDGenerate},
\texttt{cleanBuildXSDCompile} \textbar{}
\href{https://docs.gradle.org/current/dsl/org.gradle.api.tasks.compile.JavaCompile.html}{\texttt{JavaCompile}}
\textbar{} Compiles the generated Java types.

By default, the \texttt{buildXSD} task looks for XSD files in the
\texttt{\$\{project.projectDir\}/xsd} directory, and saves the generated
JAR file as
\texttt{\$\{project.projectDir\}/lib/\$\{project.archivesBaseName\}-xbean.jar}.

If the
\href{https://docs.gradle.org/current/userguide/war_plugin.html}{\texttt{war}}
plugin is applied, the task looks for XSD files in the
\texttt{\$\{project.webAppDir\}/WEB-INF/xsd} directory, and saves the
generated JAR file as
\texttt{\$\{project.webAppDir\}/WEB-INF/lib/\$\{project.archivesBaseName\}-xbean.jar}.

\subsubsection{BuildXSDTask}\label{buildxsdtask}

Tasks of type \texttt{BuildXSDTask} extend
\href{https://docs.gradle.org/current/dsl/org.gradle.api.tasks.bundling.Zip.html}{\texttt{Zip}}.
They also have the following properties set by default:

Property Name \textbar{} Default Value
\href{https://docs.gradle.org/current/dsl/org.gradle.api.tasks.bundling.Zip.html\#org.gradle.api.tasks.bundling.Zip:appendix}{\texttt{appendix}}
\textbar{} \texttt{"xbean"}
\href{https://docs.gradle.org/current/dsl/org.gradle.api.tasks.bundling.Zip.html\#org.gradle.api.tasks.bundling.Zip:extension}{\texttt{extension}}
\textbar{} \texttt{"jar"}
\href{https://docs.gradle.org/current/dsl/org.gradle.api.tasks.bundling.Zip.html\#org.gradle.api.tasks.bundling.Zip:version}{\texttt{version}}
\textbar{} \texttt{null}

For each task of type \texttt{BuildXSDTask}, the following helper tasks
are created:

\begin{itemize}
\tightlist
\item
  \texttt{\$\{buildXSDTask.name\}Compile}
\item
  \texttt{\$\{buildXSDTask.name\}Generate}
\end{itemize}

\paragraph{Task Properties}\label{task-properties-37}

Property Name \textbar{} Type \textbar{} Default Value \textbar{}
Description \texttt{inputDir} \textbar{} \texttt{File} \textbar{}
\texttt{null} \textbar{} A directory containing XSD files from which to
generate \href{https://xmlbeans.apache.org/}{Apache XMLBeans} bindings.

The properties of type \texttt{File} support any type that can be
resolved by
\href{https://docs.gradle.org/current/dsl/org.gradle.api.Project.html\#org.gradle.api.Project:file(java.lang.Object)}{\texttt{project.file}}.

\subsection{Additional Configuration}\label{additional-configuration-20}

There are additional configurations that can help you use the XSD
Builder.

\subsubsection{Apache XMLBeans
Dependency}\label{apache-xmlbeans-dependency}

By default, the XSD Builder Gradle plugin creates a configuration called
\texttt{xsdBuilder} and adds a dependency to the 2.5.0 version of Apache
XMLBeans. It is possible to override this setting and use a specific
version of the library by manually adding a dependency to the
\texttt{xsdBuilder} configuration:

\begin{verbatim}
dependencies {
    xsdBuilder group: "org.apache.xmlbeans", name: "xmlbeans", version: "2.6.0"
}
\end{verbatim}

\section{Felix Gogo Shell}\label{felix-gogo-shell}

To interact with Liferay DXP's module framework on a local server
machine, you can use the Felix Gogo shell within
\href{/docs/7-0/tutorials/-/knowledge_base/t/blade-cli}{Blade CLI}.

Here's the command syntax:

\begin{verbatim}
blade sh <gogoShellCommand>
\end{verbatim}

\noindent\hrulefill

If you're not using Blade CLI, you can start the Gogo shell from a local
telnet session.

\begin{verbatim}
 telnet localhost 11311
\end{verbatim}

To disconnect the session, execute the \texttt{disconnect} command.

\textbf{Warning}: Commands \texttt{shutdown}, \texttt{close}, and
\texttt{exit} stop the OSGi framework. So make sure to use the
\texttt{disconnect} command to end the telnet Gogo Shell session.

\noindent\hrulefill

Here are some useful Gogo shell commands:

\texttt{help}: lists all the available Gogo shell commands. Notice that
each command has two parts to its name, separated by a colon. For
example, the full name of the \texttt{help} command is
\texttt{felix:help}. The first part is the command scope while the
second part is the command function. The scope allows commands with the
same name to be disambiguated. E.g., scope allows the
\texttt{felix:refresh} command to be distinguished from the
\texttt{equinox:refresh} command.

\texttt{help\ {[}COMMAND\_NAME{]}}: lists information about a specific
command including a description of the command, the scope of the
command, and information about any flags or parameters that can be
supplied when invoking the command.

\texttt{lb}: lists all of the bundles installed in Liferay's module
framework. Use the \texttt{-s} flag to list the bundles using the
bundles' symbolic names.

\texttt{b\ {[}BUNDLE\_ID{]}}: lists information about a specific bundle
including the bundle's symbolic name, bundle ID, data root, registered
(provided) and used services, imported and exported packages, and more

\texttt{headers\ {[}BUNDLE\_ID{]}}: lists metadata about the bundle from
the bundle's \texttt{MANIFEST.MF} file

\texttt{diag\ {[}BUNDLE\_ID{]}}: lists information about why the
specified bundle is not working (e.g., unresolved dependencies, etc.)

\texttt{packages\ {[}PACKAGE\_NAME{]}}: lists all of the named package's
dependencies

\texttt{scr:list}: lists all of the components registered in the module
framework (\emph{scr} stands for service component runtime)

\texttt{scr:info\ {[}COMPONENT\_NAME{]}}: lists information about a
specific component including the component's description, services,
properties, configuration, references, and more.

\texttt{services}: lists all of the services that have been registered
in Liferay's module framework

\texttt{inspect\ capability\ service\ {[}BUNDLE\_ID{]}}: lists services
exposed by a bundle

\texttt{install\ {[}PATH\_TO\_JAR\_FILE{]}}: installs the specified
bundle into Liferay's module framework

\texttt{start\ {[}BUNDLE\_ID{]}}: starts the specified bundle

\texttt{stop\ {[}BUNDLE\_ID{]}}: stops the specified bundle

\texttt{uninstall\ {[}BUNDLE\_ID{]}}: uninstalls the specified bundle
from Liferay's module framework

\texttt{system:getproperties}: lists all of the system properties

For more information about the Gogo shell, please visit
\url{http://felix.apache.org/documentation/subprojects/apache-felix-gogo.html}.

\chapter{Maven}\label{maven}

Liferay provides plugins that you can apply to your Maven project. This
reference documentation describes configuration properties for your
Maven project's \texttt{pom.xml} for each plugin. If you're looking for
instructions on using Maven with your Liferay modules, see the
\href{/docs/7-0/tutorials/-/knowledge_base/t/maven}{Maven tutorials}.

\section{Bundle Support Plugin}\label{bundle-support-plugin}

The Bundle Support plugin lets you use
\href{/docs/7-0/tutorials/-/knowledge_base/t/liferay-workspace}{Liferay
Workspace} as a Maven project. For more information on how a Maven
Workspace works and the features it provides, see the
\href{/docs/7-0/tutorials/-/knowledge_base/t/maven-workspace}{Maven
Workspace} tutorial.

\subsection{Usage}\label{usage-29}

To use the plugin, include it in your project's root \texttt{pom.xml}
file:

\begin{verbatim}
<build>
    <plugins>
    ...
        <plugin>
            <groupId>com.liferay</groupId>
            <artifactId>com.liferay.portal.tools.bundle.support</artifactId>
            <version>3.2.5</version>
            <executions>
                <execution>
                    <id>clean</id>
                    <goals>
                        <goal>clean</goal>
                    </goals>
                    <phase>clean</phase>
                    <configuration>
                    </configuration>
                </execution>
                <execution>
                    <id>deploy</id>
                    <goals>
                        <goal>deploy</goal>
                    </goals>
                    <phase>pre-integration-test</phase>
                    <configuration>
                    </configuration>
                </execution>
            </executions>
        </plugin>
        ...
    </plugins>
</build>
\end{verbatim}

\subsection{Goals}\label{goals}

The plugin adds five Maven goals to your project:

Name \textbar{} Description
\hyperref[clean-goals-available-parameters]{bundle-support:clean}
\textbar{} Deletes a file from the \texttt{deploy} directory of a
Liferay bundle.
\hyperref[create-token-goals-available-parameters]{bundle-support:create-token}
\textbar{} Creates a token used to validate your user credentials when
downloading a DXP bundle.
\hyperref[deploy-goals-available-parameters]{bundle-support:deploy}
\textbar{} Deploys the Maven project to the specified Liferay DXP
bundle. \hyperref[dist-goals-available-parameters]{bundle-support:dist}
\textbar{} Creates a distributable Liferay DXP bundle archive file
(e.g., ZIP).
\hyperref[init-goals-available-parameters]{bundle-support:init}
\textbar{} Downloads and installs the specified Liferay DXP version.

\subsection{clean Goal's Available
Parameters}\label{clean-goals-available-parameters}

You can set the following parameters in the \texttt{clean} execution's
\texttt{\textless{}configuration\textgreater{}} section of the POM:

Parameter Name \textbar{} Type \textbar{} Default Value \textbar{}
Description \texttt{liferayHome} \textbar{} \texttt{String} \textbar{}
\texttt{bundles} \textbar{} The directory where your Liferay DXP
instance resides. This can be specified from the command line as
\texttt{-DliferayHome=}. \texttt{fileName} \textbar{} \texttt{String}
\textbar{} \texttt{\$\{project.artifactId\}.\$\{project.packaging\}}
\textbar{} The name of the file to delete from your bundle.

\subsection{create-token Goal's Available
Parameters}\label{create-token-goals-available-parameters}

You can change the default parameter values of the \texttt{create-token}
goal by creating an \texttt{\textless{}execution\textgreater{}} section
containing \texttt{\textless{}configuration\textgreater{}} tags. For
example,

\begin{verbatim}
<execution>
    <id>create-token</id>
    <goals>
        <goal>create-token</goal>
    </goals>
    <configuration>
    </configuration>
</execution>
\end{verbatim}

You can set the following parameters in the \texttt{create-token}
execution's \texttt{\textless{}configuration\textgreater{}} section of
the POM:

Parameter Name \textbar{} Type \textbar{} Default Value \textbar{}
Description \texttt{emailAddress} \textbar{} \texttt{String} \textbar{}
\texttt{null} \textbar{} The email address to use when downloading a DXP
bundle. This email address must match the one registered for your DXP
subscription. \texttt{force} \textbar{} \texttt{boolean} \textbar{}
\texttt{false} \textbar{} Whether to override the existing token with a
newly generated one. \texttt{password} \textbar{} \texttt{String}
\textbar{} \texttt{null} \textbar{} The password to use when downloading
a DXP bundle. This password must match the one registered for your DXP
subscription. \texttt{passwordFile} \textbar{} \texttt{File} \textbar{}
\texttt{null} \textbar{} The file to hold your password used when
downloading a DXP bundle. \texttt{tokenFile} \textbar{} \texttt{File}
\textbar{} \texttt{\$\{user.home\}/.liferay/token} \textbar{} The file
to hold the Liferay bundle authentication token. \texttt{tokenUrl}
\textbar{} \texttt{URL} \textbar{}
\texttt{https://releases-cdn.liferay.com/portal/7.0.6-ga7/liferay-ce-portal-tomcat-7.0-ga7-20180507111753223.zip}
\textbar{} The URL pointing to the bundle Zip to download.

After executing the \texttt{create-token} goal, you're prompted for your
email address and password, both of which are used to generate your
token. It's recommended to configure your email and password from the
command line rather than specifying them in your POM file.

\subsection{deploy Goal's Available
Parameters}\label{deploy-goals-available-parameters}

You can set the following parameters in the \texttt{deploy} execution's
\texttt{\textless{}configuration\textgreater{}} section of the POM:

Parameter Name \textbar{} Type \textbar{} Default Value \textbar{}
Description \texttt{liferayHome} \textbar{} \texttt{String} \textbar{}
\texttt{bundles} \textbar{} The directory where your Liferay DXP
instance resides. This can be specified from the command line as
\texttt{-DliferayHome=}. \texttt{deployFile} \textbar{} \texttt{File}
\textbar{}
\texttt{\$\{project.build.directory\}/\$\{project.build.finalName\}.\$\{project.packaging\}}
\textbar{} The packaged file (e.g., JAR) to deploy to the Liferay
bundle. \texttt{outputFileName} \textbar{} \texttt{String} \textbar{}
\texttt{\$\{project.artifactId\}.\$\{project.packaging\}} \textbar{} The
name of the output file.

\subsection{dist Goal's Available
Parameters}\label{dist-goals-available-parameters}

You can change the default parameter values of the \texttt{dist} goal by
creating an \texttt{\textless{}execution\textgreater{}} section
containing \texttt{\textless{}configuration\textgreater{}} tags. For
example,

\begin{verbatim}
<execution>
    <id>dist</id>
    <goals>
        <goal>dist</goal>
    </goals>
    <configuration>
    </configuration>
</execution>
\end{verbatim}

You can set the following parameters in the \texttt{dist} execution's
\texttt{\textless{}configuration\textgreater{}} section of the POM:

Parameter Name \textbar{} Type \textbar{} Default Value \textbar{}
Description \texttt{liferayHome} \textbar{} \texttt{String} \textbar{}
\texttt{bundles} \textbar{} The directory where your Liferay DXP
instance resides. This can be specified from the command line as
\texttt{-DliferayHome=}. \texttt{archiveFileName} \textbar{}
\texttt{String} \textbar{} \texttt{null} \textbar{} The name for the
generated archive file. \texttt{cacheDir} \textbar{} \texttt{File}
\textbar{} \texttt{\$\{user.home\}/.liferay/bundles} \textbar{} The
directory where the downloaded bundle Zip files are stored.
\texttt{configs} \textbar{} \texttt{String} \textbar{} \texttt{configs}
\textbar{} The directory that contains the configuration files.
\texttt{deployFile} \textbar{} \texttt{File}
\textbar{}\texttt{\$\{project.build.directory\}/\$\{project.build.finalName\}.\$\{project.packaging\}}
\textbar{} The packaged file (e.g., JAR) to deploy to the Liferay
bundle. \texttt{environment} \textbar{} \texttt{String} \textbar{}
\texttt{\$\{liferay.workspace.environment\}} \textbar{} The environment
of your Liferay home deployment. (e.g., \texttt{common}, \texttt{dev},
\texttt{local}, \texttt{prod}, and \texttt{uat}). \texttt{format}
\textbar{} \texttt{String} \textbar{} \texttt{zip} \textbar{} The format
type to use when packaging the Liferay bundle as an archive.
\texttt{includeFolder} \textbar{} \texttt{boolean} \textbar{}
\texttt{true} \textbar{} Whether to add a parent folder to the archive.
\texttt{outputFileName} \textbar{} \texttt{String} \textbar{}
\texttt{\$\{project.artifactId\}.\$\{project.packaging\}} \textbar{} The
path to the archive file. \texttt{password} \textbar{} \texttt{String}
\textbar{} \texttt{null} \textbar{} The password if your Liferay
bundle's URL requires authentication. \texttt{stripComponents}
\textbar{} \texttt{int} \textbar{} \texttt{1} \textbar{} The number of
directories to strip when expanding your bundle. \texttt{token}
\textbar{} \texttt{boolean} \textbar{} \texttt{false} \textbar{} Whether
to use a token to download a Liferay DXP bundle. This should be set to
\texttt{true} when downloading a DXP bundle. \texttt{tokenFile}
\textbar{} \texttt{File} \textbar{}
\texttt{\$\{user.home\}/.liferay/token} \textbar{} The file to hold the
Liferay bundle authentication token. \texttt{url} \textbar{}
\texttt{URL} \textbar{} \texttt{\$\{liferay.workspace.bundle.url\}}
\textbar{} The URL of the Liferay bundle to expand. \texttt{userName}
\textbar{} \texttt{String} \textbar{} \texttt{null} \textbar{} The user
name if your Liferay bundle's URL requires authentication.

\subsection{init Goal's Available
Parameters}\label{init-goals-available-parameters}

You can change the default parameter values of the \texttt{init} goal by
creating an \texttt{\textless{}execution\textgreater{}} section
containing \texttt{\textless{}configuration\textgreater{}} tags. For
example,

\begin{verbatim}
<execution>
    <id>init</id>
    <goals>
        <goal>init</goal>
    </goals>
    <configuration>
    </configuration>
</execution>
\end{verbatim}

You can set the following parameters in the \texttt{init} execution's
\texttt{\textless{}configuration\textgreater{}} section of the POM:

Parameter Name \textbar{} Type \textbar{} Default Value \textbar{}
Description \texttt{liferayHome} \textbar{} \texttt{String} \textbar{}
\texttt{bundles} \textbar{} The directory where your Liferay DXP
instance resides. This can be specified from the command line as
\texttt{-DliferayHome=}. \texttt{cacheDir} \textbar{} \texttt{File}
\textbar{} \texttt{\$\{user.home\}/.liferay/bundles} \textbar{} The
directory where the downloaded bundle Zip files are stored.
\texttt{configs} \textbar{} \texttt{String} \textbar{} \texttt{configs}
\textbar{} The directory that contains the configuration files.
\texttt{environment} \textbar{} \texttt{String} \textbar{}
\texttt{\$\{liferay.workspace.environment\}} \textbar{} The environment
with the settings appropriate for current development (e.g.,
\texttt{common}, \texttt{dev}, \texttt{local}, \texttt{prod}, and
\texttt{uat}). \texttt{password} \textbar{} \texttt{String} \textbar{}
\texttt{null} \textbar{} The password if your Liferay bundle's URL
requires authentication. \texttt{stripComponents} \textbar{}
\texttt{int} \textbar{} \texttt{1} \textbar{} The number of directories
to strip when expanding your bundle. \texttt{token} \textbar{}
\texttt{boolean} \textbar{} \texttt{false} \textbar{} Whether to use a
token to download a Liferay DXP bundle. This should be set to
\texttt{true} when downloading a DXP bundle. \texttt{tokenFile}
\textbar{} \texttt{File} \textbar{}
\texttt{\$\{user.home\}/.liferay/token} \textbar{} The file to hold the
Liferay bundle authentication token. \texttt{url} \textbar{}
\texttt{URL} \textbar{} \texttt{\$\{liferay.workspace.bundle.url\}}
\textbar{} The URL of the Liferay bundle to expand. \texttt{userName}
\textbar{} \texttt{String} \textbar{} \texttt{null} \textbar{} The user
name if your Liferay bundle's URL requires authentication.

\section{CSS Builder Plugin}\label{css-builder-plugin}

The CSS Builder plugin lets you compile
\href{http://sass-lang.com/}{Sass} files in your project.

\subsection{Usage}\label{usage-30}

To use the plugin, include it in your project's root \texttt{pom.xml}
file:

\begin{verbatim}
<build>
    <plugins>
    ...
        <plugin>
            <groupId>com.liferay</groupId>
            <artifactId>com.liferay.css.builder</artifactId>
            <version>3.0.0</version>
            <executions>
                <execution>
                    <id>default-build</id>
                    <phase>compile</phase>
                    <goals>
                        <goal>build</goal>
                    </goals>
                </execution>
            </executions>
            <configuration>
            </configuration>
        </plugin>
    ...
    </plugins>
</build>
\end{verbatim}

You can view an example POM containing the CSS Builder configuration
\href{https://github.com/liferay/liferay-portal/blob/master/modules/util/css-builder/samples/pom.xml}{here}.

\subsection{Goals}\label{goals-1}

The plugin adds one Maven goal to your project:

Name \textbar{} Description \texttt{css-builder:build} \textbar{}
Compiles the Sass files in the project.

\subsection{Available Parameters}\label{available-parameters}

You can set the following parameters in the
\texttt{\textless{}configuration\textgreater{}} section of the POM:

Parameter Name \textbar{} Type \textbar{} Default Value \textbar{}
Description \texttt{appendCssImportTimestamps} \textbar{}
\texttt{boolean} \textbar{} \texttt{true} \textbar{} Whether to append
the current timestamp to the URLs in the \texttt{@import} CSS at-rules.
\texttt{baseDir} \textbar{} \texttt{File} \textbar{}
\texttt{"src/META-INF/resources"} \textbar{} The base directory that
contains the SCSS files to compile. \texttt{dirNames} \textbar{}
\texttt{List\textless{}String\textgreater{}} \textbar{}
\texttt{{[}"/"{]}} \textbar{} The name of the directories, relative to
\hyperref[basedir]{\texttt{baseDir}}, which contain the SCSS files to
compile. \texttt{generateSourceMap} \textbar{} \texttt{boolean}
\textbar{} \texttt{false} \textbar{} Whether to generate
\href{https://developers.google.com/web/tools/chrome-devtools/debug/readability/source-maps}{source
maps} for easier debugging. \texttt{importDir} \textbar{} \texttt{File}
\textbar{} \texttt{null} \textbar{} The \texttt{META-INF/resources}
directory of the
\href{https://github.com/liferay/liferay-portal/tree/master/modules/apps/foundation/frontend-css/frontend-css-common}{Liferay
Frontend Common CSS} artifact. This is required in order to make
\href{http://bourbon.io}{Bourbon} and other CSS libraries available to
the compilation. \texttt{outputDirName} \textbar{} \texttt{String}
\textbar{} \texttt{".sass-cache/"} \textbar{} The name of the
sub-directories where the SCSS files are compiled to. For each directory
that contains SCSS files, a sub-directory with this name is created.
\texttt{precision} \textbar{} \texttt{int} \textbar{} \texttt{9}
\textbar{} The numeric precision of numbers in Sass.
\texttt{rtlExcludedPathRegexps} \textbar{}
\texttt{List\textless{}String\textgreater{}} \textbar{} \textbar{} The
SCSS file patterns to exclude when converting for right-to-left (RTL)
support. \texttt{sassCompilerClassName} \textbar{} \texttt{String}
\textbar{} \texttt{"jni"} \textbar{} The type of Sass compiler to use.
Supported values are \texttt{"jni"} and \texttt{"ruby"}. The Ruby Sass
compiler requires \texttt{com.liferay.sass.compiler.ruby.jar},
\texttt{com.liferay.ruby.gems.jar}, and \texttt{jruby-complete.jar} to
be added to the classpath.

You can also manage the \texttt{com.liferay.frontend.css.common} default
theme dependency provided by the CSS Builder in your \texttt{pom.xml}.
This can be modified by adding it as a project dependency:

\begin{verbatim}
<project>
    ...
    <dependencies>
        <dependency>
            <groupId>com.liferay</groupId>
            <artifactId>com.liferay.frontend.css.common</artifactId>
            <version>3.0.1</version>
            <scope>provided</scope>
        </dependency>
        ...
    </dependencies>
</project>
\end{verbatim}

There are additional Liferay theme-related dependencies you can manage
this way that are provided by the Theme Builder. See
\href{/docs/7-1/reference/-/knowledge_base/r/theme-builder-plugin}{this
section} for more information.

\section{DB Support Plugin}\label{db-support-plugin}

The DB Support plugin lets you run the Liferay DB Support tool to
execute certain actions on a local Liferay DXP database. The following
actions are available:

\begin{itemize}
\tightlist
\item
  Cleans the Liferay database from the Service Builder tables and rows
  of a module.
\end{itemize}

\subsection{Usage}\label{usage-31}

To use the plugin, include it in your project's \texttt{pom.xml} file:

\begin{verbatim}
<build>
    <plugins>
    ...
        <plugin>
            <groupId>com.liferay</groupId>
            <artifactId>com.liferay.portal.tools.db.support</artifactId>
            <version>1.0.6</version>
            <configuration>
            </configuration>
            <dependencies>
                <dependency>
                    <groupId>org.hsqldb</groupId>
                    <artifactId>hsqldb</artifactId>
                    <version>2.4.0</version>
                </dependency>
            </dependencies>
        </plugin>
    ...
    </plugins>
</build>
\end{verbatim}

Also notice the configured plugin dependency. You must configure the
JDBC driver used by your Liferay DXP bundle so the DB Support plugin can
properly manage your database. Replace the HSQLDB driver listed above
with your custom database's JDBC driver.

\subsection{Goals}\label{goals-2}

The plugin adds one Maven goal to your project:

Name \textbar{} Description \texttt{db-support:clean-service-builder}
\textbar{} Cleans the Liferay DXP database from the Service Builder
tables and rows of a module.

\subsection{Available Parameters}\label{available-parameters-1}

You can set the following parameters in the
\texttt{\textless{}configuration\textgreater{}} section of the POM:

Parameter Name \textbar{} Type \textbar{} Default Value \textbar{}
Description \texttt{password} \textbar{} \texttt{String} \textbar{}
\texttt{jdbc.default.password} \textbar{} The user password for
connecting to the Liferay DXP database. \texttt{propertiesFile}
\textbar{} \texttt{File} \textbar{} \texttt{null} \textbar{} The
\texttt{portal-ext.properties} file which contains the JDBC settings for
connecting to the Liferay DXP database. \texttt{serviceXmlFile}
\textbar{} \texttt{File} \textbar{} \texttt{null} \textbar{} The
\texttt{service.xml} file of the module. \texttt{servletContextName}
\textbar{} \texttt{String} \textbar{} \texttt{null} \textbar{} The
servlet context name (usually the value of the
\texttt{Bundle-Symbolic-Name} manifest header) of the module.
\texttt{url} \textbar{} \texttt{String} \textbar{}
\texttt{jdbc.default.url} \textbar{} The JDBC URL for connecting to the
Liferay DXP database. \texttt{userName} \textbar{} \texttt{String}
\textbar{} \texttt{jdbc.default.username} \textbar{} The user name for
connecting to the Liferay DXP database.

\section{Deployment Helper Plugin}\label{deployment-helper-plugin}

The Deployment Helper plugin lets you create a cluster deployable WAR
from your OSGi artifacts.

\subsection{Usage}\label{usage-32}

To use the plugin, include it in your project's root \texttt{pom.xml}
file:

\begin{verbatim}
<build>
    <plugins>
    ...
        <plugin>
            <groupId>com.liferay</groupId>
            <artifactId>com.liferay.deployment.helper</artifactId>
            <version>1.0.4</version>
            <configuration>
            </configuration>
        </plugin>
    ...
    </plugins>
</build>
\end{verbatim}

You can view an example POM containing the Deployment Helper
configuration
\href{https://github.com/liferay/liferay-portal/blob/master/modules/util/deployment-helper/samples/pom.xml}{here}.

\subsection{Goals}\label{goals-3}

The plugin adds one Maven goal to your project:

Name \textbar{} Description \texttt{deployment-helper:build} \textbar{}
Builds a WAR which contains one or more files that are copied once the
WAR is deployed.

\subsection{Available Parameters}\label{available-parameters-2}

You can set the following parameters in the
\texttt{\textless{}configuration\textgreater{}} section of the POM:

Parameter Name \textbar{} Type \textbar{} Default Value \textbar{}
Description \texttt{deploymentFileNames} \textbar{} \texttt{String}
\textbar{} \texttt{null} \textbar{} The files or directories to include
in the WAR and copy once the WAR is deployed. If a directory is added to
this collection, all the JAR files contained in the directory are
included in the WAR. \texttt{deploymentPath} \textbar{} \texttt{String}
\textbar{} \texttt{null} \textbar{} The directory to which the included
files are copied. \texttt{outputFileName} \textbar{} \texttt{String}
\textbar{} \texttt{null} \textbar{} The WAR file to build.

\section{Javadoc Formatter Plugin}\label{javadoc-formatter-plugin}

The Javadoc Formatter plugin lets you format project Javadoc comments.
The tool lets you generate:

\begin{itemize}
\tightlist
\item
  Default
  \href{http://www.oracle.com/technetwork/java/javase/documentation/index-137868.html\#@author}{\texttt{@author}}
  tags to all classes.
\item
  Comment stubs to classes, fields, and methods.
\item
  Missing
  \href{https://docs.oracle.com/javase/8/docs/api/java/lang/Override.html}{\texttt{@Override}}
  annotations.
\item
  An XML representation of the Javadoc comments, which can be used by
  tools in order to index the Javadocs of the project.
\end{itemize}

\subsection{Usage}\label{usage-33}

To use the plugin, include it in your project's root \texttt{pom.xml}
file:

\begin{verbatim}
<build>
    <plugins>
    ...
        <plugin>
            <groupId>com.liferay</groupId>
            <artifactId>com.liferay.javadoc.formatter</artifactId>
            <version>1.0.32</version>
            <configuration>
            </configuration>
        </plugin>
    ...
    </plugins>
</build>
\end{verbatim}

You can view an example POM containing the Javadoc Formatter
configuration
\href{https://github.com/liferay/liferay-portal/blob/master/modules/util/javadoc-formatter/samples/pom.xml}{here}.

\subsection{Goals}\label{goals-4}

The plugin adds one Maven goal to your project:

Name \textbar{} Description \texttt{javadoc-formatter:format} \textbar{}
Runs the Liferay Javadoc Formatter to format files.

\subsection{Available Parameters}\label{available-parameters-3}

You can set the following parameters in the
\texttt{\textless{}configuration\textgreater{}} section of the POM:

Parameter Name \textbar{} Type \textbar{} Default Value \textbar{}
Description \texttt{author} \textbar{} \texttt{String} \textbar{}
\texttt{"Brian\ Wing\ Shun\ Chan"} \textbar{} The value of the
\texttt{@author} tag to add at class level if missing.
\texttt{generateXml} \textbar{} \texttt{boolean} \textbar{}
\texttt{false} \textbar{} Whether to generate a XML representation of
the Javadoc comments. The XML files are generated in the
\texttt{src/main/resources} directory only if the Java files are
contained in \texttt{src/main/java}. \texttt{initializeMissingJavadocs}
\textbar{} \texttt{boolean} \textbar{} \texttt{false} \textbar{} Whether
to add comment stubs at the class, field, and method levels. If
\texttt{false}, only the class-level \texttt{@author} is added.
\texttt{inputDirName} \textbar{} \texttt{String} \textbar{}
\texttt{"./"} \textbar{} The root directory to begin searching for Java
files to format. \texttt{limits} \textbar{} \texttt{String{[}{]}}
\textbar{} \texttt{{[}{]}} \textbar{} The Java file name patterns,
relative to the working directory, to include when formatting Javadoc
comments. The patterns must be specified without the \texttt{.java} file
type suffix. If empty, all Java files are formatted.
\texttt{outputFilePrefix} \textbar{} \texttt{String} \textbar{}
\texttt{"javadocs"} \textbar{} The file name prefix of the XML
representation of the Javadoc comments. If \texttt{generateXML} is
\texttt{false}, this property is not used. \texttt{updateJavadocs}
\textbar{} \texttt{boolean} \textbar{} \texttt{false} \textbar{} Whether
to fix existing comment blocks by adding missing tags.

\section{Lang Builder Plugin}\label{lang-builder-plugin}

The Lang Builder plugin lets you sort and translate the language keys in
your project.

\subsection{Usage}\label{usage-34}

To use the plugin, include it in your project's root \texttt{pom.xml}
file:

\begin{verbatim}
<build>
    <plugins>
    ...
        <plugin>
            <groupId>com.liferay</groupId>
            <artifactId>com.liferay.lang.builder</artifactId>
            <version>1.0.31</version>
            <configuration>
            </configuration>
        </plugin>
    ...
    </plugins>
</build>
\end{verbatim}

You can view an example POM containing the Lang Builder configuration
\href{https://github.com/liferay/liferay-portal/blob/master/modules/util/lang-builder/samples/pom.xml}{here}.

\subsection{Goals}\label{goals-5}

The plugin adds one Maven goal to your project:

Name \textbar{} Description \texttt{lang-builder:build} \textbar{} Runs
Liferay Lang Builder to translate language property files.

\subsection{Available Parameters}\label{available-parameters-4}

You can set the following parameters in the
\texttt{\textless{}configuration\textgreater{}} section of the POM:

Parameter Name \textbar{} Type \textbar{} Default Value \textbar{}
Description \texttt{excludedLanguageIds} \textbar{}
\texttt{String{[}{]}} \textbar{}
\texttt{\{"da",\ "de",\ "fi",\ "ja",\ "nl",\ "pt\_PT",\ "sv"\}}
\textbar{} The language IDs to exclude in the automatic translation.
\texttt{langDirName} \textbar{} \texttt{String} \textbar{}
\texttt{"src/content"} \textbar{} The directory where the language
properties files are saved. \texttt{langFileName} \textbar{}
\texttt{String} \textbar{} \texttt{"Language"} \textbar{} The file name
prefix of the language properties files (e.g.,
\texttt{Language\_it.properties}). \texttt{plugin} \textbar{}
\texttt{boolean} \textbar{} \texttt{true} \textbar{} Whether to check
for duplicate language keys between the project and the portal.
\texttt{portalLanguagePropertiesFileName} \textbar{} \texttt{String}
\textbar{} \texttt{null} \textbar{} The \texttt{Language.properties}
file of the portal. \texttt{translate} \textbar{} \texttt{boolean}
\textbar{} \texttt{true} \textbar{} Whether to translate the language
keys and generate a language properties file for each locale that's
supported by Liferay DXP. \texttt{translateSubscriptionKey} \textbar{}
\texttt{String} \textbar{} \texttt{null} \textbar{} The subscription key
for Microsoft Translation integration. Subscription to the Translator
Text Translation API on Microsoft Cognitive Services is required. Basic
subscriptions, up to 2 million characters a month, are free.

\section{Service Builder Plugin}\label{service-builder-plugin}

The Service Builder plugin lets you generate a service layer defined in
a
\href{/docs/7-0/tutorials/-/knowledge_base/t/what-is-service-builder}{Service
Builder} \texttt{service.xml} file. Visit the
\href{/docs/7-0/tutorials/-/knowledge_base/t/using-service-builder-in-a-maven-project}{Using
Service Builder in a Maven Project} tutorial to learn more about
applying Service Builder to your Maven project.

\subsection{Usage}\label{usage-35}

To use the plugin, include it in your project's root \texttt{pom.xml}
file:

\begin{verbatim}
<build>
    <plugins>
    ...
        <plugin>
            <groupId>com.liferay</groupId>
            <artifactId>com.liferay.portal.tools.service.builder</artifactId>
            <version>1.0.292</version>
            <configuration>
            </configuration>
        </plugin>
    ...
    </plugins>
</build>
\end{verbatim}

You can view an example POM containing the Service Builder configuration
\href{https://github.com/liferay/liferay-portal/blob/master/modules/util/portal-tools-service-builder/samples/pom.xml}{here}.

\subsection{Goals}\label{goals-6}

The plugin adds one Maven goal to your project:

Name \textbar{} Description \texttt{service-builder:build} \textbar{}
Runs the Liferay Service Builder.

\subsection{Available Parameters}\label{available-parameters-5}

You can set the following parameters in the
\texttt{\textless{}configuration\textgreater{}} section of the POM:

Parameter Name \textbar{} Type \textbar{} Default Value \textbar{}
Description \texttt{apiDirName} \textbar{} \texttt{String} \textbar{}
\texttt{"../portal-kernel/src"} \textbar{} A directory where the service
API Java source files are generated.
\texttt{autoImportDefaultReferences} \textbar{} \texttt{boolean}
\textbar{} \texttt{true} \textbar{} Whether to automatically add default
references, like \texttt{com.liferay.portal.ClassName},
\texttt{com.liferay.portal.Resource} and
\texttt{com.liferay.portal.User}, to the services.
\texttt{autoNamespaceTables} \textbar{} \texttt{boolean} \textbar{}
\texttt{null} \textbar{} Whether to prefix table names by the namespace
specified in the \texttt{service.xml} file. \texttt{beanLocatorUtil}
\textbar{} \texttt{String} \textbar{}
\texttt{"com.liferay.portal.kernel.bean.PortalBeanLocatorUtil"}
\textbar{} The fully qualified class name of a bean locator class to use
in the generated service classes. \texttt{buildNumber} \textbar{}
\texttt{long} \textbar{} \texttt{1} \textbar{} A specific value to
assign the \texttt{build.number} property in the
\texttt{service.properties} file. \texttt{buildNumberIncrement}
\textbar{} \texttt{boolean} \textbar{} \texttt{true} \textbar{} Whether
to automatically increment the \texttt{build.number} property in the
\texttt{service.properties} file by one at every service generation.
\texttt{databaseNameMaxLength} \textbar{} \texttt{int} \textbar{}
\texttt{30} \textbar{} The upper bound for database table and column
name lengths to ensure it works on all databases. \texttt{hbmFileName}
\textbar{} \texttt{String} \textbar{}
\texttt{"src/META-INF/portal-hbm.xml"} \textbar{} A Hibernate Mapping
file to generate. \texttt{implDirName} \textbar{} \texttt{String}
\textbar{} \texttt{"src"} \textbar{} A directory where the service Java
source files are generated. \texttt{inputFileName} \textbar{}
\texttt{String} \textbar{} \texttt{"service.xml"} \textbar{} The
project's \texttt{service.xml} file. \texttt{modelHintsConfigs}
\textbar{} \texttt{String} \textbar{}
\texttt{"classpath*:META-INF/portal-model-hints.xml,\ META-INF/portal-model-hints.xml,\ classpath*:META-INF/ext-model-hints.xml,\ classpath*:META-INF/portlet-model-hints.xml"}
\textbar{} Paths to the
\href{/docs/7-0/tutorials/-/knowledge_base/t/customizing-model-entities-with-model-hints}{model
hints} files for Liferay Service Builder to use in generating the
service layer. \texttt{modelHintsFileName} \textbar{} \texttt{String}
\textbar{} \texttt{"src/META-INF/portal-model-hints.xml"} \textbar{} A
model hints file for the project. \texttt{osgiModule} \textbar{}
\texttt{boolean} \textbar{} \texttt{null} \textbar{} Whether to generate
the service layer for OSGi modules. \texttt{pluginName} \textbar{}
\texttt{String} \textbar{} \texttt{null} \textbar{} If specified, a
plugin can enable additional generation features, such as \texttt{Clp}
class generation, for non-OSGi modules. \texttt{propsUtil} \textbar{}
\texttt{String} \textbar{} \texttt{"com.liferay.portal.util.PropsUtil"}
\textbar{} The fully qualified class name of the service properties util
class to generate. \texttt{readOnlyPrefixes} \textbar{} \texttt{String}
\textbar{} \texttt{"fetch,\ get,\ has,\ is,\ load,\ reindex,\ search"}
\textbar{} Prefixes of methods to consider read-only.
\texttt{resourceActionsConfigs} \textbar{} \texttt{String} \textbar{}
\texttt{"META-INF/resource-actions/default.xml,\ resource-actions/default.xml"}
\textbar{} Paths to the
\href{/docs/7-0/tutorials/-/knowledge_base/t/adding-permissions-to-resources}{resource
actions} files for Liferay Service Builder to use in generating the
service layer. \texttt{resourcesDirName} \textbar{} \texttt{String}
\textbar{} \texttt{"src"} \textbar{} A directory where the service
non-Java files are generated. \texttt{springFileName} \textbar{}
\texttt{String} \textbar{} \texttt{"src/META-INF/portal-spring.xml"}
\textbar{} A service Spring file to generate. \texttt{springNamespaces}
\textbar{} \texttt{String} \textbar{} \texttt{"beans"} \textbar{}
Namespaces of Spring XML Schemas to add to the service Spring file.
\texttt{sqlDirName} \textbar{} \texttt{String} \textbar{}
\texttt{"../sql"} \textbar{} A directory where the SQL files are
generated. \texttt{sqlFileName} \textbar{} \texttt{String} \textbar{}
\texttt{"portal-tables.sql"} \textbar{} A name (relative to
\texttt{sqlDir}) for the file in which the SQL table creation
instructions are generated. \texttt{sqlIndexesFileName} \textbar{}
\texttt{String} \textbar{} \texttt{"indexes.sql"} \textbar{} A name
(relative to \texttt{sqlDir}) for the file in which the SQL index
creation instructions are generated. \texttt{sqlSequencesFileName}
\textbar{} \texttt{String} \textbar{} \texttt{"sequences.sql"}
\textbar{} A name (relative to \texttt{sqlDir}) for the file in which
the SQL sequence creation instructions are generated.
\texttt{targetEntityName} \textbar{} \texttt{String} \textbar{}
\texttt{null} \textbar{} If specified, it's the name of the entity for
which Liferay Service Builder should generate the service.
\texttt{testDirName} \textbar{} \texttt{String} \textbar{}
\texttt{"test/integration"} \textbar{} If specified, it's a directory
where integration test Java source files are generated.

\section{Source Formatter Plugin}\label{source-formatter-plugin}

The Source Formatter plugin formats project files according to Liferay's
source formatting standards. For more documentation on Source Formatter
specific functionality, visit the tool's
\href{https://github.com/liferay/liferay-portal/tree/master/modules/util/source-formatter/documentation}{documentation}
folder.

\subsection{Usage}\label{usage-36}

To use the plugin, include it in your project's root \texttt{pom.xml}
file:

\begin{verbatim}
<build>
    <plugins>
    ...
        <plugin>
            <groupId>com.liferay</groupId>
            <artifactId>com.liferay.source.formatter</artifactId>
            <version>1.0.885</version>
            <executions>
                <execution>
                    <phase>process-sources</phase>
                    <goals>
                        <goal>format</goal>
                    </goals>
                </execution>
            </executions>
            <configuration>
            </configuration>
        </plugin>
    ...
    </plugins>
</build>
\end{verbatim}

You can view an example POM containing the Source Formatter
configuration
\href{https://github.com/liferay/liferay-portal/blob/master/modules/util/source-formatter/samples/pom.xml}{here}.

\subsection{Goals}\label{goals-7}

The plugin adds one Maven goal to your project:

Name \textbar{} Description \texttt{source-formatter:format} \textbar{}
Runs the Liferay Source Formatter to format source formatting errors.

\subsection{Available Parameters}\label{available-parameters-6}

You can set the following parameters in the
\texttt{\textless{}configuration\textgreater{}} section of the POM:

Parameter Name \textbar{} Type \textbar{} Default Value \textbar{}
Description \texttt{autoFix} \textbar{} \texttt{boolean} \textbar{}
\texttt{true} \textbar{} Whether to automatically fix source formatting
errors. \texttt{baseDir} \textbar{} \texttt{String} \textbar{}
\texttt{"./"} \textbar{} The Source Formatter base directory.
\emph{(Read-only)} \texttt{fileNames} \textbar{} \texttt{String{[}{]}}
\textbar{} \texttt{null} \textbar{} The file names to format, relative
to the project directory. If \texttt{null}, all files contained in
\texttt{baseDir} will be formatted. \texttt{formatCurrentBranch}
\textbar{} \texttt{boolean} \textbar{} \texttt{false} \textbar{} Whether
to format only the files contained in \texttt{baseDir} that are added or
modified in the current Git branch. \texttt{formatLatestAuthor}
\textbar{} \texttt{boolean} \textbar{} \texttt{false} \textbar{} Whether
to format only the files contained in \texttt{baseDir} that are added or
modified in the latest Git commits of the same author.
\texttt{formatLocalChanges} \textbar{} \texttt{boolean} \textbar{}
\texttt{false} \textbar{} Whether to format only the unstaged files
contained in \texttt{baseDir}. \texttt{gitWorkingBranchName} \textbar{}
\texttt{String} \textbar{} \texttt{"master"} \textbar{} The Git working
branch name. \texttt{includeSubrepositories} \textbar{} \texttt{boolean}
\textbar{} \texttt{false} \textbar{} Whether to format files that are in
read-only subrepositories. \texttt{maxLineLength} \textbar{}
\texttt{int} \textbar{} \texttt{80} \textbar{} The maximum number of
characters allowed in Java files. \texttt{printErrors} \textbar{}
\texttt{boolean} \textbar{} \texttt{true} \textbar{} Whether to print
formatting errors on the Standard Output stream.
\texttt{processorThreadCount} \textbar{} \texttt{int} \textbar{}
\texttt{5} \textbar{} The number of threads used by Source Formatter.
\texttt{showDocumentation} \textbar{} \texttt{boolean} \textbar{}
\texttt{false} \textbar{} Whether to show the documentation for the
source formatting issues, if present. \texttt{throwException} \textbar{}
\texttt{boolean} \textbar{} \texttt{false} \textbar{} Whether to fail
the build if formatting errors are found.

\section{Theme Builder Plugin}\label{theme-builder-plugin}

The Theme Builder plugin lets you build Liferay theme files in your
project. Visit the
\href{/docs/7-0/tutorials/-/knowledge_base/t/building-themes-in-a-maven-project}{Building
Themes in a Maven Project} tutorial to learn more about applying Theme
Builder to your Maven project.

\subsection{Usage}\label{usage-37}

To use the plugin, include it in your project's root \texttt{pom.xml}
file:

\begin{verbatim}
<build>
    <plugins>
    ...
        <plugin>
            <groupId>com.liferay</groupId>
            <artifactId>com.liferay.portal.tools.theme.builder</artifactId>
            <version>1.1.7</version>
            <executions>
                <execution>
                    <phase>generate-resources</phase>
                    <goals>
                        <goal>build</goal>
                    </goals>
                    <configuration>
                    </configuration>
                </execution>
            </executions>
        </plugin>
        ...
    </plugins>
</build>
\end{verbatim}

You can view an example POM containing the Theme Builder configuration
\href{https://github.com/liferay/liferay-portal/blob/master/modules/util/portal-tools-theme-builder/samples/pom.xml}{here}.

\subsection{Goals}\label{goals-8}

The plugin adds one Maven goal to your project:

Name \textbar{} Description \texttt{theme-builder:build} \textbar{}
Builds the theme files.

\subsection{Available Parameters}\label{available-parameters-7}

You can set the following parameters in the
\texttt{\textless{}configuration\textgreater{}} section of the POM:

Parameter Name \textbar{} Type \textbar{} Default Value \textbar{}
Description \texttt{diffsDir} \textbar{} \texttt{File} \textbar{}
\texttt{\$\{maven.war.src\}} \textbar{} The directory that contains the
files to copy over the parent theme. \texttt{name} \textbar{}
\texttt{String} \textbar{} \texttt{\$\{project.artifactId\}} \textbar{}
The name of the new theme. \texttt{outputDir} \textbar{} \texttt{File}
\textbar{}
\texttt{\$\{project.build.directory\}/\$\{project.build.finalName\}}
\textbar{} The directory where to build the theme. \texttt{parentDir}
\textbar{} \texttt{File} \textbar{} \texttt{null} \textbar{} The
directory of the parent theme. \texttt{parentName} \textbar{}
\texttt{String} \textbar{} \texttt{null} \textbar{} The name of the
parent theme. \texttt{templateExtension} \textbar{} \texttt{String}
\textbar{} \texttt{"ftl"} \textbar{} The extension of the template
files, usually \texttt{"ftl"} or \texttt{"vm"}. \texttt{unstyledDir}
\textbar{} \texttt{File} \textbar{} \texttt{null} \textbar{} The
directory of
\href{https://github.com/liferay/liferay-portal/tree/master/modules/apps/foundation/frontend-theme/frontend-theme-unstyled}{Liferay
Frontend Theme Unstyled}.

You can also manage the \texttt{com.liferay.frontend.theme.styled} and
\texttt{com.liferay.frontend.theme.unstyled} default theme dependencies
provided by the Theme Builder in your \texttt{pom.xml}. They can be
modified by adding them as project dependencies:

\begin{verbatim}
<project>
    ...
    <dependencies>
        ...
        <dependency>
            <groupId>com.liferay</groupId>
            <artifactId>com.liferay.frontend.theme.styled</artifactId>
            <version>3.0.4</version>
            <scope>provided</scope>
        </dependency>
        <dependency>
            <groupId>com.liferay</groupId>
            <artifactId>com.liferay.frontend.theme.unstyled</artifactId>
            <version>3.0.4</version>
            <scope>provided</scope>
        </dependency>
    </dependencies>
</project>
\end{verbatim}

There is an additional Liferay theme-related dependency you can manage
this way that's provided by the CSS Builder. See
\href{/docs/7-1/reference/-/knowledge_base/r/css-builder-plugin}{this
section} for more information.

\section{TLD Formatter Plugin}\label{tld-formatter-plugin}

The TLD Formatter plugin lets you format a project's TLD files.

\subsection{Usage}\label{usage-38}

To use the plugin, include it in your project's root \texttt{pom.xml}
file:

\begin{verbatim}
<build>
    <plugins>
    ...
        <plugin>
            <groupId>com.liferay</groupId>
            <artifactId>com.liferay.tld.formatter</artifactId>
            <version>1.0.5</version>
            <configuration>
            </configuration>
        </plugin>
    ...
    </plugins>
</build>
\end{verbatim}

You can view an example POM containing the TLD Formatter configuration
\href{https://github.com/liferay/liferay-portal/blob/master/modules/util/tld-formatter/samples/pom.xml}{here}.

\subsection{Goals}\label{goals-9}

The plugin adds one Maven goal to your project:

Name \textbar{} Description \texttt{tld-formatter:format} \textbar{}
Runs the Liferay TLD Formatter to format files.

\subsection{Available Parameters}\label{available-parameters-8}

You can set the following parameters in the
\texttt{\textless{}configuration\textgreater{}} section of the POM:

Parameter Name \textbar{} Type \textbar{} Default Value \textbar{}
Description \texttt{baseDirName} \textbar{} \texttt{String} \textbar{}
\texttt{"./"} \textbar{} The base directory to begin searching for TLD
files to format. \texttt{plugin} \textbar{} \texttt{boolean} \textbar{}
\texttt{true} \textbar{} Whether to format all the TLD files contained
in the working directory. If \texttt{false}, all
\texttt{liferay-portlet-ext.tld} files are ignored.

\section{WSDD Builder Plugin}\label{wsdd-builder-plugin}

The WSDD Builder plugin lets you generate the
\href{http://axis.apache.org/axis/}{Apache Axis} Web Service Deployment
Descriptor (WSDD) files from a
\href{/docs/7-0/tutorials/-/knowledge_base/t/what-is-service-builder}{Service
Builder} \texttt{service.xml} file.

\subsection{Usage}\label{usage-39}

To use the plugin, include it in your project's root \texttt{pom.xml}
file:

\begin{verbatim}
<build>
    <plugins>
    ...
        <plugin>
            <groupId>com.liferay</groupId>
            <artifactId>com.liferay.portal.tools.wsdd.builder</artifactId>
            <version>1.0.10</version>
            <configuration>
            </configuration>
        </plugin>
    ...
    </plugins>
</build>
\end{verbatim}

You can view an example POM containing the WSDD Builder configuration
\href{https://github.com/liferay/liferay-portal/blob/master/modules/util/portal-tools-wsdd-builder/samples/pom.xml}{here}.

\subsection{Goals}\label{goals-10}

The plugin adds one Maven goal to your project:

Name \textbar{} Description \texttt{wsdd-builder:build} \textbar{} Runs
the Liferay WSDD Builder to generate the WSDD files.

\subsection{Available Parameters}\label{available-parameters-9}

You can set the following parameters in the
\texttt{\textless{}configuration\textgreater{}} section of the POM:

Parameter Name \textbar{} Type \textbar{} Default Value \textbar{}
Description \texttt{classPath} \textbar{} \texttt{String} \textbar{}
\texttt{null} \textbar{} The classpath that the Liferay WSDD Builder
uses to generate WSDD files. \texttt{inputFileName} \textbar{}
\texttt{String} \textbar{} \texttt{"service.xml"} \textbar{} The file
from which to generate the WSDD files. \texttt{outputDirName} \textbar{}
\texttt{String} \textbar{} \texttt{"src"} \textbar{} The directory where
the \texttt{*\_deploy.wsdd} and \texttt{*\_undeploy.wsdd} files are
generated. \texttt{serverConfigFileName} \textbar{} \texttt{String}
\textbar{} \texttt{"server-config.wsdd"} \textbar{} The file to
generate. \texttt{serviceNamespace} \textbar{} \texttt{String}
\textbar{} \texttt{"Plugin"} \textbar{} The namespace for the WSDD
Service.

\section{XML Formatter Plugin}\label{xml-formatter-plugin}

The XML Formatter plugin lets you format a project's XML files.

\subsection{Usage}\label{usage-40}

To use the plugin, include it in your project's root \texttt{pom.xml}
file:

\begin{verbatim}
<build>
    <plugins>
    ...
        <plugin>
            <groupId>com.liferay</groupId>
            <artifactId>com.liferay.xml.formatter</artifactId>
            <version>1.0.5</version>
            <configuration>
            </configuration>
        </plugin>
    ...
    </plugins>
</build>
\end{verbatim}

You can view an example POM containing the XML Formatter configuration
\href{https://github.com/liferay/liferay-portal/blob/master/modules/util/xml-formatter/samples/pom.xml}{here}.

\subsection{Goals}\label{goals-11}

The plugin adds one Maven goal to your project:

Name \textbar{} Description \texttt{xml-formatter:format} \textbar{}
Runs the Liferay XML Formatter to format the project files.

\subsection{Available Parameters}\label{available-parameters-10}

You can set the following parameters in the
\texttt{\textless{}configuration\textgreater{}} section of the POM:

Parameter Name \textbar{} Type \textbar{} Default Value \textbar{}
Description \texttt{fileName} \textbar{} \texttt{String} \textbar{}
\texttt{null} \textbar{} The XML file to format. This plugin only lets
you format one XML file at a time. \texttt{stripComments} \textbar{}
\texttt{boolean} \textbar{} \texttt{false} \textbar{} Whether to remove
all the comments from the XML file.

\chapter{Project Templates}\label{project-templates}

Liferay provides project templates that you can use to generate starter
projects formatted in an opinionated way. These templates can be used by
most build tools (e.g., Gradle, Maven, Liferay @ide@) to generate your
desired project structure.

Some popular project templates include

\begin{itemize}
\tightlist
\item
  API project
\item
  Fragment project
\item
  MVC Portlet project
\item
  Service Builder project
\item
  Template Context Contributor project
\item
  Theme project (WAR)
\item
  etc.
\end{itemize}

If you're using
\href{/docs/7-0/tutorials/-/knowledge_base/t/blade-cli}{Blade CLI},
execute the following command to display a full list of project
templates:

\begin{verbatim}
blade create -l
\end{verbatim}

If you're using
\href{/docs/7-0/tutorials/-/knowledge_base/t/maven}{Maven}, you can view
and use the project templates as Maven archetypes. Execute the following
command to list them:

\begin{verbatim}
mvn archetype:generate -Dfilter=liferay
\end{verbatim}

Archetypes with the \texttt{com.liferay.project.templates} prefix are
the latest templates offered by Liferay.

If you're using
\href{/docs/7-0/tutorials/-/knowledge_base/t/liferay-ide}{Liferay
@ide@}, navigate to \emph{File} → \emph{New} → \emph{Liferay Module
Project} and view the project templates from the \emph{Project Template
Name} drop-down menu.

In this section of reference articles, each project template is outlined
with the appropriate generation command and folder structure. Visit the
project template article you're most interested in to start building
your own project!

\section{Activator Template}\label{activator-template}

In this article, you'll learn how to create a Liferay activator as a
Liferay module. To create a Liferay activator via the command line using
Blade CLI or Maven, use one of the commands with the following
parameters:

\begin{verbatim}
blade create -t activator -v 7.0 [-p packageName] [-c className] projectName
\end{verbatim}

or

\begin{verbatim}
mvn archetype:generate \
    -DarchetypeGroupId=com.liferay \
    -DarchetypeArtifactId=com.liferay.project.templates.activator \
    -DartifactId=[projectName] \
    -Dpackage=[packageName] \
    -DclassName=[className] \
    -DliferayVersion=7.0
\end{verbatim}

You can also insert the \texttt{-b\ maven} parameter in the Blade
command to generate a Maven project using Blade CLI.

The template for this kind of project is \texttt{activator}. Suppose you
want to create an activator project called \texttt{my-activator-project}
with a package name of \texttt{com.liferay.docs.activator} and a class
name of \texttt{Activator}. You could run the following command to
accomplish this:

\begin{verbatim}
blade create -t activator -v 7.0 -p com.liferay.docs.activator -c Activator my-activator-project
\end{verbatim}

or

\begin{verbatim}
mvn archetype:generate \
    -DarchetypeGroupId=com.liferay \
    -DarchetypeArtifactId=com.liferay.project.templates.activator \
    -DgroupId=com.liferay \
    -DartifactId=my-activator-project \
    -Dpackage=com.liferay.docs.activator \
    -Dversion=1.0 \
    -DclassName=Activator \
    -Dauthor=Joe Bloggs \
    -DliferayVersion=7.0
\end{verbatim}

Note that in your class, you're implementing the
\texttt{org.osgi.framework.BundleActivator} interface.

After running the command above, your project's directory structure
looks like this:

\begin{itemize}
\tightlist
\item
  \texttt{my-activator-project}

  \begin{itemize}
  \tightlist
  \item
    \texttt{gradle} (only in Blade CLI generated projects)

    \begin{itemize}
    \tightlist
    \item
      \texttt{wrapper}

      \begin{itemize}
      \tightlist
      \item
        \texttt{gradle-wrapper.jar}
      \item
        \texttt{gradle-wrapper.properties}
      \end{itemize}
    \end{itemize}
  \item
    \texttt{src}

    \begin{itemize}
    \tightlist
    \item
      \texttt{main}

      \begin{itemize}
      \tightlist
      \item
        \texttt{java}

        \begin{itemize}
        \tightlist
        \item
          \texttt{com/liferay/docs/activator}

          \begin{itemize}
          \tightlist
          \item
            \texttt{Activator.java}
          \end{itemize}
        \end{itemize}
      \end{itemize}
    \end{itemize}
  \item
    \texttt{bnd.bnd}
  \item
    \texttt{build.gradle}
  \item
    \texttt{{[}gradlew\textbar{}pom.xml{]}}
  \end{itemize}
\end{itemize}

The generated module is functional and is deployable to a Liferay DXP
instance. To build upon the generated app, modify the project by adding
logic and additional files to the folders outlined above.

\section{API Template}\label{api-template}

In this tutorial, you'll learn how to create a Liferay API as a Liferay
module. To create a Liferay API via the command line using Blade CLI or
Maven, use one of the commands with the following parameters:

\begin{verbatim}
blade create -t api -v 7.0 [-p packageName] [-c className] projectName
\end{verbatim}

or

\begin{verbatim}
mvn archetype:generate \
    -DarchetypeGroupId=com.liferay \
    -DarchetypeArtifactId=com.liferay.project.templates.api \
    -DartifactId=[projectName] \
    -Dpackage=[packageName] \
    -DclassName=[className] \
    -DliferayVersion=7.0
\end{verbatim}

You can also insert the \texttt{-b\ maven} parameter in the Blade
command to generate a Maven project using Blade CLI.

The template for this kind of project is \texttt{api}. The \texttt{api}
template creates a simple \texttt{api} module with an empty public
interface. For example, suppose you want to create an API project called
\texttt{my-api-project} with a package name of
\texttt{com.liferay.docs.api} and a class name of \texttt{MyApi}. You
could run the following command to accomplish this:

\begin{verbatim}
blade create -t api -v 7.0 -p com.liferay.docs -c MyApi my-api-project
\end{verbatim}

or

\begin{verbatim}
mvn archetype:generate \
    -DarchetypeGroupId=com.liferay \
    -DarchetypeArtifactId=com.liferay.project.templates.api \
    -DgroupId=com.liferay \
    -DartifactId=my-api-project \
    -Dpackage=com.liferay.docs \
    -Dversion=1.0 \
    -DclassName=MyApi \
    -Dauthor=Joe Bloggs \
    -DliferayVersion=7.0
\end{verbatim}

After running the command above, your project's directory structure
looks like this:

\begin{itemize}
\tightlist
\item
  \texttt{my-api-project}

  \begin{itemize}
  \tightlist
  \item
    \texttt{gradle} (only in Blade CLI generated projects)

    \begin{itemize}
    \tightlist
    \item
      \texttt{wrapper}

      \begin{itemize}
      \tightlist
      \item
        \texttt{gradle-wrapper.jar}
      \item
        \texttt{gradle-wrapper.properties}
      \end{itemize}
    \end{itemize}
  \item
    \texttt{src}

    \begin{itemize}
    \tightlist
    \item
      \texttt{main}

      \begin{itemize}
      \tightlist
      \item
        \texttt{java}

        \begin{itemize}
        \tightlist
        \item
          \texttt{com/liferay/docs/api}

          \begin{itemize}
          \tightlist
          \item
            \texttt{MyApi.java}
          \end{itemize}
        \end{itemize}
      \item
        resources

        \begin{itemize}
        \tightlist
        \item
          \texttt{com/liferay/docs/api}

          \begin{itemize}
          \tightlist
          \item
            \texttt{packageinfo}
          \end{itemize}
        \end{itemize}
      \end{itemize}
    \end{itemize}
  \item
    \texttt{bnd.bnd}
  \item
    \texttt{build.gradle}
  \item
    \texttt{{[}gradlew\textbar{}pom.xml{]}}
  \end{itemize}
\end{itemize}

The generated module is a working application and is deployable to a
Liferay DXP instance. To build upon the generated app, modify the
project by adding logic and additional files to the folders outlined
above.

\section{Control Menu Entry Template}\label{control-menu-entry-template}

In this article, you'll learn how to create a Liferay Control Menu entry
as a Liferay module. To create a Liferay Control Menu entry via the
command line using Blade CLI or Maven, use one of the commands with the
following parameters:

\begin{verbatim}
blade create -t control-menu-entry -v 7.0 [-p packageName] [-c className] projectName
\end{verbatim}

or

\begin{verbatim}
mvn archetype:generate \
    -DarchetypeGroupId=com.liferay \
    -DarchetypeArtifactId=com.liferay.project.templates.control.menu.entry \
    -DartifactId=[projectName] \
    -Dpackage=[packageName] \
    -DclassName=[className] \
    -DliferayVersion=7.0
\end{verbatim}

You can also insert the \texttt{-b\ maven} parameter in the Blade
command to generate a Maven project using Blade CLI.

The template for this kind of project is \texttt{control-menu-entry}.
Suppose you want to create a control menu entry project called
\texttt{my-control-menu-entry-project} with a package name of
\texttt{com.liferay.docs.entry.control.menu} and a class name of
\texttt{SampleProductNavigationControlMenuEntry}. You could run the
following command to accomplish this:

\begin{verbatim}
blade create -t control-menu-entry -v 7.0 -p com.liferay.docs.entry -c Sample my-control-menu-entry-project
\end{verbatim}

or

\begin{verbatim}
mvn archetype:generate \
    -DarchetypeGroupId=com.liferay \
    -DarchetypeArtifactId=com.liferay.project.templates.control.menu.entry \
    -DgroupId=com.liferay \
    -DartifactId=my-control-menu-entry-project \
    -Dpackage=com.liferay.docs.entry \
    -Dversion=1.0 \
    -DclassName=Sample \
    -Dauthor=Joe Bloggs \
    -DliferayVersion=7.0
\end{verbatim}

After running the command above, your project's directory structure
would look like this:

\begin{itemize}
\tightlist
\item
  \texttt{my-control-menu-entry-project}

  \begin{itemize}
  \tightlist
  \item
    \texttt{gradle} (only in Blade CLI generated projects)

    \begin{itemize}
    \tightlist
    \item
      \texttt{wrapper}

      \begin{itemize}
      \tightlist
      \item
        \texttt{gradle-wrapper.jar}
      \item
        \texttt{gradle-wrapper.properties}
      \end{itemize}
    \end{itemize}
  \item
    \texttt{src}

    \begin{itemize}
    \tightlist
    \item
      \texttt{main}

      \begin{itemize}
      \tightlist
      \item
        \texttt{java}

        \begin{itemize}
        \tightlist
        \item
          \texttt{com/liferay/docs/entry/control/menu}

          \begin{itemize}
          \tightlist
          \item
            \texttt{SampleProductNavigationControlMenuEntry.java}
          \end{itemize}
        \end{itemize}
      \item
        \texttt{resources}

        \begin{itemize}
        \tightlist
        \item
          \texttt{content}

          \begin{itemize}
          \tightlist
          \item
            \texttt{Language.properties}
          \end{itemize}
        \end{itemize}
      \end{itemize}
    \end{itemize}
  \item
    \texttt{bnd.bnd}
  \item
    \texttt{build.gradle}
  \item
    \texttt{{[}gradlew\textbar{}pom.xml{]}}
  \end{itemize}
\end{itemize}

The generated module is functional and is deployable to a Liferay DXP
instance. To build upon the generated app, modify the project by adding
logic and additional files to the folders outlined above. You can visit
the
\href{/docs/7-0/reference/-/knowledge_base/r/control-menu-entry}{control-menu-entry}
sample project for a more expanded sample of a Control Menu entry.
Likewise, see the
\href{/docs/7-0/tutorials/-/knowledge_base/t/customizing-the-control-menu}{Customizing
the Control Menu} tutorial for instructions on customizing a Control
Menu entry project.

\section{Form Field Template}\label{form-field-template}

In this article, you'll learn how to create a Liferay form field as a
Liferay module. To create a Liferay form field via the command line
using Blade CLI or Maven, use one of the commands with the following
parameters:

\begin{verbatim}
blade create -t form-field -v 7.0 [-p packageName] [-c className] projectName
\end{verbatim}

or

\begin{verbatim}
mvn archetype:generate \
    -DarchetypeGroupId=com.liferay \
    -DarchetypeArtifactId=com.liferay.project.templates.form.field \
    -DartifactId=[projectName] \
    -Dpackage=[packageName] \
    -DclassName=[className] \
    -DliferayVersion=7.0
\end{verbatim}

You can also insert the \texttt{-b\ maven} parameter in the Blade
command to generate a Maven project using Blade CLI.

The template for this kind of project is \texttt{form-field}. Suppose
you want to create a form field project called
\texttt{my-form-field-project} with a package name of
\texttt{com.liferay.docs.form.field} and a class name prefix of
\texttt{MyFormField}. You could run one of the following commands to
accomplish this:

\begin{verbatim}
blade create -t form-field -v 7.0 -p com.liferay.docs -c MyFormField my-form-field-project
\end{verbatim}

or

\begin{verbatim}
mvn archetype:generate \
    -DarchetypeGroupId=com.liferay \
    -DarchetypeArtifactId=com.liferay.project.templates.form.field \
    -DgroupId=com.liferay \
    -DartifactId=my-form-field-project \
    -Dpackage=com.liferay.docs \
    -Dversion=1.0 \
    -DclassName=MyFormField \
    -Dauthor=Joe Bloggs \
    -DliferayVersion=7.0
\end{verbatim}

After running the command above, your project's directory structure
looks like this:

\begin{itemize}
\tightlist
\item
  \texttt{my-form-field-project}

  \begin{itemize}
  \tightlist
  \item
    \texttt{gradle} (only in Blade CLI generated projects)

    \begin{itemize}
    \tightlist
    \item
      \texttt{wrapper}

      \begin{itemize}
      \tightlist
      \item
        \texttt{gradle-wrapper.jar}
      \item
        \texttt{gradle-wrapper.properties}
      \end{itemize}
    \end{itemize}
  \item
    \texttt{src}

    \begin{itemize}
    \tightlist
    \item
      \texttt{main}

      \begin{itemize}
      \tightlist
      \item
        \texttt{java}

        \begin{itemize}
        \tightlist
        \item
          \texttt{com/liferay/docs/form/field}

          \begin{itemize}
          \tightlist
          \item
            \texttt{MyFormFieldDDMFormFieldRenderer.java}
          \item
            \texttt{MyFormFieldDDMFormFieldType.java}
          \end{itemize}
        \end{itemize}
      \item
        \texttt{resources}

        \begin{itemize}
        \tightlist
        \item
          \texttt{content}

          \begin{itemize}
          \tightlist
          \item
            \texttt{Language.properties}
          \end{itemize}
        \item
          \texttt{META-INF}

          \begin{itemize}
          \tightlist
          \item
            \texttt{resources}

            \begin{itemize}
            \tightlist
            \item
              \texttt{config.js}
            \item
              \texttt{my-form-field-project.soy}
            \item
              \texttt{my-form-field-project\_field.js}
            \end{itemize}
          \end{itemize}
        \end{itemize}
      \end{itemize}
    \end{itemize}
  \item
    \texttt{bnd.bnd}
  \item
    \texttt{build.gradle}
  \item
    \texttt{{[}gradlew\textbar{}pom.xml{]}}
  \end{itemize}
\end{itemize}

The generated module is a working form field and is deployable to a
Liferay DXP instance. To build upon the generated app, modify the
project by adding logic and additional files to the folders outlined
above.

\section{Fragment Template}\label{fragment-template}

In this article, you'll learn how to create a Liferay fragment as a
Liferay module. You can learn more about fragment modules in the
\href{/docs/7-0/tutorials/-/knowledge_base/t/overriding-a-modules-jsps\#declaring-a-fragment-host}{Declaring
a Fragment Host} article and in section 3.14 of the
\href{https://osgi.org/download/r6/osgi.core-6.0.0.pdf}{OSGi Alliance's
core specification document}.

To create a Liferay fragment via the command line using Blade CLI or
Maven, use one of the commands with the following parameters:

\begin{verbatim}
blade create -t fragment -v 7.0 [-h hostBundleName] [-H hostBundleVersion] projectName
\end{verbatim}

or

\begin{verbatim}
mvn archetype:generate \
    -DarchetypeGroupId=com.liferay \
    -DarchetypeArtifactId=com.liferay.project.templates.fragment \
    -DartifactId=[projectName] \
    -Dpackage=[packageName] \
    -DclassName=[className] \
    -DliferayVersion=7.0
\end{verbatim}

You can also insert the \texttt{-b\ maven} parameter in the Blade
command to generate a Maven project using Blade CLI.

The template for this kind of project is \texttt{fragment}. Suppose you
want to create a fragment project called \texttt{my-fragment-project}
with a host bundle symbolic name of \texttt{com.liferay.login.web} and
host bundle version of \texttt{1.0.0}. You could run the following
command to accomplish this:

\begin{verbatim}
blade create -t fragment -v 7.0 -h com.liferay.login.web -H 1.0.0 my-fragment-project
\end{verbatim}

or

\begin{verbatim}
mvn archetype:generate \
    -DarchetypeGroupId=com.liferay \
    -DarchetypeArtifactId=com.liferay.project.templates.fragment \
    -DgroupId=com.liferay \
    -DartifactId=my-fragment-project \
    -Dversion=1.0 \
    -Dpackage= \
    -DhostBundleSymbolicName=com.liferay.login.web \
    -DhostBundleVersion=1.0.0 \
    -DliferayVersion=7.0
\end{verbatim}

The folder structure is created, but there are no files. The only files
created are the \texttt{bnd.bnd} and \texttt{build.gradle} files, which
specify your host bundle and its information, and your build tool's
files. After running the command above, your project's directory
structure looks like this:

\begin{itemize}
\tightlist
\item
  \texttt{my-fragment-project}

  \begin{itemize}
  \tightlist
  \item
    \texttt{gradle} (only in Blade CLI generated projects)

    \begin{itemize}
    \tightlist
    \item
      \texttt{wrapper}

      \begin{itemize}
      \tightlist
      \item
        \texttt{gradle-wrapper.jar}
      \item
        \texttt{gradle-wrapper.properties}
      \end{itemize}
    \end{itemize}
  \item
    \texttt{src}

    \begin{itemize}
    \tightlist
    \item
      \texttt{main}

      \begin{itemize}
      \tightlist
      \item
        \texttt{java} -\texttt{resources} -\texttt{META-INF}
        -\texttt{resources}
      \end{itemize}
    \end{itemize}
  \item
    \texttt{bnd.bnd}
  \item
    \texttt{build.gradle}
  \item
    \texttt{{[}gradlew\textbar{}pom.xml{]}}
  \end{itemize}
\end{itemize}

The generated module is functional and is deployable to a Liferay DXP
instance. To build upon the generated app, modify the project by adding
logic and additional files to the folders outlined above.

\section{FreeMarker Portlet Template}\label{freemarker-portlet-template}

In this article, you'll learn how to create a Liferay FreeMarker portlet
application as a Liferay module. To create a Liferay FreeMarker portlet
application via the command line using Blade CLI or Maven, use one of
the commands with the following parameters:

\begin{verbatim}
blade create -t freemarker-portlet -v 7.0 [-p packageName] [-c className] projectName
\end{verbatim}

or

\begin{verbatim}
mvn archetype:generate \
    -DarchetypeGroupId=com.liferay \
    -DarchetypeArtifactId=com.liferay.project.templates.freemarker.portlet \
    -DartifactId=[projectName] \
    -Dpackage=[packageName] \
    -DclassName=[className] \
    -DliferayVersion=7.0
\end{verbatim}

You can also insert the \texttt{-b\ maven} parameter in the Blade
command to generate a Maven project using Blade CLI.

The template for this kind of project is \texttt{freemarker-portlet}.
Suppose you want to create a FreeMarker portlet project called
\texttt{my-freemarker-portlet-project} with a package name of
\texttt{com.liferay.docs.freemarkerportlet} and a class name of
\texttt{MyFreemarkerPortlet}. Also, you'd like to create a service of
type \texttt{javax.portlet.Portlet} that extends the
\texttt{com.liferay.util.bridges.freemarker.FreeMarkerPortlet} class.
Here, \emph{service} means an OSGi service, not a Liferay API. Another
way to say \emph{service type} is to say \emph{component type}. You
could run the following command to accomplish this:

\begin{verbatim}
blade create -t freemarker-portlet -v 7.0 -p com.liferay.docs.freemarkerportlet -c MyFreemarkerPortlet my-freemarker-portlet-project
\end{verbatim}

or

\begin{verbatim}
mvn archetype:generate \
    -DarchetypeGroupId=com.liferay \
    -DarchetypeArtifactId=com.liferay.project.templates.freemarker.portlet \
    -DgroupId=com.liferay \
    -DartifactId=my-freemarker-portlet-project \
    -Dpackage=com.liferay.docs.freemarkerportlet \
    -Dversion=1.0 \
    -DclassName=MyFreemarkerPortlet \
    -Dauthor=Joe Bloggs \
    -DliferayVersion=7.0
\end{verbatim}

After running the command above, your project's directory structure
looks like this:

\begin{itemize}
\tightlist
\item
  \texttt{my-freemarker-portlet-project}

  \begin{itemize}
  \tightlist
  \item
    \texttt{gradle} (only in Blade CLI generated projects)

    \begin{itemize}
    \tightlist
    \item
      \texttt{wrapper}

      \begin{itemize}
      \tightlist
      \item
        \texttt{gradle-wrapper.jar}
      \item
        \texttt{gradle-wrapper.properties}
      \end{itemize}
    \end{itemize}
  \item
    \texttt{src}

    \begin{itemize}
    \tightlist
    \item
      \texttt{main}

      \begin{itemize}
      \tightlist
      \item
        \texttt{java}

        \begin{itemize}
        \tightlist
        \item
          \texttt{com/liferay/docs/freemarkerportlet}

          \begin{itemize}
          \tightlist
          \item
            \texttt{constants}

            \begin{itemize}
            \tightlist
            \item
              \texttt{MyFreemarkerPortletKeys.java}
            \end{itemize}
          \item
            \texttt{portlet}

            \begin{itemize}
            \tightlist
            \item
              \texttt{MyFreemarkerPortlet.java}
            \end{itemize}
          \end{itemize}
        \end{itemize}
      \item
        \texttt{resources}

        \begin{itemize}
        \tightlist
        \item
          \texttt{content}

          \begin{itemize}
          \tightlist
          \item
            \texttt{Language.properties}
          \end{itemize}
        \item
          \texttt{META-INF}

          \begin{itemize}
          \tightlist
          \item
            \texttt{resources}

            \begin{itemize}
            \tightlist
            \item
              \texttt{css}

              \begin{itemize}
              \tightlist
              \item
                \texttt{main.scss}
              \end{itemize}
            \end{itemize}
          \end{itemize}
        \item
          \texttt{templates}

          \begin{itemize}
          \tightlist
          \item
            \texttt{init.ftl}
          \item
            \texttt{view.ftl}
          \end{itemize}
        \end{itemize}
      \end{itemize}
    \end{itemize}
  \item
    \texttt{bnd.bnd}
  \item
    \texttt{build.gradle}
  \item
    \texttt{{[}gradlew\textbar{}pom.xml{]}}
  \end{itemize}
\end{itemize}

The generated module is a working application and is deployable to a
Liferay DXP instance. To build upon the generated app, modify the
project by adding logic and additional files to the folders outlined
above.

\section{Layout Template}\label{layout-template}

In this article, you'll learn how to create a Liferay layout template as
a WAR project. To create a Liferay layout template via the command line
using Blade CLI or Maven, use one of the commands with the following
parameters:

\begin{verbatim}
blade create -t layout-template -v 7.0 projectName
\end{verbatim}

or

\begin{verbatim}
mvn archetype:generate \
    -DarchetypeGroupId=com.liferay \
    -DarchetypeArtifactId=com.liferay.project.templates.layout.template \
    -DartifactId=[projectName] \
    -DliferayVersion=7.0
\end{verbatim}

You can also insert the \texttt{-b\ maven} parameter in the Blade
command to generate a Maven project using Blade CLI.

The template for this kind of project is \texttt{layout-template}.
Suppose you want to create a layout template project called
\texttt{my-layout-template-project}. You could run one of the following
commands to accomplish this:

\begin{verbatim}
blade create -t layout-template -v 7.0 my-layout-template-project
\end{verbatim}

or

\begin{verbatim}
mvn archetype:generate \
    -DarchetypeGroupId=com.liferay \
    -DarchetypeArtifactId=com.liferay.project.templates.layout.template \
    -DgroupId=com.liferay \
    -DartifactId=my-layout-template-project \
    -Dversion=1.0 \
    -Dauthor=Joe Bloggs \
    -DliferayVersion=7.0
\end{verbatim}

After running the command above, your project's directory structure
looks like this:

\begin{itemize}
\tightlist
\item
  \texttt{my-layout-template-project}

  \begin{itemize}
  \tightlist
  \item
    \texttt{gradle} (only in Blade CLI generated projects)

    \begin{itemize}
    \tightlist
    \item
      \texttt{wrapper}

      \begin{itemize}
      \tightlist
      \item
        \texttt{gradle-wrapper.jar}
      \item
        \texttt{gradle-wrapper.properties}
      \end{itemize}
    \end{itemize}
  \item
    \texttt{src}

    \begin{itemize}
    \tightlist
    \item
      \texttt{main}

      \begin{itemize}
      \tightlist
      \item
        \texttt{webapp}

        \begin{itemize}
        \tightlist
        \item
          \texttt{WEB-INF}

          \begin{itemize}
          \tightlist
          \item
            \texttt{liferay-layout-templates.xml}
          \item
            \texttt{liferay-plugin-package.properties}
          \end{itemize}
        \item
          \texttt{my-layout-template-project.png}
        \item
          \texttt{my-layout-template-project.tpl}
        \end{itemize}
      \end{itemize}
    \end{itemize}
  \item
    \texttt{build.gradle}
  \item
    \texttt{{[}gradlew\textbar{}pom.xml{]}}
  \end{itemize}
\end{itemize}

The generated WAR is a working layout template and is deployable to a
Liferay DXP instance. To build upon the generated layout template,
modify the project by adding logic and additional files to the folders
outlined above.

\section{MVC Portlet Template}\label{mvc-portlet-template}

In this article, you'll learn how to create a Liferay MVC portlet
application as a Liferay module. To create a Liferay MVC portlet
application via the command line using Blade CLI or Maven, use one of
the commands with the following parameters:

\begin{verbatim}
blade create -t mvc-portlet -v 7.0 [-p packageName] [-c className] projectName
\end{verbatim}

or

\begin{verbatim}
mvn archetype:generate \
    -DarchetypeGroupId=com.liferay \
    -DarchetypeArtifactId=com.liferay.project.templates.mvc.portlet \
    -DartifactId=[projectName] \
    -Dpackage=[packageName] \
    -DclassName=[className] \
    -DliferayVersion=7.0
\end{verbatim}

You can also insert the \texttt{-b\ maven} parameter in the Blade
command to generate a Maven project using Blade CLI.

The template for this kind of project is \texttt{mvc-portlet}. Suppose
you want to create an MVC portlet project called
\texttt{my-mvc-portlet-project} with a package name of
\texttt{com.liferay.docs.mvcportlet} and a class name of
\texttt{MyMvcPortlet}. Also, you'd like to create a service of type
\texttt{javax.portlet.Portlet} that extends the
\texttt{com.liferay.portal.kernel.portlet.bridges.mvc.MVCPortlet} class.
Here, \emph{service} means an OSGi service, not a Liferay API. Another
way to say \emph{service type} is to say \emph{component type}. You
could run the following command to accomplish this:

\begin{verbatim}
blade create -t mvc-portlet -v 7.0 -p com.liferay.docs.mvcportlet -c MyMvcPortlet my-mvc-portlet-project
\end{verbatim}

or

\begin{verbatim}
mvn archetype:generate \
    -DarchetypeGroupId=com.liferay \
    -DarchetypeArtifactId=com.liferay.project.templates.mvc.portlet \
    -DgroupId=com.liferay \
    -DartifactId=my-mvc-portlet-project \
    -Dpackage=com.liferay.docs.mvcportlet \
    -Dversion=1.0 \
    -DclassName=MyMvcPortlet \
    -Dauthor=Joe Bloggs \
    -DliferayVersion=7.0
\end{verbatim}

After running the command above, your project's directory structure
looks like this:

\begin{itemize}
\tightlist
\item
  \texttt{my-mvc-portlet-project}

  \begin{itemize}
  \tightlist
  \item
    \texttt{gradle} (only in Blade CLI generated projects)

    \begin{itemize}
    \tightlist
    \item
      \texttt{wrapper}

      \begin{itemize}
      \tightlist
      \item
        \texttt{gradle-wrapper.jar}
      \item
        \texttt{gradle-wrapper.properties}
      \end{itemize}
    \end{itemize}
  \item
    \texttt{src}

    \begin{itemize}
    \tightlist
    \item
      \texttt{main}

      \begin{itemize}
      \tightlist
      \item
        \texttt{java}

        \begin{itemize}
        \tightlist
        \item
          \texttt{com/liferay/docs/mvcportlet}

          \begin{itemize}
          \tightlist
          \item
            \texttt{MyMvcPortlet.java}
          \end{itemize}
        \end{itemize}
      \item
        \texttt{resources}

        \begin{itemize}
        \tightlist
        \item
          \texttt{content}

          \begin{itemize}
          \tightlist
          \item
            \texttt{Language.properties}
          \end{itemize}
        \item
          \texttt{META-INF}

          \begin{itemize}
          \tightlist
          \item
            \texttt{resources}

            \begin{itemize}
            \tightlist
            \item
              \texttt{init.jsp}
            \item
              \texttt{view.jsp}
            \end{itemize}
          \end{itemize}
        \end{itemize}
      \end{itemize}
    \end{itemize}
  \item
    \texttt{bnd.bnd}
  \item
    \texttt{build.gradle}
  \item
    \texttt{{[}gradlew\textbar{}pom.xml{]}}
  \end{itemize}
\end{itemize}

The generated module is a working application and is deployable to a
Liferay DXP instance. To build upon the generated app, modify the
project by adding logic and additional files to the folders outlined
above.

\section{npm Angular Portlet
Template}\label{npm-angular-portlet-template}

In this article, you'll learn how to create an npm Angular portlet as a
Liferay module. To create an npm Angular portlet via the command line
using Blade CLI or Maven, use one of the commands with the following
parameters:

\begin{verbatim}
blade create -t npm-angular-portlet -v 7.0 [-p packageName] [-c className] projectName
\end{verbatim}

or

\begin{verbatim}
mvn archetype:generate \
    -DarchetypeGroupId=com.liferay \
    -DarchetypeArtifactId=com.liferay.project.templates.npm.angular.portlet \
    -DartifactId=[projectName] \
    -Dpackage=[packageName] \
    -DclassName=[className] \
    -DliferayVersion=7.0
\end{verbatim}

You can also insert the \texttt{-b\ maven} parameter in the Blade
command to generate a Maven project using Blade CLI.

\noindent\hrulefill

\textbf{Note:} The minifier fails on Liferay DXP 7.0 when JSDoc is
present in a portlet. To resolve this, use
\href{https://gruntjs.com/getting-started}{Grunt}
\href{https://www.npmjs.com/package/grunt-contrib-uglify}{uglify} to
remove the JSDoc comments. This process may take a long time, depending
on the number of files that require an update.

\noindent\hrulefill

The template for this kind of project is \texttt{npm-angular-portlet}.
Suppose you want to create an npm Angular portlet project called
\texttt{my-npm-angular-portlet} with a package name of
\texttt{com.liferay.npm.angular} and a class name of
\texttt{MyNpmAngularPortlet}. Also, you'd like to create a service of
type \texttt{javax.portlet.Portlet} that extends the
\texttt{com.liferay.portal.kernel.portlet.bridges.mvc.MVCPortlet} class.
Here, \emph{service} means an OSGi service, not a Liferay API. Another
way to say \emph{service type} is to say \emph{component type}. You
could run the following command to accomplish this:

\begin{verbatim}
blade create -t npm-angular-portlet -v 7.0 -p com.liferay.npm.angular -c MyNpmAngularPortlet my-npm-angular-portlet
\end{verbatim}

or

\begin{verbatim}
mvn archetype:generate \
    -DarchetypeGroupId=com.liferay \
    -DarchetypeArtifactId=com.liferay.project.templates.npm.angular.portlet \
    -DgroupId=com.liferay \
    -DartifactId=my-npm-angular-portlet \
    -Dpackage=com.liferay.npm.angular \
    -Dversion=1.0 \
    -DclassName=MyNpmAngularPortlet \
    -DpackageJsonVersion=1.0.0 \
    -DliferayVersion=7.0
\end{verbatim}

After running the command above, your project's directory structure
looks like this:

\begin{itemize}
\tightlist
\item
  \texttt{my-npm-angular-portlet}

  \begin{itemize}
  \tightlist
  \item
    \texttt{{[}gradle\textbar{}.mvn{]}}

    \begin{itemize}
    \tightlist
    \item
      \texttt{wrapper}

      \begin{itemize}
      \tightlist
      \item
        \texttt{{[}gradle\textbar{}maven{]}-wrapper.jar}
      \item
        \texttt{{[}gradle\textbar{}maven{]}-wrapper.properties}
      \end{itemize}
    \end{itemize}
  \item
    \texttt{src}

    \begin{itemize}
    \tightlist
    \item
      \texttt{main}

      \begin{itemize}
      \tightlist
      \item
        \texttt{java}

        \begin{itemize}
        \tightlist
        \item
          \texttt{com/liferay/npm/angular}

          \begin{itemize}
          \tightlist
          \item
            \texttt{constants}

            \begin{itemize}
            \tightlist
            \item
              \texttt{MyNpmAngularPortletKeys.java}
            \end{itemize}
          \item
            \texttt{portlet}

            \begin{itemize}
            \tightlist
            \item
              \texttt{MyNpmAngularPortlet.java}
            \end{itemize}
          \end{itemize}
        \end{itemize}
      \item
        \texttt{resources}

        \begin{itemize}
        \tightlist
        \item
          \texttt{content}

          \begin{itemize}
          \tightlist
          \item
            \texttt{Language.properties}
          \end{itemize}
        \item
          \texttt{META-INF}

          \begin{itemize}
          \tightlist
          \item
            \texttt{resources}

            \begin{itemize}
            \tightlist
            \item
              \texttt{js}

              \begin{itemize}
              \tightlist
              \item
                \texttt{app}

                \begin{itemize}
                \tightlist
                \item
                  \texttt{app.component.ts}
                \item
                  \texttt{app.module.ts}
                \end{itemize}
              \item
                \texttt{angular-loader.ts}
              \item
                \texttt{main.ts}
              \end{itemize}
            \item
              \texttt{init.jsp}
            \item
              \texttt{view.jsp}
            \end{itemize}
          \end{itemize}
        \end{itemize}
      \end{itemize}
    \end{itemize}
  \item
    \texttt{.babelrc}
  \item
    \texttt{.npmbundlerrc}
  \item
    \texttt{bnd.bnd}
  \item
    \texttt{{[}build.gradle\textbar{}pom.xml{]}}
  \item
    \texttt{{[}gradlew\textbar{}mvnw{]}}
  \item
    \texttt{package.json}
  \item
    \texttt{tsconfig.json}
  \end{itemize}
\end{itemize}

The generated module is a working application and is deployable to a
Liferay DXP instance. To build upon the generated portlet, modify the
project by adding logic and additional files to the folders outlined
above.

\section{npm Billboard.js Portlet
Template}\label{npm-billboard.js-portlet-template}

In this article, you'll learn how to create an npm Billboard.js portlet
as a Liferay module. To create an npm Billboard.js portlet via the
command line using Blade CLI or Maven, use one of the commands with the
following parameters:

\begin{verbatim}
blade create -t npm-billboardjs-portlet -v 7.0 [-p packageName] [-c className] projectName
\end{verbatim}

or

\begin{verbatim}
mvn archetype:generate \
    -DarchetypeGroupId=com.liferay \
    -DarchetypeArtifactId=com.liferay.project.templates.npm.billboardjs.portlet \
    -DartifactId=[projectName] \
    -Dpackage=[packageName] \
    -DclassName=[className] \
    -DliferayVersion=7.0
\end{verbatim}

You can also insert the \texttt{-b\ maven} parameter in the Blade
command to generate a Maven project using Blade CLI.

\noindent\hrulefill

\textbf{Note:} The minifier fails on Liferay DXP 7.0 when JSDoc is
present in a portlet. To resolve this, use
\href{https://gruntjs.com/getting-started}{Grunt}
\href{https://www.npmjs.com/package/grunt-contrib-uglify}{uglify} to
remove the JSDoc comments. This process may take a long time, depending
on the number of files that require an update.

\noindent\hrulefill

The template for this kind of project is
\texttt{npm-billboardjs-portlet}. Suppose you want to create an npm
Billboard.js portlet project called \texttt{my-npm-billboardjs-portlet}
with a package name of \texttt{com.liferay.npm.billboardjs} and a class
name of \texttt{MyNpmBillboardjsPortlet}. Also, you'd like to create a
service of type \texttt{javax.portlet.Portlet} that extends the
\texttt{com.liferay.portal.kernel.portlet.bridges.mvc.MVCPortlet} class.
Here, \emph{service} means an OSGi service, not a Liferay API. Another
way to say \emph{service type} is to say \emph{component type}. You
could run the following command to accomplish this:

\begin{verbatim}
blade create -t npm-billboardjs-portlet -v 7.0 -p com.liferay.npm.billboardjs -c MyNpmBillboardjsPortlet my-npm-billboardjs-portlet
\end{verbatim}

or

\begin{verbatim}
mvn archetype:generate \
    -DarchetypeGroupId=com.liferay \
    -DarchetypeArtifactId=com.liferay.project.templates.npm.billboardjs.portlet \
    -DgroupId=com.liferay \
    -DartifactId=my-npm-billboardjs-portlet \
    -Dpackage=com.liferay.npm.billboardjs \
    -Dversion=1.0 \
    -DclassName=MyNpmBillboardjsPortlet \
    -DpackageJsonVersion=1.0.0 \
    -DliferayVersion=7.0
\end{verbatim}

After running the command above, your project's directory structure
looks like this:

\begin{itemize}
\tightlist
\item
  \texttt{my-npm-billboardjs-portlet}

  \begin{itemize}
  \tightlist
  \item
    \texttt{{[}gradle\textbar{}.mvn{]}}

    \begin{itemize}
    \tightlist
    \item
      \texttt{wrapper}

      \begin{itemize}
      \tightlist
      \item
        \texttt{{[}gradle\textbar{}maven{]}-wrapper.jar}
      \item
        \texttt{{[}gradle\textbar{}maven{]}-wrapper.properties}
      \end{itemize}
    \end{itemize}
  \item
    \texttt{src}

    \begin{itemize}
    \tightlist
    \item
      \texttt{main}

      \begin{itemize}
      \tightlist
      \item
        \texttt{java}

        \begin{itemize}
        \tightlist
        \item
          \texttt{com/liferay/npm/billboardjs}

          \begin{itemize}
          \tightlist
          \item
            \texttt{constants}

            \begin{itemize}
            \tightlist
            \item
              \texttt{MyNpmBillboardjsPortletKeys.java}
            \end{itemize}
          \item
            \texttt{portlet}

            \begin{itemize}
            \tightlist
            \item
              \texttt{MyNpmBillboardjsPortlet.java}
            \end{itemize}
          \end{itemize}
        \end{itemize}
      \item
        \texttt{resources}

        \begin{itemize}
        \tightlist
        \item
          \texttt{content}

          \begin{itemize}
          \tightlist
          \item
            \texttt{Language.properties}
          \end{itemize}
        \item
          \texttt{META-INF}

          \begin{itemize}
          \tightlist
          \item
            \texttt{resources}

            \begin{itemize}
            \tightlist
            \item
              \texttt{js}

              \begin{itemize}
              \tightlist
              \item
                \texttt{data.json}
              \item
                \texttt{index.es.js}
              \end{itemize}
            \item
              \texttt{init.jsp}
            \item
              \texttt{view.jsp}
            \end{itemize}
          \end{itemize}
        \end{itemize}
      \end{itemize}
    \end{itemize}
  \item
    \texttt{.babelrc}
  \item
    \texttt{.npmbundlerrc}
  \item
    \texttt{bnd.bnd}
  \item
    \texttt{{[}build.gradle\textbar{}pom.xml{]}}
  \item
    \texttt{{[}gradlew\textbar{}mvnw{]}}
  \item
    \texttt{package.json}
  \end{itemize}
\end{itemize}

The generated module is a working application and is deployable to a
Liferay DXP instance. To build upon the generated portlet, modify the
project by adding logic and additional files to the folders outlined
above.

\section{npm Isomorphic Portlet
Template}\label{npm-isomorphic-portlet-template}

In this article, you'll learn how to create an npm Isomorphic portlet as
a Liferay module. To create an npm Isomorphic portlet via the command
line using Blade CLI or Maven, use one of the commands with the
following parameters:

\begin{verbatim}
blade create -t npm-isomorphic-portlet -v 7.0 [-p packageName] [-c className] projectName
\end{verbatim}

or

\begin{verbatim}
mvn archetype:generate \
    -DarchetypeGroupId=com.liferay \
    -DarchetypeArtifactId=com.liferay.project.templates.npm.isomorphic.portlet \
    -DartifactId=[projectName] \
    -Dpackage=[packageName] \
    -DclassName=[className] \
    -DliferayVersion=7.0
\end{verbatim}

You can also insert the \texttt{-b\ maven} parameter in the Blade
command to generate a Maven project using Blade CLI.

\noindent\hrulefill

\textbf{Note:} The minifier fails on Liferay DXP 7.0 when JSDoc is
present in a portlet. To resolve this, use
\href{https://gruntjs.com/getting-started}{Grunt}
\href{https://www.npmjs.com/package/grunt-contrib-uglify}{uglify} to
remove the JSDoc comments. This process may take a long time, depending
on the number of files that require an update.

\noindent\hrulefill

The template for this kind of project is
\texttt{npm-isomorphic-portlet}. Suppose you want to create an npm
Isomorphic portlet project called \texttt{my-npm-isomorphic-portlet}
with a package name of \texttt{com.liferay.npm.isomorphic} and a class
name of \texttt{MyNpmIsomorphicPortlet}. Also, you'd like to create a
service of type \texttt{javax.portlet.Portlet} that extends the
\texttt{com.liferay.portal.kernel.portlet.bridges.mvc.MVCPortlet} class.
Here, \emph{service} means an OSGi service, not a Liferay API. Another
way to say \emph{service type} is to say \emph{component type}. You
could run the following command to accomplish this:

\begin{verbatim}
blade create -t npm-isomorphic-portlet -v 7.0 -p com.liferay.npm.isomorphic -c MyNpmIsomorphicPortlet my-npm-isomorphic-portlet
\end{verbatim}

or

\begin{verbatim}
mvn archetype:generate \
    -DarchetypeGroupId=com.liferay \
    -DarchetypeArtifactId=com.liferay.project.templates.npm.isomorphic.portlet \
    -DgroupId=com.liferay \
    -DartifactId=my-npm-isomorphic-portlet \
    -Dpackage=com.liferay.npm.isomorphic \
    -Dversion=1.0 \
    -DclassName=MyNpmIsomorphicPortlet \
    -DpackageJsonVersion=1.0.0 \
    -DliferayVersion=7.0
\end{verbatim}

After running the command above, your project's directory structure
looks like this:

\begin{itemize}
\tightlist
\item
  \texttt{my-npm-isomorphic-portlet}

  \begin{itemize}
  \tightlist
  \item
    \texttt{{[}gradle\textbar{}.mvn{]}}

    \begin{itemize}
    \tightlist
    \item
      \texttt{wrapper}

      \begin{itemize}
      \tightlist
      \item
        \texttt{{[}gradle\textbar{}maven{]}-wrapper.jar}
      \item
        \texttt{{[}gradle\textbar{}maven{]}-wrapper.properties}
      \end{itemize}
    \end{itemize}
  \item
    \texttt{src}

    \begin{itemize}
    \tightlist
    \item
      \texttt{main}

      \begin{itemize}
      \tightlist
      \item
        \texttt{java}

        \begin{itemize}
        \tightlist
        \item
          \texttt{com/liferay/npm/isomorphic}

          \begin{itemize}
          \tightlist
          \item
            \texttt{constants}

            \begin{itemize}
            \tightlist
            \item
              \texttt{MyNpmIsomorphicPortletKeys.java}
            \end{itemize}
          \item
            \texttt{portlet}

            \begin{itemize}
            \tightlist
            \item
              \texttt{MyNpmIsomorphicPortlet.java}
            \end{itemize}
          \end{itemize}
        \end{itemize}
      \item
        \texttt{resources}

        \begin{itemize}
        \tightlist
        \item
          \texttt{content}

          \begin{itemize}
          \tightlist
          \item
            \texttt{Language.properties}
          \end{itemize}
        \item
          \texttt{META-INF}

          \begin{itemize}
          \tightlist
          \item
            \texttt{resources}

            \begin{itemize}
            \tightlist
            \item
              \texttt{js}

              \begin{itemize}
              \tightlist
              \item
                \texttt{index.es.js}
              \end{itemize}
            \item
              \texttt{init.jsp}
            \item
              \texttt{view.jsp}
            \end{itemize}
          \end{itemize}
        \end{itemize}
      \end{itemize}
    \end{itemize}
  \item
    \texttt{.babelrc}
  \item
    \texttt{.npmbundlerrc}
  \item
    \texttt{bnd.bnd}
  \item
    \texttt{{[}build.gradle\textbar{}pom.xml{]}}
  \item
    \texttt{{[}gradlew\textbar{}mvnw{]}}
  \item
    \texttt{package.json}
  \end{itemize}
\end{itemize}

The generated module is a working application and is deployable to a
Liferay DXP instance. To build upon the generated portlet, modify the
project by adding logic and additional files to the folders outlined
above.

\section{npm jQuery Portlet Template}\label{npm-jquery-portlet-template}

In this article, you'll learn how to create an npm jQuery portlet as a
Liferay module. To create an npm jQuery portlet via the command line
using Blade CLI or Maven, use one of the commands with the following
parameters:

\begin{verbatim}
blade create -t npm-jquery-portlet -v 7.0 [-p packageName] [-c className] projectName
\end{verbatim}

or

\begin{verbatim}
mvn archetype:generate \
    -DarchetypeGroupId=com.liferay \
    -DarchetypeArtifactId=com.liferay.project.templates.npm.jquery.portlet \
    -DartifactId=[projectName] \
    -Dpackage=[packageName] \
    -DclassName=[className] \
    -DliferayVersion=7.0
\end{verbatim}

You can also insert the \texttt{-b\ maven} parameter in the Blade
command to generate a Maven project using Blade CLI.

\noindent\hrulefill

\textbf{Note:} The minifier fails on Liferay DXP 7.0 when JSDoc is
present in a portlet. To resolve this, use
\href{https://gruntjs.com/getting-started}{Grunt}
\href{https://www.npmjs.com/package/grunt-contrib-uglify}{uglify} to
remove the JSDoc comments. This process may take a long time, depending
on the number of files that require an update.

\noindent\hrulefill

The template for this kind of project is \texttt{npm-jquery-portlet}.
Suppose you want to create an npm jQuery portlet project called
\texttt{my-npm-jquery-portlet} with a package name of
\texttt{com.liferay.npm.jquery} and a class name of
\texttt{MyNpmjQueryPortlet}. Also, you'd like to create a service of
type \texttt{javax.portlet.Portlet} that extends the
\texttt{com.liferay.portal.kernel.portlet.bridges.mvc.MVCPortlet} class.
Here, \emph{service} means an OSGi service, not a Liferay API. Another
way to say \emph{service type} is to say \emph{component type}. You
could run the following command to accomplish this:

\begin{verbatim}
blade create -t npm-jquery-portlet -v 7.0 -p com.liferay.npm.jquery -c MyNpmjQueryPortlet my-npm-jquery-portlet
\end{verbatim}

or

\begin{verbatim}
mvn archetype:generate \
    -DarchetypeGroupId=com.liferay \
    -DarchetypeArtifactId=com.liferay.project.templates.npm.jquery.portlet \
    -DgroupId=com.liferay \
    -DartifactId=my-npm-jquery-portlet \
    -Dpackage=com.liferay.npm.jquery \
    -Dversion=1.0 \
    -DclassName=MyNpmjQueryPortlet \
    -DpackageJsonVersion=1.0.0 \
    -DliferayVersion=7.0
\end{verbatim}

After running the command above, your project's directory structure
looks like this:

\begin{itemize}
\tightlist
\item
  \texttt{my-npm-jquery-portlet}

  \begin{itemize}
  \tightlist
  \item
    \texttt{{[}gradle\textbar{}.mvn{]}}

    \begin{itemize}
    \tightlist
    \item
      \texttt{wrapper}

      \begin{itemize}
      \tightlist
      \item
        \texttt{{[}gradle\textbar{}maven{]}-wrapper.jar}
      \item
        \texttt{{[}gradle\textbar{}maven{]}-wrapper.properties}
      \end{itemize}
    \end{itemize}
  \item
    \texttt{src}

    \begin{itemize}
    \tightlist
    \item
      \texttt{main}

      \begin{itemize}
      \tightlist
      \item
        \texttt{java}

        \begin{itemize}
        \tightlist
        \item
          \texttt{com/liferay/npm/jquery}

          \begin{itemize}
          \tightlist
          \item
            \texttt{constants}

            \begin{itemize}
            \tightlist
            \item
              \texttt{MyNpmjQueryPortletKeys.java}
            \end{itemize}
          \item
            \texttt{portlet}

            \begin{itemize}
            \tightlist
            \item
              \texttt{MyNpmjQueryPortlet.java}
            \end{itemize}
          \end{itemize}
        \end{itemize}
      \item
        \texttt{resources}

        \begin{itemize}
        \tightlist
        \item
          \texttt{content}

          \begin{itemize}
          \tightlist
          \item
            \texttt{Language.properties}
          \end{itemize}
        \item
          \texttt{META-INF}

          \begin{itemize}
          \tightlist
          \item
            \texttt{resources}

            \begin{itemize}
            \tightlist
            \item
              \texttt{js}

              \begin{itemize}
              \tightlist
              \item
                \texttt{index.es.js}
              \end{itemize}
            \item
              \texttt{init.jsp}
            \item
              \texttt{view.jsp}
            \end{itemize}
          \end{itemize}
        \end{itemize}
      \end{itemize}
    \end{itemize}
  \item
    \texttt{.babelrc}
  \item
    \texttt{.npmbundlerrc}
  \item
    \texttt{bnd.bnd}
  \item
    \texttt{{[}build.gradle\textbar{}pom.xml{]}}
  \item
    \texttt{{[}gradlew\textbar{}mvnw{]}}
  \item
    \texttt{package.json}
  \end{itemize}
\end{itemize}

The generated module is a working application and is deployable to a
Liferay DXP instance. To build upon the generated portlet, modify the
project by adding logic and additional files to the folders outlined
above.

\section{npm Metal.js Portlet
Template}\label{npm-metal.js-portlet-template}

In this article, you'll learn how to create an npm Metal.js portlet as a
Liferay module. To create an npm Metal.js portlet via the command line
using Blade CLI or Maven, use one of the commands with the following
parameters:

\begin{verbatim}
blade create -t npm-metaljs-portlet -v 7.0 [-p packageName] [-c className] projectName
\end{verbatim}

or

\begin{verbatim}
mvn archetype:generate \
    -DarchetypeGroupId=com.liferay \
    -DarchetypeArtifactId=com.liferay.project.templates.npm.metaljs.portlet \
    -DartifactId=[projectName] \
    -Dpackage=[packageName] \
    -DclassName=[className] \
    -DliferayVersion=7.0
\end{verbatim}

You can also insert the \texttt{-b\ maven} parameter in the Blade
command to generate a Maven project using Blade CLI.

\noindent\hrulefill

\textbf{Note:} The minifier fails on Liferay DXP 7.0 when JSDoc is
present in a portlet. To resolve this, use
\href{https://gruntjs.com/getting-started}{Grunt}
\href{https://www.npmjs.com/package/grunt-contrib-uglify}{uglify} to
remove the JSDoc comments. This process may take a long time, depending
on the number of files that require an update.

\noindent\hrulefill

The template for this kind of project is \texttt{npm-metaljs-portlet}.
Suppose you want to create an npm Metal.js portlet project called
\texttt{my-npm-metaljs-portlet} with a package name of
\texttt{com.liferay.npm.metaljs} and a class name of
\texttt{MyNpmMetaljsPortlet}. Also, you'd like to create a service of
type \texttt{javax.portlet.Portlet} that extends the
\texttt{com.liferay.portal.kernel.portlet.bridges.mvc.MVCPortlet} class.
Here, \emph{service} means an OSGi service, not a Liferay API. Another
way to say \emph{service type} is to say \emph{component type}. You
could run the following command to accomplish this:

\begin{verbatim}
blade create -t npm-metaljs-portlet -v 7.0 -p com.liferay.npm.metaljs -c MyNpmMetaljsPortlet my-npm-metaljs-portlet
\end{verbatim}

or

\begin{verbatim}
mvn archetype:generate \
    -DarchetypeGroupId=com.liferay \
    -DarchetypeArtifactId=com.liferay.project.templates.npm.metaljs.portlet \
    -DgroupId=com.liferay \
    -DartifactId=my-npm-metaljs-portlet \
    -Dpackage=com.liferay.npm.metaljs \
    -Dversion=1.0 \
    -DclassName=MyNpmMetaljsPortlet \
    -DpackageJsonVersion=1.0.0 \
    -DliferayVersion=7.0
\end{verbatim}

After running the command above, your project's directory structure
looks like this:

\begin{itemize}
\tightlist
\item
  \texttt{my-npm-metaljs-portlet}

  \begin{itemize}
  \tightlist
  \item
    \texttt{{[}gradle\textbar{}.mvn{]}}

    \begin{itemize}
    \tightlist
    \item
      \texttt{wrapper}

      \begin{itemize}
      \tightlist
      \item
        \texttt{{[}gradle\textbar{}maven{]}-wrapper.jar}
      \item
        \texttt{{[}gradle\textbar{}maven{]}-wrapper.properties}
      \end{itemize}
    \end{itemize}
  \item
    \texttt{src}

    \begin{itemize}
    \tightlist
    \item
      \texttt{main}

      \begin{itemize}
      \tightlist
      \item
        \texttt{java}

        \begin{itemize}
        \tightlist
        \item
          \texttt{com/liferay/npm/metaljs}

          \begin{itemize}
          \tightlist
          \item
            \texttt{constants}

            \begin{itemize}
            \tightlist
            \item
              \texttt{MyNpmMetaljsPortletKeys.java}
            \end{itemize}
          \item
            \texttt{portlet}

            \begin{itemize}
            \tightlist
            \item
              \texttt{MyNpmMetaljsPortlet.java}
            \end{itemize}
          \end{itemize}
        \end{itemize}
      \item
        \texttt{resources}

        \begin{itemize}
        \tightlist
        \item
          \texttt{content}

          \begin{itemize}
          \tightlist
          \item
            \texttt{Language.properties}
          \end{itemize}
        \item
          \texttt{META-INF}

          \begin{itemize}
          \tightlist
          \item
            \texttt{resources}

            \begin{itemize}
            \tightlist
            \item
              \texttt{js}

              \begin{itemize}
              \tightlist
              \item
                \texttt{HelloWorld.soy}
              \item
                \texttt{index.es.js}
              \end{itemize}
            \item
              \texttt{init.jsp}
            \item
              \texttt{view.jsp}
            \end{itemize}
          \end{itemize}
        \end{itemize}
      \end{itemize}
    \end{itemize}
  \item
    \texttt{.babelrc}
  \item
    \texttt{.npmbundlerrc}
  \item
    \texttt{bnd.bnd}
  \item
    \texttt{{[}build.gradle\textbar{}pom.xml{]}}
  \item
    \texttt{{[}gradlew\textbar{}mvnw{]}}
  \item
    \texttt{package.json}
  \end{itemize}
\end{itemize}

The generated module is a working application and is deployable to a
Liferay DXP instance. To build upon the generated portlet, modify the
project by adding logic and additional files to the folders outlined
above.

\section{npm Portlet Template}\label{npm-portlet-template}

In this article, you'll learn how to create an npm portlet as a Liferay
module. To create an npmportlet via the command line using Blade CLI or
Maven, use one of the commands with the following parameters:

\begin{verbatim}
blade create -t npm-portlet -v 7.0 [-p packageName] [-c className] projectName
\end{verbatim}

or

\begin{verbatim}
mvn archetype:generate \
    -DarchetypeGroupId=com.liferay \
    -DarchetypeArtifactId=com.liferay.project.templates.npm.portlet \
    -DartifactId=[projectName] \
    -Dpackage=[packageName] \
    -DclassName=[className] \
    -DliferayVersion=7.0
\end{verbatim}

You can also insert the \texttt{-b\ maven} parameter in the Blade
command to generate a Maven project using Blade CLI.

\noindent\hrulefill

\textbf{Note:} The minifier fails on Liferay DXP 7.0 when JSDoc is
present in a portlet. To resolve this, use
\href{https://gruntjs.com/getting-started}{Grunt}
\href{https://www.npmjs.com/package/grunt-contrib-uglify}{uglify} to
remove the JSDoc comments. This process may take a long time, depending
on the number of files that require an update.

\noindent\hrulefill

The template for this kind of project is \texttt{npm-portlet}. Suppose
you want to create an npm portlet project called \texttt{my-npm-portlet}
with a package name of \texttt{com.liferay.npm} and a class name of
\texttt{MyNpmPortlet}. Also, you'd like to create a service of type
\texttt{javax.portlet.Portlet} that extends the
\texttt{com.liferay.portal.kernel.portlet.bridges.mvc.MVCPortlet} class.
Here, \emph{service} means an OSGi service, not a Liferay API. Another
way to say \emph{service type} is to say \emph{component type}. You
could run the following command to accomplish this:

\begin{verbatim}
blade create -t npm-portlet -v 7.0 -p com.liferay.npm -c MyNpmPortlet my-npm-portlet
\end{verbatim}

or

\begin{verbatim}
mvn archetype:generate \
    -DarchetypeGroupId=com.liferay \
    -DarchetypeArtifactId=com.liferay.project.templates.npm.portlet \
    -DgroupId=com.liferay \
    -DartifactId=my-npm-portlet \
    -Dpackage=com.liferay.npm \
    -Dversion=1.0 \
    -DclassName=MyNpmPortlet \
    -DpackageJsonVersion=1.0.0 \
    -DliferayVersion=7.0
\end{verbatim}

After running the command above, your project's directory structure
looks like this:

\begin{itemize}
\tightlist
\item
  \texttt{my-npm-portlet}

  \begin{itemize}
  \tightlist
  \item
    \texttt{{[}gradle\textbar{}.mvn{]}}

    \begin{itemize}
    \tightlist
    \item
      \texttt{wrapper}

      \begin{itemize}
      \tightlist
      \item
        \texttt{{[}gradle\textbar{}maven{]}-wrapper.jar}
      \item
        \texttt{{[}gradle\textbar{}maven{]}-wrapper.properties}
      \end{itemize}
    \end{itemize}
  \item
    \texttt{src}

    \begin{itemize}
    \tightlist
    \item
      \texttt{main}

      \begin{itemize}
      \tightlist
      \item
        \texttt{java}

        \begin{itemize}
        \tightlist
        \item
          \texttt{com/liferay/npm}

          \begin{itemize}
          \tightlist
          \item
            \texttt{constants}

            \begin{itemize}
            \tightlist
            \item
              \texttt{MyNpmPortletKeys.java}
            \end{itemize}
          \item
            \texttt{portlet}

            \begin{itemize}
            \tightlist
            \item
              \texttt{MyNpmPortlet.java}
            \end{itemize}
          \end{itemize}
        \end{itemize}
      \item
        \texttt{resources}

        \begin{itemize}
        \tightlist
        \item
          \texttt{content}

          \begin{itemize}
          \tightlist
          \item
            \texttt{Language.properties}
          \end{itemize}
        \item
          \texttt{META-INF}

          \begin{itemize}
          \tightlist
          \item
            \texttt{resources}

            \begin{itemize}
            \tightlist
            \item
              \texttt{js}

              \begin{itemize}
              \tightlist
              \item
                \texttt{index.es.js}
              \end{itemize}
            \item
              \texttt{init.jsp}
            \item
              \texttt{view.jsp}
            \end{itemize}
          \end{itemize}
        \end{itemize}
      \end{itemize}
    \end{itemize}
  \item
    \texttt{.babelrc}
  \item
    \texttt{.npmbundlerrc}
  \item
    \texttt{bnd.bnd}
  \item
    \texttt{{[}build.gradle\textbar{}pom.xml{]}}
  \item
    \texttt{{[}gradlew\textbar{}mvnw{]}}
  \item
    \texttt{package.json}
  \end{itemize}
\end{itemize}

The generated module is a working application and is deployable to a
Liferay DXP instance. To build upon the generated portlet, modify the
project by adding logic and additional files to the folders outlined
above.

\section{npm React Portlet Template}\label{npm-react-portlet-template}

In this article, you'll learn how to create an npm React portlet as a
Liferay module. To create an npm React portlet via the command line
using Blade CLI or Maven, use one of the commands with the following
parameters:

\begin{verbatim}
blade create -t npm-react-portlet -v 7.0 [-p packageName] [-c className] projectName
\end{verbatim}

or

\begin{verbatim}
mvn archetype:generate \
    -DarchetypeGroupId=com.liferay \
    -DarchetypeArtifactId=com.liferay.project.templates.npm.react.portlet \
    -DartifactId=[projectName] \
    -Dpackage=[packageName] \
    -DclassName=[className] \
    -DliferayVersion=7.0
\end{verbatim}

You can also insert the \texttt{-b\ maven} parameter in the Blade
command to generate a Maven project using Blade CLI.

\noindent\hrulefill

\textbf{Note:} The minifier fails on Liferay DXP 7.0 when JSDoc is
present in a portlet. To resolve this, use
\href{https://gruntjs.com/getting-started}{Grunt}
\href{https://www.npmjs.com/package/grunt-contrib-uglify}{uglify} to
remove the JSDoc comments. This process may take a long time, depending
on the number of files that require an update.

\noindent\hrulefill

The template for this kind of project is \texttt{npm-react-portlet}.
Suppose you want to create an npm React portlet project called
\texttt{my-npm-react-portlet} with a package name of
\texttt{com.liferay.npm.react} and a class name of
\texttt{MyNpmReactPortlet}. Also, you'd like to create a service of type
\texttt{javax.portlet.Portlet} that extends the
\texttt{com.liferay.portal.kernel.portlet.bridges.mvc.MVCPortlet} class.
Here, \emph{service} means an OSGi service, not a Liferay API. Another
way to say \emph{service type} is to say \emph{component type}. You
could run the following command to accomplish this:

\begin{verbatim}
blade create -t npm-react-portlet -v 7.0 -p com.liferay.npm.react -c MyNpmReactPortlet my-npm-react-portlet
\end{verbatim}

or

\begin{verbatim}
mvn archetype:generate \
    -DarchetypeGroupId=com.liferay \
    -DarchetypeArtifactId=com.liferay.project.templates.npm.react.portlet \
    -DgroupId=com.liferay \
    -DartifactId=my-npm-react-portlet \
    -Dpackage=com.liferay.npm.react \
    -Dversion=1.0 \
    -DclassName=MyNpmReactPortlet \
    -DpackageJsonVersion=1.0.0 \
    -DliferayVersion=7.0
\end{verbatim}

After running the command above, your project's directory structure
looks like this:

\begin{itemize}
\tightlist
\item
  \texttt{my-npm-react-portlet}

  \begin{itemize}
  \tightlist
  \item
    \texttt{{[}gradle\textbar{}.mvn{]}}

    \begin{itemize}
    \tightlist
    \item
      \texttt{wrapper}

      \begin{itemize}
      \tightlist
      \item
        \texttt{{[}gradle\textbar{}maven{]}-wrapper.jar}
      \item
        \texttt{{[}gradle\textbar{}maven{]}-wrapper.properties}
      \end{itemize}
    \end{itemize}
  \item
    \texttt{src}

    \begin{itemize}
    \tightlist
    \item
      \texttt{main}

      \begin{itemize}
      \tightlist
      \item
        \texttt{java}

        \begin{itemize}
        \tightlist
        \item
          \texttt{com/liferay/npm/react}

          \begin{itemize}
          \tightlist
          \item
            \texttt{constants}

            \begin{itemize}
            \tightlist
            \item
              \texttt{MyNpmReactPortletKeys.java}
            \end{itemize}
          \item
            \texttt{portlet}

            \begin{itemize}
            \tightlist
            \item
              \texttt{MyNpmReactPortlet.java}
            \end{itemize}
          \end{itemize}
        \end{itemize}
      \item
        \texttt{resources}

        \begin{itemize}
        \tightlist
        \item
          \texttt{content}

          \begin{itemize}
          \tightlist
          \item
            \texttt{Language.properties}
          \end{itemize}
        \item
          \texttt{META-INF}

          \begin{itemize}
          \tightlist
          \item
            \texttt{resources}

            \begin{itemize}
            \tightlist
            \item
              \texttt{js}

              \begin{itemize}
              \tightlist
              \item
                \texttt{index.es.js}
              \end{itemize}
            \item
              \texttt{init.jsp}
            \item
              \texttt{view.jsp}
            \end{itemize}
          \end{itemize}
        \end{itemize}
      \end{itemize}
    \end{itemize}
  \item
    \texttt{.babelrc}
  \item
    \texttt{.npmbundlerrc}
  \item
    \texttt{bnd.bnd}
  \item
    \texttt{{[}build.gradle\textbar{}pom.xml{]}}
  \item
    \texttt{{[}gradlew\textbar{}mvnw{]}}
  \item
    \texttt{package.json}
  \end{itemize}
\end{itemize}

The generated module is a working application and is deployable to a
Liferay DXP instance. To build upon the generated portlet, modify the
project by adding logic and additional files to the folders outlined
above.

\section{npm Vue.js Portlet Template}\label{npm-vue.js-portlet-template}

In this article, you'll learn how to create an npm Vue.js portlet as a
Liferay module. To create an npm Vue.js portlet via the command line
using Blade CLI or Maven, use one of the commands with the following
parameters:

\begin{verbatim}
blade create -t npm-vuejs-portlet -v 7.0 [-p packageName] [-c className] projectName
\end{verbatim}

or

\begin{verbatim}
mvn archetype:generate \
    -DarchetypeGroupId=com.liferay \
    -DarchetypeArtifactId=com.liferay.project.templates.npm.vuejs.portlet \
    -DartifactId=[projectName] \
    -Dpackage=[packageName] \
    -DclassName=[className] \
    -DliferayVersion=7.0
\end{verbatim}

You can also insert the \texttt{-b\ maven} parameter in the Blade
command to generate a Maven project using Blade CLI.

\noindent\hrulefill

\textbf{Note:} The minifier fails on Liferay DXP 7.0 when JSDoc is
present in a portlet. To resolve this, use
\href{https://gruntjs.com/getting-started}{Grunt}
\href{https://www.npmjs.com/package/grunt-contrib-uglify}{uglify} to
remove the JSDoc comments. This process may take a long time, depending
on the number of files that require an update.

\noindent\hrulefill

The template for this kind of project is \texttt{npm-vuejs-portlet}.
Suppose you want to create an npm Vue.js portlet project called
\texttt{my-npm-vuejs-portlet} with a package name of
\texttt{com.liferay.npm.vuejs} and a class name of
\texttt{MyNpmVuejsPortlet}. Also, you'd like to create a service of type
\texttt{javax.portlet.Portlet} that extends the
\texttt{com.liferay.portal.kernel.portlet.bridges.mvc.MVCPortlet} class.
Here, \emph{service} means an OSGi service, not a Liferay API. Another
way to say \emph{service type} is to say \emph{component type}. You
could run the following command to accomplish this:

\begin{verbatim}
blade create -t npm-vuejs-portlet -v 7.0 -p com.liferay.npm.vuejs -c MyNpmVuejsPortlet my-npm-vuejs-portlet
\end{verbatim}

or

\begin{verbatim}
mvn archetype:generate \
    -DarchetypeGroupId=com.liferay \
    -DarchetypeArtifactId=com.liferay.project.templates.npm.vuejs.portlet \
    -DgroupId=com.liferay \
    -DartifactId=my-npm-vuejs-portlet \
    -Dpackage=com.liferay.npm.vuejs \
    -Dversion=1.0 \
    -DclassName=MyNpmVuejsPortlet \
    -DpackageJsonVersion=1.0.0 \
    -DliferayVersion=7.0
\end{verbatim}

\textbf{Important:} This sample works for Liferay DXP 7.0 Fix Pack 44+
and Liferay Portal CE GA7+.

After running the command above, your project's directory structure
looks like this:

\begin{itemize}
\tightlist
\item
  \texttt{my-npm-vuejs-portlet}

  \begin{itemize}
  \tightlist
  \item
    \texttt{.mvn} (only in Maven Blade CLI generated projects)

    \begin{itemize}
    \tightlist
    \item
      \texttt{wrapper}

      \begin{itemize}
      \tightlist
      \item
        \texttt{maven-wrapper.jar}
      \item
        \texttt{maven-wrapper.properties}
      \end{itemize}
    \end{itemize}
  \item
    \texttt{src}

    \begin{itemize}
    \tightlist
    \item
      \texttt{main}

      \begin{itemize}
      \tightlist
      \item
        \texttt{java}

        \begin{itemize}
        \tightlist
        \item
          \texttt{com/liferay/npm/vuejs}

          \begin{itemize}
          \tightlist
          \item
            \texttt{constants}

            \begin{itemize}
            \tightlist
            \item
              \texttt{MyNpmVuejsPortletKeys.java}
            \end{itemize}
          \item
            \texttt{portlet}

            \begin{itemize}
            \tightlist
            \item
              \texttt{MyNpmVuejsPortlet.java}
            \end{itemize}
          \end{itemize}
        \end{itemize}
      \item
        \texttt{resources}

        \begin{itemize}
        \tightlist
        \item
          \texttt{content}

          \begin{itemize}
          \tightlist
          \item
            \texttt{Language.properties}
          \end{itemize}
        \item
          \texttt{META-INF}

          \begin{itemize}
          \tightlist
          \item
            \texttt{resources}

            \begin{itemize}
            \tightlist
            \item
              \texttt{lib}

              \begin{itemize}
              \tightlist
              \item
                \texttt{index.es.js}
              \end{itemize}
            \item
              \texttt{init.jsp}
            \item
              \texttt{view.jsp}
            \end{itemize}
          \end{itemize}
        \end{itemize}
      \end{itemize}
    \end{itemize}
  \item
    \texttt{.babelrc}
  \item
    \texttt{.npmbundlerrc}
  \item
    \texttt{bnd.bnd}
  \item
    \texttt{{[}build.gradle\textbar{}pom.xml{]}}
  \item
    \texttt{mvnw} (only in Maven Blade CLI generated projects)
  \item
    \texttt{mvnw.cmd} (only in Maven Blade CLI generated projects)
  \item
    \texttt{package.json}
  \end{itemize}
\end{itemize}

The generated module is a working application and is deployable to a
Liferay DXP instance. To build upon the generated portlet, modify the
project by adding logic and additional files to the folders outlined
above.

\section{Panel App Template}\label{panel-app-template}

In this article, you'll learn how to create a Liferay panel app and
category as a Liferay module. To create a Liferay panel app and category
via the command line using Blade CLI or Maven, use one of the commands
with the following parameters:

\begin{verbatim}
blade create -t panel-app -v 7.0 [-p packageName] [-c className] projectName
\end{verbatim}

or

\begin{verbatim}
mvn archetype:generate \
    -DarchetypeGroupId=com.liferay \
    -DarchetypeArtifactId=com.liferay.project.templates.panel.app \
    -DartifactId=[projectName] \
    -Dpackage=[packageName] \
    -DclassName=[className] \
    -DliferayVersion=7.0
\end{verbatim}

You can also insert the \texttt{-b\ maven} parameter in the Blade
command to generate a Maven project using Blade CLI.

The template for this kind of project is \texttt{panel-app}. Suppose you
want to create a panel app project called \texttt{my-panel-app-project}
with a package name prefix of \texttt{com.liferay.docs} and a class name
prefix of \texttt{Sample}. You could run the following command to
accomplish this:

\begin{verbatim}
blade create -t panel-app -v 7.0 -p com.liferay.docs -c Sample my-panel-app-project
\end{verbatim}

or

\begin{verbatim}
mvn archetype:generate \
    -DarchetypeGroupId=com.liferay \
    -DarchetypeArtifactId=com.liferay.project.templates.panel.app \
    -DgroupId=com.liferay \
    -DartifactId=my-panel-app-project \
    -Dpackage=com.liferay.docs \
    -Dversion=1.0 \
    -DclassName=Sample \
    -Dauthor=Joe Bloggs \
    -DliferayVersion=7.0
\end{verbatim}

After running the command above, your project's directory structure
would look like this

\begin{itemize}
\tightlist
\item
  \texttt{my-panel-app-project}

  \begin{itemize}
  \tightlist
  \item
    \texttt{gradle} (only in Blade CLI generated projects)

    \begin{itemize}
    \tightlist
    \item
      \texttt{wrapper}

      \begin{itemize}
      \tightlist
      \item
        \texttt{gradle-wrapper.jar}
      \item
        \texttt{gradle-wrapper.properties}
      \end{itemize}
    \end{itemize}
  \item
    \texttt{src}

    \begin{itemize}
    \tightlist
    \item
      \texttt{main}

      \begin{itemize}
      \tightlist
      \item
        \texttt{java}

        \begin{itemize}
        \tightlist
        \item
          \texttt{com/liferay/docs/}

          \begin{itemize}
          \tightlist
          \item
            \texttt{application/list}

            \begin{itemize}
            \tightlist
            \item
              \texttt{SamplePanelApp.java}
            \item
              \texttt{SamplePanelCategory.java}
            \end{itemize}
          \item
            \texttt{constants}

            \begin{itemize}
            \tightlist
            \item
              \texttt{SamplePanelCategoryKeys.java}
            \item
              \texttt{SamplePortletKeys.java}
            \end{itemize}
          \item
            \texttt{portlet}

            \begin{itemize}
            \tightlist
            \item
              \texttt{SamplePortlet.java}
            \end{itemize}
          \end{itemize}
        \end{itemize}
      \item
        \texttt{resources}

        \begin{itemize}
        \tightlist
        \item
          \texttt{content}

          \begin{itemize}
          \tightlist
          \item
            \texttt{Language.properties}
          \end{itemize}
        \item
          \texttt{META-INF}

          \begin{itemize}
          \tightlist
          \item
            \texttt{resources}

            \begin{itemize}
            \tightlist
            \item
              \texttt{init.jsp}
            \item
              \texttt{view.jsp}
            \end{itemize}
          \end{itemize}
        \end{itemize}
      \end{itemize}
    \end{itemize}
  \item
    \texttt{bnd.bnd}
  \item
    \texttt{build.gradle}
  \item
    \texttt{{[}gradlew\textbar{}pom.xml{]}}
  \end{itemize}
\end{itemize}

The generated module is functional and is deployable to a Liferay DXP
instance. The generated module, by default, creates a panel category
with a panel app in Liferay DXP's Product Menu. To build upon the
generated app, modify the project by adding logic and additional files
to the folders outlined above. You can visit the
\href{/docs/7-0/tutorials/-/knowledge_base/t/customizing-the-product-menu}{Customizing
the Product Menu} tutorial for instructions on customizing a panel app
project.

\section{Portlet Configuration Icon}\label{portlet-configuration-icon}

In this article, you'll learn how to create a Liferay portlet
configuration icon as a Liferay module. To create a portlet
configuration icon via the command line using Blade CLI or Maven, use
one of the commands with the following parameters:

\begin{verbatim}
blade create -t portlet-configuration-icon -v 7.0 [-p packageName] [-c className] projectName
\end{verbatim}

or

\begin{verbatim}
mvn archetype:generate \
    -DarchetypeGroupId=com.liferay \
    -DarchetypeArtifactId=com.liferay.project.templates.portlet.configuration.icon \
    -DartifactId=[projectName] \
    -Dpackage=[packageName] \
    -DclassName=[className] \
    -DliferayVersion=7.0
\end{verbatim}

You can also insert the \texttt{-b\ maven} parameter in the Blade
command to generate a Maven project using Blade CLI.

The template for this kind of project is
\texttt{portlet-configuration-icon}. Suppose you want to create a
portlet configuration icon project called
\texttt{my-portlet-config-icon} with a package name of
\texttt{com.liferay.docs.portlet.configuration.icon} and a class name of
\texttt{SamplePortletConfigurationIcon}. You could run the following
command to accomplish this:

\begin{verbatim}
blade create -t portlet-configuration-icon -v 7.0 -p com.liferay.docs -c Sample my-portlet-config-icon
\end{verbatim}

or

\begin{verbatim}
mvn archetype:generate \
    -DarchetypeGroupId=com.liferay \
    -DarchetypeArtifactId=com.liferay.project.templates.portlet.configuration.icon \
    -DgroupId=com.liferay \
    -DartifactId=my-portlet-config-project \
    -Dpackage=com.liferay.docs \
    -Dversion=1.0 \
    -DclassName=Sample \
    -Dauthor=Joe Bloggs \
    -DliferayVersion=7.0
\end{verbatim}

After running the command above, your project's directory structure
would look like this

\begin{itemize}
\tightlist
\item
  \texttt{my-portlet-config-icon}

  \begin{itemize}
  \tightlist
  \item
    \texttt{gradle} (only in Blade CLI generated projects)

    \begin{itemize}
    \tightlist
    \item
      \texttt{wrapper}

      \begin{itemize}
      \tightlist
      \item
        \texttt{gradle-wrapper.jar}
      \item
        \texttt{gradle-wrapper.properties}
      \end{itemize}
    \end{itemize}
  \item
    \texttt{src}

    \begin{itemize}
    \tightlist
    \item
      \texttt{main}

      \begin{itemize}
      \tightlist
      \item
        \texttt{java}

        \begin{itemize}
        \tightlist
        \item
          \texttt{com/liferay/docs/portlet/configuration/icon}

          \begin{itemize}
          \tightlist
          \item
            \texttt{SamplePortletConfigurationIcon.java}
          \end{itemize}
        \end{itemize}
      \item
        \texttt{resources}

        \begin{itemize}
        \tightlist
        \item
          \texttt{content}

          \begin{itemize}
          \tightlist
          \item
            \texttt{Language.properties}
          \end{itemize}
        \end{itemize}
      \end{itemize}
    \end{itemize}
  \item
    \texttt{bnd.bnd}
  \item
    \texttt{build.gradle}
  \item
    \texttt{{[}gradlew\textbar{}pom.xml{]}}
  \end{itemize}
\end{itemize}

The generated module is functional and is deployable to a Liferay DXP
instance. The generated module, by default, creates a sample link in the
Hello World portlet's Options menu. To build upon the generated app,
modify the project by adding logic and additional files to the folders
outlined above. You can visit the
\href{https://github.com/liferay/liferay-blade-samples/tree/7.0/gradle/extensions/portlet-configuration-icon}{portlet-configuration-icon}
sample project for a more expanded sample of a portlet configuration
icon.

\section{Portlet Template}\label{portlet-template}

In this article, you'll learn how to create a Liferay portlet
application as a Liferay module. To create a Liferay portlet application
via the command line using Blade CLI or Maven, use one of the commands
with the following parameters:

\begin{verbatim}
blade create -t portlet -v 7.0 [-p packageName] [-c className] projectName
\end{verbatim}

or

\begin{verbatim}
mvn archetype:generate \
    -DarchetypeGroupId=com.liferay \
    -DarchetypeArtifactId=com.liferay.project.templates.portlet \
    -DartifactId=[projectName] \
    -Dpackage=[packageName] \
    -DclassName=[className] \
    -DliferayVersion=7.0
\end{verbatim}

You can also insert the \texttt{-b\ maven} parameter in the Blade
command to generate a Maven project using Blade CLI.

The template for this kind of project is \texttt{portlet}. Suppose you
want to create a portlet project called \texttt{my-portlet-project} with
a package name of \texttt{com.liferay.docs.portlet} and a class name of
\texttt{MyPortlet}. Also, you'd like to create a service of type
\texttt{javax.portlet.Portlet} that extends the
\texttt{javax.portlet.GenericPortlet} class. Here, \emph{service} means
an OSGi service, not a Liferay API. Another way to say \emph{service
type} is to say \emph{component type}. You could run the following
command to accomplish this:

\begin{verbatim}
blade create -t portlet -v 7.0 -p com.liferay.docs.portlet -c MyPortlet my-portlet-project
\end{verbatim}

or

\begin{verbatim}
mvn archetype:generate \
    -DarchetypeGroupId=com.liferay \
    -DarchetypeArtifactId=com.liferay.project.templates.portlet \
    -DgroupId=com.liferay \
    -DartifactId=my-portlet-project \
    -Dpackage=com.liferay.docs.portlet \
    -Dversion=1.0 \
    -DclassName=MyPortlet \
    -Dauthor=Joe Bloggs \
    -DliferayVersion=7.0
\end{verbatim}

After running the command above, your project's directory structure
looks like this:

\begin{itemize}
\tightlist
\item
  \texttt{my-portlet-project}

  \begin{itemize}
  \tightlist
  \item
    \texttt{gradle} (only in Blade CLI generated projects)

    \begin{itemize}
    \tightlist
    \item
      \texttt{wrapper}

      \begin{itemize}
      \tightlist
      \item
        \texttt{gradle-wrapper.jar}
      \item
        \texttt{gradle-wrapper.properties}
      \end{itemize}
    \end{itemize}
  \item
    \texttt{src}

    \begin{itemize}
    \tightlist
    \item
      \texttt{main}

      \begin{itemize}
      \tightlist
      \item
        \texttt{java}

        \begin{itemize}
        \tightlist
        \item
          \texttt{com/liferay/docs/portlet}

          \begin{itemize}
          \tightlist
          \item
            \texttt{MyPortlet.java}
          \end{itemize}
        \end{itemize}
      \end{itemize}
    \end{itemize}
  \item
    \texttt{bnd.bnd}
  \item
    \texttt{build.gradle}
  \item
    \texttt{{[}gradlew\textbar{}pom.xml{]}}
  \end{itemize}
\end{itemize}

The generated module is a working application and is deployable to a
Liferay DXP instance. To build upon the generated app, modify the
project by adding logic and additional files to the folders outlined
above.

\section{Portlet Provider Template}\label{portlet-provider-template}

In this article, you'll learn how to create a Liferay portlet provider
as a Liferay module. To create a Liferay portlet provider via the
command line using Blade CLI or Maven, use one of the commands with the
following parameters:

\begin{verbatim}
blade create -t portlet-provider -v 7.0 [-p packageName] [-c className] projectName
\end{verbatim}

or

\begin{verbatim}
mvn archetype:generate \
    -DarchetypeGroupId=com.liferay \
    -DarchetypeArtifactId=com.liferay.project.templates.portlet.provider \
    -DartifactId=[projectName] \
    -Dpackage=[packageName] \
    -DclassName=[className] \
    -DliferayVersion=7.0
\end{verbatim}

You can also insert the \texttt{-b\ maven} parameter in the Blade
command to generate a Maven project using Blade CLI.

The template for this kind of project is \texttt{portlet-provider}.
Suppose you want to create a portlet provider project called
\texttt{my-portlet-provider-project} with a package name of
\texttt{com.liferay.docs.portlet} and a class name prefix of
\texttt{Sample}. You could run the following command to accomplish this:

\begin{verbatim}
blade create -t portlet-provider -v 7.0 -p com.liferay.docs -c Sample my-portlet-provider-project
\end{verbatim}

or

\begin{verbatim}
mvn archetype:generate \
    -DarchetypeGroupId=com.liferay \
    -DarchetypeArtifactId=com.liferay.project.templates.portlet.provider \
    -DgroupId=com.liferay \
    -DartifactId=my-portlet-provider-project \
    -Dpackage=com.liferay.docs \
    -Dversion=1.0 \
    -DclassName=Sample \
    -Dauthor=Joe Bloggs \
    -DliferayVersion=7.0
\end{verbatim}

After running the command above, your project's directory structure
would look like this

\begin{itemize}
\tightlist
\item
  \texttt{my-portlet-provider-project}

  \begin{itemize}
  \tightlist
  \item
    \texttt{gradle} (only in Blade CLI generated projects)

    \begin{itemize}
    \tightlist
    \item
      \texttt{wrapper}

      \begin{itemize}
      \tightlist
      \item
        \texttt{gradle-wrapper.jar}
      \item
        \texttt{gradle-wrapper.properties}
      \end{itemize}
    \end{itemize}
  \item
    \texttt{src}

    \begin{itemize}
    \tightlist
    \item
      \texttt{main}

      \begin{itemize}
      \tightlist
      \item
        \texttt{java}

        \begin{itemize}
        \tightlist
        \item
          \texttt{com/liferay/docs/portlet}

          \begin{itemize}
          \tightlist
          \item
            \texttt{SampleAddPortletProvider.java}
          \item
            \texttt{SamplePortlet.java}
          \end{itemize}
        \end{itemize}
      \item
        \texttt{resources}

        \begin{itemize}
        \tightlist
        \item
          \texttt{META-INF}

          \begin{itemize}
          \tightlist
          \item
            \texttt{resources}

            \begin{itemize}
            \tightlist
            \item
              \texttt{init.jsp}
            \item
              \texttt{view.jsp}
            \end{itemize}
          \end{itemize}
        \end{itemize}
      \end{itemize}
    \end{itemize}
  \item
    \texttt{bnd.bnd}
  \item
    \texttt{build.gradle}
  \item
    \texttt{{[}gradlew\textbar{}pom.xml{]}}
  \end{itemize}
\end{itemize}

The generated module is functional and is deployable to a Liferay DXP
instance. To build upon the generated app, modify the project by adding
logic and additional files to the folders outlined above. You can visit
the
\href{/docs/7-0/tutorials/-/knowledge_base/t/providing-portlets-to-manage-requests}{Providing
Portlets to Manage Requests} tutorial for instructions on customizing a
portlet provider project.

\section{Portlet Toolbar Contributor
Template}\label{portlet-toolbar-contributor-template}

In this article, you'll learn how to create a Liferay portlet toolbar
contributor as a Liferay module. To create a portlet toolbar contributor
entry via the command line using Blade CLI or Maven, use one of the
commands with the following parameters:

\begin{verbatim}
blade create -t portlet-toolbar-contributor -v 7.0 [-p packageName] [-c className] projectName
\end{verbatim}

or

\begin{verbatim}
mvn archetype:generate \
    -DarchetypeGroupId=com.liferay \
    -DarchetypeArtifactId=com.liferay.project.templates.portlet.toolbar.contributor \
    -DartifactId=[projectName] \
    -Dpackage=[packageName] \
    -DclassName=[className] \
    -DliferayVersion=7.0
\end{verbatim}

You can also insert the \texttt{-b\ maven} parameter in the Blade
command to generate a Maven project using Blade CLI.

The template for this kind of project is
\texttt{portlet-toolbar-contributor}. Suppose you want to create a
portlet toolbar contributor project called
\texttt{my-portlet-toolbar-contributor} with a package name of
\texttt{com.liferay.docs.portlet.toolbar.contributor} and a class name
of \texttt{SamplePortletToolbarContributor}. You could run the following
command to accomplish this:

\begin{verbatim}
blade create -t portlet-toolbar-contributor -v 7.0 -p com.liferay.docs -c Sample my-portlet-toolbar-contributor
\end{verbatim}

or

\begin{verbatim}
mvn archetype:generate \
    -DarchetypeGroupId=com.liferay \
    -DarchetypeArtifactId=com.liferay.project.templates.portlet.toolbar.contributor \
    -DgroupId=com.liferay \
    -DartifactId=my-portlet-toolbar-contributor \
    -Dpackage=com.liferay.docs \
    -Dversion=1.0 \
    -DclassName=Sample \
    -Dauthor=Joe Bloggs \
    -DliferayVersion=7.0
\end{verbatim}

After running the command above, your project's directory structure
would look like this

\begin{itemize}
\tightlist
\item
  \texttt{my-portlet-toolbar-contributor}

  \begin{itemize}
  \tightlist
  \item
    \texttt{gradle} (only in Blade CLI generated projects)

    \begin{itemize}
    \tightlist
    \item
      \texttt{wrapper}

      \begin{itemize}
      \tightlist
      \item
        \texttt{gradle-wrapper.jar}
      \item
        \texttt{gradle-wrapper.properties}
      \end{itemize}
    \end{itemize}
  \item
    \texttt{src}

    \begin{itemize}
    \tightlist
    \item
      \texttt{main}

      \begin{itemize}
      \tightlist
      \item
        \texttt{java}

        \begin{itemize}
        \tightlist
        \item
          \texttt{com/liferay/docs/portlet/toolbar/contributor}

          \begin{itemize}
          \tightlist
          \item
            \texttt{SamplePortletToolbarContributor.java}
          \end{itemize}
        \end{itemize}
      \item
        \texttt{resources}

        \begin{itemize}
        \tightlist
        \item
          \texttt{content}

          \begin{itemize}
          \tightlist
          \item
            \texttt{Language.properties}
          \end{itemize}
        \end{itemize}
      \end{itemize}
    \end{itemize}
  \item
    \texttt{bnd.bnd}
  \item
    \texttt{build.gradle}
  \item
    \texttt{{[}gradlew\textbar{}pom.xml{]}}
  \end{itemize}
\end{itemize}

The generated module is functional and is deployable to a Liferay DXP
instance. To build upon the generated app, modify the project by adding
logic and additional files to the folders outlined above. This generated
project, by default, creates a new button on the Hello World portlet's
toolbar. You can visit the
\href{https://github.com/liferay/liferay-blade-samples/tree/7.0/gradle/extensions/portlet-toolbar-contributor}{portlet-toolbar-contributor}
sample project for a more expanded sample of a portlet toolbar
contributor.

\section{REST Template}\label{rest-template}

In this article, you'll learn how to create a Liferay RESTful web
service packaged in a Liferay module. To create a Liferay RESTful web
service via the command line using Blade CLI or Maven, use one of the
commands with the following parameters:

\begin{verbatim}
blade create -t rest -v 7.0 [-p packageName] [-c className] projectName
\end{verbatim}

or

\begin{verbatim}
mvn archetype:generate \
    -DarchetypeGroupId=com.liferay \
    -DarchetypeArtifactId=com.liferay.project.templates.rest \
    -DartifactId=[projectName] \
    -Dpackage=[packageName] \
    -DclassName=[className] \
    -DliferayVersion=7.0
\end{verbatim}

You can also insert the \texttt{-b\ maven} parameter in the Blade
command to generate a Maven project using Blade CLI.

The template for this kind of project is \texttt{rest}. Suppose you want
to create a RESTful web service project called \texttt{my-rest-project}
with a package name of \texttt{com.liferay.docs.application} and a class
name prefix of \texttt{Rest}. You could run one of the following
commands to accomplish this:

\begin{verbatim}
blade create -t rest -v 7.0 -p com.liferay.docs -c Rest my-rest-project
\end{verbatim}

or

\begin{verbatim}
mvn archetype:generate \
    -DarchetypeGroupId=com.liferay \
    -DarchetypeArtifactId=com.liferay.project.templates.rest \
    -DgroupId=com.liferay \
    -DartifactId=my-rest-project \
    -Dpackage=com.liferay.docs \
    -Dversion=1.0 \
    -DclassName=Rest \
    -Dauthor=Joe Bloggs \
    -DliferayVersion=7.0
\end{verbatim}

After running the command above, your project's directory structure
looks like this:

\begin{itemize}
\tightlist
\item
  \texttt{my-rest-project}

  \begin{itemize}
  \tightlist
  \item
    \texttt{gradle} (only in Blade CLI generated projects)

    \begin{itemize}
    \tightlist
    \item
      \texttt{wrapper}

      \begin{itemize}
      \tightlist
      \item
        \texttt{gradle-wrapper.jar}
      \item
        \texttt{gradle-wrapper.properties}
      \end{itemize}
    \end{itemize}
  \item
    \texttt{src}

    \begin{itemize}
    \tightlist
    \item
      \texttt{main}

      \begin{itemize}
      \tightlist
      \item
        \texttt{java}

        \begin{itemize}
        \tightlist
        \item
          \texttt{com/liferay/docs/application}

          \begin{itemize}
          \tightlist
          \item
            \texttt{RestApplication.java}
          \end{itemize}
        \end{itemize}
      \item
        \texttt{resources}

        \begin{itemize}
        \tightlist
        \item
          \texttt{configuration}

          \begin{itemize}
          \tightlist
          \item
            \texttt{com.liferay.portal.remote.cxf.common.configuration.CXFEndpointPublisherConfiguration-cxf}
          \item
            \texttt{com.liferay.portal.remote.rest.extender.configuration.RestExtenderConfiguration-rest}
          \end{itemize}
        \end{itemize}
      \end{itemize}
    \end{itemize}
  \item
    \texttt{bnd.bnd}
  \item
    \texttt{build.gradle}
  \item
    \texttt{{[}gradlew\textbar{}pom.xml{]}}
  \end{itemize}
\end{itemize}

The generated module is a working RESTful web service and is deployable
to a Liferay DXP instance. To build upon the generated app, modify the
project by adding logic and additional files to the folders outlined
above.

\section{Service Template}\label{service-template}

In this article, you'll learn how to create a Liferay service as a
Liferay module. To create a Liferay service via the command line using
Blade CLI or Maven, use one of the commands with the following
parameters:

\begin{verbatim}
blade create -t service -v 7.0 [-p packageName] [-c className] [-s serviceName] projectName
\end{verbatim}

or

\begin{verbatim}
mvn archetype:generate \
    -DarchetypeGroupId=com.liferay \
    -DarchetypeArtifactId=com.liferay.project.templates.service \
    -DartifactId=[projectName] \
    -Dpackage=[packageName] \
    -DclassName=[className] \
    -DserviceName=[serviceName] \
    -DliferayVersion=7.0
\end{verbatim}

You can also insert the \texttt{-b\ maven} parameter in the Blade
command to generate a Maven project using Blade CLI.

The template for this kind of project is \texttt{service}. Suppose you
want to create a service project called \texttt{my-service-project} with
a package name of \texttt{com.liferay.docs.service} and a class name of
\texttt{Service}. Also, you'd like to create a service of type
\texttt{com.liferay.portal.kernel.events.LifecycleAction} that also
implements that same service. You could run the following command to
accomplish this:

\begin{verbatim}
blade create -t service -v 7.0 -p com.liferay.docs.service -c Service -s com.liferay.portal.kernel.events.LifecycleAction  my-service-project
\end{verbatim}

or

\begin{verbatim}
mvn archetype:generate \
    -DarchetypeGroupId=com.liferay \
    -DarchetypeArtifactId=com.liferay.project.templates.service \
    -DgroupId=com.liferay \
    -DartifactId=my-service-project \
    -Dpackage=com.liferay.docs \
    -Dversion=1.0 \
    -DclassName=Service \
    -DclassName=com.liferay.portal.kernel.events.LifecycleAction \
    -Dauthor=Joe Bloggs \
    -DliferayVersion=7.0
\end{verbatim}

After running the command above, your project's directory structure
would look like this

\begin{itemize}
\tightlist
\item
  \texttt{my-service-project}

  \begin{itemize}
  \tightlist
  \item
    \texttt{gradle} (only in Blade CLI generated projects)

    \begin{itemize}
    \tightlist
    \item
      \texttt{wrapper}

      \begin{itemize}
      \tightlist
      \item
        \texttt{gradle-wrapper.jar}
      \item
        \texttt{gradle-wrapper.properties}
      \end{itemize}
    \end{itemize}
  \item
    \texttt{src}

    \begin{itemize}
    \tightlist
    \item
      \texttt{main}

      \begin{itemize}
      \tightlist
      \item
        \texttt{java}

        \begin{itemize}
        \tightlist
        \item
          \texttt{com/liferay/docs/service}

          \begin{itemize}
          \tightlist
          \item
            \texttt{Service.java}
          \end{itemize}
        \end{itemize}
      \end{itemize}
    \end{itemize}
  \item
    \texttt{bnd.bnd}
  \item
    \texttt{build.gradle}
  \item
    \texttt{{[}gradlew\textbar{}pom.xml{]}}
  \end{itemize}
\end{itemize}

The generated module is functional and is deployable to a Liferay DXP
instance. To build upon the generated app, modify the project by adding
logic and additional files to the folders outlined above.

\section{Service Builder Template}\label{service-builder-template}

In this article, you'll learn how to create a Liferay portlet
application that uses Service Builder as Liferay modules. To create a
Liferay Service Builder project via the command line using Blade CLI or
Maven, use one of the commands with the following parameters:

\begin{verbatim}
blade create -t service-builder -v 7.0 [-p packageName] projectName
\end{verbatim}

or

\begin{verbatim}
mvn archetype:generate \
    -DarchetypeGroupId=com.liferay \
    -DarchetypeArtifactId=com.liferay.project.templates.service.builder \
    -DartifactId=[projectName] \
    -Dpackage=[packageName] \
    -DapiPath=[apiPath] \
    -DliferayVersion=7.0
\end{verbatim}

You can also insert the \texttt{-b\ maven} parameter in the Blade
command to generate a Maven project using Blade CLI.

The template for this kind of project is \texttt{service-builder}.
Suppose you want to create a Service Builder project called
\texttt{tasks} with a package name of \texttt{com.liferay.docs.tasks}.
You could run the following command to accomplish this:

\begin{verbatim}
blade create -t service-builder -v 7.0 -p com.liferay.docs.tasks tasks
\end{verbatim}

or

\begin{verbatim}
mvn archetype:generate \
    -DarchetypeGroupId=com.liferay \
    -DarchetypeArtifactId=com.liferay.project.templates.service.builder \
    -DgroupId=com.liferay \
    -DartifactId=tasks \
    -Dpackage=com.liferay.docs.tasks \
    -Dversion=1.0 \
    -DapiPath=com.liferay.api.path \
    -DliferayVersion=7.0
\end{verbatim}

This task creates the \texttt{tasks-api} and \texttt{tasks-service}
folders. In many cases, a Service Builder project also requires a
\texttt{-web} folder to hold, for example, portlet classes. This should
be created manually. After running the command above, your project's
directory structure looks like this:

\begin{itemize}
\tightlist
\item
  \texttt{tasks}

  \begin{itemize}
  \tightlist
  \item
    \texttt{gradle} (only in Blade CLI generated projects)

    \begin{itemize}
    \tightlist
    \item
      \texttt{wrapper}

      \begin{itemize}
      \tightlist
      \item
        \texttt{gradle-wrapper.jar}
      \item
        \texttt{gradle-wrapper.properties}
      \end{itemize}
    \end{itemize}
  \item
    \texttt{tasks-api}

    \begin{itemize}
    \tightlist
    \item
      \texttt{bnd.bnd}
    \item
      \texttt{build.gradle}
    \end{itemize}
  \item
    \texttt{tasks-service}

    \begin{itemize}
    \tightlist
    \item
      \texttt{bnd.bnd}
    \item
      \texttt{build.gradle}
    \item
      \texttt{service.xml}
    \end{itemize}
  \item
    \texttt{build.gradle}
  \item
    \texttt{{[}gradlew\textbar{}pom.xml{]}}
  \item
    \texttt{settings.gradle}
  \end{itemize}
\end{itemize}

To generate your service and API classes for the \texttt{*-api} and
\texttt{*-service} modules, replace the \texttt{service.xml} file in the
\texttt{*-service} module. Depending on your build tool, you can build
your services by executing

\begin{verbatim}
blade gw buildService
\end{verbatim}

or

\begin{verbatim}
mvn service-builder:build
\end{verbatim}

from the \texttt{tasks} root directory. Note that \texttt{blade\ gw}
only works if the Gradle wrapper can be detected. To ensure the
availability of the Gradle wrapper, be sure to work in a Liferay
Workspace.

The \texttt{mvn\ service-builder:build} command only works if you're
using the \texttt{com.liferay.portal.tools.service.builder} plugin
version 1.0.145+. Maven projects using an earlier version of the Service
Builder plugin should update their POM accordingly.

The generated module is functional and is deployable to a Liferay DXP
instance. To build upon the generated app, modify the project by adding
logic and additional files to the folders outlined above.

\subsection{Related Topics}\label{related-topics-3}

\href{/docs/7-0/tutorials/-/knowledge_base/t/running-service-builder-and-understanding-the-generated-code}{Running
Service Builder and Understanding the Generated Code}

\href{/docs/7-0/tutorials/-/knowledge_base/t/using-service-builder-in-a-maven-project}{Using
Service Builder in a Maven Project}

\href{/docs/7-0/reference/-/knowledge_base/r/service-builder-with-maven}{Service
Builder with Maven}

\section{Service Wrapper Template}\label{service-wrapper-template}

In this article, you'll learn how to create a Liferay service wrapper as
a Liferay module. To create a Liferay service wrapper via the command
line using Blade CLI or Maven, use one of the commands with the
following parameters:

\begin{verbatim}
blade create -t service-wrapper -v 7.0 [-p packageName] [-c className] [-s serviceWrapperClass] projectName
\end{verbatim}

or

\begin{verbatim}
mvn archetype:generate \
    -DarchetypeGroupId=com.liferay \
    -DarchetypeArtifactId=com.liferay.project.templates.service.wrapper \
    -DartifactId=[projectName] \
    -Dpackage=[packageName] \
    -DclassName=[className] \
    -DserviceWrapperClass=[serviceWrapperClass] \
    -DliferayVersion=7.0
\end{verbatim}

You can also insert the \texttt{-b\ maven} parameter in the Blade
command to generate a Maven project using Blade CLI.

The template for this kind of project is \texttt{service-wrapper}.
Suppose you want to create a service wrapper project called
\texttt{service-override} with a package name of
\texttt{com.liferay.docs.serviceoverride} and a class name of
\texttt{UserLocalServiceOverride}. Also, you'd like to create a service
of type \texttt{com.liferay.portal.kernel.service.ServiceWrapper} that
extends the \texttt{com.liferay.portal.service.UserLocalServiceWrapper}
class. You could run the following command to accomplish this:

\begin{verbatim}
blade create -t service-wrapper -v 7.0 -p com.liferay.docs.serviceoverride -c UserLocalServiceOverride -s com.liferay.portal.kernel.service.UserLocalServiceWrapper service-override
\end{verbatim}

or

\begin{verbatim}
mvn archetype:generate \
    -DarchetypeGroupId=com.liferay \
    -DarchetypeArtifactId=com.liferay.project.templates.service.wrapper \
    -DgroupId=com.liferay \
    -DartifactId=service-override \
    -Dpackage=com.liferay.docs.serviceoverride \
    -Dversion=1.0 \
    -DclassName=UserLocalServiceOverride \
    -DserviceWrapperClass=com.liferay.portal.kernel.service.UserLocalServiceWrapper \
    -Dauthor=Joe Bloggs \
    -DliferayVersion=7.0
\end{verbatim}

Here, \emph{service} means an OSGi service, not a Liferay API. Another
way to say \emph{service type} is to say \emph{component type}.

After running the command above, your project's directory structure
looks like this:

\begin{itemize}
\tightlist
\item
  \texttt{service-override}

  \begin{itemize}
  \tightlist
  \item
    \texttt{gradle} (only in Blade CLI generated projects)

    \begin{itemize}
    \tightlist
    \item
      \texttt{wrapper}

      \begin{itemize}
      \tightlist
      \item
        \texttt{gradle-wrapper.jar}
      \item
        \texttt{gradle-wrapper.properties}
      \end{itemize}
    \end{itemize}
  \item
    \texttt{src}

    \begin{itemize}
    \tightlist
    \item
      \texttt{main}

      \begin{itemize}
      \tightlist
      \item
        \texttt{java}

        \begin{itemize}
        \tightlist
        \item
          \texttt{com/liferay/docs/serviceoverride}

          \begin{itemize}
          \tightlist
          \item
            \texttt{UserLocalServiceOverride.java}
          \end{itemize}
        \end{itemize}
      \end{itemize}
    \end{itemize}
  \item
    \texttt{bnd.bnd}
  \item
    \texttt{build.gradle}
  \item
    \texttt{{[}gradlew\textbar{}pom.xml{]}}
  \end{itemize}
\end{itemize}

The generated module is a working application and is deployable to a
Liferay DXP instance. To build upon the generated app, modify the
project by adding logic and additional files to the folders outlined
above.

\section{Simulation Panel Entry
Template}\label{simulation-panel-entry-template}

In this article, you'll learn how to create a Liferay simulation panel
entry as a Liferay module. To create a simulation panel entry via the
command line using Blade CLI or Maven, use one of the commands with the
following parameters:

\begin{verbatim}
blade create -t simulation-panel-entry -v 7.0 [-p packageName] [-c className] projectName
\end{verbatim}

or

\begin{verbatim}
mvn archetype:generate \
    -DarchetypeGroupId=com.liferay \
    -DarchetypeArtifactId=com.liferay.project.templates.simulation.panel.entry \
    -DartifactId=[projectName] \
    -Dpackage=[packageName] \
    -DclassName=[className] \
    -DliferayVersion=7.0
\end{verbatim}

You can also insert the \texttt{-b\ maven} parameter in the Blade
command to generate a Maven project using Blade CLI.

The template for this kind of project is
\texttt{simulation-panel-entry}. Suppose you want to create a simulation
panel entry project called \texttt{my-simulation-panel-entry} with a
package name of \texttt{com.liferay.docs.application.list} and a class
name of \texttt{SampleSimulationPanelApp}. You could run the following
command to accomplish this:

\begin{verbatim}
blade create -t simulation-panel-entry -v 7.0 -p com.liferay.docs -c Sample my-simulation-panel-entry
\end{verbatim}

or

\begin{verbatim}
mvn archetype:generate \
    -DarchetypeGroupId=com.liferay \
    -DarchetypeArtifactId=com.liferay.project.templates.simulation.panel.entry \
    -DgroupId=com.liferay \
    -DartifactId=my-simulation-panel-entry \
    -Dpackage=com.liferay.docs \
    -Dversion=1.0 \
    -DclassName=Sample \
    -Dauthor=Joe Bloggs \
    -DliferayVersion=7.0
\end{verbatim}

After running the command above, your project's directory structure
would look like this

\begin{itemize}
\tightlist
\item
  \texttt{my-simulation-panel-entry}

  \begin{itemize}
  \tightlist
  \item
    \texttt{gradle} (only in Blade CLI generated projects)

    \begin{itemize}
    \tightlist
    \item
      \texttt{wrapper}

      \begin{itemize}
      \tightlist
      \item
        \texttt{gradle-wrapper.jar}
      \item
        \texttt{gradle-wrapper.properties}
      \end{itemize}
    \end{itemize}
  \item
    \texttt{src}

    \begin{itemize}
    \tightlist
    \item
      \texttt{main}

      \begin{itemize}
      \tightlist
      \item
        \texttt{java}

        \begin{itemize}
        \tightlist
        \item
          \texttt{com/liferay/docs/application/list}

          \begin{itemize}
          \tightlist
          \item
            \texttt{SampleSimulationPanelApp.java}
          \end{itemize}
        \end{itemize}
      \item
        \texttt{resources}

        \begin{itemize}
        \tightlist
        \item
          \texttt{content}

          \begin{itemize}
          \tightlist
          \item
            \texttt{Language.properties}
          \end{itemize}
        \item
          \texttt{META-INF}

          \begin{itemize}
          \tightlist
          \item
            \texttt{resources}

            \begin{itemize}
            \tightlist
            \item
              \texttt{simulation\_panel.jsp}
            \end{itemize}
          \end{itemize}
        \end{itemize}
      \end{itemize}
    \end{itemize}
  \item
    \texttt{bnd.bnd}
  \item
    \texttt{build.gradle}
  \item
    \texttt{{[}gradlew\textbar{}pom.xml{]}}
  \end{itemize}
\end{itemize}

The generated module is functional and is deployable to a Liferay DXP
instance. To build upon the generated app, modify the project by adding
logic and additional files to the folders outlined above. You can visit
the
\href{https://github.com/liferay/liferay-blade-samples/tree/7.0/gradle/apps/simulation-panel-app}{simulation-panel-app}
sample project for a more expanded sample of a control menu entry.
Likewise, see the
\href{/docs/7-0/tutorials/-/knowledge_base/t/extending-the-simulation-menu}{Extending
the Simulation Menu} tutorial for instructions on customizing a
simulation panel entry project.

\section{Soy Portlet Template}\label{soy-portlet-template}

In this article, you'll learn how to create a Soy portlet application as
a Liferay module. To create a Soy portlet as a module via the command
line using Blade CLI or Maven, use one of the commands with the
following parameters:

\begin{verbatim}
blade create -t soy-portlet -v 7.0 [-p packageName] [-c className] projectName
\end{verbatim}

or

\begin{verbatim}
mvn archetype:generate \
    -DarchetypeGroupId=com.liferay \
    -DarchetypeArtifactId=com.liferay.project.templates.soy.portlet \
    -DartifactId=[projectName] \
    -Dpackage=[packageName] \
    -DclassName=[className] \
    -DliferayVersion=7.0
\end{verbatim}

You can also insert the \texttt{-b\ maven} parameter in the Blade
command to generate a Maven project using Blade CLI.

The template for this kind of project is \texttt{soy-portlet}. Suppose
you want to create an Soy portlet project called
\texttt{my-soy-portlet-project} with a package name of
\texttt{com.liferay.docs.soyportlet} and a class name of
\texttt{MySoyPortlet}. Also, you'd like to create a service of type
\texttt{javax.portlet.Portlet} that extends the
\texttt{com.liferay.portal.portlet.bridge.soy.SoyPortlet} class. Here,
\emph{service} means an OSGi service, not a Liferay API. Another way to
say \emph{service type} is to say \emph{component type}. You could run
the following command to accomplish this:

\begin{verbatim}
blade create -t soy-portlet -v 7.0 -p com.liferay.docs.soyportlet -c MySoyPortlet my-soy-portlet-project
\end{verbatim}

or

\begin{verbatim}
mvn archetype:generate \
    -DarchetypeGroupId=com.liferay \
    -DarchetypeArtifactId=com.liferay.project.templates.soy.portlet \
    -DgroupId=com.liferay \
    -DartifactId=my-soy-portlet-project \
    -Dpackage=com.liferay.docs.soyportlet \
    -Dversion=1.0 \
    -DclassName=MySoyPortlet \
    -Dauthor=Joe Bloggs \
    -DliferayVersion=7.0
\end{verbatim}

After running the command above, your project's directory structure
looks like this:

\begin{itemize}
\tightlist
\item
  \texttt{my-soy-portlet-project}

  \begin{itemize}
  \tightlist
  \item
    \texttt{gradle} (only in Blade CLI generated projects)

    \begin{itemize}
    \tightlist
    \item
      \texttt{wrapper}

      \begin{itemize}
      \tightlist
      \item
        \texttt{gradle-wrapper.jar}
      \item
        \texttt{gradle-wrapper.properties}
      \end{itemize}
    \end{itemize}
  \item
    \texttt{src}

    \begin{itemize}
    \tightlist
    \item
      \texttt{main}

      \begin{itemize}
      \tightlist
      \item
        \texttt{java}

        \begin{itemize}
        \tightlist
        \item
          \texttt{com/liferay/docs/soyportlet}

          \begin{itemize}
          \tightlist
          \item
            \texttt{constants}

            \begin{itemize}
            \tightlist
            \item
              \texttt{MySoyPortletKeys.java}
            \end{itemize}
          \item
            \texttt{portlet}

            \begin{itemize}
            \tightlist
            \item
              \texttt{action}

              \begin{itemize}
              \tightlist
              \item
                \texttt{MySoyPortletNavigationMVCRenderCommand.java}
              \end{itemize}
            \item
              \texttt{MySoyPortlet.java}
            \end{itemize}
          \end{itemize}
        \end{itemize}
      \item
        \texttt{resources}

        \begin{itemize}
        \tightlist
        \item
          \texttt{content}

          \begin{itemize}
          \tightlist
          \item
            \texttt{Language.properties}
          \end{itemize}
        \item
          \texttt{META-INF}

          \begin{itemize}
          \tightlist
          \item
            \texttt{resources}

            \begin{itemize}
            \tightlist
            \item
              \texttt{Footer.es}
            \item
              \texttt{Footer.soy}
            \item
              \texttt{Header.es}
            \item
              \texttt{Header.soy}
            \item
              \texttt{Navigation.es}
            \item
              \texttt{Navigation.soy}
            \item
              \texttt{View.es}
            \item
              \texttt{View.soy}
            \end{itemize}
          \end{itemize}
        \end{itemize}
      \end{itemize}
    \end{itemize}
  \item
    \texttt{bnd.bnd}
  \item
    \texttt{build.gradle}
  \item
    \texttt{package.json}
  \item
    \texttt{{[}gradlew\textbar{}pom.xml{]}}
  \end{itemize}
\end{itemize}

The generated module is a working application and is deployable to a
Liferay DXP instance. To build upon the generated app, modify the
project by adding logic and additional files to the folders outlined
above.

\section{Spring MVC Portlet Template}\label{spring-mvc-portlet-template}

In this article, you'll learn how to create a Liferay Spring MVC portlet
application as a WAR. To create a Liferay Spring MVC portlet via the
command line using Blade CLI or Maven, use one of the commands with the
following parameters:

\begin{verbatim}
blade create -t spring-mvc-portlet -v 7.0 [-p packageName] [-c className] projectName
\end{verbatim}

or

\begin{verbatim}
mvn archetype:generate \
    -DarchetypeGroupId=com.liferay \
    -DarchetypeArtifactId=com.liferay.project.templates.spring.mvc.portlet \
    -DartifactId=[projectName] \
    -Dpackage=[packageName] \
    -DclassName=[className] \
    -DliferayVersion=7.0
\end{verbatim}

You can also insert the \texttt{-b\ maven} parameter in the Blade
command to generate a Maven project using Blade CLI.

The template for this kind of project is \texttt{spring-mvc-portlet}.
Suppose you want to create a Spring MVC portlet project called
\texttt{my-spring-mvc-portlet-project} with a package name of
\texttt{com.liferay.docs.springmvcportlet} and a class name of
\texttt{MySpringMvcPortlet}. Also, you'd like to create a
Spring-annotated portlet class named
\texttt{MySpringMvcPortletViewController}.

\begin{verbatim}
blade create -t spring-mvc-portlet -v 7.0 -p com.liferay.docs.springmvcportlet -c MySpringMvcPortlet my-spring-mvc-portlet-project
\end{verbatim}

or

\begin{verbatim}
mvn archetype:generate \
    -DarchetypeGroupId=com.liferay \
    -DarchetypeArtifactId=com.liferay.project.templates.spring.mvc.portlet \
    -DgroupId=com.liferay \
    -DartifactId=my-spring-mvc-portlet-project \
    -Dpackage=com.liferay.docs.springmvcportlet \
    -Dversion=1.0 \
    -DclassName=MySpringMvcPortlet \
    -Dauthor=Joe Bloggs \
    -DliferayVersion=7.0
\end{verbatim}

After running the command above, your project's directory structure
looks like this:

\begin{itemize}
\tightlist
\item
  \texttt{my-spring-mvc-portlet-project}

  \begin{itemize}
  \tightlist
  \item
    \texttt{gradle} (only in Blade CLI generated projects)

    \begin{itemize}
    \tightlist
    \item
      \texttt{wrapper}

      \begin{itemize}
      \tightlist
      \item
        \texttt{gradle-wrapper.jar}
      \item
        \texttt{gradle-wrapper.properties}
      \end{itemize}
    \end{itemize}
  \item
    \texttt{src}

    \begin{itemize}
    \tightlist
    \item
      \texttt{main}

      \begin{itemize}
      \tightlist
      \item
        \texttt{java}

        \begin{itemize}
        \tightlist
        \item
          \texttt{com/liferay/docs/springmvcportlet/portlet}

          \begin{itemize}
          \tightlist
          \item
            \texttt{MySpringMvcPortletViewController}
          \end{itemize}
        \end{itemize}
      \item
        \texttt{resources}

        \begin{itemize}
        \tightlist
        \item
          \texttt{content}

          \begin{itemize}
          \tightlist
          \item
            \texttt{Language.properties}
          \end{itemize}
        \end{itemize}
      \item
        \texttt{webapp}

        \begin{itemize}
        \tightlist
        \item
          \texttt{css}

          \begin{itemize}
          \tightlist
          \item
            \texttt{main.scss}
          \end{itemize}
        \item
          \texttt{WEB-INF}

          \begin{itemize}
          \tightlist
          \item
            \texttt{jsp}

            \begin{itemize}
            \tightlist
            \item
              \texttt{init.jsp}
            \item
              \texttt{view.jsp}
            \end{itemize}
          \item
            \texttt{spring-context}

            \begin{itemize}
            \tightlist
            \item
              \texttt{portlet}

              \begin{itemize}
              \tightlist
              \item
                \texttt{my-spring-mvc-portlet-project.xml}
              \end{itemize}
            \item
              \texttt{portlet-application-context.xml}
            \end{itemize}
          \item
            \texttt{tld}

            \begin{itemize}
            \tightlist
            \item
              \texttt{liferay-portlet.tld}
            \item
              \texttt{liferay-portlet-ext.tld}
            \item
              \texttt{liferay-security.tld}
            \item
              \texttt{liferay-theme.tld}
            \item
              \texttt{liferay-ui.tld}
            \item
              \texttt{liferay-util.tld}
            \end{itemize}
          \item
            \texttt{liferay-display.xml}
          \item
            \texttt{liferay-plugin-package.properties}
          \item
            \texttt{liferay-portlet.xml}
          \item
            \texttt{portlet.xml}
          \item
            \texttt{web.xml}
          \end{itemize}
        \item
          \texttt{icon.png}
        \end{itemize}
      \end{itemize}
    \end{itemize}
  \item
    \texttt{build.gradle}
  \item
    \texttt{{[}gradlew\textbar{}pom.xml{]}}
  \end{itemize}
\end{itemize}

The generated WAR is a working application and is deployable to a
Liferay DXP instance. To build upon the generated app, modify the
project by adding logic and additional files to the folders outlined
above. You can visit the
\href{/docs/7-0/reference/-/knowledge_base/r/spring-mvc-portlet}{springmvc-portlet}
sample project for a more expanded sample of a Spring MVC portlet.

\section{Template Context Contributor
Template}\label{template-context-contributor-template}

In this article, you'll learn how to create a Liferay template context
contributor as a Liferay module. To create a template context
contributor via the command line using Blade CLI or Maven, use one of
the commands with the following parameters:

\begin{verbatim}
blade create -t template-context-contributor -v 7.0 [-p packageName] [-c className] projectName
\end{verbatim}

or

\begin{verbatim}
mvn archetype:generate \
    -DarchetypeGroupId=com.liferay \
    -DarchetypeArtifactId=com.liferay.project.templates.template.context.contributor \
    -DartifactId=[projectName] \
    -Dpackage=[packageName] \
    -DclassName=[className] \
    -DliferayVersion=7.0
\end{verbatim}

You can also insert the \texttt{-b\ maven} parameter in the Blade
command to generate a Maven project using Blade CLI.

The template for this kind of project is
\texttt{template-context-contributor}. Suppose you want to create a
template context contributor project called
\texttt{my-template-context-contributor} with a package name of
\texttt{com.liferay.docs.theme.contributor} and a class name of
\texttt{SampleTemplateContextContributor}. You could run the following
command to accomplish this:

\begin{verbatim}
blade create -t template-context-contributor -v 7.0 -p com.liferay.docs -c Sample my-template-context-contributor
\end{verbatim}

or

\begin{verbatim}
mvn archetype:generate \
    -DarchetypeGroupId=com.liferay \
    -DarchetypeArtifactId=com.liferay.project.templates.template.context.contributor \
    -DgroupId=com.liferay \
    -DartifactId=my-template-context-contributor \
    -Dpackage=com.liferay.docs \
    -Dversion=1.0 \
    -DclassName=Sample \
    -Dauthor=Joe Bloggs \
    -DliferayVersion=7.0
\end{verbatim}

After running the command above, your project's directory structure
would look like this

\begin{itemize}
\tightlist
\item
  \texttt{my-template-context-contributor}

  \begin{itemize}
  \tightlist
  \item
    \texttt{gradle} (only in Blade CLI generated projects)

    \begin{itemize}
    \tightlist
    \item
      \texttt{wrapper}

      \begin{itemize}
      \tightlist
      \item
        \texttt{gradle-wrapper.jar}
      \item
        \texttt{gradle-wrapper.properties}
      \end{itemize}
    \end{itemize}
  \item
    \texttt{src}

    \begin{itemize}
    \tightlist
    \item
      \texttt{main}

      \begin{itemize}
      \tightlist
      \item
        \texttt{java}

        \begin{itemize}
        \tightlist
        \item
          \texttt{com/liferay/docs/theme/contributor}

          \begin{itemize}
          \tightlist
          \item
            \texttt{SampleTemplateContextContributor.java}
          \end{itemize}
        \end{itemize}
      \end{itemize}
    \end{itemize}
  \item
    \texttt{bnd.bnd}
  \item
    \texttt{build.gradle}
  \item
    \texttt{{[}gradlew\textbar{}pom.xml{]}}
  \end{itemize}
\end{itemize}

The generated module is functional and is deployable to a Liferay DXP
instance. To build upon the generated app, modify the project by adding
logic and additional files to the folders outlined above. You can visit
the
\href{https://github.com/liferay/liferay-blade-samples/tree/7.0/gradle/themes/template-context-contributor}{template-context-contributor}
sample project for a more expanded sample of a template context
contributor. Likewise, see the
\href{/docs/7-0/tutorials/-/knowledge_base/t/context-contributors}{Context
Contributors} tutorial for instructions on customizing a template
context contributor project.

\section{Theme Template}\label{theme-template}

In this article, you'll learn how to create a Liferay theme as a WAR
project. To create a Liferay theme via the command line using Blade CLI
or Maven, use one of the commands with the following parameters:

\begin{verbatim}
blade create -t theme -v 7.0 projectName
\end{verbatim}

or

\begin{verbatim}
mvn archetype:generate \
    -DarchetypeGroupId=com.liferay \
    -DarchetypeArtifactId=com.liferay.project.templates.theme \
    -DartifactId=[projectName] \
    -DliferayVersion=7.0
\end{verbatim}

You can also insert the \texttt{-b\ maven} parameter in the Blade
command to generate a Maven project using Blade CLI.

The template for this kind of project is \texttt{theme}. Suppose you
want to create a theme project called \texttt{my-theme-project} as a WAR
file. You could run the following command to accomplish this:

\begin{verbatim}
blade create -t theme -v 7.0 my-theme-project
\end{verbatim}

or

\begin{verbatim}
mvn archetype:generate \
    -DarchetypeGroupId=com.liferay \
    -DarchetypeArtifactId=com.liferay.project.templates.theme \
    -DgroupId=com.liferay \
    -DartifactId=my-theme-project \
    -Dversion=1.0 \
    -DliferayVersion=7.0
\end{verbatim}

After running the command above, your project's folder structure looks
like this:

\begin{itemize}
\tightlist
\item
  \texttt{my-theme-project}

  \begin{itemize}
  \tightlist
  \item
    \texttt{gradle} (only in Blade CLI generated projects)

    \begin{itemize}
    \tightlist
    \item
      \texttt{wrapper}

      \begin{itemize}
      \tightlist
      \item
        \texttt{gradle-wrapper.jar}
      \item
        \texttt{gradle-wrapper.properties}
      \end{itemize}
    \end{itemize}
  \item
    \texttt{src}

    \begin{itemize}
    \tightlist
    \item
      \texttt{main}

      \begin{itemize}
      \tightlist
      \item
        \texttt{resources}

        \begin{itemize}
        \tightlist
        \item
          \texttt{resources-importer}

          \begin{itemize}
          \tightlist
          \item
            \texttt{sitemap.json}
          \end{itemize}
        \end{itemize}
      \item
        \texttt{webapp}

        \begin{itemize}
        \tightlist
        \item
          \texttt{css}

          \begin{itemize}
          \tightlist
          \item
            \texttt{\_custom.scss}
          \end{itemize}
        \item
          \texttt{WEB-INF}

          \begin{itemize}
          \tightlist
          \item
            \texttt{liferay-plugin-package.properties}
          \item
            \texttt{web.xml}
          \end{itemize}
        \end{itemize}
      \end{itemize}
    \end{itemize}
  \item
    \texttt{build.gradle}
  \item
    \texttt{{[}gradlew\textbar{}pom.xml{]}}
  \end{itemize}
\end{itemize}

The generated theme is functional and is deployable to a Liferay DXP
instance. To build upon the generated project, modify the project by
adding logic and additional files to the folders outlined above. You can
visit the
\href{/docs/7-0/reference/-/knowledge_base/r/theme}{simple-theme}
project for a more expanded sample of a theme. Likewise, see the
\href{/docs/7-0/tutorials/-/knowledge_base/t/themes-and-layout-templates}{Themes
and Layout Templates} tutorial section for more information on creating
themes.

\section{Theme Contributor Template}\label{theme-contributor-template}

In this article, you'll learn how to create a Liferay theme contributor
as a Liferay module. To create a theme contributor via the command line
using Blade CLI or Maven, use one of the commands with the following
parameters:

\begin{verbatim}
blade create -t theme-contributor -v 7.0 [--contributor-type contributorType] [-p packageName] projectName
\end{verbatim}

or

\begin{verbatim}
mvn archetype:generate \
    -DarchetypeGroupId=com.liferay \
    -DarchetypeArtifactId=com.liferay.project.templates.theme.contributor \
    -DartifactId=[projectName] \
    -Dpackage=[packageName] \
    -DcontributorType=[contributorType] \
    -DliferayVersion=7.0
\end{verbatim}

You can also insert the \texttt{-b\ maven} parameter in the Blade
command to generate a Maven project using Blade CLI.

The template for this kind of project is \texttt{theme-contributor}.
Suppose you want to create a theme contributor project called
\texttt{my-theme-contributor} with a package name of
\texttt{com.liferay.docs.theme.contributor} and a contributor type of
\texttt{my-contributor}. You could run the following command to
accomplish this:

\begin{verbatim}
blade create -t theme-contributor -v 7.0 --contributor-type my-contributor -p com.liferay.docs.theme.contributor my-theme-contributor
\end{verbatim}

or

\begin{verbatim}
mvn archetype:generate \
    -DarchetypeGroupId=com.liferay \
    -DarchetypeArtifactId=com.liferay.project.templates.theme.contributor \
    -DgroupId=com.liferay \
    -DartifactId=my-theme-contributor \
    -Dpackage=com.liferay.docs.theme.contributor \
    -Dversion=1.0 \
    -DcontributorType=my-contributor \
    -DliferayVersion=7.0
\end{verbatim}

After running the command above, your project's folder structure would
look like this:

\begin{itemize}
\tightlist
\item
  \texttt{my-theme-contributor}

  \begin{itemize}
  \tightlist
  \item
    \texttt{src}

    \begin{itemize}
    \tightlist
    \item
      \texttt{main}

      \begin{itemize}
      \tightlist
      \item
        \texttt{java}

        \begin{itemize}
        \tightlist
        \item
          \texttt{com/liferay/docs/theme/contributor}
        \end{itemize}
      \item
        \texttt{resources/META-INF/resources}

        \begin{itemize}
        \tightlist
        \item
          \texttt{css}

          \begin{itemize}
          \tightlist
          \item
            \texttt{my-contributor}

            \begin{itemize}
            \tightlist
            \item
              \texttt{\_body.scss}
            \item
              \texttt{\_control\_menu.scss}
            \item
              \texttt{\_product\_menu.scss}
            \item
              \texttt{\_simulation\_panel.scss}
            \end{itemize}
          \item
            \texttt{my-contributor.scss}
          \end{itemize}
        \item
          \texttt{js}

          \begin{itemize}
          \tightlist
          \item
            \texttt{my-contributor.js}
          \end{itemize}
        \end{itemize}
      \end{itemize}
    \end{itemize}
  \item
    \texttt{bnd.bnd}
  \item
    \texttt{build.gradle} (only in Gradle Blade CLI generated projects)
  \item
    \texttt{mvnw} (only in Maven Blade CLI generated projects)
  \item
    \texttt{mvnw.cmd} (only in Maven Blade CLI generated projects)
  \item
    \texttt{pom.xml} (only in Maven-related projects)
  \end{itemize}
\end{itemize}

The generated module is functional and is deployable to a Liferay DXP
instance. To build upon the generated app, modify the project by adding
logic and additional files to the folders outlined above. You can visit
the
\href{/docs/7-0/reference/-/knowledge_base/r/theme-contributor}{Blade
Theme Contributor} sample project for a more expanded sample of a theme
contributor. Likewise, see the
\href{/docs/7-0/tutorials/-/knowledge_base/t/theme-contributors}{Theme
Contributors} tutorial for instructions on customizing a theme
contributor project.

\section{WAR Hook Template}\label{war-hook-template}

In this article, you'll learn how to create a Liferay WAR hook project.
To create a Liferay WAR hook via the command line using Blade CLI or
Maven, use one of the commands with the following parameters:

\begin{verbatim}
blade create -t war-hook -v 7.0 [-p packageName] [-c className] projectName
\end{verbatim}

or

\begin{verbatim}
mvn archetype:generate \
    -DarchetypeGroupId=com.liferay \
    -DarchetypeArtifactId=com.liferay.project.templates.war.hook \
    -DartifactId=[projectName]
    -Dpackage=[packageName] \
    -DclassName=[className] \
    -DliferayVersion=7.0
\end{verbatim}

You can also insert the \texttt{-b\ maven} parameter in the Blade
command to generate a Maven project using Blade CLI.

The template for this kind of project is \texttt{war-hook}. Suppose you
want to create a WAR hook project called \texttt{my-war-hook-project}
with a package name of \texttt{com.liferay.docs} and a class name of
\texttt{MyWarHook}. You could run the following command to accomplish
this:

\begin{verbatim}
blade create -t war-hook -v 7.0 -p com.liferay.docs -c MyWarHook my-war-hook-project
\end{verbatim}

or

\begin{verbatim}
mvn archetype:generate \
    -DarchetypeGroupId=com.liferay \
    -DarchetypeArtifactId=com.liferay.project.templates.war.hook \
    -DgroupId=com.liferay \
    -DartifactId=my-war-hook-project \
    -Dpackage=com.liferay.docs \
    -DclassName=MyWarHook \
    -Dversion=1.0 \
    -DliferayVersion=7.0
\end{verbatim}

After running the command above, your project's folder structure looks
like this:

\begin{itemize}
\tightlist
\item
  \texttt{my-war-hook-project}

  \begin{itemize}
  \tightlist
  \item
    \texttt{{[}gradle\textbar{}.mvn{]}}

    \begin{itemize}
    \tightlist
    \item
      \texttt{wrapper}

      \begin{itemize}
      \tightlist
      \item
        \texttt{{[}gradle\textbar{}maven{]}-wrapper.jar}
      \item
        \texttt{{[}gradle\textbar{}maven{]}-wrapper.properties}
      \end{itemize}
    \end{itemize}
  \item
    \texttt{src}

    \begin{itemize}
    \tightlist
    \item
      \texttt{main}

      \begin{itemize}
      \tightlist
      \item
        \texttt{java}

        \begin{itemize}
        \tightlist
        \item
          \texttt{com/liferay/docs}

          \begin{itemize}
          \tightlist
          \item
            \texttt{MyWarHookLoginPostAction}
          \item
            \texttt{MyWarHookStartupAction}
          \end{itemize}
        \end{itemize}
      \item
        \texttt{resources}

        \begin{itemize}
        \tightlist
        \item
          \texttt{portal.properties}
        \end{itemize}
      \item
        \texttt{webapp}

        \begin{itemize}
        \tightlist
        \item
          \texttt{WEB-INF}

          \begin{itemize}
          \tightlist
          \item
            \texttt{liferay-hook.xml}
          \item
            \texttt{liferay-plugin-package.properties}
          \item
            \texttt{web.xml}
          \end{itemize}
        \end{itemize}
      \end{itemize}
    \end{itemize}
  \item
    \texttt{{[}build.gradle\textbar{}pom.xml{]}}
  \item
    \texttt{{[}gradlew\textbar{}mvnw{]}}
  \end{itemize}
\end{itemize}

The generated WAR hook is functional and is deployable to a Liferay DXP
instance. To build upon the generated project, modify the project by
adding logic and additional files to the folders outlined above.
Deploying WAR hooks is supported for 7.0, however, it is recommended to
optimize your WAR hooks to fragments or other applicable module
projects. You can visit the
\href{/docs/7-0/tutorials/-/knowledge_base/t/customizing}{Customizing}
section for info on how to do this for many project types. See the
\href{/docs/6-2/tutorials/-/knowledge_base/t/customizing-liferay-portal}{Customizing
Liferay Portal} section for more information on WAR hooks.

\section{WAR MVC Portlet Template}\label{war-mvc-portlet-template}

In this article, you'll learn how to create a Liferay MVC portlet
project as a WAR file. To create a Liferay MVC portlet project as a WAR
via the command line using Blade CLI or Maven, use one of the commands
with the following parameters:

\begin{verbatim}
blade create -t war-mvc-portlet -v 7.0 [-p packageName] projectName
\end{verbatim}

or

\begin{verbatim}
mvn archetype:generate \
    -DarchetypeGroupId=com.liferay \
    -DarchetypeArtifactId=com.liferay.project.templates.war.mvc.portlet \
    -DartifactId=[projectName] \
    -Dpackage=[packageName] \
    -DliferayVersion=7.0
\end{verbatim}

You can also insert the \texttt{-b\ maven} parameter in the Blade
command to generate a Maven project using Blade CLI.

The template for this kind of project is \texttt{war-mvc-portlet}.
Suppose you want to create a WAR MVC portlet project called
\texttt{my-war-mvc-portlet-project} with a package name of
\texttt{com.liferay.docs.war.mvc} and a class name of
\texttt{MyWarMvcPortlet}. You could run the following command to
accomplish this:

\begin{verbatim}
blade create -t war-mvc-portlet -v 7.0 -p com.liferay.docs.war.mvc my-war-mvc-portlet-project
\end{verbatim}

or

\begin{verbatim}
mvn archetype:generate \
    -DarchetypeGroupId=com.liferay \
    -DarchetypeArtifactId=com.liferay.project.templates.war.mvc.portlet \
    -DgroupId=com.liferay \
    -DartifactId=my-war-mvc-portlet-project \
    -Dpackage=com.liferay.docs.war.mvc \
    -Dversion=1.0 \
    -DliferayVersion=7.0
\end{verbatim}

After running the command above, your project's folder structure looks
like this:

\begin{itemize}
\tightlist
\item
  \texttt{my-war-mvc-portlet-project}

  \begin{itemize}
  \tightlist
  \item
    \texttt{{[}gradle\textbar{}.mvn{]}}

    \begin{itemize}
    \tightlist
    \item
      \texttt{wrapper}

      \begin{itemize}
      \tightlist
      \item
        \texttt{{[}gradle\textbar{}maven{]}-wrapper.jar}
      \item
        \texttt{{[}gradle\textbar{}maven{]}-wrapper.properties}
      \end{itemize}
    \end{itemize}
  \item
    \texttt{src}

    \begin{itemize}
    \tightlist
    \item
      \texttt{main}

      \begin{itemize}
      \tightlist
      \item
        \texttt{java}

        \begin{itemize}
        \tightlist
        \item
          \texttt{com/liferay/docs/war/mvc}
        \end{itemize}
      \item
        \texttt{resources}

        \begin{itemize}
        \tightlist
        \item
          \texttt{content}

          \begin{itemize}
          \tightlist
          \item
            \texttt{Language.properties}
          \end{itemize}
        \end{itemize}
      \item
        \texttt{webapp}

        \begin{itemize}
        \tightlist
        \item
          \texttt{css}

          \begin{itemize}
          \tightlist
          \item
            \texttt{.sass-cache}

            \begin{itemize}
            \tightlist
            \item
              \texttt{main.css}
            \item
              \texttt{main\_rtl.css}
            \end{itemize}
          \item
            \texttt{main.scss}
          \end{itemize}
        \item
          \texttt{WEB-INF}

          \begin{itemize}
          \tightlist
          \item
            \texttt{tld}

            \begin{itemize}
            \tightlist
            \item
              \texttt{liferay-portlet.tld}
            \item
              \texttt{liferay-portlet-ext.tld}
            \item
              \texttt{liferay-security.tld}
            \item
              \texttt{liferay-theme.tld}
            \item
              \texttt{liferay-ui.tld}
            \item
              \texttt{liferay-util.tld}
            \end{itemize}
          \item
            \texttt{liferay-display.xml}
          \item
            \texttt{liferay-plugin-package.properties}
          \item
            \texttt{liferay-portlet.xml}
          \item
            \texttt{portlet.xml}
          \item
            \texttt{web.xml}
          \end{itemize}
        \item
          \texttt{init.jsp}
        \item
          \texttt{view.jsp}
        \end{itemize}
      \end{itemize}
    \end{itemize}
  \item
    \texttt{{[}build.gradle\textbar{}pom.xml{]}}
  \item
    \texttt{{[}gradlew\textbar{}mvnw{]}}
  \end{itemize}
\end{itemize}

The generated WAR MVC portlet is functional and is deployable to a
Liferay DXP instance. To build upon the generated project, modify the
project by adding logic and additional files to the folders outlined
above. Deploying WAR MVC portlets is supported for 7.0, however, it is
recommended to optimize your WAR portlet to a module project, if
possible. You can visit the
\href{/docs/7-0/tutorials/-/knowledge_base/t/from-liferay-6-to-liferay-7}{From
Liferay Portal 6 to 7} section for info on how to do this.

\chapter{Sample Projects}\label{sample-projects}

\noindent\hrulefill

\textbf{Note:} This section of articles does not provide documentation
for \emph{all} sample projects residing in the
\texttt{liferay-blade-samples} repo. The documentation for these samples
is in progress and will grow over time.

\noindent\hrulefill

Liferay provides sample projects that target different integration
points in Liferay DXP. These projects reside in the
\href{https://github.com/liferay/liferay-blade-samples}{liferay-blade-samples}
Github repository and can be easily copy/pasted to your local
environment. The sample projects are grouped into three different parent
folders based on the build tools used to generate them:

\begin{itemize}
\tightlist
\item
  \texttt{gradle}
\item
  \texttt{liferay-workspace}
\item
  \texttt{maven}
\end{itemize}

\noindent\hrulefill

\textbf{Note:} The Liferay Workspace folder stores WAR-type samples in a
separate folder named
\href{https://github.com/liferay/liferay-blade-samples/tree/7.0/liferay-workspace/wars}{wars}.
The Gradle and Maven tool folders mix WAR samples with the other sample
types (apps, extensions, etc.).

\noindent\hrulefill

For more information on these sample projects, visit the
\href{/docs/7-0/tutorials/-/knowledge_base/t/liferay-sample-modules}{Liferay
Sample Projects} tutorial.

\chapter{Apps}\label{apps}

This section focuses on Liferay sample applications. You can view these
sample apps by visiting the \texttt{apps} folder corresponding to your
preferred build tool:

\begin{itemize}
\tightlist
\item
  \href{https://github.com/liferay/liferay-blade-samples/tree/7.0/gradle/apps}{Gradle
  sample apps}
\item
  \href{https://github.com/liferay/liferay-blade-samples/tree/7.0/liferay-workspace/apps}{Liferay
  Workspace sample apps}
\item
  \href{https://github.com/liferay/liferay-blade-samples/tree/7.0/maven/apps}{Maven
  sample apps}
\end{itemize}

The following samples are documented:

\begin{itemize}
\tightlist
\item
  \href{greedy-policy-option-portlet}{Greedy Policy Option Portlet}
\item
  \href{kotlin-portlet}{Kotlin Portlet}
\item
  \href{npm-samples}{npm Samples}
\item
  \href{service-builder-samples}{Service Builder Samples}
\item
  \href{shared-language-keys}{Shared Language Keys}
\item
  \href{simulation-panel-app}{Simulation Panel App}
\item
  \href{spring-mvc-portlet}{Spring MVC Portlet}
\end{itemize}

Visit a particular sample page to learn more!

\section{Greedy Policy Option
Application}\label{greedy-policy-option-application}

The Greedy Policy Option sample provides two portlets that can be added
to a Liferay DXP page: Greedy Portlet and Reluctant Portlet.

\begin{figure}
\centering
\includegraphics{./images/greedy-policy-app.png}
\caption{The Greedy Policy Option app provides two portlets that only
print text. You'll dive deeper later to discover their interesting
capabilities involving services.}
\end{figure}

These two portlets do not provide anything useful out-of-the-box. They
are, however, very effective at demonstrating the ability to reference
services using greedy and reluctant policy options. You'll learn how to
do this later.

\subsection{What API(s) and/or code components does this sample
highlight?}\label{what-apis-andor-code-components-does-this-sample-highlight}

This sample provides two modules referencing services using greedy and
reluctant policy options.

\begin{itemize}
\item
  \texttt{service-reference}: Provides an OSGi service interface called
  \texttt{SomeService}, a default implementation of it, and portlets
  that refer to service instances. One portlet refers to new higher
  ranked instances of the service automatically. The other portlet is
  reluctant to use new higher ranked instances and continues to use its
  bound service. The reluctant portlet can, however, be configured
  dynamically to use other service instances.
\item
  \texttt{higher-ranked-service}: Has a higher ranked
  \texttt{SomeService} implementation.
\end{itemize}

Here are each module's file structures:

\begin{itemize}
\tightlist
\item
  \texttt{service-reference/}

  \begin{itemize}
  \tightlist
  \item
    \texttt{bnd.bnd}
  \item
    \texttt{configs/}

    \begin{itemize}
    \tightlist
    \item
      \texttt{com.liferay.blade.reluctant.vs.greedy.portlet.portlet.ReluctantPortlet.config}
      → \texttt{ReluctantPortlet} configuration file for Liferay DXP DE
      7.0 Fix Pack 8 or later and Liferay CE Portal 7.0 GA4 or later
    \item
      \texttt{com.liferay.blade.reluctant.vs.greedy.portlet.portlet.ReluctantPortlet.cfg}
      → \texttt{ReluctantPortlet} configuration file for Liferay DXP DE
      7.0 Fix Packs earlier than Fix Pack 8 and Liferay CE Portal 7.0
      GA3 or earlier
    \end{itemize}
  \item
    \texttt{src/main/java/com/liferay/blade/reluctant/vs/greedy/portlet/}

    \begin{itemize}
    \tightlist
    \item
      \texttt{api/}

      \begin{itemize}
      \tightlist
      \item
        \texttt{SomeService.java} → Service interface
      \end{itemize}
    \item
      \texttt{constants/}

      \begin{itemize}
      \tightlist
      \item
        \texttt{ReluctantPortletVsGreedyPortletKeys.java} → Portlet
        constants
      \end{itemize}
    \item
      \texttt{portlet/}

      \begin{itemize}
      \tightlist
      \item
        \texttt{DefaultSomeService.java} → Zero ranked service
        implementation
      \item
        \texttt{GreedyPortlet.java} → Refers to \texttt{SomeService}
        using a greedy service policy option
      \item
        \texttt{ReluctantPortletPortlet.java} → Refers to
        \texttt{SomeService} using a reluctant service policy option by
        default.
      \end{itemize}
    \end{itemize}
  \end{itemize}
\item
  \texttt{higher-ranked-service/}

  \begin{itemize}
  \tightlist
  \item
    \texttt{bnd.bnd}
  \item
    \texttt{src/main/java/com/liferay/blade/reluctant/vs/greedy/svc/HigherRankedService.java}
    → Service implementation with service ranking value of \texttt{100}
  \end{itemize}
\end{itemize}

\subsection{How does this sample leverage the API(s) and/or code
component?}\label{how-does-this-sample-leverage-the-apis-andor-code-component}

Here are the things you can learn using the sample modules:

\begin{enumerate}
\def\labelenumi{\arabic{enumi}.}
\item
  \hyperref[binding-a-newly-deployed-components-service-reference-to-the-highest-ranking-service-instance-thats-available-initially]{Binding
  a component's service reference to the highest ranked service instance
  that's available initially.}
\item
  \hyperref[deploying-a-module-with-a-higher-ranked-service-instance-for-binding-to-greedy-references-immediately]{Deploying
  a module with a higher ranked service instance for binding to greedy
  references immediately.}
\item
  \hyperref[configuring-a-component-to-reference-a-different-service-instance-dynamically]{Configuring
  a component to reference a different service instance dynamically.}
\end{enumerate}

Let's walk through the demonstration.

\subsubsection{Binding a newly deployed component's service reference to
the highest ranking service instance that's available
initially}\label{binding-a-newly-deployed-components-service-reference-to-the-highest-ranking-service-instance-thats-available-initially}

On deploying a component that references a service, it binds to the
highest ranking service instance that matches its target filter (if
specified).

The portlet classes refer to instances of interface
\texttt{SomeService}. The \texttt{doSomething} method returns a
\texttt{String}.

\begin{verbatim}
public interface SomeService {

    public String doSomething();

}
\end{verbatim}

Class \texttt{DefaultSomeService} implements \texttt{SomeService}. Its
\texttt{doSomething} method returns the \texttt{String} ``I am
Default!''.

\begin{verbatim}
@Component
public class DefaultSomeService implements SomeService {

    @Override
    public String doSomething() {
        return "I am Default!";
    }

}
\end{verbatim}

When module's portlets refer to \texttt{DefaultSomeService}, they
display the \texttt{String} ``I am Default!''.

The \texttt{ReluctantPortlet} class's \texttt{SomeService} reference's
policy option is the default: static and reluctant. This policy option
keeps the reference bound to its current service instance unless that
instance stops or the reference is reconfigured to refer to a different
service instance.

\begin{verbatim}
@Component(
   immediate = true,
   property = {
       "com.liferay.portlet.display-category=category.sample",
       "com.liferay.portlet.instanceable=true",
       "javax.portlet.display-name=Reluctant Portlet",
       "javax.portlet.init-param.template-path=/",
       "javax.portlet.init-param.view-template=/view.jsp",
       "javax.portlet.name=" + ReluctantVsGreedyPortletKeys.Reluctant,
       "javax.portlet.resource-bundle=content.Language",
       "javax.portlet.security-role-ref=power-user,user"
   },
   service = Portlet.class
)
public class ReluctantPortlet extends MVCPortlet {

   @Override
   public void doView(
           RenderRequest renderRequest, RenderResponse renderResponse)
       throws IOException, PortletException {

       renderRequest.setAttribute("doSomething", _someService.doSomething());

       super.doView(renderRequest, renderResponse);
   }

   @Reference
   private SomeService _someService;

}
\end{verbatim}

The \texttt{ReluctantPortlet}'s method \texttt{doView} sets render
request attribute \texttt{doSomething} to the value returned from the
\texttt{SomeService} instance's \texttt{doSomething} method (e.g.,
\texttt{DefaultService} returns ``I am default!'').

The \texttt{GreedyPortlet} class is similar to
\texttt{ReluctantPortlet}, except its \texttt{SomeService} reference's
policy option is static and greedy (i.e.,
\texttt{ReferencePolicyOption.GREEDY}).

\begin{verbatim}
public class GreedyPortlet extends MVCPortlet {

    @Override
    public void doView(
            RenderRequest renderRequest, RenderResponse renderResponse)
        throws IOException, PortletException {

        renderRequest.setAttribute("doSomething", _someService.doSomething());

        super.doView(renderRequest, renderResponse);
    }

    @Reference (policyOption = ReferencePolicyOption.GREEDY)
    private SomeService _someService;

}
\end{verbatim}

The greedy policy option lets the component switch to using a higher
ranked \texttt{SomeService} instance if one becomes active in the
system. The section
\hyperref[deploying-a-module-with-a-higher-ranked-service-instance-for-binding-to-greedy-references-immediately]{\emph{Deploying
a module with a higher ranked service instance for binding to greedy
references immediately}} demonstrates this portlet switching to a higher
ranked service.

It's time to see this module's portlets and service in action.

\begin{enumerate}
\def\labelenumi{\arabic{enumi}.}
\item
  Stop module \texttt{higher-ranked-service} if it's active.
\item
  Deploy the \texttt{service-reference} module.
\item
  Add the \emph{Reluctant Portlet} from the \emph{Add} →
  \emph{Applications} → \emph{Sample} category to a site page.

  The portlet displays the message ``SomeService says I am
  default!''--whose latter part comes from the render request attribute
  set by the \texttt{DefaultService} instance.

  \begin{figure}
  \centering
  \includegraphics{./images/reluctant-portlet-using-default.png}
  \caption{\emph{Reluctant Portlet} displays the message ``SomeService
  says I am default!''}
  \end{figure}
\item
  Add the \emph{Greedy Portlet} from the \emph{Add} →
  \emph{Applications} → \emph{Sample} category to a site page.

  The portlet displays the message ``SomeService says I am better, use
  me!''. Both portlets are referencing a \texttt{DefaultService}
  instance.

  \begin{figure}
  \centering
  \includegraphics{./images/greedy-portlet-using-default.png}
  \caption{\emph{Greedy Portlet} displays the message ``SomeService says
  I am better, use me!''}
  \end{figure}
\end{enumerate}

Since \texttt{DefaultService} is the only active \texttt{SomeService}
instance in the system, the portlets refer to it for their
\texttt{SomeService} fields.

The \texttt{DefaultService} and portlets \emph{Reluctant Portlet} and
\emph{Greedy Portlet} are active. Let's activate a higher ranked
\texttt{SomeService} instance and see how the portlets react to it.

\subsubsection{Deploying a module with a higher ranked service instance
for binding to greedy references
immediately}\label{deploying-a-module-with-a-higher-ranked-service-instance-for-binding-to-greedy-references-immediately}

Module \texttt{higher-ranked-service} provides a \texttt{SomeService}
implementation called \texttt{HigherRankedService}.
\texttt{HigherRankedService}'s service ranking is \texttt{100}--that's
\texttt{100} more than \texttt{DefaultService}'s ranking \texttt{0}. Its
\texttt{doSomething} method returns the \texttt{String} ``I am better,
use me!''.

\begin{enumerate}
\def\labelenumi{\arabic{enumi}.}
\tightlist
\item
  Deploy the \texttt{higher-ranked-service} module.
\item
  Refresh your page that has the portlets \emph{Reluctant Portlet} and
  \emph{Greedy Portlet}.
\end{enumerate}

\emph{Reluctant Portlet} continues displaying message ``SomeService says
I am better, use me!''. It's ``reluctant'' to unbind from the
\texttt{DefaultService} instance and bind to the newly activated
\texttt{HigherRankedService} service.

\emph{Greedy Portlet} displays a new message ``SomeService says I am
better, use me!''. The part of the message ``I am better, use me!''
comes from the \texttt{HigherRankedService} instance to which it refers.

\begin{figure}
\centering
\includegraphics{./images/greedy-portlet-using-higher-ranked-service.png}
\caption{The \emph{Greedy Portlet} is using a
\texttt{HigherRankedService} instance}
\end{figure}

Next, learn how to bind the \emph{Reluctant Portlet} to a
\texttt{HigherRankedService} instance.

\subsubsection{Configuring a component to reference a different service
instance
dynamically}\label{configuring-a-component-to-reference-a-different-service-instance-dynamically}

The \emph{Reluctant Portlet} is currently bound to a
\texttt{DefaultService} instance. It's ``reluctant'' to unbind from it
and bind to a different service. OSGi Configuration Administration lets
you reconfigure service references to filter on and bind to different
service instances.

The \texttt{service-reference} module's configuration files and
\texttt{com.liferay.blade.reluctant.vs.greedy.portlet.portlet.ReluctantPortlet.config}
and
\texttt{com.liferay.blade.reluctant.vs.greedy.portlet.portlet.ReluctantPortlet.cfg}
configure the \texttt{ReluctantPortlet} component to use a
\texttt{HigherRankedService} instance.

\begin{verbatim}
_someService.target=(component.name=com.liferay.blade.reluctant.vs.greedy.service.HigherRankedService)
\end{verbatim}

The service configuration filters on a service whose
\texttt{component.name} is
\texttt{com.liferay.blade.reluctant.vs.greedy.service.HigherRankedService}.

Note: For deploying to Liferay DXP DE 7.0 Fix Pack 8 or later or Liferay
CE Portal 7.0 GA4 or later, use file with suffix \texttt{.config}. For
earlier versions, use the file with suffix \texttt{.cfg}.

Here are the steps to reconfigure \texttt{ReluctantPortlet} to use
\texttt{HigherRankedService}:

\begin{enumerate}
\def\labelenumi{\arabic{enumi}.}
\tightlist
\item
  Copy the configuration file to
  \texttt{{[}Liferay-Home{]}/osgi/configs}.
\item
  Refresh your browser.
\end{enumerate}

\emph{Reluctant Portlet} displays a new message ``SomeService says I am
better, use me!''.

\begin{figure}
\centering
\includegraphics{./images/reluctant-portlet-using-higher-ranked-service.png}
\caption{\emph{Reluctant Portlet} is using the
\texttt{HigherRankedService} instance instead of a
\texttt{DefaultService} instance.}
\end{figure}

\emph{Reluctant Portlet} is using \texttt{HigherRankedService} instance
instead of a \texttt{DefaultService} instance. You've configured
\emph{Reluctant Portlet} to use a \texttt{HigherRankedService} instance!

\subsection{Where Is This Sample?}\label{where-is-this-sample}

There are three different versions of this sample, each built with a
different build tool:

\begin{itemize}
\tightlist
\item
  \href{https://github.com/liferay/liferay-blade-samples/tree/7.0/gradle/apps/greedy-policy-option-portlet}{Gradle}
\item
  \href{https://github.com/liferay/liferay-blade-samples/tree/7.0/liferay-workspace/apps/greedy-policy-option-portlet}{Liferay
  Workspace}
\item
  \href{https://github.com/liferay/liferay-blade-samples/tree/7.0/maven/apps/greedy-policy-option-portlet}{Maven}
\end{itemize}

\section{Kotlin Portlet}\label{kotlin-portlet}

The Kotlin Portlet sample provides an input form that accepts a name.
Once submitting a name, the portlet renders a greeting message.

\begin{figure}
\centering
\includegraphics{./images/kotlin-portlet.png}
\caption{After saving the inputted name, it's is displayed as a greeting
on the portlet page.}
\end{figure}

\subsection{What API(s) and/or code components does this sample
highlight?}\label{what-apis-andor-code-components-does-this-sample-highlight-1}

This sample highlights the use of the
\href{https://kotlinlang.org/}{Kotlin} programming language in
conjunction with Liferay's MVC framework. Specifically, this sample
leverages the
\href{@platform-ref@/7.0-latest/javadocs/portal-kernel/com/liferay/portal/kernel/portlet/bridges/mvc/MVCActionCommand.html}{MVCActionCommand}
interface.

\subsection{How does this sample leverage the API(s) and/or code
component?}\label{how-does-this-sample-leverage-the-apis-andor-code-component-1}

This sample uses the
\href{/docs/7-0/tutorials/-/knowledge_base/t/mvc-action-command}{MVC
Action Command}'s \texttt{processAction(...)} method to process the
inputted text (i.e., name). The text is set as an attribute in the
\texttt{KotlinGreeterActionCommandKt.kt} class using an
\texttt{ActionRequest} and then is retrieved in the JSP using a
\texttt{RenderRequest}.

\subsection{Where Is This Sample?}\label{where-is-this-sample-1}

This sample is built with the following build tools:

\begin{itemize}
\tightlist
\item
  \href{https://github.com/liferay/liferay-blade-samples/tree/7.0/gradle/apps/kotlin-portlet}{Gradle}
\item
  \href{https://github.com/liferay/liferay-blade-samples/tree/7.0/liferay-workspace/apps/kotlin-portlet}{Liferay
  Workspace}
\end{itemize}

\chapter{npm Samples}\label{npm-samples}

This section focuses on Liferay npm sample portlets built with Gradle.
You can view these samples by visiting the
\href{https://github.com/liferay/liferay-blade-samples/tree/7.0/gradle/apps/npm}{gradle/apps/npm}
folder in the \texttt{liferay-blade-samples} Github repository.

The following npm samples are documented:

\begin{itemize}
\tightlist
\item
  \href{/docs/7-0/reference/-/knowledge_base/r/angular-npm-portlet}{Angular
  npm Portlet}
\item
  \href{/docs/7-0/reference/-/knowledge_base/r/billboard-js-npm-portlet}{Billboard.js
  npm Portlet}
\item
  \href{/docs/7-0/reference/-/knowledge_base/r/isomorphic-npm-portlet}{Isomorphic
  npm Portlet}
\item
  \href{/docs/7-0/reference/-/knowledge_base/r/jquery-npm-portlet}{jQuery
  npm Portlet}
\item
  \href{/docs/7-0/reference/-/knowledge_base/r/metal-js-npm-portlet}{Metal.js
  npm Portlet}
\item
  \href{/docs/7-0/reference/-/knowledge_base/r/react-npm-portlet}{React
  npm Portlet}
\item
  \href{/docs/7-0/reference/-/knowledge_base/r/simple-npm-portlet}{Simple
  npm Portlet}
\item
  \href{/docs/7-0/reference/-/knowledge_base/r/vue-js-npm-portlet}{Vue.js
  npm Portlet}
\end{itemize}

\noindent\hrulefill

\textbf{Note:} The minifier fails on Liferay DXP 7.0 when JSDoc is
present in a portlet. To resolve this, use
\href{https://gruntjs.com/getting-started}{Grunt}
\href{https://www.npmjs.com/package/grunt-contrib-uglify}{uglify} to
remove the JSDoc comments. This process may take a long time, depending
on the number of files that require an update.

\noindent\hrulefill

Visit a particular sample page to learn more!

\section{Angular npm Portlet}\label{angular-npm-portlet}

The Angular npm Portlet sample provides a portlet that uses the
\href{https://angular.io/}{Angular} framework to render its output.

\begin{figure}
\centering
\includegraphics{./images/angular-npm-sample.png}
\caption{Type custom text in the field and watch it instantaneously
displayed in the portlet.}
\end{figure}

This portlet showcases Angular's speed and performance when rendering a
user interface.

\noindent\hrulefill

\textbf{Note:} The minifier fails on Liferay DXP 7.0 when JSDoc is
present in a portlet. To resolve this, use
\href{https://gruntjs.com/getting-started}{Grunt}
\href{https://www.npmjs.com/package/grunt-contrib-uglify}{uglify} to
remove the JSDoc comments. This process may take a long time, depending
on the number of files that require an update.

\noindent\hrulefill

\textbf{Important:} This sample works for Liferay DXP 7.0 Fix Pack 44+
and Liferay Portal CE GA7+.

\subsection{What API(s) and/or code components does this sample
highlight?}\label{what-apis-andor-code-components-does-this-sample-highlight-2}

This sample leverages the
\href{/docs/7-0/tutorials/-/knowledge_base/t/using-npm-in-your-portlets}{npm
development workflow support}.

\subsection{How does this sample leverage the API(s) and/or code
component?}\label{how-does-this-sample-leverage-the-apis-andor-code-component-2}

This sample uses the \href{https://www.npmjs.com/}{npm registry} to
download project dependencies and uses the
\href{https://github.com/liferay/liferay-npm-build-tools/tree/master/packages/liferay-npm-bundler}{liferay-npm-bundler
tool} to bundle the project dependencies inside the OSGi bundle JAR
file.

To accomplish the bundling, the project's build process relies on a
\texttt{build} script inside its \texttt{package.json} file:

\begin{verbatim}
"scripts": {
    "build": "tsc && liferay-npm-bundler"
},
\end{verbatim}

\subsection{Where Is This Sample?}\label{where-is-this-sample-2}

This sample is built with the following build tool:

\begin{itemize}
\tightlist
\item
  \href{https://github.com/liferay/liferay-blade-samples/tree/7.0/gradle/apps/npm/angular-npm-portlet}{Gradle}
\end{itemize}

\section{Billboard.js npm Portlet}\label{billboard.js-npm-portlet}

The Billboard.js npm Portlet sample provides a portlet that uses the
\href{https://naver.github.io/billboard.js/}{Billboard.js} framework to
render its output.

\begin{figure}
\centering
\includegraphics{./images/billboardjs-npm-sample.png}
\caption{The Billboard.js npm Portlet shows off some nice looking graphs
using Billboard.js.}
\end{figure}

This portlet showcases the power of graphing by displaying a set of
default charts and a more advanced custom chart. These are all built
using Billboard.js.

\noindent\hrulefill

\textbf{Note:} The minifier fails on Liferay DXP 7.0 when JSDoc is
present in a portlet. To resolve this, use
\href{https://gruntjs.com/getting-started}{Grunt}
\href{https://www.npmjs.com/package/grunt-contrib-uglify}{uglify} to
remove the JSDoc comments. This process may take a long time, depending
on the number of files that require an update.

\noindent\hrulefill

\textbf{Important:} This sample works for Liferay DXP 7.0 Fix Pack 44+
and Liferay Portal CE GA7+.

\subsection{What API(s) and/or code components does this sample
highlight?}\label{what-apis-andor-code-components-does-this-sample-highlight-3}

This sample leverages the
\href{/docs/7-0/tutorials/-/knowledge_base/t/using-npm-in-your-portlets}{npm
development workflow support}.

\subsection{How does this sample leverage the API(s) and/or code
component?}\label{how-does-this-sample-leverage-the-apis-andor-code-component-3}

This sample uses the \href{https://www.npmjs.com/}{npm registry} to
download project dependencies and uses the
\href{https://github.com/liferay/liferay-npm-build-tools/tree/master/packages/liferay-npm-bundler}{liferay-npm-bundler
tool} to bundle the project dependencies inside the OSGi bundle JAR
file.

To accomplish the bundling, the project's build process relies on a
\texttt{build} script inside its \texttt{package.json} file:

\begin{verbatim}
"scripts": {
    "build": "babel --source-maps -d build/resources/main/META-INF/resources src/main/resources/META-INF/resources && liferay-npm-bundler"
},
\end{verbatim}

\subsection{Where Is This Sample?}\label{where-is-this-sample-3}

This sample is built with the following build tool:

\begin{itemize}
\tightlist
\item
  \href{https://github.com/liferay/liferay-blade-samples/tree/7.0/gradle/apps/npm/billboardjs-npm-portlet}{Gradle}
\end{itemize}

\section{Isomorphic npm Portlet}\label{isomorphic-npm-portlet}

The Isomorphic npm Portlet sample provides a portlet that uses
\href{https://en.wikipedia.org/wiki/Isomorphic_JavaScript}{isomorphic}
code (i.e., can run from client and/or server side) on the client side.

\begin{figure}
\centering
\includegraphics{./images/isomorphic-npm-sample.png}
\caption{This sample portlet displays the results of running code
designed for the server in the browser.}
\end{figure}

This portlet showcases running code designed to execute in the server in
the browser. Note that this portlet does \textbf{not} run JavaScript
code in the server; it's executing isomorphic JavaScript code in the
browser.

\noindent\hrulefill

\textbf{Note:} The minifier fails on Liferay DXP 7.0 when JSDoc is
present in a portlet. To resolve this, use
\href{https://gruntjs.com/getting-started}{Grunt}
\href{https://www.npmjs.com/package/grunt-contrib-uglify}{uglify} to
remove the JSDoc comments. This process may take a long time, depending
on the number of files that require an update.

\noindent\hrulefill

\textbf{Important:} This sample works for Liferay DXP 7.0 Fix Pack 44+
and Liferay Portal CE GA7+.

\subsection{What API(s) and/or code components does this sample
highlight?}\label{what-apis-andor-code-components-does-this-sample-highlight-4}

This sample leverages the
\href{/docs/7-0/tutorials/-/knowledge_base/t/using-npm-in-your-portlets}{npm
development workflow support}.

You can do many things with isomorphic code. You can run it in

\begin{itemize}
\tightlist
\item
  the server only (e.g., Node.js)
\item
  the client only (e.g., browser)
\item
  both the server and client (e.g., Node.js + browser)
\end{itemize}

Isomorphic code cannot run server-side because Liferay DXP is Java based
and cannot execute JavaScript that way. This sample portlet shows how
Liferay's npm bundler can transform server-side code to make it work in
the client (e.g., emulates some of Node.js' APIs in the client).

\subsection{How does this sample leverage the API(s) and/or code
component?}\label{how-does-this-sample-leverage-the-apis-andor-code-component-4}

This sample uses the \href{https://www.npmjs.com/}{npm registry} to
download project dependencies and uses the
\href{https://github.com/liferay/liferay-npm-build-tools/tree/master/packages/liferay-npm-bundler}{liferay-npm-bundler
tool} to bundle the project dependencies inside the OSGi bundle JAR
file.

To accomplish the bundling, the project's build process relies on a
\texttt{build} script inside its \texttt{package.json} file:

\begin{verbatim}
"scripts": {
    "build": "babel --source-maps -d build/resources/main/META-INF/resources src/main/resources/META-INF/resources && liferay-npm-bundler"
},
\end{verbatim}

\subsection{Where Is This Sample?}\label{where-is-this-sample-4}

This sample is built with the following build tool:

\begin{itemize}
\tightlist
\item
  \href{https://github.com/liferay/liferay-blade-samples/tree/7.0/gradle/apps/npm/isomorphic-npm-portlet}{Gradle}
\end{itemize}

\section{jQuery npm Portlet}\label{jquery-npm-portlet}

The jQuery npm Portlet sample provides a portlet that uses the
\href{https://jquery.com/}{jQuery} framework to render its output.

\begin{figure}
\centering
\includegraphics{./images/jquery-npm-sample.png}
\caption{Clicking on the portlet's hand symbol displays a message.}
\end{figure}

This portlet showcases the fast HTML document traversal jQuery offers.

\noindent\hrulefill

\textbf{Note:} The minifier fails on Liferay DXP 7.0 when JSDoc is
present in a portlet. To resolve this, use
\href{https://gruntjs.com/getting-started}{Grunt}
\href{https://www.npmjs.com/package/grunt-contrib-uglify}{uglify} to
remove the JSDoc comments. This process may take a long time, depending
on the number of files that require an update.

\noindent\hrulefill

\textbf{Important:} This sample works for Liferay DXP 7.0 Fix Pack 44+
and Liferay Portal CE GA7+.

\subsection{What API(s) and/or code components does this sample
highlight?}\label{what-apis-andor-code-components-does-this-sample-highlight-5}

This sample leverages the
\href{/docs/7-0/tutorials/-/knowledge_base/t/using-npm-in-your-portlets}{npm
development workflow support}.

\subsection{How does this sample leverage the API(s) and/or code
component?}\label{how-does-this-sample-leverage-the-apis-andor-code-component-5}

This sample uses the \href{https://www.npmjs.com/}{npm registry} to
download project dependencies and uses the
\href{https://github.com/liferay/liferay-npm-build-tools/tree/master/packages/liferay-npm-bundler}{liferay-npm-bundler
tool} to bundle the project dependencies inside the OSGi bundle JAR
file.

To accomplish the bundling, the project's build process relies on a
\texttt{build} script inside its \texttt{package.json} file:

\begin{verbatim}
"scripts": {
    "build": "babel --source-maps -d build/resources/main/META-INF/resources src/main/resources/META-INF/resources && liferay-npm-bundler"
},
\end{verbatim}

\subsection{Where Is This Sample?}\label{where-is-this-sample-5}

This sample is built with the following build tool:

\begin{itemize}
\tightlist
\item
  \href{https://github.com/liferay/liferay-blade-samples/tree/7.0/gradle/apps/npm/jquery-npm-portlet}{Gradle}
\end{itemize}

\section{Metal.js npm Portlet}\label{metal.js-npm-portlet}

The Metal.js npm Portlet sample provides a portlet that uses the
\href{https://metaljs.com/}{Metal.js} framework to render its output.

\begin{figure}
\centering
\includegraphics{./images/metaljs-npm-sample.png}
\caption{Clicking the button returns displays a dialog window.}
\end{figure}

This portlet displays a Metal.js based dialog that has been rendered
using SOY templates.

\noindent\hrulefill

\textbf{Note:} The minifier fails on Liferay DXP 7.0 when JSDoc is
present in a portlet. To resolve this, use
\href{https://gruntjs.com/getting-started}{Grunt}
\href{https://www.npmjs.com/package/grunt-contrib-uglify}{uglify} to
remove the JSDoc comments. This process may take a long time, depending
on the number of files that require an update.

\noindent\hrulefill

\textbf{Important:} This sample works for Liferay DXP 7.0 Fix Pack 44+
and Liferay Portal CE GA7+.

\subsection{What API(s) and/or code components does this sample
highlight?}\label{what-apis-andor-code-components-does-this-sample-highlight-6}

This sample leverages the
\href{/docs/7-0/tutorials/-/knowledge_base/t/using-npm-in-your-portlets}{npm
development workflow support}.

\subsection{How does this sample leverage the API(s) and/or code
component?}\label{how-does-this-sample-leverage-the-apis-andor-code-component-6}

This sample uses the \href{https://www.npmjs.com/}{npm registry} to
download project dependencies and uses the
\href{https://github.com/liferay/liferay-npm-build-tools/tree/master/packages/liferay-npm-bundler}{liferay-npm-bundler
tool} to bundle the project dependencies inside the OSGi bundle JAR
file.

To accomplish the bundling, the project's build process relies on a
\texttt{build} script inside its \texttt{package.json} file:

\begin{verbatim}
"scripts": {
    "build": "metalsoy && babel --source-maps -d build/resources/main/META-INF/resources src/main/resources/META-INF/resources && liferay-npm-bundler"
},
\end{verbatim}

\subsection{Where Is This Sample?}\label{where-is-this-sample-6}

This sample is built with the following build tool:

\begin{itemize}
\tightlist
\item
  \href{https://github.com/liferay/liferay-blade-samples/tree/7.0/gradle/apps/npm/metaljs-npm-portlet}{Gradle}
\end{itemize}

\section{React npm Portlet}\label{react-npm-portlet}

The React npm Portlet sample provides a portlet that uses the
\href{https://reactjs.org/}{React} framework to render its output.

\begin{figure}
\centering
\includegraphics{./images/react-npm-sample.png}
\caption{You can play the game Tic-tac-toe with this sample portlet.}
\end{figure}

This portlet showcases the how efficiently React can render components
based on user interaction.

\noindent\hrulefill

\textbf{Note:} The minifier fails on Liferay DXP 7.0 when JSDoc is
present in a portlet. To resolve this, use
\href{https://gruntjs.com/getting-started}{Grunt}
\href{https://www.npmjs.com/package/grunt-contrib-uglify}{uglify} to
remove the JSDoc comments. This process may take a long time, depending
on the number of files that require an update.

\noindent\hrulefill

\textbf{Important:} This sample works for Liferay DXP 7.0 Fix Pack 44+
and Liferay Portal CE GA7+.

\subsection{What API(s) and/or code components does this sample
highlight?}\label{what-apis-andor-code-components-does-this-sample-highlight-7}

This sample leverages the
\href{/docs/7-0/tutorials/-/knowledge_base/t/using-npm-in-your-portlets}{npm
development workflow support}.

\subsection{How does this sample leverage the API(s) and/or code
component?}\label{how-does-this-sample-leverage-the-apis-andor-code-component-7}

This sample uses the \href{https://www.npmjs.com/}{npm registry} to
download project dependencies and uses the
\href{https://github.com/liferay/liferay-npm-build-tools/tree/master/packages/liferay-npm-bundler}{liferay-npm-bundler
tool} to bundle the project dependencies inside the OSGi bundle JAR
file.

To accomplish the bundling, the project's build process relies on a
\texttt{build} script inside its \texttt{package.json} file:

\begin{verbatim}
"scripts": {
    "build": "babel --source-maps -d build/resources/main/META-INF/resources src/main/resources/META-INF/resources && liferay-npm-bundler"
},
\end{verbatim}

\subsection{Where Is This Sample?}\label{where-is-this-sample-7}

This sample is built with the following build tool:

\begin{itemize}
\tightlist
\item
  \href{https://github.com/liferay/liferay-blade-samples/tree/7.0/gradle/apps/npm/react-npm-portlet}{Gradle}
\end{itemize}

\section{Simple npm Portlet}\label{simple-npm-portlet}

The Simple npm Portlet sample provides a portlet that uses the
\href{https://www.npmjs.com/package/isarray}{isarray npm package} when
rendering its output.

\begin{figure}
\centering
\includegraphics{./images/simple-npm-sample.png}
\caption{The portlet's status and actions are displayed as output.}
\end{figure}

\noindent\hrulefill

\textbf{Note:} The minifier fails on Liferay DXP 7.0 when JSDoc is
present in a portlet. To resolve this, use
\href{https://gruntjs.com/getting-started}{Grunt}
\href{https://www.npmjs.com/package/grunt-contrib-uglify}{uglify} to
remove the JSDoc comments. This process may take a long time, depending
on the number of files that require an update.

\noindent\hrulefill

\textbf{Important:} This sample works for Liferay DXP 7.0 Fix Pack 44+
and Liferay Portal CE GA7+.

\subsection{What API(s) and/or code components does this sample
highlight?}\label{what-apis-andor-code-components-does-this-sample-highlight-8}

This sample leverages the
\href{/docs/7-0/tutorials/-/knowledge_base/t/using-npm-in-your-portlets}{npm
development workflow support}.

\subsection{How does this sample leverage the API(s) and/or code
component?}\label{how-does-this-sample-leverage-the-apis-andor-code-component-8}

This sample uses the \href{https://www.npmjs.com/}{npm registry} to
download project dependencies and uses the
\href{https://github.com/liferay/liferay-npm-build-tools/tree/master/packages/liferay-npm-bundler}{liferay-npm-bundler
tool} to bundle the project dependencies inside the OSGi bundle JAR
file.

To accomplish the bundling, the project's build process relies on a
\texttt{build} script inside its \texttt{package.json} file:

\begin{verbatim}
"scripts": {
    "build": "babel --source-maps -d build/resources/main/META-INF/resources src/main/resources/META-INF/resources && liferay-npm-bundler"
},
\end{verbatim}

\subsection{Where Is This Sample?}\label{where-is-this-sample-8}

This sample is built with the following build tool:

\begin{itemize}
\tightlist
\item
  \href{https://github.com/liferay/liferay-blade-samples/tree/7.0/gradle/apps/npm/simple-npm-portlet}{Gradle}
\end{itemize}

\section{Vue.js npm Portlet}\label{vue.js-npm-portlet}

The Vue.js npm Portlet sample provides a portlet that uses the
\href{https://vuejs.org/}{Vue.js} framework to render its output.

\begin{figure}
\centering
\includegraphics{./images/vuejs-npm-sample.png}
\caption{Clicking the portlet's button reverses the message.}
\end{figure}

This portlet showcases Vue.js's speed and performance when rendering a
user interface.

\noindent\hrulefill

\textbf{Note:} The minifier fails on Liferay DXP 7.0 when JSDoc is
present in a portlet. To resolve this, use
\href{https://gruntjs.com/getting-started}{Grunt}
\href{https://www.npmjs.com/package/grunt-contrib-uglify}{uglify} to
remove the JSDoc comments. This process may take a long time, depending
on the number of files that require an update.

\noindent\hrulefill

\textbf{Important:} This sample works for Liferay DXP 7.0 Fix Pack 44+
and Liferay Portal CE GA7+.

\subsection{What API(s) and/or code components does this sample
highlight?}\label{what-apis-andor-code-components-does-this-sample-highlight-9}

This sample leverages the
\href{/docs/7-0/tutorials/-/knowledge_base/t/using-npm-in-your-portlets}{npm
development workflow support}.

\subsection{How does this sample leverage the API(s) and/or code
component?}\label{how-does-this-sample-leverage-the-apis-andor-code-component-9}

This sample uses the \href{https://www.npmjs.com/}{npm registry} to
download project dependencies and uses the
\href{https://github.com/liferay/liferay-npm-build-tools/tree/master/packages/liferay-npm-bundler}{liferay-npm-bundler
tool} to bundle the project dependencies inside the OSGi bundle JAR
file.

To accomplish the bundling, the project's build process relies on a
\texttt{build} script inside its \texttt{package.json} file:

\begin{verbatim}
"scripts": {
    "build": "babel --source-maps -d build/resources/main/META-INF/resources src/main/resources/META-INF/resources && liferay-npm-bundler"
},
\end{verbatim}

\subsection{Where Is This Sample?}\label{where-is-this-sample-9}

This sample is built with the following build tool:

\begin{itemize}
\tightlist
\item
  \href{https://github.com/liferay/liferay-blade-samples/tree/7.0/gradle/apps/npm/vuejs-npm-portlet}{Gradle}
\end{itemize}

\chapter{Service Builder Samples}\label{service-builder-samples}

This section focuses on Liferay Service Builder sample projects built
with various build tools. You can view these samples by visiting the
\texttt{apps/service-builder} folder corresponding to your preferred
build tool:

\begin{itemize}
\tightlist
\item
  \href{https://github.com/liferay/liferay-blade-samples/tree/7.0/gradle/apps/service-builder}{Gradle
  Service Builder sample apps}
\item
  \href{https://github.com/liferay/liferay-blade-samples/tree/7.0/liferay-workspace/apps/service-builder}{Liferay
  Service Builder Workspace sample apps}
\item
  \href{https://github.com/liferay/liferay-blade-samples/tree/7.0/maven/apps/service-builder}{Maven
  Service Builder sample apps}
\end{itemize}

The following Service Builder samples are documented:

\begin{itemize}
\tightlist
\item
  \href{/docs/7-0/reference/-/knowledge_base/r/service-builder-application-demonstrating-actionable-dynamic-query}{Service
  Builder application demonstrating Actionable Dynamic Query}
\item
  \href{/docs/7-0/reference/-/knowledge_base/r/service-builder-application-using-external-database-via-jdbc}{Service
  Builder application with JDBC connection}
\item
  \href{/docs/7-0/reference/-/knowledge_base/r/service-builder-application-using-external-database-via-jndi}{Service
  Builder application with JNDI connection}
\end{itemize}

Visit a particular sample page to learn more!

\section{Service Builder Application Demonstrating Actionable Dynamic
Query}\label{service-builder-application-demonstrating-actionable-dynamic-query}

This sample is similar to the
\href{https://github.com/liferay/liferay-blade-samples/tree/7.0/gradle/apps/service-builder/basic}{\texttt{basic}
Service Builder sample}, which lets you perform CRUD (create, read,
update, delete) operations on service builder entities. The distinctive
feature of the Service Builder Actionable Dynamic Query (ADQ) sample is
that it also lets you perform a mass update on all existing service
builder entities.

\begin{figure}
\centering
\includegraphics{./images/adq-sample.png}
\caption{This sample provides options to add entities and perform a mass
update.}
\end{figure}

To see the ADQ Service Builder sample in action, complete the following
steps:

\begin{enumerate}
\def\labelenumi{\arabic{enumi}.}
\item
  Add the sample to a page by navigating to \emph{Add}
  (\includegraphics{./images/icon-control-menu-add.png}) →
  \emph{Applications} → \emph{Sample} and dragging it to the page.
\item
  Select the app's \emph{Add} button and add an entity. Do this several
  times to create multiple entities.
\item
  Click the \emph{Mass Update} button and click \emph{Save} to invoke
  the update.

  \begin{figure}
  \centering
  \includegraphics{./images/adq-sample-mass-update.png}
  \caption{Clicking the \emph{Save} button executes the mass update.}
  \end{figure}

  After invoking the update, each entity's \texttt{field3} value (whose
  value is less than 100) is incremented.
\end{enumerate}

You've leveraged the actionable dynamic query API in your sample!

\subsection{What API(s) and/or code components does this sample
highlight?}\label{what-apis-andor-code-components-does-this-sample-highlight-10}

This sample demonstrates Liferay DXP's actionable dynamic query API.
Specifically, it demonstrates how to create an ADQ, add criteria to an
ADQ, specify an action for the ADQ, and execute the ADQ.

\subsection{How does this sample leverage the API(s) and/or code
component?}\label{how-does-this-sample-leverage-the-apis-andor-code-component-10}

An action request is sent to the \texttt{JSPPortlet} with a \texttt{cmd}
request parameter. When the \texttt{JSPPortlet}'s \texttt{processAction}
method processes the request, the value of the \texttt{cmd} parameter is
parsed and then the portlet's \texttt{massUpdate} method is invoked. The
\texttt{massUpdate} method, in turn, invokes the \texttt{massUpdate}
method defined in the \texttt{adq-service} module's
\texttt{BarLocalServiceImpl}. This is where the sample leverages the
actionable dynamic query API:

\begin{verbatim}
public void massUpdate() {
    ActionableDynamicQuery adq = getActionableDynamicQuery();

    adq.setAddCriteriaMethod(
        new ActionableDynamicQuery.AddCriteriaMethod() {

            @Override
            public void addCriteria(DynamicQuery dynamicQuery) {
                dynamicQuery.add(RestrictionsFactoryUtil.lt("field3", 100));
            }

        });

    adq.setPerformActionMethod(
        new ActionableDynamicQuery.PerformActionMethod<Bar>() {

            @Override
            public void performAction(Bar bar) {
                int field3 = bar.getField3();

                field3++;
                bar.setField3(field3);

                updateBar(bar);
            }

        });

    try {
        adq.performActions();
    }
    catch (Exception e) {
        e.printStackTrace();
    }
}
\end{verbatim}

For more information on the actionable dynamic query API, visit its
dedicated
\href{develop/tutorials/-/knowledge_base/7-0/dynamic-query\#actionable-dynamic-queries}{tutorial}.

\section{Service Builder Application Using External Database via
JDBC}\label{service-builder-application-using-external-database-via-jdbc}

This sample demonstrates how to connect a Liferay Service Builder
application to an external database via a JDBC connection. Here, an
external database means any database other than Liferay DXP's database.
For this sample to work correctly, you must prepare such an external
database and configure Liferay DXP to use it. Follow the steps below to
make the required preparations before deploying the application.

\begin{enumerate}
\def\labelenumi{\arabic{enumi}.}
\item
  Create the external database to which your Service Builder application
  will connect. For example, create a MariaDB database called
  \texttt{external}. Add a table to this database called
  \texttt{country} with a \texttt{BIGINT} column called \texttt{Id} and
  a \texttt{VARCHAR(255)} column called \texttt{Name}. Add at least one
  record to this table. Here are the MariaDB commands to accomplish
  this:

\begin{verbatim}
create database external character set utf8;

use external;

create table country(id bigint not null primary key, name varchar(255));

insert into country(id, name) values(1, 'Australia');
\end{verbatim}

  Make sure that your database commands were successful: Running
  \texttt{select\ *\ from\ country;} should return the record you added.
\item
  Create a \texttt{portal-ext.properties} file in your Liferay DXP
  instance's \texttt{{[}LIFERAY\_HOME{]}} folder (this folder should be
  marked by the presence of a \texttt{.liferay-home} file). In your
  \texttt{portal-ext.properties} file, define the details of your JDBC
  data source connection:

\begin{verbatim}
jdbc.ext.driverClassName=org.mariadb.jdbc.Driver
jdbc.ext.password=userpassword
jdbc.ext.url=jdbc:mariadb://localhost/external?useUnicode=true&characterEncoding=UTF-8&useFastDateParsing=false
jdbc.ext.username=yourusername
\end{verbatim}

  Note that Liferay DXP's primary data source is specified by the
  \texttt{jdbc.default} prefix. These details are often specified in a
  \texttt{portal-setup-wizard.properties} file. Here, we've chosen to
  use the \texttt{jdbc.ext} prefix for our alternate data source.
\item
  Create a
  \texttt{com.liferay.blade.samples.jdbcservicebuilder.service-log4j-ext.xml}
  in your Liferay instance's \texttt{{[}LIFERAY\_HOME{]}/osgi/log4}
  folder. Create this folder if it doesn't yet exist. Add this content
  to the XML file that you created:

\begin{verbatim}
<?xml version="1.0"?>
<!DOCTYPE log4j:configuration SYSTEM "log4j.dtd">

<log4j:configuration xmlns:log4j="http://jakarta.apache.org/log4j/">
    <category name="com.liferay.blade.samples.jdbcservicebuilder.service.impl">
        <priority value="INFO" />
    </category>
</log4j:configuration>
\end{verbatim}

  This XML file defines the log level for the classes in the
  \texttt{com.liferay.blade.samples.jdbcservicebuilder.service.impl}
  package. The
  \texttt{com.liferay.blade.samples.jdbcservicebuilder.service.impl.CountryLocalServiceImpl}
  is the class that will produce log messages when the sample portlet is
  viewed.
\end{enumerate}

Now your sample is ready for deployment! Make sure to build and deploy
each of the three modules that comprise the sample application:

\begin{itemize}
\tightlist
\item
  \texttt{jdbc-api}
\item
  \texttt{jdbc-service}
\item
  \texttt{jdbc-web}
\end{itemize}

After these modules have been deployed, add the \texttt{-web} portlet to
a Liferay DXP page.

\begin{figure}
\centering
\includegraphics{./images/jdbc-sb-sample.png}
\caption{This sample prints out the values previously inputted into the
database.}
\end{figure}

A sample table is printed in the portlet's view, representing the info
inputted into the database.

\subsection{What API(s) and/or code components does this sample
highlight?}\label{what-apis-andor-code-components-does-this-sample-highlight-11}

This sample demonstrates two ways to access data from an external
database defined by a JDBC connection:

\begin{itemize}
\tightlist
\item
  extract data directly from the raw data source by explicitly
  specifying a SQL query.
\item
  read data using the helper methods that Service Builder generates in
  your application's persistence layer.
\end{itemize}

\subsection{How does this sample leverage the API(s) and/or code
component?}\label{how-does-this-sample-leverage-the-apis-andor-code-component-11}

Once you've added the \texttt{-web} portlet to a page, the
\texttt{CountryLocalService.useJDBC} method is invoked. This method
accesses the database defined by the JDBC connection you specified and
logs information about the rows in the \texttt{country} table to Liferay
DXP's log.

The first way of accessing data from the external database is to extract
it directly from the raw data source by explicitly specifying a SQL
query. This technique is demonstrated by the
\texttt{CountryLocalServiceImpl.useJDBC} method. That method obtains the
Spring-defined data source that's injected into the
\texttt{countryPersistence} bean, opens a new connection, and reads data
from the data source. This is the technique used by the sample
application to write the data to Liferay DXP's log.

The second way of accessing data from the external database is to read
data using the helper methods that Service Builder generates in your
application's persistence layer. This technique is demonstrated by the
\texttt{UseJDBC.getCountries} method which first obtains an instance of
the \texttt{CountryLocalService} OSGi service and then invokes
\texttt{countryLocalService.getCountries}. The
\texttt{countryLocalService.getCountries} and
\texttt{countryLocalService.getCountriesCount} methods are two examples
of the persistence layer helper methods that Service Builder generates.
This is the technique used by the sample application to actually display
the data. The portlet's \texttt{view.jsp} uses the
\texttt{\textless{}search-container\textgreater{}} JSP tag to display a
list of results. The results are obtained by the
\texttt{UseJDBC.getCountries} method mentioned above.

\section{Service Builder Application Using External Database via
JNDI}\label{service-builder-application-using-external-database-via-jndi}

This sample demonstrates how to connect a Liferay Service Builder
application to an external database via a JNDI connection. Here, an
external database means any database other than Liferay DXP's database.
For this sample to work correctly, you must prepare such an external
database and configure Liferay DXP to use it. Follow the steps below to
make the required preparations before deploying the application.

\begin{enumerate}
\def\labelenumi{\arabic{enumi}.}
\item
  Create the external database to which your Service Builder application
  will connect. For example, create a MariaDB database called
  \texttt{external}. Add a table to this database called \texttt{region}
  with a \texttt{BIGINT} column called \texttt{Id} and a
  \texttt{VARCHAR(255)} column called \texttt{Name}. Add at least one
  record to this table. Here are the MariaDB commands to accomplish
  this:

\begin{verbatim}
create database external character set utf8;

use external;

create table region(id bigint not null primary key, name varchar(255));

insert into region(id, name) values(1, 'Tasmania');
\end{verbatim}

  Make sure that your database commands were successful: Running
  \texttt{select\ *\ from\ region;} should return the record you added.
\item
  Now you need to define a JNDI connection to your database. The way
  this is done depends on your application server. Here we demonstrate
  how to specify the JNDI connection for Tomcat. First, edit your
  \texttt{{[}LIFERAY\_HOME{]}/tomcat-8.0.32/conf/server.xml} file and
  add this resource element inside of the
  \texttt{\textless{}GlobalNamingResources\textgreater{}} element:

\begin{verbatim}
<Resource
    name="jdbc/externalDataSource"
    auth="Container"
    type="javax.sql.DataSource"
    factory="org.apache.tomcat.jdbc.pool.DataSourceFactory"
    driverClassName="org.mariadb.jdbc.Driver"
    url="jdbc:mariadb://localhost/external"
    username="yourusername"
    password="yourpassword"
    maxActive="20"
    maxIdle="5"
    maxWait="10000"
/>
\end{verbatim}

  Replace the specified username and password with the correct values
  for your database.
\item
  Edit your \texttt{{[}LIFERAY\_HOME{]}/tomcat-8.0.32/conf/context.xml}
  file and add this resource link element inside of the
  \texttt{\textless{}Context\textgreater{}} element:

\begin{verbatim}
<ResourceLink name="jdbc/externalDataSource" global="jdbc/externalDataSource" type="javax.sql.DataSource"/>
\end{verbatim}

  Now your data source is defined at Tomcat's scope.
\item
  Create a
  \texttt{com.liferay.blade.samples.jndiservicebuilder.service-log4j-ext.xml}
  in your Liferay DXP instance's \texttt{{[}LIFERAY\_HOME{]}/osgi/log4}
  folder. Create this folder if it doesn't yet exist. Add this content
  to the XML file that you created:

\begin{verbatim}
<?xml version="1.0"?>
<!DOCTYPE log4j:configuration SYSTEM "log4j.dtd">

<log4j:configuration xmlns:log4j="http://jakarta.apache.org/log4j/">
    <category name="com.liferay.blade.samples.jndiservicebuilder.service.impl">
        <priority value="INFO" />
    </category>
</log4j:configuration>
\end{verbatim}

  This XML file defines the log level for the classes in the
  \texttt{com.liferay.blade.samples.jndiservicebuilder.service.impl}
  package. The
  \texttt{com.liferay.blade.samples.jndiservicebuilder.service.impl.RegionLocalServiceImpl}
  is the class that will produce log messages when the sample portlet is
  viewed.
\end{enumerate}

Now your sample is ready for deployment! Make sure to build and deploy
each of the three modules that comprise the sample application:

\begin{itemize}
\tightlist
\item
  \texttt{jndi-api}
\item
  \texttt{jndi-service}
\item
  \texttt{jndi-web}
\end{itemize}

After these modules have been deployed, add the \texttt{jndi-web}
portlet to a Liferay DXP page.

\begin{figure}
\centering
\includegraphics{./images/jndi-sb-sample.png}
\caption{This sample prints out the values previously inputted into the
database.}
\end{figure}

A sample table is printed in the portlet's view, representing the info
inputted into the database.

\subsection{What API(s) and/or code components does this sample
highlight?}\label{what-apis-andor-code-components-does-this-sample-highlight-12}

This sample demonstrates two ways to access data from an external
database defined by a JNDI connection:

\begin{itemize}
\tightlist
\item
  extract data directly from the raw data source by explicitly
  specifying a SQL query.
\item
  read data using the helper methods that Service Builder generates in
  your application's persistence layer.
\end{itemize}

\subsection{How does this sample leverage the API(s) and/or code
component?}\label{how-does-this-sample-leverage-the-apis-andor-code-component-12}

Once you've added the \texttt{jndi-web} portlet to a page, the
\texttt{RegionLocalServiceUtil.useJNDI} method is invoked. This method
accesses the database defined by the JNDI connection you specified and
logs information about the rows in the \texttt{region} table to Liferay
DXP's log.

The first way of accessing data from the external database is to extract
data directly from the raw data source by explicitly specifying a SQL
query. This technique is demonstrated by the
\texttt{RegionLocalServiceImpl.useJNDI} method. That method obtains the
Spring-defined data source that's injected into the
\texttt{regionPersistence} bean, opens a new connection, and reads data
from the data source. This is the technique used by the sample
application to write the data to Liferay DXP's log.

The second way of accessing data from the external database is to read
data using the helper methods that Service Builder generates in your
application's persistence layer. This technique is demonstrated by the
\texttt{UseJNDI.getRegions} method which first obtains an instance of
the \texttt{RegionLocalService} OSGi service and then invokes
\texttt{regionLocalService.getRegions}. The
\texttt{regionLocalService.getRegions} and
\texttt{regionLocalService.getRegionsCount} methods are two examples of
the persistence layer helper methods that Service Builder generates.
This is the technique used by the sample application to actually display
the data. The portlet's \texttt{view.jsp} uses the
\texttt{\textless{}search-container\textgreater{}} JSP tag to display a
list of results. The results are obtained by the
\texttt{UseJNDI.getRegions} method mentioned above.

\section{Shared Language Keys}\label{shared-language-keys}

The Shared Language Keys sample provides a JSP portlet that displays
language keys.

\begin{figure}
\centering
\includegraphics{./images/language-web-portlet.png}
\caption{The sample JSP portlet displays three language keys.}
\end{figure}

The language keys displayed in the portlet come from two different
modules.

\subsection{What API(s) and/or code components does this sample
highlight?}\label{what-apis-andor-code-components-does-this-sample-highlight-13}

This sample is broken into two modules:

\begin{itemize}
\tightlist
\item
  \texttt{language}
\item
  \texttt{language-web}
\end{itemize}

The \texttt{language-web} module provides a JSP portlet with unique
language keys that it displays. The \texttt{language} module provides a
resource module which only holds language keys. Its sole purpose is to
share language keys with the JSP portlet provided in
\texttt{language-web}. This sample conveys Liferay's recommended
approach to sharing language keys through OSGi services.

\subsection{How does this sample leverage the API(s) and/or code
component?}\label{how-does-this-sample-leverage-the-apis-andor-code-component-13}

You must deploy both \texttt{language-web} and \texttt{language} modules
to simulate this sample's targeted demonstration.

First, note the language keys provided by each module:

\begin{itemize}
\tightlist
\item
  \texttt{language-web}

  \begin{itemize}
  \tightlist
  \item
    \texttt{blade\_language\_web\_LanguageWebPortlet.caption=Hello\ from\ BLADE\ Language\ Web!}
  \item
    \texttt{blade\_language\_web\_override\_LanguageWebPortlet.caption=I\ have\ overridden\ the\ key\ from\ BLADE\ Language\ Module!}
  \end{itemize}
\item
  \texttt{language}

  \begin{itemize}
  \tightlist
  \item
    \texttt{blade\_language\_LanguageWebPortlet.caption=Hello\ from\ the\ BLADE\ Language\ Module!}
  \item
    \texttt{blade\_language\_web\_override\_LanguageWebPortlet.caption=Hello\ from\ the\ BLADE\ Language\ Module\ but\ you\ won\textquotesingle{}t\ see\ me!}
  \end{itemize}
\end{itemize}

When you place the sample BLADE Language Web portlet on a Liferay DXP
page, you're presented with three language keys.

\begin{figure}
\centering
\includegraphics{./images/shared-language-keys.png}
\caption{The Language Web portlet displays three phrases, two of which
are shared from a different module.}
\end{figure}

The first message is provided by the \texttt{language-web} module. The
second message is from the \texttt{language} module. The third message
is provided by both modules; as you can see, the \texttt{language-web}'s
message is used, overriding the \texttt{language} module's identically
named language key.

This sample shows what takes precedence when displaying language keys.
The order for this example goes

\begin{enumerate}
\def\labelenumi{\arabic{enumi}.}
\tightlist
\item
  \texttt{language-web} module language keys
\item
  \texttt{language} module language keys
\item
  Liferay DXP language keys
\end{enumerate}

So how does sharing language keys work?

By default, the \texttt{ResourceBundleLoaderAnalyzerPlugin} expands
modules with \texttt{/content/Language.properties} files to add provided
capabilities:

\begin{itemize}
\tightlist
\item
  \texttt{bundle.symbolic.name}
\item
  \texttt{resource.bundle.base.name}
\end{itemize}

Then the deployed \texttt{LanguageExtender} scans modules with those
capabilities to automatically register an associated
\texttt{ResourceBundleLoader}.

You can leverage this functionality to use keys from common language
modules by republishing an aggregate \texttt{ResourceBundleLoader}. This
can be done two ways:

\begin{enumerate}
\def\labelenumi{\arabic{enumi}.}
\item
  Via Components

  You can get a reference to the registered service in your components
  as detailed in the
  \href{/docs/7-0/tutorials/-/knowledge_base/t/overriding-language-keys\#overriding-a-modules-language-keys}{Overriding
  a Module's Language Keys} tutorial. The main disadvantage of this
  approach is that it forces you to provide a specific implementation of
  the \texttt{ResourceBundleLoader}, making it harder to modularize in
  the future.
\item
  Via Provide Capability

  The same \texttt{LanguageExtender} that registers the services
  supports an extended syntax that lets you register an aggregate of a
  collection of bundles:

\begin{verbatim}
 -liferay-aggregate-resource-bundles: \
     blade.language
\end{verbatim}

  This approach has the advantage of easier extensibility. When language
  keys change, only the common language modules must be built and
  redeployed for the modules referencing them to recognize their
  updates.
\end{enumerate}

For more information on sharing language keys, visit the
\href{/docs/7-0/tutorials/-/knowledge_base/t/internationalization}{Internationalization}
tutorials.

\subsection{Where Is This Sample?}\label{where-is-this-sample-10}

There are three different versions of this sample, each built with a
different build tool:

\begin{itemize}
\tightlist
\item
  \href{https://github.com/liferay/liferay-blade-samples/tree/7.0/gradle/apps/shared-language-keys}{Gradle}
\item
  \href{https://github.com/liferay/liferay-blade-samples/tree/7.0/liferay-workspace/apps/shared-language-keys}{Liferay
  Workspace}
\item
  \href{https://github.com/liferay/liferay-blade-samples/tree/7.0/maven/apps/shared-language-keys}{Maven}
\end{itemize}

\section{Simulation Panel App}\label{simulation-panel-app}

The Simulation Panel App provides new functionality in @product's
Simulation Menu. When deploying this sample with no customizations, the
\emph{Simulation Sample} feature is provided in the Simulation Menu with
four options.

\begin{figure}
\centering
\includegraphics{./images/simulation-panel-app.png}
\caption{A simulation panel app adds new functionality to the Simulation
Menu.}
\end{figure}

\subsection{What API(s) and/or code components does this sample
highlight?}\label{what-apis-andor-code-components-does-this-sample-highlight-14}

This sample leverages the
\href{@app-ref@/web-experience/latest/javadocs/com/liferay/application/list/PanelApp.html}{PanelApp}
API.

\subsection{How does this sample leverage the API(s) and/or code
component?}\label{how-does-this-sample-leverage-the-apis-andor-code-component-14}

This sample leverages the \texttt{PanelApp} interface as an OSGi service
via the \texttt{@Component} annotation:

\begin{verbatim}
@Component(
    immediate = true,
    property = {
        "panel.app.order:Integer=500",
        "panel.category.key=" + SimulationPanelCategory.SIMULATION
    },
    service = PanelApp.class
)
\end{verbatim}

There are also two properties provided via the \texttt{@Component}
annotation:

\begin{itemize}
\tightlist
\item
  \texttt{panel.app.order}: the order in which the panel app is
  displayed among other panel apps in the chosen category. Entries are
  ordered from top to bottom. For example, an entry with order
  \texttt{1} will be listed above an entry with order \texttt{2}. If the
  order is not specified, it's chosen at random based on which service
  was registered first in the OSGi container.
\item
  \texttt{panel.category.key}: the host panel category for your panel
  app, which should be the Simulation Menu category.
\end{itemize}

The simulation panel app extends the
\href{https://docs.liferay.com/ce/apps/web-experience/latest/javadocs/com/liferay/application/list/BaseJSPPanelApp.html}{BaseJSPPanelApp},
which provides a skeletal implementation of the
\href{https://docs.liferay.com/ce/apps/web-experience/latest/javadocs/com/liferay/application/list/PanelApp.html}{PanelApp}
interface with JSP support. JSPs, however, are not the only way to
provide frontend functionality to your panel categories/apps. You can
create your own class implementing \texttt{PanelApp} to use other
technologies, such as FreeMarker.

To learn more about Liferay Portal's product navigation using panel
categories and panel apps, see the
\href{/docs/7-0/tutorials/-/knowledge_base/t/customizing-the-product-menu}{Customizing
the Product Menu} tutorial. For more information on extending the
Simulation Menu, see the
\href{/docs/7-0/tutorials/-/knowledge_base/t/extending-the-simulation-menu}{Extending
the Simulation Menu} tutorial.

\subsection{Where Is This Sample?}\label{where-is-this-sample-11}

There are three different versions of this sample, each built with a
different build tool:

\begin{itemize}
\tightlist
\item
  \href{https://github.com/liferay/liferay-blade-samples/tree/7.0/gradle/apps/simulation-panel-app}{Gradle}
\item
  \href{https://github.com/liferay/liferay-blade-samples/tree/7.0/liferay-workspace/apps/simulation-panel-app}{Liferay
  Workspace}
\item
  \href{https://github.com/liferay/liferay-blade-samples/tree/7.0/maven/apps/simulation-panel-app}{Maven}
\end{itemize}

\section{Spring MVC Portlet}\label{spring-mvc-portlet}

The Spring MVC portlet provides a way to add various different fields
into the database and display them in a table. This project is a Spring
MVC based portlet WAR that implements the same functionality as the
\texttt{apps/service-builder/basic-web} sample project. It manages JSP
pages for display, uses a Spring-annotated portlet class, and invokes
the \texttt{apps/service-builder/basic-api} module to call services.

\noindent\hrulefill

\textbf{Note:} If you're planning to package this sample using Maven,
you must complete a few additional steps to avoid build errors. This
sample relies on the \texttt{service-builder/basic-api} module. Since
the \texttt{basic-api} bundle is not available on Liferay's CDN repo or
Maven Central, this sample can not reference it, resulting in build
failures.

To satisfy this dependency, you must install the bundle dependency to
your local \texttt{\textasciitilde{}/.m2\ repo}, along with the parent
BND plugin and root Maven project. Here are the steps to accomplish
this:

\begin{enumerate}
\def\labelenumi{\arabic{enumi}.}
\tightlist
\item
  Run \texttt{mvn\ clean\ install} on
  \texttt{maven/apps/service-builder/basic-api}.
\item
  Run \texttt{mvn\ clean\ install} on
  \texttt{maven/parent.bnd.bundle.plugin}.
\item
  Run \texttt{mvn\ clean\ install\ -N} in the root
  \texttt{liferay-blade-samples/maven} folder.
\end{enumerate}

Now you can build this sample successfully.

\noindent\hrulefill

\begin{figure}
\centering
\includegraphics{./images/spring-mvc-portlet.png}
\caption{Click \emph{Add} and fill out the sample fields to generate a
custom entry in the portlet's table.}
\end{figure}

Unlike the \texttt{service-builder/basic-web} module, Spring MVC
portlets must be delivered as portlet WAR projects. This project builds
to a WAR file but leverages all of the Liferay Workspace tools and
Gradle to build the WAR. You must build and deploy the
\texttt{service-builder/basic-api} and
\texttt{service-builder/basic-service} modules for this sample to work
properly. For more information on using Spring MVC portlets in Liferay
DXP, visit the
\href{/docs/7-0/tutorials/-/knowledge_base/t/spring-mvc}{Spring MVC}
tutorial.

\subsection{What API(s) and/or code components does this sample
highlight?}\label{what-apis-andor-code-components-does-this-sample-highlight-15}

This sample demonstrates a Liferay DXP portlet built using the
\href{https://docs.spring.io/spring/docs/current/spring-framework-reference/html/mvc.html}{Spring
Web MVC framework}.

\subsection{How does this sample leverage the API(s) and/or code
component?}\label{how-does-this-sample-leverage-the-apis-andor-code-component-15}

You can easily modify this sample by customizing its
\texttt{SpringMVCPortletViewController} Java class or any of its JSPs
stored in the \texttt{src/main/webapp/WEB-INF/jsp} folder. For more
information on customizing this sample, see the Javadoc listed in this
sample's \texttt{SpringMVCPortletViewController} Java class.

\subsection{Where Is This Sample?}\label{where-is-this-sample-12}

There are three different versions of this sample, each built with a
different build tool:

\begin{itemize}
\tightlist
\item
  \href{https://github.com/liferay/liferay-blade-samples/tree/7.0/gradle/apps/springmvc-portlet}{Gradle}
\item
  \href{https://github.com/liferay/liferay-blade-samples/tree/7.0/liferay-workspace/wars/springmvc-portlet}{Liferay
  Workspace}
\item
  \href{https://github.com/liferay/liferay-blade-samples/tree/7.0/maven/apps/springmvc-portlet}{Maven}
\end{itemize}

\chapter{Extensions}\label{extensions}

This section focuses on Liferay sample extensions. You can view these
sample extensions by visiting the \texttt{extensions} folder
corresponding to your preferred build tool:

\begin{itemize}
\tightlist
\item
  \href{https://github.com/liferay/liferay-blade-samples/tree/7.0/gradle/extensions}{Gradle
  sample extensions}
\item
  \href{https://github.com/liferay/liferay-blade-samples/tree/7.0/liferay-workspace/extensions}{Liferay
  Workspace sample extensions}
\item
  \href{https://github.com/liferay/liferay-blade-samples/tree/7.0/maven/extensions}{Maven
  sample extensions}
\end{itemize}

The following samples are documented:

\begin{itemize}
\tightlist
\item
  \href{control-menu-entry}{Control Menu Entry}
\item
  \href{document-action}{Document Action}
\item
  \href{gogo-shell-command}{Gogo Shell Command}
\item
  \href{indexer-post-processor}{Indexer Post Processor}
\item
  \href{model-listener}{Model Listener}
\item
  \href{screen-name-validator}{Screen Name Validator}
\end{itemize}

Visit a particular sample page to learn more!

\section{Control Menu Entry}\label{control-menu-entry}

The Control Menu Entry sample provides a customizable button that is
added to Liferay Portal's default Control Menu. When deploying this
sample with no customizations, an additional button is added to the User
(right side) portion of the Control Menu.

\begin{figure}
\centering
\includegraphics{./images/controlmenuentry.png}
\caption{The User area of the Control Menu is provided an additional
link button when the Control Menu Entry sample is deployed to Liferay
DXP.}
\end{figure}

The button navigates the user to Liferay's website:
https://www.liferay.com.

\subsection{What API(s) and/or code components does this sample
highlight?}\label{what-apis-andor-code-components-does-this-sample-highlight-16}

This sample leverages the
\href{@app-ref@/web-experience/latest/javadocs/com/liferay/product/navigation/control/menu/ProductNavigationControlMenuEntry.html}{ProductNavigationControlMenuEntry}
API.

\subsection{How does this sample leverage the API(s) and/or code
component?}\label{how-does-this-sample-leverage-the-apis-andor-code-component-16}

This sample first leverages the
\texttt{ProductNavigationControlMenuEntry} interface as an OSGi service
via the \texttt{@Component} annotation:

\begin{verbatim}
@Component(
    immediate = true,
    property = {
        "product.navigation.control.menu.category.key=" + ProductNavigationControlMenuCategoryKeys.USER,
        "product.navigation.control.menu.entry.order:Integer=1"
    },
    service = ProductNavigationControlMenuEntry.class
)
\end{verbatim}

There are also two properties provided via the \texttt{@Component}
annotation:

\begin{itemize}
\tightlist
\item
  \texttt{product.navigation.control.menu.category.key}: the category in
  which your entry should reside. The default Control Menu provides
  three categories: \emph{SITES} (left portion), \emph{TOOLS} (middle
  portion), and \emph{USER} (right portion).
\item
  \texttt{product.navigation.control.menu.entry.order:Integer}: the
  order in which your entry will be displayed in the category. Entries
  are ordered from left to right. For example, an entry with order
  \texttt{1} will be listed to the left of an entry with order
  \texttt{2}. If the order is not specified, it's chosen at random based
  on which service was registered first in the OSGi container.
\end{itemize}

This sample also implements the
\texttt{ProductNavigationControlMenuEntry} interface. The following
methods are implemented:

\begin{itemize}
\tightlist
\item
  \texttt{getIcon(HttpServletRequest)}
\item
  \texttt{getLabel(Locale)}
\item
  \texttt{getURL(HttpServletRequest)}
\item
  \texttt{isShow(HttpServletRequest)}
\end{itemize}

Refer to this sample's \texttt{BladeProductNavigationControlMenuEntry}
class for Javadocs describing these methods. For more information on how
to customize Liferay Portal's Control Menu, visit the
\href{/docs/7-0/tutorials/-/knowledge_base/t/customizing-the-control-menu}{Customizing
the Control Menu} tutorial.

\subsection{Where Is This Sample?}\label{where-is-this-sample-13}

There are three different versions of this sample, each built with a
different build tool:

\begin{itemize}
\tightlist
\item
  \href{https://github.com/liferay/liferay-blade-samples/tree/7.0/gradle/extensions/control-menu-entry}{Gradle}
\item
  \href{https://github.com/liferay/liferay-blade-samples/tree/7.0/liferay-workspace/extensions/control-menu-entry}{Liferay
  Workspace}
\item
  \href{https://github.com/liferay/liferay-blade-samples/tree/7.0/maven/extensions/control-menu-entry}{Maven}
\end{itemize}

\section{Document Action}\label{document-action}

The Document Action sample shows how to add a context menu option to an
entry in the Documents and Media portlet. When deploying this sample
with no customizations, an additional menu option is available in the
Documents and Media Admin portlet and the Documents and Media portlet.
This sample creates a \emph{Blade Basic Info} option that displays basic
information about the entry (e.g., file name, type, version, etc.). For
example, the Admin portlet provides the new option as illustrated in the
images below:

\begin{figure}
\centering
\includegraphics{./images/documents-and-media-admin-portlet.png}
\caption{The new \emph{Blade Basic Info} option is available from the
entry's Options menu.}
\end{figure}

\begin{figure}
\centering
\includegraphics{./images/documents-and-media-admin-portlet-detail.png}
\caption{The new option is also available from the portlet's Document
Details.}
\end{figure}

Likewise, the Documents and Media portlet provides the same option after
selecting \emph{Show Actions} from the portlet's Configuration menu.

\begin{figure}
\centering
\includegraphics{./images/documents-and-media-portlet.png}
\caption{You can access the new \emph{Blade Basic Info} option from the
Documents and Media portlet added to a page.}
\end{figure}

\begin{figure}
\centering
\includegraphics{./images/documents-and-media-portlet-detail.png}
\caption{The Documents And Media portlet provides the option from its
Document Detail too.}
\end{figure}

\subsection{What API(s) and/or code components does this sample
highlight?}\label{what-apis-andor-code-components-does-this-sample-highlight-17}

This sample leverages the
\href{@product-ref@/portal-kernel/com/liferay/portal/kernel/portlet/configuration/icon/PortletConfigurationIcon.html}{PortletConfigurationIcon}
API.

\subsection{How does this sample leverage the API(s) and/or code
component?}\label{how-does-this-sample-leverage-the-apis-andor-code-component-17}

There are four Java classes used in this sample:

\begin{itemize}
\tightlist
\item
  \texttt{BladeActionConfigurationIcon}: Adds the new context menu
  option to the Document Detail screen options
  (\includegraphics{./images/icon-options.png}) (top right
  corner) of the Documents and Media Admin portlet. See the
  \href{/docs/7-0/tutorials/-/knowledge_base/t/configuring-your-admin-apps-actions-menu}{Configuring
  Your Admin App's Actions Menu} tutorial for more details.
\item
  \texttt{BladeActionDisplayContext}: Adds the Display Context for the
  document action. More about Display Contexts are described later.
\item
  \texttt{BladeActionDisplayContextFactory}: Adds the Display Context
  factory for the document action.
\item
  \texttt{BladeDocumentActionPortlet}: Provides the portlet class, which
  extends the
  \href{https://portals.apache.org/pluto/portlet-2.0-apidocs/javax/portlet/GenericPortlet.html}{GenericPortlet}.
  This class generates what is shown when the context menu option is
  selected.
\end{itemize}

A Display Context is a Java class that controls access to a portlet
screen's UI elements. For example, the Document Library would use
Display Contexts to provide its screens all their UI elements. It would
use one Display Context for its document edit screen, another for its
document view screen, etc. A portlet ideally uses a different Display
Context for each of its screens.

A screen's JSP calls on the Display Context (DC) to get elements to
render and to decide whether to render certain types of elements. Some
of the DC methods return a collection of UI elements (e.g., a menu
object of menu items), while other DC methods return booleans that
determine whether to show particular element types. The DC decides which
objects to display, while the JSP organizes the rendered objects and
implements the screen's look and feel. You don't have to decide which
elements to display in your JSP; simply call the DC methods to populate
UI components with objects to render.

To customize or extend a portlet screen that uses a DC, you can extend
the DC and override the methods that control access to the elements that
interest you. For example, you can turn off displaying certain types of
elements (e.g., actions) by overriding the DC method that makes that
decision. You can add new custom elements (e.g., new actions) or remove
existing elements (e.g., a delete action) from a collection of elements
a DC method returns. The beauty of customizing via a DC is that you
don't have to modify the JSP. You only modify the particular methods
that are related to the UI customization goals. And JSP updates won't
break the DC customizations. Replacing a JSP, on the other hand, can
lead to missing an important JSP modification that a new Liferay version
introduces.

As you create custom portlets, you may want to implement DCs. You can
benefit from the separation of concerns that DCs provide and customers
can extend your portlet DCs to specify which UI elements to display. And
they don't need to worry about missing out on the updates you make to
the JSPs.

\subsection{Where Is This Sample?}\label{where-is-this-sample-14}

There are three different versions of this sample, each built with a
different build tool:

\begin{itemize}
\tightlist
\item
  \href{https://github.com/liferay/liferay-blade-samples/tree/7.0/gradle/extensions/document-action}{Gradle}
\item
  \href{https://github.com/liferay/liferay-blade-samples/tree/7.0/liferay-workspace/extensions/document-action}{Liferay
  Workspace}
\item
  \href{https://github.com/liferay/liferay-blade-samples/tree/7.0/maven/extensions/document-action}{Maven}
\end{itemize}

\section{Gogo Shell Command}\label{gogo-shell-command}

The Gogo Shell Command sample demonstrates adding a custom command to
Liferay DXP's Gogo shell environment. All Liferay DXP installations have a
Gogo shell environment, which lets system administrators interact with
Liferay DXP's module framework on a local server machine.

This example adds a new custom Gogo shell command called
\texttt{usercount} under the \texttt{blade} scope. It prints out the
number of registered users on your Liferay DXP installation.

To test this sample, follow the instructions below:

\begin{enumerate}
\def\labelenumi{\arabic{enumi}.}
\item
  Start a Liferay DXP installation.
\item
  Using a command line tool, connect to your local Gogo shell. For
  example, you can do this by executing
  \texttt{telnet\ localhost\ 11311}.
\item
  Run \texttt{help} to view all the available commands. The sample Gogo
  shell command is listed.

  \begin{figure}
  \centering
  \includegraphics{./images/gogo-shell-1.png}
  \caption{The sample Gogo shell command is listed with all the
  available commands.}
  \end{figure}
\item
  Execute \texttt{usercount} to execute the new custom command. The
  number of users on your running Liferay Portal installation is
  printed.

  \begin{figure}
  \centering
  \includegraphics{./images/gogo-shell-2.png}
  \caption{The outcome of executing the \texttt{usercount} command.}
  \end{figure}
\end{enumerate}

\subsection{What API(s) and/or code components does this sample
highlight?}\label{what-apis-andor-code-components-does-this-sample-highlight-18}

This sample demonstrates creating a new Gogo shell command by leveraging
\texttt{osgi.command.*} properties in a Java class.

\subsection{How does this sample leverage the API(s) and/or code
component?}\label{how-does-this-sample-leverage-the-apis-andor-code-component-18}

To add this new Gogo shell command, you must implement the logic in a
Java class with the following two properties:

\begin{itemize}
\tightlist
\item
  \texttt{osgi.command.function}: the command's name, which must match
  the method name in the registered service implementation.
\item
  \texttt{osgi.command.scope}: the general scope or namespace for the
  command.
\end{itemize}

These properties are set in your class's \texttt{@Component} annotation
like this:

\begin{verbatim}
@Component(
    property = {"osgi.command.function=usercount", "osgi.command.scope=blade"},
    service = Object.class
)
\end{verbatim}

The logic for the \texttt{usercount} command is specified in the method
with the same name:

\begin{verbatim}
public void usercount() {
    System.out.println(
        "# of users: " + getUserLocalService().getUsersCount());
}
\end{verbatim}

This method uses \emph{Declarative Services} to get a reference for the
\texttt{UserLocalService} to invoke the \texttt{getUsersCount} method.
This lets you find the number of users currently in the system.

For more information on using the Gogo shell, see the
\href{/docs/7-0/reference/-/knowledge_base/r/using-the-felix-gogo-shell}{Using
the Felix Gogo Shell} tutorial.

\subsection{Where Is This Sample?}\label{where-is-this-sample-15}

There are three different versions of this sample, each built with a
different build tool:

\begin{itemize}
\tightlist
\item
  \href{https://github.com/liferay/liferay-blade-samples/tree/7.0/gradle/extensions/gogo}{Gradle}
\item
  \href{https://github.com/liferay/liferay-blade-samples/tree/7.0/liferay-workspace/extensions/gogo}{Liferay
  Workspace}
\item
  \href{https://github.com/liferay/liferay-blade-samples/tree/7.0/maven/extensions/gogo}{Maven}
\end{itemize}

\section{Indexer Post Processor}\label{indexer-post-processor}

The Indexer Post Processor sample demonstrates using the
\texttt{IndexerPostProcessor} interface, which is provided to customize
search queries and documents before they're sent to the search engine,
and/or customize result summaries when they're returned to end users.
This basic demonstration prints a message in the log when one of the
\texttt{*IndexerPostProcessor} methods is called.

To see this sample's messages in Liferay DXP's log, you must add a
logging category to the portal. Navigate to \emph{Control Panel} →
\emph{Configuration} → \emph{Server Administration} and click on
\emph{Log Levels} → \emph{Add Category}. Then fill out the form:

\begin{itemize}
\tightlist
\item
  \emph{Logger Name}:
  \texttt{com.liferay.blade.samples.indexerpostprocessor}
\item
  \emph{Log Level}: \texttt{INFO}
\end{itemize}

Once you save the new logging category, you can witness the sample
indexer post processor in action. For example, you can test the sample's
\texttt{BlogsIndexerPostProcessor} implementation by creating a blog
entry. When you publish the blog, the following message is logged in the
console:

\begin{verbatim}
18:27:30,737 INFO  [http-nio-8080-exec-8][BlogsIndexerPostProcessor:76] postProcessDocument
\end{verbatim}

\subsection{What API(s) and/or code components does this sample
highlight?}\label{what-apis-andor-code-components-does-this-sample-highlight-19}

This sample leverages the
\href{@platform-ref@/7.0-latest/javadocs/portal-kernel/com/liferay/portal/kernel/search/IndexerPostProcessor.html}{IndexerPostProcessor}
API.

\subsection{How does this sample leverage the API(s) and/or code
component?}\label{how-does-this-sample-leverage-the-apis-andor-code-component-19}

This sample contains four implementations of the
\texttt{IndexerPostProcessor} interface:

\begin{itemize}
\tightlist
\item
  \texttt{BlogsIndexerPostProcessor}
\item
  \texttt{MultipleEntityIndexerPostProcessor}
\item
  \texttt{MultipleIndexerPostProcessor}
\item
  \texttt{UserEntityIndexerPostProcessor}
\end{itemize}

All these classes leverage the interface as an OSGi service via the
\texttt{@Component} annotation. For example, the \texttt{@Component}
annotation of the \texttt{UserEntityIndexerPostProcessor} looks like
this:

\begin{verbatim}
@Component(
    immediate = true,
    property = {
        "indexer.class.name=com.liferay.portal.kernel.model.User",
        "indexer.class.name=com.liferay.portal.kernel.model.UserGroup"
    },
    service = IndexerPostProcessor.class
)
\end{verbatim}

There's one property type provided via the \texttt{@Component}
annotation:

\begin{itemize}
\tightlist
\item
  \texttt{indexer.class.name}: the fully qualified class name of the
  indexed entity or an \texttt{Indexer} class itself.
\end{itemize}

This sample's implementations of the \texttt{IndexerPostProcessor}
interface override the following methods:

\begin{itemize}
\tightlist
\item
  \texttt{postProcessContextBooleanFilter}
\item
  \texttt{postProcessContextQuery}
\item
  \texttt{postProcessDocument}
\item
  \texttt{postProcessFullQuery}
\item
  \texttt{postProcessSearchQuery(BooleanQuery,\ BooleanFilter)}
\item
  \texttt{postProcessSearchQuery(BooleanQuery,\ SearchContext)}
\item
  \texttt{postProcessSummary}
\end{itemize}

For more information on Liferay's Search API, refer to the
\href{/docs/7-0/tutorials/-/knowledge_base/t/introduction-to-liferay-search}{Introduction
to Liferay Search} tutorial.

\subsection{Where Is This Sample?}\label{where-is-this-sample-16}

There are three different versions of this sample, each built with a
different build tool:

\begin{itemize}
\tightlist
\item
  \href{https://github.com/liferay/liferay-blade-samples/tree/7.0/gradle/extensions/indexer-post-processor}{Gradle}
\item
  \href{https://github.com/liferay/liferay-blade-samples/tree/7.0/liferay-workspace/extensions/indexer-post-processor}{Liferay
  Workspace}
\item
  \href{https://github.com/liferay/liferay-blade-samples/tree/7.0/maven/extensions/indexer-post-processor}{Maven}
\end{itemize}

\section{Model Listener}\label{model-listener}

The Model Listener sample demonstrates adding a custom model listener to
a Liferay Portal out-of-the-box entity. When deploying this sample with
no customizations, a custom model listener is added to the portal's
layouts, listening for \texttt{onBeforeCreate} events. This means that
any page creation will trigger this listener, which will execute before
the new page is created.

For example, if a new page is added with the name \emph{My Test Page},
the following message is printed to the console:

\begin{figure}
\centering
\includegraphics{./images/model-listener-1.png}
\caption{The sample model listener's message in the console.}
\end{figure}

You can also verify that the model listener sample was executed by
navigating to the new page's \emph{Configure Page} → \emph{SEO} option.
The HTML Title field looks like this:

\begin{figure}
\centering
\includegraphics{./images/model-listener-2.png}
\caption{The page's HTML title updated by the model listener sample.}
\end{figure}

\subsection{What API(s) and/or code components does this sample
highlight?}\label{what-apis-andor-code-components-does-this-sample-highlight-20}

This sample leverages the
\href{@platform-ref@/7.0-latest/javadocs/portal-kernel/com/liferay/portal/kernel/model/ModelListener.html}{ModelListener}
API.

\subsection{How does this sample leverage the API(s) and/or code
component?}\label{how-does-this-sample-leverage-the-apis-andor-code-component-20}

Model Listeners are used to listen for persistence events on models and
take actions as a result of those events. Actions can be executed on an
entity's database table before or after a \texttt{create},
\texttt{remove}, \texttt{update}, \texttt{addAssociation}, or
\texttt{removeAssociation} event. It's possible to have more than one
model listener on a single model too; the execution order is not
guaranteed.

There are two steps to create a new model listener:

\begin{itemize}
\tightlist
\item
  Implement a Model Listener class
\item
  Register the new service in Liferay's OSGi runtime
\end{itemize}

This sample adds the model listener logic in a new Java class named
\texttt{CustomLayoutListener} that extends
\href{@platform-ref@/7.0-latest/javadocs/portal-kernel/com/liferay/portal/kernel/model/BaseModelListener.html}{BaseModelListener}.

\begin{verbatim}
public class CustomLayoutListener extends BaseModelListener<Layout> {

    @Override
    public void onBeforeCreate(Layout model) throws ModelListenerException {
        System.out.println(
            "About to create layout: " + model.getNameCurrentValue());

        model.setTitle("Title generated by model listener!");
    }

}
\end{verbatim}

Important things to note in this code snippet are

\begin{itemize}
\tightlist
\item
  The entity to be targeted by this model listener is specified as the
  parameterized type (e.g., \texttt{Layout}).
\item
  The overridden methods dictate the type of event(s) that are listened
  for (e.g., \texttt{onBeforeCreate}); they also trigger the logic
  execution.
\end{itemize}

The final step is registering the service in Liferay's OSGi runtime,
which is accomplished by the following annotation (if using Declarative
Services):

\begin{verbatim}
@Component(immediate = true, service = ModelListener.class)
\end{verbatim}

For more information on model listeners, see the
\href{/docs/7-0/tutorials/-/knowledge_base/t/model-listeners}{Creating
Model Listeners} tutorial.

\subsection{Where Is This Sample?}\label{where-is-this-sample-17}

There are three different versions of this sample, each built with a
different build tool:

\begin{itemize}
\tightlist
\item
  \href{https://github.com/liferay/liferay-blade-samples/tree/7.0/gradle/extensions/model-listener}{Gradle}
\item
  \href{https://github.com/liferay/liferay-blade-samples/tree/7.0/liferay-workspace/extensions/model-listener}{Liferay
  Workspace}
\item
  \href{https://github.com/liferay/liferay-blade-samples/tree/7.0/maven/extensions/model-listener}{Maven}
\end{itemize}

\section{Screen Name Validator}\label{screen-name-validator}

The Screen Name Validator sample provides a way to validate a user's
inputted screen name. During validation, the screen name is tested
client-side and server-side.

This sample checks if a user's screen name contains reserved words that
are configured in the \emph{Control Panel} → \emph{Configuration} →
\emph{System Settings} → \emph{Foundation} → \emph{ScreenName Validator}
menu. The default values for the screen name validator's reserved words
are \emph{admin} and \emph{user}.

\begin{figure}
\centering
\includegraphics{./images/screenname-validator-config.png}
\caption{Enter reserved words for the screen name validator.}
\end{figure}

You can test this sample by following the following steps:

\begin{enumerate}
\def\labelenumi{\arabic{enumi}.}
\tightlist
\item
  Deploy the Screen Name Validator to your portal installation.
\item
  Navigate to the \emph{Control Panel} → \emph{Users} → \emph{Users and
  Organizations} menu.
\item
  Create a new user by selecting the \emph{Add User}
  (\includegraphics{./images/icon-add.png}) button.
\item
  Adding a screen name that contains the word \emph{admin} or
  \emph{user}.
\end{enumerate}

\begin{figure}
\centering
\includegraphics{./images/screenname-validator-test.png}
\caption{The error message displays when inputting a reserved word for
the screen name.}
\end{figure}

\subsection{What API(s) and/or code components does this sample
highlight?}\label{what-apis-andor-code-components-does-this-sample-highlight-21}

This sample leverages the
\href{@product-ref@/portal-kernel/com/liferay/portal/kernel/security/auth/ScreenNameValidator.html}{ScreenNameValidator}
API.

\subsection{How does this sample leverage the API(s) and/or code
component?}\label{how-does-this-sample-leverage-the-apis-andor-code-component-21}

To customize this sample, modify its
\texttt{com.liferay.blade.samples.screenname.validator.internal.CustomScreenNameValidator}
class.

You can also customize this sample's configuration by adding more
properties in its
\texttt{com.liferay.blade.samples.screenname.validator.CustomScreenNameConfiguration}
class.

For more information on customizing the Validation sample to fit your
needs, see the Javadoc provided in this sample's Java classes.

\subsection{Where Is This Sample?}\label{where-is-this-sample-18}

There are three different versions of this sample, each built with a
different build tool:

\begin{itemize}
\tightlist
\item
  \href{https://github.com/liferay/liferay-blade-samples/tree/7.0/gradle/extensions/screen-name-validator}{Gradle}
\item
  \href{https://github.com/liferay/liferay-blade-samples/tree/7.0/liferay-workspace/extensions/screen-name-validator}{Liferay
  Workspace}
\item
  \href{https://github.com/liferay/liferay-blade-samples/tree/7.0/maven/extensions/screen-name-validator}{Maven}
\end{itemize}

\section{Servlet}\label{servlet}

The Servlet sample provides an OSGi Whiteboard Servlet in Liferay DXP.
When deploying this sample and configuring the servlet, a \emph{Hello
World} message is displayed when accessing the servlet page URL. Log
info is also outputted to your console.

\begin{figure}
\centering
\includegraphics{./images/servlet-sample.png}
\caption{The servlet displays \emph{Hello World} from the configured
servlet page URL.}
\end{figure}

\begin{figure}
\centering
\includegraphics{./images/servlet-sample-log.png}
\caption{The servlet also logs info in the console.}
\end{figure}

To configure the servlet in Liferay DXP, complete the following steps:

\begin{enumerate}
\def\labelenumi{\arabic{enumi}.}
\item
  Navigate to the \emph{Control Panel} → \emph{Configuration} →
  \emph{Server Administration} → \emph{Log Levels}.
\item
  Select \emph{Add Category}.
\item
  Insert \emph{com.liferay.blade.samples.servlet.BladeServlet} for the
  Logger Name and \emph{INFO} for the Log Level.
\item
  Navigate to the http://localhost:8080/o/blade/servlet URL.
\end{enumerate}

\subsection{What API(s) and/or code components does this sample
highlight?}\label{what-apis-andor-code-components-does-this-sample-highlight-22}

This sample leverages the
\href{https://tomcat.apache.org/tomcat-5.5-doc/servletapi/javax/servlet/http/HttpServlet.html}{HttpServlet}
API.

\subsection{How does this sample leverage the API(s) and/or code
component?}\label{how-does-this-sample-leverage-the-apis-andor-code-component-22}

To customize this sample, modify its
\texttt{com.liferay.blade.samples.servlet.BladeServlet} class. This
class extends the \texttt{HttpServlet} class. Creating your own servlet
for Liferay DXP is useful when you need to implement servlet actions.
For example, if you wanted to implement the CMIS server by yourself with
\href{https://chemistry.apache.org/}{Apache Chemistry}, you would need
to implement your own servlet, managing requests at a low level.

\subsection{Where Is This Sample?}\label{where-is-this-sample-19}

There are three different versions of this sample, each built with a
different build tool:

\begin{itemize}
\tightlist
\item
  \href{https://github.com/liferay/liferay-blade-samples/tree/7.0/gradle/extensions/servlet}{Gradle}
\item
  \href{https://github.com/liferay/liferay-blade-samples/tree/7.0/liferay-workspace/extensions/servlet}{Liferay
  Workspace}
\item
  \href{https://github.com/liferay/liferay-blade-samples/tree/7.0/maven/extensions/servlet}{Maven}
\end{itemize}

\chapter{Overrides}\label{overrides}

This section focuses on Liferay sample overrides. You can view these
sample overrides by visiting the \texttt{overrides} folder corresponding
to your preferred build tool:

\begin{itemize}
\tightlist
\item
  \href{https://github.com/liferay/liferay-blade-samples/tree/7.0/gradle/overrides}{Gradle
  sample overrides}
\item
  \href{https://github.com/liferay/liferay-blade-samples/tree/7.0/liferay-workspace/overrides}{Liferay
  Workspace sample overrides}
\item
  \href{https://github.com/liferay/liferay-blade-samples/tree/7.0/maven/overrides}{Maven
  sample overrides}
\end{itemize}

The following samples are documented:

\begin{itemize}
\tightlist
\item
  \href{core-jsp-override}{Core JSP Override}
\item
  \href{module-jsp-override}{Module JSP Override}
\item
  \href{resource-bundle-override}{Resource Bundle Override}
\end{itemize}

Visit a particular sample page to learn more!

\section{Core JSP Override}\label{core-jsp-override}

The Core JSP Override sample lets you override core/kernel JSPs by
adding them to the module's \texttt{META-INF/jsps} folder. This module
overrides the Liferay DXP's \texttt{bottom.jsp} file by inserting the
\texttt{bottom-ext.jsp} file in the
\texttt{META-INF/jsps/html/common/themes} folder. When deploying this
sample with no customizations, sample text is added to the bottom of
Liferay's default theme.

\begin{figure}
\centering
\includegraphics{./images/blade-core-jsp-hook.png}
\caption{Deploying a core JSP override overrides core functionality,
like Liferay DXP's default theme.}
\end{figure}

For more information on how to customize Liferay's Core using JSP
overrides, visit the
\href{/docs/7-0/tutorials/-/knowledge_base/t/overriding-core-jsps}{Overriding
Core JSPs} tutorial.

\subsection{What API(s) and/or code components does this sample
highlight?}\label{what-apis-andor-code-components-does-this-sample-highlight-23}

This sample leverages the
\href{@platform-ref@/7.0-latest/javadocs/portal-impl/com/liferay/portal/deploy/hot/CustomJspBag.html}{CustomJspBag}
API.

\textbf{Important:} Using core JSP overrides should be a last resort
option only when there is no other way to customize functionality in
your Liferay installation. It's up to the maintainer of this JSP
override to properly maintain and adapt to changes in the underlying JSP
implementation.

\subsection{How does this sample leverage the API(s) and/or code
component?}\label{how-does-this-sample-leverage-the-apis-andor-code-component-23}

You can easily modify this sample by customizing its
\texttt{com.liferay.blade.samples.corejsphook.BladeCustomJspBag} Java
class or adding additional JSPs in the configured JSP folder. You can
modify the custom JSP folder's path by editing the
\texttt{BladeCustomJspBag.getCustomJspDir()} method to return a
different folder path.

For more information on customizing the Core JSP Override sample to fit
your needs, see the Javadoc listed in this sample's
\texttt{BladeCustomJspBag} class.

\subsection{Where Is This Sample?}\label{where-is-this-sample-20}

There are three different versions of this sample, each built with a
different build tool:

\begin{itemize}
\tightlist
\item
  \href{https://github.com/liferay/liferay-blade-samples/tree/7.0/gradle/overrides/core-jsp-override}{Gradle}
\item
  \href{https://github.com/liferay/liferay-blade-samples/tree/7.0/liferay-workspace/overrides/core-jsp-override}{Liferay
  Workspace}
\item
  \href{https://github.com/liferay/liferay-blade-samples/tree/7.0/maven/overrides/core-jsp-override}{Maven}
\end{itemize}

\section{Module JSP Override}\label{module-jsp-override}

The Module JSP Override sample conveys Liferay's recommended approach to
override an application's JSP by leveraging OSGi fragment modules. This
example overrides the default \texttt{login.jsp} file in the
\texttt{com.liferay.login.web} bundle by adding the red text
\emph{changed} to the Sign In form.

\begin{figure}
\centering
\includegraphics{./images/hook-jsp.png}
\caption{The customized Sign In form with the new \emph{changed} text.}
\end{figure}

\subsection{What API(s) and/or code components does this sample
highlight?}\label{what-apis-andor-code-components-does-this-sample-highlight-24}

This sample demonstrates how to create a fragment host module and
configure it to override an existing module's JSP.

\subsection{How does this sample leverage the API(s) and/or code
component?}\label{how-does-this-sample-leverage-the-apis-andor-code-component-24}

You can create your own JSP override by

\begin{itemize}
\tightlist
\item
  Declaring the fragment host.
\item
  Providing the JSP that will override the original one.
\end{itemize}

To properly declare the fragment host in the \texttt{bnd.bnd} file, you
must specify the host module's (where the original JSP is located)
Bundle Symbolic Name and the host module's exact version to which the
fragment belongs. In this example, this is configured like this:

\begin{verbatim}
Fragment-Host: com.liferay.login.web;bundle-version="1.0.0"
\end{verbatim}

Then you must provide the new JSP intended to override the original one.
Be sure to mimic the host module's folder structure when overriding its
JAR. For this example, since the original JSP is in the folder
\texttt{/META-INF/resources/login.jsp}, the new JSP file resides in the
folder \texttt{src/main/resources/META-INF/resources/login.jsp}.

If needed, you can also target the original JSP following one of the two
possible naming conventions: \texttt{original} or \texttt{portal}. This
pattern looks like

\begin{verbatim}
<liferay-util:include
    page="/login.original.jsp"
    servletContext="<%= application %>"
/>
\end{verbatim}

or

\begin{verbatim}
<liferay-util:include
    page="/login.portal.jsp"
    servletContext="<%= application %>"
/>
\end{verbatim}

This approach can be used to override any application JSP (i.e., JSPs
residing in a module). You can also add new JSPs to an existing module
with this technique. If you need to override a core JSP, see the
\texttt{core-jsp-override} sample.

For more information on using fragment bundles to override application
JSPs, see the
\href{/docs/7-0/tutorials/-/knowledge_base/t/overriding-a-modules-jsps}{Overriding
App JSPs} tutorial.

\subsection{Where Is This Sample?}\label{where-is-this-sample-21}

There are three different versions of this sample, each built with a
different build tool:

\begin{itemize}
\tightlist
\item
  \href{https://github.com/liferay/liferay-blade-samples/tree/7.0/gradle/overrides/module-jsp-override}{Gradle}
\item
  \href{https://github.com/liferay/liferay-blade-samples/tree/7.0/liferay-workspace/overrides/module-jsp-override}{Liferay
  Workspace}
\item
  \href{https://github.com/liferay/liferay-blade-samples/tree/7.0/maven/overrides/module-jsp-override}{Maven}
\end{itemize}

\section{Resource Bundle Override}\label{resource-bundle-override}

This example overrides the default \texttt{add-blog-entry} language key
(English and Spanish) for Liferay DXP's default Blogs application. After
deploying this sample hook to Liferay DXP, the Blogs application's
\emph{Add Blog Entry} button is modified to display \emph{Overriden Add
Blog Entry}. If you change Liferay DXP's default language to Spanish,
the modified language key is translated to display in that language. For
example, the text changes to \emph{Añadir entrada sobreescrita}.

\begin{figure}
\centering
\includegraphics{./images/hook-resourcebundle.png}
\caption{The customized Blogs application displays the new
\texttt{add-blog-entry} language key in English.}
\end{figure}

For reference, the Blogs application's language keys are stored in the
\href{https://github.com/liferay/liferay-portal}{liferay-portal} Github
repo's
\texttt{modules/apps/collaboration/blogs/blogs-web/src/main/resources/content}
folder.

\subsection{What API(s) and/or code components does this sample
highlight?}\label{what-apis-andor-code-components-does-this-sample-highlight-25}

This sample leverages the
\href{@product-ref@/portal-kernel/com/liferay/portal/kernel/util/ResourceBundleLoader.html}{ResourceBundleLoader}
API.

\subsection{How does this sample leverage the API(s) and/or code
component?}\label{how-does-this-sample-leverage-the-apis-andor-code-component-25}

This sample conveys the recommended approach to override an
application's language keys file for any module that is deployed to
Liferay DXP's OSGi runtime (not applicable to Liferay DXP's core
language keys).

The steps to override applications' language keys are

\begin{itemize}
\tightlist
\item
  Implement a resource bundle loader.
\item
  Register the service.
\item
  Provide the new language keys that will override the original ones.
\end{itemize}

The resource bundle loader is a class that should implement the
interface \texttt{com.liferay.portal.kernel.util.ResourceBundleLoader}.
Specifically, you must implement the \texttt{loadResourceBundle} method,
which returns the loaded resource bundle:

\begin{verbatim}
@Override
public ResourceBundle loadResourceBundle(String languageId) {
    return _resourceBundleLoader.loadResourceBundle(languageId);
}
\end{verbatim}

Then you must set the resource bundle loader to load the resource
bundles as an \texttt{AggregateResourceBundleLoader}.

\begin{verbatim}
@Reference(target = "(&(bundle.symbolic.name=com.liferay.blogs.web)(!(component.name=com.liferay.blade.samples.hook.resourcebundle.ResourceBundleLoaderComponent)))")
public void setResourceBundleLoader(
    ResourceBundleLoader resourceBundleLoader) {

    _resourceBundleLoader = new AggregateResourceBundleLoader(
        new CacheResourceBundleLoader(
            new ClassResourceBundleLoader(
                "content.Language",
                ResourceBundleLoaderComponent.class.getClassLoader())),
    resourceBundleLoader);
}
\end{verbatim}

The \texttt{@Reference} annotation targets the original Blogs module by
specifying its symbolic name \texttt{com.liferay.blogs.web}. This
sample's own component name (i.e.,
\texttt{com.liferay.blade.samples.hook.resourcebundle.ResourceBundleLoaderComponent})
is not targeted to use this resource bundle.

Also note the required parameters to set the resource bundle loader:

\begin{itemize}
\tightlist
\item
  The base language file name (e.g., \texttt{content.Language}).
\item
  The classloader for your resource bundle loader.
\item
  The resource bundle loader from the method's parameter.
\end{itemize}

The class should also register the resource bundle loader in the OSGi
runtime. This is done by setting the following three properties:

\begin{itemize}
\tightlist
\item
  \texttt{bundle.symbolic.name}: The symbolic name of the target module
  (i.e., the module's keys you're overriding).
\item
  \texttt{resource.bundle.base.name}: The resource bundle base name that
  points to your language files.
\item
  \texttt{servlet.context.name}: The servlet context name of the target
  module.
\end{itemize}

These properties are set in your class's \texttt{@Component} annotation
like this:

\begin{verbatim}
@Component(
    immediate = true,
    property = {
        "bundle.symbolic.name=com.liferay.blogs.web",
        "resource.bundle.base.name=content.Language",
        "servlet.context.name=blogs-web"
    }
)
\end{verbatim}

Lastly, the new \texttt{language.properties} files should be added to
the folder \texttt{src/content} for each locale's keys you want to
override. Since this example's goal is to only override the English and
Spanish keys, the \texttt{Language\_en.properties} and
\texttt{Language\_es.properties} are added.

This approach can be used to override any application's language keys
(i.e., \texttt{language.properties} files that are inside a module
deployed to Liferay DXP's OSGi runtime). If you need to override Liferay
DXP's core language keys, see the
\href{/docs/7-0/tutorials/-/knowledge_base/t/overriding-language-keys\#modifying-global-language-keys}{Modifying
Global Language Keys} tutorial.

For more information on using a resource bundle to override an
application's language keys, see the
\href{/docs/7-0/tutorials/-/knowledge_base/t/overriding-language-keys\#overriding-a-modules-language-keys}{Overriding
a Module's Language Keys} tutorial.

\subsection{Where Is This Sample?}\label{where-is-this-sample-22}

There are three different versions of this sample, each built with a
different build tool:

\begin{itemize}
\tightlist
\item
  \href{https://github.com/liferay/liferay-blade-samples/tree/7.0/gradle/overrides/resource-bundle-override}{Gradle}
\item
  \href{https://github.com/liferay/liferay-blade-samples/tree/7.0/liferay-workspace/overrides/resource-bundle-override}{Liferay
  Workspace}
\item
  \href{https://github.com/liferay/liferay-blade-samples/tree/7.0/maven/overrides/resource-bundle-override}{Maven}
\end{itemize}

\chapter{Themes}\label{themes-22}

This section focuses on Liferay sample themes. You can view these sample
themes by visiting the \texttt{themes} folder corresponding to your
preferred build tool:

\begin{itemize}
\tightlist
\item
  \href{https://github.com/liferay/liferay-blade-samples/tree/7.0/gradle/themes}{Gradle
  sample themes}
\item
  \href{https://github.com/liferay/liferay-blade-samples/tree/7.0/liferay-workspace/themes}{Liferay
  Workspace sample themes}
\item
  \href{https://github.com/liferay/liferay-blade-samples/tree/7.0/maven/themes}{Maven
  sample themes}
\end{itemize}

The following samples are documented:

\begin{itemize}
\tightlist
\item
  \href{theme}{Simple Theme}
\item
  \href{template-context-contributor}{Template Context Contributor}
\item
  \href{theme-contributor}{Theme Contributor}
\end{itemize}

Visit a particular sample page to learn more!

\section{Simple Theme}\label{simple-theme}

The Simple Theme sample provides the base files for a theme, using the
\href{/docs/7-0/reference/-/knowledge_base/r/theme-builder-gradle-plugin}{Theme
Builder Gradle plugin}. When deploying this sample with no
customizations, a theme based off of the \texttt{\_styled} base theme is
created.

\begin{figure}
\centering
\includegraphics{./images/theme.png}
\caption{A theme based off of the Styled base theme is created when the
Theme Blade sample is deployed to Liferay Portal.}
\end{figure}

For more information on themes, visit the
\href{/docs/7-0/tutorials/-/knowledge_base/t/introduction-to-themes}{Introduction
to Themes} tutorial.

\subsection{What API(s) and/or code components does this sample
highlight?}\label{what-apis-andor-code-components-does-this-sample-highlight-26}

This sample demonstrates a way to create a simple theme in Liferay DXP.

\subsection{How does this sample leverage the API(s) and/or code
component?}\label{how-does-this-sample-leverage-the-apis-andor-code-component-26}

To modify this sample, add the \texttt{images}, \texttt{js}, or
\texttt{templates} folder, along with your modified files, to the
\texttt{src/main/webapp} folder. The sample already provides the
\texttt{src/main/resources/resources-importer},
\texttt{src/main/webapp/WEB-INF}, and \texttt{src/main/webapp/css}
folders for you. Add your style modifications to the provided
\texttt{css/\_custom.scss} file. For a complete explanation of a theme's
files, see the
\href{/docs/7-0/reference/-/knowledge_base/r/theme-reference-guide}{Theme
Reference Guide}.

\subsection{Where Is This Sample?}\label{where-is-this-sample-23}

There are three different versions of this sample, each built with a
different build tool:

\begin{itemize}
\tightlist
\item
  \href{https://github.com/liferay/liferay-blade-samples/tree/7.0/gradle/themes/simple-theme}{Gradle}
\item
  \href{https://github.com/liferay/liferay-blade-samples/tree/7.0/liferay-workspace/wars/simple-theme}{Liferay
  Workspace}
\item
  \href{https://github.com/liferay/liferay-blade-samples/tree/7.0/maven/themes/simple-theme}{Maven}
\end{itemize}

\section{Template Context
Contributor}\label{template-context-contributor}

The Template Context Contributor sample injects a new variable into
Liferay DXP's theme context. When deploying this sample with no
customizations, you can use the \texttt{\$\{sample\_text\}} variable
from any theme.

\subsection{What API(s) and/or code components does this sample
highlight?}\label{what-apis-andor-code-components-does-this-sample-highlight-27}

Many developers prefer using templating frameworks like FreeMarker and
Velocity, but don't have access to the common objects offered to those
working with JSPs. Context contributors allow non-JSP developers an easy
way to inject variables into their Liferay templates.

This sample leverages the
\href{@product-ref@/portal-kernel/com/liferay/portal/kernel/template/TemplateContextContributor.html}{TemplateContextContributor}
API.

\subsection{How does this sample leverage the API(s) and/or code
component?}\label{how-does-this-sample-leverage-the-apis-andor-code-component-27}

You can easily modify this sample by customizing its
\texttt{BladeTemplateContextContributor.java} Java class. For example,
the default context contributor sample provides the
\texttt{\$\{sample\_text\}} variable by injecting it into Liferay's
\texttt{contextObjects}, which is a map provided by default to offer
common variables to non-JSP template developers. You can easily inject
your own variables into the \texttt{contextObjects} map usable by any
theme deployed to Liferay DXP.

Are you working with templates that aren't themes (e.g., ADTs, DDM
templates, etc.)? You can change the context in which your variables are
injected by modifying the \texttt{property} attribute in the
\texttt{@Component} annotation. If you want your variable available for
all templates, change it to

\begin{verbatim}
property = {"type=" + TemplateContextContributor.TYPE_GLOBAL}
\end{verbatim}

For more information on customizing the Template Context Contributor
sample to fit your needs, see the Javadoc listed in this sample's
\texttt{com.liferay.blade.samples.theme.contributorBladeTemplateContextContributor}
class. For more information on context contributors and how to create
them in Liferay DXP, visit the
\href{/docs/7-0/tutorials/-/knowledge_base/t/context-contributors}{Context
Contributors} tutorial.

\subsection{Where Is This Sample?}\label{where-is-this-sample-24}

There are three different versions of this sample, each built with a
different build tool:

\begin{itemize}
\tightlist
\item
  \href{https://github.com/liferay/liferay-blade-samples/tree/7.0/gradle/themes/template-context-contributor}{Gradle}
\item
  \href{https://github.com/liferay/liferay-blade-samples/tree/7.0/liferay-workspace/themes/template-context-contributor}{Liferay
  Workspace}
\item
  \href{https://github.com/liferay/liferay-blade-samples/tree/7.0/maven/themes/template-context-contributor}{Maven}
\end{itemize}

\section{Theme Contributor}\label{theme-contributor}

The Theme Contributor sample contributes updates to the UI of the theme
body, Control Menu, Product Menu, and Simulation Panel. When deploying
this sample with no customizations, the colors of the theme and
aforementioned menus are updated.

\begin{figure}
\centering
\includegraphics{./images/theme-contributor-yellow.png}
\caption{Your Liferay DXP pages and menu fonts now have a yellow tint.}
\end{figure}

Also, there's a simple JavaScript update that is provided, which logs a
message to the browser's console window that states \emph{Hello Blade
Theme Contributor!}.

\begin{figure}
\centering
\includegraphics{./images/theme-contributor-console-output.png}
\caption{The message is printed to your browser's console window using
JavaScript.}
\end{figure}

\subsection{What API(s) and/or code components does this sample
highlight?}\label{what-apis-andor-code-components-does-this-sample-highlight-28}

This sample demonstrates a way to contribute updates to a Liferay DXP
theme. Theme Contributors let you package UI resources (e.g., CSS and
JS) independent of a theme to include on a Liferay DXP page.

\subsection{How does this sample leverage the API(s) and/or code
component?}\label{how-does-this-sample-leverage-the-apis-andor-code-component-28}

To modify this sample, replace the corresponding JS or SCSS file with
the JavaScript or styles that you want, or add your own JS or SCSS
files. For example, this sample provides an update to the Control Menu's
\texttt{background-color} in its
\texttt{src/main/resources/META-INF/resources/css/blade.theme.contributor/\_control\_menu.scss}
file:

\begin{verbatim}
body {
        .control-menu {
                background-color: darkkhaki;
        }
}
\end{verbatim}

All of the SCSS files used in this sample are imported into the main
\texttt{blade.theme.contributor.scss} file:

\begin{verbatim}
@import "bourbon";
@import "mixins";

@import "blade.theme.contributor/body";
@import "blade.theme.contributor/control_menu";
@import "blade.theme.contributor/product_menu";
@import "blade.theme.contributor/simulation_panel";
\end{verbatim}

If you add your own \texttt{SCSS} files, you must add them to the list
of imports in the \texttt{blade.theme.contributor.scss} file.

Likewise, the sample \texttt{blade.theme.contributor.js} logs a message
to your browser's console window using the following JS logic:

\begin{verbatim}
console.log('Hello Blade Theme Contributor!');
\end{verbatim}

For more information on Theme Contributors, visit the
\href{/docs/7-0/tutorials/-/knowledge_base/t/theme-contributors}{Theme
Contributors} tutorial.

\subsection{Where Is This Sample?}\label{where-is-this-sample-25}

There are three different versions of this sample, each built with a
different build tool:

\begin{itemize}
\tightlist
\item
  \href{https://github.com/liferay/liferay-blade-samples/tree/7.0/gradle/themes/theme-contributor}{Gradle}
\item
  \href{https://github.com/liferay/liferay-blade-samples/tree/7.0/liferay-workspace/themes/theme-contributor}{Liferay
  Workspace}
\item
  \href{https://github.com/liferay/liferay-blade-samples/tree/7.0/maven/themes/theme-contributor}{Maven}
\end{itemize}

\section{Third Party Packages Portal
Exports}\label{third-party-packages-portal-exports}

The \texttt{com.liferay.portal.bootstrap} module exports many third
party Java packages that can cause problems if used improperly. If your
WAR's Gradle file, for example, uses the \texttt{compile} scope for a
dependency that Liferay's OSGi runtime already provides, the dependency
JAR is included in the WAR's \texttt{WEB-INF/lib} and deployed in the
resulting WAB, and two versions of dependency classes wind up on the
classpath. This can cause weird errors that are hard to debug.

To find a list of the packages exported by
\texttt{com.liferay.portal.bootstrap}, go to the source file
\texttt{modules/core/portal-bootstrap/system.packages.extra.bnd}. If you
don't have access to the source code, the same list (in a less
user-friendly format) is in the
\texttt{META-INF/system.packages.extra.mf} file in
\texttt{{[}LIFERAY\_HOME{]}/osgi/core/com.liferay.portal.bootstrap.jar}.
These packages are installed and available in Liferay's OSGi runtime. If
your module or WAR uses one of them, specify the corresponding
dependency as being ``provided'' (provided by Liferay DXP). Here's how
to specify a provided dependency:

Maven:
\texttt{\textless{}scope\textgreater{}provided\textless{}/scope\textgreater{}}

Gradle: \texttt{providedCompile}

Now you can safely leverage third party packages Liferay DXP provides!

\subsection{Related topics}\label{related-topics-4}

\href{/docs/7-0/tutorials/-/knowledge_base/t/resolving-a-plugins-dependencies}{Resolving
a Plugin's Dependencies}

\href{/docs/7-0/tutorials/-/knowledge_base/t/configuring-dependencies}{Configuring
Dependencies}

\section{Resolving Common Output Errors Reported by the resolve
Task}\label{resolving-common-output-errors-reported-by-the-resolve-task}

\href{/docs/7-0/tutorials/-/knowledge_base/t/liferay-workspace}{Liferay
Workspace} provides the \texttt{resolve} Gradle task to
\href{/docs/7-0/tutorials/-/knowledge_base/t/resolving-your-modules}{validate
modules}. This is very useful for finding issues and reporting them as
output before deployment. For more information on running this task from
Liferay Workspace, see the
\href{/docs/7-0/tutorials/-/knowledge_base/t/validating-modules-against-the-target-platform}{Validating
Modules Against the Target Platform} tutorial section. For general help
with OSGi related issues, visit the
\href{/docs/7-0/tutorials/-/knowledge_base/t/troubleshooting}{Troubleshooting
FAQ} tutorial section.

For help interpreting the \texttt{resolve} task's output, see the list
below for common output errors, what they mean, and how to fix them.

\subsection{Missing Import Error}\label{missing-import-error}

When your module refers to an unavailable import, the container throws
this error. For example, suppose you have a module \texttt{test-service}
that depends on the \texttt{com.google.common.base} package. If the
container can't find that package, it throws this error:

\begin{verbatim}
Resolution exception in project 'modules:test-service': Unresolved requirements in root project 'modules:test-service':
    Mandatory:
        [osgi.wiring.package ] com.google.common.base; version=[23.0.0,24.0.0)
        [osgi.identity       ] test.service
\end{verbatim}

This kind of error can also occur when separate modules require
different versions of another module. If you have \emph{module A}
requiring \emph{module Test version 1} and \emph{module B} requiring
\emph{module Test version 4}, without running the resolver, both modules
A and B would compile successfully. When they were deployed, however,
one would fail in the OSGi runtime because both dependencies cannot be
satisfied. These types of scenarios are difficult to diagnose, but with
the \texttt{resolve} task, can be found with ease.

To fix missing import errors, you may need to adjust the
\href{/docs/7-0/tutorials/-/knowledge_base/t/exporting-packages}{export}
and/or
\href{/docs/7-0/tutorials/-/knowledge_base/t/importing-packages}{import}
configuration of your modules. Also, see the
\href{/docs/7-0/tutorials/-/knowledge_base/t/adding-third-party-libraries-to-a-module}{Resolving
Third Party Library Package Dependencies} tutorial for more information
on resolving import errors. Sometimes, this kind of error can be solved
by editing the \texttt{resolve} task's list of capabilities. See the
\href{/docs/7-0/tutorials/-/knowledge_base/t/modifying-the-target-platforms-capabilities\#depending-on-third-party-libraries-not-included-in-product}{Depending
on Third Party Libraries Not Included in Liferay DXP} section to learn
how to do this.

\subsection{Missing Service Reference}\label{missing-service-reference}

If your module references a non-existent service, an error is thrown.
This is helpful because service reference issues are hard to diagnose
during deployment without using the
\href{/docs/7-0/reference/-/knowledge_base/r/using-the-felix-gogo-shell}{Gogo
Shell}.

For example, if your module \texttt{test-portlet} references a service
(e.g., \texttt{test.api.TestApi}) it does not have access to, the
following error is thrown:

\begin{verbatim}
Resolution exception in project 'modules:test-portlet': Unresolved requirements in project 'modules:test-portlet':
    Mandatory:
        [osgi.identity ] test.portlet
        [osgi.service  ] objectClass=test.api.TestApi
\end{verbatim}

To fix this, you must make the service available to your module. If
you're expecting the service to be provided by your target platform,
check to make sure it's being provided. If it's a service provided by a
custom module, check that service provider module and ensure it's
correctly providing that service to your module. To check the target
platform for available services, follow the steps below:

\begin{enumerate}
\def\labelenumi{\arabic{enumi}.}
\item
  Start your target platform instance.
\item
  Start the Gogo shell from a local telnet session (e.g.,
  \texttt{telnet\ localhost\ \ \ \ \ 11311}).
\item
  List all services containing a keyword by running
  \texttt{services\ \textbar{}\ grep\ \ \ \ \ "SERVICE\_NAME"}. It's
  easiest to do this rather than listing all services since there are
  usually too many to sift through.
\item
  You can also list services provided by a component. Run
  \texttt{lb\ -s} to list all provided bundles by their bundle symbolic
  name (BSN). Find the BSN for the desired component and then run
  \texttt{scr:info\ \textless{}BSN\textgreater{}}.
\end{enumerate}

If you're unable to track down your missing service, it may be provided
by a customized Liferay DXP core feature or an external Liferay DXP
feature. If this is the case, it isn't included in the target platform's
default capabilities. You can make the custom service capability
available to reference by
\href{/docs/7-0/tutorials/-/knowledge_base/t/modifying-the-target-platforms-capabilities\#depending-on-a-customized-distribution-of-product}{generating
a new custom distro JAR}.

\subsection{Missing Fragment Host}\label{missing-fragment-host}

Referring to a non-existent fragment host throws an error. For example,
if your \texttt{test.login} fragment is configured to modify a fragment
host named \texttt{com.liferay.login.web} that cannot be referenced, the
following error is thrown:

\begin{verbatim}
Resolution exception in project 'modules:test.login': Unresolved requirements in project 'modules:test-login':
    Mandatory:
        [osgi.identity    ] test.login
        [osgi.wiring.host ] com.liferay.login.web; version=1.0.10
\end{verbatim}

Configuring a fragment host in your module is typically done with the
\texttt{Fragment-Host} header in the \texttt{bnd.bnd} file:

\begin{verbatim}
Fragment-Host: com.liferay.login.web;bundle-version="[1.0.0,1.0.1)"
\end{verbatim}

To fix this, inspect your target platform to ensure it includes the JAR
you're attempting to add a fragment for. Your fragment host header may
be referencing an incorrect bundle symbolic name (BSN) or version. The
easiest way to check this is by using the
\href{/docs/7-0/reference/-/knowledge_base/r/using-the-felix-gogo-shell}{Gogo
Shell}. Follow the steps below to find the bundle symbolic name:

\begin{enumerate}
\def\labelenumi{\arabic{enumi}.}
\item
  Start your target platform instance.
\item
  Start the Gogo shell from a local telnet session (e.g.,
  \texttt{telnet\ localhost\ \ \ \ \ 11311}).
\item
  List all installed bundles by BSN with the command \texttt{lb\ -s}.
  You can search through the output to find the BSN. If you already know
  the BSN and want to check the version, run
  \texttt{lb\ -s\ \textbar{}\ grep\ "\textless{}BSN\textgreater{}"}.
\end{enumerate}

Once you know the correct BSN/version to reference, update your
\texttt{Fragment-Host} header to resolve the error.

For more information on fragments, see the
\href{/docs/7-0/tutorials/-/knowledge_base/t/overriding-a-modules-jsps}{JSP
Overrides Using OSGi Fragments} tutorial.

\section{CKEditor Plugin Reference
Guide}\label{ckeditor-plugin-reference-guide}

This reference guide provides a list of the default CKEditor plugins
bundled with Liferay DXP's AlloyEditor. Each plugin below links to its
\texttt{plugin.js} file for reference:

\begin{itemize}
\tightlist
\item
  \href{https://github.com/ckeditor/ckeditor-dev/tree/master/plugins/about/plugin.js}{about}
\item
  \href{https://github.com/ckeditor/ckeditor-dev/tree/master/plugins/a11yhelp/plugin.js}{allyhelp}
\item
  \href{https://github.com/liferay/liferay-portal/tree/7.0.x/modules/apps/foundation/frontend-editor/frontend-editor-ckeditor-web/src/main/resources/META-INF/resources/_diffs/plugins/a11yhelpbtn/plugin.js}{allyhelpbtn}
\item
  \href{https://github.com/liferay/liferay-portal/tree/7.0.x/modules/apps/foundation/frontend-editor/frontend-editor-ckeditor-web/src/main/resources/META-INF/resources/_diffs/plugins/ajaxsave/plugin.js}{ajaxsave}
\item
  \href{https://github.com/liferay/liferay-portal/tree/7.0.x/modules/apps/foundation/frontend-editor/frontend-editor-ckeditor-web/src/main/resources/META-INF/resources/_diffs/plugins/autocomplete/plugin.js}{autocomplete}
\item
  \href{https://github.com/ckeditor/ckeditor-dev/tree/master/plugins/basicstyles/plugin.js}{basicstyles}
\item
  \href{https://github.com/liferay/liferay-portal/tree/7.0.x/modules/apps/foundation/frontend-editor/frontend-editor-ckeditor-web/src/main/resources/META-INF/resources/_diffs/plugins/bbcode/plugin.js}{bbcode}
\item
  \href{https://github.com/ckeditor/ckeditor-dev/tree/master/plugins/bidi/plugin.js}{bidi}
\item
  \href{https://github.com/ckeditor/ckeditor-dev/tree/master/plugins/blockquote/plugin.js}{blockquote}
\item
  \href{https://github.com/ckeditor/ckeditor-dev/tree/master/plugins/clipboard/plugin.js}{clipboard}
\item
  \href{https://github.com/ckeditor/ckeditor-dev/tree/master/plugins/colorbutton/plugin.js}{colorbutton}
\item
  \href{https://github.com/ckeditor/ckeditor-dev/tree/master/plugins/colordialog/plugin.js}{colordialog}
\item
  \href{https://github.com/ckeditor/ckeditor-dev/blob/master/plugins/contextmenu/plugin.js}{contextmenu}
\item
  \href{https://github.com/liferay/liferay-portal/blob/7.0.x/modules/apps/foundation/frontend-editor/frontend-editor-ckeditor-web/src/main/resources/META-INF/resources/_diffs/plugins/creole/plugin.js}{creole}
\item
  \href{https://github.com/ckeditor/ckeditor-dev/blob/master/plugins/dialogadvtab/plugin.js}{dialogadvtab}
\item
  \href{https://github.com/ckeditor/ckeditor-dev/blob/master/plugins/div/plugin.js}{div}
\item
  \href{https://github.com/ckeditor/ckeditor-dev/blob/master/plugins/elementspath/plugin.js}{elementspath}
\item
  \href{https://github.com/ckeditor/ckeditor-dev/blob/master/plugins/enterkey/plugin.js}{enterkey}
\item
  \href{https://github.com/ckeditor/ckeditor-dev/blob/master/plugins/entities/plugin.js}{entities}
\item
  \href{https://github.com/ckeditor/ckeditor-dev/blob/master/plugins/filebrowser/plugin.js}{filebrowse}
\item
  \href{https://github.com/ckeditor/ckeditor-dev/blob/master/plugins/find/plugin.js}{find}
\item
  \href{https://github.com/ckeditor/ckeditor-dev/blob/master/plugins/flash/plugin.js}{flash}
\item
  \href{https://github.com/ckeditor/ckeditor-dev/blob/master/plugins/floatingspace/plugin.js}{floatingspace}
\item
  \href{https://github.com/ckeditor/ckeditor-dev/blob/master/plugins/font/plugin.js}{font}
\item
  \href{https://github.com/ckeditor/ckeditor-dev/blob/master/plugins/forformat/plugin.js}{format}
\item
  \href{https://github.com/ckeditor/ckeditor-dev/blob/master/plugins/forms/plugin.js}{forms}
\item
  \href{https://github.com/ckeditor/ckeditor-dev/blob/master/plugins/horizontalrule/plugin.js}{horizontalrule}
\item
  \href{https://github.com/ckeditor/ckeditor-dev/blob/master/plugins/htmlwriter/plugin.js}{htmlwriter}
\item
  \href{https://github.com/ckeditor/ckeditor-dev/blob/master/plugins/image/plugin.js}{image}
\item
  \href{https://github.com/ckeditor/ckeditor-dev/blob/master/plugins/iframe/plugin.js}{iframe}
\item
  \href{https://github.com/ckeditor/ckeditor-dev/blob/master/plugins/indindent/plugin.js}{indent}
\item
  \href{https://github.com/liferay/liferay-portal/blob/7.0.x/modules/apps/foundation/frontend-editor/frontend-editor-ckeditor-web/src/main/resources/META-INF/resources/_diffs/plugins/itemselector/plugin.js}{itemselector}
\item
  \href{https://github.com/ckeditor/ckeditor-dev/blob/master/plugins/justify/plugin.js}{justify}
\item
  \href{https://github.com/ckeditor/ckeditor-dev/blob/master/plugins/link/plugin.js}{link}
\item
  \href{https://github.com/ckeditor/ckeditor-dev/blob/master/plugins/list/plugin.js}{list}
\item
  \href{https://github.com/ckeditor/ckeditor-dev/blob/master/plugins/liststyle/plugin.js}{liststyle}
\item
  \href{https://github.com/liferay/liferay-portal/blob/7.0.x/modules/apps/foundation/frontend-editor/frontend-editor-ckeditor-web/src/main/resources/META-INF/resources/_diffs/plugins/lfrpopup/plugin.js}{lfrpopup}
\item
  \href{https://github.com/ckeditor/ckeditor-dev/blob/master/plugins/magicline/plugin.js}{magicline}
\item
  \href{https://github.com/liferay/liferay-portal/blob/7.0.x/modules/apps/foundation/frontend-editor/frontend-editor-ckeditor-web/src/main/resources/META-INF/resources/_diffs/plugins/media/plugin.js}{media}
\item
  \href{https://github.com/ckeditor/ckeditor-dev/blob/master/plugins/newpage/plugin.js}{newpage}
\item
  \href{https://github.com/ckeditor/ckeditor-dev/blob/master/plugins/pagebreak/plugin.js}{pagebreak}
\item
  \href{https://github.com/ckeditor/ckeditor-dev/blob/master/plugins/pastefromword/plugin.js}{pastefromword}
\item
  \href{https://github.com/ckeditor/ckeditor-dev/blob/master/plugins/pastetext/plugin.js}{pastetext}
\item
  \href{https://github.com/ckeditor/ckeditor-dev/blob/master/plugins/preview/plugin.js}{preview}
\item
  \href{https://github.com/ckeditor/ckeditor-dev/blob/master/plugins/removeformat/plugin.js}{removeformat}
\item
  \href{https://github.com/ckeditor/ckeditor-dev/blob/master/plugins/resize/plugin.js}{resize}
\item
  \href{https://github.com/liferay/liferay-portal/blob/7.0.x/modules/apps/foundation/frontend-editor/frontend-editor-ckeditor-web/src/main/resources/META-INF/resources/_diffs/plugins/restore/plugin.js}{restore}
\item
  \href{https://github.com/ckeditor/ckeditor-dev/blob/master/plugins/selectall/plugin.js}{selectall}
\item
  \href{https://github.com/ckeditor/ckeditor-dev/blob/master/plugins/showblocks/plugin.js}{showblocks}
\item
  \href{https://github.com/ckeditor/ckeditor-dev/blob/master/plugins/showborders/plugin.js}{showborders}
\item
  \href{https://github.com/ckeditor/ckeditor-dev/blob/master/plugins/smiley/plugin.js}{smiley}
\item
  \href{https://github.com/ckeditor/ckeditor-dev/blob/master/plugins/sourcearea/plugin.js}{sourcearea}
\item
  \href{https://github.com/ckeditor/ckeditor-dev/blob/master/plugins/specialchar/plugin.js}{specialchar}
\item
  \href{https://github.com/ckeditor/ckeditor-dev/blob/master/plugins/stylescombo/plugin.js}{stylescombo}
\item
  \href{https://github.com/ckeditor/ckeditor-dev/blob/master/plugins/tab/plugin.js}{tab}
\item
  \href{https://github.com/ckeditor/ckeditor-dev/blob/master/plugins/table/plugin.js}{table}
\item
  \href{https://github.com/ckeditor/ckeditor-dev/blob/master/plugins/tabletools/plugin.js}{tabletools}
\item
  \href{https://github.com/ckeditor/ckeditor-dev/blob/master/plugins/templates/plugin.js}{templates}
\item
  \href{https://github.com/ckeditor/ckeditor-dev/blob/master/plugins/toolbar/plugin.js}{toolbar}
\item
  \href{https://github.com/ckeditor/ckeditor-dev/blob/master/plugins/undo/plugin.js}{undo}
\item
  \href{https://github.com/liferay/liferay-portal/blob/7.0.x/modules/apps/foundation/frontend-editor/frontend-editor-ckeditor-web/src/main/resources/META-INF/resources/_diffs/plugins/wikilink/plugin.js}{wikilink}
\item
  \href{https://github.com/ckeditor/ckeditor-dev/blob/master/plugins/wysiwygarea/plugin.js}{wysiwygarea}
\end{itemize}

\noindent\hrulefill

\textbf{Note:} The following CKEditor plugins are not available for
inline mode in AlloyEditor at this time; however, you can still use them
in the classic CKEditor:

\begin{itemize}
\tightlist
\item
  \href{https://github.com/ckeditor/ckeditor-dev/blob/master/plugins/maximize/plugin.js}{maximize}
\item
  \href{https://github.com/ckeditor/ckeditor-dev/blob/master/plugins/print/plugin.js}{print}
\item
  \href{https://github.com/ckeditor/ckeditor-dev/blob/master/plugins/save/plugin.js}{save}
\end{itemize}

To use the Classic CKEditor instead of AlloyEditor, there are a few
properties you can use, depending on the portlet. Add the
\href{https://github.com/liferay/liferay-portal/blob/7.0.x/portal-impl/src/portal.properties\#L6030-L6039}{properties}
that you need to your \texttt{portal-ext.properties} file:

\begin{verbatim}
 editor.wysiwyg.default=ckeditor
 editor.wysiwyg.portal-impl.portlet.ddm.text_html.ftl=ckeditor
 editor.wysiwyg.portal-web.docroot.html.portlet.announcements.edit_entry.jsp=ckeditor
 editor.wysiwyg.portal-web.docroot.html.portlet.blogs.edit_entry.jsp=ckeditor
 editor.wysiwyg.portal-web.docroot.html.portlet.mail.edit.jsp=ckeditor
 editor.wysiwyg.portal-web.docroot.html.portlet.mail.edit_message.jsp=ckeditor
 editor.wysiwyg.portal-web.docroot.html.portlet.message_boards.edit_message.html.jsp=ckeditor
 editor.wysiwyg.portal-web.docroot.html.taglib.ui.discussion.jsp=ckeditor
 editor.wysiwyg.portal-web.docroot.html.taglib.ui.email_notification_settings.jsp=ckeditor
\end{verbatim}

\section{Item Selector Criterion and Return
Types}\label{item-selector-criterion-and-return-types}

Liferay DXP bundles have apps and app suites containing
\href{@app-ref@/collaboration/latest/javadocs/com/liferay/item/selector/ItemSelectorCriterion.html}{\texttt{ItemSelectorCriterion}
classes} and
\href{@app-ref@/collaboration/latest/javadocs/com/liferay/item/selector/ItemSelectorReturnType.html}{\texttt{ItemSelectorReturnType}
classes} developers can use.

\subsection{Item Selector Criterion
Classes}\label{item-selector-criterion-classes}

\textbf{Collaboration App Suite Modules:}

\begin{itemize}
\item
  \texttt{com.liferay.item.selector.criteria.api}:

  \begin{itemize}
  \item
    \href{@app-ref@/collaboration/latest/javadocs/com/liferay/item/selector/criteria/image/criterion/ImageItemSelectorCriterion.html}{ImageItemSelectorCriterion}:
    Image file entity type.
  \item
    \href{@app-ref@/collaboration/latest/javadocs/com/liferay/item/selector/criteria/audio/criterion/AudioItemSelectorCriterion.html}{AudioItemSelectorCriterion}:
    Audio file entity type.
  \item
    \href{@app-ref@/collaboration/latest/javadocs/com/liferay/item/selector/criteria/criteria/file/criterion/FileItemSelectorCriterion.html}{FileItemSelectorCriterion}:
    Document Library file entity type.
  \item
    \href{@app-ref@/collaboration/latest/javadocs/com/liferay/item/selector/criteria/upload/criterion/UploadItemSelectorCriterion.html}{UploadItemSelectorCriterion}:
    Uploadable file entity type.
  \item
    \href{@app-ref@/collaboration/latest/javadocs/com/liferay/item/selector/criteria/url/criterion/URLItemSelectorCriterion.html}{URLItemSelectorCriterion}:
    URL entity type.
  \item
    \href{@app-ref@/collaboration/latest/javadocs/com/liferay/item/selector/criteria/video/criterion/VideoItemSelectorCriterion.html}{VideoItemSelectorCriterion}:
    Video file entity type.
  \end{itemize}
\item
  \href{@app-ref@/collaboration/latest/javadocs/com/liferay/wiki/item/selector/criterion/package-summary.html}{\texttt{com.liferay.wiki.api}}
  has wiki criterion.
\end{itemize}

\textbf{Web Experience App Suite Modules:}

\begin{itemize}
\item
  \href{@app-ref@/web-experience/latest/javadocs/com/liferay/site/item/selector/criterion/package-summary.html}{\texttt{com.liferay.site.item.selector.api}}
  has site criterion.
\item
  \href{@app-ref@/web-experience/latest/javadocs/com/liferay/layout/item/selector/criterion/package-summary.html}{\texttt{com.liferay.layout.item.selector.api}}
  has layout criterion.
\item
  \href{@app-ref@/web-experience/latest/javadocs/com/liferay/journal/item/selector/criterion/package-summary.html}{\texttt{com.liferay.journal.item.selector.api}}
  has web content criterion.
\end{itemize}

If there's no criterion class for your entity, you can create your own
\href{@app-ref@/collaboration/latest/javadocs/com/liferay/item/selector/ItemSelectorCriterion.html}{\texttt{ItemSelectorCriterion}
class} (tutorial coming soon).

\subsection{Item Selector Return Type
Classes}\label{item-selector-return-type-classes}

The Liferay Collaboration app suite's
\href{@app-ref@/collaboration/latest/javadocs/com/liferay/item/selector/criteria/package-summary.html}{\texttt{com.liferay.item.selector.criteria.api}
module} includes the following return types:

\begin{itemize}
\item
  \href{@app-ref@/collaboration/latest/javadocs/com/liferay/item/selector/criteria/Base64ItemSelectorReturnType.html}{Base64ItemSelectorReturnType}:
  Base64 encoding of the entity as a \texttt{String}.
\item
  \href{@app-ref@/collaboration/latest/javadocs/com/liferay/item/selector/criteria/FileEntryItemSelectorReturnType.html}{FileEntryItemSelectorReturnType}:
  File entry information as a JSON object.
\item
  \href{@app-ref@/collaboration/latest/javadocs/com/liferay/item/selector/criteria/URLItemSelectorReturnType.html}{URLItemSelectorReturnType}:
  URL of the entity as a \texttt{String}.
\item
  \href{@app-ref@/collaboration/latest/javadocs/com/liferay/item/selector/criteria/UUIDItemSelectorReturnType.html}{UUIDItemSelectorReturnType}:
  Universally Unique Identifier (UUID) of the entity as a
  \texttt{String}.
\end{itemize}

If there's no return type class that meets your needs, you can implement
your own
\href{@app-ref@/collaboration/latest/javadocs/com/liferay/item/selector/ItemSelectorReturnType.html}{\texttt{ItemSelectorReturnType}
class} (tutorial coming soon).

\section{Breaking Changes}\label{breaking-changes}

This document presents a chronological list of changes that break
existing functionality, APIs, or contracts with third party Liferay
developers or users. We try our best to minimize these disruptions, but
sometimes they are unavoidable.

The breaking changes covered in this article apply to both the
commercial and open source versions of Liferay.

Here are some of the types of changes documented in this file:

\begin{itemize}
\tightlist
\item
  Functionality that is removed or replaced
\item
  API incompatibilities: Changes to public Java or JavaScript APIs
\item
  Changes to context variables available to templates
\item
  Changes in CSS classes available to Liferay themes and portlets
\item
  Configuration changes: Changes in configuration files, like
  \texttt{portal.properties}, \texttt{system.properties}, etc.
\item
  Execution requirements: Java version, J2EE Version, browser versions,
  etc.
\item
  Deprecations or end of support: For example, warning that a certain
  feature or API will be dropped in an upcoming version.
\item
  Recommendations: For example, recommending using a newly introduced
  API that replaces an old API, in spite of the old API being kept in
  Liferay Portal for backwards compatibility.
\end{itemize}

\subsection{Breaking Changes List}\label{breaking-changes-list-1}

\subsubsection{The liferay-ui:logo-selector Tag Requires Parameter
Changes}\label{the-liferay-uilogo-selector-tag-requires-parameter-changes}

\begin{itemize}
\tightlist
\item
  \textbf{Date:} 2013-Dec-05
\item
  \textbf{JIRA Ticket:} LPS-42645
\end{itemize}

\paragraph{What changed?}\label{what-changed-10}

The Logo Selector tag now supports uploading an image, storing it as a
temporary file, cropping it, and canceling edits. The tag no longer
requires creating a UI to include the image. Consequently, the
\texttt{editLogoURL} parameter is no longer needed and has been removed.
The tag now uses the following parameters to support the new features:

\begin{itemize}
\tightlist
\item
  \texttt{currentLogoURL}: the URL to display the image being stored
\item
  \texttt{maxFileSize}: the size limit for the logo to be uploaded
\item
  \texttt{tempImageFileName}: the unique identifier to store the
  temporary image on upload
\end{itemize}

\paragraph{Who is affected?}\label{who-is-affected-10}

Plugins or templates that are using the
\texttt{liferay-ui:logo-selector} tag need to update their usage of the
tag.

\paragraph{How should I update my
code?}\label{how-should-i-update-my-code-10}

You should remove the parameter \texttt{editLogoURL} and include (if
neccessary) the parameters \texttt{currentLogoURL},
\texttt{maxFileSize}, and/or \texttt{tempImageFileName}.

\textbf{Example}

Old way:

\begin{verbatim}
<portlet:renderURL var="editUserPortraitURL" windowState="<%= LiferayWindowState.POP_UP.toString() %>">
    <portlet:param name="struts_action" value="/users_admin/edit_user_portrait" />
    <portlet:param name="redirect" value="<%= currentURL %>" />
    <portlet:param name="p_u_i_d" value="<%= String.valueOf(selUser.getUserId()) %>" />
    <portlet:param name="portrait_id" value="<%= String.valueOf(selUser.getPortraitId()) %>" />
</portlet:renderURL>

<liferay-ui:logo-selector
    defaultLogoURL="<%= UserConstants.getPortraitURL(themeDisplay.getPathImage(), selUser.isMale(), 0) %>"
    editLogoURL="<%= editUserPortraitURL %>"
    imageId="<%= selUser.getPortraitId() %>"
    logoDisplaySelector=".user-logo"
/>
\end{verbatim}

New way:

\begin{verbatim}
<liferay-ui:logo-selector
    currentLogoURL="<%= selUser.getPortraitURL(themeDisplay) %>"
    defaultLogoURL="<%= UserConstants.getPortraitURL(themeDisplay.getPathImage(), selUser.isMale(), 0) %>"
    imageId="<%= selUser.getPortraitId() %>"
    logoDisplaySelector=".user-logo"
    maxFileSize="<%= PrefsPropsUtil.getLong(PropsKeys.USERS_IMAGE_MAX_SIZE) / 1024 %>"
    tempImageFileName="<%= String.valueOf(selUser.getUserId()) %>"
/>
\end{verbatim}

\paragraph{Why was this change made?}\label{why-was-this-change-made-10}

This change helps keep a unified UI and consistent experience for
uploading logos in the portal. The logos can be customized from a single
location and used throughout the portal. In addition, the change adds
new features such as image cropping and support for canceling image
upload.

\subsubsection{Merged Configured Email Signature Field into the Body of
Email Messages from Message Boards and
Wiki}\label{merged-configured-email-signature-field-into-the-body-of-email-messages-from-message-boards-and-wiki}

\begin{itemize}
\tightlist
\item
  \textbf{Date:} 2014-Feb-28
\item
  \textbf{JIRA Ticket:} LPS-44599
\end{itemize}

\paragraph{What changed?}\label{what-changed-11}

The configuration for email signatures of notifications from Message
Boards and Wiki has been removed. An automatic update process is
available that appends existing signatures into respective email message
bodies for Message Boards and Wiki notifications. The upgrade process
only applies to configured signatures in the database. In case you
declared signatures in portal properties (e.g.,
\texttt{portal-ext.properties}), you must make the manual changes
explained below.

\paragraph{Who is affected?}\label{who-is-affected-11}

Users and system administrators who have configured email signatures for
Message Boards or Wiki notifications are affected. System administrators
who have configured portal properties (e.g.,
\texttt{portal-ext.properties}) must make the manual changes described
below.

\paragraph{How should I update my
code?}\label{how-should-i-update-my-code-11}

You should modify your \texttt{portal-ext.properties} file to remove the
properties \texttt{message.boards.email.message.added.signature},
\texttt{message.boards.email.message.updated.signature},
\texttt{wiki.email.page.added.signature}, and
\texttt{wiki.email.page.updated.signature}. Then, you should append the
contents of the signatures to the bodies you had previously configured
in your \texttt{portal-ext.properties} file.

\textbf{Example}

Old way:

\begin{verbatim}
wiki.email.page.updated.body=A wiki page was updated.
wiki.email.page.updated.signature=For any doubts email the system administrator
\end{verbatim}

New way:

\begin{verbatim}
wiki.email.page.updated.body=A wiki page was updated.\n--\nFor any doubts email the system administrator
\end{verbatim}

\paragraph{Why was this change made?}\label{why-was-this-change-made-11}

This change helps simplify the user interface. The signatures can still
be set inside the message body. There was no real benefit in keeping the
signature and body fields separate.

\subsubsection{Removed get and format Methods that Used PortletConfig
Parameters}\label{removed-get-and-format-methods-that-used-portletconfig-parameters}

\begin{itemize}
\tightlist
\item
  \textbf{Date:} 2014-Mar-07
\item
  \textbf{JIRA Ticket:} LPS-44342
\end{itemize}

\paragraph{What changed?}\label{what-changed-12}

All the methods \texttt{get()} and \texttt{format()} which had the
PortletConfig as a parameter have been removed.

\paragraph{Who is affected?}\label{who-is-affected-12}

Any invocations from Java classes or JSPs to these methods in
\texttt{LanguageUtil} and \texttt{UnicodeLanguageUtil} are affected.

\paragraph{How should I update my
code?}\label{how-should-i-update-my-code-12}

Replace invocations to these methods with invocations to methods of the
same name that take a \texttt{ResourceBundle} parameter, instead of
taking a \texttt{PortletConfig} parameter.

\textbf{Example}

Old call:

\begin{verbatim}
LanguageUtil.get(portletConfig, locale, key);
\end{verbatim}

New call:

\begin{verbatim}
LanguageUtil.get(portletConfig.getResourceBundle(locale), key);
\end{verbatim}

\paragraph{Why was this change made?}\label{why-was-this-change-made-12}

The removed methods didn't work properly and would never work properly,
since they didn't have all the information they required. Since we
expected the methods were rarely used, we thought it better to remove
them without deprecation than to leave them as buggy methods in the API.

\subsubsection{Web Content Articles Now Require a Structure and
Template}\label{web-content-articles-now-require-a-structure-and-template}

\begin{itemize}
\tightlist
\item
  \textbf{Date:} 2014-Mar-18
\item
  \textbf{JIRA Ticket:} LPS-45107
\end{itemize}

\paragraph{What changed?}\label{what-changed-13}

Web content is now required to use a structure and template. A default
structure and template named \emph{Basic Web Content} was added to the
global scope, and can be modified or deleted.

\paragraph{Who is affected?}\label{who-is-affected-13}

Applications that use the Journal API to create web content without a
structure or template are affected.

\paragraph{How should I update my
code?}\label{how-should-i-update-my-code-13}

You should always use a structure and template when creating web
content. You can still use the \emph{Basic Web Content} from the global
scope (using the structure key \texttt{basic-web-content}), but you
should keep in mind that users can modify or delete it.

\paragraph{Why was this change made?}\label{why-was-this-change-made-13}

This change gives users the flexibility to modify the default structure
and template.

\subsubsection{Changed the AssetRenderer and Indexer APIs to Include the
PortletRequest and PortletResponse
Parameters}\label{changed-the-assetrenderer-and-indexer-apis-to-include-the-portletrequest-and-portletresponse-parameters}

\begin{itemize}
\tightlist
\item
  \textbf{Date:} 2014-May-07
\item
  \textbf{JIRA Ticket:} LPS-44639 and LPS-44894
\end{itemize}

\paragraph{What changed?}\label{what-changed-14}

The \texttt{getSummary()} method in the AssetRenderer API and the
\texttt{doGetSummary()} method in the Indexer API have changed and must
include a \texttt{PortletRequest} and \texttt{PortletResponse} parameter
as part of their signatures.

\paragraph{Who is affected?}\label{who-is-affected-14}

These methods must be updated in all AssetRenderer and Indexer
implementations.

\paragraph{How should I update my
code?}\label{how-should-i-update-my-code-14}

Add a \texttt{PortletRequest} and \texttt{PortletResponse} parameter to
the signatures of these methods.

\textbf{Example 1}

Old signature:

\begin{verbatim}
protected Summary doGetSummary(Document document, Locale locale, String snippet, PortletURL portletURL)
\end{verbatim}

New signature:

\begin{verbatim}
protected Summary doGetSummary(Document document, Locale locale, String snippet, PortletRequest portletRequest, PortletResponse portletResponse)
\end{verbatim}

\textbf{Example 2}

Old signature:

\begin{verbatim}
public String getSummary(Locale locale)
\end{verbatim}

New signature:

\begin{verbatim}
public String getSummary(PortletRequest portletRequest, PortletResponse portletResponse)
\end{verbatim}

\paragraph{Why was this change made?}\label{why-was-this-change-made-14}

Some content (such as web content) needs the \texttt{PortletRequest} and
\texttt{PortletResponse} parameters in order to be rendered.

\subsubsection{Only One Portlet Instance's Settings is Used Per
Portlet}\label{only-one-portlet-instances-settings-is-used-per-portlet}

\begin{itemize}
\tightlist
\item
  \textbf{Date:} 2014-Jun-06
\item
  \textbf{JIRA Ticket:} LPS-43134
\end{itemize}

\paragraph{What changed?}\label{what-changed-15}

Previously, some portlets allowed separate setups per portlet instance,
regardless of whether the instances were in the same page or in
different pages. For some of the portlet setup fields, however, it
didn't make sense to allow different values in different instances. The
flexibility of these fields was unnecessary and confused users. As part
of this change, these fields have been moved from portlet instance setup
to Site Administration.

The upgrade process takes care of making the necessary database changes.
In the case of several portlet instances having different
configurations, however, only one configuration is preserved.

For example, if you configured three Bookmarks portlets where the mail
configuration was the same, upgrade will be the same and you won't have
any problem. But if you configured the three portlet instances
differently, only one configuration will be chosen. To find out which
configuration is chosen, you can check the log generated in the console
by the upgrade process.

Since configuring instances of the same portlet type differently is
highly discouraged and notoriously problematic, we expect this change
will inconvenience only a very low minority of portal users.

\paragraph{Who is affected?}\label{who-is-affected-15}

Affected users are those who have specified varying configurations for
multiple portlet instances of a portlet type, that stores configurations
at the layout level.

\paragraph{How should I update my
code?}\label{how-should-i-update-my-code-15}

The upgrade process chooses one portlet instance's configurations and
stores it at the service level. After the upgrade, you should review the
portlet's configuration and make any necessary modifications.

\paragraph{Why was this change made?}\label{why-was-this-change-made-15}

Unifying portlet and service configuration facilitates managing them.

\subsubsection{DDM Structure Local Service API No Longer Has the
updateXSDFieldMetadata
operation}\label{ddm-structure-local-service-api-no-longer-has-the-updatexsdfieldmetadata-operation}

\begin{itemize}
\tightlist
\item
  \textbf{Date:} 2014-Jun-11
\item
  \textbf{JIRA Ticket:} LPS-47559
\end{itemize}

\paragraph{What changed?}\label{what-changed-16}

The \texttt{updateXSDFieldMetadata()} operation was removed from the DDM
Structure Local Service API.

DDM Structure Local API users should reference a structure's internal
representation; any call to modify a DDM structure's content should be
done through the DDMForm model.

\paragraph{Who is affected?}\label{who-is-affected-16}

Applications that use the DDM Structure Local Service API might be
affected.

\paragraph{How should I update my
code?}\label{how-should-i-update-my-code-16}

You should always use DDMForm to update the DDM Structure content. You
can retrieve it by calling \texttt{ddmStructure.getDDMForm()}. Perform
any changes to it and then call
\texttt{DDMStructureLocalServiceUtil.updateDDMStructure(ddmStructure)}.

\paragraph{Why was this change made?}\label{why-was-this-change-made-16}

This change gives users the flexibility to modify the structure content
without concerning themselves with the DDM Structure's internal content
representation of data.

\subsubsection{The aui:input Tag for Type checkbox No Longer Creates a
Hidden
Input}\label{the-auiinput-tag-for-type-checkbox-no-longer-creates-a-hidden-input}

\begin{itemize}
\tightlist
\item
  \textbf{Date:} 2014-Jun-16
\item
  \textbf{JIRA Ticket:} LPS-44228
\end{itemize}

\paragraph{What changed?}\label{what-changed-17}

Whenever the aui:input tag is used to generate an input of type
checkbox, only an input tag will be generated, instead of the checkbox
and hidden field it was generating before.

\paragraph{Who is affected?}\label{who-is-affected-17}

Anyone trying to grab the previously generated fields is affected. The
change mostly affects JavaScript code trying to add some additional
actions when clicking on the checkboxes.

\paragraph{How should I update my
code?}\label{how-should-i-update-my-code-17}

In your front-end JavaScript code, follow these steps:

\begin{itemize}
\tightlist
\item
  Remove the \texttt{Checkbox} suffix when querying for the node in any
  of its forms, like \texttt{A.one(...)}, \texttt{\$(...)}, etc.
\item
  Remove any action that tries to set the value of the checkbox on the
  previously generated hidden field.
\end{itemize}

\paragraph{Why was this change made?}\label{why-was-this-change-made-17}

This change makes generated forms more standard and interoperable since
it falls back to the checkboxes default behavior. It allows the form to
be submitted properly even when JavaScript is disabled.

\subsubsection{Using util-taglib No Longer Binds You to Using
portal-kernel's javax.servlet.jsp
Implementation}\label{using-util-taglib-no-longer-binds-you-to-using-portal-kernels-javax.servlet.jsp-implementation}

\begin{itemize}
\tightlist
\item
  \textbf{Date:} 2014-Jun-19
\item
  \textbf{JIRA Ticket:} LPS-47682
\end{itemize}

\paragraph{What changed?}\label{what-changed-18}

Several APIs in \texttt{portal-kernel.jar} contained references to the
\texttt{javax.servlet.jsp} package. This forced \texttt{util-taglib},
which depended on many of the package's features, to be bound to the
same JSP implementation.

Due to this, the following APIs had breaking changes:

\begin{itemize}
\tightlist
\item
  \texttt{LanguageUtil}
\item
  \texttt{UnicodeLanguageUtil}
\item
  \texttt{VelocityTaglibImpl}
\item
  \texttt{ThemeUtil}
\item
  \texttt{RuntimePageUtil}
\item
  \texttt{PortletDisplayTemplateUtil}
\item
  \texttt{DDMXSDUtil}
\item
  \texttt{PortletResourceBundles}
\item
  \texttt{ResourceActionsUtil}
\item
  \texttt{PortalUtil}
\end{itemize}

\paragraph{Who is affected?}\label{who-is-affected-18}

This affects anyone calling the classes listed above.

\paragraph{How should I update my
code?}\label{how-should-i-update-my-code-18}

Code invoking the APIs listed above should be updated to use an
\texttt{HttpServletRequest} parameter instead of the formerly used
\texttt{PageContext} parameter.

\paragraph{Why was this change made?}\label{why-was-this-change-made-18}

As stated previously, the use of the \texttt{javax.servlet.jsp} API in
\texttt{portal-kernel} prevented the use of any other JSP impl within
plugins (OSGi or otherwise). This limited what Liferay could change with
respect to providing its own JSP implementation within OSGi.

\subsubsection{Changes in Exceptions Thrown by User
Services}\label{changes-in-exceptions-thrown-by-user-services}

\begin{itemize}
\tightlist
\item
  \textbf{Date:} 2014-Jul-03
\item
  \textbf{JIRA Ticket:} LPS-47130
\end{itemize}

\paragraph{What changed?}\label{what-changed-19}

In order to provide more information about the root cause of an
exception, several exceptions have been extended with static inner
classes, one for each cause. As a result of this effort, some exceptions
have been identified that really belong as static inner subclasses of
existing exceptions.

\paragraph{Who is affected?}\label{who-is-affected-19}

Client code which is handling any of the following exceptions:

\begin{itemize}
\tightlist
\item
  \texttt{DuplicateUserScreenNameException}
\item
  \texttt{DuplicateUserEmailAddressException}
\end{itemize}

\paragraph{How should I update my
code?}\label{how-should-i-update-my-code-19}

Replace the old exception with the equivalent inner class exception as
follows:

\begin{itemize}
\tightlist
\item
  \texttt{DuplicateUserScreenNameException} →
  \texttt{UserScreenNameException.MustNotBeDuplicate}
\item
  \texttt{DuplicateUserEmailAddressException} →
  \texttt{UserEmailAddressException.MustNotBeDuplicate}
\end{itemize}

\paragraph{Why was this change made?}\label{why-was-this-change-made-19}

This change provides more information to clients of the services API
about the root cause of an error. It provides a more helpful error
message to the end-user and it allows for easier recovery, when
possible.

\subsubsection{Removed Trash Logic from DLAppHelperLocalService
Methods}\label{removed-trash-logic-from-dlapphelperlocalservice-methods}

\begin{itemize}
\tightlist
\item
  \textbf{Date:} 2014-Jul-22
\item
  \textbf{JIRA Ticket:} LPS-47508
\end{itemize}

\paragraph{What changed?}\label{what-changed-20}

The \texttt{deleteFileEntry()} and \texttt{deleteFolder()} methods in
\texttt{DLAppHelperLocalService} deleted the corresponding trash entry
in the database. This logic has been removed from these methods.

\paragraph{Who is affected?}\label{who-is-affected-20}

Every caller of the \texttt{deleteFileEntry()} and
\texttt{deleteFolder()} methods is affected.

\paragraph{How should I update my
code?}\label{how-should-i-update-my-code-20}

There is no direct replacement. Trash operations are now accessible
through the \texttt{TrashCapability} implementations for each
repository. The following code demonstrates using a
\texttt{TrashCapability} instance to delete a \texttt{FileEntry}:

\begin{verbatim}
Repository repository = getRepository();

TrashCapability trashCapability = repository.getCapability(
    TrashCapability.class);

FileEntry fileEntry = repository.getFileEntry(fileEntryId);

trashCapability.deleteFileEntry(fileEntry);
\end{verbatim}

Note that the \texttt{deleteFileEntry()} and \texttt{deleteFolder()}
methods in \texttt{TrashCapability} not only remove the trash entry, but
also remove the folder or file entry itself, and any associated data,
such as assets, previews, etc.

\paragraph{Why was this change made?}\label{why-was-this-change-made-20}

This change was made to allow different kinds of repositories to support
trash operations in a uniform way.

\subsubsection{Removed Sync Logic from DLAppHelperLocalService
Methods}\label{removed-sync-logic-from-dlapphelperlocalservice-methods}

\begin{itemize}
\tightlist
\item
  \textbf{Date:} 2014-Sep-05
\item
  \textbf{JIRA Ticket:} LPS-48895
\end{itemize}

\paragraph{What changed?}\label{what-changed-21}

The \texttt{moveFileEntry()} and \texttt{moveFolder()} methods in
\texttt{DLAppHelperLocalService} fired Liferay Sync events. These
methods have been removed.

\paragraph{Who is affected?}\label{who-is-affected-21}

Every caller of the \texttt{moveFileEntry()} and \texttt{moveFolder()}
methods is affected.

\paragraph{How should I update my
code?}\label{how-should-i-update-my-code-21}

There is no direct replacement. Sync operations are now accessible
through the \texttt{SyncCapability} implementations for each repository.
The following code demonstrates using a \texttt{SyncCapability} instance
to move a \texttt{FileEntry}:

\begin{verbatim}
Repository repository = getRepository();

SyncCapability syncCapability = repository.getCapability(
    SyncCapability.class);

FileEntry fileEntry = repository.getFileEntry(fileEntryId);

syncCapability.moveFileEntry(fileEntry);
\end{verbatim}

\paragraph{Why was this change made?}\label{why-was-this-change-made-21}

There are repositories that don't support Liferay Sync operations.

\subsubsection{Removed the .aui Namespace from
Bootstrap}\label{removed-the-.aui-namespace-from-bootstrap}

\begin{itemize}
\tightlist
\item
  \textbf{Date:} 2014-Sep-26
\item
  \textbf{JIRA Ticket:} LPS-50348
\end{itemize}

\paragraph{What changed?}\label{what-changed-22}

The \texttt{.aui} namespace was removed from prefixing all of
Bootstrap's CSS.

\paragraph{Who is affected?}\label{who-is-affected-22}

Theme and plugin developers that targeted their CSS to rely on the
namespace are affected.

\paragraph{How should I update my
code?}\label{how-should-i-update-my-code-22}

Theme developers can still manually add an \texttt{aui.css} file in
their \texttt{\_diffs} directory, and add it back in. The \texttt{aui}
CSS class can also be added to the \texttt{\$root\_css\_class} variable.

\paragraph{Why was this change made?}\label{why-was-this-change-made-22}

Due to changes in the Sass parser, the nesting of third-party libraries
was causing some syntax errors which broke other functionality (e.g.,
RTL conversion). There was also a lot of additional complexity for a
relatively minor benefit.

\subsubsection{Moved MVCPortlet, ActionCommand and ActionCommandCache
from util-bridges.jar to
portal-kernel.jar}\label{moved-mvcportlet-actioncommand-and-actioncommandcache-from-util-bridges.jar-to-portal-kernel.jar}

\begin{itemize}
\tightlist
\item
  \textbf{Date:} 2014-Sep-26
\item
  \textbf{JIRA Ticket:} LPS-50156
\end{itemize}

\paragraph{What changed?}\label{what-changed-23}

The classes from package \texttt{com.liferay.util.bridges.mvc} in
\texttt{util-bridges.jar} were moved to a new package
\texttt{com.liferay.portal.kernel.portlet.bridges.mvc} in
\texttt{portal-kernel.jar}.

Old classes:

\begin{verbatim}
com.liferay.util.bridges.mvc.ActionCommand
com.liferay.util.bridges.mvc.BaseActionCommand
\end{verbatim}

New classes:

\begin{verbatim}
com.liferay.portal.kernel.portlet.bridges.mvc.BaseMVCActionCommand
com.liferay.portal.kernel.portlet.bridges.mvc.MVCActionCommand
\end{verbatim}

In addition, \texttt{com.liferay.util.bridges.mvc.MVCPortlet} is
deprecated, but was made to extend
\texttt{com.liferay.portal.kernel.portlet.bridges.mvc.MVCPortlet}.

The classes in the
\texttt{com.liferay.portal.kernel.portlet.bridges.mvc} package have been
renamed to add the \texttt{MVC} prefix. These modifications were made
after this breaking change, and can be referenced in
\href{https://issues.liferay.com/browse/LPS-56372}{LPS-56372}.

\paragraph{Who is affected?}\label{who-is-affected-23}

This will affect any implementations of \texttt{ActionCommand}.

\paragraph{How should I update my
code?}\label{how-should-i-update-my-code-23}

Replace imports of \texttt{com.liferay.util.bridges.mvc.ActionCommand}
with
\texttt{com.liferay.portal.kernel.portlet.bridges.mvc.MVCActionCommand}
and imports of \texttt{com.liferay.util.bridges.mvc.BaseActionCommand}
with
\texttt{com.liferay.portal.kernel.portlet.bridges.mvc.BaseMVCActionCommand}.

\paragraph{Why was this change made?}\label{why-was-this-change-made-23}

This change was made to avoid duplication of an implementable interface
in the system. Duplication can cause \texttt{ClassCastException}s.

\subsubsection{Convert Process Classes Are No Longer Specified via the
convert.processes Portal Property, but Are Contributed as OSGi
Modules}\label{convert-process-classes-are-no-longer-specified-via-the-convert.processes-portal-property-but-are-contributed-as-osgi-modules}

\begin{itemize}
\tightlist
\item
  \textbf{Date:} 2014-Oct-09
\item
  \textbf{JIRA Ticket:} LPS-50604
\end{itemize}

\paragraph{What changed?}\label{what-changed-24}

The implementation class
\texttt{com.liferay.portal.convert.ConvertProcess} was renamed
\texttt{com.liferay.portal.convert.BaseConvertProcess}. An interface
named \texttt{com.liferay.portal.convert.ConvertProcess} was created for
it.

The \texttt{convert.processes} key was removed from
\texttt{portal.properties}. Consequentially, \texttt{ConvertProcess}
implementations must register as OSGi components.

\paragraph{Who is affected?}\label{who-is-affected-24}

This affects any implementations of the former \texttt{ConvertProcess}
class, including \texttt{ConvertProcess} class implementations in EXT
plugins. Until version 6.2, this type of service could only be
implemented with an EXT plugin, given that the \texttt{ConvertProcess}
class resided in \texttt{portal-impl}.

\paragraph{How should I update my
code?}\label{how-should-i-update-my-code-24}

You should replace
\texttt{extends\ com.liferay.portal.convert.ConvertProcess} with
\texttt{extends\ com.liferay.portal.convert.BaseConvertProcess} and
annotate the class with
\texttt{@Component(service=ConvertProcess.class)}.

Then turn your EXT plugin into an OSGi bundle and deploy it to the
portal. You should see your convert process in the configuration UI.

\paragraph{Why was this change made?}\label{why-was-this-change-made-24}

This change was made as a part of the ongoing strategy to modularize
Liferay Portal by means of an OSGi container.

\subsubsection{Migration of the Field Type from the Journal Article API
into a
Vocabulary}\label{migration-of-the-field-type-from-the-journal-article-api-into-a-vocabulary}

\begin{itemize}
\tightlist
\item
  \textbf{Date:} 2014-Oct-13
\item
  \textbf{JIRA Ticket:} LPS-50764
\end{itemize}

\paragraph{What changed?}\label{what-changed-25}

The field \emph{type} from the Journal Article entity has been removed.
The Journal API no longer supports this parameter. A new vocabulary
called \emph{Web Content Types} is created when migrating from previous
versions of Liferay, and the types from the existing articles are kept
as categories of this vocabulary.

\paragraph{Who is affected?}\label{who-is-affected-25}

This affects any caller of the removed methods
\texttt{JournalArticle.getType()} and \texttt{JournalFeed.getType()},
and callers of \texttt{ArticleTypeException}'s methods, that attempt to
use the former \texttt{type} parameter of the \texttt{JournalArticle} or
\texttt{JournalFeed} service.

\paragraph{How should I update my
code?}\label{how-should-i-update-my-code-25}

If your logic was not affected by the type, you can simply remove the
\texttt{type} parameter from the Journal API call. If your logic was
affected by the type, you should now use the
\texttt{AssetCategoryService} to obtain the category of the journal
articles.

\paragraph{Why was this change made?}\label{why-was-this-change-made-25}

Web Content Types had to be updated in a properties file and could not
be translated easily. Categories provide a much more flexible behavior
and a better UI. In addition, all the features, such as filters,
developed for categories can be used now in asset publishers and faceted
search.

\subsubsection{Removed the getClassNamePortletId(String) Method from
PortalUtil
Class}\label{removed-the-getclassnameportletidstring-method-from-portalutil-class}

\begin{itemize}
\tightlist
\item
  \textbf{Date:} 2014-Nov-11
\item
  \textbf{JIRA Ticket:} LPS-50604
\end{itemize}

\paragraph{What changed?}\label{what-changed-26}

The \texttt{getClassNamePortletId(String)} method from the
\texttt{PortalUtil} class has been removed.

\paragraph{Who is affected?}\label{who-is-affected-26}

This affects any plugin using the method.

\paragraph{How should I update my
code?}\label{how-should-i-update-my-code-26}

If you are using the method, you should implement it yourself in a
private utility class.

\paragraph{Why was this change made?}\label{why-was-this-change-made-26}

This change was needed in order to modularize the portal. Also, the
method is no longer being used inside Liferay Portal.

\subsubsection{Removed the Header Web Content and Footer Web Content
Preferences from the RSS
Portlet}\label{removed-the-header-web-content-and-footer-web-content-preferences-from-the-rss-portlet}

\begin{itemize}
\tightlist
\item
  \textbf{Date:} 2014-Nov-12
\item
  \textbf{JIRA Ticket:} LPS-46984
\end{itemize}

\paragraph{What changed?}\label{what-changed-27}

The \emph{Header Web Content} and \emph{Footer Web Content} preferences
from the RSS portlet have been removed. The portlet now supports
Application Display Templates (ADT), which provide templating
capabilities that can apply web content to the portlet's header and
footer.

\paragraph{Who is affected?}\label{who-is-affected-27}

This affects RSS portlets that are displayed on pages and that use these
preferences. These preferences are no longer used in the RSS portlet.

\paragraph{How should I update my
code?}\label{how-should-i-update-my-code-27}

Even though these preferences have been removed, an ADT can be created
to produce the same result. Liferay will publish this ADT so that it can
be used in the RSS portlet.

\paragraph{Why was this change made?}\label{why-was-this-change-made-27}

The support for ADTs in the RSS portlet not only covers this use case,
but also covers many other use cases, providing a much simpler way to
create custom preferences.

\subsubsection{Removed the createFlyouts Method from liferay/util.js and
Related
Resources}\label{removed-the-createflyouts-method-from-liferayutil.js-and-related-resources}

\begin{itemize}
\tightlist
\item
  \textbf{Date:} 2014-Dec-18
\item
  \textbf{JIRA Ticket:} LPS-52275
\end{itemize}

\paragraph{What changed?}\label{what-changed-28}

The \texttt{Liferay.Util.createFlyouts} method has been completely
removed from core files.

\paragraph{Who is affected?}\label{who-is-affected-28}

This only affects third party developers who are explicitly calling
\texttt{Liferay.Util.createFlyouts} for the creation of flyout menus. It
will not affect any menus in core files.

\paragraph{How should I update my
code?}\label{how-should-i-update-my-code-28}

If you are using the method, you can achieve the same behavior with CSS.

\paragraph{Why was this change made?}\label{why-was-this-change-made-28}

This method was removed due to there being no working use cases in
Portal, and its overall lack of functionality.

\subsubsection{Removed Support for Flat Thread View in Discussion
Comments}\label{removed-support-for-flat-thread-view-in-discussion-comments}

\begin{itemize}
\tightlist
\item
  \textbf{Date:} 2014-Dec-30
\item
  \textbf{JIRA Ticket:} LPS-51876
\end{itemize}

\paragraph{What changed?}\label{what-changed-29}

Discussion comments are now displayed using the \emph{Combination}
thread view, and the number of levels displayed in the tree is limited.

\paragraph{Who is affected?}\label{who-is-affected-29}

This affects installations that specify portal property setting
\texttt{discussion.thread.view=flat}, which was the default setting.

\paragraph{How should I update my
code?}\label{how-should-i-update-my-code-29}

There is no need to update anything since the portal property has been
removed and the \texttt{combination} thread view is now hard-coded.

\paragraph{Why was this change made?}\label{why-was-this-change-made-29}

Flat view comments were originally implemented as an option to tree view
comments, which were having performance issues with comment pagination.

Portal now uses a new pagination implementation that performs well. It
allows comments to display in a hierarchical view, making it easier to
see reply history. Therefore, the \texttt{flat} thread view is no longer
needed.

\subsubsection{Removed Asset Tag
Properties}\label{removed-asset-tag-properties}

\begin{itemize}
\tightlist
\item
  \textbf{Date:} 2015-Jan-13
\item
  \textbf{JIRA Ticket:} LPS-52588
\end{itemize}

\paragraph{What changed?}\label{what-changed-30}

The \emph{Asset Tag Properties} have been removed. The service no longer
exists and the Asset Tag Service API no longer has this parameter. The
behavior associated with tag properties in the Asset Publisher and XSL
portlets has also been removed.

\paragraph{Who is affected?}\label{who-is-affected-30}

This affects any plugin that uses the Asset Tag Properties service.

\paragraph{How should I update my
code?}\label{how-should-i-update-my-code-30}

If you are using this functionality, you can achieve the same behavior
with \emph{Asset Category Properties}. If you are using the Asset Tag
Service, remove the \texttt{String{[}{]}} tag properties parameter from
your calls to the service's methods.

\paragraph{Why was this change made?}\label{why-was-this-change-made-30}

The Asset Tag Properties were deprecated for the 6.2 version of Liferay
Portal.

\subsubsection{Removed the asset.publisher.asset.entry.query.processors
Property}\label{removed-the-asset.publisher.asset.entry.query.processors-property}

\begin{itemize}
\tightlist
\item
  \textbf{Date:} 2015-Jan-22
\item
  \textbf{JIRA Ticket:} LPS-52966
\end{itemize}

\paragraph{What changed?}\label{what-changed-31}

The \texttt{asset.publisher.asset.entry.query.processors} property has
been removed from \texttt{portal.properties}.

\paragraph{Who is affected?}\label{who-is-affected-31}

This affects any hook that uses the
\texttt{asset.publisher.asset.entry.query.processors} property.

\paragraph{How should I update my
code?}\label{how-should-i-update-my-code-31}

If you are using this property to register Asset Entry Query Processors,
your Asset Entry Query Processor must implement the
\texttt{com.liferay.portlet.assetpublisher.util.AssetEntryQueryProcessor}
interface and must specify the
\texttt{@Component(service=AssetEntryQueryProcessor.class)} annotation.

\paragraph{Why was this change made?}\label{why-was-this-change-made-31}

This change was made as a part of the ongoing strategy to modularize
Liferay Portal.

\subsubsection{Replaced the ReservedUserScreenNameException with
UserScreenNameException.MustNotBeReserved in
UserLocalService}\label{replaced-the-reserveduserscreennameexception-with-userscreennameexception.mustnotbereserved-in-userlocalservice}

\begin{itemize}
\tightlist
\item
  \textbf{Date:} 2015-Jan-29
\item
  \textbf{JIRA Ticket:} LPS-53113
\end{itemize}

\paragraph{What changed?}\label{what-changed-32}

Previous to Liferay 7, several methods of \texttt{UserLocalService}
could throw a \texttt{ReservedUserScreenNameException} when a user set a
screen name that was not allowed. That exception has been deprecated and
replaced with \texttt{UserScreenNameException.MustNotBeReserved}.

\paragraph{Who is affected?}\label{who-is-affected-32}

This affects developers who have written code that catches the
\texttt{ReservedUserScreenNameException} while calling the affected
methods.

\paragraph{How should I update my
code?}\label{how-should-i-update-my-code-32}

You should replace catching exception
\texttt{ReservedUserScreenNameException} with catching exception
\texttt{UserScreenNameException.MustNotBeReserved}.

\paragraph{Why was this change made?}\label{why-was-this-change-made-32}

A new pattern has been defined for exceptions that provides higher
expressivity in their names and also more information regarding why the
exception was thrown.

The new exception \texttt{UserScreenNameException.MustNotBeReserved} has
all the necessary information about why the exception was thrown and its
context. In particular, it contains the user ID, the problematic screen
name, and the list of reserved screen names.

\subsubsection{Replaced the ReservedUserEmailAddressException with
UserEmailAddressException Inner Classes in User
Services}\label{replaced-the-reserveduseremailaddressexception-with-useremailaddressexception-inner-classes-in-user-services}

\begin{itemize}
\tightlist
\item
  \textbf{Date:} 2015-Feb-03
\item
  \textbf{JIRA Ticket:} LPS-53279
\end{itemize}

\paragraph{What changed?}\label{what-changed-33}

Previous to Liferay 7, several methods of \texttt{UserLocalService} and
\texttt{UserService} could throw a
\texttt{ReservedUserEmailAddressException} when a user set an email
address that was not allowed. That exception has been deprecated and
replaced with \texttt{UserEmailAddressException.MustNotUseCompanyMx},
\texttt{UserEmailAddressException.MustNotBePOP3User}, and
\texttt{UserEmailAddressException.MustNotBeReserved}.

\paragraph{Who is affected?}\label{who-is-affected-33}

This affects developers who have written code that catches the
\texttt{ReservedUserEmailAddressException} while calling the affected
methods.

\paragraph{How should I update my
code?}\label{how-should-i-update-my-code-33}

Depending on the method you're calling and the context in which you're
calling it, you should replace catching exception
\texttt{ReservedUserEmailAddressException} with catching exception
\texttt{UserEmailAddressException.MustNotUseCompanyMx},
\texttt{UserEmailAddressException.MustNotBePOP3User}, or
\texttt{UserEmailAddressException.MustNotBeReserved}.

\paragraph{Why was this change made?}\label{why-was-this-change-made-33}

A new pattern has been defined for exceptions. This pattern requires
using higher expressivity in exception names and requires that each
exception provide more information regarding why it was thrown.

Each new exception provides its context and has all the necessary
information about why the exception was thrown. For example, the
\texttt{UserEmailAddressException.MustNotBeReserved} exception contains
the problematic email address and the list of reserved email addresses.

\subsubsection{Replaced ReservedUserIdException with UserIdException
Inner
Classes}\label{replaced-reserveduseridexception-with-useridexception-inner-classes}

\begin{itemize}
\tightlist
\item
  \textbf{Date:} 2015-Feb-10
\item
  \textbf{JIRA Ticket:} LPS-53487
\end{itemize}

\paragraph{What changed?}\label{what-changed-34}

The \texttt{ReservedUserIdException} has been deprecated and replaced
with \texttt{UserIdException.MustNotBeReserved}.

\paragraph{Who is affected?}\label{who-is-affected-34}

This affects developers who have written code that catches the
\texttt{ReservedUserIdException} while calling the affected methods.

\paragraph{How should I update my
code?}\label{how-should-i-update-my-code-34}

You should replace catching exception \texttt{ReservedUserIdException}
with catching exception \texttt{UserIdException.MustNotBeReserved}.

\paragraph{Why was this change made?}\label{why-was-this-change-made-34}

A new pattern has been defined for exceptions that provides higher
expressivity in their names and also more information regarding why the
exception was thrown.

The new exception \texttt{UserIdException.MustNotBeReserved} provides
its context and has all the necessary information about why the
exception was thrown. In particular, it contains the problematic user ID
and the list of reserved user IDs.

\subsubsection{Moved the AssetPublisherUtil Class and Removed It from
the Public
API}\label{moved-the-assetpublisherutil-class-and-removed-it-from-the-public-api}

\begin{itemize}
\tightlist
\item
  \textbf{Date:} 2015-Feb-11
\item
  \textbf{JIRA Ticket:} LPS-52744
\end{itemize}

\paragraph{What changed?}\label{what-changed-35}

The class \texttt{AssetPublisherUtil} from the \texttt{portal-kernel}
module has been moved to the module \texttt{AssetPublisher} and it is no
longer a part of the public API.

\paragraph{Who is affected?}\label{who-is-affected-35}

This affects developers who have written code that uses the
\texttt{AssetPublisherUtil} class.

\paragraph{How should I update my
code?}\label{how-should-i-update-my-code-35}

This \texttt{AssetPublisherUtil} class should no longer be used from
other modules since it contains utility methods for the Asset Publisher
portlet. If needed, you can define a dependency with the Asset Publisher
module and use the new class.

\paragraph{Why was this change made?}\label{why-was-this-change-made-35}

This change has been made as part of the modularization efforts to
decouple the different parts of the portal.

\subsubsection{Removed Operations That Used the Fields Class from the
StorageAdapter
Interface}\label{removed-operations-that-used-the-fields-class-from-the-storageadapter-interface}

\begin{itemize}
\tightlist
\item
  \textbf{Date:} 2015-Feb-11
\item
  \textbf{JIRA Ticket:} LPS-53021
\end{itemize}

\paragraph{What changed?}\label{what-changed-36}

All operations that used the \texttt{Fields} class have been removed
from the \texttt{StorageAdapter} interface.

\paragraph{Who is affected?}\label{who-is-affected-36}

This affects developers who have written code that directly calls these
operations.

\paragraph{How should I update my
code?}\label{how-should-i-update-my-code-36}

You should update your code to use the \texttt{DDMFormValues} class
instead of the \texttt{Fields} class.

\paragraph{Why was this change made?}\label{why-was-this-change-made-36}

This change has been made due to the deprecation of the \texttt{Fields}
class.

\subsubsection{Created a New getType Method That is Implemented in
DLProcessor}\label{created-a-new-gettype-method-that-is-implemented-in-dlprocessor}

\begin{itemize}
\tightlist
\item
  \textbf{Date:} 2015-Feb-17
\item
  \textbf{JIRA Ticket:} LPS-53574
\end{itemize}

\paragraph{What changed?}\label{what-changed-37}

The \texttt{DLProcessor} interface has a new method \texttt{getType()}.

\paragraph{Who is affected?}\label{who-is-affected-37}

This affects developers who have created a \texttt{DLProcessor}.

\paragraph{How should I update my
code?}\label{how-should-i-update-my-code-37}

You should implement the new method and return the type of processor.
You can check the class \texttt{DLProcessorConstants} to see processor
types.

\paragraph{Why was this change made?}\label{why-was-this-change-made-37}

Previous to Liferay 7, developers were forced to extend one of the
existing \texttt{DLProcessor} classes and developers using the extended
class had to check the instance of that class to determine its processor
type.

With this change, developers no longer need to extend any particular
class to create their own \texttt{DLProcessor} and their processor's
type can be clearly specified by a constant from the class
\texttt{DLProcessorConstants}.

\subsubsection{Changed the Usage of the liferay-ui:restore-entry
Tag}\label{changed-the-usage-of-the-liferay-uirestore-entry-tag}

\begin{itemize}
\tightlist
\item
  \textbf{Date:} 2015-Mar-01
\item
  \textbf{JIRA Ticket:} LPS-54106
\end{itemize}

\paragraph{What changed?}\label{what-changed-38}

The usage of the taglib tag \texttt{liferay-ui:restore-entry} serves a
different purpose now. It renders the UI to restore elements from the
Recycle Bin.

\paragraph{Who is affected?}\label{who-is-affected-38}

This affects developers using the tag \texttt{liferay-ui:restore-entry}.

\paragraph{How should I update my
code?}\label{how-should-i-update-my-code-38}

You should replace your calls to the tag with code like the listing
below:

\begin{verbatim}
<aui:script use="liferay-restore-entry">
    new Liferay.RestoreEntry(
    {
            checkEntryURL: '<%= checkEntryURL.toString() %>',
            duplicateEntryURL: '<%= duplicateEntryURL.toString() %>',
            namespace: '<portlet:namespace />'
        }
    );
</aui:script>
\end{verbatim}

In the above code, the \texttt{checkEntryURL} should be an
\texttt{ActionURL} of your portlet, which checks whether the current
entry can be restored from the Recycle Bin. The
\texttt{duplicateEntryURL} should be a \texttt{RenderURL} of your
portlet, that renders the UI to restore the entry, resolving any
existing conflicts. In order to generate that URL, you can use the tag
\texttt{liferay-ui:restore-entry}, which has been refactored for this
usage.

\paragraph{Why was this change made?}\label{why-was-this-change-made-38}

This change allows the Trash portlet to be an independent module. Its
actions and views are no longer used by the tag; they are now the
responsability of each plugin.

\subsubsection{Added Required Parameter resourceClassNameId for DDM
Template Search
Operations}\label{added-required-parameter-resourceclassnameid-for-ddm-template-search-operations}

\begin{itemize}
\tightlist
\item
  \textbf{Date:} 2015-Mar-03
\item
  \textbf{JIRA Ticket:} LPS-52990
\end{itemize}

\paragraph{What changed?}\label{what-changed-39}

The DDM template \texttt{search} and \texttt{searchCount} operations
have a new parameter called \texttt{resourceClassNameId}.

\paragraph{Who is affected?}\label{who-is-affected-39}

This affects developers who have direct calls to the
\texttt{DDMTemplateService} or \texttt{DDMTemplateLocalService}.

\paragraph{How should I update my
code?}\label{how-should-i-update-my-code-39}

You should add the \texttt{resourceClassNameId} parameter to your calls.
This parameter represents the resource that owns the permission for the
DDM template. For example, if the template is a WCM template, the
\texttt{resourceClassNameId} points to the \texttt{JournalArticle}'s
\texttt{classNameId}. If the template is a DDL template, the
\texttt{resourceClassNameId} points to the \texttt{DDLRecordSet}'s
\texttt{classNameId}. If the template is an ADT template, the
\texttt{resourceClassNameId} points to the
\texttt{PortletDisplayTemplate}'s \texttt{classNameId}.

\paragraph{Why was this change made?}\label{why-was-this-change-made-39}

This change was made in order to implement model resource permissions
for DDM templates, such as \texttt{VIEW}, \texttt{DELETE},
\texttt{PERMISSIONS}, and \texttt{UPDATE}.

\subsubsection{Replaced the Breadcrumb Portlet's Display Styles with
ADTs}\label{replaced-the-breadcrumb-portlets-display-styles-with-adts}

\begin{itemize}
\tightlist
\item
  \textbf{Date:} 2015-Mar-12
\item
  \textbf{JIRA Ticket:} LPS-53577
\end{itemize}

\paragraph{What changed?}\label{what-changed-40}

The custom display styles of the breadcrumb tag added using JSPs no
longer work. They have been replaced by Application Display Templates
(ADT).

\paragraph{Who is affected?}\label{who-is-affected-40}

This affects developers that use the following properties:

\begin{verbatim}
breadcrumb.display.style.default=horizontal

breadcrumb.display.style.options=horizontal,vertical
\end{verbatim}

\paragraph{How should I update my
code?}\label{how-should-i-update-my-code-40}

To style the Breadcrumb portlet, you should use ADTs instead of using
custom styles in your JSPs. ADTs can be created from the UI of the
portal by navigating to \emph{Site Settings} → \emph{Application Display
Templates}. ADTs can also be created programatically.

\paragraph{Why was this change made?}\label{why-was-this-change-made-40}

ADTs allow you to change an application's look and feel without changing
its JSP code.

\subsubsection{Changed Usage of the liferay-ui:ddm-template-selector
Tag}\label{changed-usage-of-the-liferay-uiddm-template-selector-tag}

\begin{itemize}
\tightlist
\item
  \textbf{Date:} 2015-Mar-16
\item
  \textbf{JIRA Ticket:} LPS-53790
\end{itemize}

\paragraph{What changed?}\label{what-changed-41}

The attribute \texttt{classNameId} of the
\texttt{liferay-ui:ddm-template-selector} taglib tag has been renamed
\texttt{className}.

\paragraph{Who is affected?}\label{who-is-affected-41}

This affects developers using the
\texttt{liferay-ui:ddm-template-selector} tag.

\paragraph{How should I update my
code?}\label{how-should-i-update-my-code-41}

In your \texttt{liferay-ui:ddm-template-selector} tags, rename the
\texttt{classNameId} attribute to \texttt{className}.

\paragraph{Why was this change made?}\label{why-was-this-change-made-41}

Application Display Templates were being referenced by their UUID, which
was usually not known by the developer. Referencing all DDM templates by
their class name simplifies using this tag.

\subsubsection{Changed the Usage of Asset
Preview}\label{changed-the-usage-of-asset-preview}

\begin{itemize}
\tightlist
\item
  \textbf{Date:} 2015-Mar-16
\item
  \textbf{JIRA Ticket:} LPS-53972
\end{itemize}

\paragraph{What changed?}\label{what-changed-42}

Instead of directly including the JSP referenced by the
\texttt{AssetRenderer}'s \texttt{getPreviewPath} method to preview an
asset, you now use a taglib tag.

\paragraph{Who is affected?}\label{who-is-affected-42}

This affects developers who have written code that directly calls an
\texttt{AssetRenderer}'s \texttt{getPreviewPath} method to preview an
asset.

\paragraph{How should I update my
code?}\label{how-should-i-update-my-code-42}

JSP code that previews an asset by calling an \texttt{AssetRenderer}'s
\texttt{getPreviewPath} method, such as in the example code below, must
be replaced:

\begin{verbatim}
<liferay-util:include
    page="<%= assetRenderer.getPreviewPath(liferayPortletRequest, liferayPortletResponse) %>"
    portletId="<%= assetRendererFactory.getPortletId() %>"
    servletContext="<%= application %>"
/>
\end{verbatim}

To preview an asset, you should instead use the
\texttt{liferay-ui:asset-display} tag, passing it an instance of the
asset entry and an asset renderer preview template. Here's an example of
using the tag:

\begin{verbatim}
<liferay-ui:asset-display
    assetEntry="<%= assetEntry %>"
    template="<%= AssetRenderer.TEMPLATE_PREVIEW %>"
/>
\end{verbatim}

\paragraph{Why was this change made?}\label{why-was-this-change-made-42}

This change simplifies using asset previews.

\subsubsection{Added New Methods in the ScreenNameValidator
Interface}\label{added-new-methods-in-the-screennamevalidator-interface}

\begin{itemize}
\tightlist
\item
  \textbf{Date:} 2015-Mar-17
\item
  \textbf{JIRA Ticket:} LPS-53409
\end{itemize}

\paragraph{What changed?}\label{what-changed-43}

The \texttt{ScreenNameValidator} interface has new methods
\texttt{getDescription(Locale)} and \texttt{getJSValidation()}.

\paragraph{Who is affected?}\label{who-is-affected-43}

This affects developers who have implemented a custom screen name
validator with the \texttt{ScreenNameValidator} interface.

\paragraph{How should I update my
code?}\label{how-should-i-update-my-code-43}

You should implement the new methods introduced in the interface.

\begin{itemize}
\item
  \texttt{getDescription(Locale)}: returns a description of what the
  screen name validator validates.
\item
  \texttt{getJSValidation()}: returns the JavaScript input validator on
  the client side.
\end{itemize}

\paragraph{Why was this change made?}\label{why-was-this-change-made-43}

Previous to Liferay 7, validation for user screen name characters was
hard-coded in \texttt{UserLocalService}. A new portal property named
\texttt{users.screen.name.special.characters} has been added to provide
configurability of special characters allowed in screen names.

In addition, developers can now specify a custom input validator for the
screen name on the client side by providing a JavaScript validator in
\texttt{getJSValidation()}.

\subsubsection{Replaced the Language Portlet's Display Styles with
ADTs}\label{replaced-the-language-portlets-display-styles-with-adts}

\begin{itemize}
\tightlist
\item
  \textbf{Date:} 2015-Mar-30
\item
  \textbf{JIRA Ticket:} LPS-54419
\end{itemize}

\paragraph{What changed?}\label{what-changed-44}

The custom display styles of the language tag added using JSPs no longer
work. They have been replaced by Application Display Templates (ADT).

\paragraph{Who is affected?}\label{who-is-affected-44}

This affects developers that use the following properties:

\begin{verbatim}
language.display.style.default=icon

language.display.style.options=icon,long-text
\end{verbatim}

\paragraph{How should I update my
code?}\label{how-should-i-update-my-code-44}

To style the Language portlet, you should use ADTs instead of using
custom styles in your JSPs. ADTs can be created from the UI of the
portal by navigating to \emph{Site Settings} → \emph{Application Display
Templates}. ADTs can also be created programatically.

\paragraph{Why was this change made?}\label{why-was-this-change-made-44}

ADTs allow you to change an application's look and feel without changing
its JSP code.

\subsubsection{Added Required Parameter groupId for Adding Tags,
Categories, and
Vocabularies}\label{added-required-parameter-groupid-for-adding-tags-categories-and-vocabularies}

\begin{itemize}
\tightlist
\item
  \textbf{Date:} 2015-Mar-31
\item
  \textbf{JIRA Ticket:} LPS-54570
\end{itemize}

\paragraph{What changed?}\label{what-changed-45}

The API for adding tags, categories, and vocabularies now requires
passing the \texttt{groupId} parameter. Previously, it had to be
included in the \texttt{ServiceContext} parameter passed to the method.

\paragraph{Who is affected?}\label{who-is-affected-45}

This affects developers who have direct calls to the following methods:

\begin{itemize}
\tightlist
\item
  \texttt{addTag} in \texttt{AssetTagService} or
  \texttt{AssetTagLocalService}
\item
  \texttt{addCategory} in \texttt{AssetCategoryService} or
  \texttt{AssetCategoryLocalService}
\item
  \texttt{addVocabulary} in \texttt{AssetVocabularyService} or
  \texttt{AssetVocabularyLocalService}
\item
  \texttt{updateFolder} in \texttt{JournalFolderService} or
  \texttt{JournalFolderLocalService}
\end{itemize}

\paragraph{How should I update my
code?}\label{how-should-i-update-my-code-45}

You should add the \texttt{groupId} parameter to your calls. This
parameter represents the site in which you are creating the tag,
category, or vocabulary. It can be obtained from the
\texttt{themeDisplay} or \texttt{serviceContext} using
\texttt{themeDisplay.getScopeGroupId()} or
\texttt{serviceContext.getScopeGroupId()}, respectively.

\paragraph{Why was this change made?}\label{why-was-this-change-made-45}

This change was made in order improve the API. The \texttt{groupId}
parameter was always required, but it was hidden by the
\texttt{ServiceContext} object.

\subsubsection{Removed the Tags that Start with
portlet:icon-}\label{removed-the-tags-that-start-with-portleticon-}

\begin{itemize}
\tightlist
\item
  \textbf{Date:} 2015-Mar-31
\item
  \textbf{JIRA Ticket:} LPS-54620
\end{itemize}

\paragraph{What changed?}\label{what-changed-46}

The following tags have been removed:

\begin{itemize}
\tightlist
\item
  \texttt{liferay-portlet:icon-close}
\item
  \texttt{liferay-portlet:icon-configuration}
\item
  \texttt{liferay-portlet:icon-edit}
\item
  \texttt{liferay-portlet:icon-edit-defaults}
\item
  \texttt{liferay-portlet:icon-edit-guest}
\item
  \texttt{liferay-portlet:icon-export-import}
\item
  \texttt{liferay-portlet:icon-help}
\item
  \texttt{liferay-portlet:icon-maximize}
\item
  \texttt{liferay-portlet:icon-minimize}
\item
  \texttt{liferay-portlet:icon-portlet-css}
\item
  \texttt{liferay-portlet:icon-print}
\item
  \texttt{liferay-portlet:icon-refresh}
\item
  \texttt{liferay-portlet:icon-staging}
\end{itemize}

\paragraph{Who is affected?}\label{who-is-affected-46}

This affects developers who have written code that uses these tags.

\paragraph{How should I update my
code?}\label{how-should-i-update-my-code-46}

The tag \texttt{liferay-ui:icon} can replace the call to the previous
tags. All the previous tags have been converted into Java classes that
implement the methods that the \texttt{icon} tag requires.

See the modules \texttt{portlet-configuration-icon-*} in the
\texttt{modules/apps/web-experience/portlet-configuration} folder.

\paragraph{Why was this change made?}\label{why-was-this-change-made-46}

These tags were used to generate the configuration icon of portlets.
This functionality will now be managed from OSGi modules instead of tags
since OSGi modules provide more flexibility and can be included in any
app.

\subsubsection{Changed the Default Value of the copy-request-parameters
Init Parameter for MVC
Portlets}\label{changed-the-default-value-of-the-copy-request-parameters-init-parameter-for-mvc-portlets}

\begin{itemize}
\tightlist
\item
  \textbf{Date:} 2015-Apr-15
\item
  \textbf{JIRA Ticket:} LPS-54798
\end{itemize}

\paragraph{What changed?}\label{what-changed-47}

The \texttt{copy-request-parameters} init parameter's default value is
now set to \texttt{true} in all portlets that extend
\texttt{MVCPortlet}.

\paragraph{Who is affected?}\label{who-is-affected-47}

This affects developers that have created portlets that extend
\texttt{MVCPortlet}.

\paragraph{How should I update my
code?}\label{how-should-i-update-my-code-47}

To continue using the property the same way you did before this change
was implemented, you'll need to change the default property. To change
the property, set the init parameter to \texttt{false} in your class
extending \texttt{MVCPortlet}:

\begin{verbatim}
javax.portlet.init-param.copy-request-parameters=false
\end{verbatim}

\paragraph{Why was this change made?}\label{why-was-this-change-made-47}

This change was made to allow for backwards compatibility.

\subsubsection{Removed Portal Properties Used to Display Sections in
Form
Navigators}\label{removed-portal-properties-used-to-display-sections-in-form-navigators}

\begin{itemize}
\tightlist
\item
  \textbf{Date:} 2015-Apr-16
\item
  \textbf{JIRA Ticket:} LPS-54903
\end{itemize}

\paragraph{What changed?}\label{what-changed-48}

The following portal properties (and the equivalent \texttt{PropsKeys}
and \texttt{PropsValues}) that were used to decide what sections would
be displayed in the \texttt{form-navigator} have been removed:

\begin{itemize}
\tightlist
\item
  \texttt{company.settings.form.configuration}
\item
  \texttt{company.settings.form.identification}
\item
  \texttt{company.settings.form.miscellaneous}
\item
  \texttt{company.settings.form.social}
\item
  \texttt{journal.article.form.add}
\item
  \texttt{journal.article.form.update}
\item
  \texttt{journal.article.form.default.values}
\item
  \texttt{layout.form.add}
\item
  \texttt{layout.form.update}
\item
  \texttt{layout.set.form.update}
\item
  \texttt{organizations.form.add.identification}
\item
  \texttt{organizations.form.add.main}
\item
  \texttt{organizations.form.add.miscellaneous}
\item
  \texttt{organizations.form.update.identification}
\item
  \texttt{organizations.form.update.main}
\item
  \texttt{organizations.form.update.miscellaneous}
\item
  \texttt{sites.form.add.advanced}
\item
  \texttt{sites.form.add.main}
\item
  \texttt{sites.form.add.miscellaneous}
\item
  \texttt{sites.form.add.seo}
\item
  \texttt{sites.form.update.advanced}
\item
  \texttt{sites.form.update.main}
\item
  \texttt{sites.form.update.miscellaneous}
\item
  \texttt{sites.form.update.seo}
\item
  \texttt{users.form.add.identification}
\item
  \texttt{users.form.add.main}
\item
  \texttt{users.form.add.miscellaneous}
\item
  \texttt{users.form.my.account.identification}
\item
  \texttt{users.form.my.account.main}
\item
  \texttt{users.form.my.account.miscellaneous}
\item
  \texttt{users.form.update.identification}
\item
  \texttt{users.form.update.main}
\item
  \texttt{users.form.update.miscellaneous}
\end{itemize}

The sections and categories of form navigators are now OSGi components.

\paragraph{Who is affected?}\label{who-is-affected-48}

This affects administrators who may have added, removed, or reordered
sections using those portal properties. Developers using the constants
defined in \texttt{PropsKeys} or \texttt{PropsValues} for those portal
properties will also be affected.

\paragraph{How should I update my
code?}\label{how-should-i-update-my-code-48}

Since those properties no longer exist, you cannot rely on them.
References to the constants of \texttt{PropsKeys} and
\texttt{PropsValues} will need to be updated. You can use
\texttt{FormNavigatorCategoryUtil} and \texttt{FormNavigatorEntryUtil}
to obtain a list of the available sections and categories for a form
navigator instance.

Changes to remove or reorder specific sections will need to be done
through the OSGi console to update the service ranking or stop the
components.

Adding new sections with Liferay Hooks will still work as a legacy
feature, but the recommended way is using OSGi components to add new
sections.

\paragraph{Why was this change made?}\label{why-was-this-change-made-48}

The old mechanism to add new sections to \texttt{form-navigator} tags
was very limited because it could only depend on portal for services and
utils due to the new section that was rendered from the portal
classloader.

There was a need to add new sections and categories to
\texttt{form-navigator} tags via OSGi plugins in a more extensible way,
allowing the developer to include new sections to access to their own
utils and services.

\subsubsection{Removed the Type Setting breadcrumbShowParentGroups from
Groups}\label{removed-the-type-setting-breadcrumbshowparentgroups-from-groups}

\begin{itemize}
\tightlist
\item
  \textbf{Date:} 2015-Apr-21
\item
  \textbf{JIRA Ticket:} LPS-54791
\end{itemize}

\paragraph{What changed?}\label{what-changed-49}

The type setting \texttt{breadcrumbShowParentGroups} was removed from
groups and is no longer available in the site configuration. Now, it is
only available in the breadcrumb configuration.

\paragraph{Who is affected?}\label{who-is-affected-49}

This affects all site administrators that have set the
\texttt{showParentGroups} preference in Site Administration.

\paragraph{How should I update my
code?}\label{how-should-i-update-my-code-49}

There are no code updates required. This should only be updated at the
portlet instance level.

\paragraph{Why was this change made?}\label{why-was-this-change-made-49}

This change was introduced to support the new Settings API.

\subsubsection{Changed Return Value of the Method getText of the
Editor's Window
API}\label{changed-return-value-of-the-method-gettext-of-the-editors-window-api}

\begin{itemize}
\tightlist
\item
  \textbf{Date:} 2015-Apr-28
\item
  \textbf{JIRA Ticket:} LPS-52698
\end{itemize}

\paragraph{What changed?}\label{what-changed-50}

The method \texttt{getText} now returns the editor's content, without
any HTML markup.

\paragraph{Who is affected?}\label{who-is-affected-50}

This affects developers that are using the \texttt{getText} method of
the editor's window API.

\paragraph{How should I update my
code?}\label{how-should-i-update-my-code-50}

To continue using the editor the same way you did before this change was
implemented, you should change calls to the \texttt{getText} method to
instead call the \texttt{getHTML} method.

\paragraph{Why was this change made?}\label{why-was-this-change-made-50}

This change was made in the editor's window API to provide a proper
\texttt{getText} method that returns just the editor's content, without
any HTML markup. This change is used for the blog abstract field.

\subsubsection{Moved the Contact Name Exception Classes to Inner Classes
of
ContactNameException}\label{moved-the-contact-name-exception-classes-to-inner-classes-of-contactnameexception}

\begin{itemize}
\tightlist
\item
  \textbf{Date:} 2015-May-05
\item
  \textbf{JIRA Ticket:} LPS-55364
\end{itemize}

\paragraph{What changed?}\label{what-changed-51}

The use of classes \texttt{ContactFirstNameException},
\texttt{ContactFullNameException}, and \texttt{ContactLastNameException}
has been moved to inner classes in a new class called
\texttt{ContactNameException}.

\paragraph{Who is affected?}\label{who-is-affected-51}

This affects developers who may have included one of the three classes
above in their code.

\paragraph{How should I update my
code?}\label{how-should-i-update-my-code-51}

While the old classes remain for backwards-compatibility, they are being
deprecated. You're encouraged to use the new pattern of inner classes
for exceptions wherever possible. For example, instead of using
\texttt{ContactFirstNameExeception}, use
\texttt{ContactNameException.MustHaveFirstName}.

\paragraph{Why was this change made?}\label{why-was-this-change-made-51}

This change was made in accordance with the new exceptions pattern being
applied throughout Portal. It also allows the new localized user name
configuration feature to be thoroughly covered by exceptions for
different configurations.

\subsubsection{Removed USERS\_LAST\_NAME\_REQUIRED from
portal.properties in Favor of language.properties
Configurations}\label{removed-users_last_name_required-from-portal.properties-in-favor-of-language.properties-configurations}

\begin{itemize}
\tightlist
\item
  \textbf{Date:} 2015-May-07
\item
  \textbf{JIRA Ticket:} LPS-54956
\end{itemize}

\paragraph{What changed?}\label{what-changed-52}

The \texttt{USERS\_LAST\_NAME\_REQUIRED} property has been removed from
\texttt{portal.properties} and the corresponding UI. Required names are
now handled on a per-language basis via the \texttt{language.properties}
files. It has also been removed as an option from the Portal Settings
section of the Control Panel.

\paragraph{Who is affected?}\label{who-is-affected-52}

This affects anyone who uses the \texttt{USERS\_LAST\_NAME\_REQUIRED}
portal property.

\paragraph{How should I update my
code?}\label{how-should-i-update-my-code-52}

If you need to require the user's last name, list it on the
\texttt{lang.user.name.required.field.names} line of the appropriate
\texttt{language.properties} files:

\begin{verbatim}
lang.user.name.required.field.names=last-name
\end{verbatim}

\paragraph{Why was this change made?}\label{why-was-this-change-made-52}

Portal property \texttt{USERS\_LAST\_NAME\_REQUIRED} didn't support the
multicultural user name configurations introduced in LPS-48406. Language
property files (e.g., \texttt{language.properties}) now support these
configurations. Control of all user name configuration, except with
regards to first name, is relegated to language property files. First
name is required and always present.

\subsubsection{Removed Methods getGroupLocalRepositoryImpl and
getLocalRepositoryImpl from RepositoryLocalService and
RepositoryService}\label{removed-methods-getgrouplocalrepositoryimpl-and-getlocalrepositoryimpl-from-repositorylocalservice-and-repositoryservice}

\begin{itemize}
\tightlist
\item
  \textbf{Date:} 2015-May-14
\item
  \textbf{JIRA Ticket:} LPS-55566
\end{itemize}

\paragraph{What changed?}\label{what-changed-53}

The methods \texttt{getGroupLocalRepositoryImpl(...)} and
\texttt{getLocalRepositoryImpl(...)} have been removed from
\texttt{RepositoryLocalService} and \texttt{RepositoryService}. Although
the methods are related to the service, they belong in a different level
of abstraction.

\paragraph{Who is affected?}\label{who-is-affected-53}

This affects anyone who uses those methods.

\paragraph{How should I update my
code?}\label{how-should-i-update-my-code-53}

The removed methods were generic and had long signatures with optional
parameters. They now have one specialized version per parameter and are
in the \texttt{RepositoryProvider} service.

\textbf{Example}

Old call:

\begin{verbatim}
RepositoryLocalServiceUtil.getRepositoryImpl(0, fileEntryId, 0)
\end{verbatim}

New call:

\begin{verbatim}
RepositoryProviderUtil.getLocalRepositoryByFileEntryId(fileEntryId)
\end{verbatim}

\paragraph{Why was this change made?}\label{why-was-this-change-made-53}

This change was made to enhance the Repository API and facilitate
decoupling the API from the Document Library, as a part of the portal
modularization effort.

\subsubsection{Removed addFileEntry Method from
DLAppHelperLocalService}\label{removed-addfileentry-method-from-dlapphelperlocalservice}

\begin{itemize}
\tightlist
\item
  \textbf{Date:} 2015-May-20
\item
  \textbf{JIRA Ticket:} LPS-47645
\end{itemize}

\paragraph{What changed?}\label{what-changed-54}

The \texttt{addFileEntry} method has been removed from
\texttt{DLAppHelperLocalService}.

\paragraph{Who is affected?}\label{who-is-affected-54}

This affects anyone who calls the \texttt{addFileEntry} method.

\paragraph{How should I update my
code?}\label{how-should-i-update-my-code-54}

If you need to invoke the \texttt{addFileEntry} method as part of a
custom repository implementation, use the provided repository
capabilities instead. See \texttt{LiferayRepositoryDefiner} for examples
on their use.

For other use cases, you may need to explicitly invoke each of the
service methods used by \texttt{addFileEntry}.

\paragraph{Why was this change made?}\label{why-was-this-change-made-54}

The logic inside the \texttt{addFileEntry} method was moved, out from
\texttt{DLAppHelperLocalService} and into repository capabilities, to
further decouple core repository implementations from additional
(optional) functionality.

\subsubsection{Indexers Called from Document Library Now Receive
FileEntry Instead of
DLFileEntry}\label{indexers-called-from-document-library-now-receive-fileentry-instead-of-dlfileentry}

\begin{itemize}
\tightlist
\item
  \textbf{Date:} 2015-May-20
\item
  \textbf{JIRA Ticket:} LPS-55613
\end{itemize}

\paragraph{What changed?}\label{what-changed-55}

Indexers that previously received a \texttt{DLFileEntry} object (e.g.,
in the \texttt{addRelatedEntryFields} method) no longer receive a
\texttt{DLFileEntry}, but a \texttt{FileEntry}.

\paragraph{Who is affected?}\label{who-is-affected-55}

This affects anyone who implements an Indexer handling
\texttt{DLFileEntry} objects.

\paragraph{How should I update my
code?}\label{how-should-i-update-my-code-55}

You should try to use methods in \texttt{FileEntry} or exported
repository capabilities to obtain the value you were using. If no
capability exists for your use case, you can resort to calling
\texttt{fileEntry.getModel()} and casting the result to a
\texttt{DLFileEntry}. However, this breaks all encapsulation and may
result in future failures or compatibility problems.

Old code:

\begin{verbatim}
@Override
public void addRelatedEntryFields(Document document, Object obj)
    throws Exception {

    DLFileEntry dlFileEntry = (DLFileEntry)obj;

    long fileEntryId = dlFileEntry.getFileEntryId();
\end{verbatim}

New Code:

\begin{verbatim}
@Override
public void addRelatedEntryFields(Document document, Object obj)
    throws Exception {

    FileEntry fileEntry = (FileEntry)obj;

    long fileEntryId = fileEntry.getFileEntryId();
\end{verbatim}

\paragraph{Why was this change made?}\label{why-was-this-change-made-55}

This change was made to enhance the Repository API and make decoupling
from Document Library easier when modularizing the portal.

\subsubsection{Removed permissionClassName, permissionClassPK, and
permissionOwner Parameters from MBMessage
API}\label{removed-permissionclassname-permissionclasspk-and-permissionowner-parameters-from-mbmessage-api}

\begin{itemize}
\tightlist
\item
  \textbf{Date:} 2015-May-27
\item
  \textbf{JIRA Ticket:} LPS-55877
\end{itemize}

\paragraph{What changed?}\label{what-changed-56}

The parameters \texttt{permissionClassName}, \texttt{permissionClassPK},
and \texttt{permissionOwner} have been removed from the Message Boards
API and Discussion tag.

\paragraph{Who is affected?}\label{who-is-affected-56}

This affects anyone who invokes the affected methods (locally or
remotely) and any view that uses the Discussion tag.

\paragraph{How should I update my
code?}\label{how-should-i-update-my-code-56}

It suffices to remove the parameters from the method calls (for
consumers of the API) or the attributes in tag invocations.

\paragraph{Why was this change made?}\label{why-was-this-change-made-56}

Those API methods were exposed in the remote services, allowing any
consumer to bypass the permission system by providing customized
\texttt{className}, \texttt{classPK}, or \texttt{ownerId} parameters.

\subsubsection{Moved Indexer.addRelatedEntryFields and
Indexer.reindexDDMStructures, and Removed
Indexer.getQueryString}\label{moved-indexer.addrelatedentryfields-and-indexer.reindexddmstructures-and-removed-indexer.getquerystring}

\begin{itemize}
\tightlist
\item
  \textbf{Date:} 2015-May-27
\item
  \textbf{JIRA Ticket:} LPS-55928
\end{itemize}

\paragraph{What changed?}\label{what-changed-57}

Method \texttt{Indexer.addRelatedEntryFields(Document,\ Object)} has
been moved into \texttt{RelatedEntryIndexer}.

\texttt{Indexer.reindexDDMStructures(List\textless{}Long\textgreater{})}
has been moved into \texttt{DDMStructureIndexer}.

\texttt{Indexer.getQueryString(SearchContext,\ Query)} has been removed,
in favor of calling
\texttt{SearchEngineUtil.getQueryString(SearchContext,\ Query)}

\paragraph{Who is affected?}\label{who-is-affected-57}

This affects any code that invokes the affected methods, as well as any
code that implements the interface methods.

\paragraph{How should I update my
code?}\label{how-should-i-update-my-code-57}

Any code implementing \texttt{Indexer.addRelatedEntryFields(...)} should
implement the \texttt{RelatedEntryIndexer} interface.

Any code calling \texttt{Indexer.addRelatedEntryFields(...)} should
determine first if the \texttt{Indexer} is an instance of
\texttt{RelatedEntryIndexer}.

Old code:

\begin{verbatim}
mbMessageIndexer.addRelatedEntryFields(...);
\end{verbatim}

New code:

\begin{verbatim}
if (mbMessageIndexer instanceof RelatedEntryIndexer) {
    RelatedEntryIndexer relatedEntryIndexer =
        (RelatedEntryIndexer)mbMessageIndexer;

    relatedEntryIndexer.addRelatedEntryFields(...);
}
\end{verbatim}

Any code implementing \texttt{Indexer.reindexDDMStructures(...)} should
implement the \texttt{DDMStructureIndexer} interface.

Any code calling \texttt{Indexer.reindexDDMStructures(...)} should
determine first if the \texttt{Indexer} is an instance of
\texttt{DDMStructureIndexer}.

Old code:

\begin{verbatim}
mbMessageIndexer.reindexDDMStructures(...);
\end{verbatim}

New code:

\begin{verbatim}
if (journalIndexer instanceof DDMStructureIndexer) {
    DDMStructureIndexer ddmStructureIndexer =
        (DDMStructureIndexer)journalIndexer;

    ddmStructureIndexer.reindexDDMStructures(...);
}
\end{verbatim}

Any code calling \texttt{Indexer.getQueryString(...)} should call
\texttt{SearchEngineUtil.getQueryString(...)}.

Old code:

\begin{verbatim}
mbMessageIndexer.getQueryString(...);
\end{verbatim}

New code:

\begin{verbatim}
SearchEngineUtil.getQueryString(...);
\end{verbatim}

\paragraph{Why was this change made?}\label{why-was-this-change-made-57}

The \texttt{addRelatedEntryFields} and \texttt{reindexDDMStructures}
methods were not related to core indexing functions. They were functions
of specialized indexers.

The \texttt{getQueryString} method was an unnecessary convenience
method.

\subsubsection{Removed mbMessages and fileEntryTuples Attributes from
app-view-search-entry
Tag}\label{removed-mbmessages-and-fileentrytuples-attributes-from-app-view-search-entry-tag}

\begin{itemize}
\tightlist
\item
  \textbf{Date:} 2015-May-27
\item
  \textbf{JIRA Ticket:} LPS-55886
\end{itemize}

\paragraph{What changed?}\label{what-changed-58}

The \texttt{mbMessages} and \texttt{fileEntryTuples} attributes from the
\texttt{app-view-search-entry} tag have been removed. Related methods
\texttt{getMbMessages}, \texttt{getFileEntryTuples}, and
\texttt{addMbMessage} have also been removed from the
\texttt{SearchResult} class.

\paragraph{Who is affected?}\label{who-is-affected-58}

This affects developers that use the \texttt{app-view-search-entry} tag
in their views, have developed hooks to customize the tag JSP, or have
developed a portlet that uses that tag. Also, any custom code that uses
the \texttt{SearchResult} class may be affected.

\paragraph{How should I update my
code?}\label{how-should-i-update-my-code-58}

The new attributes \texttt{commentRelatedSearchResults} and
\texttt{fileEntryRelatedSearchResults} should be used instead. The
expected value is the one returned by the
\texttt{getCommentRelatedSearchResults} and
\texttt{getFileEntryRelatedSearchResults} methods in
\texttt{SearchResult}.

When adding comments to the \texttt{SearchResult}, the new
\texttt{addComment} method should be used instead of the
\texttt{addMbMessage} method.

\paragraph{Why was this change made?}\label{why-was-this-change-made-58}

As part of the modularization efforts, references to \texttt{MBMessage}
needed to be removed for the Message Boards portlet to be placed into
its own OSGi bundle.

\subsubsection{Replaced Method getPermissionQuery with
getPermissionFilter in SearchPermissionChecker, and getFacetQuery with
getFacetBooleanFilter in
Indexer}\label{replaced-method-getpermissionquery-with-getpermissionfilter-in-searchpermissionchecker-and-getfacetquery-with-getfacetbooleanfilter-in-indexer}

\begin{itemize}
\tightlist
\item
  \textbf{Date:} 2015-Jun-02
\item
  \textbf{JIRA Ticket:} LPS-56064
\end{itemize}

\paragraph{What changed?}\label{what-changed-59}

Method
\texttt{SearchPermissionChecker.getPermissionQuery(\ long,\ long{[}{]},\ long,\ String,\ Query,\ SearchContext)}
has been replaced by
\texttt{SearchPermissionChecker.getPermissionBooleanFilter(\ long,\ long{[}{]},\ long,\ String,\ BooleanFilter,\ SearchContext)}.

Method \texttt{Indexer.getFacetQuery(String,\ SearchContext)} has been
replaced by
\texttt{Indexer.getFacetBooleanFilter(String,\ SearchContext)}.

\paragraph{Who is affected?}\label{who-is-affected-59}

This affects any code that invokes the affected methods, as well as any
code that implements the interface methods.

\paragraph{How should I update my
code?}\label{how-should-i-update-my-code-59}

Any code calling/implementing
\texttt{SearchPermissionChecker.getPermissionQuery(...)} should instead
call/implement
\texttt{SearchPermissionChecker.getPermissionBooleanFilter(...)}.

Any code calling/implementing \texttt{Indexer.getFacetQuery(...)} should
instead call/implement \texttt{Indexer.getFacetBooleanFilter(...)}.

\paragraph{Why was this change made?}\label{why-was-this-change-made-59}

Permission constraints placed on search should not affect the score for
returned search results. Thus, these constraints should be applied as
search filters. \texttt{SearchPermissionChecker} is also a very deep
internal interface within the permission system. Thus, to limit
confusion in the logic for maintainability, the
\texttt{SearchPermissionChecker.getPermissionQuery(...)} method was
removed as opposed to deprecated.

Similarly, constraints applied to facets should not affect the scoring
or facet counts. Since \texttt{Indexer.getFacetQuery(...)} was only
utilized by the \texttt{AssetEntriesFacet}, and used to reduce the
impact of changes for
\texttt{SearchPermissionChecker.getPermissionBooleanFilter(...)}, the
method was removed as opposed to deprecated.

\subsubsection{Added userId Parameter to Update Operations of
DDMStructureLocalService and
DDMTemplateLocalService}\label{added-userid-parameter-to-update-operations-of-ddmstructurelocalservice-and-ddmtemplatelocalservice}

\begin{itemize}
\tightlist
\item
  \textbf{Date:} 2015-Jun-05
\item
  \textbf{JIRA Ticket:} LPS-50939
\end{itemize}

\paragraph{What changed?}\label{what-changed-60}

A new parameter \texttt{userId} has been added to the
\texttt{updateStructure} and \texttt{updateTemplate} methods of the
\texttt{DDMStructureLocalService} and \texttt{DDMTemplateLocalService}
classes, respectively.

\paragraph{Who is affected?}\label{who-is-affected-60}

This affects any code that invokes the affected methods, as well as any
code that implements the interface methods.

\paragraph{How should I update my
code?}\label{how-should-i-update-my-code-60}

Any code calling/implementing
\texttt{DDMStructureLocalServiceUtil.updateStructure(...)} or
\texttt{DDMTemplateLocalServiceUtil.updateTemplate(...)} should pass the
new \texttt{userId} parameter.

\paragraph{Why was this change made?}\label{why-was-this-change-made-60}

For the service to keep track of which user is modifying the structure
or template, the \texttt{userId} parameter was required. In order to add
support to structure and template versions, audit columns were also
added to such models.

\subsubsection{Removed Method getEntries from DL, DLImpl, and DLUtil
Classes}\label{removed-method-getentries-from-dl-dlimpl-and-dlutil-classes}

\begin{itemize}
\tightlist
\item
  \textbf{Date:} 2015-Jun-10
\item
  \textbf{JIRA Ticket:} LPS-56247
\end{itemize}

\paragraph{What changed?}\label{what-changed-61}

The method \texttt{getEntries} has been removed from the \texttt{DL},
\texttt{DLImpl}, and \texttt{DLUtil} classes.

\paragraph{Who is affected?}\label{who-is-affected-61}

This affects any caller of the \texttt{getEntries} method.

\paragraph{How should I update my
code?}\label{how-should-i-update-my-code-61}

You may use the \texttt{SearchResultUtil} class to process the search
results. Note that this class is not completely equivalent; if you need
exactly the same behavior as the removed method, you will need to add
custom code.

\paragraph{Why was this change made?}\label{why-was-this-change-made-61}

The \texttt{getEntries} method was no longer used, and contained
hardcoded references to classes that will be moved into OSGi bundles.

\subsubsection{Removed WikiUtil.getEntries
Method}\label{removed-wikiutil.getentries-method}

\begin{itemize}
\tightlist
\item
  \textbf{Date:} 2015-Jun-10
\item
  \textbf{JIRA Ticket:} LPS-56242
\end{itemize}

\paragraph{What changed?}\label{what-changed-62}

The method \texttt{getEntries()} has been removed from class
\texttt{WikiUtil}.

\paragraph{Who is affected?}\label{who-is-affected-62}

Any JSP hook or ext plugin that uses this method is affected. As the
class was located in portal-impl, regular portlets and other safe
extension points won't be affected.

\paragraph{How should I update my
code?}\label{how-should-i-update-my-code-62}

You should review the JSP or ext plugin, updating it to remove any
reference to the new class and mimicking the original JSP code. In case
you need equivalent functionality to the one provided by
\texttt{WikiUtil.getEntries()} you may use the \texttt{SearchResultUtil}
class. While not totally equivalent, it offers similar functionality.

\paragraph{Why was this change made?}\label{why-was-this-change-made-62}

The \texttt{WikiUtil.getEntries()} method was no longer used, and it
contained hardcoded references to classes that will be moved into OSGi
modules.

\subsubsection{Removed render Method from ConfigurationAction
API}\label{removed-render-method-from-configurationaction-api}

\begin{itemize}
\tightlist
\item
  \textbf{Date:} 2015-Jun-14
\item
  \textbf{JIRA Ticket:} LPS-56300
\end{itemize}

\paragraph{What changed?}\label{what-changed-63}

The method \texttt{render} has been removed from the interface
\texttt{ConfigurationAction}.

\paragraph{Who is affected?}\label{who-is-affected-63}

This affects any Java code calling the method \texttt{render} on a
\texttt{ConfigurationAction} class, or Java classes overriding the
\texttt{render} method of a \texttt{ConfigurationAction} class.

\paragraph{How should I update my
code?}\label{how-should-i-update-my-code-63}

The method \texttt{render} was used to return the path of a JSP,
including the configuration of a portlet. That method is now available
for configurations extending the
\texttt{BaseJSPSettingsConfigurationAction} class, and is called
\texttt{getJspPath}.

If any logic was added to override the \texttt{render} method, it can
now be added in the \texttt{include} method.

\paragraph{Why was this change made?}\label{why-was-this-change-made-63}

This change was part of needed modifications to support adding
configuration for portlets based on other technology different than JSP
(e.g., FreeMarker). The method \texttt{include} can now be used to
create configuration UIs written in FreeMarker or any other framework.

\subsubsection{Removed ckconfig Files Used for CKEditor
Configuration}\label{removed-ckconfig-files-used-for-ckeditor-configuration}

\begin{itemize}
\tightlist
\item
  \textbf{Date:} 2015-Jun-16
\item
  \textbf{JIRA Ticket:} LPS-55518
\end{itemize}

\paragraph{What changed?}\label{what-changed-64}

The files \texttt{ckconfig.jsp}, \texttt{ckconfig-ext.jsp},
\texttt{ckconfig\_bbcode.jsp}, \texttt{ckconfig\_bbcode-ext.jsp},
\texttt{ckconfig\_creole.jsp}, and \texttt{ckconfig\_creole-ext.jsp}
have been removed and are no longer used to configure the CKEditor
instances created using the \texttt{liferay-ui:input-editor} tag.

\paragraph{Who is affected?}\label{who-is-affected-64}

This affects any hook or plugin-ext overriding these files to modify the
editor configuration.

\paragraph{How should I update my
code?}\label{how-should-i-update-my-code-64}

Depending on the changes, different extension methods are available:

\begin{itemize}
\tightlist
\item
  For CKEditor configuration options, an implementation of
  \texttt{EditorConfigContributor} can be created to pass or modify the
  expected parameters.
\item
  For CKEditor instance manipulation (setting attributes, adding
  listeners, etc.), the \texttt{DynamicInclude} extension point
  \texttt{\#ckeditor{[}\_creole\textbar{}\_bbcode{]}\#onEditorCreated}
  has been added to provide the possibility of injecting JavaScript,
  when needed.
\end{itemize}

\paragraph{Why was this change made?}\label{why-was-this-change-made-64}

This change is part of a greater effort to provide mechanisms to extend
and configure any editor in Liferay Portal in a coherent and extensible
way.

\subsubsection{Renamed ActionCommand Classes Used in the MVCPortlet
Framework}\label{renamed-actioncommand-classes-used-in-the-mvcportlet-framework}

\begin{itemize}
\tightlist
\item
  \textbf{Date:} 2015-Jun-16
\item
  \textbf{JIRA Ticket:} LPS-56372
\end{itemize}

\paragraph{What changed?}\label{what-changed-65}

The classes located in the
\texttt{com.liferay.portal.kernel.portlet.bridges.mvc} package have been
renamed to include the \emph{MVC} prefix.

Old Classes:

\begin{itemize}
\tightlist
\item
  \texttt{BaseActionCommand}
\item
  \texttt{BaseTransactionalActionCommand}
\item
  \texttt{ActionCommand}
\item
  \texttt{ActionCommandCache}
\end{itemize}

New Classes:

\begin{itemize}
\tightlist
\item
  \texttt{BaseMVCActionCommand}
\item
  \texttt{BaseMVCTransactionalActionCommand}
\item
  \texttt{MVCActionCommand}
\item
  \texttt{MVCActionCommandCache}
\end{itemize}

Also, the property \texttt{action.command.name} has been renamed to
\texttt{mvc.command.name}. The code snippet below shows the new property
in its context.

\begin{verbatim}
@Component(
    immediate = true,
    property = {
            "javax.portlet.name=" + InvitationPortletKeys.INVITATION,
            "mvc.command.name=view"
    },
    service = MVCActionCommand.class
)
\end{verbatim}

\paragraph{Who is affected?}\label{who-is-affected-65}

This affects any Java code calling the \texttt{ActionCommand} classes
used in the \texttt{MVCPortlet} framework.

\paragraph{How should I update my
code?}\label{how-should-i-update-my-code-65}

You should update the old \texttt{ActionCommand} class names with the
new \emph{MVC} prefix.

\paragraph{Why was this change made?}\label{why-was-this-change-made-65}

This change adds consistency to the MVC framework, and makes it
self-explanatory what classes should be used for the MVC portlet.

\subsubsection{Extended MVC Framework to Use Same Key for Registering
ActionURL and
ResourceURL}\label{extended-mvc-framework-to-use-same-key-for-registering-actionurl-and-resourceurl}

\begin{itemize}
\tightlist
\item
  \textbf{Date:} 2015-Jun-16
\item
  \textbf{JIRA Ticket:} LPS-56372
\end{itemize}

\paragraph{What changed?}\label{what-changed-66}

Previously, a single \texttt{ActionCommand} was valid for both
\texttt{ActionURL} and \texttt{ResourceURL}. Now you must distinguish
both an \texttt{ActionURL} and \texttt{ResourceURL} as different
actions, which means you can register both with the same key.

\paragraph{Who is affected?}\label{who-is-affected-66}

This affects developers that were using the \texttt{ActionCommand} for
\texttt{actionURL}s and \texttt{resourceURL}s.

\paragraph{How should I update my
code?}\label{how-should-i-update-my-code-66}

You should replace the \texttt{ActionCommand}s used for
\texttt{actionURL}s and \texttt{resourceURL}s to use
\texttt{MVCActionCommand} and \texttt{MVCResourceCommand}, respectively.
For example, for the new \texttt{MVCResourceCommand}, you'll need to use
the \texttt{resourceID} of the \texttt{resourceURL} instead of using
\texttt{ActionRequest.ACTION\_NAME}.

Old Code:

\begin{verbatim}
<liferay-portlet:resourceURL copyCurrentRenderParameters="<%= false %>" var="exportRecordSetURL">
    <portlet:param name="<%= ActionRequest.ACTION_NAME %>" value="exportRecordSet" />
    <portlet:param name="recordSetId" value="<%= String.valueOf(recordSet.getRecordSetId()) %>" />
</liferay-portlet:resourceURL>
\end{verbatim}

New Code:

\begin{verbatim}
<liferay-portlet:resourceURL copyCurrentRenderParameters="<%= false %>" id="exportRecordSet" var="exportRecordSetURL">
    <portlet:param name="recordSetId" value="<%= String.valueOf(recordSet.getRecordSetId()) %>" />
</liferay-portlet:resourceURL>
\end{verbatim}

\paragraph{Why was this change made?}\label{why-was-this-change-made-66}

This change was made to extend the MVC framework to have better support
for \texttt{actionURL}s and \texttt{resourceURL}s.

\subsubsection{Changed Java Package Names for Portlets Extracted as
Modules}\label{changed-java-package-names-for-portlets-extracted-as-modules}

\begin{itemize}
\tightlist
\item
  \textbf{Date:} 2015-Jun-29
\item
  \textbf{JIRA Ticket:} LPS-56383 and others
\end{itemize}

\paragraph{What changed?}\label{what-changed-67}

The Java package names changed for portlets that were extracted as OSGi
modules in 7.0. Here is the complete list:

\begin{itemize}
\tightlist
\item
  \texttt{com.liferay.portlet.bookmarks} →
  \texttt{com.liferay.bookmarks}
\item
  \texttt{com.liferay.portlet.dynamicdatalists} →
  \texttt{com.liferay.dynamicdatalists}
\item
  \texttt{com.liferay.portlet.journal} → \texttt{com.liferay.journal}
\item
  \texttt{com.liferay.portlet.polls} → \texttt{com.liferay.polls}
\item
  \texttt{com.liferay.portlet.wiki} → \texttt{com.liferay.wiki}
\end{itemize}

\paragraph{Who is affected?}\label{who-is-affected-67}

This affects developers using the portlets API from their own plugins.

\paragraph{How should I update my
code?}\label{how-should-i-update-my-code-67}

Update the package imports to use the new package names. Any literal
usage of the portlet \texttt{className} should also be updated.

\paragraph{Why was this change made?}\label{why-was-this-change-made-67}

Package names have been adapted to the new condition of Liferay portlets
as OSGi services.

\subsubsection{Removed the DLFileEntryTypes\_DDMStructures Mapping
Table}\label{removed-the-dlfileentrytypes_ddmstructures-mapping-table}

\begin{itemize}
\tightlist
\item
  \textbf{Date:} 2015-Jul-01
\item
  \textbf{JIRA Ticket:} LPS-56660
\end{itemize}

\paragraph{What changed?}\label{what-changed-68}

The \texttt{DLFileEntryTypes\_DDMStructures} mapping table is no longer
available.

\paragraph{Who is affected?}\label{who-is-affected-68}

This affects developers using the Document Library File Entry Type Local
Service API.

\paragraph{How should I update my
code?}\label{how-should-i-update-my-code-68}

Update the calls to \texttt{addDDMStructureLinks},
\texttt{deleteDDMStructureLinks}, and \texttt{updateDDMStructureLinks}
if you want to add, delete, or update references between
\texttt{DLFileEntryType} and \texttt{DDMStructures}.

\paragraph{Why was this change made?}\label{why-was-this-change-made-68}

This change was made to reduce the coupling between the two
applications.

\subsubsection{Removed render Method from AssetRenderer API and
WorkflowHandler
API}\label{removed-render-method-from-assetrenderer-api-and-workflowhandler-api}

\begin{itemize}
\tightlist
\item
  \textbf{Date:} 2015-Jul-03
\item
  \textbf{JIRA Ticket:} LPS-56705
\end{itemize}

\paragraph{What changed?}\label{what-changed-69}

The method \texttt{render} has been removed from the interfaces
\texttt{AssetRenderer} and \texttt{WorkflowHandler}.

\paragraph{Who is affected?}\label{who-is-affected-69}

This affects any Java code calling the method \texttt{render} on an
\texttt{AssetRenderer} or \texttt{WorkflowHandler} class, or Java
classes overriding the \texttt{render} method of these classes.

\paragraph{How should I update my
code?}\label{how-should-i-update-my-code-69}

The method \texttt{render} was used to return the path of a JSP,
including the configuration of a portlet. That method is now available
for the same AssetRender API extending the \texttt{BaseJSPAssetRenderer}
class, and is called \texttt{getJspPath}.

If any logic was added to override the \texttt{render} method, it can
now be added in the \texttt{include} method.

\paragraph{Why was this change made?}\label{why-was-this-change-made-69}

This change was part of needed modifications to support adding asset
renderers and workflow handlers for portlets based on other technology
different than JSP (e.g., FreeMarker). The method \texttt{include} can
now be used to create asset renderers or workflow handlers with UIs
written in FreeMarker or any other framework.

\subsubsection{Renamed ADMIN\_INSTANCE to PORTAL\_INSTANCES in
PortletKeys}\label{renamed-admin_instance-to-portal_instances-in-portletkeys}

\begin{itemize}
\tightlist
\item
  \textbf{Date:} 2015-Jul-08
\item
  \textbf{JIRA Ticket:} LPS-56867
\end{itemize}

\paragraph{What changed?}\label{what-changed-70}

The constant \texttt{PortletKeys.ADMIN\_INSTANCE} has been renamed as
\texttt{PortletKeys.PORTAL\_INSTANCES}.

\paragraph{Who is affected?}\label{who-is-affected-70}

This affects developers using the old constant in their code; for
example, creating a direct link to it. This is not common and usually
not a good practice, so this should not affect many people.

\paragraph{How should I update my
code?}\label{how-should-i-update-my-code-70}

You should rename the constant \texttt{ADMIN\_INSTANCE} to
\texttt{PORTAL\_INSTANCES} everywhere it is used.

\paragraph{Why was this change made?}\label{why-was-this-change-made-70}

This change was part of needed modifications to extract the Portal
Instances portlet from the Admin portlet. The constant's old name was
not accurate, since it originated from the old Admin portlet. Since the
Portal Instances portlet is now extracted to its own module, the old
name no longer resembles its usage.

\subsubsection{Removed Support for filterFindBy Generation or
InlinePermissionUtil Usage for Tables When the Primary Key Type Is Not
long}\label{removed-support-for-filterfindby-generation-or-inlinepermissionutil-usage-for-tables-when-the-primary-key-type-is-not-long}

\begin{itemize}
\tightlist
\item
  \textbf{Date:} 2015-Jul-21
\item
  \textbf{JIRA Ticket:} LPS-54590
\end{itemize}

\paragraph{What changed?}\label{what-changed-71}

ServiceBuilder and inline permission filter support has been removed for
non-\texttt{long} primary key types.

\paragraph{Who is affected?}\label{who-is-affected-71}

This affects code that is using \texttt{int}, \texttt{float},
\texttt{double}, \texttt{boolean}, or \texttt{short} type primary keys
in the \texttt{service.xml} with inline permissions.

\paragraph{How should I update my
code?}\label{how-should-i-update-my-code-71}

You should change the primary key type to \texttt{long}.

\paragraph{Why was this change made?}\label{why-was-this-change-made-71}

Inline permissioning was using the \texttt{join} method between two
different data types and that caused significant performance degradation
with \texttt{filterFindBy} queries.

\subsubsection{Removed Vaadin 6 from Liferay
Core}\label{removed-vaadin-6-from-liferay-core}

\begin{itemize}
\tightlist
\item
  \textbf{Date:} 2015-Jul-31
\item
  \textbf{JIRA Ticket:} LPS-57525
\end{itemize}

\paragraph{What changed?}\label{what-changed-72}

The bundled Vaadin 6.x JAR file has been removed from portal core.

\paragraph{Who is affected?}\label{who-is-affected-72}

This affects developers who are creating Vaadin portlet applications in
Liferay Portal.

\paragraph{How should I update my
code?}\label{how-should-i-update-my-code-72}

You should upgrade to Vaadin 7, bundle your \texttt{vaadin.jar} with
your plugin, or deploy Vaadin libraries to Liferay's OSGi container.

\paragraph{Why was this change made?}\label{why-was-this-change-made-72}

Vaadin 6.x is outdated and there are no plans for any new projects to be
created with it. Therefore, developers should begin using Vaadin 7.x.

\subsubsection{Replaced the Navigation Menu Portlet's Display Styles
with
ADTs}\label{replaced-the-navigation-menu-portlets-display-styles-with-adts}

\begin{itemize}
\tightlist
\item
  \textbf{Date:} 2015-Jul-31
\item
  \textbf{JIRA Ticket:} LPS-27113
\end{itemize}

\paragraph{What changed?}\label{what-changed-73}

The custom display styles of the navigation tag added using JSPs no
longer work. They have been replaced by Application Display Templates
(ADT).

\paragraph{Who is affected?}\label{who-is-affected-73}

This affects developers that use portlet properties with the following
prefix:

\begin{verbatim}
navigation.display.style
\end{verbatim}

This also affects developers that use the following attribute in the
navigation tag:

\begin{verbatim}
displayStyleDefinition
\end{verbatim}

\paragraph{How should I update my
code?}\label{how-should-i-update-my-code-73}

To style the Navigation portlet, you should use ADTs instead of using
custom styles in your JSPs. ADTs can be created from the UI of the
portal by navigating to \emph{Site Settings} → \emph{Application Display
Templates}. ADTs can also be created programatically.

Developers should use the \texttt{ddmTemplateGroupId} and
\texttt{ddmTemplateKey} attributes of the navigation tag to set the ADT
that defines the style of the navigation.

\paragraph{Why was this change made?}\label{why-was-this-change-made-73}

ADTs allow you to change an application's look and feel without changing
its JSP code.

\subsubsection{Renamed URI Attribute Used to Generate AUI Tag
Library}\label{renamed-uri-attribute-used-to-generate-aui-tag-library}

\begin{itemize}
\tightlist
\item
  \textbf{Date:} 2015-Aug-12
\item
  \textbf{JIRA Ticket:} LPS-57809
\end{itemize}

\paragraph{What changed?}\label{what-changed-74}

The URI attribute used to identify the AUI taglib has been renamed.

\paragraph{Who is affected?}\label{who-is-affected-74}

This affects developers that use the URI
\texttt{http://alloy.liferay.com/tld/aui} in their JSPs, XMLs, etc.

\paragraph{How should I update my
code?}\label{how-should-i-update-my-code-74}

You should use the new AUI URI declaration:

Old:

\begin{verbatim}
http://alloy.liferay.com/tld/aui
\end{verbatim}

New:

\begin{verbatim}
http://liferay.com/tld/aui
\end{verbatim}

\paragraph{Why was this change made?}\label{why-was-this-change-made-74}

To stay consistent with other taglibs provided by Liferay, the AUI
\texttt{.tld} file was modified to start with the prefix
\texttt{liferay-}. Due to this change, the XML files used to
automatically generate the AUI taglib were modified, changing the AUI
URI declaration.

\subsubsection{Removed Support for runtime-portlet Tag in Body of Web
Content
Articles}\label{removed-support-for-runtime-portlet-tag-in-body-of-web-content-articles}

\begin{itemize}
\tightlist
\item
  \textbf{Date:} 2015-Sep-17
\item
  \textbf{JIRA Ticket:} LPS-58736
\end{itemize}

\paragraph{What changed?}\label{what-changed-75}

The tag \texttt{runtime-portlet} is no longer replaced by a portlet if
it is found in the body of a web content article.

\paragraph{Who is affected?}\label{who-is-affected-75}

This affects any web content in the database (\texttt{JournalArticle}
table) that uses this tag.

\paragraph{How should I update my
code?}\label{how-should-i-update-my-code-75}

Embedding another portlet is only supported from a template. You should
embed the portlet by passing its name in a call to
\texttt{theme.runtime} or using the right tag in FreeMarker.

\textbf{Example}

In Velocity:

\begin{verbatim}
$theme.runtime("145")
\end{verbatim}

In FreeMarker:

\begin{verbatim}
<#assign liferay_portlet = PortalJspTagLibs["/WEB-INF/tld/liferay-portlet-ext.tld"] />

<@liferay_portlet["runtime"] portletName="145" />
\end{verbatim}

\paragraph{Why was this change made?}\label{why-was-this-change-made-75}

This change improves the performance of web content articles while
enforcing a single way to embed portlets into the page for better
testing.

\subsubsection{Removed the liferay-ui:control-panel-site-selector
Tag}\label{removed-the-liferay-uicontrol-panel-site-selector-tag}

\begin{itemize}
\tightlist
\item
  \textbf{Date:} 2015-Sep-23
\item
  \textbf{JIRA Ticket:} LPS-58210
\end{itemize}

\paragraph{What changed?}\label{what-changed-76}

The tag \texttt{liferay-ui:control-panel-site-selector} has been
deleted.

\paragraph{Who is affected?}\label{who-is-affected-76}

This affects developers who use this tag in their code.

\paragraph{How should I update my
code?}\label{how-should-i-update-my-code-76}

You should consider using the tag \texttt{liferay-ui:my-sites}, or
create your own markup using the \texttt{GroupService} API.

\paragraph{Why was this change made?}\label{why-was-this-change-made-76}

This tag is no longer used and will no longer be maintained properly.

\subsubsection{Removed Methods Related to Control Panel in
PortalUtil}\label{removed-methods-related-to-control-panel-in-portalutil}

\begin{itemize}
\tightlist
\item
  \textbf{Date:} 2015-Sep-23
\item
  \textbf{JIRA Ticket:} LPS-58210
\end{itemize}

\paragraph{What changed?}\label{what-changed-77}

The following methods have been deleted:

\begin{itemize}
\tightlist
\item
  \texttt{getControlPanelCategoriesMap}
\item
  \texttt{getControlPanelCategory}
\item
  \texttt{getControlPanelPortlets}
\item
  \texttt{getFirstMyAccountPortlet}
\item
  \texttt{getFirstSiteAdministrationPortlet}
\item
  \texttt{getSiteAdministrationCategoriesMap}
\item
  \texttt{getSiteAdministrationURL}
\item
  \texttt{isCompanyControlPanelVisible}
\end{itemize}

\paragraph{Who is affected?}\label{who-is-affected-77}

This affects developers that use any of the methods listed above.

\paragraph{How should I update my
code?}\label{how-should-i-update-my-code-77}

In order to work with applications displayed in the Product Menu,
developers should call the \texttt{PanelCategoryRegistry} and
\texttt{PanelAppRegistry} classes located in the
\texttt{application-list-api} module. These classes allow developers to
interact with categories and applications in the Control Panel.

\paragraph{Why was this change made?}\label{why-was-this-change-made-77}

These methods are no longer used and they will not work properly since
they cannot call the \texttt{application-list-api} from the portal
context.

\subsubsection{Removed ThemeDisplay Methods Related to Control Panel and
Site
Administration}\label{removed-themedisplay-methods-related-to-control-panel-and-site-administration}

\begin{itemize}
\tightlist
\item
  \textbf{Date:} 2015-Sep-23
\item
  \textbf{JIRA Ticket:} LPS-58210
\end{itemize}

\paragraph{What changed?}\label{what-changed-78}

The following methods have been deleted:

\begin{itemize}
\tightlist
\item
  \texttt{getControlPanelCategory}
\item
  \texttt{getURLSiteAdministration}
\end{itemize}

\paragraph{Who is affected?}\label{who-is-affected-78}

This affects developers that use either of the methods listed above.

\paragraph{How should I update my
code?}\label{how-should-i-update-my-code-78}

Site Administration is not a site per se; some applications are
displayed in that context. To create a link to an application that is
displayed in Site Administration, developers should use the method
\texttt{PortalUtil.getControlPanelURL}. In order to obtain the first
application displayed in a section of the Product Menu, developers
should use the \texttt{application-list-api} module to call the
\texttt{PanelCategoryRegistry} and \texttt{PanelAppRegistry} classes.

\paragraph{Why was this change made?}\label{why-was-this-change-made-78}

These methods are no longer used and they will not work properly since
they cannot call the \texttt{application-list-api} from the portal
context.

\subsubsection{Removed Control Panel from List of Sites Returned by
Methods Group.getUserSitesGroups and
User.getMySiteGroups}\label{removed-control-panel-from-list-of-sites-returned-by-methods-group.getusersitesgroups-and-user.getmysitegroups}

\begin{itemize}
\tightlist
\item
  \textbf{Date:} 2015-Sep-23
\item
  \textbf{JIRA Ticket:} LPS-58862
\end{itemize}

\paragraph{What changed?}\label{what-changed-79}

The following methods had a boolean parameter to determine whether to
include the Control Panel group:

\begin{itemize}
\tightlist
\item
  \texttt{Group.getUserSitesGroups}
\item
  \texttt{User.getMySiteGroups}
\end{itemize}

This boolean parameter should no longer be used.

\paragraph{Who is affected?}\label{who-is-affected-79}

This affects developers that use either of the methods listed above
passing the \texttt{includeControlPanel} parameter as \texttt{true}.

\paragraph{How should I update my
code?}\label{how-should-i-update-my-code-79}

If you don't need the Control Panel, remove the \texttt{false}
parameter. If you still want to obtain a link to the Control Panel, you
should do it in a different way.

The Control Panel is not a site per se; some applications are displayed
in that context. To create a link to an application that is displayed in
the Control Panel, developers should use the method
\texttt{PortalUtil.getControlPanelURL}. In order to obtain the first
application displayed in a section of the Product Menu, developers
should use the \texttt{application-list-api} module to call the
\texttt{PanelCategoryRegistry} and \texttt{PanelAppRegistry} classes.

\paragraph{Why was this change made?}\label{why-was-this-change-made-79}

The Control Panel is no longer a site per se, but just a context in
which some applications are displayed. This concept conflicts with the
idea of returning a site called Control Panel in the Sites API.

\subsubsection{Changed Exception Thrown by Documents and Media Services
When Duplicate Files are
Found}\label{changed-exception-thrown-by-documents-and-media-services-when-duplicate-files-are-found}

\begin{itemize}
\tightlist
\item
  \textbf{Date:} 2015-Sep-24
\item
  \textbf{JIRA Ticket:} LPS-53819
\end{itemize}

\paragraph{What changed?}\label{what-changed-80}

When a duplicate file entry is found by Documents and Media (D\&M)
services, a \texttt{DuplicateFileEntryException} will be thrown.
Previously, the exception \texttt{DuplicateFileException} was used.

The \texttt{DuplicateFileException} is now raised only by \texttt{Store}
implementations.

\paragraph{Who is affected?}\label{who-is-affected-80}

Any caller of the \texttt{addFileEntry} methods in \texttt{DLApp} and
\texttt{DLFileEntry} local and remote services is affected.

\paragraph{How should I update my
code?}\label{how-should-i-update-my-code-80}

Change the exception type from \texttt{DuplicateFileException} to
\texttt{DuplicateFileEntryException} in \texttt{try-catch} blocks
surrounding calls to D\&M services.

\paragraph{Why was this change made?}\label{why-was-this-change-made-80}

The \texttt{DuplicateFileException} exception was used in two different
contexts:

\begin{itemize}
\tightlist
\item
  When creating a new file through D\&M and a row in the database
  already existed for a file entry with the same title.
\item
  When the stores tried to save a file and the underlying storage unit
  (a file in the case of \texttt{FileSystemStore}) already existed.
\end{itemize}

This made it impossible to detect and recover from store corruption
issues, as they were undifferentiable from other errors.

\subsubsection{Removed All References to Windows Live
Messenger}\label{removed-all-references-to-windows-live-messenger}

\begin{itemize}
\tightlist
\item
  \textbf{Date:} 2015-Oct-15
\item
  \textbf{JIRA Ticket:} LPS-30883
\end{itemize}

\paragraph{What changed?}\label{what-changed-81}

All references to the \texttt{msnSn} column in the Contacts table have
been removed from portal. All references to Windows Live Messenger have
been removed from properties, tests, classes, and the frontend. Also,
the \texttt{getMsnSn} and \texttt{setMsnSn} methods have been removed
from the \texttt{Contact} and \texttt{LDAPUser} models.

The following classes have been removed:

\begin{itemize}
\tightlist
\item
  \texttt{MSNConnector}
\item
  \texttt{MSNMessageAdapter}
\end{itemize}

The following constants have been removed:

\begin{itemize}
\tightlist
\item
  \texttt{CalEventConstants.REMIND\_BY\_MSN}
\item
  \texttt{ContactConverterKeys.MSN\_SN}
\item
  \texttt{PropsKeys.MSN\_LOGIN}
\item
  \texttt{PropsKeys.MSN\_PASSWORD}
\end{itemize}

The following methods have been removed:

\begin{itemize}
\tightlist
\item
  \texttt{Contact.getMsnSn}
\item
  \texttt{Contact.setMsnSn}
\item
  \texttt{LDAPUser.getMsnSn}
\item
  \texttt{LDAPUser.setMsnSn}
\end{itemize}

The following methods have been changed:

\begin{itemize}
\tightlist
\item
  \texttt{AdminUtil.updateUser}
\item
  \texttt{ContactLocalServiceUtil.addContact}
\item
  \texttt{ContactLocalServiceUtil.updateContact}
\item
  \texttt{UserLocalServiceUtil.addContact}
\item
  \texttt{UserLocalServiceUtil.updateContact}
\item
  \texttt{UserLocalServiceUtil.updateUser}
\item
  \texttt{UserServiceUtil.updateUser}
\end{itemize}

\paragraph{Who is affected?}\label{who-is-affected-81}

This affects developers who use any of the classes, constants, or
methods listed above.

\paragraph{How should I update my
code?}\label{how-should-i-update-my-code-81}

When updating or adding a user or contact using one of the changed
methods above, remove the \texttt{msnSn} argument from the method call.
If you are using one of the removed items above, you should remove all
references to them from your code and look for alternatives, if
necessary. Lastly, remove any references to the \texttt{msnSN} column in
the Contacts table from your SQL queries.

\paragraph{Why was this change made?}\label{why-was-this-change-made-81}

Since Microsoft dropped support for Windows Live Messenger, Liferay will
no longer continue to support it.

\subsubsection{Removed Support for AIM, ICQ, MySpace, and Yahoo
Messenger}\label{removed-support-for-aim-icq-myspace-and-yahoo-messenger}

\begin{itemize}
\tightlist
\item
  \textbf{Date:} 2015-Oct-22
\item
  \textbf{JIRA Ticket:} LPS-59716
\end{itemize}

\paragraph{What changed?}\label{what-changed-82}

Liferay no longer supports integration with MySpace and AIM, ICQ, and
Yahoo Messenger instant messaging services. The corresponding
\texttt{aimSn}, \texttt{icqSn}, \texttt{mySpaceSn}, and \texttt{ymSn}
columns have been removed from the \texttt{Contacts} table.

The following classes have been removed:

\begin{itemize}
\tightlist
\item
  \texttt{AIMConnector}
\item
  \texttt{ICQConnector}
\item
  \texttt{YMConnector}
\end{itemize}

The following constants have been removed:

\begin{itemize}
\tightlist
\item
  \texttt{CalEventConstants.REMIND\_BY\_AIM}
\item
  \texttt{CalEventConstants.REMIND\_BY\_ICQ}
\item
  \texttt{CalEventConstants.REMIND\_BY\_YM}
\item
  \texttt{ContactConverterKeys.AIM\_SN}
\item
  \texttt{ContactConverterKeys.ICQ\_SN}
\item
  \texttt{ContactConverterKeys.MYSPACE\_SN}
\item
  \texttt{ContactConverterKeys.YM\_SN}
\item
  \texttt{PropsKeys.AIM\_LOGIN}
\item
  \texttt{PropsKeys.AIM\_PASSWORD}
\item
  \texttt{PropsKeys.ICQ\_JAR}
\item
  \texttt{PropsKeys.ICQ\_LOGIN}
\item
  \texttt{PropsKeys.ICQ\_PASSWORD}
\item
  \texttt{PropsKeys.YM\_LOGIN}
\item
  \texttt{PropsKeys.YM\_PASSWORD}
\end{itemize}

The following methods have been removed:

\begin{itemize}
\tightlist
\item
  \texttt{getAimSn}
\item
  \texttt{getIcqSn}
\item
  \texttt{getMySpaceSn}
\item
  \texttt{getYmSn}
\item
  \texttt{setAimSn}
\item
  \texttt{setIcqSn}
\item
  \texttt{setMySpaceSn}
\item
  \texttt{setYmSn}
\end{itemize}

The following methods have been changed:

\begin{itemize}
\tightlist
\item
  \texttt{updateUser}
\item
  \texttt{addContact}
\end{itemize}

The following portal properties have been removed:

\begin{itemize}
\tightlist
\item
  \texttt{aim.login}
\item
  \texttt{aim.password}
\item
  \texttt{icq.jar}
\item
  \texttt{icq.login}
\item
  \texttt{icq.password}
\item
  \texttt{ym.login}
\item
  \texttt{ym.password}
\end{itemize}

\paragraph{Who is affected?}\label{who-is-affected-82}

This affects developers who use any of the classes, constants, methods,
or properties listed above.

\paragraph{How should I update my
code?}\label{how-should-i-update-my-code-82}

When updating or adding a user or contact using one of the changed
methods above, remove the \texttt{aimSn}, \texttt{icqSn},
\texttt{mySpaceSn}, and \texttt{ymSn} arguments from the method call. If
you are using one of the removed items above, you should remove all
references to them from your code and look for alternatives, if
necessary. Lastly, remove from your SQL queries any references to former
\texttt{Contacts} table columns \texttt{aimSn}, \texttt{icqSn},
\texttt{mySpaceSn}, and \texttt{ymSn}.

Also, a reference to any one of the removed portal properties above no
longer returns a value.

\paragraph{Why was this change made?}\label{why-was-this-change-made-82}

The services removed in this change are no longer popular enough to
merit continued support.

\subsubsection{Removed All Methods from SchedulerEngineHelper that
Explicitly Schedule Jobs Using SchedulerEntry or Specify MessageListener
Class
Names}\label{removed-all-methods-from-schedulerenginehelper-that-explicitly-schedule-jobs-using-schedulerentry-or-specify-messagelistener-class-names}

\begin{itemize}
\tightlist
\item
  \textbf{Date:} 2015-Oct-29
\item
  \textbf{JIRA Ticket:} LPS-59681
\end{itemize}

\paragraph{What changed?}\label{what-changed-83}

The following methods were removed from \texttt{SchedulerEngine}:

\begin{itemize}
\tightlist
\item
  \texttt{SchedulerEngineHelper.addJob(Trigger,\ StorageType,\ String,\ String,\ Message,\ String,\ String,\ int)}
\item
  \texttt{SchedulerEngineHelper.addJob(Trigger,\ StorageType,\ String,\ String,\ Object,\ String,\ String,\ int)}
\item
  \texttt{SchedulerEngineHelper.schedule(SchedulerEntry,\ StorageType,\ String,\ int)}
\end{itemize}

\paragraph{Who is affected?}\label{who-is-affected-83}

This affects developers that use the above methods to schedule jobs into
the \texttt{SchedulerEngine}.

\paragraph{How should I update my
code?}\label{how-should-i-update-my-code-83}

You should update your code to call one of these methods:

\begin{itemize}
\tightlist
\item
  \texttt{SchedulerEngineHelper.schedule(Trigger,\ StorageType,\ String,\ String,\ Message,\ int)}
\item
  \texttt{SchedulerEngineHelper.schedule(Trigger,\ StorageType,\ String,\ String,\ Object,\ int)}
\end{itemize}

Instead of simply providing the class name of your scheduled job
listener, you should follow these steps:

\begin{enumerate}
\def\labelenumi{\arabic{enumi}.}
\item
  Instantiate your MessageListener.
\item
  Call
  \texttt{SchedulerEngineHelper.register(MessageListener,\ SchedulerEntry)}
  to register your \texttt{SchedulerEventMessageListener}.
\end{enumerate}

\paragraph{Why was this change made?}\label{why-was-this-change-made-83}

The deleted methods provided facilities that aren't compatible with
using declarative services in an OSGi container. The new approach allows
for proper injection of dependencies into scheduled event message
listeners.

\subsubsection{Removed the asset.publisher.query.form.configuration
Property}\label{removed-the-asset.publisher.query.form.configuration-property}

\begin{itemize}
\tightlist
\item
  \textbf{Date:} 2015-Nov-03
\item
  \textbf{JIRA Ticket:} LPS-60119
\end{itemize}

\paragraph{What changed?}\label{what-changed-84}

The \texttt{asset.publisher.query.form.configuration} property has been
removed from \texttt{portal.properties}.

\paragraph{Who is affected?}\label{who-is-affected-84}

This affects any hook that uses the
\texttt{asset.publisher.query.form.configuration} property.

\paragraph{How should I update my
code?}\label{how-should-i-update-my-code-84}

If you are using this property to generate the UI for an Asset Entry
Query Processor, your Asset Entry Query Processor must now implement the
\texttt{include} method to generate the UI.

\paragraph{Why was this change made?}\label{why-was-this-change-made-84}

This change was made as a part of the ongoing strategy to modularize
Liferay Portal.

\subsubsection{Removed Hover and Alternate Style Features of Search
Container
Tag}\label{removed-hover-and-alternate-style-features-of-search-container-tag}

\begin{itemize}
\tightlist
\item
  \textbf{Date:} 2015-Nov-03
\item
  \textbf{JIRA Ticket:} LPS-58854
\end{itemize}

\paragraph{What changed?}\label{what-changed-85}

The following attributes and methods have been removed:

\begin{itemize}
\tightlist
\item
  The attribute \texttt{hover} of the
  \texttt{liferay-ui:search-container} tag.
\item
  The method \texttt{isHover()} of the \texttt{SearchContainerTag}
  class.
\item
  The attributes \texttt{classNameHover}, \texttt{hover},
  \texttt{rowClassNameAlternate}, \texttt{rowClassNameAlternateHover},
  \texttt{rowClassNameBody}, \texttt{rowClassNameBodyHover} of the
  \texttt{liferay-search-container} JavaScript module.
\end{itemize}

\paragraph{Who is affected?}\label{who-is-affected-85}

This affects developers that use the \texttt{hover} attribute of the
\texttt{liferay-ui:search-container} tag.

\paragraph{How should I update my
code?}\label{how-should-i-update-my-code-85}

You should update your code changing the CSS selector that defines how
rows look on hover to use the \texttt{:hover} and \texttt{:nth-of-type}
CSS pseudo selectors instead.

\paragraph{Why was this change made?}\label{why-was-this-change-made-85}

Browsers support better ways to style content on hover in a way that
doesn't penalize performance. Therefore, this change was made to
increase the performance of hovering over content in Liferay.

\subsubsection{Removed AppViewMove and AppViewSelect JavaScript
Modules}\label{removed-appviewmove-and-appviewselect-javascript-modules}

\begin{itemize}
\tightlist
\item
  \textbf{Date:} 2015-Nov-03
\item
  \textbf{JIRA Ticket:} LPS-58854
\end{itemize}

\paragraph{What changed?}\label{what-changed-86}

The JavaScript modules \texttt{AppViewMove} and \texttt{AppViewSelect}
have been removed.

\paragraph{Who is affected?}\label{who-is-affected-86}

This affects developers that use these modules to configure
\emph{select} and \emph{move} actions inside their applications.

\paragraph{How should I update my
code?}\label{how-should-i-update-my-code-86}

If you are using any of these modules, you can make use of the following
\texttt{SearchContainer} APIs:

\begin{itemize}
\tightlist
\item
  Listen to the \texttt{rowToggled} event of the search container to be
  notified about changes to the search container state.
\item
  Configure your search container \emph{move} options creating a
  \texttt{RowMover} and define the allowed \emph{move} targets and
  associated actions.
\item
  Use the \texttt{registerAction} method of the search container to
  execute your \emph{move} logic when the user completes a \emph{move}
  action.
\end{itemize}

\paragraph{Why was this change made?}\label{why-was-this-change-made-86}

The removed JavaScript modules contained too much logic and were
difficult to decipher. It was also difficult to add this to an existing
app. With this change, every app using a search container can use this
functionality much easier.

\subsubsection{Removed the mergeLayoutTags Preference from Asset
Publisher}\label{removed-the-mergelayouttags-preference-from-asset-publisher}

\begin{itemize}
\tightlist
\item
  \textbf{Date:} 2015-Nov-20
\item
  \textbf{JIRA Ticket:} LPS-60677
\end{itemize}

\paragraph{What changed?}\label{what-changed-87}

The \texttt{mergeLayoutTags} preference has been removed from the Asset
Publisher.

\paragraph{Who is affected?}\label{who-is-affected-87}

This affects any Asset Publisher portlet that uses this preference.

\paragraph{How should I update my
code?}\label{how-should-i-update-my-code-87}

There is nothing to update since this functionality is no longer used.

\paragraph{Why was this change made?}\label{why-was-this-change-made-87}

In previous versions of Liferay, some applications such as Blogs and
Wiki shared the tags of their entries within the page. The Asset
Publisher was able to use them to show other assets with the same tags.
This functionality has changed, so the preference is no longer used.

\subsubsection{Removed the liferay-ui:navigation Tag and Replaced with
liferay-site-navigation:navigation
Tag}\label{removed-the-liferay-uinavigation-tag-and-replaced-with-liferay-site-navigationnavigation-tag}

\begin{itemize}
\tightlist
\item
  \textbf{Date:} 2015-Nov-20
\item
  \textbf{JIRA Ticket:} LPS-60328
\end{itemize}

\paragraph{What changed?}\label{what-changed-88}

The \texttt{liferay-ui:navigation} tag has been removed and replaced
with the \texttt{liferay-site-navigation:navigation} tag.

\paragraph{Who is affected?}\label{who-is-affected-88}

Plugins or templates that are using the \texttt{liferay-ui:navigation}
tag need to update their usage of the tag.

\paragraph{How should I update my
code?}\label{how-should-i-update-my-code-88}

You should import the \texttt{liferay-site-navigation} tag library (if
necessary) and update the tag namespace from
\texttt{liferay-ui:navigation} to
\texttt{liferay-site-navigation:navigation}.

\paragraph{Why was this change made?}\label{why-was-this-change-made-88}

This change was made as a part of the ongoing strategy to modularize
Liferay Portal by means of an OSGi container.

\subsubsection{Removed Software Catalog Portlet and
Services}\label{removed-software-catalog-portlet-and-services}

\begin{itemize}
\tightlist
\item
  \textbf{Date:} 2015-Nov-21
\item
  \textbf{JIRA Ticket:} LPS-60705
\end{itemize}

\paragraph{What changed?}\label{what-changed-89}

The Software Catalog portlet and its associated services are no longer
part of Liferay's source code or binaries.

\paragraph{Who is affected?}\label{who-is-affected-89}

This affects portals which were making use of the Software Catalog
portlet to manage a catalog of their software. Developers who were
making use of the software catalog services from their custom code are
also affected.

\paragraph{How should I update my
code?}\label{how-should-i-update-my-code-89}

There is no direct replacement for invocations to the Software Catalog
services. In cases where it is really needed, it is possible to obtain
the code from a previous release and include it in the custom product
(subject to licensing).

\paragraph{Why was this change made?}\label{why-was-this-change-made-89}

The Software Catalog was developed to implement the very first versions
of what later become Liferay's Marketplace. It was later replaced and
has not been used by Liferay since then. It has also been used minimally
outside of Liferay. The decision was made to remove it so Liferay could
be more lightweight and free time to focus on other areas of the product
that add more value.

\subsubsection{Removed the liferay-ui:asset-categories-navigation Tag
and Replaced with
liferay-asset:asset-categories-navigation}\label{removed-the-liferay-uiasset-categories-navigation-tag-and-replaced-with-liferay-assetasset-categories-navigation}

\begin{itemize}
\tightlist
\item
  \textbf{Date:} 2015-Nov-25
\item
  \textbf{JIRA Ticket:} LPS-60753
\end{itemize}

\paragraph{What changed?}\label{what-changed-90}

The \texttt{liferay-ui:asset-categories-navigation} tag has been removed
and replaced with the \texttt{liferay-asset:asset-categories-navigation}
tag.

\paragraph{Who is affected?}\label{who-is-affected-90}

Plugins or templates that are using the
\texttt{liferay-ui:asset-categories-navigation} tag need to update their
usage of the tag.

\paragraph{How should I update my
code?}\label{how-should-i-update-my-code-90}

You should import the \texttt{liferay-asset} tag library (if necessary)
and update the tag namespace from
\texttt{liferay-ui:asset-categories-navigation} to
\texttt{liferay-asset:asset-categories-navigation}.

\paragraph{Why was this change made?}\label{why-was-this-change-made-90}

This change was made as a part of the ongoing strategy to modularize
Liferay Portal by means of an OSGi container.

\subsubsection{Removed the liferay-ui:trash-empty Tag and Replaced with
liferay-trash:empty}\label{removed-the-liferay-uitrash-empty-tag-and-replaced-with-liferay-trashempty}

\begin{itemize}
\tightlist
\item
  \textbf{Date:} 2015-Nov-30
\item
  \textbf{JIRA Ticket:} LPS-60779
\end{itemize}

\paragraph{What changed?}\label{what-changed-91}

The \texttt{liferay-ui:trash-empty} tag has been removed and replaced
with the \texttt{liferay-trash:empty} tag.

\paragraph{Who is affected?}\label{who-is-affected-91}

Plugins and templates that are using the \texttt{liferay-ui:trash-empty}
tag need to update their usage of the tag.

\paragraph{How should I update my
code?}\label{how-should-i-update-my-code-91}

You should import the \texttt{liferay-trash} tag library (if necessary)
and update the tag namespace from \texttt{liferay-ui:trash-empty} to
\texttt{liferay-trash:empty}.

\paragraph{Why was this change made?}\label{why-was-this-change-made-91}

This change was made as a part of the ongoing strategy to modularize
Liferay Portal by means of an OSGi container.

\subsubsection{Removed the liferay-ui:trash-undo Tag and Replaced with
liferay-trash:undo}\label{removed-the-liferay-uitrash-undo-tag-and-replaced-with-liferay-trashundo}

\begin{itemize}
\tightlist
\item
  \textbf{Date:} 2015-Nov-30
\item
  \textbf{JIRA Ticket:} LPS-60779
\end{itemize}

\paragraph{What changed?}\label{what-changed-92}

The \texttt{liferay-ui:trash-undo} taglib has been removed and replaced
with the \texttt{liferay-trash:undo} tag.

\paragraph{Who is affected?}\label{who-is-affected-92}

Plugins and templates that are using the \texttt{liferay-ui:trash-undo}
tag need to update their usage of the tag.

\paragraph{How should I update my
code?}\label{how-should-i-update-my-code-92}

You should import the \texttt{liferay-trash} tag library (if necessary)
and update the tag namespace from \texttt{liferay-ui:trash-undo} to
\texttt{liferay-trash:undo}.

\paragraph{Why was this change made?}\label{why-was-this-change-made-92}

This change was made as a part of the ongoing strategy to modularize
Liferay Portal by means of an OSGi container.

\subsubsection{Removed the getPageOrderByComparator Method from
WikiUtil}\label{removed-the-getpageorderbycomparator-method-from-wikiutil}

\begin{itemize}
\tightlist
\item
  \textbf{Date:} 2015-Dec-01
\item
  \textbf{JIRA Ticket:} LPS-60843
\end{itemize}

\paragraph{What changed?}\label{what-changed-93}

The \texttt{getPageOrderByComparator} method has been removed from
\texttt{WikiUtil}.

\paragraph{Who is affected?}\label{who-is-affected-93}

This affects developers that use this method in their code.

\paragraph{How should I update my
code?}\label{how-should-i-update-my-code-93}

You should update your code to invoke
\texttt{WikiPortletUtil.getPageOrderByComparator(String,\ String)}.

\paragraph{Why was this change made?}\label{why-was-this-change-made-93}

As part of the modularization efforts it has been considered that that
this logic belongs to wiki-web module.

\subsubsection{Custom AUI Validators Are No Longer Implicitly
Required}\label{custom-aui-validators-are-no-longer-implicitly-required}

\begin{itemize}
\tightlist
\item
  \textbf{Date:} 2015-Dec-02
\item
  \textbf{JIRA Ticket:} LPS-60995
\end{itemize}

\paragraph{What changed?}\label{what-changed-94}

The AUI Validator tag no longer forces custom validators (e.g.,
\texttt{name="custom"}) to be required, and are now optional by default.

\paragraph{Who is affected?}\label{who-is-affected-94}

This affects developers using custom validators, especially ones who
relied on the field being implicitly required via the custom validator.

\paragraph{How should I update my
code?}\label{how-should-i-update-my-code-94}

There are several cases where you should update your code to compensate
for this change. First, blank value checking is no longer necessary, so
places where blank values are checked should be updated.

Old Code:

\begin{verbatim}
<aui:input name="privateVirtualHost">
    <aui:validator errorMessage="please-enter-a-unique-virtual-host" name="custom">
        function(val, fieldNode, ruleValue) {
            return !val || val != A.one('#<portlet:namespace />publicVirtualHost').val();
        }
    </aui:validator>
</aui:input>
\end{verbatim}

New Code:

\begin{verbatim}
<aui:input name="privateVirtualHost">
    <aui:validator errorMessage="please-enter-a-unique-virtual-host" name="custom">
        function(val, fieldNode, ruleValue) {
            return val != A.one('#<portlet:namespace />publicVirtualHost').val();
        }
    </aui:validator>
</aui:input>
\end{verbatim}

Also, instead of using custom validators to determine if a field is
required, you should now use a conditional \texttt{required} validator.

Old Code:

\begin{verbatim}
<aui:input name="file" type="file" />

<aui:input name="title">
    <aui:validator errorMessage="you-must-specify-a-file-or-a-title" name="custom">
        function(val, fieldNode, ruleValue) {
            return !!val || !!A.one('#<portlet:namespace />file').val();
        }
    </aui:validator>
</aui:input>
\end{verbatim}

New Code:

\begin{verbatim}
<aui:input name="file" type="file" />

<aui:input name="title">
    <aui:validator errorMessage="you-must-specify-a-file-or-a-title" name="required">
        function(fieldNode) {
            return !A.one('#<portlet:namespace />file').val();
        }
    </aui:validator>
</aui:input>
\end{verbatim}

Lastly, custom validators that assumed validation would always run must
now explicitly pass the \texttt{required} validator. This is done by
passing in the
\texttt{\textless{}aui:validator\ name="required"\ /\textgreater{}}
element. The \texttt{\textless{}aui:input\textgreater{}} tag listed
below is an example of how to explicity pass the \texttt{required}
validator:

\begin{verbatim}
<aui:input name="vowelsOnly">
    <aui:validator errorMessage="must-contain-only-the-following-characters" name="custom">
        function(val, fieldNode, ruleValue) {
            var allowedCharacters = 'aeiouy';
            var regex = new RegExp('[^' + allowedCharacters + ']');

            return !regex.test(val);
        }
    </aui:validator>
    <aui:validator name="required" />
</aui:input>
\end{verbatim}

\paragraph{Why was this change made?}\label{why-was-this-change-made-94}

A custom validator caused the field to be implicitly required. This
meant that all validators for the field would be evaluated. This created
a condition where you could not combine custom validators with another
validator for an optional field.

For example, imagine an optional field which has an email validator,
plus a custom validator which checks for email addresses within a
specific domain (e.g., \texttt{example.com}). There was no way for this
optional field to pass validation. Even if you handled blank values in
your custom validator, that blank value would fail the email validator.

This change requires most custom validators to be refactored, but allows
greater flexibility for all developers.

\subsubsection{Moved Recycle Bin Logic Into a New DLTrashService
Interface}\label{moved-recycle-bin-logic-into-a-new-dltrashservice-interface}

\begin{itemize}
\tightlist
\item
  \textbf{Date:} 2015-Dec-02
\item
  \textbf{JIRA Ticket:} LPS-60810
\end{itemize}

\paragraph{What changed?}\label{what-changed-95}

All Recycle Bin logic in Documents and Media services was moved from
\texttt{DLAppService} into the new \texttt{DLTrashService} service
interface. All moved methods have the same name and signatures.

\paragraph{Who is affected?}\label{who-is-affected-95}

This affects any local or remote caller of \texttt{DLAppService}.

\paragraph{How should I update my
code?}\label{how-should-i-update-my-code-95}

As all methods have been simply moved into the new service, calling the
equivalent method on \texttt{DLTrashService} suffices.

\paragraph{Why was this change made?}\label{why-was-this-change-made-95}

Documents and Media services have complex interdependencies that result
in circular dependencies. Until now, \texttt{DLAppService} was
responsible for exposing the Recycle Bin logic, delegating it to other
components. The problem was, the components depended on
\texttt{DLAppService} to implement their logic. Extracting the services
from \texttt{DLAppService} was the only sensible solution to this
circularity.

\subsubsection{Deprecated the liferay-ui:flags Tag and Replaced with
liferay-flags:flags}\label{deprecated-the-liferay-uiflags-tag-and-replaced-with-liferay-flagsflags}

\begin{itemize}
\tightlist
\item
  \textbf{Date:} 2015-Dec-02
\item
  \textbf{JIRA Ticket:} LPS-60967
\end{itemize}

\paragraph{What changed?}\label{what-changed-96}

The \texttt{liferay-ui:flags} tag has been deprecated and replaced with
the \texttt{liferay-flags:flags} tag.

\paragraph{Who is affected?}\label{who-is-affected-96}

Plugins or templates that are using the \texttt{liferay-ui:flags} tag
need to update their usage of the tag.

\paragraph{How should I update my
code?}\label{how-should-i-update-my-code-96}

You should import the \texttt{liferay-flags} tag library (if necessary)
and update the tag namespace from \texttt{liferay-ui:flags} to
\texttt{liferay-flags:flags}.

\paragraph{Why was this change made?}\label{why-was-this-change-made-96}

This change was made as a part of the ongoing strategy to modularize
Liferay Portal by means of an OSGi container.

\subsubsection{Removed the liferay-ui:diff Tag and Replaced with
liferay-frontend:diff}\label{removed-the-liferay-uidiff-tag-and-replaced-with-liferay-frontenddiff}

\begin{itemize}
\tightlist
\item
  \textbf{Date:} 2015-Dec-14
\item
  \textbf{JIRA Ticket:} LPS-61326
\end{itemize}

\paragraph{What changed?}\label{what-changed-97}

The \texttt{liferay-ui:diff} tag has been removed and replaced with the
\texttt{liferay-frontend:diff} tag.

\paragraph{Who is affected?}\label{who-is-affected-97}

Plugins and templates that are using the \texttt{liferay-ui:diff} tag
need to update their usage of the tag.

\paragraph{How should I update my
code?}\label{how-should-i-update-my-code-97}

You should import the \texttt{liferay-frontend} tag library (if
necessary) and update the tag namespace from \texttt{liferay-ui:diff} to
\texttt{liferay-frontend:diff}.

\paragraph{Why was this change made?}\label{why-was-this-change-made-97}

This change was made as a part of the ongoing strategy to modularize
Liferay Portal by means of an OSGi container.

\subsubsection{Taglibs Are No Longer Accessible via the theme Variable
in
FreeMarker}\label{taglibs-are-no-longer-accessible-via-the-theme-variable-in-freemarker}

\begin{itemize}
\tightlist
\item
  \textbf{Date:} 2016-Jan-06
\item
  \textbf{JIRA Ticket:} LPS-61683
\end{itemize}

\paragraph{What changed?}\label{what-changed-98}

The \texttt{\$\{theme\}} variable previously injected in the FreeMarker
context providing access to various tags and utilities is no longer
available.

\paragraph{Who is affected?}\label{who-is-affected-98}

This affects FreeMarker templates that are using the
\texttt{\$\{theme\}} variable.

\paragraph{How should I update my
code?}\label{how-should-i-update-my-code-98}

All the tags and utility methods formerly accessed via the
\texttt{\$\{theme\}} variable should now be accessed directly via tags.

\textbf{Example 1}

\begin{verbatim}
${theme.runtime("com.liferay.portal.kernel.servlet.taglib.ui.BreadcrumbEntry", portletProviderAction.VIEW, "", default_preferences)}
\end{verbatim}

can be replaced by:

\begin{verbatim}
<@liferay_portlet["runtime"]
    defaultPreferences=default_preferences
    portletProviderAction=portletProviderAction.VIEW
    portletProviderClassName="com.liferay.portal.kernel.servlet.taglib.ui.BreadcrumbEntry"
/>
\end{verbatim}

\textbf{Example 2}

\begin{verbatim}
${theme.include(content_include)}
\end{verbatim}

can be replaced by:

\begin{verbatim}
<@liferay_util["include"] page=content_include />
\end{verbatim}

\textbf{Example 3}

\begin{verbatim}
${theme.wrapPortlet("portlet.ftl", content_include)}
\end{verbatim}

can be replaced by:

\begin{verbatim}
<@liferay_theme["wrap-portlet"] page="portlet.ftl">
    <@liferay_util["include"] page=content_include />
</@>
\end{verbatim}

\textbf{Example 4}

\begin{verbatim}
${theme.iconHelp(portlet_description)}
\end{verbatim}

can be replaced by:

\begin{verbatim}
<@liferay_ui["icon-help"] message=portlet_description />
\end{verbatim}

\textbf{Example 5}

\begin{verbatim}
${nav_item.icon()}
\end{verbatim}

can be replaced by:

\begin{verbatim}
<@liferay_theme["layout-icon"] layout=${nav_item.getLayout()} />
\end{verbatim}

\paragraph{Why was this change made?}\label{why-was-this-change-made-98}

Previously, the \texttt{\{\$theme\}} variable was being injected with
the \texttt{VelocityTaglibImpl} class. This created coupling between
template engines and coupling between specific tags and template engines
at the same time.

FreeMarker already offers native support for tags which cover all the
functionality originally provided by the \texttt{\{\$theme\}} variable.
Removing this coupling helps future development while still keeping all
the existing functionality.

\subsubsection{Portlet Configuration Options May Not Always Be
Displayed}\label{portlet-configuration-options-may-not-always-be-displayed}

\begin{itemize}
\tightlist
\item
  \textbf{Date:} 2016-Jan-07
\item
  \textbf{JIRA Ticket:} LPS-54620 and LPS-61820
\end{itemize}

\paragraph{What changed?}\label{what-changed-99}

The portlet configuration options (e.g., configuration, export/import,
look and feel, etc.) were always displayed in every view of the portlet
and couldn't be customized.

With Lexicon, the configuration options displayed are based on the
portlet's context, so not all options will always be displayed.

\paragraph{Who is affected?}\label{who-is-affected-99}

This affects portlets that should always display all configuration
options no matter which view of the portlet is rendered.

\paragraph{How should I update my
code?}\label{how-should-i-update-my-code-99}

If you don't apply any change to your source code, you will experience
the following behaviors based on the portlet type:

\begin{itemize}
\item
  \textbf{Struts Portlet:} If you've defined a \texttt{view-action} init
  parameter, the configuration options are only displayed for that
  particular view when invoking a URL with a parameter
  \texttt{struts\_action} with the value indicated in the
  \texttt{view-action} init parameter and also in the default view of
  the portlet (when there is no \texttt{struts\_action} parameter in the
  request).
\item
  \textbf{Liferay MVC Portlet:} If you've defined a
  \texttt{view-template} init parameter, the configuration options are
  only displayed when that template is rendered by invoking a URL with a
  parameter \texttt{mvcPath} with the value indicated in the
  \texttt{view-template} init parameter. and also in the default view of
  the portlet (when there is no \texttt{mvcPath} parameter in the
  request).
\item
  If it's a portlet using any other framework, the configuration options
  are always displayed.
\end{itemize}

In order to keep the old behavior of adding the configuration options in
every view, you need to add the init parameter
\texttt{always-display-default-configuration-icons} with the value
\texttt{true}.

\paragraph{Why was this change made?}\label{why-was-this-change-made-99}

Lexicon patterns require the ability to specify different configuration
options depending on the view of the portlet by adding or removing
options. This can be easily achieved by using the
\texttt{PortletConfigurationIcon} classes.

\subsubsection{The getURLView Method of AssetRenderer Returns String
Instead of
PortletURL}\label{the-geturlview-method-of-assetrenderer-returns-string-instead-of-portleturl}

\begin{itemize}
\tightlist
\item
  \textbf{Date:} 2016-Jan-08
\item
  \textbf{JIRA Ticket:} LPS-61853
\end{itemize}

\paragraph{What changed?}\label{what-changed-100}

The \texttt{AssetRenderer} interface's \texttt{getURLView} method has
changed and now returns \texttt{String} instead of \texttt{PortletURL}.

\paragraph{Who is affected?}\label{who-is-affected-100}

This affects all custom assets that implement the \texttt{AssetRenderer}
interface.

\paragraph{How should I update my
code?}\label{how-should-i-update-my-code-100}

You should update the method signature to reflect that it returns a
\texttt{String} and you should adapt your implementation accordingly.

In general, it should be as easy as returning
\texttt{portletURL.toString()}.

\paragraph{Why was this change
made?}\label{why-was-this-change-made-100}

The API was forcing implementations to return a \texttt{PortletURL},
making it difficult to return another type of link. For example, in the
case of Bookmarks, developers wanted to automatically redirect to other
potential URLs.

\subsubsection{Removed the icon Method from
NavItem}\label{removed-the-icon-method-from-navitem}

\begin{itemize}
\tightlist
\item
  \textbf{Date:} 2016-Jan-11
\item
  \textbf{JIRA Ticket:} LPS-61900
\end{itemize}

\paragraph{What changed?}\label{what-changed-101}

The \texttt{NavItem} interface has changed and the method \texttt{icon}
that would render the nav item icon has been removed.

\paragraph{Who is affected?}\label{who-is-affected-101}

This affects all themes using the \texttt{nav\_item.icon()} method.

\paragraph{How should I update my
code?}\label{how-should-i-update-my-code-101}

You should update your code to call the method
\texttt{nav\_item.iconURL} to return the image's URL and then use it as
you prefer.

\textbf{Example:}

\begin{verbatim}
<img alt="Page Icon" class="layout-logo" src="<%= nav_item.iconURL()" />
\end{verbatim}

To keep the previous behavior in Velocity:

\begin{verbatim}
$theme.layoutIcon($nav_item.getLayout())
\end{verbatim}

To keep the previous behavior in FreeMarker:

\begin{verbatim}
<@liferay_theme["layout-icon"] layout=nav_item_layout />
\end{verbatim}

\paragraph{Why was this change
made?}\label{why-was-this-change-made-101}

The API was forcing developers to have a dependency on a taglib, which
didn't allow for much flexibility.

\subsubsection{Renamed Packages to Fix the Split Packages
Problem}\label{renamed-packages-to-fix-the-split-packages-problem}

\begin{itemize}
\tightlist
\item
  \textbf{Date:} 2016-Jan-19
\item
  \textbf{JIRA Ticket:} LPS-61952
\end{itemize}

\paragraph{What changed?}\label{what-changed-102}

Split packages are caused when two or more bundles export the same
package name and version. When the classloader loads a package, exactly
one exporter of that package is chosen; so if a package is split across
multiple bundles, then an importer only sees a subset of the package.

\paragraph{Who is affected?}\label{who-is-affected-102}

The \texttt{portal-kernel} and \texttt{portal-impl} folders have many
packages with the same name. Therefore, all of these packages are
affected by the split package problem.

\paragraph{How should I update my
code?}\label{how-should-i-update-my-code-102}

You should rename duplicated package names if they currently exist
somewhere else.

\textbf{Example}

\begin{itemize}
\item
  \texttt{com.liferay.counter} → \texttt{com.liferay.counter.kernel}
\item
  \texttt{com.liferay.mail.model} →
  \texttt{com.liferay.mail.kernel.model}
\item
  \texttt{com.liferay.mail.service} →
  \texttt{com.liferay.mail.kernel.service}
\item
  \texttt{com.liferay.mail.util} → \texttt{com.liferay.mail.kernel.util}
\item
  \texttt{com.liferay.portal.exception} →
  \texttt{com.liferay.portal.kernel.exception}
\item
  \texttt{com.liferay.portal.jdbc.pool.metrics} →
  \texttt{com.liferay.portal.kernel.jdbc.pool.metrics}
\item
  \texttt{com.liferay.portal.kernel.mail} →
  \texttt{com.liferay.mail.kernel.model}
\item
  \texttt{com.liferay.portal.layoutconfiguration.util} →
  \texttt{com.liferay.portal.kernel.layoutconfiguration.util}
\item
  \texttt{com.liferay.portal.layoutconfiguration.util.xml} →
  \texttt{com.liferay.portal.kernel.layoutconfiguration.util.xml}
\item
  \texttt{com.liferay.portal.mail} →
  \texttt{com.liferay.portal.kernel.mail}
\item
  \texttt{com.liferay.portal.model} →
  \texttt{com.liferay.portal.kernel.model}
\item
  \texttt{com.liferay.portal.model.adapter} →
  \texttt{com.liferay.portal.kernel.model.adapter}
\item
  \texttt{com.liferay.portal.model.impl} →
  \texttt{com.liferay.portal.kernel.model.impl}
\item
  \texttt{com.liferay.portal.portletfilerepository} →
  \texttt{com.liferay.portal.kernel.portletfilerepository}
\item
  \texttt{com.liferay.portal.repository.proxy} →
  \texttt{com.liferay.portal.kernel.repository.proxy}
\item
  \texttt{com.liferay.portal.security.auth} →
  \texttt{com.liferay.portal.kernel.security.auth}
\item
  \texttt{com.liferay.portal.security.exportimport} →
  \texttt{com.liferay.portal.kernel.security.exportimport}
\item
  \texttt{com.liferay.portal.security.ldap} →
  \texttt{com.liferay.portal.kernel.security.ldap}
\item
  \texttt{com.liferay.portal.security.membershippolicy} →
  \texttt{com.liferay.portal.kernel.security.membershippolicy}
\item
  \texttt{com.liferay.portal.security.permission} →
  \texttt{com.liferay.portal.kernel.security.permission}
\item
  \texttt{com.liferay.portal.security.permission.comparator} →
  \texttt{com.liferay.portal.kernel.security.permission.comparator}
\item
  \texttt{com.liferay.portal.security.pwd} →
  \texttt{com.liferay.portal.kernel.security.pwd}
\item
  \texttt{com.liferay.portal.security.xml} →
  \texttt{com.liferay.portal.kernel.security.xml}
\item
  \texttt{com.liferay.portal.service.configuration} →
  \texttt{com.liferay.portal.kernel.service.configuration}
\item
  \texttt{com.liferay.portal.service.http} →
  \texttt{com.liferay.portal.kernel.service.http}
\item
  \texttt{com.liferay.portal.service.permission} →
  \texttt{com.liferay.portal.kernel.service.permission}
\item
  \texttt{com.liferay.portal.service.persistence.impl} →
  \texttt{com.liferay.portal.kernel.service.persistence.impl}
\item
  \texttt{com.liferay.portal.theme} →
  \texttt{com.liferay.portal.kernel.theme}
\item
  \texttt{com.liferay.portal.util} →
  \texttt{com.liferay.portal.kernel.util}
\item
  \texttt{com.liferay.portal.util.comparator} →
  \texttt{com.liferay.portal.kernel.util.comparator}
\item
  \texttt{com.liferay.portal.verify.model} →
  \texttt{com.liferay.portal.kernel.verify.model}
\item
  \texttt{com.liferay.portal.webserver} →
  \texttt{com.liferay.portal.kernel.webserver}
\item
  \texttt{com.liferay.portlet} →
  \texttt{com.liferay.portal.kernel.portlet}
\item
  \texttt{com.liferay.portlet.admin.util} →
  \texttt{com.liferay.admin.kernel.util}
\item
  \texttt{com.liferay.portlet.announcements} →
  \texttt{com.liferay.announcements.kernel}
\item
  \texttt{com.liferay.portlet.asset} → \texttt{com.liferay.asset.kernel}
\item
  \texttt{com.liferay.portlet.backgroundtask.util.comparator} →
  \texttt{com.liferay.background.task.kernel.util.comparator}
\item
  \texttt{com.liferay.portlet.blogs} → \texttt{com.liferay.blogs.kernel}
\item
  \texttt{com.liferay.portlet.blogs.exception} →
  \texttt{com.liferay.blogs.kernel.exception}
\item
  \texttt{com.liferay.portlet.blogs.model} →
  \texttt{com.liferay.blogs.kernel.model}
\item
  \texttt{com.liferay.portlet.blogs.service} →
  \texttt{com.liferay.blogs.kernel.service}
\item
  \texttt{com.liferay.portlet.blogs.service.persistence} →
  \texttt{com.liferay.blogs.service.persistence}
\item
  \texttt{com.liferay.portlet.blogs.util.comparator} →
  \texttt{com.liferay.blogs.kernel.util.comparator}
\item
  \texttt{com.liferay.portlet.documentlibrary} →
  \texttt{com.liferay.document.library.kernel}
\item
  \texttt{com.liferay.portlet.dynamicdatamapping} →
  \texttt{com.liferay.dynamic.data.mapping.kernel}
\item
  \texttt{com.liferay.portlet.expando} →
  \texttt{com.liferay.expando.kernel}
\item
  \texttt{com.liferay.portlet.exportimport} →
  \texttt{com.liferay.exportimport.kernel}
\item
  \texttt{com.liferay.portlet.imagegallerydisplay.display.context} →
  \texttt{com.liferay.image.gallery.display.kernel.display.context}
\item
  \texttt{com.liferay.portlet.journal.util} →
  \texttt{com.liferay.journal.kernel.util}
\item
  \texttt{com.liferay.portlet.layoutsadmin.util} →
  \texttt{com.liferay.layouts.admin.kernel.util}
\item
  \texttt{com.liferay.portlet.messageboards} →
  \texttt{com.liferay.message.boards.kernel}
\item
  \texttt{com.liferay.portlet.messageboards.constants} →
  \texttt{com.liferay.message.boards.kernel.constants}
\item
  \texttt{com.liferay.portlet.messageboards.exception} →
  \texttt{com.liferay.message.boards.kernel.exception}
\item
  \texttt{com.liferay.portlet.messageboards.model} →
  \texttt{com.liferay.message.boards.kernel.model}
\item
  \texttt{com.liferay.portlet.messageboards.service} →
  \texttt{com.liferay.message.boards.kernel.service}
\item
  \texttt{com.liferay.portlet.messageboards.service.persistence} →
  \texttt{com.liferay.message.boards.kernel.service.persistence}
\item
  \texttt{com.liferay.portlet.messageboards.util} →
  \texttt{com.liferay.message.boards.kernel.util}
\item
  \texttt{com.liferay.portlet.messageboards.util.comparator} →
  \texttt{com.liferay.message.boards.kernel.util.comparator}
\item
  \texttt{com.liferay.portlet.mobiledevicerules} →
  \texttt{com.liferay.mobile.device.rules}
\item
  \texttt{com.liferay.portlet.portletconfiguration.util} →
  \texttt{com.liferay.portlet.configuration.kernel.util}
\item
  \texttt{com.liferay.portlet.rolesadmin.util} →
  \texttt{com.liferay.roles.admin.kernel.util}
\item
  \texttt{com.liferay.portlet.sites.util} →
  \texttt{com.liferay.sites.kernel.util}
\item
  \texttt{com.liferay.portlet.social} →
  \texttt{com.liferay.social.kernel}
\item
  \texttt{com.liferay.portlet.trash} → \texttt{com.liferay.trash.kernel}
\item
  \texttt{com.liferay.portlet.useradmin.util} →
  \texttt{com.liferay.users.admin.kernel.util}
\item
  \texttt{com.liferay.portlet.ratings} →
  \texttt{com.liferay.ratings.kernel}
\item
  \texttt{com.liferay.portlet.ratings.definition} →
  \texttt{com.liferay.ratings.kernel.definition}
\item
  \texttt{com.liferay.portlet.ratings.display.context} →
  \texttt{com.liferay.ratings.kernel.display.context}
\item
  \texttt{com.liferay.portlet.ratings.exception} →
  \texttt{com.liferay.ratings.kernel.exception}
\item
  \texttt{com.liferay.portlet.ratings.model} →
  \texttt{com.liferay.ratings.kernel.model}
\item
  \texttt{com.liferay.portlet.ratings.service} →
  \texttt{com.liferay.ratings.kernel.service}
\item
  \texttt{com.liferay.portlet.ratings.service.persistence} →
  \texttt{com.liferay.ratings.kernel.service.persistence}
\item
  \texttt{com.liferay.portlet.ratings.transformer} →
  \texttt{com.liferay.ratings.kernel.transformer}
\end{itemize}

\paragraph{Why was this change
made?}\label{why-was-this-change-made-102}

This change was necessary to solve the current split package problems
and prevent future ones.

\subsubsection{Removed the aui:column Tag and Replaced with
aui:col}\label{removed-the-auicolumn-tag-and-replaced-with-auicol}

\begin{itemize}
\tightlist
\item
  \textbf{Date:} 2016-Jan-19
\item
  \textbf{JIRA Ticket:} LPS-62208
\end{itemize}

\paragraph{What changed?}\label{what-changed-103}

The \texttt{aui:column} tag has been removed and replaced with the
\texttt{aui:col} tag.

\paragraph{Who is affected?}\label{who-is-affected-103}

Plugins or templates that are using the \texttt{aui:column} tag must
update their usage of the tag.

\paragraph{How should I update my
code?}\label{how-should-i-update-my-code-103}

You should import the \texttt{aui} tag library (if necessary) and update
the tag namespace from \texttt{aui:column} to \texttt{aui:col}.

\paragraph{Why was this change
made?}\label{why-was-this-change-made-103}

This change was made as a part of the ongoing strategy to modularize
Liferay Portal by means of an OSGi container.

\subsubsection{The title Field of FileEntry Models is Now
Mandatory}\label{the-title-field-of-fileentry-models-is-now-mandatory}

\begin{itemize}
\tightlist
\item
  \textbf{Date:} 2016-Jan-25
\item
  \textbf{JIRA Ticket:} LPS-62251
\end{itemize}

\paragraph{What changed?}\label{what-changed-104}

The \texttt{title} field of file entries was optional as long as a
source file name was provided. To avoid confusion, the title is now
required by the API and is filled automatically by the UI when a source
file name is present.

\paragraph{Who is affected?}\label{who-is-affected-104}

This affects any user of the local or remote API. Users of the Web UI
are unaffected.

\paragraph{How should I update my
code?}\label{how-should-i-update-my-code-104}

You should pass a non-null, non-empty string for the \texttt{title}
parameter of the \texttt{addFileEntry} and \texttt{updateFileEntry}
methods.

\paragraph{Why was this change
made?}\label{why-was-this-change-made-104}

The \texttt{title} field was marked as mandatory, but it was possible to
create a document without filling it, as the backend would infer a value
from the source file name automatically. This was considered confusing
from a UX perspective.

\subsubsection{DLUtil.getImagePreviewURL and DLUtil.getThumbnailSrc Can
Return Empty
Strings}\label{dlutil.getimagepreviewurl-and-dlutil.getthumbnailsrc-can-return-empty-strings}

\begin{itemize}
\tightlist
\item
  \textbf{Date:} 2016-Jan-28
\item
  \textbf{JIRA Ticket:} LPS-62643
\end{itemize}

\paragraph{What changed?}\label{what-changed-105}

The \texttt{DLUtil.getImagePreviewURL} and
\texttt{DLUtil.getThumbnailSrc} methods return an empty string if there
are no previews or thumbnails for the specific image, video, or
document.

Previously, if there were no previews or thumbnails, these methods would
return a URL to an image based on the document.

\paragraph{Who is affected?}\label{who-is-affected-105}

This affects any developer invoking \texttt{DLUtil.getImagePreviewURL}
or \texttt{DLUtil.getThumbnailSrc}.

\paragraph{How should I update my
code?}\label{how-should-i-update-my-code-105}

You should be aware that the method could return an empty string and act
accordingly. For example, you could display the
\texttt{documents-and-media} Lexicon icon instead.

\paragraph{Why was this change
made?}\label{why-was-this-change-made-105}

In order to display the \texttt{documents-and-media} Lexicon icon in
Documents and Media, this change was necessary.

\subsubsection{Removed the aui:button-item Tag and Replaced with
aui:button}\label{removed-the-auibutton-item-tag-and-replaced-with-auibutton}

\begin{itemize}
\tightlist
\item
  \textbf{Date:} 2016-Feb-04
\item
  \textbf{JIRA Ticket:} LPS-62922
\end{itemize}

\paragraph{What changed?}\label{what-changed-106}

The \texttt{aui:button-item} tag has been removed and replaced with the
\texttt{aui:button} tag.

\paragraph{Who is affected?}\label{who-is-affected-106}

Plugins or templates that are using the \texttt{aui:button-item} tag
must update their usage of the tag.

\paragraph{How should I update my
code?}\label{how-should-i-update-my-code-106}

You should import the \texttt{aui} tag library (if necessary) and update
the tag namespace from \texttt{aui:button-item} to \texttt{aui:button}.

\paragraph{Why was this change
made?}\label{why-was-this-change-made-106}

This change was made as a part of the ongoing strategy to remove
deprecated code.

\subsubsection{Removed the WAP
Functionality}\label{removed-the-wap-functionality}

\begin{itemize}
\tightlist
\item
  \textbf{Date:} 2016-Feb-05
\item
  \textbf{JIRA Ticket:} LPS-62920
\end{itemize}

\paragraph{What changed?}\label{what-changed-107}

The WAP functionality has been removed.

\paragraph{Who is affected?}\label{who-is-affected-107}

This affects developers that use the WAP functionality.

\paragraph{How should I update my
code?}\label{how-should-i-update-my-code-107}

If you are using any of the following methods, you need to remove the
parameters in those methods related to WAP.

\begin{itemize}
\tightlist
\item
  \texttt{LayoutLocalServiceUtil.updateLookAndFeel}
\item
  \texttt{LayoutRevisionLocalServiceUtil.addLayoutRevision}
\item
  \texttt{LayoutRevisionLocalServiceUtil.updateLayoutRevision}
\item
  \texttt{LayoutRevisionServiceUtil.addLayoutRevision}
\item
  \texttt{LayoutServiceUtil.updateLookAndFeel}
\item
  \texttt{LayoutSetLocalServiceUtil.updateLookAndFeel}
\item
  \texttt{LayoutSetServiceUtil.updateLookAndFeel}
\item
  \texttt{ThemeLocalServiceUtil.getColorScheme}
\item
  \texttt{ThemeLocalServiceUtil.getControlPanelThemes}
\item
  \texttt{ThemeLocalServiceUtil.getPageThemes}
\item
  \texttt{ThemeLocalServiceUtil.getTheme}
\end{itemize}

\paragraph{Why was this change
made?}\label{why-was-this-change-made-107}

This change was made because WAP is an obsolete functionality.

\subsubsection{Removed the aui:layout Tag with No Direct
Replacement}\label{removed-the-auilayout-tag-with-no-direct-replacement}

\begin{itemize}
\tightlist
\item
  \textbf{Date:} 2016-Feb-08
\item
  \textbf{JIRA Ticket:} LPS-62935
\end{itemize}

\paragraph{What changed?}\label{what-changed-108}

The \texttt{aui:layout} tag has been removed with no direct replacement.

\paragraph{Who is affected?}\label{who-is-affected-108}

Plugins or templates that are using the \texttt{aui:layout} tag must
remove their usage of the tag.

\paragraph{How should I update my
code?}\label{how-should-i-update-my-code-108}

There is no direct replacement. You should remove all usages of the
\texttt{aui:layout} tag.

\paragraph{Why was this change
made?}\label{why-was-this-change-made-108}

This change was made as a part of the ongoing strategy to remove
deprecated tags.

\subsubsection{Deprecated the liferay-portlet:icon-back Tag with No
Direct
Replacement}\label{deprecated-the-liferay-portleticon-back-tag-with-no-direct-replacement}

\begin{itemize}
\tightlist
\item
  \textbf{Date:} 2016-Feb-10
\item
  \textbf{JIRA Ticket:} LPS-63101
\end{itemize}

\paragraph{What changed?}\label{what-changed-109}

The \texttt{liferay-portlet:icon-back} tag has been deprecated with no
direct replacement.

\paragraph{Who is affected?}\label{who-is-affected-109}

Plugins or templates that are using the
\texttt{liferay-portlet:icon-back} tag must remove their usage of the
tag.

\paragraph{How should I update my
code?}\label{how-should-i-update-my-code-109}

There is no direct replacement. You should remove all usages of the
\texttt{liferay-portlet:icon-back} tag.

\paragraph{Why was this change
made?}\label{why-was-this-change-made-109}

This change was made as a part of the ongoing strategy to deprecate
unused tags.

\subsubsection{Deprecated the liferay-security:encrypt Tag with No
Direct
Replacement}\label{deprecated-the-liferay-securityencrypt-tag-with-no-direct-replacement}

\begin{itemize}
\tightlist
\item
  \textbf{Date:} 2016-Feb-10
\item
  \textbf{JIRA Ticket:} LPS-63106
\end{itemize}

\paragraph{What changed?}\label{what-changed-110}

The \texttt{liferay-security:encrypt} tag has been deprecated with no
direct replacement.

\paragraph{Who is affected?}\label{who-is-affected-110}

Plugins or templates that are using the
\texttt{liferay-security:encrypt} tag must remove their usage of the
tag.

\paragraph{How should I update my
code?}\label{how-should-i-update-my-code-110}

There is no direct replacement. You should remove all usages of the
\texttt{liferay-security:encrypt} tag.

\paragraph{Why was this change
made?}\label{why-was-this-change-made-110}

This change was made as a part of the ongoing strategy to deprecate
unused tags.

\subsubsection{Removed the Ability to Specify Class Loaders in
Scripting}\label{removed-the-ability-to-specify-class-loaders-in-scripting}

\begin{itemize}
\tightlist
\item
  \textbf{Date:} 2016-Feb-17
\item
  \textbf{JIRA Ticket:} LPS-63180
\end{itemize}

\paragraph{What changed?}\label{what-changed-111}

\begin{itemize}
\tightlist
\item
  \texttt{com.liferay.portal.kernel.scripting.ScriptingExecutor} no
  longer uses the provided class loaders in the eval methods.
\item
  \texttt{com.liferay.portal.kernel.scripting.Scripting} no longer uses
  the provided class loaders and servlet context names in eval and exec
  methods.
\end{itemize}

\paragraph{Who is affected?}\label{who-is-affected-111}

\begin{itemize}
\tightlist
\item
  All implementations of
  \texttt{com.liferay.portal.kernel.scripting.ScriptingExecutor} are
  affected.
\item
  All classes that call
  \texttt{com.liferay.portal.kernel.scripting.Scripting} are affected.
\end{itemize}

\paragraph{How should I update my
code?}\label{how-should-i-update-my-code-111}

You should remove class loader and servlect context parameters from
calls to the modified methods.

\paragraph{Why was this change
made?}\label{why-was-this-change-made-111}

This change was made since custom class loader management is no longer
necessary in the OSGi container.

\subsubsection{User Operation and Importer/Exporter Classes and
Utilities Have Been Moved or Removed From
portal-kernel}\label{user-operation-and-importerexporter-classes-and-utilities-have-been-moved-or-removed-from-portal-kernel}

\begin{itemize}
\tightlist
\item
  \textbf{Date:} 2016-Feb-17
\item
  \textbf{JIRA Ticket:} LPS-63205
\end{itemize}

\paragraph{What changed?}\label{what-changed-112}

\begin{itemize}
\item
  \texttt{com.liferay.portal.kernel.security.exportimport.UserImporter},
  \texttt{com.liferay.portal.kernel.security.exportimport.UserExporter},
  and
  \texttt{com.liferay.portal.kernel.security.exportimport.UserOperation}
  have been moved from portal-kernel to the
  portal-security-export-import-api module.
\item
  \texttt{com.liferay.portal.kernel.security.exportimport.UserImporterUtil}
  and
  \texttt{com.liferay.portal.kernel.security.exportimport.UserExporterUtil}
  have been removed with no replacement.
\end{itemize}

\paragraph{Who is affected?}\label{who-is-affected-112}

\begin{itemize}
\item
  All implementations of
  \texttt{com.liferay.portal.kernel.security.exportimport.UserImporter}
  or
  \texttt{com.liferay.portal.kernel.security.exportimport.UserExporter}
  are affected.
\item
  All code that uses
  \texttt{com.liferay.portal.kernel.security.exportimport.UserImporterUtil},
  \texttt{com.liferay.portal.kernel.security.exportimport.UserExporterUtil},
  \texttt{com.liferay.portal.kernel.security.exportimport.UserImporter},
  or
  \texttt{com.liferay.portal.kernel.security.exportimport.UserExporter}
  is affected.
\end{itemize}

\paragraph{How should I update my
code?}\label{how-should-i-update-my-code-112}

If you are in an OSGi module, you can simply inject the UserImporter or
UserExporter references

\begin{verbatim}
@Reference
private UserExporter_userExporter;

@Reference
private UserImporter _userImporter;
\end{verbatim}

If you are in a legacy WAR or WAB, you will need a snippet like:

\begin{verbatim}
Bundle bundle = FrameworkUtil.getBundle(getClass());

BundleContext bundleContext = bundle.getBundleContext();

ServiceReference<UserImporter> serviceReference =
    bundleContext.getServiceReference(UserImporter.class);

UserImporter userImporter = bundleContext.getService(serviceReference);
\end{verbatim}

\paragraph{Why was this change
made?}\label{why-was-this-change-made-112}

The change was made to improve modularity of the user import/export
subsystem in the product.

\subsubsection{Deprecated Category Entry for
Users}\label{deprecated-category-entry-for-users}

\begin{itemize}
\tightlist
\item
  \textbf{Date:} 2016-Feb-22
\item
  \textbf{JIRA Ticket:} LPS-63466
\end{itemize}

\paragraph{What changed?}\label{what-changed-113}

The category entry for Site Administration → Users has been deprecated
in favor of Site Administration → Members.

\paragraph{Who is affected?}\label{who-is-affected-113}

All developers who specified a \texttt{control-panel-entry-category} to
be visible in Site Administration → Users are affected.

\paragraph{How should I update my
code?}\label{how-should-i-update-my-code-113}

You should change the entry from \texttt{site\_administration.users} to
\texttt{site\_administration.members} to make it visible in the
category.

\paragraph{Why was this change
made?}\label{why-was-this-change-made-113}

This change standardizes naming conventions and separates concepts
between Users in the Control Panel and Site Members.

\subsubsection{Deprecated Category Entry for
Pages}\label{deprecated-category-entry-for-pages}

\begin{itemize}
\tightlist
\item
  \textbf{Date:} 2016-Feb-25
\item
  \textbf{JIRA Ticket:} LPS-63667
\end{itemize}

\paragraph{What changed?}\label{what-changed-114}

The category entry for Site Administration → Pages has been deprecated
in favor of Site Administration → Navigation.

\paragraph{Who is affected?}\label{who-is-affected-114}

All developers who specified a \texttt{control-panel-entry-category} to
be visible in Site Administration → Pages are affected.

\paragraph{How should I update my
code?}\label{how-should-i-update-my-code-114}

You should change the entry from \texttt{site\_administration.pages} to
\texttt{site\_administration.navigation} to make it visible in the
category.

\paragraph{Why was this change
made?}\label{why-was-this-change-made-114}

This change standardizes naming conventions and separates concepts in
Product Menu

\subsubsection{Removed the
com.liferay.dynamic.data.mapping.util.DDMXMLUtil
Class}\label{removed-the-com.liferay.dynamic.data.mapping.util.ddmxmlutil-class}

\begin{itemize}
\tightlist
\item
  \textbf{Date:} 2016-Mar-03
\item
  \textbf{JIRA Ticket:} LPS-63928
\end{itemize}

\paragraph{What changed?}\label{what-changed-115}

The class \texttt{com.liferay.dynamic.data.mapping.util.DDMXMLUtil} has
been removed with no replacement.

\paragraph{Who is affected?}\label{who-is-affected-115}

All code that uses
\texttt{com.liferay.dynamic.data.mapping.util.DDMXMLUtil} is affected.

\paragraph{How should I update my
code?}\label{how-should-i-update-my-code-115}

In an OSGi module, simply inject the DDMXML reference:

\begin{verbatim}
@Reference
private DDMXML _ddmXML;
\end{verbatim}

In a legacy WAR or WAB, you need to get a DDMXML service reference from
the bundle context:

\begin{verbatim}
Bundle bundle = FrameworkUtil.getBundle(getClass());

BundleContext bundleContext = bundle.getBundleContext();

ServiceReference<UserImporter> serviceReference =
    bundleContext.getServiceReference(DDMXML.class);

DDMXML ddmXML = bundleContext.getService(serviceReference);
\end{verbatim}

\paragraph{Why was this change
made?}\label{why-was-this-change-made-115}

This change was made to improve modularity of the dynamic data mapping
subsystem.

\subsubsection{FlagsEntryService.addEntry Method Throws
PortalException}\label{flagsentryservice.addentry-method-throws-portalexception}

\begin{itemize}
\tightlist
\item
  \textbf{Date:} 2016-Mar-04
\item
  \textbf{JIRA Ticket:} LPS-63109
\end{itemize}

\paragraph{What changed?}\label{what-changed-116}

The method \texttt{FlagsEntryService.addEntry} now throws a
\texttt{PortalException} if the \texttt{reporterEmailAddress} is not a
valid email address.

\paragraph{Who is affected?}\label{who-is-affected-116}

Any caller of the method \texttt{FlagsEntryService.addEntry} is
affected.

\paragraph{How should I update my
code?}\label{how-should-i-update-my-code-116}

You should consider checking for the \texttt{PortalException} in
try-catch blocks and adapt your code accordingly.

\paragraph{Why was this change
made?}\label{why-was-this-change-made-116}

This change prevents providing an incorrect email address when adding
flag entries.

\subsubsection{Removed PHP Portlet
Support}\label{removed-php-portlet-support}

\begin{itemize}
\tightlist
\item
  \textbf{Date:} 2016-Mar-10
\item
  \textbf{JIRA Ticket:} LPS-64052
\end{itemize}

\paragraph{What changed?}\label{what-changed-117}

PHP portlets are no longer supported.

\paragraph{Who is affected?}\label{who-is-affected-117}

This affects any portlet using the class
\texttt{com.liferay.util.bridges.php.PHPPortlet}.

\paragraph{How should I update my
code?}\label{how-should-i-update-my-code-117}

You should port your PHP portlet to a different technology.

\paragraph{Why was this change
made?}\label{why-was-this-change-made-117}

This change simplifies future maintenance of the portal. This support
could be added back in the future as an independent module.

\subsubsection{Removed Liferay Frontend Editor BBCode Web, Previously
Known as Liferay BBCode
Editor}\label{removed-liferay-frontend-editor-bbcode-web-previously-known-as-liferay-bbcode-editor}

\begin{itemize}
\tightlist
\item
  \textbf{Date:} 2016-Mar-16
\item
  \textbf{JIRA Ticket:} LPS-48334
\end{itemize}

\paragraph{What changed?}\label{what-changed-118}

The following things have been changed:

\begin{itemize}
\tightlist
\item
  Removed the \texttt{com.liferay.frontend.editor.bbcode.web} OSGi
  bundle
\item
  Removed all hardcoded references/logic for the editor
\item
  Added a log warning and logic to upgrade the editor property to
  \texttt{ckeditor\_bbcode} if the old \texttt{bbcode} is being used.
  This log warning and logic will be removed in the future, along with
  \href{https://issues.liferay.com/browse/LPS-64099}{LPS-64099}.
\end{itemize}

\paragraph{Who is affected?}\label{who-is-affected-118}

This affects anyone who has the property
\texttt{editor.wysiwyg.portal-web.docroot.html.portlet.message\_boards.edit\_message.bb\_code.jsp}
set to \texttt{bbcode} in portal properties (e.g.,
\texttt{portal-ext.properties}).

\paragraph{How should I update my
code?}\label{how-should-i-update-my-code-118}

You should modify your \texttt{portal-ext.properties} file to remove the
property
\texttt{editor.wysiwyg.portal-web.docroot.html.portlet.message\_boards.edit\_message.bb\_code.jsp}.

\paragraph{Why was this change
made?}\label{why-was-this-change-made-118}

Since Liferay Frontend Editor BBCode Web has been deprecated since 6.1,
it was time to remove it completely. This frees up development and
support resources to focus on supported features.

\subsubsection{Removed the asset.entry.validator
Property}\label{removed-the-asset.entry.validator-property}

\begin{itemize}
\tightlist
\item
  \textbf{Date:} 2016-Mar-17
\item
  \textbf{JIRA Ticket:} LPS-64370
\end{itemize}

\paragraph{What changed?}\label{what-changed-119}

The property \texttt{asset.entry.validator} has been removed from
\texttt{portal.properties}.

\paragraph{Who is affected?}\label{who-is-affected-119}

This affects any installation with a customized asset validator.

\paragraph{How should I update my
code?}\label{how-should-i-update-my-code-119}

You should create a new OSGi component that implements
\texttt{AssetEntryValidator} and define for which models it will be
applicable by using the \texttt{model.class.name} OSGi property, or an
asterisk if it applies to any model.

If you were using the \texttt{MinimalAssetEntryValidator}, this
functionality can still be added by deploying the module
\texttt{asset-tags-validator}.

\paragraph{Why was this change
made?}\label{why-was-this-change-made-119}

This change has been made as part of the modularization efforts to
decouple different parts of the portal.

\subsubsection{Removed the swfupload and video\_player
Utilities}\label{removed-the-swfupload-and-video_player-utilities}

\begin{itemize}
\tightlist
\item
  \textbf{Date:} 2016-May-13
\item
  \textbf{JIRA Ticket:} LPS-54111
\end{itemize}

\paragraph{What changed?}\label{what-changed-120}

The utilities \texttt{swfupload} and \texttt{video\_player} have been
removed.

\paragraph{Who is affected?}\label{who-is-affected-120}

This affects anyone who is using the \texttt{swfupload} AlloyUI module
or any of the associated \texttt{swfupload\_f*.swf} and
\texttt{mpw\_player.swf} flash movies.

\paragraph{How should I update my
code?}\label{how-should-i-update-my-code-120}

There are better, more standard ways to achieve upload currently. For
instance, you can use
\href{http://alloyui.com/api/classes/Uploader.html}{A.Uploader} to
manage your uploads consistently across browsers.

For audio/video reproduction, you should update your code to use
\href{http://alloyui.com/api/classes/A.Audio.html}{A.Audio} and
\href{http://alloyui.com/api/classes/A.Video.html}{A.Video}.

\paragraph{Why was this change
made?}\label{why-was-this-change-made-120}

This change removes outdated code no longer being used in the platform.
In addition, this change avoids future security issues from outdated
flash movies.

\subsubsection{Moved Journal Portlet Properties to OSGi
Configuration}\label{moved-journal-portlet-properties-to-osgi-configuration}

\begin{itemize}
\tightlist
\item
  \textbf{Date:} 2016-Jul-29
\item
  \textbf{JIRA Ticket:} LPS-58672
\end{itemize}

\paragraph{What changed?}\label{what-changed-121}

All Journal portlet properties have been moved to an OSGi configuration.

\paragraph{Who is affected?}\label{who-is-affected-121}

This affects anyone who is overriding the Journal portlet's
\texttt{portlet.properties} file.

\paragraph{How should I update my
code?}\label{how-should-i-update-my-code-121}

Instead of overriding the Journal portlet's \texttt{portlet.properties}
file, you can manage the properties from Portal's configuration
administrator. This can be accessed by navigating to Liferay's Control
Panel → \emph{System Settings} → \emph{Web Experience} and selecting the
appropriate Web Content category.

\paragraph{Why was this change
made?}\label{why-was-this-change-made-121}

This change was made as part of modularization efforts to ease portlet
configuration changes.

\subsubsection{Moved the liferay-ui:journal-article Tag to
Journal}\label{moved-the-liferay-uijournal-article-tag-to-journal}

\begin{itemize}
\tightlist
\item
  \textbf{Date:} 2016-Nov-24
\item
  \textbf{JIRA Ticket:} LPS-69321
\end{itemize}

\paragraph{What changed?}\label{what-changed-122}

The \texttt{liferay-ui:journal-article} tag has been moved to the
Journal (Web Content) application.

\paragraph{Who is affected?}\label{who-is-affected-122}

This affects developers using the \texttt{liferay-ui:journal-article}
tag.

\paragraph{How should I update my
code?}\label{how-should-i-update-my-code-122}

You should use the \texttt{liferay-journal:journal-article} tag instead.

\textbf{Example}

Old code:

\begin{verbatim}
<liferay-ui:journal-article
    articleId="<%= article.getArticleId() %>"
/>
\end{verbatim}

New code:

\begin{verbatim}
<liferay-journal:journal-article
    articleId="<%= article.getArticleId() %>"
    groupId="<%= article.getGroupId() %>"
/>
\end{verbatim}

If you still want to use the \texttt{liferay-ui:journal-article} tag,
you must deploy the \texttt{journal-taglib} module to your Liferay
installation.

\paragraph{Why was this change
made?}\label{why-was-this-change-made-122}

This change was made as part of the modularization efforts for the Web
Content application.

\subsubsection{Deprecated the liferay-ui:captcha Tag and Replaced with
liferay-captcha:captcha}\label{deprecated-the-liferay-uicaptcha-tag-and-replaced-with-liferay-captchacaptcha}

\begin{itemize}
\tightlist
\item
  \textbf{Date:} 2016-Nov-29
\item
  \textbf{JIRA Ticket:} LPS-69383
\end{itemize}

\paragraph{What changed?}\label{what-changed-123}

The \texttt{liferay-ui:captcha} tag has been deprecated and replaced
with the \texttt{liferay-captcha:captcha} tag.

\paragraph{Who is affected?}\label{who-is-affected-123}

Plugins or templates that are using the \texttt{liferay-ui:captcha} tag
need to update their usage of the tag.

\paragraph{How should I update my
code?}\label{how-should-i-update-my-code-123}

You should import the \texttt{liferay-captcha} tag library (if
necessary) and update the tag namespace from \texttt{liferay-ui:captcha}
to \texttt{liferay-captcha:captcha}.

\paragraph{Why was this change
made?}\label{why-was-this-change-made-123}

This change was made as a part of the ongoing strategy to modularize
Liferay Portal by means of an OSGi container.

\subsubsection{Moved Shopping File Uploads Portlet Properties to OSGi
Configuration}\label{moved-shopping-file-uploads-portlet-properties-to-osgi-configuration}

\begin{itemize}
\tightlist
\item
  \textbf{Date:} 2016-Dec-08
\item
  \textbf{JIRA Ticket:} LPS-69210
\end{itemize}

\paragraph{What changed?}\label{what-changed-124}

The Shopping file uploads portlet properties have been moved from Server
Administration to an OSGi configuration named
\texttt{ShoppingFileUploadsConfiguration.java} in the
\texttt{shopping-api} module.

\paragraph{Who is affected?}\label{who-is-affected-124}

This affects anyone who is using the following portlet properties:

\begin{itemize}
\tightlist
\item
  \texttt{shopping.image.extensions}
\item
  \texttt{shopping.image.large.max.size}
\item
  \texttt{shopping.image.medium.max.size}
\item
  \texttt{shopping.image.small.max.size}
\end{itemize}

\paragraph{How should I update my
code?}\label{how-should-i-update-my-code-124}

Instead of overriding the \texttt{portal.properties} file, you can
manage the properties from Portal's configuration administrator. This
can be accessed by navigating to Liferay's \emph{Control Panel} →
\emph{Configuration} → \emph{System Settings} → \emph{Shopping Cart
Images} and editing the settings there.

If you would like to include the new configuration in your application,
follow the instructions for
\href{/docs/7-0/tutorials/-/knowledge_base/t/making-your-applications-configurable}{making
your applications configurable in Liferay 7.0}.

\paragraph{Why was this change
made?}\label{why-was-this-change-made-124}

This change was made as part of the modularization efforts to ease
portal configuration changes.

\subsubsection{Moved the Expando Custom Field Tags to liferay-expando
Taglib}\label{moved-the-expando-custom-field-tags-to-liferay-expando-taglib}

\begin{itemize}
\tightlist
\item
  \textbf{Date:} 2016-Dec-12
\item
  \textbf{JIRA Ticket:} LPS-69400
\end{itemize}

\paragraph{What changed?}\label{what-changed-125}

The following tags have been deprecated and replaced:

\begin{itemize}
\tightlist
\item
  \texttt{liferay-ui:custom-attribute}
\item
  \texttt{liferay-ui:custom-attribute-list}
\item
  \texttt{liferay-ui:custom-attributes-available}
\end{itemize}

\paragraph{Who is affected?}\label{who-is-affected-125}

Plugins and templates that are using the aforementioned tags must update
their usage of the tag.

\paragraph{How should I update my
code?}\label{how-should-i-update-my-code-125}

You should import the \texttt{liferay-expando} tag library (if
necessary) and update the tag namespace from \texttt{liferay-ui} to
\texttt{liferay-expando}:

\begin{itemize}
\tightlist
\item
  \texttt{liferay-ui:custom-attribute} →
  \texttt{liferay-expando:custom-attribute}
\item
  \texttt{liferay-ui:custom-attribute-list} →
  \texttt{liferay-expando:custom-attribute-list}
\item
  \texttt{liferay-ui:custom-attributes-available} →
  \texttt{liferay-expando:custom-attributes-available}
\end{itemize}

\paragraph{Why was this change
made?}\label{why-was-this-change-made-125}

This change was made as part of the ongoing strategy to modularize
Liferay Portal by means of an OSGi container.

\subsubsection{Moved Journal File Uploads Portlet Properties to OSGi
Configuration}\label{moved-journal-file-uploads-portlet-properties-to-osgi-configuration}

\begin{itemize}
\tightlist
\item
  \textbf{Date:} 2017-Jan-04
\item
  \textbf{JIRA Ticket:} LPS-69209
\end{itemize}

\paragraph{What changed?}\label{what-changed-126}

The Journal File Uploads portlet properties have been moved from Server
Administration to an OSGi configuration named
\texttt{JournalFileUploadsConfiguration.java} in the
\texttt{journal-service} module.

\paragraph{Who is affected?}\label{who-is-affected-126}

This affects anyone who is using the following portlet properties:

\begin{itemize}
\tightlist
\item
  \texttt{journal.image.extensions}
\item
  \texttt{journal.image.small.max.size}
\end{itemize}

\paragraph{How should I update my
code?}\label{how-should-i-update-my-code-126}

Instead of overriding the \texttt{portal.properties} file, you can
manage the properties from Portal's configuration administrator. This
can be accessed by navigating to Liferay's \emph{Control Panel} →
\emph{Configuration} → \emph{System Settings} → \emph{Web Content File
Uploads} and editing the settings there.

If you would like to include the new configuration in your application,
follow the instructions for
\href{/docs/7-0/tutorials/-/knowledge_base/t/making-your-applications-configurable}{making
your applications configurable in Liferay 7.0}.

\paragraph{Why was this change
made?}\label{why-was-this-change-made-126}

This change was made as part of the modularization efforts to ease
portal configuration changes.

\subsubsection{Deprecated the aui:tool Tag with No Direct
Replacement}\label{deprecated-the-auitool-tag-with-no-direct-replacement}

\begin{itemize}
\tightlist
\item
  \textbf{Date:} 2017-Feb-02
\item
  \textbf{JIRA Ticket:} LPS-70422
\end{itemize}

\paragraph{What changed?}\label{what-changed-127}

The \texttt{aui:tool} tag has been deprecated with no direct
replacement.

\paragraph{Who is affected?}\label{who-is-affected-127}

Plugins or templates that are using the \texttt{aui:tool} tag must
remove their usage of the tag.

\paragraph{How should I update my
code?}\label{how-should-i-update-my-code-127}

There is no direct replacement. You should remove all usages of the
\texttt{aui:tool} tag.

\paragraph{Why was this change
made?}\label{why-was-this-change-made-127}

This change was made as a part of the ongoing strategy to deprecate
unused tags.

\subsubsection{Build Auto Upgrade}\label{build-auto-upgrade}

\begin{itemize}
\tightlist
\item
  \textbf{Date:} 2017-Aug-17
\item
  \textbf{JIRA Ticket:} LPS-73967
\end{itemize}

\paragraph{What changed?}\label{what-changed-128}

The \texttt{build.auto.upgrade} property in \texttt{service.properties}
for Liferay Portal 6.x Service Builder portlets applies Liferay Service
schema changes on rebuilding the services and redeploying the portlets.

Since 7.0, the per portlet property \texttt{build.auto.upgrade} is
deprecated.

This change reintroduces Build Auto Upgrade in a new global property
\texttt{schema.module.build.auto.upgrade} in the
\texttt{{[}Liferay\_Home{]}/portal-developer.properties} file.

Setting global property \texttt{schema.module.build.auto.upgrade} to
\texttt{true} applies module schema changes for redeployed modules whose
service build numbers have incremented. The \texttt{build.number}
property in the module's \texttt{service.properties} file indicates the
service build number.

\paragraph{Who is affected?}\label{who-is-affected-128}

This feature is available for developers to use in development only.

\textbf{WARNING}: DO NOT USE the Build Auto Upgrade feature in
production. Liferay DOES NOT support Build Auto Upgrade in production.

\paragraph{How should I update my
code?}\label{how-should-i-update-my-code-128}

To use this feature in development, set global property
\texttt{schema.module.build.auto.upgrade} in
\texttt{{[}Liferay\_Home{]}/portal-developer.properties} to
\texttt{true}, increment your module's \texttt{build.number} in the
\texttt{service.properties} file, and deploy the module.

\paragraph{Why was this change
made?}\label{why-was-this-change-made-128}

This change was made so that 7.0 developers could test database schema
changes on the fly, without having to write upgrade processes.

\subsubsection{Removed Exports from Dynamic Data Lists
Web}\label{removed-exports-from-dynamic-data-lists-web}

\begin{itemize}
\tightlist
\item
  \textbf{Date:} 2017-Nov-27
\item
  \textbf{JIRA Ticket:} LPS-75778
\end{itemize}

\paragraph{What changed?}\label{what-changed-129}

The \texttt{Dynamic\ Data\ Lists\ Web} module no longer exports the
\texttt{com.liferay.dynamic.data.lists.web.asset} package.

\paragraph{Who is affected?}\label{who-is-affected-129}

This change affects anyone who is using the
\texttt{com.liferay.dynamic.data.lists.web.asset} package. This
particularly affects anyone using
\texttt{com.liferay.dynamic.data.lists.web.asset.DDLRecordAssetRendererFactory}
and casting the return \texttt{AssetRenderer} to
\texttt{com.liferay.dynamic.data.lists.web.asset.DDLRecordAssetRenderer}.

\paragraph{How should I update my
code?}\label{how-should-i-update-my-code-129}

There are no replacements for this package; you must remove all usages.
\texttt{DDLRecordAssetRendererFactory} can still be used as an OSGi
service; however, you can no longer cast the returned
\texttt{AssetRenderer} to \texttt{DDLRecordAssetRenderer}.

\paragraph{Why was this change
made?}\label{why-was-this-change-made-129}

This change was made to clean up LPKG dependencies.

\subsubsection{Deprecated the social.activity.sets.enabled Property with
No Direct
Replacement}\label{deprecated-the-social.activity.sets.enabled-property-with-no-direct-replacement}

\begin{itemize}
\tightlist
\item
  \textbf{Date:} 2018-Jan-24
\item
  \textbf{JIRA Ticket:} LPS-63635
\end{itemize}

\paragraph{What changed?}\label{what-changed-130}

The \texttt{social.activity.sets.enabled} property is no longer
recognized by the Social Activity portlet. From Liferay Portal 7.0
onwards, Social Activity Sets will always be used.

\paragraph{Who is affected?}\label{who-is-affected-130}

This change affects anyone who has set the
\texttt{social.activity.sets.enabled} property to \texttt{false}.

\paragraph{How should I update my
code?}\label{how-should-i-update-my-code-130}

No changes are necessary.

\paragraph{Why was this change
made?}\label{why-was-this-change-made-130}

The Social Activity portlet had two different versions with slightly
different behaviors; one used in Liferay Portal and the other one in
Social Office. To sync both components, and simplify its internal logic,
activity sets are always enabled by default, with no option to disable
them.

\subsubsection{Removed Description HTML Escaping in
PortletDisplay}\label{removed-description-html-escaping-in-portletdisplay}

\begin{itemize}
\tightlist
\item
  \textbf{Date:} 2018-Jul-17
\item
  \textbf{JIRA Ticket:} LPS-83185
\end{itemize}

\paragraph{What changed?}\label{what-changed-131}

The portlet description stored in \texttt{PortletDisplay.java} is no
longer escaped automatically.

\paragraph{Who is affected?}\label{who-is-affected-131}

This affects anyone who used the portlet description's escaped value to
generate HTML. A small UI change could occur, as some characters may be
unescaped.

\paragraph{How should I update my
code?}\label{how-should-i-update-my-code-131}

If you were using the \texttt{portletDescription} value to generate
HTML, you should escape it using the proper escape sequence:
\texttt{HtmlUtil.escape}.

\paragraph{Why was this change
made?}\label{why-was-this-change-made-131}

This change corrects a best practice violation regarding content
escaping.

\subsubsection{Removed Cache Bootstrap
Feature}\label{removed-cache-bootstrap-feature}

\begin{itemize}
\tightlist
\item
  \textbf{Date:} 2020-Jan-8
\item
  \textbf{JIRA Ticket:} LPS-96563
\end{itemize}

\paragraph{What changed?}\label{what-changed-132}

The cache bootstrap feature has been removed. These properties can no
longer be used to enable/configure cache bootstrap:

\texttt{ehcache.bootstrap.cache.loader.enabled},
\texttt{ehcache.bootstrap.cache.loader.properties.default},
\texttt{ehcache.bootstrap.cache.loader.properties.\$\{specific.cache.name\}}.

\paragraph{Who is affected?}\label{who-is-affected-132}

This affects anyone using the properties listed above.

\paragraph{How should I update my
code?}\label{how-should-i-update-my-code-132}

There's no direct replacement for the removed feature. If you have code
that depends on it, you must implement it yourself.

\paragraph{Why was this change
made?}\label{why-was-this-change-made-132}

This change was made to avoid security issues.

\section{What Changed Between Liferay npm Bundler 1.x and
2.x}\label{what-changed-between-liferay-npm-bundler-1.x-and-2.x}

This reference doc outlines the key changes between liferay-npm-bundler
version 1.x and 2.x.

\subsection{Automatically Formatting Modules for
AMD}\label{automatically-formatting-modules-for-amd}

In version series 1.x of the bundler it was the developer's
responsibility to wrap project modules in an AMD \texttt{define()} call.
However, since 2.x the bundler does it for you, so the only requisite is
that the project's code is transpiled/written for CommonJS modules model
(the standard model for module handling in Node.js, that uses
\texttt{require()} calls to load modules).

\subsection{Isolating Project
Dependencies}\label{isolating-project-dependencies}

Package names are prefixed with the bundle name since version 2.0.0 of
the bundler, but were left intact in previous versions. This strategy is
used to isolate packages from different bundles. You can still deploy
bundler 1.x packages (without prefix), and they will still work as they
did for previous versions of the bundler.

\subsection{Improved Peer Dependency
Support}\label{improved-peer-dependency-support}

In bundler 1.x, there was only one shared peer dependency package
available between portlets. With isolated dependencies per portlet, it's
easy to honor peer dependencies perfectly. Peer dependencies can be
resolved exactly as stated in projects because their names are prefixed
with the project's name. This is possible because of the new
\href{https://github.com/liferay/liferay-npm-build-tools/tree/master/packages/liferay-npm-bundler-plugin-inject-peer-dependencies}{liferay-npm-bundler-plugin-inject-peer-dependencies}
plugin. It scans all JS modules for \texttt{require} calls. If the
bundler finds a required package in the \texttt{main.js} file, but it is
not declared in the \texttt{package.json}, it resolves it to the proper
version that is found in the \texttt{node\_modules} folder. The plugin
then injects a new dependency in the output \texttt{package.json} for
the required package.

Note that injected dependency version constraints are the specific
version number required, without caret or any other semantic version
operator. This is to honor the exact peer dependency found in the
project. Injecting more relaxed semantic version expressions could lead
to unstable results.

\subsection{Manually De-duplicating Through
Importing}\label{manually-de-duplicating-through-importing}

Namespacing means that each portlet gets its own dependencies. Only
using the bundler this way obtains the same functionality as standard
bundlers like webpack or Browserify, so you wouldn't need a specific
tool like liferay-npm-bundler. Since Liferay DXP is a portlet based
architecture, sharing dependencies among different portlets would be
very beneficial.

In bundler 1.x that deduplication was made automatically, but there was
no control over it. However, with version 2.x, you may now import
packages from an external OSGi bundle, instead of using your own. This
lets you put shared dependencies in one project, and reference them from
the rest. Though This new way of de-duplication is not automatic, it
leads to full control (during build time) of how each package is
resolved.
